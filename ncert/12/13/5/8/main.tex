From the given information, 
\begin{align}
    p_X(k) = \comb{n}{k} p^k (1-p)^{n-k}, \quad 
	n = 6, p =\frac{1}{2}.
\end{align}
yielding
\begin{align}
    p_X(k) &= \comb{n}{k} \brak{\frac{1}{2}}^k \brak{\frac{1}{2}}^{n-k}\\
    &= \comb{n}{k} \brak{\frac{1}{2}}^n
\end{align}
upon substituting for $p$.
For $p_X\brak{k}$ to be maximum, 
\begin{align}
	\comb{n}{k} &\geq \comb{n}{k-1} \label{eq: 1.1.3}\quad \text{and} \\
	\comb{n}{k} &\geq \comb{n}{k+1} \label{eq: 1.1.4}
	\\
\because    \comb{n}{k}&= \frac{n!}{(n-k)!k!},
    \label{eq: 1.1.2}
\end{align}
from \eqref{eq: 1.1.2} and \eqref{eq: 1.1.3},  
\begin{align}
	\frac{n!}{(n-k)!k!} &\geq \frac{n!}{(n-k+1)!(k-1)!}\\
	\implies \frac{n!}{(n-k)!k!} &\geq \frac{n!}{(n-k)!k!}\frac{k}{n-k+1}\\
	\implies 1 &\geq \frac{k}{n-k+1}\\
	\therefore k &\leq \frac{n+1}{2} \label{eq: 1.1.5}
\end{align}
From \eqref{eq: 1.1.2} and \eqref{eq: 1.1.4},
\begin{align}
	\frac{n!}{(n-k)!k!} &\geq \frac{n!}{(n-k-1)!(k+1)!}\\
	\implies \frac{n!}{(n-k)!k!} &\geq \frac{n!}{(n-k)!k!}\frac{n-k}{k+1}\\
	\implies 1 &\geq \frac{n-k}{k+1}\\
	\therefore k &\geq \frac{n-1}{2}  \label{eq: 1.1.6}
\end{align} 
Thus, from \eqref{eq: 1.1.5} and \eqref{eq: 1.1.6},
\begin{align}
	\frac{n-1}{2} &\leq k \leq \frac{n+1}{2} \label{eq: Final}
    \\
    \implies
    k &=
    \begin{cases}
        \frac{n}{2}, &  n \text{ even} \\
        \frac{n+1}{2} \text{ or } \frac{n-1}{2}, &  n \text{ odd} 
    \end{cases}
\end{align}
Since
\begin{align}
   	n=6,
   	 k=\frac{n}{2}
   	=3
\end{align}
See \figref{fig:Triangle}.
\begin{figure}[H]
\centering
\includegraphics[width=\columnwidth]{ncert/12/13/5/8/figs/figure1.png}
\caption{}
\label{fig:Triangle}
\end{figure}
