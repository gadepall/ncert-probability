\begin{table}[h!]
    \begin{center}
       \begin{tabular}{|l|c|r|}
       \hline
       Parameter & Values & Description\\
       \hline
       $n$ & 10 & Number of articles\\
       \hline
       $p$ & 0.05 & Probability of being defective\\
       \hline
       $Y$ & $0\leq Y \leq 10$
       &  Number of defective elements\\
       \hline
       $\mu=np$ & $0.5$ & mean\\
       \hline
       $\sigma=\sqrt{np(1-p)}$ & $0.475$ & standard deviation\\
       \hline
       \end{tabular}
       \end{center}
   \end{table}
\text{Gaussian Distribution}\\
\begin{enumerate}
\item Central limit theorm:
\begin{align}
Y &\sim \gauss{\mu}{\frac{\sigma}{\sqrt{n}}}\\
\end{align}
Due to continuity correction \pr{X=x} can be approximated using gaussian distribution as
\begin{align}
	p_Y\brak{x}&\approx\pr{x-0.05<Y<x+0.05}\\
	&\approx\pr{Y<x+0.05}-\pr{Y<x-0.05}	\\
	&\approx F_Y\brak{x+0.05}-F_Y\brak{x-0.05}
\end{align}
Now, we get:
\begin{align}
F_Y\brak{1} &= p_{Y}\brak{1.05}\\
&= 1 - \qfunc{\frac{1.05 - 0.5}{\sqrt{0.05}}}\\
&= 1-\qfunc{\frac{0.55}{0.2236}}\\
&= 1-\qfunc{2.4596}\\
&= 0.99304
\end{align}
\item Binomial Distribution:
\begin{align}
n=10 ; p=\frac{1}{20}
\end{align}
Pmf of $X$ for $0 \leq k \leq 10$ is
\begin{align}
p_X(k)&=\comb{n}{k}p^k(1-p)^{n-k}
\end{align}
Then the probability is given as:
\begin{align}
p_X(0)+p_X(1)=\comb{10}{0}\left(\frac{1}{20}\right)^0\left(1-\frac{1}{20}\right)^{10}
+\comb{10}{1}\left(\frac{1}{20}\right)^1\left(1-\frac{1}{20}\right)^{9}
\end{align}
Hence we get;
\begin{align}
p_X(0)+p_X(1)=29\left(\frac{19^9}{20^{10}}\right)
            &=0.91386
\end{align}
Hence we can say probability calculated through central limit theorem is very close to 
the one calculated through binomial distribution.
\end{enumerate}
\begin{figure}[H]
\centering
\includegraphics[width=\columnwidth]{ncert/9/3/2/figs/figure1.png}
\caption{Binomial vs Gaussian}
\label{fig:9.3.2_gauss}
\end{figure}
