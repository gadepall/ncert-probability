See 
\tabref{tab:11/16/3/11}.
\figref{fig:exemplar/11/16/3/11} is used to obtain the input probabilities.
\begin{enumerate}
	\item
\begin{align}
BC^\prime	
=
	BC^\prime\brak{A+A^\prime}=	
	BC^\prime A+BC^\prime A^\prime	
\label{eq:11/16/3/11/1}
\end{align}
Also, 
\begin{align}
	AB&=
	AB\brak{C+C^\prime}
	\\
	&=	
	ABC+ABC^\prime = ABC^\prime  \, \because AC=0.
\label{eq:11/16/3/11/2}
\end{align}
From 
\eqref{eq:11/16/3/11/1}
and
\eqref{eq:11/16/3/11/2},
\begin{align}
BC^\prime = AB+A^\prime BC^\prime
\\
\implies
	\pr{BC^\prime} = \pr{AB}+\pr{A^\prime BC^\prime}
\end{align}
\item Also, 
\begin{align}
\pr{B} = \pr{BC}+\pr{BC^\prime} = 0.17+0.15 = 0.32
\end{align}
from 
\tabref{tab:11/16/3/11}.  This is used to evaluate $\pr{A+B}$.
\item $\because AC=0$
\begin{align}
	\pr{AB}=\pr{AB^\prime C^\prime} +\pr{A B^\prime C} &= \pr{AB^\prime C^\prime}
	\\
	\pr{B^\prime C} &= \pr{A^\prime B^\prime C}
\end{align}
\end{enumerate}
