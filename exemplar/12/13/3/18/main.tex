\iffalse
\let\negmedspace\undefined
\let\negthickspace\undefined
\documentclass[journal,12pt,onecolumn]{IEEEtran}
\usepackage{cite}
\usepackage{amsmath,amssymb,amsfonts,amsthm}
\usepackage{algorithmic}
\usepackage{graphicx}
\usepackage{textcomp}
\usepackage{xcolor}
\usepackage{txfonts}
\usepackage{listings}
\usepackage{enumitem}
\usepackage{mathtools}
\usepackage{gensymb}
\usepackage{comment}
\usepackage[breaklinks=true]{hyperref}
\usepackage{tkz-euclide} 
\usepackage{listings}
\usepackage{gvv}                                        
\def\inputGnumericTable{}                                 
\usepackage[latin1]{inputenc}                                
\usepackage{color}                                            
\usepackage{array}                                            
\usepackage{longtable}                                       
\usepackage{calc}                                             
\usepackage{multirow}                                         
\usepackage{hhline}                                           
\usepackage{ifthen}                                           
\usepackage{lscape}

\newtheorem{theorem}{Theorem}[section]
\newtheorem{problem}{Problem}
\newtheorem{proposition}{Proposition}[section]
\newtheorem{lemma}{Lemma}[section]
\newtheorem{corollary}[theorem]{Corollary}
\newtheorem{example}{Example}[section]
\newtheorem{definition}[problem]{Definition}
\newcommand{\BEQA}{\begin{eqnarray}}
\newcommand{\EEQA}{\end{eqnarray}}
\newcommand{\define}{\stackrel{\triangle}{=}}
\theoremstyle{remark}
\newtheorem{rem}{Remark}
\begin{document}

\bibliographystyle{IEEEtran}
\vspace{3cm}


\title{Question 12.13.3.18}
\author{EE22BTECH11051}

\maketitle
\vspace{3cm}

\textbf{Question:} A box has 5 blue and 4 red balls. One ball is drawn at random and not replaced.
Its colour is also not noted. Then another ball is drawn at random. What is the
probability of second ball being blue?

\textbf{Solution:} \\
\fi
Let X and Y denote the random variables for the first and second draw respectively as follows:
\begin{table}[h]
    \centering
    %%%%%%%%%%%%%%%%%%%%%%%%%%%%%%%%%%%%%%%%%%%%%%%%%%%%%%%%%%%%%%%%%%%%%%
%%                                                                  %%
%%  This is a LaTeX2e table fragment exported from Gnumeric.        %%
%%                                                                  %%
%%%%%%%%%%%%%%%%%%%%%%%%%%%%%%%%%%%%%%%%%%%%%%%%%%%%%%%%%%%%%%%%%%%%%%

\begin{center}
    \begin{tabular}{|c|c|c|}
    \hline
    \textbf{RV}& \textbf{Values} & \textbf{Description} \\ \hline
    $X$		   & 	$\{0,1\}$	&  1st draw :- 0: blue, 1: red\\ \hline
    $Y$ 		   & 	$\{0,1\}$	&  2nd draw :- 0: blue, 1: red\\ \hline
    \end{tabular}
    \end{center}
    \caption{Random Variables}
    \label{12.13.3.8_table_1}
    \end{table}
\\
The probabilities are given as:
\begin{align}
    \pr{X = 0} = \frac{5}{9}\\
    \pr{X = 1} = \frac{4}{9}
\end{align}
\begin{align}
    \pr{Y = 0 | X = 0} = \frac{\pr{Y = 0,X = 0}}{\pr{X = 0}} = \frac{1}{2}\\
    \pr{Y = 0 | X = 1} = \frac{\pr{Y = 0,X = 1}}{\pr{X = 1}} = \frac{5}{8}
\end{align}
    \\
The probability of the second ball beign drawn being blue is given as:
\begin{align}
\pr{Y = 0} &= \pr{X = 0}\pr{Y = 0 | X = 0} + \pr{X = 1}\pr{Y = 0 | X = 1}\\
           &= {\frac{5}{9}}\times{\frac{1}{2}} + {\frac{4}{9}}\times{{\frac{5}{8}}}\\
           &= \frac{5}{9}
\end{align}
