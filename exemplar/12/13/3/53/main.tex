	See \tabref{tab:exemplar/12/13/3/53}
\begin{table}[H]
 	\input{exemplar/12/13/3/53/table/table.tex}
	\caption{}
	\label{tab:exemplar/12/13/3/53}
\end{table}
The pmf of $X$ is given by
\begin{align}
p_X(0) &= \frac{48}{52} \times \frac{47}{51}
= \frac{188}{221}\\
p_X(1) &= \frac{4}{52} \times \frac{48}{51} + \frac{48}{52} \times \frac{4}{51}
= \frac{32}{221}\\
p_X(2) &= \frac{4}{52} \times \frac{3}{51}
= \frac{1}{221}
\end{align}
Therefore,
\begin{align}
\mu = E\brak{X} &= \sum_{k=0}^2 kp_X(k)\\
&= 0p_X(0)+1p_X(1)+2p_X(2)\\
&= \frac{34}{221} 
= \frac{2}{13} \label{eq:exemplar/12/13/3/53/1}
\end{align}
and
\begin{align}
E\brak{X^2} &= \sum_{k=0}^2 k^2p_X(k)\\
&= 0p_X(0) + 1p_X(1) +4p_X(2)\\
&= \frac{36}{221} \label{eq:exemplar/12/13/3/53/2}
\end{align}
Now,
\begin{align}
Var\brak{X} = E\brak{X^2} - \brak{E\brak{X}}^2
\end{align}
Using \eqref{eq:exemplar/12/13/3/53/1} and \eqref{eq:exemplar/12/13/3/53/2}
\begin{align}
\sigma^2 &= \frac{36}{221} - \brak{\frac{2}{13}}^2
=\frac{400}{2873}\\
\implies \sigma &= \sqrt{\frac{400}{2873}}
\approx 0.373
\end{align}



