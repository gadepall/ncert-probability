\documentclass[12pt,twocolumn,notitlepage]{article}
\usepackage[margin=0.5in]{geometry}
\usepackage{amsmath}
\usepackage{gensymb}
\usepackage{graphicx}
\usepackage{amsthm}
\usepackage{mathrsfs}
\usepackage{txfonts}
\usepackage{cite}
\usepackage{cases}
\usepackage{subfig}
\usepackage[breaklinks=true]{hyperref}
\usepackage{listings}
\usepackage[latin1]{inputenc}
\usepackage{color}
\usepackage{array}
\usepackage{longtable}
\usepackage{calc}
\usepackage{multirow}
\usepackage{hhline}
\usepackage{ifthen}
\usepackage{amssymb}
\providecommand{\pr}[1]{\ensuremath{\Pr\left(#1\right)}}
\providecommand{\sbrak}[1]{\ensuremath{{}\left[#1\right]}}
\providecommand{\lsbrak}[1]{\ensuremath{{}\left[#1\right.}}
\providecommand{\rsbrak}[1]{\ensuremath{{}\left.#1\right]}}
\providecommand{\brak}[1]{\ensuremath{\left(#1\right)}}
\providecommand{\lbrak}[1]{\ensuremath{\left(#1\right.}}
\providecommand{\rbrak}[1]{\ensuremath{\left.#1\right)}}
\providecommand{\cbrak}[1]{\ensuremath{\left\{#1\right\}}}
\providecommand{\lcbrak}[1]{\ensuremath{\left\{#1\right.}}
\providecommand{\rcbrak}[1]{\ensuremath{\left.#1\right\}}}

\newcommand*{\comb}[2]{{}^{#1}C_{#2}}

\title{Probability Assignment 1 (12.13.5.2)}
\author{Aditya Varun V (AI22BTECH11001)}
\date{}

\begin{document}

\maketitle
\subsection*{Question}
A pair of dice is thrown 4 times. If getting a doublet is considered a success, find
the probability of two successes.


\subsection*{Solution}

Let X denote the number of doublets/successes obtained after the 4 trials. Clearly, X has the binomial distribution with $n=4$ and p being the probability of obtaining a doublet,
\begin{align}
    p &= \frac{6}{36} \\
    &= \frac{1}{6} 
\end{align}

Now, since X has the binomial distribution, the probability mass function is given by
\begin{align}
    P_X\brak{r} &= \comb{n}{r}\brak{\frac{1}{6}}^{r}\brak{\frac{5}{6}}^{n-r} 
\end{align}

Hence, the probability of two successes is
\begin{align}
    P_X\brak{2} &= \comb{4}{2}\brak{\frac{1}{6}}^{2}\brak{\frac{5}{6}}^{2} \\
    &= \frac{25}{216} 
\end{align}

\end{document}

