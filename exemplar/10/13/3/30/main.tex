Number of cards left after removing all jacks, queens and kings 
\begin{align}
N	= 52 - 4\times 3
	= 40
\end{align}
%\begin{table}[H]
%\def\arraystretch{1.2}
%\begin{tabular}{|c|c|c|}
%\hline
%	\textbf{Parameter} &\textbf{Value} &\textbf{Description}\\ \hline
%	$X$ &1-10 &Represents the value of the card picked \\ \hline
%\end{tabular}
%\end{table}
Let $1 \le X \le 10$ be the value of the card picked.  Then,
\begin{align}
	p_X(k) &= \Pr(X=k)\ \forall\ 1 \leq k \leq 10\\
	&= \frac{4\times 1}{40}\\
	&= \frac{1}{10}\\
	\therefore p_X(k) &= 
	\begin{cases}
		\frac{1}{10} & 1 \leq k \leq 10\\
		0 & \text{otherwise}
	\end{cases}
\end{align}
and
\begin{align}
	F_{X}(k) &= \sum_{m=0}^{k}p_{X}(m) \quad 1 \leq k \leq 10\\
	&= \frac{k}{10}\\
	\therefore F_{X}(k) &= 
	\begin{cases}
		0 & k \leq 0\\
		\frac{k}{10} & 1\leq k \leq 10\\
		1 & k > 10 
	\end{cases}
\end{align}
\begin{enumerate}
	\item Probability that card has value equal to 7 is
		\begin{align}
			 p_{X}(7)
			= \frac{1}{10}
		\end{align}
	\item Probability that card has value greater than 7 is
		\begin{align}
			1 - F_X(7)
			&= 1 - \frac{7}{10}
			\\
			&= \frac{3}{10}
		\end{align}
	\item Probability that card has value less than 7 is
		\begin{align}
			 F_{X}(6)
			=\frac{6}{10}
		\end{align}
\end{enumerate}
