\documentclass[journal]{IEEEtran}
\usepackage[a5paper, margin=10mm]{geometry}
%\usepackage{lmodern} % Ensure lmodern is loaded for pdflatex
\usepackage{tfrupee} % Include tfrupee package


\setlength{\headheight}{1cm} % Set the height of the header box
\setlength{\headsep}{0mm}     % Set the distance between the header box and the top of the text


%\usepackage[a5paper, top=10mm, bottom=10mm, left=10mm, right=10mm]{geometry}

%
\usepackage{gvv-book}
\usepackage{gvv}
%\setlength{\intextsep}{10pt} % Space between text and floats

\makeindex

\begin{document}
\bibliographystyle{IEEEtran}
\onecolumn


\title{
	%\begin{flushleft}
	\begin{center}
	Random Variables through Simulation 
		%Probability
	%MATRICES \\ In Geometry
	%Matrices In Geometry
%	\\
%\rule{0.4\columnwidth}{0.4pt}
%\end{flushleft}
\end{center}
}
\author{
\vspace{11cm}
	%\begin{flushleft}
	\begin{center}
\includegraphics[width=0.2\columnwidth]{figs/logo.jpg}
\\
		{	\huge G. V. V. Sharma}
	\end{center}
	%\end{flushleft}
%\IEEEpubid{\makebox[\columnwidth]{978-1-7281-5966-1/20/\$31.00 ©2020 IEEE \hfill} \hspace{\columnsep}\makebox[\columnwidth]{ }}
}
\maketitle
\newpage

\section*{About this Book}

This book introduces random variables through high school probability. All problems in the book are from NCERT mathematics textbooks from Class 9-12.   
  A lot of college level concepts related to random variables are covered in the process.
The content is sufficient for random variable simulations using Python/C.
There is no copyright, so readers are free to print and share.  
\begin{flushright}
\today
\end{flushright}
Github: https://github.com/gadepall/ncert-probability
		\\
License: https://creativecommons.org/licenses/by-sa/3.0/
\\
and
\\
https://www.gnu.org/licenses/fdl-1.3.en.html

\newpage

\tableofcontents

\newpage
%\twocolumn


%\renewcommand{\theequation}{\theenumi}
\numberwithin{equation}{enumi}
\numberwithin{figure}{enumi}
\numberwithin{table}{enumi}
%\numberwithin{table}{section} 
%\renewcommand{\thefigure}{\theenumi}
%\renewcommand{\thetable}{\theenumi}

%\section{Axioms}
\section{Definitions}
\subsection{NCERT}
\begin{enumerate}[label=\thesection.\arabic*,ref=\thesection.\theenumi]
	\item One card is drawn from a well-shuffled deck of 52 cards. Find the probability of getting
\begin{enumerate}
\item A king of red colour 
\item A face card 
\item A red face card
\item The jack of hearts
\item A spade
\item The queen of diamonds

\end{enumerate}
\solution
		%\begin{table}[H]
	\centering
\begin{tabular}{|c|c|c|}
\hline
Random variable &Value &Definition\\ \hline
\multirow{3}{*}{X} &0 &Slips of Rs 1\\
&1 &Slips of Rs 5\\
&2 &Slips of Rs 13\\ \hline
\multirow{2}{*}{Y} &0 &Box A\\
&1 &Box B\\\hline
\end{tabular}
\caption{}
\label{tab:Distribution}
\end{table}
See \tabref{tab:Distribution}.
\begin{align}
p_{Y}\brak{k}= \begin{cases} 
      \frac{1}{3} & {k=0} \\
      \frac{2}{3 }& {k=1} 
   \end{cases}
   \\
p_{Y|X}\brak{0|0} = \frac{19}{25}\, 
p_{Y|X}\brak{0|1} = \frac{6}{25}\,
p_{Y|X}\brak{1|0} = \frac{45}{50}\,
p_{Y|X}\brak{1|2} = \frac{5}{50}
\end{align}
The desired probability is the probability that a slip drawn at random is marked other than Rs 1,
\begin{align}
&=1-p_X\brak{0}\\
&= p_X(1) + p_X(2)
\end{align}
Using Bayes theorem,
\begin{align}
&= p_Y\brak{0} \times \pr{Y=0 | X=1} + p_Y\brak{1} \times \pr{Y=1|X=2}\\
&=\frac{1}{3} \times \frac{6}{25} + \frac{2}{3} \times \frac{5}{50}\\
&=\frac{11}{75}
\end{align}

\newpage

%\tableofcontents

\bigskip

\renewcommand{\thefigure}{\theenumi}
\renewcommand{\thetable}{\theenumi}
%\renewcommand{\theequation}{\theenumi}

%\begin{abstract}
%%\boldmath
%In this letter, an algorithm for evaluating the exact analytical bit error rate  (BER)  for the piecewise linear (PL) combiner for  multiple relays is presented. Previous results were available only for upto three relays. The algorithm is unique in the sense that  the actual mathematical expressions, that are prohibitively large, need not be explicitly obtained. The diversity gain due to multiple relays is shown through plots of the analytical BER, well supported by simulations. 
%
%\end{abstract}
% IEEEtran.cls defaults to using nonbold math in the Abstract.
% This preserves the distinction between vectors and scalars. However,
% if the journal you are submitting to favors bold math in the abstract,
% then you can use LaTeX's standard command \boldmath at the very start
% of the abstract to achieve this. Many IEEE journals frown on math
% in the abstract anyway.

% Note that keywords are not normally used for peerreview papers.
%\begin{IEEEkeywords}
%Cooperative diversity, decode and forward, piecewise linear
%\end{IEEEkeywords}



% For peer review papers, you can put extra information on the cover
% page as needed:
% \ifCLASSOPTIONpeerreview
% \begin{center} \bfseries EDICS Category: 3-BBND \end{center}
% \fi
%
% For peerreview papers, this IEEEtran command inserts a page break and
% creates the second title. It will be ignored for other modes.
%\IEEEpeerreviewmaketitle




	\item Five cards—the ten, jack, queen, king and ace of diamonds, are well-shuffled with their face downwards. One card is then picked up at random.
\begin{enumerate}
\item
What is the probability that the card is the queen? 
\item
If the queen is drawn and put aside, what is the probability that the second card picked up is (a) an ace? (b) a queen?\\
\end{enumerate}
\solution
		%\begin{enumerate}[label=\thesection.\arabic*,ref=\thesection.\theenumi]
	\item One card is drawn from a well-shuffled deck of 52 cards. Find the probability of getting
\begin{enumerate}
\item A king of red colour 
\item A face card 
\item A red face card
\item The jack of hearts
\item A spade
\item The queen of diamonds

\end{enumerate}
\solution
		%\begin{table}[H]
	\centering
\begin{tabular}{|c|c|c|}
\hline
Random variable &Value &Definition\\ \hline
\multirow{3}{*}{X} &0 &Slips of Rs 1\\
&1 &Slips of Rs 5\\
&2 &Slips of Rs 13\\ \hline
\multirow{2}{*}{Y} &0 &Box A\\
&1 &Box B\\\hline
\end{tabular}
\caption{}
\label{tab:Distribution}
\end{table}
See \tabref{tab:Distribution}.
\begin{align}
p_{Y}\brak{k}= \begin{cases} 
      \frac{1}{3} & {k=0} \\
      \frac{2}{3 }& {k=1} 
   \end{cases}
   \\
p_{Y|X}\brak{0|0} = \frac{19}{25}\, 
p_{Y|X}\brak{0|1} = \frac{6}{25}\,
p_{Y|X}\brak{1|0} = \frac{45}{50}\,
p_{Y|X}\brak{1|2} = \frac{5}{50}
\end{align}
The desired probability is the probability that a slip drawn at random is marked other than Rs 1,
\begin{align}
&=1-p_X\brak{0}\\
&= p_X(1) + p_X(2)
\end{align}
Using Bayes theorem,
\begin{align}
&= p_Y\brak{0} \times \pr{Y=0 | X=1} + p_Y\brak{1} \times \pr{Y=1|X=2}\\
&=\frac{1}{3} \times \frac{6}{25} + \frac{2}{3} \times \frac{5}{50}\\
&=\frac{11}{75}
\end{align}

\newpage

%\tableofcontents

\bigskip

\renewcommand{\thefigure}{\theenumi}
\renewcommand{\thetable}{\theenumi}
%\renewcommand{\theequation}{\theenumi}

%\begin{abstract}
%%\boldmath
%In this letter, an algorithm for evaluating the exact analytical bit error rate  (BER)  for the piecewise linear (PL) combiner for  multiple relays is presented. Previous results were available only for upto three relays. The algorithm is unique in the sense that  the actual mathematical expressions, that are prohibitively large, need not be explicitly obtained. The diversity gain due to multiple relays is shown through plots of the analytical BER, well supported by simulations. 
%
%\end{abstract}
% IEEEtran.cls defaults to using nonbold math in the Abstract.
% This preserves the distinction between vectors and scalars. However,
% if the journal you are submitting to favors bold math in the abstract,
% then you can use LaTeX's standard command \boldmath at the very start
% of the abstract to achieve this. Many IEEE journals frown on math
% in the abstract anyway.

% Note that keywords are not normally used for peerreview papers.
%\begin{IEEEkeywords}
%Cooperative diversity, decode and forward, piecewise linear
%\end{IEEEkeywords}



% For peer review papers, you can put extra information on the cover
% page as needed:
% \ifCLASSOPTIONpeerreview
% \begin{center} \bfseries EDICS Category: 3-BBND \end{center}
% \fi
%
% For peerreview papers, this IEEEtran command inserts a page break and
% creates the second title. It will be ignored for other modes.
%\IEEEpeerreviewmaketitle




	\item Five cards—the ten, jack, queen, king and ace of diamonds, are well-shuffled with their face downwards. One card is then picked up at random.
\begin{enumerate}
\item
What is the probability that the card is the queen? 
\item
If the queen is drawn and put aside, what is the probability that the second card picked up is (a) an ace? (b) a queen?\\
\end{enumerate}
\solution
		%\begin{enumerate}[label=\thesection.\arabic*,ref=\thesection.\theenumi]
	\item One card is drawn from a well-shuffled deck of 52 cards. Find the probability of getting
\begin{enumerate}
\item A king of red colour 
\item A face card 
\item A red face card
\item The jack of hearts
\item A spade
\item The queen of diamonds

\end{enumerate}
\solution
		%\input{ncert/10/15/1/14/main.tex}
	\item Five cards—the ten, jack, queen, king and ace of diamonds, are well-shuffled with their face downwards. One card is then picked up at random.
\begin{enumerate}
\item
What is the probability that the card is the queen? 
\item
If the queen is drawn and put aside, what is the probability that the second card picked up is (a) an ace? (b) a queen?\\
\end{enumerate}
\solution
		%\input{ncert/10/15/1/15/defs.tex}
	\item A bag contains $5$ red balls and some blue balls. If the probability of drawing a blue ball is double that if a red ball, determine the number of blue balls in the bag. 
		\\
\solution
		%\input{ncert/10/15/2/3/defs.tex}
	\item A card is selected from a pack of 52 cards.
 \begin{enumerate}[label=(\alph*)] 
                 \item How many points are there in the sample space?
                 \item Calculate the probability that the card is an ace of spades.
                 \item Calculate the probability that the card is (i) an ace and (ii) black card.
 \end{enumerate}
\solution
		%\input{ncert/11/16/3/4/main.tex}
\item Four cards are drawn from a well-shuffled deck of 52 cards. What is the probability of obtaining 3 diamonds and one spade.
\\
\solution
		%\input{ncert/11/16/4/2/defs.tex}
\item In a certain lottery 10,000 tickets are sold and ten equal prizes are awarded. What is the probability of not getting a prize if you buy (a) one ticket (b) two tickets (c) 10 tickets ?	
\\
\solution
		%\input{ncert/11/16/4/4/defs.tex}
		%
\item 
Out of 100 students, two sections of 40 and 60 are formed. If you and your friend are among the 100 students, what is the probability that
\begin{enumerate}
\item you both enter the same section?
\item you both enter the different sections?
\end{enumerate}
\solution
		%\input{ncert/11/16/4/5/defs.tex}
	\item 
The number lock of a suitcase has 4 wheels each labelled with ten digits i.e. from 0 to 9.The lock opens with a sequence of four digits with no repeats.What is the probability of a person getting the right sequence to open the suitcase.
\\
\solution
		%\input{ncert/11/16/4/10/defs.tex}
		%
\item 
Two cards are drawn at random and without replacement from a pack of 52 playing cards. Find the probability that both the cards are black.
\\
\solution
		%\input{ncert/12/13/2/2/defs.tex}
		\item A box of oranges is inspected by examining three randomly selected oranges drawn without replacement. If all the three oranges are good, the box is approved for sale, otherwise, it is rejected. Find the probability that a box containing 15 oranges out of which 12 are good and 3 are bad ones will be approved for sale.
		\label{ncert/12/13/2/3/defs.tex}
		\item Two balls are drawn at random with replacement from a box containing 10 black and 8 red balls. Find the probability that
		\label{ncert/12/13/2/12}
\begin{enumerate}
\item both balls are red.
\item first ball is black and second is red.
\item one of them is black and other is red.
\end{enumerate}

\item In a hostel, 60\% of the students read Hindi newspaper, 40\% read English newspaper and 20\% read both Hindi and English newspapers. A student is selected at random.
		\label{ncert/12/13/2/15}
\begin{enumerate}
\item Find the probability that she reads neither Hindi nor English newspapers.
\item If she reads Hindi newspaper, find the probability that she reads English newspaper.
\item If she reads English newspaper, find the probability that she reads Hindi newspaper.\\
\end{enumerate}
\item The probability of obtaining an even prime number on each die, when a pair of dice is rolled is 
\begin{enumerate}
    \item $0$ 
    
    \item $\frac{1}{3}$ 
    
    \item $\frac{1}{12}$ 
    
    \item $\frac{1}{36}$ 
\end{enumerate}
\solution
		%\input{ncert/12/13/2/17/defs.tex}
	\item A bag contains 4 red and 4 black balls, another bag contains 2 red and 6 black balls. One of the two bags is selected at random and a ball is drawn from the bag which is found to be red. Find the probability that the ball is drawn from the first bag.
\\
\solution
		%\input{ncert/12/13/3/2/main.tex}
  \item
  Cards with numbers 2 to 101 are placed in a box. A card is selected at random.Find the probability that the card has
\begin{enumerate}[label=(\roman*)]
	\item an even number 
	\item a square number
\end{enumerate}
\solution
%\input{exemplar/10/13/3/32/main.tex}
\item
The king, queen and jack of clubs are removed from a deck of 52 playing cards and then well shuffled. Now one card is drawn at random from the remaining cards.  Determine the probability that the card is
\begin{enumerate}[label=(\roman*)]
\item a club
\item 10 of hearts
\end{enumerate}
\solution
%\input{exemplar/10/13/3/29/main.tex}
\item A team of medical students doing their internship have to assist during surgeries
at a city hospital. The probabilities of surgeries rated as very complex, complex,
routine, simple or very simple are respectively, 0.15, 0.20, 0.31, 0.26, .08. Find
the probabilities that a particular surgery will be rated
\begin{enumerate}
	\item complex or very complex;
	\item neither very complex nor very simple;
	\item routine or complex
	\item routine or simple
\end{enumerate}
\solution
%\input{exemplar/11/16/3/8(1)/main.tex}
\item A card is selected from a pack of 52 cards.
\begin{enumerate}[label=(\alph*)]
    \item How many points are there in the sample space?
    \item Calculate the probability that the card is an ace of spades.
    \item Calculate the probability that the card is (i) an ace and (ii) black card.
\end{enumerate}
\solution
%\input{exemplar/11/16/3/4/main2.tex}
\item The probability that a non leap year selected at random will contain 53 sundays.
\\
\solution
%\input{exemplar/10/13/1/19/main.tex}
\item One of the four persons John, Rita, Aslam or Gurpreet will be promoted next
month. Consequently the sample space consists of four elementary outcomes
S = {John promoted, Rita promoted, Aslam promoted, Gurpreet promoted}
You are told that the chances of John’s promotion is same as that of Gurpreet,
Rita’s chances of promotion are twice as likely as Johns. Aslam’s chances are
four times that of John.
\begin{enumerate}
	\item Determine
	\begin{enumerate}
		\item P (John promoted)
		\item P (Rita promoted)
		\item P (Aslam promoted)
		\item P (Gurpreet promoted)
	\end{enumerate}
	\item If A = {John promoted or Gurpreet promoted}, find P (A).
\end{enumerate}
\solution
%\input{exemplar/11/16/3/10/main.tex}
\item A card is drawn from a deck of 52 cards. Find the probability of getting a king or a heart or a red card.\\
\solution
%\input{exemplar/11/16/3/15/main.tex}
\item The probability that a student will pass his examination is 0.73, the probability of
the student getting a compartment is 0.13, and the probability that the student will
either pass or get compartment is 0.96. State True or False.\\
\solution
%\input{exemplar/11/16/3/31/main.tex}
\item A card is selected from a pack of 52 cards\\
\begin{enumerate}[label=(\alph*)]
\item How many points are there in the sample space?
\item Calculate the probability that the cards is an ace of spades.
\item Calculate the probability that the card is (i) an ace (ii)black card.\\
\end{enumerate}
%\input{ncert/11/16/3/4_1/Prob_4.tex}
\item In a non-leap year, the probability of having 53 tuesdays or 53 wednesdays is\\
\solution
%\input{exemplar/11/16/3/18/main.tex}
\item There are 1000 sealed envelopes in a box, 10 of them contain a cash prize of
Rs 100 each, 100 of them contain a cash prize of Rs 50 each and 200 of them
contain a cash prize of Rs 10 each and rest do not contain any cash prize. If they
are well shuffled and an envelope is picked up out, what is the probability that it
contains no cash prize?\\
\solution
%\input{exemplar/10/13/3/34/main.tex}
\item 
A die is thrown and a card is selected at random from a deck of 52 playing cards. The probability of getting an even number on the die and a spade card.\\
\solution
%\input{exemplar/12/13/3/78/main.tex}
\item
If 4-digit numbers greater than 5,000 are randomly formed from the digits 0, 1, 3, 5, and 7, what is the probability of forming a number divisible by 5 when:
\begin{enumerate}
    \item The digits are repeated?
    \item The repetition of digits is not allowed?
\end{enumerate}
\solution
%\input{ncert/11/16/4/9/main.tex}
\item Consider the probability space $\brak{\Omega, \mathcal{G}, P}$ where $\Omega = [0,2]$ and $\mathcal{G} = \cbrak{\phi, \Omega, [0,1], (1,2]}$. Let $X$ and $Y$ be two functions on $\Omega$ defined as
\begin{align*}
    X(\omega) = 
    \begin{cases}
        1 & \text{if }\omega \in [0, 1]\\
        2 & \text{if }\omega \in (1, 2]
    \end{cases}
\end{align*}
and
\begin{align*}
    Y(\omega) = 
    \begin{cases}
        2 & \text{if }\omega \in [0, 1.5]\\
        3 & \text{if }\omega \in (1.5, 2].
    \end{cases}
\end{align*}
Then which one of the following statements is true?
\begin{enumerate}
    \item [(A)] $X$ is a random variable with respect to $\mathcal{G}$, but $Y$ is not a random variable with respect to $\mathcal{G}$.
    \item [(B)] $Y$ is a random variable with respect to $\mathcal{G}$, but $X$ is not a random variable with respect to $\mathcal{G}$.
    \item [(C)] Neither $X$ nor $Y$ is a random variable with respect to $\mathcal{G}$.
    \item [(D)] Both $X$ and $Y$ are random variables with respect to $\mathcal{G}$.
\end{enumerate} \hfill (GATE ST 2023)\\
\solution
%\input{gate/ST/2023/14/main.tex}
	\item  A die is loaded in such a way that each odd number is twice as likely to occur as
each even number. Find $P(G)$, where $G$ is the event that a number greater than
3 occurs on a single roll of the die.
\\
\solution
		%\input{exemplar/11/16/3/5/main.tex}
	\item All the jacks, queens and kings are removed from a deck of 52 playing cards. The remaining cards are well shuffled and then one card is drawn at random. Giving ace a value 1 similar value for other cards, find the probability that the card has a value 
		\begin{enumerate}
			\item 7
			\item greater than 7
			\item less than 7
		\end{enumerate}
		%\input{exemplar/10/13/3/30/main.tex}
  \item A Lot consists of 48 mobile phones of which 42 are good, 3 have only minor defects and 3 have major defects.Varnika will buy a phone if it is good but the trader will only buy a mobile if it has no major defects. One phone is selected at random from the lot. What is the probability that it is
\begin{enumerate}
	\item acceptable to Varnika?
            \item acceptable to the trader?
\end{enumerate}
\solution
	%\input{exemplar/10/13/3/40/main.tex}
 \item A student says that if you throw a die, it will show up 1 or not 1. Therefore, the probability of getting 1 and the probability of getting 'not 1' each is equal to $\frac{1}{2}$. Is this correct? Give reasons.\\
 \solution
        %\input{exemplar/10/13/2/9/main.tex}
   \item Four candidates A, B, C, D have ap-
plied for the assignment to coach a school cricket
team. If A is twice as likely to be selected as B, and
B and C are given about the same chance of being
selected, while C is twice as likely to be selected
as D, what are the probabilities that
\begin{enumerate}
\item C will be selected?
\item A will not be selected?
\end{enumerate}
	%\input{exemplar/11/16/3/9/main.tex}
 \item A bag contain 24 balls of which $x$ balls are red, $2x$ are white and $3x$ are blue. A ball is selected at random, What is the probability that it is
\begin{enumerate}[label=\alph*)]
\item not red ?
\item white ?
\end{enumerate}
%\input{exemplar/10/13/3/41/main.tex}
If the letters of the word ASSASSINATION are arranged at random. Find the Probability that
\begin{enumerate}[label=(\alph*)]
\item Four $S's$ come consecutively in the word
\item Two  $I's$ and two $N's$ come together
\item All $A's$ are not coming together
\item No two $A's$ are coming together
\end{enumerate}
%\input{exemplar/11/16/3/14/main.tex}
	\item One urn contains two black balls (labelled B1 and B2) and one white ball. A
	second urn contains one black ball and two white balls (labelled W1 and W2).
	Suppose the following experiment is performed. One of the two urns is chosen
	at random. Next a ball is randomly chosen from the urn. Then a second ball is
	chosen at random from the same urn without replacing the first ball.
	
	\begin{enumerate}
	\item What is the probability that two black balls are chosen?
	
	\item What is the probability that two balls of opposite colour are chosen?
	\end{enumerate}
	\solution
	%\input{exemplar/11/16/3/12/main1.tex}
\end{enumerate}

	\item A bag contains $5$ red balls and some blue balls. If the probability of drawing a blue ball is double that if a red ball, determine the number of blue balls in the bag. 
		\\
\solution
		%\begin{enumerate}[label=\thesection.\arabic*,ref=\thesection.\theenumi]
	\item One card is drawn from a well-shuffled deck of 52 cards. Find the probability of getting
\begin{enumerate}
\item A king of red colour 
\item A face card 
\item A red face card
\item The jack of hearts
\item A spade
\item The queen of diamonds

\end{enumerate}
\solution
		%\input{ncert/10/15/1/14/main.tex}
	\item Five cards—the ten, jack, queen, king and ace of diamonds, are well-shuffled with their face downwards. One card is then picked up at random.
\begin{enumerate}
\item
What is the probability that the card is the queen? 
\item
If the queen is drawn and put aside, what is the probability that the second card picked up is (a) an ace? (b) a queen?\\
\end{enumerate}
\solution
		%\input{ncert/10/15/1/15/defs.tex}
	\item A bag contains $5$ red balls and some blue balls. If the probability of drawing a blue ball is double that if a red ball, determine the number of blue balls in the bag. 
		\\
\solution
		%\input{ncert/10/15/2/3/defs.tex}
	\item A card is selected from a pack of 52 cards.
 \begin{enumerate}[label=(\alph*)] 
                 \item How many points are there in the sample space?
                 \item Calculate the probability that the card is an ace of spades.
                 \item Calculate the probability that the card is (i) an ace and (ii) black card.
 \end{enumerate}
\solution
		%\input{ncert/11/16/3/4/main.tex}
\item Four cards are drawn from a well-shuffled deck of 52 cards. What is the probability of obtaining 3 diamonds and one spade.
\\
\solution
		%\input{ncert/11/16/4/2/defs.tex}
\item In a certain lottery 10,000 tickets are sold and ten equal prizes are awarded. What is the probability of not getting a prize if you buy (a) one ticket (b) two tickets (c) 10 tickets ?	
\\
\solution
		%\input{ncert/11/16/4/4/defs.tex}
		%
\item 
Out of 100 students, two sections of 40 and 60 are formed. If you and your friend are among the 100 students, what is the probability that
\begin{enumerate}
\item you both enter the same section?
\item you both enter the different sections?
\end{enumerate}
\solution
		%\input{ncert/11/16/4/5/defs.tex}
	\item 
The number lock of a suitcase has 4 wheels each labelled with ten digits i.e. from 0 to 9.The lock opens with a sequence of four digits with no repeats.What is the probability of a person getting the right sequence to open the suitcase.
\\
\solution
		%\input{ncert/11/16/4/10/defs.tex}
		%
\item 
Two cards are drawn at random and without replacement from a pack of 52 playing cards. Find the probability that both the cards are black.
\\
\solution
		%\input{ncert/12/13/2/2/defs.tex}
		\item A box of oranges is inspected by examining three randomly selected oranges drawn without replacement. If all the three oranges are good, the box is approved for sale, otherwise, it is rejected. Find the probability that a box containing 15 oranges out of which 12 are good and 3 are bad ones will be approved for sale.
		\label{ncert/12/13/2/3/defs.tex}
		\item Two balls are drawn at random with replacement from a box containing 10 black and 8 red balls. Find the probability that
		\label{ncert/12/13/2/12}
\begin{enumerate}
\item both balls are red.
\item first ball is black and second is red.
\item one of them is black and other is red.
\end{enumerate}

\item In a hostel, 60\% of the students read Hindi newspaper, 40\% read English newspaper and 20\% read both Hindi and English newspapers. A student is selected at random.
		\label{ncert/12/13/2/15}
\begin{enumerate}
\item Find the probability that she reads neither Hindi nor English newspapers.
\item If she reads Hindi newspaper, find the probability that she reads English newspaper.
\item If she reads English newspaper, find the probability that she reads Hindi newspaper.\\
\end{enumerate}
\item The probability of obtaining an even prime number on each die, when a pair of dice is rolled is 
\begin{enumerate}
    \item $0$ 
    
    \item $\frac{1}{3}$ 
    
    \item $\frac{1}{12}$ 
    
    \item $\frac{1}{36}$ 
\end{enumerate}
\solution
		%\input{ncert/12/13/2/17/defs.tex}
	\item A bag contains 4 red and 4 black balls, another bag contains 2 red and 6 black balls. One of the two bags is selected at random and a ball is drawn from the bag which is found to be red. Find the probability that the ball is drawn from the first bag.
\\
\solution
		%\input{ncert/12/13/3/2/main.tex}
  \item
  Cards with numbers 2 to 101 are placed in a box. A card is selected at random.Find the probability that the card has
\begin{enumerate}[label=(\roman*)]
	\item an even number 
	\item a square number
\end{enumerate}
\solution
%\input{exemplar/10/13/3/32/main.tex}
\item
The king, queen and jack of clubs are removed from a deck of 52 playing cards and then well shuffled. Now one card is drawn at random from the remaining cards.  Determine the probability that the card is
\begin{enumerate}[label=(\roman*)]
\item a club
\item 10 of hearts
\end{enumerate}
\solution
%\input{exemplar/10/13/3/29/main.tex}
\item A team of medical students doing their internship have to assist during surgeries
at a city hospital. The probabilities of surgeries rated as very complex, complex,
routine, simple or very simple are respectively, 0.15, 0.20, 0.31, 0.26, .08. Find
the probabilities that a particular surgery will be rated
\begin{enumerate}
	\item complex or very complex;
	\item neither very complex nor very simple;
	\item routine or complex
	\item routine or simple
\end{enumerate}
\solution
%\input{exemplar/11/16/3/8(1)/main.tex}
\item A card is selected from a pack of 52 cards.
\begin{enumerate}[label=(\alph*)]
    \item How many points are there in the sample space?
    \item Calculate the probability that the card is an ace of spades.
    \item Calculate the probability that the card is (i) an ace and (ii) black card.
\end{enumerate}
\solution
%\input{exemplar/11/16/3/4/main2.tex}
\item The probability that a non leap year selected at random will contain 53 sundays.
\\
\solution
%\input{exemplar/10/13/1/19/main.tex}
\item One of the four persons John, Rita, Aslam or Gurpreet will be promoted next
month. Consequently the sample space consists of four elementary outcomes
S = {John promoted, Rita promoted, Aslam promoted, Gurpreet promoted}
You are told that the chances of John’s promotion is same as that of Gurpreet,
Rita’s chances of promotion are twice as likely as Johns. Aslam’s chances are
four times that of John.
\begin{enumerate}
	\item Determine
	\begin{enumerate}
		\item P (John promoted)
		\item P (Rita promoted)
		\item P (Aslam promoted)
		\item P (Gurpreet promoted)
	\end{enumerate}
	\item If A = {John promoted or Gurpreet promoted}, find P (A).
\end{enumerate}
\solution
%\input{exemplar/11/16/3/10/main.tex}
\item A card is drawn from a deck of 52 cards. Find the probability of getting a king or a heart or a red card.\\
\solution
%\input{exemplar/11/16/3/15/main.tex}
\item The probability that a student will pass his examination is 0.73, the probability of
the student getting a compartment is 0.13, and the probability that the student will
either pass or get compartment is 0.96. State True or False.\\
\solution
%\input{exemplar/11/16/3/31/main.tex}
\item A card is selected from a pack of 52 cards\\
\begin{enumerate}[label=(\alph*)]
\item How many points are there in the sample space?
\item Calculate the probability that the cards is an ace of spades.
\item Calculate the probability that the card is (i) an ace (ii)black card.\\
\end{enumerate}
%\input{ncert/11/16/3/4_1/Prob_4.tex}
\item In a non-leap year, the probability of having 53 tuesdays or 53 wednesdays is\\
\solution
%\input{exemplar/11/16/3/18/main.tex}
\item There are 1000 sealed envelopes in a box, 10 of them contain a cash prize of
Rs 100 each, 100 of them contain a cash prize of Rs 50 each and 200 of them
contain a cash prize of Rs 10 each and rest do not contain any cash prize. If they
are well shuffled and an envelope is picked up out, what is the probability that it
contains no cash prize?\\
\solution
%\input{exemplar/10/13/3/34/main.tex}
\item 
A die is thrown and a card is selected at random from a deck of 52 playing cards. The probability of getting an even number on the die and a spade card.\\
\solution
%\input{exemplar/12/13/3/78/main.tex}
\item
If 4-digit numbers greater than 5,000 are randomly formed from the digits 0, 1, 3, 5, and 7, what is the probability of forming a number divisible by 5 when:
\begin{enumerate}
    \item The digits are repeated?
    \item The repetition of digits is not allowed?
\end{enumerate}
\solution
%\input{ncert/11/16/4/9/main.tex}
\item Consider the probability space $\brak{\Omega, \mathcal{G}, P}$ where $\Omega = [0,2]$ and $\mathcal{G} = \cbrak{\phi, \Omega, [0,1], (1,2]}$. Let $X$ and $Y$ be two functions on $\Omega$ defined as
\begin{align*}
    X(\omega) = 
    \begin{cases}
        1 & \text{if }\omega \in [0, 1]\\
        2 & \text{if }\omega \in (1, 2]
    \end{cases}
\end{align*}
and
\begin{align*}
    Y(\omega) = 
    \begin{cases}
        2 & \text{if }\omega \in [0, 1.5]\\
        3 & \text{if }\omega \in (1.5, 2].
    \end{cases}
\end{align*}
Then which one of the following statements is true?
\begin{enumerate}
    \item [(A)] $X$ is a random variable with respect to $\mathcal{G}$, but $Y$ is not a random variable with respect to $\mathcal{G}$.
    \item [(B)] $Y$ is a random variable with respect to $\mathcal{G}$, but $X$ is not a random variable with respect to $\mathcal{G}$.
    \item [(C)] Neither $X$ nor $Y$ is a random variable with respect to $\mathcal{G}$.
    \item [(D)] Both $X$ and $Y$ are random variables with respect to $\mathcal{G}$.
\end{enumerate} \hfill (GATE ST 2023)\\
\solution
%\input{gate/ST/2023/14/main.tex}
	\item  A die is loaded in such a way that each odd number is twice as likely to occur as
each even number. Find $P(G)$, where $G$ is the event that a number greater than
3 occurs on a single roll of the die.
\\
\solution
		%\input{exemplar/11/16/3/5/main.tex}
	\item All the jacks, queens and kings are removed from a deck of 52 playing cards. The remaining cards are well shuffled and then one card is drawn at random. Giving ace a value 1 similar value for other cards, find the probability that the card has a value 
		\begin{enumerate}
			\item 7
			\item greater than 7
			\item less than 7
		\end{enumerate}
		%\input{exemplar/10/13/3/30/main.tex}
  \item A Lot consists of 48 mobile phones of which 42 are good, 3 have only minor defects and 3 have major defects.Varnika will buy a phone if it is good but the trader will only buy a mobile if it has no major defects. One phone is selected at random from the lot. What is the probability that it is
\begin{enumerate}
	\item acceptable to Varnika?
            \item acceptable to the trader?
\end{enumerate}
\solution
	%\input{exemplar/10/13/3/40/main.tex}
 \item A student says that if you throw a die, it will show up 1 or not 1. Therefore, the probability of getting 1 and the probability of getting 'not 1' each is equal to $\frac{1}{2}$. Is this correct? Give reasons.\\
 \solution
        %\input{exemplar/10/13/2/9/main.tex}
   \item Four candidates A, B, C, D have ap-
plied for the assignment to coach a school cricket
team. If A is twice as likely to be selected as B, and
B and C are given about the same chance of being
selected, while C is twice as likely to be selected
as D, what are the probabilities that
\begin{enumerate}
\item C will be selected?
\item A will not be selected?
\end{enumerate}
	%\input{exemplar/11/16/3/9/main.tex}
 \item A bag contain 24 balls of which $x$ balls are red, $2x$ are white and $3x$ are blue. A ball is selected at random, What is the probability that it is
\begin{enumerate}[label=\alph*)]
\item not red ?
\item white ?
\end{enumerate}
%\input{exemplar/10/13/3/41/main.tex}
If the letters of the word ASSASSINATION are arranged at random. Find the Probability that
\begin{enumerate}[label=(\alph*)]
\item Four $S's$ come consecutively in the word
\item Two  $I's$ and two $N's$ come together
\item All $A's$ are not coming together
\item No two $A's$ are coming together
\end{enumerate}
%\input{exemplar/11/16/3/14/main.tex}
	\item One urn contains two black balls (labelled B1 and B2) and one white ball. A
	second urn contains one black ball and two white balls (labelled W1 and W2).
	Suppose the following experiment is performed. One of the two urns is chosen
	at random. Next a ball is randomly chosen from the urn. Then a second ball is
	chosen at random from the same urn without replacing the first ball.
	
	\begin{enumerate}
	\item What is the probability that two black balls are chosen?
	
	\item What is the probability that two balls of opposite colour are chosen?
	\end{enumerate}
	\solution
	%\input{exemplar/11/16/3/12/main1.tex}
\end{enumerate}

	\item A card is selected from a pack of 52 cards.
 \begin{enumerate}[label=(\alph*)] 
                 \item How many points are there in the sample space?
                 \item Calculate the probability that the card is an ace of spades.
                 \item Calculate the probability that the card is (i) an ace and (ii) black card.
 \end{enumerate}
\solution
		%\begin{table}[H]
	\centering
\begin{tabular}{|c|c|c|}
\hline
Random variable &Value &Definition\\ \hline
\multirow{3}{*}{X} &0 &Slips of Rs 1\\
&1 &Slips of Rs 5\\
&2 &Slips of Rs 13\\ \hline
\multirow{2}{*}{Y} &0 &Box A\\
&1 &Box B\\\hline
\end{tabular}
\caption{}
\label{tab:Distribution}
\end{table}
See \tabref{tab:Distribution}.
\begin{align}
p_{Y}\brak{k}= \begin{cases} 
      \frac{1}{3} & {k=0} \\
      \frac{2}{3 }& {k=1} 
   \end{cases}
   \\
p_{Y|X}\brak{0|0} = \frac{19}{25}\, 
p_{Y|X}\brak{0|1} = \frac{6}{25}\,
p_{Y|X}\brak{1|0} = \frac{45}{50}\,
p_{Y|X}\brak{1|2} = \frac{5}{50}
\end{align}
The desired probability is the probability that a slip drawn at random is marked other than Rs 1,
\begin{align}
&=1-p_X\brak{0}\\
&= p_X(1) + p_X(2)
\end{align}
Using Bayes theorem,
\begin{align}
&= p_Y\brak{0} \times \pr{Y=0 | X=1} + p_Y\brak{1} \times \pr{Y=1|X=2}\\
&=\frac{1}{3} \times \frac{6}{25} + \frac{2}{3} \times \frac{5}{50}\\
&=\frac{11}{75}
\end{align}

\newpage

%\tableofcontents

\bigskip

\renewcommand{\thefigure}{\theenumi}
\renewcommand{\thetable}{\theenumi}
%\renewcommand{\theequation}{\theenumi}

%\begin{abstract}
%%\boldmath
%In this letter, an algorithm for evaluating the exact analytical bit error rate  (BER)  for the piecewise linear (PL) combiner for  multiple relays is presented. Previous results were available only for upto three relays. The algorithm is unique in the sense that  the actual mathematical expressions, that are prohibitively large, need not be explicitly obtained. The diversity gain due to multiple relays is shown through plots of the analytical BER, well supported by simulations. 
%
%\end{abstract}
% IEEEtran.cls defaults to using nonbold math in the Abstract.
% This preserves the distinction between vectors and scalars. However,
% if the journal you are submitting to favors bold math in the abstract,
% then you can use LaTeX's standard command \boldmath at the very start
% of the abstract to achieve this. Many IEEE journals frown on math
% in the abstract anyway.

% Note that keywords are not normally used for peerreview papers.
%\begin{IEEEkeywords}
%Cooperative diversity, decode and forward, piecewise linear
%\end{IEEEkeywords}



% For peer review papers, you can put extra information on the cover
% page as needed:
% \ifCLASSOPTIONpeerreview
% \begin{center} \bfseries EDICS Category: 3-BBND \end{center}
% \fi
%
% For peerreview papers, this IEEEtran command inserts a page break and
% creates the second title. It will be ignored for other modes.
%\IEEEpeerreviewmaketitle




\item Four cards are drawn from a well-shuffled deck of 52 cards. What is the probability of obtaining 3 diamonds and one spade.
\\
\solution
		%\begin{enumerate}[label=\thesection.\arabic*,ref=\thesection.\theenumi]
	\item One card is drawn from a well-shuffled deck of 52 cards. Find the probability of getting
\begin{enumerate}
\item A king of red colour 
\item A face card 
\item A red face card
\item The jack of hearts
\item A spade
\item The queen of diamonds

\end{enumerate}
\solution
		%\input{ncert/10/15/1/14/main.tex}
	\item Five cards—the ten, jack, queen, king and ace of diamonds, are well-shuffled with their face downwards. One card is then picked up at random.
\begin{enumerate}
\item
What is the probability that the card is the queen? 
\item
If the queen is drawn and put aside, what is the probability that the second card picked up is (a) an ace? (b) a queen?\\
\end{enumerate}
\solution
		%\input{ncert/10/15/1/15/defs.tex}
	\item A bag contains $5$ red balls and some blue balls. If the probability of drawing a blue ball is double that if a red ball, determine the number of blue balls in the bag. 
		\\
\solution
		%\input{ncert/10/15/2/3/defs.tex}
	\item A card is selected from a pack of 52 cards.
 \begin{enumerate}[label=(\alph*)] 
                 \item How many points are there in the sample space?
                 \item Calculate the probability that the card is an ace of spades.
                 \item Calculate the probability that the card is (i) an ace and (ii) black card.
 \end{enumerate}
\solution
		%\input{ncert/11/16/3/4/main.tex}
\item Four cards are drawn from a well-shuffled deck of 52 cards. What is the probability of obtaining 3 diamonds and one spade.
\\
\solution
		%\input{ncert/11/16/4/2/defs.tex}
\item In a certain lottery 10,000 tickets are sold and ten equal prizes are awarded. What is the probability of not getting a prize if you buy (a) one ticket (b) two tickets (c) 10 tickets ?	
\\
\solution
		%\input{ncert/11/16/4/4/defs.tex}
		%
\item 
Out of 100 students, two sections of 40 and 60 are formed. If you and your friend are among the 100 students, what is the probability that
\begin{enumerate}
\item you both enter the same section?
\item you both enter the different sections?
\end{enumerate}
\solution
		%\input{ncert/11/16/4/5/defs.tex}
	\item 
The number lock of a suitcase has 4 wheels each labelled with ten digits i.e. from 0 to 9.The lock opens with a sequence of four digits with no repeats.What is the probability of a person getting the right sequence to open the suitcase.
\\
\solution
		%\input{ncert/11/16/4/10/defs.tex}
		%
\item 
Two cards are drawn at random and without replacement from a pack of 52 playing cards. Find the probability that both the cards are black.
\\
\solution
		%\input{ncert/12/13/2/2/defs.tex}
		\item A box of oranges is inspected by examining three randomly selected oranges drawn without replacement. If all the three oranges are good, the box is approved for sale, otherwise, it is rejected. Find the probability that a box containing 15 oranges out of which 12 are good and 3 are bad ones will be approved for sale.
		\label{ncert/12/13/2/3/defs.tex}
		\item Two balls are drawn at random with replacement from a box containing 10 black and 8 red balls. Find the probability that
		\label{ncert/12/13/2/12}
\begin{enumerate}
\item both balls are red.
\item first ball is black and second is red.
\item one of them is black and other is red.
\end{enumerate}

\item In a hostel, 60\% of the students read Hindi newspaper, 40\% read English newspaper and 20\% read both Hindi and English newspapers. A student is selected at random.
		\label{ncert/12/13/2/15}
\begin{enumerate}
\item Find the probability that she reads neither Hindi nor English newspapers.
\item If she reads Hindi newspaper, find the probability that she reads English newspaper.
\item If she reads English newspaper, find the probability that she reads Hindi newspaper.\\
\end{enumerate}
\item The probability of obtaining an even prime number on each die, when a pair of dice is rolled is 
\begin{enumerate}
    \item $0$ 
    
    \item $\frac{1}{3}$ 
    
    \item $\frac{1}{12}$ 
    
    \item $\frac{1}{36}$ 
\end{enumerate}
\solution
		%\input{ncert/12/13/2/17/defs.tex}
	\item A bag contains 4 red and 4 black balls, another bag contains 2 red and 6 black balls. One of the two bags is selected at random and a ball is drawn from the bag which is found to be red. Find the probability that the ball is drawn from the first bag.
\\
\solution
		%\input{ncert/12/13/3/2/main.tex}
  \item
  Cards with numbers 2 to 101 are placed in a box. A card is selected at random.Find the probability that the card has
\begin{enumerate}[label=(\roman*)]
	\item an even number 
	\item a square number
\end{enumerate}
\solution
%\input{exemplar/10/13/3/32/main.tex}
\item
The king, queen and jack of clubs are removed from a deck of 52 playing cards and then well shuffled. Now one card is drawn at random from the remaining cards.  Determine the probability that the card is
\begin{enumerate}[label=(\roman*)]
\item a club
\item 10 of hearts
\end{enumerate}
\solution
%\input{exemplar/10/13/3/29/main.tex}
\item A team of medical students doing their internship have to assist during surgeries
at a city hospital. The probabilities of surgeries rated as very complex, complex,
routine, simple or very simple are respectively, 0.15, 0.20, 0.31, 0.26, .08. Find
the probabilities that a particular surgery will be rated
\begin{enumerate}
	\item complex or very complex;
	\item neither very complex nor very simple;
	\item routine or complex
	\item routine or simple
\end{enumerate}
\solution
%\input{exemplar/11/16/3/8(1)/main.tex}
\item A card is selected from a pack of 52 cards.
\begin{enumerate}[label=(\alph*)]
    \item How many points are there in the sample space?
    \item Calculate the probability that the card is an ace of spades.
    \item Calculate the probability that the card is (i) an ace and (ii) black card.
\end{enumerate}
\solution
%\input{exemplar/11/16/3/4/main2.tex}
\item The probability that a non leap year selected at random will contain 53 sundays.
\\
\solution
%\input{exemplar/10/13/1/19/main.tex}
\item One of the four persons John, Rita, Aslam or Gurpreet will be promoted next
month. Consequently the sample space consists of four elementary outcomes
S = {John promoted, Rita promoted, Aslam promoted, Gurpreet promoted}
You are told that the chances of John’s promotion is same as that of Gurpreet,
Rita’s chances of promotion are twice as likely as Johns. Aslam’s chances are
four times that of John.
\begin{enumerate}
	\item Determine
	\begin{enumerate}
		\item P (John promoted)
		\item P (Rita promoted)
		\item P (Aslam promoted)
		\item P (Gurpreet promoted)
	\end{enumerate}
	\item If A = {John promoted or Gurpreet promoted}, find P (A).
\end{enumerate}
\solution
%\input{exemplar/11/16/3/10/main.tex}
\item A card is drawn from a deck of 52 cards. Find the probability of getting a king or a heart or a red card.\\
\solution
%\input{exemplar/11/16/3/15/main.tex}
\item The probability that a student will pass his examination is 0.73, the probability of
the student getting a compartment is 0.13, and the probability that the student will
either pass or get compartment is 0.96. State True or False.\\
\solution
%\input{exemplar/11/16/3/31/main.tex}
\item A card is selected from a pack of 52 cards\\
\begin{enumerate}[label=(\alph*)]
\item How many points are there in the sample space?
\item Calculate the probability that the cards is an ace of spades.
\item Calculate the probability that the card is (i) an ace (ii)black card.\\
\end{enumerate}
%\input{ncert/11/16/3/4_1/Prob_4.tex}
\item In a non-leap year, the probability of having 53 tuesdays or 53 wednesdays is\\
\solution
%\input{exemplar/11/16/3/18/main.tex}
\item There are 1000 sealed envelopes in a box, 10 of them contain a cash prize of
Rs 100 each, 100 of them contain a cash prize of Rs 50 each and 200 of them
contain a cash prize of Rs 10 each and rest do not contain any cash prize. If they
are well shuffled and an envelope is picked up out, what is the probability that it
contains no cash prize?\\
\solution
%\input{exemplar/10/13/3/34/main.tex}
\item 
A die is thrown and a card is selected at random from a deck of 52 playing cards. The probability of getting an even number on the die and a spade card.\\
\solution
%\input{exemplar/12/13/3/78/main.tex}
\item
If 4-digit numbers greater than 5,000 are randomly formed from the digits 0, 1, 3, 5, and 7, what is the probability of forming a number divisible by 5 when:
\begin{enumerate}
    \item The digits are repeated?
    \item The repetition of digits is not allowed?
\end{enumerate}
\solution
%\input{ncert/11/16/4/9/main.tex}
\item Consider the probability space $\brak{\Omega, \mathcal{G}, P}$ where $\Omega = [0,2]$ and $\mathcal{G} = \cbrak{\phi, \Omega, [0,1], (1,2]}$. Let $X$ and $Y$ be two functions on $\Omega$ defined as
\begin{align*}
    X(\omega) = 
    \begin{cases}
        1 & \text{if }\omega \in [0, 1]\\
        2 & \text{if }\omega \in (1, 2]
    \end{cases}
\end{align*}
and
\begin{align*}
    Y(\omega) = 
    \begin{cases}
        2 & \text{if }\omega \in [0, 1.5]\\
        3 & \text{if }\omega \in (1.5, 2].
    \end{cases}
\end{align*}
Then which one of the following statements is true?
\begin{enumerate}
    \item [(A)] $X$ is a random variable with respect to $\mathcal{G}$, but $Y$ is not a random variable with respect to $\mathcal{G}$.
    \item [(B)] $Y$ is a random variable with respect to $\mathcal{G}$, but $X$ is not a random variable with respect to $\mathcal{G}$.
    \item [(C)] Neither $X$ nor $Y$ is a random variable with respect to $\mathcal{G}$.
    \item [(D)] Both $X$ and $Y$ are random variables with respect to $\mathcal{G}$.
\end{enumerate} \hfill (GATE ST 2023)\\
\solution
%\input{gate/ST/2023/14/main.tex}
	\item  A die is loaded in such a way that each odd number is twice as likely to occur as
each even number. Find $P(G)$, where $G$ is the event that a number greater than
3 occurs on a single roll of the die.
\\
\solution
		%\input{exemplar/11/16/3/5/main.tex}
	\item All the jacks, queens and kings are removed from a deck of 52 playing cards. The remaining cards are well shuffled and then one card is drawn at random. Giving ace a value 1 similar value for other cards, find the probability that the card has a value 
		\begin{enumerate}
			\item 7
			\item greater than 7
			\item less than 7
		\end{enumerate}
		%\input{exemplar/10/13/3/30/main.tex}
  \item A Lot consists of 48 mobile phones of which 42 are good, 3 have only minor defects and 3 have major defects.Varnika will buy a phone if it is good but the trader will only buy a mobile if it has no major defects. One phone is selected at random from the lot. What is the probability that it is
\begin{enumerate}
	\item acceptable to Varnika?
            \item acceptable to the trader?
\end{enumerate}
\solution
	%\input{exemplar/10/13/3/40/main.tex}
 \item A student says that if you throw a die, it will show up 1 or not 1. Therefore, the probability of getting 1 and the probability of getting 'not 1' each is equal to $\frac{1}{2}$. Is this correct? Give reasons.\\
 \solution
        %\input{exemplar/10/13/2/9/main.tex}
   \item Four candidates A, B, C, D have ap-
plied for the assignment to coach a school cricket
team. If A is twice as likely to be selected as B, and
B and C are given about the same chance of being
selected, while C is twice as likely to be selected
as D, what are the probabilities that
\begin{enumerate}
\item C will be selected?
\item A will not be selected?
\end{enumerate}
	%\input{exemplar/11/16/3/9/main.tex}
 \item A bag contain 24 balls of which $x$ balls are red, $2x$ are white and $3x$ are blue. A ball is selected at random, What is the probability that it is
\begin{enumerate}[label=\alph*)]
\item not red ?
\item white ?
\end{enumerate}
%\input{exemplar/10/13/3/41/main.tex}
If the letters of the word ASSASSINATION are arranged at random. Find the Probability that
\begin{enumerate}[label=(\alph*)]
\item Four $S's$ come consecutively in the word
\item Two  $I's$ and two $N's$ come together
\item All $A's$ are not coming together
\item No two $A's$ are coming together
\end{enumerate}
%\input{exemplar/11/16/3/14/main.tex}
	\item One urn contains two black balls (labelled B1 and B2) and one white ball. A
	second urn contains one black ball and two white balls (labelled W1 and W2).
	Suppose the following experiment is performed. One of the two urns is chosen
	at random. Next a ball is randomly chosen from the urn. Then a second ball is
	chosen at random from the same urn without replacing the first ball.
	
	\begin{enumerate}
	\item What is the probability that two black balls are chosen?
	
	\item What is the probability that two balls of opposite colour are chosen?
	\end{enumerate}
	\solution
	%\input{exemplar/11/16/3/12/main1.tex}
\end{enumerate}

\item In a certain lottery 10,000 tickets are sold and ten equal prizes are awarded. What is the probability of not getting a prize if you buy (a) one ticket (b) two tickets (c) 10 tickets ?	
\\
\solution
		%\begin{enumerate}[label=\thesection.\arabic*,ref=\thesection.\theenumi]
	\item One card is drawn from a well-shuffled deck of 52 cards. Find the probability of getting
\begin{enumerate}
\item A king of red colour 
\item A face card 
\item A red face card
\item The jack of hearts
\item A spade
\item The queen of diamonds

\end{enumerate}
\solution
		%\input{ncert/10/15/1/14/main.tex}
	\item Five cards—the ten, jack, queen, king and ace of diamonds, are well-shuffled with their face downwards. One card is then picked up at random.
\begin{enumerate}
\item
What is the probability that the card is the queen? 
\item
If the queen is drawn and put aside, what is the probability that the second card picked up is (a) an ace? (b) a queen?\\
\end{enumerate}
\solution
		%\input{ncert/10/15/1/15/defs.tex}
	\item A bag contains $5$ red balls and some blue balls. If the probability of drawing a blue ball is double that if a red ball, determine the number of blue balls in the bag. 
		\\
\solution
		%\input{ncert/10/15/2/3/defs.tex}
	\item A card is selected from a pack of 52 cards.
 \begin{enumerate}[label=(\alph*)] 
                 \item How many points are there in the sample space?
                 \item Calculate the probability that the card is an ace of spades.
                 \item Calculate the probability that the card is (i) an ace and (ii) black card.
 \end{enumerate}
\solution
		%\input{ncert/11/16/3/4/main.tex}
\item Four cards are drawn from a well-shuffled deck of 52 cards. What is the probability of obtaining 3 diamonds and one spade.
\\
\solution
		%\input{ncert/11/16/4/2/defs.tex}
\item In a certain lottery 10,000 tickets are sold and ten equal prizes are awarded. What is the probability of not getting a prize if you buy (a) one ticket (b) two tickets (c) 10 tickets ?	
\\
\solution
		%\input{ncert/11/16/4/4/defs.tex}
		%
\item 
Out of 100 students, two sections of 40 and 60 are formed. If you and your friend are among the 100 students, what is the probability that
\begin{enumerate}
\item you both enter the same section?
\item you both enter the different sections?
\end{enumerate}
\solution
		%\input{ncert/11/16/4/5/defs.tex}
	\item 
The number lock of a suitcase has 4 wheels each labelled with ten digits i.e. from 0 to 9.The lock opens with a sequence of four digits with no repeats.What is the probability of a person getting the right sequence to open the suitcase.
\\
\solution
		%\input{ncert/11/16/4/10/defs.tex}
		%
\item 
Two cards are drawn at random and without replacement from a pack of 52 playing cards. Find the probability that both the cards are black.
\\
\solution
		%\input{ncert/12/13/2/2/defs.tex}
		\item A box of oranges is inspected by examining three randomly selected oranges drawn without replacement. If all the three oranges are good, the box is approved for sale, otherwise, it is rejected. Find the probability that a box containing 15 oranges out of which 12 are good and 3 are bad ones will be approved for sale.
		\label{ncert/12/13/2/3/defs.tex}
		\item Two balls are drawn at random with replacement from a box containing 10 black and 8 red balls. Find the probability that
		\label{ncert/12/13/2/12}
\begin{enumerate}
\item both balls are red.
\item first ball is black and second is red.
\item one of them is black and other is red.
\end{enumerate}

\item In a hostel, 60\% of the students read Hindi newspaper, 40\% read English newspaper and 20\% read both Hindi and English newspapers. A student is selected at random.
		\label{ncert/12/13/2/15}
\begin{enumerate}
\item Find the probability that she reads neither Hindi nor English newspapers.
\item If she reads Hindi newspaper, find the probability that she reads English newspaper.
\item If she reads English newspaper, find the probability that she reads Hindi newspaper.\\
\end{enumerate}
\item The probability of obtaining an even prime number on each die, when a pair of dice is rolled is 
\begin{enumerate}
    \item $0$ 
    
    \item $\frac{1}{3}$ 
    
    \item $\frac{1}{12}$ 
    
    \item $\frac{1}{36}$ 
\end{enumerate}
\solution
		%\input{ncert/12/13/2/17/defs.tex}
	\item A bag contains 4 red and 4 black balls, another bag contains 2 red and 6 black balls. One of the two bags is selected at random and a ball is drawn from the bag which is found to be red. Find the probability that the ball is drawn from the first bag.
\\
\solution
		%\input{ncert/12/13/3/2/main.tex}
  \item
  Cards with numbers 2 to 101 are placed in a box. A card is selected at random.Find the probability that the card has
\begin{enumerate}[label=(\roman*)]
	\item an even number 
	\item a square number
\end{enumerate}
\solution
%\input{exemplar/10/13/3/32/main.tex}
\item
The king, queen and jack of clubs are removed from a deck of 52 playing cards and then well shuffled. Now one card is drawn at random from the remaining cards.  Determine the probability that the card is
\begin{enumerate}[label=(\roman*)]
\item a club
\item 10 of hearts
\end{enumerate}
\solution
%\input{exemplar/10/13/3/29/main.tex}
\item A team of medical students doing their internship have to assist during surgeries
at a city hospital. The probabilities of surgeries rated as very complex, complex,
routine, simple or very simple are respectively, 0.15, 0.20, 0.31, 0.26, .08. Find
the probabilities that a particular surgery will be rated
\begin{enumerate}
	\item complex or very complex;
	\item neither very complex nor very simple;
	\item routine or complex
	\item routine or simple
\end{enumerate}
\solution
%\input{exemplar/11/16/3/8(1)/main.tex}
\item A card is selected from a pack of 52 cards.
\begin{enumerate}[label=(\alph*)]
    \item How many points are there in the sample space?
    \item Calculate the probability that the card is an ace of spades.
    \item Calculate the probability that the card is (i) an ace and (ii) black card.
\end{enumerate}
\solution
%\input{exemplar/11/16/3/4/main2.tex}
\item The probability that a non leap year selected at random will contain 53 sundays.
\\
\solution
%\input{exemplar/10/13/1/19/main.tex}
\item One of the four persons John, Rita, Aslam or Gurpreet will be promoted next
month. Consequently the sample space consists of four elementary outcomes
S = {John promoted, Rita promoted, Aslam promoted, Gurpreet promoted}
You are told that the chances of John’s promotion is same as that of Gurpreet,
Rita’s chances of promotion are twice as likely as Johns. Aslam’s chances are
four times that of John.
\begin{enumerate}
	\item Determine
	\begin{enumerate}
		\item P (John promoted)
		\item P (Rita promoted)
		\item P (Aslam promoted)
		\item P (Gurpreet promoted)
	\end{enumerate}
	\item If A = {John promoted or Gurpreet promoted}, find P (A).
\end{enumerate}
\solution
%\input{exemplar/11/16/3/10/main.tex}
\item A card is drawn from a deck of 52 cards. Find the probability of getting a king or a heart or a red card.\\
\solution
%\input{exemplar/11/16/3/15/main.tex}
\item The probability that a student will pass his examination is 0.73, the probability of
the student getting a compartment is 0.13, and the probability that the student will
either pass or get compartment is 0.96. State True or False.\\
\solution
%\input{exemplar/11/16/3/31/main.tex}
\item A card is selected from a pack of 52 cards\\
\begin{enumerate}[label=(\alph*)]
\item How many points are there in the sample space?
\item Calculate the probability that the cards is an ace of spades.
\item Calculate the probability that the card is (i) an ace (ii)black card.\\
\end{enumerate}
%\input{ncert/11/16/3/4_1/Prob_4.tex}
\item In a non-leap year, the probability of having 53 tuesdays or 53 wednesdays is\\
\solution
%\input{exemplar/11/16/3/18/main.tex}
\item There are 1000 sealed envelopes in a box, 10 of them contain a cash prize of
Rs 100 each, 100 of them contain a cash prize of Rs 50 each and 200 of them
contain a cash prize of Rs 10 each and rest do not contain any cash prize. If they
are well shuffled and an envelope is picked up out, what is the probability that it
contains no cash prize?\\
\solution
%\input{exemplar/10/13/3/34/main.tex}
\item 
A die is thrown and a card is selected at random from a deck of 52 playing cards. The probability of getting an even number on the die and a spade card.\\
\solution
%\input{exemplar/12/13/3/78/main.tex}
\item
If 4-digit numbers greater than 5,000 are randomly formed from the digits 0, 1, 3, 5, and 7, what is the probability of forming a number divisible by 5 when:
\begin{enumerate}
    \item The digits are repeated?
    \item The repetition of digits is not allowed?
\end{enumerate}
\solution
%\input{ncert/11/16/4/9/main.tex}
\item Consider the probability space $\brak{\Omega, \mathcal{G}, P}$ where $\Omega = [0,2]$ and $\mathcal{G} = \cbrak{\phi, \Omega, [0,1], (1,2]}$. Let $X$ and $Y$ be two functions on $\Omega$ defined as
\begin{align*}
    X(\omega) = 
    \begin{cases}
        1 & \text{if }\omega \in [0, 1]\\
        2 & \text{if }\omega \in (1, 2]
    \end{cases}
\end{align*}
and
\begin{align*}
    Y(\omega) = 
    \begin{cases}
        2 & \text{if }\omega \in [0, 1.5]\\
        3 & \text{if }\omega \in (1.5, 2].
    \end{cases}
\end{align*}
Then which one of the following statements is true?
\begin{enumerate}
    \item [(A)] $X$ is a random variable with respect to $\mathcal{G}$, but $Y$ is not a random variable with respect to $\mathcal{G}$.
    \item [(B)] $Y$ is a random variable with respect to $\mathcal{G}$, but $X$ is not a random variable with respect to $\mathcal{G}$.
    \item [(C)] Neither $X$ nor $Y$ is a random variable with respect to $\mathcal{G}$.
    \item [(D)] Both $X$ and $Y$ are random variables with respect to $\mathcal{G}$.
\end{enumerate} \hfill (GATE ST 2023)\\
\solution
%\input{gate/ST/2023/14/main.tex}
	\item  A die is loaded in such a way that each odd number is twice as likely to occur as
each even number. Find $P(G)$, where $G$ is the event that a number greater than
3 occurs on a single roll of the die.
\\
\solution
		%\input{exemplar/11/16/3/5/main.tex}
	\item All the jacks, queens and kings are removed from a deck of 52 playing cards. The remaining cards are well shuffled and then one card is drawn at random. Giving ace a value 1 similar value for other cards, find the probability that the card has a value 
		\begin{enumerate}
			\item 7
			\item greater than 7
			\item less than 7
		\end{enumerate}
		%\input{exemplar/10/13/3/30/main.tex}
  \item A Lot consists of 48 mobile phones of which 42 are good, 3 have only minor defects and 3 have major defects.Varnika will buy a phone if it is good but the trader will only buy a mobile if it has no major defects. One phone is selected at random from the lot. What is the probability that it is
\begin{enumerate}
	\item acceptable to Varnika?
            \item acceptable to the trader?
\end{enumerate}
\solution
	%\input{exemplar/10/13/3/40/main.tex}
 \item A student says that if you throw a die, it will show up 1 or not 1. Therefore, the probability of getting 1 and the probability of getting 'not 1' each is equal to $\frac{1}{2}$. Is this correct? Give reasons.\\
 \solution
        %\input{exemplar/10/13/2/9/main.tex}
   \item Four candidates A, B, C, D have ap-
plied for the assignment to coach a school cricket
team. If A is twice as likely to be selected as B, and
B and C are given about the same chance of being
selected, while C is twice as likely to be selected
as D, what are the probabilities that
\begin{enumerate}
\item C will be selected?
\item A will not be selected?
\end{enumerate}
	%\input{exemplar/11/16/3/9/main.tex}
 \item A bag contain 24 balls of which $x$ balls are red, $2x$ are white and $3x$ are blue. A ball is selected at random, What is the probability that it is
\begin{enumerate}[label=\alph*)]
\item not red ?
\item white ?
\end{enumerate}
%\input{exemplar/10/13/3/41/main.tex}
If the letters of the word ASSASSINATION are arranged at random. Find the Probability that
\begin{enumerate}[label=(\alph*)]
\item Four $S's$ come consecutively in the word
\item Two  $I's$ and two $N's$ come together
\item All $A's$ are not coming together
\item No two $A's$ are coming together
\end{enumerate}
%\input{exemplar/11/16/3/14/main.tex}
	\item One urn contains two black balls (labelled B1 and B2) and one white ball. A
	second urn contains one black ball and two white balls (labelled W1 and W2).
	Suppose the following experiment is performed. One of the two urns is chosen
	at random. Next a ball is randomly chosen from the urn. Then a second ball is
	chosen at random from the same urn without replacing the first ball.
	
	\begin{enumerate}
	\item What is the probability that two black balls are chosen?
	
	\item What is the probability that two balls of opposite colour are chosen?
	\end{enumerate}
	\solution
	%\input{exemplar/11/16/3/12/main1.tex}
\end{enumerate}

		%
\item 
Out of 100 students, two sections of 40 and 60 are formed. If you and your friend are among the 100 students, what is the probability that
\begin{enumerate}
\item you both enter the same section?
\item you both enter the different sections?
\end{enumerate}
\solution
		%\begin{enumerate}[label=\thesection.\arabic*,ref=\thesection.\theenumi]
	\item One card is drawn from a well-shuffled deck of 52 cards. Find the probability of getting
\begin{enumerate}
\item A king of red colour 
\item A face card 
\item A red face card
\item The jack of hearts
\item A spade
\item The queen of diamonds

\end{enumerate}
\solution
		%\input{ncert/10/15/1/14/main.tex}
	\item Five cards—the ten, jack, queen, king and ace of diamonds, are well-shuffled with their face downwards. One card is then picked up at random.
\begin{enumerate}
\item
What is the probability that the card is the queen? 
\item
If the queen is drawn and put aside, what is the probability that the second card picked up is (a) an ace? (b) a queen?\\
\end{enumerate}
\solution
		%\input{ncert/10/15/1/15/defs.tex}
	\item A bag contains $5$ red balls and some blue balls. If the probability of drawing a blue ball is double that if a red ball, determine the number of blue balls in the bag. 
		\\
\solution
		%\input{ncert/10/15/2/3/defs.tex}
	\item A card is selected from a pack of 52 cards.
 \begin{enumerate}[label=(\alph*)] 
                 \item How many points are there in the sample space?
                 \item Calculate the probability that the card is an ace of spades.
                 \item Calculate the probability that the card is (i) an ace and (ii) black card.
 \end{enumerate}
\solution
		%\input{ncert/11/16/3/4/main.tex}
\item Four cards are drawn from a well-shuffled deck of 52 cards. What is the probability of obtaining 3 diamonds and one spade.
\\
\solution
		%\input{ncert/11/16/4/2/defs.tex}
\item In a certain lottery 10,000 tickets are sold and ten equal prizes are awarded. What is the probability of not getting a prize if you buy (a) one ticket (b) two tickets (c) 10 tickets ?	
\\
\solution
		%\input{ncert/11/16/4/4/defs.tex}
		%
\item 
Out of 100 students, two sections of 40 and 60 are formed. If you and your friend are among the 100 students, what is the probability that
\begin{enumerate}
\item you both enter the same section?
\item you both enter the different sections?
\end{enumerate}
\solution
		%\input{ncert/11/16/4/5/defs.tex}
	\item 
The number lock of a suitcase has 4 wheels each labelled with ten digits i.e. from 0 to 9.The lock opens with a sequence of four digits with no repeats.What is the probability of a person getting the right sequence to open the suitcase.
\\
\solution
		%\input{ncert/11/16/4/10/defs.tex}
		%
\item 
Two cards are drawn at random and without replacement from a pack of 52 playing cards. Find the probability that both the cards are black.
\\
\solution
		%\input{ncert/12/13/2/2/defs.tex}
		\item A box of oranges is inspected by examining three randomly selected oranges drawn without replacement. If all the three oranges are good, the box is approved for sale, otherwise, it is rejected. Find the probability that a box containing 15 oranges out of which 12 are good and 3 are bad ones will be approved for sale.
		\label{ncert/12/13/2/3/defs.tex}
		\item Two balls are drawn at random with replacement from a box containing 10 black and 8 red balls. Find the probability that
		\label{ncert/12/13/2/12}
\begin{enumerate}
\item both balls are red.
\item first ball is black and second is red.
\item one of them is black and other is red.
\end{enumerate}

\item In a hostel, 60\% of the students read Hindi newspaper, 40\% read English newspaper and 20\% read both Hindi and English newspapers. A student is selected at random.
		\label{ncert/12/13/2/15}
\begin{enumerate}
\item Find the probability that she reads neither Hindi nor English newspapers.
\item If she reads Hindi newspaper, find the probability that she reads English newspaper.
\item If she reads English newspaper, find the probability that she reads Hindi newspaper.\\
\end{enumerate}
\item The probability of obtaining an even prime number on each die, when a pair of dice is rolled is 
\begin{enumerate}
    \item $0$ 
    
    \item $\frac{1}{3}$ 
    
    \item $\frac{1}{12}$ 
    
    \item $\frac{1}{36}$ 
\end{enumerate}
\solution
		%\input{ncert/12/13/2/17/defs.tex}
	\item A bag contains 4 red and 4 black balls, another bag contains 2 red and 6 black balls. One of the two bags is selected at random and a ball is drawn from the bag which is found to be red. Find the probability that the ball is drawn from the first bag.
\\
\solution
		%\input{ncert/12/13/3/2/main.tex}
  \item
  Cards with numbers 2 to 101 are placed in a box. A card is selected at random.Find the probability that the card has
\begin{enumerate}[label=(\roman*)]
	\item an even number 
	\item a square number
\end{enumerate}
\solution
%\input{exemplar/10/13/3/32/main.tex}
\item
The king, queen and jack of clubs are removed from a deck of 52 playing cards and then well shuffled. Now one card is drawn at random from the remaining cards.  Determine the probability that the card is
\begin{enumerate}[label=(\roman*)]
\item a club
\item 10 of hearts
\end{enumerate}
\solution
%\input{exemplar/10/13/3/29/main.tex}
\item A team of medical students doing their internship have to assist during surgeries
at a city hospital. The probabilities of surgeries rated as very complex, complex,
routine, simple or very simple are respectively, 0.15, 0.20, 0.31, 0.26, .08. Find
the probabilities that a particular surgery will be rated
\begin{enumerate}
	\item complex or very complex;
	\item neither very complex nor very simple;
	\item routine or complex
	\item routine or simple
\end{enumerate}
\solution
%\input{exemplar/11/16/3/8(1)/main.tex}
\item A card is selected from a pack of 52 cards.
\begin{enumerate}[label=(\alph*)]
    \item How many points are there in the sample space?
    \item Calculate the probability that the card is an ace of spades.
    \item Calculate the probability that the card is (i) an ace and (ii) black card.
\end{enumerate}
\solution
%\input{exemplar/11/16/3/4/main2.tex}
\item The probability that a non leap year selected at random will contain 53 sundays.
\\
\solution
%\input{exemplar/10/13/1/19/main.tex}
\item One of the four persons John, Rita, Aslam or Gurpreet will be promoted next
month. Consequently the sample space consists of four elementary outcomes
S = {John promoted, Rita promoted, Aslam promoted, Gurpreet promoted}
You are told that the chances of John’s promotion is same as that of Gurpreet,
Rita’s chances of promotion are twice as likely as Johns. Aslam’s chances are
four times that of John.
\begin{enumerate}
	\item Determine
	\begin{enumerate}
		\item P (John promoted)
		\item P (Rita promoted)
		\item P (Aslam promoted)
		\item P (Gurpreet promoted)
	\end{enumerate}
	\item If A = {John promoted or Gurpreet promoted}, find P (A).
\end{enumerate}
\solution
%\input{exemplar/11/16/3/10/main.tex}
\item A card is drawn from a deck of 52 cards. Find the probability of getting a king or a heart or a red card.\\
\solution
%\input{exemplar/11/16/3/15/main.tex}
\item The probability that a student will pass his examination is 0.73, the probability of
the student getting a compartment is 0.13, and the probability that the student will
either pass or get compartment is 0.96. State True or False.\\
\solution
%\input{exemplar/11/16/3/31/main.tex}
\item A card is selected from a pack of 52 cards\\
\begin{enumerate}[label=(\alph*)]
\item How many points are there in the sample space?
\item Calculate the probability that the cards is an ace of spades.
\item Calculate the probability that the card is (i) an ace (ii)black card.\\
\end{enumerate}
%\input{ncert/11/16/3/4_1/Prob_4.tex}
\item In a non-leap year, the probability of having 53 tuesdays or 53 wednesdays is\\
\solution
%\input{exemplar/11/16/3/18/main.tex}
\item There are 1000 sealed envelopes in a box, 10 of them contain a cash prize of
Rs 100 each, 100 of them contain a cash prize of Rs 50 each and 200 of them
contain a cash prize of Rs 10 each and rest do not contain any cash prize. If they
are well shuffled and an envelope is picked up out, what is the probability that it
contains no cash prize?\\
\solution
%\input{exemplar/10/13/3/34/main.tex}
\item 
A die is thrown and a card is selected at random from a deck of 52 playing cards. The probability of getting an even number on the die and a spade card.\\
\solution
%\input{exemplar/12/13/3/78/main.tex}
\item
If 4-digit numbers greater than 5,000 are randomly formed from the digits 0, 1, 3, 5, and 7, what is the probability of forming a number divisible by 5 when:
\begin{enumerate}
    \item The digits are repeated?
    \item The repetition of digits is not allowed?
\end{enumerate}
\solution
%\input{ncert/11/16/4/9/main.tex}
\item Consider the probability space $\brak{\Omega, \mathcal{G}, P}$ where $\Omega = [0,2]$ and $\mathcal{G} = \cbrak{\phi, \Omega, [0,1], (1,2]}$. Let $X$ and $Y$ be two functions on $\Omega$ defined as
\begin{align*}
    X(\omega) = 
    \begin{cases}
        1 & \text{if }\omega \in [0, 1]\\
        2 & \text{if }\omega \in (1, 2]
    \end{cases}
\end{align*}
and
\begin{align*}
    Y(\omega) = 
    \begin{cases}
        2 & \text{if }\omega \in [0, 1.5]\\
        3 & \text{if }\omega \in (1.5, 2].
    \end{cases}
\end{align*}
Then which one of the following statements is true?
\begin{enumerate}
    \item [(A)] $X$ is a random variable with respect to $\mathcal{G}$, but $Y$ is not a random variable with respect to $\mathcal{G}$.
    \item [(B)] $Y$ is a random variable with respect to $\mathcal{G}$, but $X$ is not a random variable with respect to $\mathcal{G}$.
    \item [(C)] Neither $X$ nor $Y$ is a random variable with respect to $\mathcal{G}$.
    \item [(D)] Both $X$ and $Y$ are random variables with respect to $\mathcal{G}$.
\end{enumerate} \hfill (GATE ST 2023)\\
\solution
%\input{gate/ST/2023/14/main.tex}
	\item  A die is loaded in such a way that each odd number is twice as likely to occur as
each even number. Find $P(G)$, where $G$ is the event that a number greater than
3 occurs on a single roll of the die.
\\
\solution
		%\input{exemplar/11/16/3/5/main.tex}
	\item All the jacks, queens and kings are removed from a deck of 52 playing cards. The remaining cards are well shuffled and then one card is drawn at random. Giving ace a value 1 similar value for other cards, find the probability that the card has a value 
		\begin{enumerate}
			\item 7
			\item greater than 7
			\item less than 7
		\end{enumerate}
		%\input{exemplar/10/13/3/30/main.tex}
  \item A Lot consists of 48 mobile phones of which 42 are good, 3 have only minor defects and 3 have major defects.Varnika will buy a phone if it is good but the trader will only buy a mobile if it has no major defects. One phone is selected at random from the lot. What is the probability that it is
\begin{enumerate}
	\item acceptable to Varnika?
            \item acceptable to the trader?
\end{enumerate}
\solution
	%\input{exemplar/10/13/3/40/main.tex}
 \item A student says that if you throw a die, it will show up 1 or not 1. Therefore, the probability of getting 1 and the probability of getting 'not 1' each is equal to $\frac{1}{2}$. Is this correct? Give reasons.\\
 \solution
        %\input{exemplar/10/13/2/9/main.tex}
   \item Four candidates A, B, C, D have ap-
plied for the assignment to coach a school cricket
team. If A is twice as likely to be selected as B, and
B and C are given about the same chance of being
selected, while C is twice as likely to be selected
as D, what are the probabilities that
\begin{enumerate}
\item C will be selected?
\item A will not be selected?
\end{enumerate}
	%\input{exemplar/11/16/3/9/main.tex}
 \item A bag contain 24 balls of which $x$ balls are red, $2x$ are white and $3x$ are blue. A ball is selected at random, What is the probability that it is
\begin{enumerate}[label=\alph*)]
\item not red ?
\item white ?
\end{enumerate}
%\input{exemplar/10/13/3/41/main.tex}
If the letters of the word ASSASSINATION are arranged at random. Find the Probability that
\begin{enumerate}[label=(\alph*)]
\item Four $S's$ come consecutively in the word
\item Two  $I's$ and two $N's$ come together
\item All $A's$ are not coming together
\item No two $A's$ are coming together
\end{enumerate}
%\input{exemplar/11/16/3/14/main.tex}
	\item One urn contains two black balls (labelled B1 and B2) and one white ball. A
	second urn contains one black ball and two white balls (labelled W1 and W2).
	Suppose the following experiment is performed. One of the two urns is chosen
	at random. Next a ball is randomly chosen from the urn. Then a second ball is
	chosen at random from the same urn without replacing the first ball.
	
	\begin{enumerate}
	\item What is the probability that two black balls are chosen?
	
	\item What is the probability that two balls of opposite colour are chosen?
	\end{enumerate}
	\solution
	%\input{exemplar/11/16/3/12/main1.tex}
\end{enumerate}

	\item 
The number lock of a suitcase has 4 wheels each labelled with ten digits i.e. from 0 to 9.The lock opens with a sequence of four digits with no repeats.What is the probability of a person getting the right sequence to open the suitcase.
\\
\solution
		%\begin{enumerate}[label=\thesection.\arabic*,ref=\thesection.\theenumi]
	\item One card is drawn from a well-shuffled deck of 52 cards. Find the probability of getting
\begin{enumerate}
\item A king of red colour 
\item A face card 
\item A red face card
\item The jack of hearts
\item A spade
\item The queen of diamonds

\end{enumerate}
\solution
		%\input{ncert/10/15/1/14/main.tex}
	\item Five cards—the ten, jack, queen, king and ace of diamonds, are well-shuffled with their face downwards. One card is then picked up at random.
\begin{enumerate}
\item
What is the probability that the card is the queen? 
\item
If the queen is drawn and put aside, what is the probability that the second card picked up is (a) an ace? (b) a queen?\\
\end{enumerate}
\solution
		%\input{ncert/10/15/1/15/defs.tex}
	\item A bag contains $5$ red balls and some blue balls. If the probability of drawing a blue ball is double that if a red ball, determine the number of blue balls in the bag. 
		\\
\solution
		%\input{ncert/10/15/2/3/defs.tex}
	\item A card is selected from a pack of 52 cards.
 \begin{enumerate}[label=(\alph*)] 
                 \item How many points are there in the sample space?
                 \item Calculate the probability that the card is an ace of spades.
                 \item Calculate the probability that the card is (i) an ace and (ii) black card.
 \end{enumerate}
\solution
		%\input{ncert/11/16/3/4/main.tex}
\item Four cards are drawn from a well-shuffled deck of 52 cards. What is the probability of obtaining 3 diamonds and one spade.
\\
\solution
		%\input{ncert/11/16/4/2/defs.tex}
\item In a certain lottery 10,000 tickets are sold and ten equal prizes are awarded. What is the probability of not getting a prize if you buy (a) one ticket (b) two tickets (c) 10 tickets ?	
\\
\solution
		%\input{ncert/11/16/4/4/defs.tex}
		%
\item 
Out of 100 students, two sections of 40 and 60 are formed. If you and your friend are among the 100 students, what is the probability that
\begin{enumerate}
\item you both enter the same section?
\item you both enter the different sections?
\end{enumerate}
\solution
		%\input{ncert/11/16/4/5/defs.tex}
	\item 
The number lock of a suitcase has 4 wheels each labelled with ten digits i.e. from 0 to 9.The lock opens with a sequence of four digits with no repeats.What is the probability of a person getting the right sequence to open the suitcase.
\\
\solution
		%\input{ncert/11/16/4/10/defs.tex}
		%
\item 
Two cards are drawn at random and without replacement from a pack of 52 playing cards. Find the probability that both the cards are black.
\\
\solution
		%\input{ncert/12/13/2/2/defs.tex}
		\item A box of oranges is inspected by examining three randomly selected oranges drawn without replacement. If all the three oranges are good, the box is approved for sale, otherwise, it is rejected. Find the probability that a box containing 15 oranges out of which 12 are good and 3 are bad ones will be approved for sale.
		\label{ncert/12/13/2/3/defs.tex}
		\item Two balls are drawn at random with replacement from a box containing 10 black and 8 red balls. Find the probability that
		\label{ncert/12/13/2/12}
\begin{enumerate}
\item both balls are red.
\item first ball is black and second is red.
\item one of them is black and other is red.
\end{enumerate}

\item In a hostel, 60\% of the students read Hindi newspaper, 40\% read English newspaper and 20\% read both Hindi and English newspapers. A student is selected at random.
		\label{ncert/12/13/2/15}
\begin{enumerate}
\item Find the probability that she reads neither Hindi nor English newspapers.
\item If she reads Hindi newspaper, find the probability that she reads English newspaper.
\item If she reads English newspaper, find the probability that she reads Hindi newspaper.\\
\end{enumerate}
\item The probability of obtaining an even prime number on each die, when a pair of dice is rolled is 
\begin{enumerate}
    \item $0$ 
    
    \item $\frac{1}{3}$ 
    
    \item $\frac{1}{12}$ 
    
    \item $\frac{1}{36}$ 
\end{enumerate}
\solution
		%\input{ncert/12/13/2/17/defs.tex}
	\item A bag contains 4 red and 4 black balls, another bag contains 2 red and 6 black balls. One of the two bags is selected at random and a ball is drawn from the bag which is found to be red. Find the probability that the ball is drawn from the first bag.
\\
\solution
		%\input{ncert/12/13/3/2/main.tex}
  \item
  Cards with numbers 2 to 101 are placed in a box. A card is selected at random.Find the probability that the card has
\begin{enumerate}[label=(\roman*)]
	\item an even number 
	\item a square number
\end{enumerate}
\solution
%\input{exemplar/10/13/3/32/main.tex}
\item
The king, queen and jack of clubs are removed from a deck of 52 playing cards and then well shuffled. Now one card is drawn at random from the remaining cards.  Determine the probability that the card is
\begin{enumerate}[label=(\roman*)]
\item a club
\item 10 of hearts
\end{enumerate}
\solution
%\input{exemplar/10/13/3/29/main.tex}
\item A team of medical students doing their internship have to assist during surgeries
at a city hospital. The probabilities of surgeries rated as very complex, complex,
routine, simple or very simple are respectively, 0.15, 0.20, 0.31, 0.26, .08. Find
the probabilities that a particular surgery will be rated
\begin{enumerate}
	\item complex or very complex;
	\item neither very complex nor very simple;
	\item routine or complex
	\item routine or simple
\end{enumerate}
\solution
%\input{exemplar/11/16/3/8(1)/main.tex}
\item A card is selected from a pack of 52 cards.
\begin{enumerate}[label=(\alph*)]
    \item How many points are there in the sample space?
    \item Calculate the probability that the card is an ace of spades.
    \item Calculate the probability that the card is (i) an ace and (ii) black card.
\end{enumerate}
\solution
%\input{exemplar/11/16/3/4/main2.tex}
\item The probability that a non leap year selected at random will contain 53 sundays.
\\
\solution
%\input{exemplar/10/13/1/19/main.tex}
\item One of the four persons John, Rita, Aslam or Gurpreet will be promoted next
month. Consequently the sample space consists of four elementary outcomes
S = {John promoted, Rita promoted, Aslam promoted, Gurpreet promoted}
You are told that the chances of John’s promotion is same as that of Gurpreet,
Rita’s chances of promotion are twice as likely as Johns. Aslam’s chances are
four times that of John.
\begin{enumerate}
	\item Determine
	\begin{enumerate}
		\item P (John promoted)
		\item P (Rita promoted)
		\item P (Aslam promoted)
		\item P (Gurpreet promoted)
	\end{enumerate}
	\item If A = {John promoted or Gurpreet promoted}, find P (A).
\end{enumerate}
\solution
%\input{exemplar/11/16/3/10/main.tex}
\item A card is drawn from a deck of 52 cards. Find the probability of getting a king or a heart or a red card.\\
\solution
%\input{exemplar/11/16/3/15/main.tex}
\item The probability that a student will pass his examination is 0.73, the probability of
the student getting a compartment is 0.13, and the probability that the student will
either pass or get compartment is 0.96. State True or False.\\
\solution
%\input{exemplar/11/16/3/31/main.tex}
\item A card is selected from a pack of 52 cards\\
\begin{enumerate}[label=(\alph*)]
\item How many points are there in the sample space?
\item Calculate the probability that the cards is an ace of spades.
\item Calculate the probability that the card is (i) an ace (ii)black card.\\
\end{enumerate}
%\input{ncert/11/16/3/4_1/Prob_4.tex}
\item In a non-leap year, the probability of having 53 tuesdays or 53 wednesdays is\\
\solution
%\input{exemplar/11/16/3/18/main.tex}
\item There are 1000 sealed envelopes in a box, 10 of them contain a cash prize of
Rs 100 each, 100 of them contain a cash prize of Rs 50 each and 200 of them
contain a cash prize of Rs 10 each and rest do not contain any cash prize. If they
are well shuffled and an envelope is picked up out, what is the probability that it
contains no cash prize?\\
\solution
%\input{exemplar/10/13/3/34/main.tex}
\item 
A die is thrown and a card is selected at random from a deck of 52 playing cards. The probability of getting an even number on the die and a spade card.\\
\solution
%\input{exemplar/12/13/3/78/main.tex}
\item
If 4-digit numbers greater than 5,000 are randomly formed from the digits 0, 1, 3, 5, and 7, what is the probability of forming a number divisible by 5 when:
\begin{enumerate}
    \item The digits are repeated?
    \item The repetition of digits is not allowed?
\end{enumerate}
\solution
%\input{ncert/11/16/4/9/main.tex}
\item Consider the probability space $\brak{\Omega, \mathcal{G}, P}$ where $\Omega = [0,2]$ and $\mathcal{G} = \cbrak{\phi, \Omega, [0,1], (1,2]}$. Let $X$ and $Y$ be two functions on $\Omega$ defined as
\begin{align*}
    X(\omega) = 
    \begin{cases}
        1 & \text{if }\omega \in [0, 1]\\
        2 & \text{if }\omega \in (1, 2]
    \end{cases}
\end{align*}
and
\begin{align*}
    Y(\omega) = 
    \begin{cases}
        2 & \text{if }\omega \in [0, 1.5]\\
        3 & \text{if }\omega \in (1.5, 2].
    \end{cases}
\end{align*}
Then which one of the following statements is true?
\begin{enumerate}
    \item [(A)] $X$ is a random variable with respect to $\mathcal{G}$, but $Y$ is not a random variable with respect to $\mathcal{G}$.
    \item [(B)] $Y$ is a random variable with respect to $\mathcal{G}$, but $X$ is not a random variable with respect to $\mathcal{G}$.
    \item [(C)] Neither $X$ nor $Y$ is a random variable with respect to $\mathcal{G}$.
    \item [(D)] Both $X$ and $Y$ are random variables with respect to $\mathcal{G}$.
\end{enumerate} \hfill (GATE ST 2023)\\
\solution
%\input{gate/ST/2023/14/main.tex}
	\item  A die is loaded in such a way that each odd number is twice as likely to occur as
each even number. Find $P(G)$, where $G$ is the event that a number greater than
3 occurs on a single roll of the die.
\\
\solution
		%\input{exemplar/11/16/3/5/main.tex}
	\item All the jacks, queens and kings are removed from a deck of 52 playing cards. The remaining cards are well shuffled and then one card is drawn at random. Giving ace a value 1 similar value for other cards, find the probability that the card has a value 
		\begin{enumerate}
			\item 7
			\item greater than 7
			\item less than 7
		\end{enumerate}
		%\input{exemplar/10/13/3/30/main.tex}
  \item A Lot consists of 48 mobile phones of which 42 are good, 3 have only minor defects and 3 have major defects.Varnika will buy a phone if it is good but the trader will only buy a mobile if it has no major defects. One phone is selected at random from the lot. What is the probability that it is
\begin{enumerate}
	\item acceptable to Varnika?
            \item acceptable to the trader?
\end{enumerate}
\solution
	%\input{exemplar/10/13/3/40/main.tex}
 \item A student says that if you throw a die, it will show up 1 or not 1. Therefore, the probability of getting 1 and the probability of getting 'not 1' each is equal to $\frac{1}{2}$. Is this correct? Give reasons.\\
 \solution
        %\input{exemplar/10/13/2/9/main.tex}
   \item Four candidates A, B, C, D have ap-
plied for the assignment to coach a school cricket
team. If A is twice as likely to be selected as B, and
B and C are given about the same chance of being
selected, while C is twice as likely to be selected
as D, what are the probabilities that
\begin{enumerate}
\item C will be selected?
\item A will not be selected?
\end{enumerate}
	%\input{exemplar/11/16/3/9/main.tex}
 \item A bag contain 24 balls of which $x$ balls are red, $2x$ are white and $3x$ are blue. A ball is selected at random, What is the probability that it is
\begin{enumerate}[label=\alph*)]
\item not red ?
\item white ?
\end{enumerate}
%\input{exemplar/10/13/3/41/main.tex}
If the letters of the word ASSASSINATION are arranged at random. Find the Probability that
\begin{enumerate}[label=(\alph*)]
\item Four $S's$ come consecutively in the word
\item Two  $I's$ and two $N's$ come together
\item All $A's$ are not coming together
\item No two $A's$ are coming together
\end{enumerate}
%\input{exemplar/11/16/3/14/main.tex}
	\item One urn contains two black balls (labelled B1 and B2) and one white ball. A
	second urn contains one black ball and two white balls (labelled W1 and W2).
	Suppose the following experiment is performed. One of the two urns is chosen
	at random. Next a ball is randomly chosen from the urn. Then a second ball is
	chosen at random from the same urn without replacing the first ball.
	
	\begin{enumerate}
	\item What is the probability that two black balls are chosen?
	
	\item What is the probability that two balls of opposite colour are chosen?
	\end{enumerate}
	\solution
	%\input{exemplar/11/16/3/12/main1.tex}
\end{enumerate}

		%
\item 
Two cards are drawn at random and without replacement from a pack of 52 playing cards. Find the probability that both the cards are black.
\\
\solution
		%\begin{enumerate}[label=\thesection.\arabic*,ref=\thesection.\theenumi]
	\item One card is drawn from a well-shuffled deck of 52 cards. Find the probability of getting
\begin{enumerate}
\item A king of red colour 
\item A face card 
\item A red face card
\item The jack of hearts
\item A spade
\item The queen of diamonds

\end{enumerate}
\solution
		%\input{ncert/10/15/1/14/main.tex}
	\item Five cards—the ten, jack, queen, king and ace of diamonds, are well-shuffled with their face downwards. One card is then picked up at random.
\begin{enumerate}
\item
What is the probability that the card is the queen? 
\item
If the queen is drawn and put aside, what is the probability that the second card picked up is (a) an ace? (b) a queen?\\
\end{enumerate}
\solution
		%\input{ncert/10/15/1/15/defs.tex}
	\item A bag contains $5$ red balls and some blue balls. If the probability of drawing a blue ball is double that if a red ball, determine the number of blue balls in the bag. 
		\\
\solution
		%\input{ncert/10/15/2/3/defs.tex}
	\item A card is selected from a pack of 52 cards.
 \begin{enumerate}[label=(\alph*)] 
                 \item How many points are there in the sample space?
                 \item Calculate the probability that the card is an ace of spades.
                 \item Calculate the probability that the card is (i) an ace and (ii) black card.
 \end{enumerate}
\solution
		%\input{ncert/11/16/3/4/main.tex}
\item Four cards are drawn from a well-shuffled deck of 52 cards. What is the probability of obtaining 3 diamonds and one spade.
\\
\solution
		%\input{ncert/11/16/4/2/defs.tex}
\item In a certain lottery 10,000 tickets are sold and ten equal prizes are awarded. What is the probability of not getting a prize if you buy (a) one ticket (b) two tickets (c) 10 tickets ?	
\\
\solution
		%\input{ncert/11/16/4/4/defs.tex}
		%
\item 
Out of 100 students, two sections of 40 and 60 are formed. If you and your friend are among the 100 students, what is the probability that
\begin{enumerate}
\item you both enter the same section?
\item you both enter the different sections?
\end{enumerate}
\solution
		%\input{ncert/11/16/4/5/defs.tex}
	\item 
The number lock of a suitcase has 4 wheels each labelled with ten digits i.e. from 0 to 9.The lock opens with a sequence of four digits with no repeats.What is the probability of a person getting the right sequence to open the suitcase.
\\
\solution
		%\input{ncert/11/16/4/10/defs.tex}
		%
\item 
Two cards are drawn at random and without replacement from a pack of 52 playing cards. Find the probability that both the cards are black.
\\
\solution
		%\input{ncert/12/13/2/2/defs.tex}
		\item A box of oranges is inspected by examining three randomly selected oranges drawn without replacement. If all the three oranges are good, the box is approved for sale, otherwise, it is rejected. Find the probability that a box containing 15 oranges out of which 12 are good and 3 are bad ones will be approved for sale.
		\label{ncert/12/13/2/3/defs.tex}
		\item Two balls are drawn at random with replacement from a box containing 10 black and 8 red balls. Find the probability that
		\label{ncert/12/13/2/12}
\begin{enumerate}
\item both balls are red.
\item first ball is black and second is red.
\item one of them is black and other is red.
\end{enumerate}

\item In a hostel, 60\% of the students read Hindi newspaper, 40\% read English newspaper and 20\% read both Hindi and English newspapers. A student is selected at random.
		\label{ncert/12/13/2/15}
\begin{enumerate}
\item Find the probability that she reads neither Hindi nor English newspapers.
\item If she reads Hindi newspaper, find the probability that she reads English newspaper.
\item If she reads English newspaper, find the probability that she reads Hindi newspaper.\\
\end{enumerate}
\item The probability of obtaining an even prime number on each die, when a pair of dice is rolled is 
\begin{enumerate}
    \item $0$ 
    
    \item $\frac{1}{3}$ 
    
    \item $\frac{1}{12}$ 
    
    \item $\frac{1}{36}$ 
\end{enumerate}
\solution
		%\input{ncert/12/13/2/17/defs.tex}
	\item A bag contains 4 red and 4 black balls, another bag contains 2 red and 6 black balls. One of the two bags is selected at random and a ball is drawn from the bag which is found to be red. Find the probability that the ball is drawn from the first bag.
\\
\solution
		%\input{ncert/12/13/3/2/main.tex}
  \item
  Cards with numbers 2 to 101 are placed in a box. A card is selected at random.Find the probability that the card has
\begin{enumerate}[label=(\roman*)]
	\item an even number 
	\item a square number
\end{enumerate}
\solution
%\input{exemplar/10/13/3/32/main.tex}
\item
The king, queen and jack of clubs are removed from a deck of 52 playing cards and then well shuffled. Now one card is drawn at random from the remaining cards.  Determine the probability that the card is
\begin{enumerate}[label=(\roman*)]
\item a club
\item 10 of hearts
\end{enumerate}
\solution
%\input{exemplar/10/13/3/29/main.tex}
\item A team of medical students doing their internship have to assist during surgeries
at a city hospital. The probabilities of surgeries rated as very complex, complex,
routine, simple or very simple are respectively, 0.15, 0.20, 0.31, 0.26, .08. Find
the probabilities that a particular surgery will be rated
\begin{enumerate}
	\item complex or very complex;
	\item neither very complex nor very simple;
	\item routine or complex
	\item routine or simple
\end{enumerate}
\solution
%\input{exemplar/11/16/3/8(1)/main.tex}
\item A card is selected from a pack of 52 cards.
\begin{enumerate}[label=(\alph*)]
    \item How many points are there in the sample space?
    \item Calculate the probability that the card is an ace of spades.
    \item Calculate the probability that the card is (i) an ace and (ii) black card.
\end{enumerate}
\solution
%\input{exemplar/11/16/3/4/main2.tex}
\item The probability that a non leap year selected at random will contain 53 sundays.
\\
\solution
%\input{exemplar/10/13/1/19/main.tex}
\item One of the four persons John, Rita, Aslam or Gurpreet will be promoted next
month. Consequently the sample space consists of four elementary outcomes
S = {John promoted, Rita promoted, Aslam promoted, Gurpreet promoted}
You are told that the chances of John’s promotion is same as that of Gurpreet,
Rita’s chances of promotion are twice as likely as Johns. Aslam’s chances are
four times that of John.
\begin{enumerate}
	\item Determine
	\begin{enumerate}
		\item P (John promoted)
		\item P (Rita promoted)
		\item P (Aslam promoted)
		\item P (Gurpreet promoted)
	\end{enumerate}
	\item If A = {John promoted or Gurpreet promoted}, find P (A).
\end{enumerate}
\solution
%\input{exemplar/11/16/3/10/main.tex}
\item A card is drawn from a deck of 52 cards. Find the probability of getting a king or a heart or a red card.\\
\solution
%\input{exemplar/11/16/3/15/main.tex}
\item The probability that a student will pass his examination is 0.73, the probability of
the student getting a compartment is 0.13, and the probability that the student will
either pass or get compartment is 0.96. State True or False.\\
\solution
%\input{exemplar/11/16/3/31/main.tex}
\item A card is selected from a pack of 52 cards\\
\begin{enumerate}[label=(\alph*)]
\item How many points are there in the sample space?
\item Calculate the probability that the cards is an ace of spades.
\item Calculate the probability that the card is (i) an ace (ii)black card.\\
\end{enumerate}
%\input{ncert/11/16/3/4_1/Prob_4.tex}
\item In a non-leap year, the probability of having 53 tuesdays or 53 wednesdays is\\
\solution
%\input{exemplar/11/16/3/18/main.tex}
\item There are 1000 sealed envelopes in a box, 10 of them contain a cash prize of
Rs 100 each, 100 of them contain a cash prize of Rs 50 each and 200 of them
contain a cash prize of Rs 10 each and rest do not contain any cash prize. If they
are well shuffled and an envelope is picked up out, what is the probability that it
contains no cash prize?\\
\solution
%\input{exemplar/10/13/3/34/main.tex}
\item 
A die is thrown and a card is selected at random from a deck of 52 playing cards. The probability of getting an even number on the die and a spade card.\\
\solution
%\input{exemplar/12/13/3/78/main.tex}
\item
If 4-digit numbers greater than 5,000 are randomly formed from the digits 0, 1, 3, 5, and 7, what is the probability of forming a number divisible by 5 when:
\begin{enumerate}
    \item The digits are repeated?
    \item The repetition of digits is not allowed?
\end{enumerate}
\solution
%\input{ncert/11/16/4/9/main.tex}
\item Consider the probability space $\brak{\Omega, \mathcal{G}, P}$ where $\Omega = [0,2]$ and $\mathcal{G} = \cbrak{\phi, \Omega, [0,1], (1,2]}$. Let $X$ and $Y$ be two functions on $\Omega$ defined as
\begin{align*}
    X(\omega) = 
    \begin{cases}
        1 & \text{if }\omega \in [0, 1]\\
        2 & \text{if }\omega \in (1, 2]
    \end{cases}
\end{align*}
and
\begin{align*}
    Y(\omega) = 
    \begin{cases}
        2 & \text{if }\omega \in [0, 1.5]\\
        3 & \text{if }\omega \in (1.5, 2].
    \end{cases}
\end{align*}
Then which one of the following statements is true?
\begin{enumerate}
    \item [(A)] $X$ is a random variable with respect to $\mathcal{G}$, but $Y$ is not a random variable with respect to $\mathcal{G}$.
    \item [(B)] $Y$ is a random variable with respect to $\mathcal{G}$, but $X$ is not a random variable with respect to $\mathcal{G}$.
    \item [(C)] Neither $X$ nor $Y$ is a random variable with respect to $\mathcal{G}$.
    \item [(D)] Both $X$ and $Y$ are random variables with respect to $\mathcal{G}$.
\end{enumerate} \hfill (GATE ST 2023)\\
\solution
%\input{gate/ST/2023/14/main.tex}
	\item  A die is loaded in such a way that each odd number is twice as likely to occur as
each even number. Find $P(G)$, where $G$ is the event that a number greater than
3 occurs on a single roll of the die.
\\
\solution
		%\input{exemplar/11/16/3/5/main.tex}
	\item All the jacks, queens and kings are removed from a deck of 52 playing cards. The remaining cards are well shuffled and then one card is drawn at random. Giving ace a value 1 similar value for other cards, find the probability that the card has a value 
		\begin{enumerate}
			\item 7
			\item greater than 7
			\item less than 7
		\end{enumerate}
		%\input{exemplar/10/13/3/30/main.tex}
  \item A Lot consists of 48 mobile phones of which 42 are good, 3 have only minor defects and 3 have major defects.Varnika will buy a phone if it is good but the trader will only buy a mobile if it has no major defects. One phone is selected at random from the lot. What is the probability that it is
\begin{enumerate}
	\item acceptable to Varnika?
            \item acceptable to the trader?
\end{enumerate}
\solution
	%\input{exemplar/10/13/3/40/main.tex}
 \item A student says that if you throw a die, it will show up 1 or not 1. Therefore, the probability of getting 1 and the probability of getting 'not 1' each is equal to $\frac{1}{2}$. Is this correct? Give reasons.\\
 \solution
        %\input{exemplar/10/13/2/9/main.tex}
   \item Four candidates A, B, C, D have ap-
plied for the assignment to coach a school cricket
team. If A is twice as likely to be selected as B, and
B and C are given about the same chance of being
selected, while C is twice as likely to be selected
as D, what are the probabilities that
\begin{enumerate}
\item C will be selected?
\item A will not be selected?
\end{enumerate}
	%\input{exemplar/11/16/3/9/main.tex}
 \item A bag contain 24 balls of which $x$ balls are red, $2x$ are white and $3x$ are blue. A ball is selected at random, What is the probability that it is
\begin{enumerate}[label=\alph*)]
\item not red ?
\item white ?
\end{enumerate}
%\input{exemplar/10/13/3/41/main.tex}
If the letters of the word ASSASSINATION are arranged at random. Find the Probability that
\begin{enumerate}[label=(\alph*)]
\item Four $S's$ come consecutively in the word
\item Two  $I's$ and two $N's$ come together
\item All $A's$ are not coming together
\item No two $A's$ are coming together
\end{enumerate}
%\input{exemplar/11/16/3/14/main.tex}
	\item One urn contains two black balls (labelled B1 and B2) and one white ball. A
	second urn contains one black ball and two white balls (labelled W1 and W2).
	Suppose the following experiment is performed. One of the two urns is chosen
	at random. Next a ball is randomly chosen from the urn. Then a second ball is
	chosen at random from the same urn without replacing the first ball.
	
	\begin{enumerate}
	\item What is the probability that two black balls are chosen?
	
	\item What is the probability that two balls of opposite colour are chosen?
	\end{enumerate}
	\solution
	%\input{exemplar/11/16/3/12/main1.tex}
\end{enumerate}

		\item A box of oranges is inspected by examining three randomly selected oranges drawn without replacement. If all the three oranges are good, the box is approved for sale, otherwise, it is rejected. Find the probability that a box containing 15 oranges out of which 12 are good and 3 are bad ones will be approved for sale.
		\label{ncert/12/13/2/3/defs.tex}
		\item Two balls are drawn at random with replacement from a box containing 10 black and 8 red balls. Find the probability that
		\label{ncert/12/13/2/12}
\begin{enumerate}
\item both balls are red.
\item first ball is black and second is red.
\item one of them is black and other is red.
\end{enumerate}

\item In a hostel, 60\% of the students read Hindi newspaper, 40\% read English newspaper and 20\% read both Hindi and English newspapers. A student is selected at random.
		\label{ncert/12/13/2/15}
\begin{enumerate}
\item Find the probability that she reads neither Hindi nor English newspapers.
\item If she reads Hindi newspaper, find the probability that she reads English newspaper.
\item If she reads English newspaper, find the probability that she reads Hindi newspaper.\\
\end{enumerate}
\item The probability of obtaining an even prime number on each die, when a pair of dice is rolled is 
\begin{enumerate}
    \item $0$ 
    
    \item $\frac{1}{3}$ 
    
    \item $\frac{1}{12}$ 
    
    \item $\frac{1}{36}$ 
\end{enumerate}
\solution
		%\begin{enumerate}[label=\thesection.\arabic*,ref=\thesection.\theenumi]
	\item One card is drawn from a well-shuffled deck of 52 cards. Find the probability of getting
\begin{enumerate}
\item A king of red colour 
\item A face card 
\item A red face card
\item The jack of hearts
\item A spade
\item The queen of diamonds

\end{enumerate}
\solution
		%\input{ncert/10/15/1/14/main.tex}
	\item Five cards—the ten, jack, queen, king and ace of diamonds, are well-shuffled with their face downwards. One card is then picked up at random.
\begin{enumerate}
\item
What is the probability that the card is the queen? 
\item
If the queen is drawn and put aside, what is the probability that the second card picked up is (a) an ace? (b) a queen?\\
\end{enumerate}
\solution
		%\input{ncert/10/15/1/15/defs.tex}
	\item A bag contains $5$ red balls and some blue balls. If the probability of drawing a blue ball is double that if a red ball, determine the number of blue balls in the bag. 
		\\
\solution
		%\input{ncert/10/15/2/3/defs.tex}
	\item A card is selected from a pack of 52 cards.
 \begin{enumerate}[label=(\alph*)] 
                 \item How many points are there in the sample space?
                 \item Calculate the probability that the card is an ace of spades.
                 \item Calculate the probability that the card is (i) an ace and (ii) black card.
 \end{enumerate}
\solution
		%\input{ncert/11/16/3/4/main.tex}
\item Four cards are drawn from a well-shuffled deck of 52 cards. What is the probability of obtaining 3 diamonds and one spade.
\\
\solution
		%\input{ncert/11/16/4/2/defs.tex}
\item In a certain lottery 10,000 tickets are sold and ten equal prizes are awarded. What is the probability of not getting a prize if you buy (a) one ticket (b) two tickets (c) 10 tickets ?	
\\
\solution
		%\input{ncert/11/16/4/4/defs.tex}
		%
\item 
Out of 100 students, two sections of 40 and 60 are formed. If you and your friend are among the 100 students, what is the probability that
\begin{enumerate}
\item you both enter the same section?
\item you both enter the different sections?
\end{enumerate}
\solution
		%\input{ncert/11/16/4/5/defs.tex}
	\item 
The number lock of a suitcase has 4 wheels each labelled with ten digits i.e. from 0 to 9.The lock opens with a sequence of four digits with no repeats.What is the probability of a person getting the right sequence to open the suitcase.
\\
\solution
		%\input{ncert/11/16/4/10/defs.tex}
		%
\item 
Two cards are drawn at random and without replacement from a pack of 52 playing cards. Find the probability that both the cards are black.
\\
\solution
		%\input{ncert/12/13/2/2/defs.tex}
		\item A box of oranges is inspected by examining three randomly selected oranges drawn without replacement. If all the three oranges are good, the box is approved for sale, otherwise, it is rejected. Find the probability that a box containing 15 oranges out of which 12 are good and 3 are bad ones will be approved for sale.
		\label{ncert/12/13/2/3/defs.tex}
		\item Two balls are drawn at random with replacement from a box containing 10 black and 8 red balls. Find the probability that
		\label{ncert/12/13/2/12}
\begin{enumerate}
\item both balls are red.
\item first ball is black and second is red.
\item one of them is black and other is red.
\end{enumerate}

\item In a hostel, 60\% of the students read Hindi newspaper, 40\% read English newspaper and 20\% read both Hindi and English newspapers. A student is selected at random.
		\label{ncert/12/13/2/15}
\begin{enumerate}
\item Find the probability that she reads neither Hindi nor English newspapers.
\item If she reads Hindi newspaper, find the probability that she reads English newspaper.
\item If she reads English newspaper, find the probability that she reads Hindi newspaper.\\
\end{enumerate}
\item The probability of obtaining an even prime number on each die, when a pair of dice is rolled is 
\begin{enumerate}
    \item $0$ 
    
    \item $\frac{1}{3}$ 
    
    \item $\frac{1}{12}$ 
    
    \item $\frac{1}{36}$ 
\end{enumerate}
\solution
		%\input{ncert/12/13/2/17/defs.tex}
	\item A bag contains 4 red and 4 black balls, another bag contains 2 red and 6 black balls. One of the two bags is selected at random and a ball is drawn from the bag which is found to be red. Find the probability that the ball is drawn from the first bag.
\\
\solution
		%\input{ncert/12/13/3/2/main.tex}
  \item
  Cards with numbers 2 to 101 are placed in a box. A card is selected at random.Find the probability that the card has
\begin{enumerate}[label=(\roman*)]
	\item an even number 
	\item a square number
\end{enumerate}
\solution
%\input{exemplar/10/13/3/32/main.tex}
\item
The king, queen and jack of clubs are removed from a deck of 52 playing cards and then well shuffled. Now one card is drawn at random from the remaining cards.  Determine the probability that the card is
\begin{enumerate}[label=(\roman*)]
\item a club
\item 10 of hearts
\end{enumerate}
\solution
%\input{exemplar/10/13/3/29/main.tex}
\item A team of medical students doing their internship have to assist during surgeries
at a city hospital. The probabilities of surgeries rated as very complex, complex,
routine, simple or very simple are respectively, 0.15, 0.20, 0.31, 0.26, .08. Find
the probabilities that a particular surgery will be rated
\begin{enumerate}
	\item complex or very complex;
	\item neither very complex nor very simple;
	\item routine or complex
	\item routine or simple
\end{enumerate}
\solution
%\input{exemplar/11/16/3/8(1)/main.tex}
\item A card is selected from a pack of 52 cards.
\begin{enumerate}[label=(\alph*)]
    \item How many points are there in the sample space?
    \item Calculate the probability that the card is an ace of spades.
    \item Calculate the probability that the card is (i) an ace and (ii) black card.
\end{enumerate}
\solution
%\input{exemplar/11/16/3/4/main2.tex}
\item The probability that a non leap year selected at random will contain 53 sundays.
\\
\solution
%\input{exemplar/10/13/1/19/main.tex}
\item One of the four persons John, Rita, Aslam or Gurpreet will be promoted next
month. Consequently the sample space consists of four elementary outcomes
S = {John promoted, Rita promoted, Aslam promoted, Gurpreet promoted}
You are told that the chances of John’s promotion is same as that of Gurpreet,
Rita’s chances of promotion are twice as likely as Johns. Aslam’s chances are
four times that of John.
\begin{enumerate}
	\item Determine
	\begin{enumerate}
		\item P (John promoted)
		\item P (Rita promoted)
		\item P (Aslam promoted)
		\item P (Gurpreet promoted)
	\end{enumerate}
	\item If A = {John promoted or Gurpreet promoted}, find P (A).
\end{enumerate}
\solution
%\input{exemplar/11/16/3/10/main.tex}
\item A card is drawn from a deck of 52 cards. Find the probability of getting a king or a heart or a red card.\\
\solution
%\input{exemplar/11/16/3/15/main.tex}
\item The probability that a student will pass his examination is 0.73, the probability of
the student getting a compartment is 0.13, and the probability that the student will
either pass or get compartment is 0.96. State True or False.\\
\solution
%\input{exemplar/11/16/3/31/main.tex}
\item A card is selected from a pack of 52 cards\\
\begin{enumerate}[label=(\alph*)]
\item How many points are there in the sample space?
\item Calculate the probability that the cards is an ace of spades.
\item Calculate the probability that the card is (i) an ace (ii)black card.\\
\end{enumerate}
%\input{ncert/11/16/3/4_1/Prob_4.tex}
\item In a non-leap year, the probability of having 53 tuesdays or 53 wednesdays is\\
\solution
%\input{exemplar/11/16/3/18/main.tex}
\item There are 1000 sealed envelopes in a box, 10 of them contain a cash prize of
Rs 100 each, 100 of them contain a cash prize of Rs 50 each and 200 of them
contain a cash prize of Rs 10 each and rest do not contain any cash prize. If they
are well shuffled and an envelope is picked up out, what is the probability that it
contains no cash prize?\\
\solution
%\input{exemplar/10/13/3/34/main.tex}
\item 
A die is thrown and a card is selected at random from a deck of 52 playing cards. The probability of getting an even number on the die and a spade card.\\
\solution
%\input{exemplar/12/13/3/78/main.tex}
\item
If 4-digit numbers greater than 5,000 are randomly formed from the digits 0, 1, 3, 5, and 7, what is the probability of forming a number divisible by 5 when:
\begin{enumerate}
    \item The digits are repeated?
    \item The repetition of digits is not allowed?
\end{enumerate}
\solution
%\input{ncert/11/16/4/9/main.tex}
\item Consider the probability space $\brak{\Omega, \mathcal{G}, P}$ where $\Omega = [0,2]$ and $\mathcal{G} = \cbrak{\phi, \Omega, [0,1], (1,2]}$. Let $X$ and $Y$ be two functions on $\Omega$ defined as
\begin{align*}
    X(\omega) = 
    \begin{cases}
        1 & \text{if }\omega \in [0, 1]\\
        2 & \text{if }\omega \in (1, 2]
    \end{cases}
\end{align*}
and
\begin{align*}
    Y(\omega) = 
    \begin{cases}
        2 & \text{if }\omega \in [0, 1.5]\\
        3 & \text{if }\omega \in (1.5, 2].
    \end{cases}
\end{align*}
Then which one of the following statements is true?
\begin{enumerate}
    \item [(A)] $X$ is a random variable with respect to $\mathcal{G}$, but $Y$ is not a random variable with respect to $\mathcal{G}$.
    \item [(B)] $Y$ is a random variable with respect to $\mathcal{G}$, but $X$ is not a random variable with respect to $\mathcal{G}$.
    \item [(C)] Neither $X$ nor $Y$ is a random variable with respect to $\mathcal{G}$.
    \item [(D)] Both $X$ and $Y$ are random variables with respect to $\mathcal{G}$.
\end{enumerate} \hfill (GATE ST 2023)\\
\solution
%\input{gate/ST/2023/14/main.tex}
	\item  A die is loaded in such a way that each odd number is twice as likely to occur as
each even number. Find $P(G)$, where $G$ is the event that a number greater than
3 occurs on a single roll of the die.
\\
\solution
		%\input{exemplar/11/16/3/5/main.tex}
	\item All the jacks, queens and kings are removed from a deck of 52 playing cards. The remaining cards are well shuffled and then one card is drawn at random. Giving ace a value 1 similar value for other cards, find the probability that the card has a value 
		\begin{enumerate}
			\item 7
			\item greater than 7
			\item less than 7
		\end{enumerate}
		%\input{exemplar/10/13/3/30/main.tex}
  \item A Lot consists of 48 mobile phones of which 42 are good, 3 have only minor defects and 3 have major defects.Varnika will buy a phone if it is good but the trader will only buy a mobile if it has no major defects. One phone is selected at random from the lot. What is the probability that it is
\begin{enumerate}
	\item acceptable to Varnika?
            \item acceptable to the trader?
\end{enumerate}
\solution
	%\input{exemplar/10/13/3/40/main.tex}
 \item A student says that if you throw a die, it will show up 1 or not 1. Therefore, the probability of getting 1 and the probability of getting 'not 1' each is equal to $\frac{1}{2}$. Is this correct? Give reasons.\\
 \solution
        %\input{exemplar/10/13/2/9/main.tex}
   \item Four candidates A, B, C, D have ap-
plied for the assignment to coach a school cricket
team. If A is twice as likely to be selected as B, and
B and C are given about the same chance of being
selected, while C is twice as likely to be selected
as D, what are the probabilities that
\begin{enumerate}
\item C will be selected?
\item A will not be selected?
\end{enumerate}
	%\input{exemplar/11/16/3/9/main.tex}
 \item A bag contain 24 balls of which $x$ balls are red, $2x$ are white and $3x$ are blue. A ball is selected at random, What is the probability that it is
\begin{enumerate}[label=\alph*)]
\item not red ?
\item white ?
\end{enumerate}
%\input{exemplar/10/13/3/41/main.tex}
If the letters of the word ASSASSINATION are arranged at random. Find the Probability that
\begin{enumerate}[label=(\alph*)]
\item Four $S's$ come consecutively in the word
\item Two  $I's$ and two $N's$ come together
\item All $A's$ are not coming together
\item No two $A's$ are coming together
\end{enumerate}
%\input{exemplar/11/16/3/14/main.tex}
	\item One urn contains two black balls (labelled B1 and B2) and one white ball. A
	second urn contains one black ball and two white balls (labelled W1 and W2).
	Suppose the following experiment is performed. One of the two urns is chosen
	at random. Next a ball is randomly chosen from the urn. Then a second ball is
	chosen at random from the same urn without replacing the first ball.
	
	\begin{enumerate}
	\item What is the probability that two black balls are chosen?
	
	\item What is the probability that two balls of opposite colour are chosen?
	\end{enumerate}
	\solution
	%\input{exemplar/11/16/3/12/main1.tex}
\end{enumerate}

	\item A bag contains 4 red and 4 black balls, another bag contains 2 red and 6 black balls. One of the two bags is selected at random and a ball is drawn from the bag which is found to be red. Find the probability that the ball is drawn from the first bag.
\\
\solution
		%\begin{table}[H]
	\centering
\begin{tabular}{|c|c|c|}
\hline
Random variable &Value &Definition\\ \hline
\multirow{3}{*}{X} &0 &Slips of Rs 1\\
&1 &Slips of Rs 5\\
&2 &Slips of Rs 13\\ \hline
\multirow{2}{*}{Y} &0 &Box A\\
&1 &Box B\\\hline
\end{tabular}
\caption{}
\label{tab:Distribution}
\end{table}
See \tabref{tab:Distribution}.
\begin{align}
p_{Y}\brak{k}= \begin{cases} 
      \frac{1}{3} & {k=0} \\
      \frac{2}{3 }& {k=1} 
   \end{cases}
   \\
p_{Y|X}\brak{0|0} = \frac{19}{25}\, 
p_{Y|X}\brak{0|1} = \frac{6}{25}\,
p_{Y|X}\brak{1|0} = \frac{45}{50}\,
p_{Y|X}\brak{1|2} = \frac{5}{50}
\end{align}
The desired probability is the probability that a slip drawn at random is marked other than Rs 1,
\begin{align}
&=1-p_X\brak{0}\\
&= p_X(1) + p_X(2)
\end{align}
Using Bayes theorem,
\begin{align}
&= p_Y\brak{0} \times \pr{Y=0 | X=1} + p_Y\brak{1} \times \pr{Y=1|X=2}\\
&=\frac{1}{3} \times \frac{6}{25} + \frac{2}{3} \times \frac{5}{50}\\
&=\frac{11}{75}
\end{align}

\newpage

%\tableofcontents

\bigskip

\renewcommand{\thefigure}{\theenumi}
\renewcommand{\thetable}{\theenumi}
%\renewcommand{\theequation}{\theenumi}

%\begin{abstract}
%%\boldmath
%In this letter, an algorithm for evaluating the exact analytical bit error rate  (BER)  for the piecewise linear (PL) combiner for  multiple relays is presented. Previous results were available only for upto three relays. The algorithm is unique in the sense that  the actual mathematical expressions, that are prohibitively large, need not be explicitly obtained. The diversity gain due to multiple relays is shown through plots of the analytical BER, well supported by simulations. 
%
%\end{abstract}
% IEEEtran.cls defaults to using nonbold math in the Abstract.
% This preserves the distinction between vectors and scalars. However,
% if the journal you are submitting to favors bold math in the abstract,
% then you can use LaTeX's standard command \boldmath at the very start
% of the abstract to achieve this. Many IEEE journals frown on math
% in the abstract anyway.

% Note that keywords are not normally used for peerreview papers.
%\begin{IEEEkeywords}
%Cooperative diversity, decode and forward, piecewise linear
%\end{IEEEkeywords}



% For peer review papers, you can put extra information on the cover
% page as needed:
% \ifCLASSOPTIONpeerreview
% \begin{center} \bfseries EDICS Category: 3-BBND \end{center}
% \fi
%
% For peerreview papers, this IEEEtran command inserts a page break and
% creates the second title. It will be ignored for other modes.
%\IEEEpeerreviewmaketitle




  \item
  Cards with numbers 2 to 101 are placed in a box. A card is selected at random.Find the probability that the card has
\begin{enumerate}[label=(\roman*)]
	\item an even number 
	\item a square number
\end{enumerate}
\solution
%\begin{table}[H]
	\centering
\begin{tabular}{|c|c|c|}
\hline
Random variable &Value &Definition\\ \hline
\multirow{3}{*}{X} &0 &Slips of Rs 1\\
&1 &Slips of Rs 5\\
&2 &Slips of Rs 13\\ \hline
\multirow{2}{*}{Y} &0 &Box A\\
&1 &Box B\\\hline
\end{tabular}
\caption{}
\label{tab:Distribution}
\end{table}
See \tabref{tab:Distribution}.
\begin{align}
p_{Y}\brak{k}= \begin{cases} 
      \frac{1}{3} & {k=0} \\
      \frac{2}{3 }& {k=1} 
   \end{cases}
   \\
p_{Y|X}\brak{0|0} = \frac{19}{25}\, 
p_{Y|X}\brak{0|1} = \frac{6}{25}\,
p_{Y|X}\brak{1|0} = \frac{45}{50}\,
p_{Y|X}\brak{1|2} = \frac{5}{50}
\end{align}
The desired probability is the probability that a slip drawn at random is marked other than Rs 1,
\begin{align}
&=1-p_X\brak{0}\\
&= p_X(1) + p_X(2)
\end{align}
Using Bayes theorem,
\begin{align}
&= p_Y\brak{0} \times \pr{Y=0 | X=1} + p_Y\brak{1} \times \pr{Y=1|X=2}\\
&=\frac{1}{3} \times \frac{6}{25} + \frac{2}{3} \times \frac{5}{50}\\
&=\frac{11}{75}
\end{align}

\newpage

%\tableofcontents

\bigskip

\renewcommand{\thefigure}{\theenumi}
\renewcommand{\thetable}{\theenumi}
%\renewcommand{\theequation}{\theenumi}

%\begin{abstract}
%%\boldmath
%In this letter, an algorithm for evaluating the exact analytical bit error rate  (BER)  for the piecewise linear (PL) combiner for  multiple relays is presented. Previous results were available only for upto three relays. The algorithm is unique in the sense that  the actual mathematical expressions, that are prohibitively large, need not be explicitly obtained. The diversity gain due to multiple relays is shown through plots of the analytical BER, well supported by simulations. 
%
%\end{abstract}
% IEEEtran.cls defaults to using nonbold math in the Abstract.
% This preserves the distinction between vectors and scalars. However,
% if the journal you are submitting to favors bold math in the abstract,
% then you can use LaTeX's standard command \boldmath at the very start
% of the abstract to achieve this. Many IEEE journals frown on math
% in the abstract anyway.

% Note that keywords are not normally used for peerreview papers.
%\begin{IEEEkeywords}
%Cooperative diversity, decode and forward, piecewise linear
%\end{IEEEkeywords}



% For peer review papers, you can put extra information on the cover
% page as needed:
% \ifCLASSOPTIONpeerreview
% \begin{center} \bfseries EDICS Category: 3-BBND \end{center}
% \fi
%
% For peerreview papers, this IEEEtran command inserts a page break and
% creates the second title. It will be ignored for other modes.
%\IEEEpeerreviewmaketitle




\item
The king, queen and jack of clubs are removed from a deck of 52 playing cards and then well shuffled. Now one card is drawn at random from the remaining cards.  Determine the probability that the card is
\begin{enumerate}[label=(\roman*)]
\item a club
\item 10 of hearts
\end{enumerate}
\solution
%\begin{table}[H]
	\centering
\begin{tabular}{|c|c|c|}
\hline
Random variable &Value &Definition\\ \hline
\multirow{3}{*}{X} &0 &Slips of Rs 1\\
&1 &Slips of Rs 5\\
&2 &Slips of Rs 13\\ \hline
\multirow{2}{*}{Y} &0 &Box A\\
&1 &Box B\\\hline
\end{tabular}
\caption{}
\label{tab:Distribution}
\end{table}
See \tabref{tab:Distribution}.
\begin{align}
p_{Y}\brak{k}= \begin{cases} 
      \frac{1}{3} & {k=0} \\
      \frac{2}{3 }& {k=1} 
   \end{cases}
   \\
p_{Y|X}\brak{0|0} = \frac{19}{25}\, 
p_{Y|X}\brak{0|1} = \frac{6}{25}\,
p_{Y|X}\brak{1|0} = \frac{45}{50}\,
p_{Y|X}\brak{1|2} = \frac{5}{50}
\end{align}
The desired probability is the probability that a slip drawn at random is marked other than Rs 1,
\begin{align}
&=1-p_X\brak{0}\\
&= p_X(1) + p_X(2)
\end{align}
Using Bayes theorem,
\begin{align}
&= p_Y\brak{0} \times \pr{Y=0 | X=1} + p_Y\brak{1} \times \pr{Y=1|X=2}\\
&=\frac{1}{3} \times \frac{6}{25} + \frac{2}{3} \times \frac{5}{50}\\
&=\frac{11}{75}
\end{align}

\newpage

%\tableofcontents

\bigskip

\renewcommand{\thefigure}{\theenumi}
\renewcommand{\thetable}{\theenumi}
%\renewcommand{\theequation}{\theenumi}

%\begin{abstract}
%%\boldmath
%In this letter, an algorithm for evaluating the exact analytical bit error rate  (BER)  for the piecewise linear (PL) combiner for  multiple relays is presented. Previous results were available only for upto three relays. The algorithm is unique in the sense that  the actual mathematical expressions, that are prohibitively large, need not be explicitly obtained. The diversity gain due to multiple relays is shown through plots of the analytical BER, well supported by simulations. 
%
%\end{abstract}
% IEEEtran.cls defaults to using nonbold math in the Abstract.
% This preserves the distinction between vectors and scalars. However,
% if the journal you are submitting to favors bold math in the abstract,
% then you can use LaTeX's standard command \boldmath at the very start
% of the abstract to achieve this. Many IEEE journals frown on math
% in the abstract anyway.

% Note that keywords are not normally used for peerreview papers.
%\begin{IEEEkeywords}
%Cooperative diversity, decode and forward, piecewise linear
%\end{IEEEkeywords}



% For peer review papers, you can put extra information on the cover
% page as needed:
% \ifCLASSOPTIONpeerreview
% \begin{center} \bfseries EDICS Category: 3-BBND \end{center}
% \fi
%
% For peerreview papers, this IEEEtran command inserts a page break and
% creates the second title. It will be ignored for other modes.
%\IEEEpeerreviewmaketitle




\item A team of medical students doing their internship have to assist during surgeries
at a city hospital. The probabilities of surgeries rated as very complex, complex,
routine, simple or very simple are respectively, 0.15, 0.20, 0.31, 0.26, .08. Find
the probabilities that a particular surgery will be rated
\begin{enumerate}
	\item complex or very complex;
	\item neither very complex nor very simple;
	\item routine or complex
	\item routine or simple
\end{enumerate}
\solution
%\begin{table}[H]
	\centering
\begin{tabular}{|c|c|c|}
\hline
Random variable &Value &Definition\\ \hline
\multirow{3}{*}{X} &0 &Slips of Rs 1\\
&1 &Slips of Rs 5\\
&2 &Slips of Rs 13\\ \hline
\multirow{2}{*}{Y} &0 &Box A\\
&1 &Box B\\\hline
\end{tabular}
\caption{}
\label{tab:Distribution}
\end{table}
See \tabref{tab:Distribution}.
\begin{align}
p_{Y}\brak{k}= \begin{cases} 
      \frac{1}{3} & {k=0} \\
      \frac{2}{3 }& {k=1} 
   \end{cases}
   \\
p_{Y|X}\brak{0|0} = \frac{19}{25}\, 
p_{Y|X}\brak{0|1} = \frac{6}{25}\,
p_{Y|X}\brak{1|0} = \frac{45}{50}\,
p_{Y|X}\brak{1|2} = \frac{5}{50}
\end{align}
The desired probability is the probability that a slip drawn at random is marked other than Rs 1,
\begin{align}
&=1-p_X\brak{0}\\
&= p_X(1) + p_X(2)
\end{align}
Using Bayes theorem,
\begin{align}
&= p_Y\brak{0} \times \pr{Y=0 | X=1} + p_Y\brak{1} \times \pr{Y=1|X=2}\\
&=\frac{1}{3} \times \frac{6}{25} + \frac{2}{3} \times \frac{5}{50}\\
&=\frac{11}{75}
\end{align}

\newpage

%\tableofcontents

\bigskip

\renewcommand{\thefigure}{\theenumi}
\renewcommand{\thetable}{\theenumi}
%\renewcommand{\theequation}{\theenumi}

%\begin{abstract}
%%\boldmath
%In this letter, an algorithm for evaluating the exact analytical bit error rate  (BER)  for the piecewise linear (PL) combiner for  multiple relays is presented. Previous results were available only for upto three relays. The algorithm is unique in the sense that  the actual mathematical expressions, that are prohibitively large, need not be explicitly obtained. The diversity gain due to multiple relays is shown through plots of the analytical BER, well supported by simulations. 
%
%\end{abstract}
% IEEEtran.cls defaults to using nonbold math in the Abstract.
% This preserves the distinction between vectors and scalars. However,
% if the journal you are submitting to favors bold math in the abstract,
% then you can use LaTeX's standard command \boldmath at the very start
% of the abstract to achieve this. Many IEEE journals frown on math
% in the abstract anyway.

% Note that keywords are not normally used for peerreview papers.
%\begin{IEEEkeywords}
%Cooperative diversity, decode and forward, piecewise linear
%\end{IEEEkeywords}



% For peer review papers, you can put extra information on the cover
% page as needed:
% \ifCLASSOPTIONpeerreview
% \begin{center} \bfseries EDICS Category: 3-BBND \end{center}
% \fi
%
% For peerreview papers, this IEEEtran command inserts a page break and
% creates the second title. It will be ignored for other modes.
%\IEEEpeerreviewmaketitle




\item A card is selected from a pack of 52 cards.
\begin{enumerate}[label=(\alph*)]
    \item How many points are there in the sample space?
    \item Calculate the probability that the card is an ace of spades.
    \item Calculate the probability that the card is (i) an ace and (ii) black card.
\end{enumerate}
\solution
%Let $X$ be an bernoulli rv defined as in \tabref{tab:exemplar/11/16/3/26}.  Then, 
\begin{equation}
    p =
        \frac{4}{11} 
\end{equation}
\begin{table}[H]
	\centering
	\input{exemplar/11/16/3/26/tables/Table2.tex}
	\caption{}
        \label{tab:exemplar/11/16/3/26}
\end{table}

\item The probability that a non leap year selected at random will contain 53 sundays.
\\
\solution
%\begin{table}[H]
	\centering
\begin{tabular}{|c|c|c|}
\hline
Random variable &Value &Definition\\ \hline
\multirow{3}{*}{X} &0 &Slips of Rs 1\\
&1 &Slips of Rs 5\\
&2 &Slips of Rs 13\\ \hline
\multirow{2}{*}{Y} &0 &Box A\\
&1 &Box B\\\hline
\end{tabular}
\caption{}
\label{tab:Distribution}
\end{table}
See \tabref{tab:Distribution}.
\begin{align}
p_{Y}\brak{k}= \begin{cases} 
      \frac{1}{3} & {k=0} \\
      \frac{2}{3 }& {k=1} 
   \end{cases}
   \\
p_{Y|X}\brak{0|0} = \frac{19}{25}\, 
p_{Y|X}\brak{0|1} = \frac{6}{25}\,
p_{Y|X}\brak{1|0} = \frac{45}{50}\,
p_{Y|X}\brak{1|2} = \frac{5}{50}
\end{align}
The desired probability is the probability that a slip drawn at random is marked other than Rs 1,
\begin{align}
&=1-p_X\brak{0}\\
&= p_X(1) + p_X(2)
\end{align}
Using Bayes theorem,
\begin{align}
&= p_Y\brak{0} \times \pr{Y=0 | X=1} + p_Y\brak{1} \times \pr{Y=1|X=2}\\
&=\frac{1}{3} \times \frac{6}{25} + \frac{2}{3} \times \frac{5}{50}\\
&=\frac{11}{75}
\end{align}

\newpage

%\tableofcontents

\bigskip

\renewcommand{\thefigure}{\theenumi}
\renewcommand{\thetable}{\theenumi}
%\renewcommand{\theequation}{\theenumi}

%\begin{abstract}
%%\boldmath
%In this letter, an algorithm for evaluating the exact analytical bit error rate  (BER)  for the piecewise linear (PL) combiner for  multiple relays is presented. Previous results were available only for upto three relays. The algorithm is unique in the sense that  the actual mathematical expressions, that are prohibitively large, need not be explicitly obtained. The diversity gain due to multiple relays is shown through plots of the analytical BER, well supported by simulations. 
%
%\end{abstract}
% IEEEtran.cls defaults to using nonbold math in the Abstract.
% This preserves the distinction between vectors and scalars. However,
% if the journal you are submitting to favors bold math in the abstract,
% then you can use LaTeX's standard command \boldmath at the very start
% of the abstract to achieve this. Many IEEE journals frown on math
% in the abstract anyway.

% Note that keywords are not normally used for peerreview papers.
%\begin{IEEEkeywords}
%Cooperative diversity, decode and forward, piecewise linear
%\end{IEEEkeywords}



% For peer review papers, you can put extra information on the cover
% page as needed:
% \ifCLASSOPTIONpeerreview
% \begin{center} \bfseries EDICS Category: 3-BBND \end{center}
% \fi
%
% For peerreview papers, this IEEEtran command inserts a page break and
% creates the second title. It will be ignored for other modes.
%\IEEEpeerreviewmaketitle




\item One of the four persons John, Rita, Aslam or Gurpreet will be promoted next
month. Consequently the sample space consists of four elementary outcomes
S = {John promoted, Rita promoted, Aslam promoted, Gurpreet promoted}
You are told that the chances of John’s promotion is same as that of Gurpreet,
Rita’s chances of promotion are twice as likely as Johns. Aslam’s chances are
four times that of John.
\begin{enumerate}
	\item Determine
	\begin{enumerate}
		\item P (John promoted)
		\item P (Rita promoted)
		\item P (Aslam promoted)
		\item P (Gurpreet promoted)
	\end{enumerate}
	\item If A = {John promoted or Gurpreet promoted}, find P (A).
\end{enumerate}
\solution
%\begin{table}[H]
	\centering
\begin{tabular}{|c|c|c|}
\hline
Random variable &Value &Definition\\ \hline
\multirow{3}{*}{X} &0 &Slips of Rs 1\\
&1 &Slips of Rs 5\\
&2 &Slips of Rs 13\\ \hline
\multirow{2}{*}{Y} &0 &Box A\\
&1 &Box B\\\hline
\end{tabular}
\caption{}
\label{tab:Distribution}
\end{table}
See \tabref{tab:Distribution}.
\begin{align}
p_{Y}\brak{k}= \begin{cases} 
      \frac{1}{3} & {k=0} \\
      \frac{2}{3 }& {k=1} 
   \end{cases}
   \\
p_{Y|X}\brak{0|0} = \frac{19}{25}\, 
p_{Y|X}\brak{0|1} = \frac{6}{25}\,
p_{Y|X}\brak{1|0} = \frac{45}{50}\,
p_{Y|X}\brak{1|2} = \frac{5}{50}
\end{align}
The desired probability is the probability that a slip drawn at random is marked other than Rs 1,
\begin{align}
&=1-p_X\brak{0}\\
&= p_X(1) + p_X(2)
\end{align}
Using Bayes theorem,
\begin{align}
&= p_Y\brak{0} \times \pr{Y=0 | X=1} + p_Y\brak{1} \times \pr{Y=1|X=2}\\
&=\frac{1}{3} \times \frac{6}{25} + \frac{2}{3} \times \frac{5}{50}\\
&=\frac{11}{75}
\end{align}

\newpage

%\tableofcontents

\bigskip

\renewcommand{\thefigure}{\theenumi}
\renewcommand{\thetable}{\theenumi}
%\renewcommand{\theequation}{\theenumi}

%\begin{abstract}
%%\boldmath
%In this letter, an algorithm for evaluating the exact analytical bit error rate  (BER)  for the piecewise linear (PL) combiner for  multiple relays is presented. Previous results were available only for upto three relays. The algorithm is unique in the sense that  the actual mathematical expressions, that are prohibitively large, need not be explicitly obtained. The diversity gain due to multiple relays is shown through plots of the analytical BER, well supported by simulations. 
%
%\end{abstract}
% IEEEtran.cls defaults to using nonbold math in the Abstract.
% This preserves the distinction between vectors and scalars. However,
% if the journal you are submitting to favors bold math in the abstract,
% then you can use LaTeX's standard command \boldmath at the very start
% of the abstract to achieve this. Many IEEE journals frown on math
% in the abstract anyway.

% Note that keywords are not normally used for peerreview papers.
%\begin{IEEEkeywords}
%Cooperative diversity, decode and forward, piecewise linear
%\end{IEEEkeywords}



% For peer review papers, you can put extra information on the cover
% page as needed:
% \ifCLASSOPTIONpeerreview
% \begin{center} \bfseries EDICS Category: 3-BBND \end{center}
% \fi
%
% For peerreview papers, this IEEEtran command inserts a page break and
% creates the second title. It will be ignored for other modes.
%\IEEEpeerreviewmaketitle




\item A card is drawn from a deck of 52 cards. Find the probability of getting a king or a heart or a red card.\\
\solution
%\begin{table}[H]
	\centering
\begin{tabular}{|c|c|c|}
\hline
Random variable &Value &Definition\\ \hline
\multirow{3}{*}{X} &0 &Slips of Rs 1\\
&1 &Slips of Rs 5\\
&2 &Slips of Rs 13\\ \hline
\multirow{2}{*}{Y} &0 &Box A\\
&1 &Box B\\\hline
\end{tabular}
\caption{}
\label{tab:Distribution}
\end{table}
See \tabref{tab:Distribution}.
\begin{align}
p_{Y}\brak{k}= \begin{cases} 
      \frac{1}{3} & {k=0} \\
      \frac{2}{3 }& {k=1} 
   \end{cases}
   \\
p_{Y|X}\brak{0|0} = \frac{19}{25}\, 
p_{Y|X}\brak{0|1} = \frac{6}{25}\,
p_{Y|X}\brak{1|0} = \frac{45}{50}\,
p_{Y|X}\brak{1|2} = \frac{5}{50}
\end{align}
The desired probability is the probability that a slip drawn at random is marked other than Rs 1,
\begin{align}
&=1-p_X\brak{0}\\
&= p_X(1) + p_X(2)
\end{align}
Using Bayes theorem,
\begin{align}
&= p_Y\brak{0} \times \pr{Y=0 | X=1} + p_Y\brak{1} \times \pr{Y=1|X=2}\\
&=\frac{1}{3} \times \frac{6}{25} + \frac{2}{3} \times \frac{5}{50}\\
&=\frac{11}{75}
\end{align}

\newpage

%\tableofcontents

\bigskip

\renewcommand{\thefigure}{\theenumi}
\renewcommand{\thetable}{\theenumi}
%\renewcommand{\theequation}{\theenumi}

%\begin{abstract}
%%\boldmath
%In this letter, an algorithm for evaluating the exact analytical bit error rate  (BER)  for the piecewise linear (PL) combiner for  multiple relays is presented. Previous results were available only for upto three relays. The algorithm is unique in the sense that  the actual mathematical expressions, that are prohibitively large, need not be explicitly obtained. The diversity gain due to multiple relays is shown through plots of the analytical BER, well supported by simulations. 
%
%\end{abstract}
% IEEEtran.cls defaults to using nonbold math in the Abstract.
% This preserves the distinction between vectors and scalars. However,
% if the journal you are submitting to favors bold math in the abstract,
% then you can use LaTeX's standard command \boldmath at the very start
% of the abstract to achieve this. Many IEEE journals frown on math
% in the abstract anyway.

% Note that keywords are not normally used for peerreview papers.
%\begin{IEEEkeywords}
%Cooperative diversity, decode and forward, piecewise linear
%\end{IEEEkeywords}



% For peer review papers, you can put extra information on the cover
% page as needed:
% \ifCLASSOPTIONpeerreview
% \begin{center} \bfseries EDICS Category: 3-BBND \end{center}
% \fi
%
% For peerreview papers, this IEEEtran command inserts a page break and
% creates the second title. It will be ignored for other modes.
%\IEEEpeerreviewmaketitle




\item The probability that a student will pass his examination is 0.73, the probability of
the student getting a compartment is 0.13, and the probability that the student will
either pass or get compartment is 0.96. State True or False.\\
\solution
%\begin{table}[H]
	\centering
\begin{tabular}{|c|c|c|}
\hline
Random variable &Value &Definition\\ \hline
\multirow{3}{*}{X} &0 &Slips of Rs 1\\
&1 &Slips of Rs 5\\
&2 &Slips of Rs 13\\ \hline
\multirow{2}{*}{Y} &0 &Box A\\
&1 &Box B\\\hline
\end{tabular}
\caption{}
\label{tab:Distribution}
\end{table}
See \tabref{tab:Distribution}.
\begin{align}
p_{Y}\brak{k}= \begin{cases} 
      \frac{1}{3} & {k=0} \\
      \frac{2}{3 }& {k=1} 
   \end{cases}
   \\
p_{Y|X}\brak{0|0} = \frac{19}{25}\, 
p_{Y|X}\brak{0|1} = \frac{6}{25}\,
p_{Y|X}\brak{1|0} = \frac{45}{50}\,
p_{Y|X}\brak{1|2} = \frac{5}{50}
\end{align}
The desired probability is the probability that a slip drawn at random is marked other than Rs 1,
\begin{align}
&=1-p_X\brak{0}\\
&= p_X(1) + p_X(2)
\end{align}
Using Bayes theorem,
\begin{align}
&= p_Y\brak{0} \times \pr{Y=0 | X=1} + p_Y\brak{1} \times \pr{Y=1|X=2}\\
&=\frac{1}{3} \times \frac{6}{25} + \frac{2}{3} \times \frac{5}{50}\\
&=\frac{11}{75}
\end{align}

\newpage

%\tableofcontents

\bigskip

\renewcommand{\thefigure}{\theenumi}
\renewcommand{\thetable}{\theenumi}
%\renewcommand{\theequation}{\theenumi}

%\begin{abstract}
%%\boldmath
%In this letter, an algorithm for evaluating the exact analytical bit error rate  (BER)  for the piecewise linear (PL) combiner for  multiple relays is presented. Previous results were available only for upto three relays. The algorithm is unique in the sense that  the actual mathematical expressions, that are prohibitively large, need not be explicitly obtained. The diversity gain due to multiple relays is shown through plots of the analytical BER, well supported by simulations. 
%
%\end{abstract}
% IEEEtran.cls defaults to using nonbold math in the Abstract.
% This preserves the distinction between vectors and scalars. However,
% if the journal you are submitting to favors bold math in the abstract,
% then you can use LaTeX's standard command \boldmath at the very start
% of the abstract to achieve this. Many IEEE journals frown on math
% in the abstract anyway.

% Note that keywords are not normally used for peerreview papers.
%\begin{IEEEkeywords}
%Cooperative diversity, decode and forward, piecewise linear
%\end{IEEEkeywords}



% For peer review papers, you can put extra information on the cover
% page as needed:
% \ifCLASSOPTIONpeerreview
% \begin{center} \bfseries EDICS Category: 3-BBND \end{center}
% \fi
%
% For peerreview papers, this IEEEtran command inserts a page break and
% creates the second title. It will be ignored for other modes.
%\IEEEpeerreviewmaketitle




\item A card is selected from a pack of 52 cards\\
\begin{enumerate}[label=(\alph*)]
\item How many points are there in the sample space?
\item Calculate the probability that the cards is an ace of spades.
\item Calculate the probability that the card is (i) an ace (ii)black card.\\
\end{enumerate}
%\input{ncert/11/16/3/4_1/Prob_4.tex}
\item In a non-leap year, the probability of having 53 tuesdays or 53 wednesdays is\\
\solution
%A non-leap year has a total of 365 days, and a week has 7 days.\\
So it can be expressed as 
\begin{align}
365\text{days} &=52\times 7+1 \text{day}
\end{align}
$\implies$ 52 tuesdays or wednesdays\\
Random variable X denotes the days of a week
\begin{align}
p_X\brak{k}&=\frac{1}{7}; \quad \brak{1<k<7}
\end{align}
So the probability of extra day being tuesday or wednesday is
\begin{align}
p_X\brak{3}+p_X\brak{4}&=\frac{1}{7}+\frac{1}{7}=\frac{2}{7}
\end{align}



\item There are 1000 sealed envelopes in a box, 10 of them contain a cash prize of
Rs 100 each, 100 of them contain a cash prize of Rs 50 each and 200 of them
contain a cash prize of Rs 10 each and rest do not contain any cash prize. If they
are well shuffled and an envelope is picked up out, what is the probability that it
contains no cash prize?\\
\solution
%\begin{table}[H]
	\centering
\begin{tabular}{|c|c|c|}
\hline
Random variable &Value &Definition\\ \hline
\multirow{3}{*}{X} &0 &Slips of Rs 1\\
&1 &Slips of Rs 5\\
&2 &Slips of Rs 13\\ \hline
\multirow{2}{*}{Y} &0 &Box A\\
&1 &Box B\\\hline
\end{tabular}
\caption{}
\label{tab:Distribution}
\end{table}
See \tabref{tab:Distribution}.
\begin{align}
p_{Y}\brak{k}= \begin{cases} 
      \frac{1}{3} & {k=0} \\
      \frac{2}{3 }& {k=1} 
   \end{cases}
   \\
p_{Y|X}\brak{0|0} = \frac{19}{25}\, 
p_{Y|X}\brak{0|1} = \frac{6}{25}\,
p_{Y|X}\brak{1|0} = \frac{45}{50}\,
p_{Y|X}\brak{1|2} = \frac{5}{50}
\end{align}
The desired probability is the probability that a slip drawn at random is marked other than Rs 1,
\begin{align}
&=1-p_X\brak{0}\\
&= p_X(1) + p_X(2)
\end{align}
Using Bayes theorem,
\begin{align}
&= p_Y\brak{0} \times \pr{Y=0 | X=1} + p_Y\brak{1} \times \pr{Y=1|X=2}\\
&=\frac{1}{3} \times \frac{6}{25} + \frac{2}{3} \times \frac{5}{50}\\
&=\frac{11}{75}
\end{align}

\newpage

%\tableofcontents

\bigskip

\renewcommand{\thefigure}{\theenumi}
\renewcommand{\thetable}{\theenumi}
%\renewcommand{\theequation}{\theenumi}

%\begin{abstract}
%%\boldmath
%In this letter, an algorithm for evaluating the exact analytical bit error rate  (BER)  for the piecewise linear (PL) combiner for  multiple relays is presented. Previous results were available only for upto three relays. The algorithm is unique in the sense that  the actual mathematical expressions, that are prohibitively large, need not be explicitly obtained. The diversity gain due to multiple relays is shown through plots of the analytical BER, well supported by simulations. 
%
%\end{abstract}
% IEEEtran.cls defaults to using nonbold math in the Abstract.
% This preserves the distinction between vectors and scalars. However,
% if the journal you are submitting to favors bold math in the abstract,
% then you can use LaTeX's standard command \boldmath at the very start
% of the abstract to achieve this. Many IEEE journals frown on math
% in the abstract anyway.

% Note that keywords are not normally used for peerreview papers.
%\begin{IEEEkeywords}
%Cooperative diversity, decode and forward, piecewise linear
%\end{IEEEkeywords}



% For peer review papers, you can put extra information on the cover
% page as needed:
% \ifCLASSOPTIONpeerreview
% \begin{center} \bfseries EDICS Category: 3-BBND \end{center}
% \fi
%
% For peerreview papers, this IEEEtran command inserts a page break and
% creates the second title. It will be ignored for other modes.
%\IEEEpeerreviewmaketitle




\item 
A die is thrown and a card is selected at random from a deck of 52 playing cards. The probability of getting an even number on the die and a spade card.\\
\solution
%\begin{table}[H]
	\centering
\begin{tabular}{|c|c|c|}
\hline
Random variable &Value &Definition\\ \hline
\multirow{3}{*}{X} &0 &Slips of Rs 1\\
&1 &Slips of Rs 5\\
&2 &Slips of Rs 13\\ \hline
\multirow{2}{*}{Y} &0 &Box A\\
&1 &Box B\\\hline
\end{tabular}
\caption{}
\label{tab:Distribution}
\end{table}
See \tabref{tab:Distribution}.
\begin{align}
p_{Y}\brak{k}= \begin{cases} 
      \frac{1}{3} & {k=0} \\
      \frac{2}{3 }& {k=1} 
   \end{cases}
   \\
p_{Y|X}\brak{0|0} = \frac{19}{25}\, 
p_{Y|X}\brak{0|1} = \frac{6}{25}\,
p_{Y|X}\brak{1|0} = \frac{45}{50}\,
p_{Y|X}\brak{1|2} = \frac{5}{50}
\end{align}
The desired probability is the probability that a slip drawn at random is marked other than Rs 1,
\begin{align}
&=1-p_X\brak{0}\\
&= p_X(1) + p_X(2)
\end{align}
Using Bayes theorem,
\begin{align}
&= p_Y\brak{0} \times \pr{Y=0 | X=1} + p_Y\brak{1} \times \pr{Y=1|X=2}\\
&=\frac{1}{3} \times \frac{6}{25} + \frac{2}{3} \times \frac{5}{50}\\
&=\frac{11}{75}
\end{align}

\newpage

%\tableofcontents

\bigskip

\renewcommand{\thefigure}{\theenumi}
\renewcommand{\thetable}{\theenumi}
%\renewcommand{\theequation}{\theenumi}

%\begin{abstract}
%%\boldmath
%In this letter, an algorithm for evaluating the exact analytical bit error rate  (BER)  for the piecewise linear (PL) combiner for  multiple relays is presented. Previous results were available only for upto three relays. The algorithm is unique in the sense that  the actual mathematical expressions, that are prohibitively large, need not be explicitly obtained. The diversity gain due to multiple relays is shown through plots of the analytical BER, well supported by simulations. 
%
%\end{abstract}
% IEEEtran.cls defaults to using nonbold math in the Abstract.
% This preserves the distinction between vectors and scalars. However,
% if the journal you are submitting to favors bold math in the abstract,
% then you can use LaTeX's standard command \boldmath at the very start
% of the abstract to achieve this. Many IEEE journals frown on math
% in the abstract anyway.

% Note that keywords are not normally used for peerreview papers.
%\begin{IEEEkeywords}
%Cooperative diversity, decode and forward, piecewise linear
%\end{IEEEkeywords}



% For peer review papers, you can put extra information on the cover
% page as needed:
% \ifCLASSOPTIONpeerreview
% \begin{center} \bfseries EDICS Category: 3-BBND \end{center}
% \fi
%
% For peerreview papers, this IEEEtran command inserts a page break and
% creates the second title. It will be ignored for other modes.
%\IEEEpeerreviewmaketitle




\item
If 4-digit numbers greater than 5,000 are randomly formed from the digits 0, 1, 3, 5, and 7, what is the probability of forming a number divisible by 5 when:
\begin{enumerate}
    \item The digits are repeated?
    \item The repetition of digits is not allowed?
\end{enumerate}
\solution
%\begin{table}[H]
	\centering
\begin{tabular}{|c|c|c|}
\hline
Random variable &Value &Definition\\ \hline
\multirow{3}{*}{X} &0 &Slips of Rs 1\\
&1 &Slips of Rs 5\\
&2 &Slips of Rs 13\\ \hline
\multirow{2}{*}{Y} &0 &Box A\\
&1 &Box B\\\hline
\end{tabular}
\caption{}
\label{tab:Distribution}
\end{table}
See \tabref{tab:Distribution}.
\begin{align}
p_{Y}\brak{k}= \begin{cases} 
      \frac{1}{3} & {k=0} \\
      \frac{2}{3 }& {k=1} 
   \end{cases}
   \\
p_{Y|X}\brak{0|0} = \frac{19}{25}\, 
p_{Y|X}\brak{0|1} = \frac{6}{25}\,
p_{Y|X}\brak{1|0} = \frac{45}{50}\,
p_{Y|X}\brak{1|2} = \frac{5}{50}
\end{align}
The desired probability is the probability that a slip drawn at random is marked other than Rs 1,
\begin{align}
&=1-p_X\brak{0}\\
&= p_X(1) + p_X(2)
\end{align}
Using Bayes theorem,
\begin{align}
&= p_Y\brak{0} \times \pr{Y=0 | X=1} + p_Y\brak{1} \times \pr{Y=1|X=2}\\
&=\frac{1}{3} \times \frac{6}{25} + \frac{2}{3} \times \frac{5}{50}\\
&=\frac{11}{75}
\end{align}

\newpage

%\tableofcontents

\bigskip

\renewcommand{\thefigure}{\theenumi}
\renewcommand{\thetable}{\theenumi}
%\renewcommand{\theequation}{\theenumi}

%\begin{abstract}
%%\boldmath
%In this letter, an algorithm for evaluating the exact analytical bit error rate  (BER)  for the piecewise linear (PL) combiner for  multiple relays is presented. Previous results were available only for upto three relays. The algorithm is unique in the sense that  the actual mathematical expressions, that are prohibitively large, need not be explicitly obtained. The diversity gain due to multiple relays is shown through plots of the analytical BER, well supported by simulations. 
%
%\end{abstract}
% IEEEtran.cls defaults to using nonbold math in the Abstract.
% This preserves the distinction between vectors and scalars. However,
% if the journal you are submitting to favors bold math in the abstract,
% then you can use LaTeX's standard command \boldmath at the very start
% of the abstract to achieve this. Many IEEE journals frown on math
% in the abstract anyway.

% Note that keywords are not normally used for peerreview papers.
%\begin{IEEEkeywords}
%Cooperative diversity, decode and forward, piecewise linear
%\end{IEEEkeywords}



% For peer review papers, you can put extra information on the cover
% page as needed:
% \ifCLASSOPTIONpeerreview
% \begin{center} \bfseries EDICS Category: 3-BBND \end{center}
% \fi
%
% For peerreview papers, this IEEEtran command inserts a page break and
% creates the second title. It will be ignored for other modes.
%\IEEEpeerreviewmaketitle




\item Consider the probability space $\brak{\Omega, \mathcal{G}, P}$ where $\Omega = [0,2]$ and $\mathcal{G} = \cbrak{\phi, \Omega, [0,1], (1,2]}$. Let $X$ and $Y$ be two functions on $\Omega$ defined as
\begin{align*}
    X(\omega) = 
    \begin{cases}
        1 & \text{if }\omega \in [0, 1]\\
        2 & \text{if }\omega \in (1, 2]
    \end{cases}
\end{align*}
and
\begin{align*}
    Y(\omega) = 
    \begin{cases}
        2 & \text{if }\omega \in [0, 1.5]\\
        3 & \text{if }\omega \in (1.5, 2].
    \end{cases}
\end{align*}
Then which one of the following statements is true?
\begin{enumerate}
    \item [(A)] $X$ is a random variable with respect to $\mathcal{G}$, but $Y$ is not a random variable with respect to $\mathcal{G}$.
    \item [(B)] $Y$ is a random variable with respect to $\mathcal{G}$, but $X$ is not a random variable with respect to $\mathcal{G}$.
    \item [(C)] Neither $X$ nor $Y$ is a random variable with respect to $\mathcal{G}$.
    \item [(D)] Both $X$ and $Y$ are random variables with respect to $\mathcal{G}$.
\end{enumerate} \hfill (GATE ST 2023)\\
\solution
%\begin{table}[H]
	\centering
\begin{tabular}{|c|c|c|}
\hline
Random variable &Value &Definition\\ \hline
\multirow{3}{*}{X} &0 &Slips of Rs 1\\
&1 &Slips of Rs 5\\
&2 &Slips of Rs 13\\ \hline
\multirow{2}{*}{Y} &0 &Box A\\
&1 &Box B\\\hline
\end{tabular}
\caption{}
\label{tab:Distribution}
\end{table}
See \tabref{tab:Distribution}.
\begin{align}
p_{Y}\brak{k}= \begin{cases} 
      \frac{1}{3} & {k=0} \\
      \frac{2}{3 }& {k=1} 
   \end{cases}
   \\
p_{Y|X}\brak{0|0} = \frac{19}{25}\, 
p_{Y|X}\brak{0|1} = \frac{6}{25}\,
p_{Y|X}\brak{1|0} = \frac{45}{50}\,
p_{Y|X}\brak{1|2} = \frac{5}{50}
\end{align}
The desired probability is the probability that a slip drawn at random is marked other than Rs 1,
\begin{align}
&=1-p_X\brak{0}\\
&= p_X(1) + p_X(2)
\end{align}
Using Bayes theorem,
\begin{align}
&= p_Y\brak{0} \times \pr{Y=0 | X=1} + p_Y\brak{1} \times \pr{Y=1|X=2}\\
&=\frac{1}{3} \times \frac{6}{25} + \frac{2}{3} \times \frac{5}{50}\\
&=\frac{11}{75}
\end{align}

\newpage

%\tableofcontents

\bigskip

\renewcommand{\thefigure}{\theenumi}
\renewcommand{\thetable}{\theenumi}
%\renewcommand{\theequation}{\theenumi}

%\begin{abstract}
%%\boldmath
%In this letter, an algorithm for evaluating the exact analytical bit error rate  (BER)  for the piecewise linear (PL) combiner for  multiple relays is presented. Previous results were available only for upto three relays. The algorithm is unique in the sense that  the actual mathematical expressions, that are prohibitively large, need not be explicitly obtained. The diversity gain due to multiple relays is shown through plots of the analytical BER, well supported by simulations. 
%
%\end{abstract}
% IEEEtran.cls defaults to using nonbold math in the Abstract.
% This preserves the distinction between vectors and scalars. However,
% if the journal you are submitting to favors bold math in the abstract,
% then you can use LaTeX's standard command \boldmath at the very start
% of the abstract to achieve this. Many IEEE journals frown on math
% in the abstract anyway.

% Note that keywords are not normally used for peerreview papers.
%\begin{IEEEkeywords}
%Cooperative diversity, decode and forward, piecewise linear
%\end{IEEEkeywords}



% For peer review papers, you can put extra information on the cover
% page as needed:
% \ifCLASSOPTIONpeerreview
% \begin{center} \bfseries EDICS Category: 3-BBND \end{center}
% \fi
%
% For peerreview papers, this IEEEtran command inserts a page break and
% creates the second title. It will be ignored for other modes.
%\IEEEpeerreviewmaketitle




	\item  A die is loaded in such a way that each odd number is twice as likely to occur as
each even number. Find $P(G)$, where $G$ is the event that a number greater than
3 occurs on a single roll of the die.
\\
\solution
		%\begin{table}[H]
	\centering
\begin{tabular}{|c|c|c|}
\hline
Random variable &Value &Definition\\ \hline
\multirow{3}{*}{X} &0 &Slips of Rs 1\\
&1 &Slips of Rs 5\\
&2 &Slips of Rs 13\\ \hline
\multirow{2}{*}{Y} &0 &Box A\\
&1 &Box B\\\hline
\end{tabular}
\caption{}
\label{tab:Distribution}
\end{table}
See \tabref{tab:Distribution}.
\begin{align}
p_{Y}\brak{k}= \begin{cases} 
      \frac{1}{3} & {k=0} \\
      \frac{2}{3 }& {k=1} 
   \end{cases}
   \\
p_{Y|X}\brak{0|0} = \frac{19}{25}\, 
p_{Y|X}\brak{0|1} = \frac{6}{25}\,
p_{Y|X}\brak{1|0} = \frac{45}{50}\,
p_{Y|X}\brak{1|2} = \frac{5}{50}
\end{align}
The desired probability is the probability that a slip drawn at random is marked other than Rs 1,
\begin{align}
&=1-p_X\brak{0}\\
&= p_X(1) + p_X(2)
\end{align}
Using Bayes theorem,
\begin{align}
&= p_Y\brak{0} \times \pr{Y=0 | X=1} + p_Y\brak{1} \times \pr{Y=1|X=2}\\
&=\frac{1}{3} \times \frac{6}{25} + \frac{2}{3} \times \frac{5}{50}\\
&=\frac{11}{75}
\end{align}

\newpage

%\tableofcontents

\bigskip

\renewcommand{\thefigure}{\theenumi}
\renewcommand{\thetable}{\theenumi}
%\renewcommand{\theequation}{\theenumi}

%\begin{abstract}
%%\boldmath
%In this letter, an algorithm for evaluating the exact analytical bit error rate  (BER)  for the piecewise linear (PL) combiner for  multiple relays is presented. Previous results were available only for upto three relays. The algorithm is unique in the sense that  the actual mathematical expressions, that are prohibitively large, need not be explicitly obtained. The diversity gain due to multiple relays is shown through plots of the analytical BER, well supported by simulations. 
%
%\end{abstract}
% IEEEtran.cls defaults to using nonbold math in the Abstract.
% This preserves the distinction between vectors and scalars. However,
% if the journal you are submitting to favors bold math in the abstract,
% then you can use LaTeX's standard command \boldmath at the very start
% of the abstract to achieve this. Many IEEE journals frown on math
% in the abstract anyway.

% Note that keywords are not normally used for peerreview papers.
%\begin{IEEEkeywords}
%Cooperative diversity, decode and forward, piecewise linear
%\end{IEEEkeywords}



% For peer review papers, you can put extra information on the cover
% page as needed:
% \ifCLASSOPTIONpeerreview
% \begin{center} \bfseries EDICS Category: 3-BBND \end{center}
% \fi
%
% For peerreview papers, this IEEEtran command inserts a page break and
% creates the second title. It will be ignored for other modes.
%\IEEEpeerreviewmaketitle




	\item All the jacks, queens and kings are removed from a deck of 52 playing cards. The remaining cards are well shuffled and then one card is drawn at random. Giving ace a value 1 similar value for other cards, find the probability that the card has a value 
		\begin{enumerate}
			\item 7
			\item greater than 7
			\item less than 7
		\end{enumerate}
		%Number of cards left after removing all jacks, queens and kings 
\begin{align}
N	= 52 - 4\times 3
	= 40
\end{align}
%\begin{table}[H]
%\def\arraystretch{1.2}
%\begin{tabular}{|c|c|c|}
%\hline
%	\textbf{Parameter} &\textbf{Value} &\textbf{Description}\\ \hline
%	$X$ &1-10 &Represents the value of the card picked \\ \hline
%\end{tabular}
%\end{table}
Let $1 \le X \le 10$ be the value of the card picked.  Then,
\begin{align}
	p_X(k) &= \Pr(X=k)\ \forall\ 1 \leq k \leq 10\\
	&= \frac{4\times 1}{40}\\
	&= \frac{1}{10}\\
	\therefore p_X(k) &= 
	\begin{cases}
		\frac{1}{10} & 1 \leq k \leq 10\\
		0 & \text{otherwise}
	\end{cases}
\end{align}
and
\begin{align}
	F_{X}(k) &= \sum_{m=0}^{k}p_{X}(m) \quad 1 \leq k \leq 10\\
	&= \frac{k}{10}\\
	\therefore F_{X}(k) &= 
	\begin{cases}
		0 & k \leq 0\\
		\frac{k}{10} & 1\leq k \leq 10\\
		1 & k > 10 
	\end{cases}
\end{align}
\begin{enumerate}
	\item Probability that card has value equal to 7 is
		\begin{align}
			 p_{X}(7)
			= \frac{1}{10}
		\end{align}
	\item Probability that card has value greater than 7 is
		\begin{align}
			1 - F_X(7)
			&= 1 - \frac{7}{10}
			\\
			&= \frac{3}{10}
		\end{align}
	\item Probability that card has value less than 7 is
		\begin{align}
			 F_{X}(6)
			=\frac{6}{10}
		\end{align}
\end{enumerate}

  \item A Lot consists of 48 mobile phones of which 42 are good, 3 have only minor defects and 3 have major defects.Varnika will buy a phone if it is good but the trader will only buy a mobile if it has no major defects. One phone is selected at random from the lot. What is the probability that it is
\begin{enumerate}
	\item acceptable to Varnika?
            \item acceptable to the trader?
\end{enumerate}
\solution
	%\begin{table}[H]
	\centering
\begin{tabular}{|c|c|c|}
\hline
Random variable &Value &Definition\\ \hline
\multirow{3}{*}{X} &0 &Slips of Rs 1\\
&1 &Slips of Rs 5\\
&2 &Slips of Rs 13\\ \hline
\multirow{2}{*}{Y} &0 &Box A\\
&1 &Box B\\\hline
\end{tabular}
\caption{}
\label{tab:Distribution}
\end{table}
See \tabref{tab:Distribution}.
\begin{align}
p_{Y}\brak{k}= \begin{cases} 
      \frac{1}{3} & {k=0} \\
      \frac{2}{3 }& {k=1} 
   \end{cases}
   \\
p_{Y|X}\brak{0|0} = \frac{19}{25}\, 
p_{Y|X}\brak{0|1} = \frac{6}{25}\,
p_{Y|X}\brak{1|0} = \frac{45}{50}\,
p_{Y|X}\brak{1|2} = \frac{5}{50}
\end{align}
The desired probability is the probability that a slip drawn at random is marked other than Rs 1,
\begin{align}
&=1-p_X\brak{0}\\
&= p_X(1) + p_X(2)
\end{align}
Using Bayes theorem,
\begin{align}
&= p_Y\brak{0} \times \pr{Y=0 | X=1} + p_Y\brak{1} \times \pr{Y=1|X=2}\\
&=\frac{1}{3} \times \frac{6}{25} + \frac{2}{3} \times \frac{5}{50}\\
&=\frac{11}{75}
\end{align}

\newpage

%\tableofcontents

\bigskip

\renewcommand{\thefigure}{\theenumi}
\renewcommand{\thetable}{\theenumi}
%\renewcommand{\theequation}{\theenumi}

%\begin{abstract}
%%\boldmath
%In this letter, an algorithm for evaluating the exact analytical bit error rate  (BER)  for the piecewise linear (PL) combiner for  multiple relays is presented. Previous results were available only for upto three relays. The algorithm is unique in the sense that  the actual mathematical expressions, that are prohibitively large, need not be explicitly obtained. The diversity gain due to multiple relays is shown through plots of the analytical BER, well supported by simulations. 
%
%\end{abstract}
% IEEEtran.cls defaults to using nonbold math in the Abstract.
% This preserves the distinction between vectors and scalars. However,
% if the journal you are submitting to favors bold math in the abstract,
% then you can use LaTeX's standard command \boldmath at the very start
% of the abstract to achieve this. Many IEEE journals frown on math
% in the abstract anyway.

% Note that keywords are not normally used for peerreview papers.
%\begin{IEEEkeywords}
%Cooperative diversity, decode and forward, piecewise linear
%\end{IEEEkeywords}



% For peer review papers, you can put extra information on the cover
% page as needed:
% \ifCLASSOPTIONpeerreview
% \begin{center} \bfseries EDICS Category: 3-BBND \end{center}
% \fi
%
% For peerreview papers, this IEEEtran command inserts a page break and
% creates the second title. It will be ignored for other modes.
%\IEEEpeerreviewmaketitle




 \item A student says that if you throw a die, it will show up 1 or not 1. Therefore, the probability of getting 1 and the probability of getting 'not 1' each is equal to $\frac{1}{2}$. Is this correct? Give reasons.\\
 \solution
        %\begin{table}[H]
	\centering
\begin{tabular}{|c|c|c|}
\hline
Random variable &Value &Definition\\ \hline
\multirow{3}{*}{X} &0 &Slips of Rs 1\\
&1 &Slips of Rs 5\\
&2 &Slips of Rs 13\\ \hline
\multirow{2}{*}{Y} &0 &Box A\\
&1 &Box B\\\hline
\end{tabular}
\caption{}
\label{tab:Distribution}
\end{table}
See \tabref{tab:Distribution}.
\begin{align}
p_{Y}\brak{k}= \begin{cases} 
      \frac{1}{3} & {k=0} \\
      \frac{2}{3 }& {k=1} 
   \end{cases}
   \\
p_{Y|X}\brak{0|0} = \frac{19}{25}\, 
p_{Y|X}\brak{0|1} = \frac{6}{25}\,
p_{Y|X}\brak{1|0} = \frac{45}{50}\,
p_{Y|X}\brak{1|2} = \frac{5}{50}
\end{align}
The desired probability is the probability that a slip drawn at random is marked other than Rs 1,
\begin{align}
&=1-p_X\brak{0}\\
&= p_X(1) + p_X(2)
\end{align}
Using Bayes theorem,
\begin{align}
&= p_Y\brak{0} \times \pr{Y=0 | X=1} + p_Y\brak{1} \times \pr{Y=1|X=2}\\
&=\frac{1}{3} \times \frac{6}{25} + \frac{2}{3} \times \frac{5}{50}\\
&=\frac{11}{75}
\end{align}

\newpage

%\tableofcontents

\bigskip

\renewcommand{\thefigure}{\theenumi}
\renewcommand{\thetable}{\theenumi}
%\renewcommand{\theequation}{\theenumi}

%\begin{abstract}
%%\boldmath
%In this letter, an algorithm for evaluating the exact analytical bit error rate  (BER)  for the piecewise linear (PL) combiner for  multiple relays is presented. Previous results were available only for upto three relays. The algorithm is unique in the sense that  the actual mathematical expressions, that are prohibitively large, need not be explicitly obtained. The diversity gain due to multiple relays is shown through plots of the analytical BER, well supported by simulations. 
%
%\end{abstract}
% IEEEtran.cls defaults to using nonbold math in the Abstract.
% This preserves the distinction between vectors and scalars. However,
% if the journal you are submitting to favors bold math in the abstract,
% then you can use LaTeX's standard command \boldmath at the very start
% of the abstract to achieve this. Many IEEE journals frown on math
% in the abstract anyway.

% Note that keywords are not normally used for peerreview papers.
%\begin{IEEEkeywords}
%Cooperative diversity, decode and forward, piecewise linear
%\end{IEEEkeywords}



% For peer review papers, you can put extra information on the cover
% page as needed:
% \ifCLASSOPTIONpeerreview
% \begin{center} \bfseries EDICS Category: 3-BBND \end{center}
% \fi
%
% For peerreview papers, this IEEEtran command inserts a page break and
% creates the second title. It will be ignored for other modes.
%\IEEEpeerreviewmaketitle




   \item Four candidates A, B, C, D have ap-
plied for the assignment to coach a school cricket
team. If A is twice as likely to be selected as B, and
B and C are given about the same chance of being
selected, while C is twice as likely to be selected
as D, what are the probabilities that
\begin{enumerate}
\item C will be selected?
\item A will not be selected?
\end{enumerate}
	%\begin{table}[H]
	\centering
\begin{tabular}{|c|c|c|}
\hline
Random variable &Value &Definition\\ \hline
\multirow{3}{*}{X} &0 &Slips of Rs 1\\
&1 &Slips of Rs 5\\
&2 &Slips of Rs 13\\ \hline
\multirow{2}{*}{Y} &0 &Box A\\
&1 &Box B\\\hline
\end{tabular}
\caption{}
\label{tab:Distribution}
\end{table}
See \tabref{tab:Distribution}.
\begin{align}
p_{Y}\brak{k}= \begin{cases} 
      \frac{1}{3} & {k=0} \\
      \frac{2}{3 }& {k=1} 
   \end{cases}
   \\
p_{Y|X}\brak{0|0} = \frac{19}{25}\, 
p_{Y|X}\brak{0|1} = \frac{6}{25}\,
p_{Y|X}\brak{1|0} = \frac{45}{50}\,
p_{Y|X}\brak{1|2} = \frac{5}{50}
\end{align}
The desired probability is the probability that a slip drawn at random is marked other than Rs 1,
\begin{align}
&=1-p_X\brak{0}\\
&= p_X(1) + p_X(2)
\end{align}
Using Bayes theorem,
\begin{align}
&= p_Y\brak{0} \times \pr{Y=0 | X=1} + p_Y\brak{1} \times \pr{Y=1|X=2}\\
&=\frac{1}{3} \times \frac{6}{25} + \frac{2}{3} \times \frac{5}{50}\\
&=\frac{11}{75}
\end{align}

\newpage

%\tableofcontents

\bigskip

\renewcommand{\thefigure}{\theenumi}
\renewcommand{\thetable}{\theenumi}
%\renewcommand{\theequation}{\theenumi}

%\begin{abstract}
%%\boldmath
%In this letter, an algorithm for evaluating the exact analytical bit error rate  (BER)  for the piecewise linear (PL) combiner for  multiple relays is presented. Previous results were available only for upto three relays. The algorithm is unique in the sense that  the actual mathematical expressions, that are prohibitively large, need not be explicitly obtained. The diversity gain due to multiple relays is shown through plots of the analytical BER, well supported by simulations. 
%
%\end{abstract}
% IEEEtran.cls defaults to using nonbold math in the Abstract.
% This preserves the distinction between vectors and scalars. However,
% if the journal you are submitting to favors bold math in the abstract,
% then you can use LaTeX's standard command \boldmath at the very start
% of the abstract to achieve this. Many IEEE journals frown on math
% in the abstract anyway.

% Note that keywords are not normally used for peerreview papers.
%\begin{IEEEkeywords}
%Cooperative diversity, decode and forward, piecewise linear
%\end{IEEEkeywords}



% For peer review papers, you can put extra information on the cover
% page as needed:
% \ifCLASSOPTIONpeerreview
% \begin{center} \bfseries EDICS Category: 3-BBND \end{center}
% \fi
%
% For peerreview papers, this IEEEtran command inserts a page break and
% creates the second title. It will be ignored for other modes.
%\IEEEpeerreviewmaketitle




 \item A bag contain 24 balls of which $x$ balls are red, $2x$ are white and $3x$ are blue. A ball is selected at random, What is the probability that it is
\begin{enumerate}[label=\alph*)]
\item not red ?
\item white ?
\end{enumerate}
%\begin{table}[H]
	\centering
\begin{tabular}{|c|c|c|}
\hline
Random variable &Value &Definition\\ \hline
\multirow{3}{*}{X} &0 &Slips of Rs 1\\
&1 &Slips of Rs 5\\
&2 &Slips of Rs 13\\ \hline
\multirow{2}{*}{Y} &0 &Box A\\
&1 &Box B\\\hline
\end{tabular}
\caption{}
\label{tab:Distribution}
\end{table}
See \tabref{tab:Distribution}.
\begin{align}
p_{Y}\brak{k}= \begin{cases} 
      \frac{1}{3} & {k=0} \\
      \frac{2}{3 }& {k=1} 
   \end{cases}
   \\
p_{Y|X}\brak{0|0} = \frac{19}{25}\, 
p_{Y|X}\brak{0|1} = \frac{6}{25}\,
p_{Y|X}\brak{1|0} = \frac{45}{50}\,
p_{Y|X}\brak{1|2} = \frac{5}{50}
\end{align}
The desired probability is the probability that a slip drawn at random is marked other than Rs 1,
\begin{align}
&=1-p_X\brak{0}\\
&= p_X(1) + p_X(2)
\end{align}
Using Bayes theorem,
\begin{align}
&= p_Y\brak{0} \times \pr{Y=0 | X=1} + p_Y\brak{1} \times \pr{Y=1|X=2}\\
&=\frac{1}{3} \times \frac{6}{25} + \frac{2}{3} \times \frac{5}{50}\\
&=\frac{11}{75}
\end{align}

\newpage

%\tableofcontents

\bigskip

\renewcommand{\thefigure}{\theenumi}
\renewcommand{\thetable}{\theenumi}
%\renewcommand{\theequation}{\theenumi}

%\begin{abstract}
%%\boldmath
%In this letter, an algorithm for evaluating the exact analytical bit error rate  (BER)  for the piecewise linear (PL) combiner for  multiple relays is presented. Previous results were available only for upto three relays. The algorithm is unique in the sense that  the actual mathematical expressions, that are prohibitively large, need not be explicitly obtained. The diversity gain due to multiple relays is shown through plots of the analytical BER, well supported by simulations. 
%
%\end{abstract}
% IEEEtran.cls defaults to using nonbold math in the Abstract.
% This preserves the distinction between vectors and scalars. However,
% if the journal you are submitting to favors bold math in the abstract,
% then you can use LaTeX's standard command \boldmath at the very start
% of the abstract to achieve this. Many IEEE journals frown on math
% in the abstract anyway.

% Note that keywords are not normally used for peerreview papers.
%\begin{IEEEkeywords}
%Cooperative diversity, decode and forward, piecewise linear
%\end{IEEEkeywords}



% For peer review papers, you can put extra information on the cover
% page as needed:
% \ifCLASSOPTIONpeerreview
% \begin{center} \bfseries EDICS Category: 3-BBND \end{center}
% \fi
%
% For peerreview papers, this IEEEtran command inserts a page break and
% creates the second title. It will be ignored for other modes.
%\IEEEpeerreviewmaketitle




If the letters of the word ASSASSINATION are arranged at random. Find the Probability that
\begin{enumerate}[label=(\alph*)]
\item Four $S's$ come consecutively in the word
\item Two  $I's$ and two $N's$ come together
\item All $A's$ are not coming together
\item No two $A's$ are coming together
\end{enumerate}
%\begin{table}[H]
	\centering
\begin{tabular}{|c|c|c|}
\hline
Random variable &Value &Definition\\ \hline
\multirow{3}{*}{X} &0 &Slips of Rs 1\\
&1 &Slips of Rs 5\\
&2 &Slips of Rs 13\\ \hline
\multirow{2}{*}{Y} &0 &Box A\\
&1 &Box B\\\hline
\end{tabular}
\caption{}
\label{tab:Distribution}
\end{table}
See \tabref{tab:Distribution}.
\begin{align}
p_{Y}\brak{k}= \begin{cases} 
      \frac{1}{3} & {k=0} \\
      \frac{2}{3 }& {k=1} 
   \end{cases}
   \\
p_{Y|X}\brak{0|0} = \frac{19}{25}\, 
p_{Y|X}\brak{0|1} = \frac{6}{25}\,
p_{Y|X}\brak{1|0} = \frac{45}{50}\,
p_{Y|X}\brak{1|2} = \frac{5}{50}
\end{align}
The desired probability is the probability that a slip drawn at random is marked other than Rs 1,
\begin{align}
&=1-p_X\brak{0}\\
&= p_X(1) + p_X(2)
\end{align}
Using Bayes theorem,
\begin{align}
&= p_Y\brak{0} \times \pr{Y=0 | X=1} + p_Y\brak{1} \times \pr{Y=1|X=2}\\
&=\frac{1}{3} \times \frac{6}{25} + \frac{2}{3} \times \frac{5}{50}\\
&=\frac{11}{75}
\end{align}

\newpage

%\tableofcontents

\bigskip

\renewcommand{\thefigure}{\theenumi}
\renewcommand{\thetable}{\theenumi}
%\renewcommand{\theequation}{\theenumi}

%\begin{abstract}
%%\boldmath
%In this letter, an algorithm for evaluating the exact analytical bit error rate  (BER)  for the piecewise linear (PL) combiner for  multiple relays is presented. Previous results were available only for upto three relays. The algorithm is unique in the sense that  the actual mathematical expressions, that are prohibitively large, need not be explicitly obtained. The diversity gain due to multiple relays is shown through plots of the analytical BER, well supported by simulations. 
%
%\end{abstract}
% IEEEtran.cls defaults to using nonbold math in the Abstract.
% This preserves the distinction between vectors and scalars. However,
% if the journal you are submitting to favors bold math in the abstract,
% then you can use LaTeX's standard command \boldmath at the very start
% of the abstract to achieve this. Many IEEE journals frown on math
% in the abstract anyway.

% Note that keywords are not normally used for peerreview papers.
%\begin{IEEEkeywords}
%Cooperative diversity, decode and forward, piecewise linear
%\end{IEEEkeywords}



% For peer review papers, you can put extra information on the cover
% page as needed:
% \ifCLASSOPTIONpeerreview
% \begin{center} \bfseries EDICS Category: 3-BBND \end{center}
% \fi
%
% For peerreview papers, this IEEEtran command inserts a page break and
% creates the second title. It will be ignored for other modes.
%\IEEEpeerreviewmaketitle




	\item One urn contains two black balls (labelled B1 and B2) and one white ball. A
	second urn contains one black ball and two white balls (labelled W1 and W2).
	Suppose the following experiment is performed. One of the two urns is chosen
	at random. Next a ball is randomly chosen from the urn. Then a second ball is
	chosen at random from the same urn without replacing the first ball.
	
	\begin{enumerate}
	\item What is the probability that two black balls are chosen?
	
	\item What is the probability that two balls of opposite colour are chosen?
	\end{enumerate}
	\solution
	%\begin{align}
    \label{eq:12.13.6.18.1}
	\because	\pr{A|B} &> \pr{A},\
\frac{\pr{AB}}{\pr{B}} > \pr{A}
\\
    \label{eq:12.13.6.18.2}
	\implies \pr{AB} &> \pr{A}\pr{B}
	\\
	\text{or, } \frac{\pr{AB}}{\pr{A}} &=\pr{B|A} > \pr{A}
\end{align}

\end{enumerate}

	\item A bag contains $5$ red balls and some blue balls. If the probability of drawing a blue ball is double that if a red ball, determine the number of blue balls in the bag. 
		\\
\solution
		%\begin{enumerate}[label=\thesection.\arabic*,ref=\thesection.\theenumi]
	\item One card is drawn from a well-shuffled deck of 52 cards. Find the probability of getting
\begin{enumerate}
\item A king of red colour 
\item A face card 
\item A red face card
\item The jack of hearts
\item A spade
\item The queen of diamonds

\end{enumerate}
\solution
		%\begin{table}[H]
	\centering
\begin{tabular}{|c|c|c|}
\hline
Random variable &Value &Definition\\ \hline
\multirow{3}{*}{X} &0 &Slips of Rs 1\\
&1 &Slips of Rs 5\\
&2 &Slips of Rs 13\\ \hline
\multirow{2}{*}{Y} &0 &Box A\\
&1 &Box B\\\hline
\end{tabular}
\caption{}
\label{tab:Distribution}
\end{table}
See \tabref{tab:Distribution}.
\begin{align}
p_{Y}\brak{k}= \begin{cases} 
      \frac{1}{3} & {k=0} \\
      \frac{2}{3 }& {k=1} 
   \end{cases}
   \\
p_{Y|X}\brak{0|0} = \frac{19}{25}\, 
p_{Y|X}\brak{0|1} = \frac{6}{25}\,
p_{Y|X}\brak{1|0} = \frac{45}{50}\,
p_{Y|X}\brak{1|2} = \frac{5}{50}
\end{align}
The desired probability is the probability that a slip drawn at random is marked other than Rs 1,
\begin{align}
&=1-p_X\brak{0}\\
&= p_X(1) + p_X(2)
\end{align}
Using Bayes theorem,
\begin{align}
&= p_Y\brak{0} \times \pr{Y=0 | X=1} + p_Y\brak{1} \times \pr{Y=1|X=2}\\
&=\frac{1}{3} \times \frac{6}{25} + \frac{2}{3} \times \frac{5}{50}\\
&=\frac{11}{75}
\end{align}

\newpage

%\tableofcontents

\bigskip

\renewcommand{\thefigure}{\theenumi}
\renewcommand{\thetable}{\theenumi}
%\renewcommand{\theequation}{\theenumi}

%\begin{abstract}
%%\boldmath
%In this letter, an algorithm for evaluating the exact analytical bit error rate  (BER)  for the piecewise linear (PL) combiner for  multiple relays is presented. Previous results were available only for upto three relays. The algorithm is unique in the sense that  the actual mathematical expressions, that are prohibitively large, need not be explicitly obtained. The diversity gain due to multiple relays is shown through plots of the analytical BER, well supported by simulations. 
%
%\end{abstract}
% IEEEtran.cls defaults to using nonbold math in the Abstract.
% This preserves the distinction between vectors and scalars. However,
% if the journal you are submitting to favors bold math in the abstract,
% then you can use LaTeX's standard command \boldmath at the very start
% of the abstract to achieve this. Many IEEE journals frown on math
% in the abstract anyway.

% Note that keywords are not normally used for peerreview papers.
%\begin{IEEEkeywords}
%Cooperative diversity, decode and forward, piecewise linear
%\end{IEEEkeywords}



% For peer review papers, you can put extra information on the cover
% page as needed:
% \ifCLASSOPTIONpeerreview
% \begin{center} \bfseries EDICS Category: 3-BBND \end{center}
% \fi
%
% For peerreview papers, this IEEEtran command inserts a page break and
% creates the second title. It will be ignored for other modes.
%\IEEEpeerreviewmaketitle




	\item Five cards—the ten, jack, queen, king and ace of diamonds, are well-shuffled with their face downwards. One card is then picked up at random.
\begin{enumerate}
\item
What is the probability that the card is the queen? 
\item
If the queen is drawn and put aside, what is the probability that the second card picked up is (a) an ace? (b) a queen?\\
\end{enumerate}
\solution
		%\begin{enumerate}[label=\thesection.\arabic*,ref=\thesection.\theenumi]
	\item One card is drawn from a well-shuffled deck of 52 cards. Find the probability of getting
\begin{enumerate}
\item A king of red colour 
\item A face card 
\item A red face card
\item The jack of hearts
\item A spade
\item The queen of diamonds

\end{enumerate}
\solution
		%\input{ncert/10/15/1/14/main.tex}
	\item Five cards—the ten, jack, queen, king and ace of diamonds, are well-shuffled with their face downwards. One card is then picked up at random.
\begin{enumerate}
\item
What is the probability that the card is the queen? 
\item
If the queen is drawn and put aside, what is the probability that the second card picked up is (a) an ace? (b) a queen?\\
\end{enumerate}
\solution
		%\input{ncert/10/15/1/15/defs.tex}
	\item A bag contains $5$ red balls and some blue balls. If the probability of drawing a blue ball is double that if a red ball, determine the number of blue balls in the bag. 
		\\
\solution
		%\input{ncert/10/15/2/3/defs.tex}
	\item A card is selected from a pack of 52 cards.
 \begin{enumerate}[label=(\alph*)] 
                 \item How many points are there in the sample space?
                 \item Calculate the probability that the card is an ace of spades.
                 \item Calculate the probability that the card is (i) an ace and (ii) black card.
 \end{enumerate}
\solution
		%\input{ncert/11/16/3/4/main.tex}
\item Four cards are drawn from a well-shuffled deck of 52 cards. What is the probability of obtaining 3 diamonds and one spade.
\\
\solution
		%\input{ncert/11/16/4/2/defs.tex}
\item In a certain lottery 10,000 tickets are sold and ten equal prizes are awarded. What is the probability of not getting a prize if you buy (a) one ticket (b) two tickets (c) 10 tickets ?	
\\
\solution
		%\input{ncert/11/16/4/4/defs.tex}
		%
\item 
Out of 100 students, two sections of 40 and 60 are formed. If you and your friend are among the 100 students, what is the probability that
\begin{enumerate}
\item you both enter the same section?
\item you both enter the different sections?
\end{enumerate}
\solution
		%\input{ncert/11/16/4/5/defs.tex}
	\item 
The number lock of a suitcase has 4 wheels each labelled with ten digits i.e. from 0 to 9.The lock opens with a sequence of four digits with no repeats.What is the probability of a person getting the right sequence to open the suitcase.
\\
\solution
		%\input{ncert/11/16/4/10/defs.tex}
		%
\item 
Two cards are drawn at random and without replacement from a pack of 52 playing cards. Find the probability that both the cards are black.
\\
\solution
		%\input{ncert/12/13/2/2/defs.tex}
		\item A box of oranges is inspected by examining three randomly selected oranges drawn without replacement. If all the three oranges are good, the box is approved for sale, otherwise, it is rejected. Find the probability that a box containing 15 oranges out of which 12 are good and 3 are bad ones will be approved for sale.
		\label{ncert/12/13/2/3/defs.tex}
		\item Two balls are drawn at random with replacement from a box containing 10 black and 8 red balls. Find the probability that
		\label{ncert/12/13/2/12}
\begin{enumerate}
\item both balls are red.
\item first ball is black and second is red.
\item one of them is black and other is red.
\end{enumerate}

\item In a hostel, 60\% of the students read Hindi newspaper, 40\% read English newspaper and 20\% read both Hindi and English newspapers. A student is selected at random.
		\label{ncert/12/13/2/15}
\begin{enumerate}
\item Find the probability that she reads neither Hindi nor English newspapers.
\item If she reads Hindi newspaper, find the probability that she reads English newspaper.
\item If she reads English newspaper, find the probability that she reads Hindi newspaper.\\
\end{enumerate}
\item The probability of obtaining an even prime number on each die, when a pair of dice is rolled is 
\begin{enumerate}
    \item $0$ 
    
    \item $\frac{1}{3}$ 
    
    \item $\frac{1}{12}$ 
    
    \item $\frac{1}{36}$ 
\end{enumerate}
\solution
		%\input{ncert/12/13/2/17/defs.tex}
	\item A bag contains 4 red and 4 black balls, another bag contains 2 red and 6 black balls. One of the two bags is selected at random and a ball is drawn from the bag which is found to be red. Find the probability that the ball is drawn from the first bag.
\\
\solution
		%\input{ncert/12/13/3/2/main.tex}
  \item
  Cards with numbers 2 to 101 are placed in a box. A card is selected at random.Find the probability that the card has
\begin{enumerate}[label=(\roman*)]
	\item an even number 
	\item a square number
\end{enumerate}
\solution
%\input{exemplar/10/13/3/32/main.tex}
\item
The king, queen and jack of clubs are removed from a deck of 52 playing cards and then well shuffled. Now one card is drawn at random from the remaining cards.  Determine the probability that the card is
\begin{enumerate}[label=(\roman*)]
\item a club
\item 10 of hearts
\end{enumerate}
\solution
%\input{exemplar/10/13/3/29/main.tex}
\item A team of medical students doing their internship have to assist during surgeries
at a city hospital. The probabilities of surgeries rated as very complex, complex,
routine, simple or very simple are respectively, 0.15, 0.20, 0.31, 0.26, .08. Find
the probabilities that a particular surgery will be rated
\begin{enumerate}
	\item complex or very complex;
	\item neither very complex nor very simple;
	\item routine or complex
	\item routine or simple
\end{enumerate}
\solution
%\input{exemplar/11/16/3/8(1)/main.tex}
\item A card is selected from a pack of 52 cards.
\begin{enumerate}[label=(\alph*)]
    \item How many points are there in the sample space?
    \item Calculate the probability that the card is an ace of spades.
    \item Calculate the probability that the card is (i) an ace and (ii) black card.
\end{enumerate}
\solution
%\input{exemplar/11/16/3/4/main2.tex}
\item The probability that a non leap year selected at random will contain 53 sundays.
\\
\solution
%\input{exemplar/10/13/1/19/main.tex}
\item One of the four persons John, Rita, Aslam or Gurpreet will be promoted next
month. Consequently the sample space consists of four elementary outcomes
S = {John promoted, Rita promoted, Aslam promoted, Gurpreet promoted}
You are told that the chances of John’s promotion is same as that of Gurpreet,
Rita’s chances of promotion are twice as likely as Johns. Aslam’s chances are
four times that of John.
\begin{enumerate}
	\item Determine
	\begin{enumerate}
		\item P (John promoted)
		\item P (Rita promoted)
		\item P (Aslam promoted)
		\item P (Gurpreet promoted)
	\end{enumerate}
	\item If A = {John promoted or Gurpreet promoted}, find P (A).
\end{enumerate}
\solution
%\input{exemplar/11/16/3/10/main.tex}
\item A card is drawn from a deck of 52 cards. Find the probability of getting a king or a heart or a red card.\\
\solution
%\input{exemplar/11/16/3/15/main.tex}
\item The probability that a student will pass his examination is 0.73, the probability of
the student getting a compartment is 0.13, and the probability that the student will
either pass or get compartment is 0.96. State True or False.\\
\solution
%\input{exemplar/11/16/3/31/main.tex}
\item A card is selected from a pack of 52 cards\\
\begin{enumerate}[label=(\alph*)]
\item How many points are there in the sample space?
\item Calculate the probability that the cards is an ace of spades.
\item Calculate the probability that the card is (i) an ace (ii)black card.\\
\end{enumerate}
%\input{ncert/11/16/3/4_1/Prob_4.tex}
\item In a non-leap year, the probability of having 53 tuesdays or 53 wednesdays is\\
\solution
%\input{exemplar/11/16/3/18/main.tex}
\item There are 1000 sealed envelopes in a box, 10 of them contain a cash prize of
Rs 100 each, 100 of them contain a cash prize of Rs 50 each and 200 of them
contain a cash prize of Rs 10 each and rest do not contain any cash prize. If they
are well shuffled and an envelope is picked up out, what is the probability that it
contains no cash prize?\\
\solution
%\input{exemplar/10/13/3/34/main.tex}
\item 
A die is thrown and a card is selected at random from a deck of 52 playing cards. The probability of getting an even number on the die and a spade card.\\
\solution
%\input{exemplar/12/13/3/78/main.tex}
\item
If 4-digit numbers greater than 5,000 are randomly formed from the digits 0, 1, 3, 5, and 7, what is the probability of forming a number divisible by 5 when:
\begin{enumerate}
    \item The digits are repeated?
    \item The repetition of digits is not allowed?
\end{enumerate}
\solution
%\input{ncert/11/16/4/9/main.tex}
\item Consider the probability space $\brak{\Omega, \mathcal{G}, P}$ where $\Omega = [0,2]$ and $\mathcal{G} = \cbrak{\phi, \Omega, [0,1], (1,2]}$. Let $X$ and $Y$ be two functions on $\Omega$ defined as
\begin{align*}
    X(\omega) = 
    \begin{cases}
        1 & \text{if }\omega \in [0, 1]\\
        2 & \text{if }\omega \in (1, 2]
    \end{cases}
\end{align*}
and
\begin{align*}
    Y(\omega) = 
    \begin{cases}
        2 & \text{if }\omega \in [0, 1.5]\\
        3 & \text{if }\omega \in (1.5, 2].
    \end{cases}
\end{align*}
Then which one of the following statements is true?
\begin{enumerate}
    \item [(A)] $X$ is a random variable with respect to $\mathcal{G}$, but $Y$ is not a random variable with respect to $\mathcal{G}$.
    \item [(B)] $Y$ is a random variable with respect to $\mathcal{G}$, but $X$ is not a random variable with respect to $\mathcal{G}$.
    \item [(C)] Neither $X$ nor $Y$ is a random variable with respect to $\mathcal{G}$.
    \item [(D)] Both $X$ and $Y$ are random variables with respect to $\mathcal{G}$.
\end{enumerate} \hfill (GATE ST 2023)\\
\solution
%\input{gate/ST/2023/14/main.tex}
	\item  A die is loaded in such a way that each odd number is twice as likely to occur as
each even number. Find $P(G)$, where $G$ is the event that a number greater than
3 occurs on a single roll of the die.
\\
\solution
		%\input{exemplar/11/16/3/5/main.tex}
	\item All the jacks, queens and kings are removed from a deck of 52 playing cards. The remaining cards are well shuffled and then one card is drawn at random. Giving ace a value 1 similar value for other cards, find the probability that the card has a value 
		\begin{enumerate}
			\item 7
			\item greater than 7
			\item less than 7
		\end{enumerate}
		%\input{exemplar/10/13/3/30/main.tex}
  \item A Lot consists of 48 mobile phones of which 42 are good, 3 have only minor defects and 3 have major defects.Varnika will buy a phone if it is good but the trader will only buy a mobile if it has no major defects. One phone is selected at random from the lot. What is the probability that it is
\begin{enumerate}
	\item acceptable to Varnika?
            \item acceptable to the trader?
\end{enumerate}
\solution
	%\input{exemplar/10/13/3/40/main.tex}
 \item A student says that if you throw a die, it will show up 1 or not 1. Therefore, the probability of getting 1 and the probability of getting 'not 1' each is equal to $\frac{1}{2}$. Is this correct? Give reasons.\\
 \solution
        %\input{exemplar/10/13/2/9/main.tex}
   \item Four candidates A, B, C, D have ap-
plied for the assignment to coach a school cricket
team. If A is twice as likely to be selected as B, and
B and C are given about the same chance of being
selected, while C is twice as likely to be selected
as D, what are the probabilities that
\begin{enumerate}
\item C will be selected?
\item A will not be selected?
\end{enumerate}
	%\input{exemplar/11/16/3/9/main.tex}
 \item A bag contain 24 balls of which $x$ balls are red, $2x$ are white and $3x$ are blue. A ball is selected at random, What is the probability that it is
\begin{enumerate}[label=\alph*)]
\item not red ?
\item white ?
\end{enumerate}
%\input{exemplar/10/13/3/41/main.tex}
If the letters of the word ASSASSINATION are arranged at random. Find the Probability that
\begin{enumerate}[label=(\alph*)]
\item Four $S's$ come consecutively in the word
\item Two  $I's$ and two $N's$ come together
\item All $A's$ are not coming together
\item No two $A's$ are coming together
\end{enumerate}
%\input{exemplar/11/16/3/14/main.tex}
	\item One urn contains two black balls (labelled B1 and B2) and one white ball. A
	second urn contains one black ball and two white balls (labelled W1 and W2).
	Suppose the following experiment is performed. One of the two urns is chosen
	at random. Next a ball is randomly chosen from the urn. Then a second ball is
	chosen at random from the same urn without replacing the first ball.
	
	\begin{enumerate}
	\item What is the probability that two black balls are chosen?
	
	\item What is the probability that two balls of opposite colour are chosen?
	\end{enumerate}
	\solution
	%\input{exemplar/11/16/3/12/main1.tex}
\end{enumerate}

	\item A bag contains $5$ red balls and some blue balls. If the probability of drawing a blue ball is double that if a red ball, determine the number of blue balls in the bag. 
		\\
\solution
		%\begin{enumerate}[label=\thesection.\arabic*,ref=\thesection.\theenumi]
	\item One card is drawn from a well-shuffled deck of 52 cards. Find the probability of getting
\begin{enumerate}
\item A king of red colour 
\item A face card 
\item A red face card
\item The jack of hearts
\item A spade
\item The queen of diamonds

\end{enumerate}
\solution
		%\input{ncert/10/15/1/14/main.tex}
	\item Five cards—the ten, jack, queen, king and ace of diamonds, are well-shuffled with their face downwards. One card is then picked up at random.
\begin{enumerate}
\item
What is the probability that the card is the queen? 
\item
If the queen is drawn and put aside, what is the probability that the second card picked up is (a) an ace? (b) a queen?\\
\end{enumerate}
\solution
		%\input{ncert/10/15/1/15/defs.tex}
	\item A bag contains $5$ red balls and some blue balls. If the probability of drawing a blue ball is double that if a red ball, determine the number of blue balls in the bag. 
		\\
\solution
		%\input{ncert/10/15/2/3/defs.tex}
	\item A card is selected from a pack of 52 cards.
 \begin{enumerate}[label=(\alph*)] 
                 \item How many points are there in the sample space?
                 \item Calculate the probability that the card is an ace of spades.
                 \item Calculate the probability that the card is (i) an ace and (ii) black card.
 \end{enumerate}
\solution
		%\input{ncert/11/16/3/4/main.tex}
\item Four cards are drawn from a well-shuffled deck of 52 cards. What is the probability of obtaining 3 diamonds and one spade.
\\
\solution
		%\input{ncert/11/16/4/2/defs.tex}
\item In a certain lottery 10,000 tickets are sold and ten equal prizes are awarded. What is the probability of not getting a prize if you buy (a) one ticket (b) two tickets (c) 10 tickets ?	
\\
\solution
		%\input{ncert/11/16/4/4/defs.tex}
		%
\item 
Out of 100 students, two sections of 40 and 60 are formed. If you and your friend are among the 100 students, what is the probability that
\begin{enumerate}
\item you both enter the same section?
\item you both enter the different sections?
\end{enumerate}
\solution
		%\input{ncert/11/16/4/5/defs.tex}
	\item 
The number lock of a suitcase has 4 wheels each labelled with ten digits i.e. from 0 to 9.The lock opens with a sequence of four digits with no repeats.What is the probability of a person getting the right sequence to open the suitcase.
\\
\solution
		%\input{ncert/11/16/4/10/defs.tex}
		%
\item 
Two cards are drawn at random and without replacement from a pack of 52 playing cards. Find the probability that both the cards are black.
\\
\solution
		%\input{ncert/12/13/2/2/defs.tex}
		\item A box of oranges is inspected by examining three randomly selected oranges drawn without replacement. If all the three oranges are good, the box is approved for sale, otherwise, it is rejected. Find the probability that a box containing 15 oranges out of which 12 are good and 3 are bad ones will be approved for sale.
		\label{ncert/12/13/2/3/defs.tex}
		\item Two balls are drawn at random with replacement from a box containing 10 black and 8 red balls. Find the probability that
		\label{ncert/12/13/2/12}
\begin{enumerate}
\item both balls are red.
\item first ball is black and second is red.
\item one of them is black and other is red.
\end{enumerate}

\item In a hostel, 60\% of the students read Hindi newspaper, 40\% read English newspaper and 20\% read both Hindi and English newspapers. A student is selected at random.
		\label{ncert/12/13/2/15}
\begin{enumerate}
\item Find the probability that she reads neither Hindi nor English newspapers.
\item If she reads Hindi newspaper, find the probability that she reads English newspaper.
\item If she reads English newspaper, find the probability that she reads Hindi newspaper.\\
\end{enumerate}
\item The probability of obtaining an even prime number on each die, when a pair of dice is rolled is 
\begin{enumerate}
    \item $0$ 
    
    \item $\frac{1}{3}$ 
    
    \item $\frac{1}{12}$ 
    
    \item $\frac{1}{36}$ 
\end{enumerate}
\solution
		%\input{ncert/12/13/2/17/defs.tex}
	\item A bag contains 4 red and 4 black balls, another bag contains 2 red and 6 black balls. One of the two bags is selected at random and a ball is drawn from the bag which is found to be red. Find the probability that the ball is drawn from the first bag.
\\
\solution
		%\input{ncert/12/13/3/2/main.tex}
  \item
  Cards with numbers 2 to 101 are placed in a box. A card is selected at random.Find the probability that the card has
\begin{enumerate}[label=(\roman*)]
	\item an even number 
	\item a square number
\end{enumerate}
\solution
%\input{exemplar/10/13/3/32/main.tex}
\item
The king, queen and jack of clubs are removed from a deck of 52 playing cards and then well shuffled. Now one card is drawn at random from the remaining cards.  Determine the probability that the card is
\begin{enumerate}[label=(\roman*)]
\item a club
\item 10 of hearts
\end{enumerate}
\solution
%\input{exemplar/10/13/3/29/main.tex}
\item A team of medical students doing their internship have to assist during surgeries
at a city hospital. The probabilities of surgeries rated as very complex, complex,
routine, simple or very simple are respectively, 0.15, 0.20, 0.31, 0.26, .08. Find
the probabilities that a particular surgery will be rated
\begin{enumerate}
	\item complex or very complex;
	\item neither very complex nor very simple;
	\item routine or complex
	\item routine or simple
\end{enumerate}
\solution
%\input{exemplar/11/16/3/8(1)/main.tex}
\item A card is selected from a pack of 52 cards.
\begin{enumerate}[label=(\alph*)]
    \item How many points are there in the sample space?
    \item Calculate the probability that the card is an ace of spades.
    \item Calculate the probability that the card is (i) an ace and (ii) black card.
\end{enumerate}
\solution
%\input{exemplar/11/16/3/4/main2.tex}
\item The probability that a non leap year selected at random will contain 53 sundays.
\\
\solution
%\input{exemplar/10/13/1/19/main.tex}
\item One of the four persons John, Rita, Aslam or Gurpreet will be promoted next
month. Consequently the sample space consists of four elementary outcomes
S = {John promoted, Rita promoted, Aslam promoted, Gurpreet promoted}
You are told that the chances of John’s promotion is same as that of Gurpreet,
Rita’s chances of promotion are twice as likely as Johns. Aslam’s chances are
four times that of John.
\begin{enumerate}
	\item Determine
	\begin{enumerate}
		\item P (John promoted)
		\item P (Rita promoted)
		\item P (Aslam promoted)
		\item P (Gurpreet promoted)
	\end{enumerate}
	\item If A = {John promoted or Gurpreet promoted}, find P (A).
\end{enumerate}
\solution
%\input{exemplar/11/16/3/10/main.tex}
\item A card is drawn from a deck of 52 cards. Find the probability of getting a king or a heart or a red card.\\
\solution
%\input{exemplar/11/16/3/15/main.tex}
\item The probability that a student will pass his examination is 0.73, the probability of
the student getting a compartment is 0.13, and the probability that the student will
either pass or get compartment is 0.96. State True or False.\\
\solution
%\input{exemplar/11/16/3/31/main.tex}
\item A card is selected from a pack of 52 cards\\
\begin{enumerate}[label=(\alph*)]
\item How many points are there in the sample space?
\item Calculate the probability that the cards is an ace of spades.
\item Calculate the probability that the card is (i) an ace (ii)black card.\\
\end{enumerate}
%\input{ncert/11/16/3/4_1/Prob_4.tex}
\item In a non-leap year, the probability of having 53 tuesdays or 53 wednesdays is\\
\solution
%\input{exemplar/11/16/3/18/main.tex}
\item There are 1000 sealed envelopes in a box, 10 of them contain a cash prize of
Rs 100 each, 100 of them contain a cash prize of Rs 50 each and 200 of them
contain a cash prize of Rs 10 each and rest do not contain any cash prize. If they
are well shuffled and an envelope is picked up out, what is the probability that it
contains no cash prize?\\
\solution
%\input{exemplar/10/13/3/34/main.tex}
\item 
A die is thrown and a card is selected at random from a deck of 52 playing cards. The probability of getting an even number on the die and a spade card.\\
\solution
%\input{exemplar/12/13/3/78/main.tex}
\item
If 4-digit numbers greater than 5,000 are randomly formed from the digits 0, 1, 3, 5, and 7, what is the probability of forming a number divisible by 5 when:
\begin{enumerate}
    \item The digits are repeated?
    \item The repetition of digits is not allowed?
\end{enumerate}
\solution
%\input{ncert/11/16/4/9/main.tex}
\item Consider the probability space $\brak{\Omega, \mathcal{G}, P}$ where $\Omega = [0,2]$ and $\mathcal{G} = \cbrak{\phi, \Omega, [0,1], (1,2]}$. Let $X$ and $Y$ be two functions on $\Omega$ defined as
\begin{align*}
    X(\omega) = 
    \begin{cases}
        1 & \text{if }\omega \in [0, 1]\\
        2 & \text{if }\omega \in (1, 2]
    \end{cases}
\end{align*}
and
\begin{align*}
    Y(\omega) = 
    \begin{cases}
        2 & \text{if }\omega \in [0, 1.5]\\
        3 & \text{if }\omega \in (1.5, 2].
    \end{cases}
\end{align*}
Then which one of the following statements is true?
\begin{enumerate}
    \item [(A)] $X$ is a random variable with respect to $\mathcal{G}$, but $Y$ is not a random variable with respect to $\mathcal{G}$.
    \item [(B)] $Y$ is a random variable with respect to $\mathcal{G}$, but $X$ is not a random variable with respect to $\mathcal{G}$.
    \item [(C)] Neither $X$ nor $Y$ is a random variable with respect to $\mathcal{G}$.
    \item [(D)] Both $X$ and $Y$ are random variables with respect to $\mathcal{G}$.
\end{enumerate} \hfill (GATE ST 2023)\\
\solution
%\input{gate/ST/2023/14/main.tex}
	\item  A die is loaded in such a way that each odd number is twice as likely to occur as
each even number. Find $P(G)$, where $G$ is the event that a number greater than
3 occurs on a single roll of the die.
\\
\solution
		%\input{exemplar/11/16/3/5/main.tex}
	\item All the jacks, queens and kings are removed from a deck of 52 playing cards. The remaining cards are well shuffled and then one card is drawn at random. Giving ace a value 1 similar value for other cards, find the probability that the card has a value 
		\begin{enumerate}
			\item 7
			\item greater than 7
			\item less than 7
		\end{enumerate}
		%\input{exemplar/10/13/3/30/main.tex}
  \item A Lot consists of 48 mobile phones of which 42 are good, 3 have only minor defects and 3 have major defects.Varnika will buy a phone if it is good but the trader will only buy a mobile if it has no major defects. One phone is selected at random from the lot. What is the probability that it is
\begin{enumerate}
	\item acceptable to Varnika?
            \item acceptable to the trader?
\end{enumerate}
\solution
	%\input{exemplar/10/13/3/40/main.tex}
 \item A student says that if you throw a die, it will show up 1 or not 1. Therefore, the probability of getting 1 and the probability of getting 'not 1' each is equal to $\frac{1}{2}$. Is this correct? Give reasons.\\
 \solution
        %\input{exemplar/10/13/2/9/main.tex}
   \item Four candidates A, B, C, D have ap-
plied for the assignment to coach a school cricket
team. If A is twice as likely to be selected as B, and
B and C are given about the same chance of being
selected, while C is twice as likely to be selected
as D, what are the probabilities that
\begin{enumerate}
\item C will be selected?
\item A will not be selected?
\end{enumerate}
	%\input{exemplar/11/16/3/9/main.tex}
 \item A bag contain 24 balls of which $x$ balls are red, $2x$ are white and $3x$ are blue. A ball is selected at random, What is the probability that it is
\begin{enumerate}[label=\alph*)]
\item not red ?
\item white ?
\end{enumerate}
%\input{exemplar/10/13/3/41/main.tex}
If the letters of the word ASSASSINATION are arranged at random. Find the Probability that
\begin{enumerate}[label=(\alph*)]
\item Four $S's$ come consecutively in the word
\item Two  $I's$ and two $N's$ come together
\item All $A's$ are not coming together
\item No two $A's$ are coming together
\end{enumerate}
%\input{exemplar/11/16/3/14/main.tex}
	\item One urn contains two black balls (labelled B1 and B2) and one white ball. A
	second urn contains one black ball and two white balls (labelled W1 and W2).
	Suppose the following experiment is performed. One of the two urns is chosen
	at random. Next a ball is randomly chosen from the urn. Then a second ball is
	chosen at random from the same urn without replacing the first ball.
	
	\begin{enumerate}
	\item What is the probability that two black balls are chosen?
	
	\item What is the probability that two balls of opposite colour are chosen?
	\end{enumerate}
	\solution
	%\input{exemplar/11/16/3/12/main1.tex}
\end{enumerate}

	\item A card is selected from a pack of 52 cards.
 \begin{enumerate}[label=(\alph*)] 
                 \item How many points are there in the sample space?
                 \item Calculate the probability that the card is an ace of spades.
                 \item Calculate the probability that the card is (i) an ace and (ii) black card.
 \end{enumerate}
\solution
		%\begin{table}[H]
	\centering
\begin{tabular}{|c|c|c|}
\hline
Random variable &Value &Definition\\ \hline
\multirow{3}{*}{X} &0 &Slips of Rs 1\\
&1 &Slips of Rs 5\\
&2 &Slips of Rs 13\\ \hline
\multirow{2}{*}{Y} &0 &Box A\\
&1 &Box B\\\hline
\end{tabular}
\caption{}
\label{tab:Distribution}
\end{table}
See \tabref{tab:Distribution}.
\begin{align}
p_{Y}\brak{k}= \begin{cases} 
      \frac{1}{3} & {k=0} \\
      \frac{2}{3 }& {k=1} 
   \end{cases}
   \\
p_{Y|X}\brak{0|0} = \frac{19}{25}\, 
p_{Y|X}\brak{0|1} = \frac{6}{25}\,
p_{Y|X}\brak{1|0} = \frac{45}{50}\,
p_{Y|X}\brak{1|2} = \frac{5}{50}
\end{align}
The desired probability is the probability that a slip drawn at random is marked other than Rs 1,
\begin{align}
&=1-p_X\brak{0}\\
&= p_X(1) + p_X(2)
\end{align}
Using Bayes theorem,
\begin{align}
&= p_Y\brak{0} \times \pr{Y=0 | X=1} + p_Y\brak{1} \times \pr{Y=1|X=2}\\
&=\frac{1}{3} \times \frac{6}{25} + \frac{2}{3} \times \frac{5}{50}\\
&=\frac{11}{75}
\end{align}

\newpage

%\tableofcontents

\bigskip

\renewcommand{\thefigure}{\theenumi}
\renewcommand{\thetable}{\theenumi}
%\renewcommand{\theequation}{\theenumi}

%\begin{abstract}
%%\boldmath
%In this letter, an algorithm for evaluating the exact analytical bit error rate  (BER)  for the piecewise linear (PL) combiner for  multiple relays is presented. Previous results were available only for upto three relays. The algorithm is unique in the sense that  the actual mathematical expressions, that are prohibitively large, need not be explicitly obtained. The diversity gain due to multiple relays is shown through plots of the analytical BER, well supported by simulations. 
%
%\end{abstract}
% IEEEtran.cls defaults to using nonbold math in the Abstract.
% This preserves the distinction between vectors and scalars. However,
% if the journal you are submitting to favors bold math in the abstract,
% then you can use LaTeX's standard command \boldmath at the very start
% of the abstract to achieve this. Many IEEE journals frown on math
% in the abstract anyway.

% Note that keywords are not normally used for peerreview papers.
%\begin{IEEEkeywords}
%Cooperative diversity, decode and forward, piecewise linear
%\end{IEEEkeywords}



% For peer review papers, you can put extra information on the cover
% page as needed:
% \ifCLASSOPTIONpeerreview
% \begin{center} \bfseries EDICS Category: 3-BBND \end{center}
% \fi
%
% For peerreview papers, this IEEEtran command inserts a page break and
% creates the second title. It will be ignored for other modes.
%\IEEEpeerreviewmaketitle




\item Four cards are drawn from a well-shuffled deck of 52 cards. What is the probability of obtaining 3 diamonds and one spade.
\\
\solution
		%\begin{enumerate}[label=\thesection.\arabic*,ref=\thesection.\theenumi]
	\item One card is drawn from a well-shuffled deck of 52 cards. Find the probability of getting
\begin{enumerate}
\item A king of red colour 
\item A face card 
\item A red face card
\item The jack of hearts
\item A spade
\item The queen of diamonds

\end{enumerate}
\solution
		%\input{ncert/10/15/1/14/main.tex}
	\item Five cards—the ten, jack, queen, king and ace of diamonds, are well-shuffled with their face downwards. One card is then picked up at random.
\begin{enumerate}
\item
What is the probability that the card is the queen? 
\item
If the queen is drawn and put aside, what is the probability that the second card picked up is (a) an ace? (b) a queen?\\
\end{enumerate}
\solution
		%\input{ncert/10/15/1/15/defs.tex}
	\item A bag contains $5$ red balls and some blue balls. If the probability of drawing a blue ball is double that if a red ball, determine the number of blue balls in the bag. 
		\\
\solution
		%\input{ncert/10/15/2/3/defs.tex}
	\item A card is selected from a pack of 52 cards.
 \begin{enumerate}[label=(\alph*)] 
                 \item How many points are there in the sample space?
                 \item Calculate the probability that the card is an ace of spades.
                 \item Calculate the probability that the card is (i) an ace and (ii) black card.
 \end{enumerate}
\solution
		%\input{ncert/11/16/3/4/main.tex}
\item Four cards are drawn from a well-shuffled deck of 52 cards. What is the probability of obtaining 3 diamonds and one spade.
\\
\solution
		%\input{ncert/11/16/4/2/defs.tex}
\item In a certain lottery 10,000 tickets are sold and ten equal prizes are awarded. What is the probability of not getting a prize if you buy (a) one ticket (b) two tickets (c) 10 tickets ?	
\\
\solution
		%\input{ncert/11/16/4/4/defs.tex}
		%
\item 
Out of 100 students, two sections of 40 and 60 are formed. If you and your friend are among the 100 students, what is the probability that
\begin{enumerate}
\item you both enter the same section?
\item you both enter the different sections?
\end{enumerate}
\solution
		%\input{ncert/11/16/4/5/defs.tex}
	\item 
The number lock of a suitcase has 4 wheels each labelled with ten digits i.e. from 0 to 9.The lock opens with a sequence of four digits with no repeats.What is the probability of a person getting the right sequence to open the suitcase.
\\
\solution
		%\input{ncert/11/16/4/10/defs.tex}
		%
\item 
Two cards are drawn at random and without replacement from a pack of 52 playing cards. Find the probability that both the cards are black.
\\
\solution
		%\input{ncert/12/13/2/2/defs.tex}
		\item A box of oranges is inspected by examining three randomly selected oranges drawn without replacement. If all the three oranges are good, the box is approved for sale, otherwise, it is rejected. Find the probability that a box containing 15 oranges out of which 12 are good and 3 are bad ones will be approved for sale.
		\label{ncert/12/13/2/3/defs.tex}
		\item Two balls are drawn at random with replacement from a box containing 10 black and 8 red balls. Find the probability that
		\label{ncert/12/13/2/12}
\begin{enumerate}
\item both balls are red.
\item first ball is black and second is red.
\item one of them is black and other is red.
\end{enumerate}

\item In a hostel, 60\% of the students read Hindi newspaper, 40\% read English newspaper and 20\% read both Hindi and English newspapers. A student is selected at random.
		\label{ncert/12/13/2/15}
\begin{enumerate}
\item Find the probability that she reads neither Hindi nor English newspapers.
\item If she reads Hindi newspaper, find the probability that she reads English newspaper.
\item If she reads English newspaper, find the probability that she reads Hindi newspaper.\\
\end{enumerate}
\item The probability of obtaining an even prime number on each die, when a pair of dice is rolled is 
\begin{enumerate}
    \item $0$ 
    
    \item $\frac{1}{3}$ 
    
    \item $\frac{1}{12}$ 
    
    \item $\frac{1}{36}$ 
\end{enumerate}
\solution
		%\input{ncert/12/13/2/17/defs.tex}
	\item A bag contains 4 red and 4 black balls, another bag contains 2 red and 6 black balls. One of the two bags is selected at random and a ball is drawn from the bag which is found to be red. Find the probability that the ball is drawn from the first bag.
\\
\solution
		%\input{ncert/12/13/3/2/main.tex}
  \item
  Cards with numbers 2 to 101 are placed in a box. A card is selected at random.Find the probability that the card has
\begin{enumerate}[label=(\roman*)]
	\item an even number 
	\item a square number
\end{enumerate}
\solution
%\input{exemplar/10/13/3/32/main.tex}
\item
The king, queen and jack of clubs are removed from a deck of 52 playing cards and then well shuffled. Now one card is drawn at random from the remaining cards.  Determine the probability that the card is
\begin{enumerate}[label=(\roman*)]
\item a club
\item 10 of hearts
\end{enumerate}
\solution
%\input{exemplar/10/13/3/29/main.tex}
\item A team of medical students doing their internship have to assist during surgeries
at a city hospital. The probabilities of surgeries rated as very complex, complex,
routine, simple or very simple are respectively, 0.15, 0.20, 0.31, 0.26, .08. Find
the probabilities that a particular surgery will be rated
\begin{enumerate}
	\item complex or very complex;
	\item neither very complex nor very simple;
	\item routine or complex
	\item routine or simple
\end{enumerate}
\solution
%\input{exemplar/11/16/3/8(1)/main.tex}
\item A card is selected from a pack of 52 cards.
\begin{enumerate}[label=(\alph*)]
    \item How many points are there in the sample space?
    \item Calculate the probability that the card is an ace of spades.
    \item Calculate the probability that the card is (i) an ace and (ii) black card.
\end{enumerate}
\solution
%\input{exemplar/11/16/3/4/main2.tex}
\item The probability that a non leap year selected at random will contain 53 sundays.
\\
\solution
%\input{exemplar/10/13/1/19/main.tex}
\item One of the four persons John, Rita, Aslam or Gurpreet will be promoted next
month. Consequently the sample space consists of four elementary outcomes
S = {John promoted, Rita promoted, Aslam promoted, Gurpreet promoted}
You are told that the chances of John’s promotion is same as that of Gurpreet,
Rita’s chances of promotion are twice as likely as Johns. Aslam’s chances are
four times that of John.
\begin{enumerate}
	\item Determine
	\begin{enumerate}
		\item P (John promoted)
		\item P (Rita promoted)
		\item P (Aslam promoted)
		\item P (Gurpreet promoted)
	\end{enumerate}
	\item If A = {John promoted or Gurpreet promoted}, find P (A).
\end{enumerate}
\solution
%\input{exemplar/11/16/3/10/main.tex}
\item A card is drawn from a deck of 52 cards. Find the probability of getting a king or a heart or a red card.\\
\solution
%\input{exemplar/11/16/3/15/main.tex}
\item The probability that a student will pass his examination is 0.73, the probability of
the student getting a compartment is 0.13, and the probability that the student will
either pass or get compartment is 0.96. State True or False.\\
\solution
%\input{exemplar/11/16/3/31/main.tex}
\item A card is selected from a pack of 52 cards\\
\begin{enumerate}[label=(\alph*)]
\item How many points are there in the sample space?
\item Calculate the probability that the cards is an ace of spades.
\item Calculate the probability that the card is (i) an ace (ii)black card.\\
\end{enumerate}
%\input{ncert/11/16/3/4_1/Prob_4.tex}
\item In a non-leap year, the probability of having 53 tuesdays or 53 wednesdays is\\
\solution
%\input{exemplar/11/16/3/18/main.tex}
\item There are 1000 sealed envelopes in a box, 10 of them contain a cash prize of
Rs 100 each, 100 of them contain a cash prize of Rs 50 each and 200 of them
contain a cash prize of Rs 10 each and rest do not contain any cash prize. If they
are well shuffled and an envelope is picked up out, what is the probability that it
contains no cash prize?\\
\solution
%\input{exemplar/10/13/3/34/main.tex}
\item 
A die is thrown and a card is selected at random from a deck of 52 playing cards. The probability of getting an even number on the die and a spade card.\\
\solution
%\input{exemplar/12/13/3/78/main.tex}
\item
If 4-digit numbers greater than 5,000 are randomly formed from the digits 0, 1, 3, 5, and 7, what is the probability of forming a number divisible by 5 when:
\begin{enumerate}
    \item The digits are repeated?
    \item The repetition of digits is not allowed?
\end{enumerate}
\solution
%\input{ncert/11/16/4/9/main.tex}
\item Consider the probability space $\brak{\Omega, \mathcal{G}, P}$ where $\Omega = [0,2]$ and $\mathcal{G} = \cbrak{\phi, \Omega, [0,1], (1,2]}$. Let $X$ and $Y$ be two functions on $\Omega$ defined as
\begin{align*}
    X(\omega) = 
    \begin{cases}
        1 & \text{if }\omega \in [0, 1]\\
        2 & \text{if }\omega \in (1, 2]
    \end{cases}
\end{align*}
and
\begin{align*}
    Y(\omega) = 
    \begin{cases}
        2 & \text{if }\omega \in [0, 1.5]\\
        3 & \text{if }\omega \in (1.5, 2].
    \end{cases}
\end{align*}
Then which one of the following statements is true?
\begin{enumerate}
    \item [(A)] $X$ is a random variable with respect to $\mathcal{G}$, but $Y$ is not a random variable with respect to $\mathcal{G}$.
    \item [(B)] $Y$ is a random variable with respect to $\mathcal{G}$, but $X$ is not a random variable with respect to $\mathcal{G}$.
    \item [(C)] Neither $X$ nor $Y$ is a random variable with respect to $\mathcal{G}$.
    \item [(D)] Both $X$ and $Y$ are random variables with respect to $\mathcal{G}$.
\end{enumerate} \hfill (GATE ST 2023)\\
\solution
%\input{gate/ST/2023/14/main.tex}
	\item  A die is loaded in such a way that each odd number is twice as likely to occur as
each even number. Find $P(G)$, where $G$ is the event that a number greater than
3 occurs on a single roll of the die.
\\
\solution
		%\input{exemplar/11/16/3/5/main.tex}
	\item All the jacks, queens and kings are removed from a deck of 52 playing cards. The remaining cards are well shuffled and then one card is drawn at random. Giving ace a value 1 similar value for other cards, find the probability that the card has a value 
		\begin{enumerate}
			\item 7
			\item greater than 7
			\item less than 7
		\end{enumerate}
		%\input{exemplar/10/13/3/30/main.tex}
  \item A Lot consists of 48 mobile phones of which 42 are good, 3 have only minor defects and 3 have major defects.Varnika will buy a phone if it is good but the trader will only buy a mobile if it has no major defects. One phone is selected at random from the lot. What is the probability that it is
\begin{enumerate}
	\item acceptable to Varnika?
            \item acceptable to the trader?
\end{enumerate}
\solution
	%\input{exemplar/10/13/3/40/main.tex}
 \item A student says that if you throw a die, it will show up 1 or not 1. Therefore, the probability of getting 1 and the probability of getting 'not 1' each is equal to $\frac{1}{2}$. Is this correct? Give reasons.\\
 \solution
        %\input{exemplar/10/13/2/9/main.tex}
   \item Four candidates A, B, C, D have ap-
plied for the assignment to coach a school cricket
team. If A is twice as likely to be selected as B, and
B and C are given about the same chance of being
selected, while C is twice as likely to be selected
as D, what are the probabilities that
\begin{enumerate}
\item C will be selected?
\item A will not be selected?
\end{enumerate}
	%\input{exemplar/11/16/3/9/main.tex}
 \item A bag contain 24 balls of which $x$ balls are red, $2x$ are white and $3x$ are blue. A ball is selected at random, What is the probability that it is
\begin{enumerate}[label=\alph*)]
\item not red ?
\item white ?
\end{enumerate}
%\input{exemplar/10/13/3/41/main.tex}
If the letters of the word ASSASSINATION are arranged at random. Find the Probability that
\begin{enumerate}[label=(\alph*)]
\item Four $S's$ come consecutively in the word
\item Two  $I's$ and two $N's$ come together
\item All $A's$ are not coming together
\item No two $A's$ are coming together
\end{enumerate}
%\input{exemplar/11/16/3/14/main.tex}
	\item One urn contains two black balls (labelled B1 and B2) and one white ball. A
	second urn contains one black ball and two white balls (labelled W1 and W2).
	Suppose the following experiment is performed. One of the two urns is chosen
	at random. Next a ball is randomly chosen from the urn. Then a second ball is
	chosen at random from the same urn without replacing the first ball.
	
	\begin{enumerate}
	\item What is the probability that two black balls are chosen?
	
	\item What is the probability that two balls of opposite colour are chosen?
	\end{enumerate}
	\solution
	%\input{exemplar/11/16/3/12/main1.tex}
\end{enumerate}

\item In a certain lottery 10,000 tickets are sold and ten equal prizes are awarded. What is the probability of not getting a prize if you buy (a) one ticket (b) two tickets (c) 10 tickets ?	
\\
\solution
		%\begin{enumerate}[label=\thesection.\arabic*,ref=\thesection.\theenumi]
	\item One card is drawn from a well-shuffled deck of 52 cards. Find the probability of getting
\begin{enumerate}
\item A king of red colour 
\item A face card 
\item A red face card
\item The jack of hearts
\item A spade
\item The queen of diamonds

\end{enumerate}
\solution
		%\input{ncert/10/15/1/14/main.tex}
	\item Five cards—the ten, jack, queen, king and ace of diamonds, are well-shuffled with their face downwards. One card is then picked up at random.
\begin{enumerate}
\item
What is the probability that the card is the queen? 
\item
If the queen is drawn and put aside, what is the probability that the second card picked up is (a) an ace? (b) a queen?\\
\end{enumerate}
\solution
		%\input{ncert/10/15/1/15/defs.tex}
	\item A bag contains $5$ red balls and some blue balls. If the probability of drawing a blue ball is double that if a red ball, determine the number of blue balls in the bag. 
		\\
\solution
		%\input{ncert/10/15/2/3/defs.tex}
	\item A card is selected from a pack of 52 cards.
 \begin{enumerate}[label=(\alph*)] 
                 \item How many points are there in the sample space?
                 \item Calculate the probability that the card is an ace of spades.
                 \item Calculate the probability that the card is (i) an ace and (ii) black card.
 \end{enumerate}
\solution
		%\input{ncert/11/16/3/4/main.tex}
\item Four cards are drawn from a well-shuffled deck of 52 cards. What is the probability of obtaining 3 diamonds and one spade.
\\
\solution
		%\input{ncert/11/16/4/2/defs.tex}
\item In a certain lottery 10,000 tickets are sold and ten equal prizes are awarded. What is the probability of not getting a prize if you buy (a) one ticket (b) two tickets (c) 10 tickets ?	
\\
\solution
		%\input{ncert/11/16/4/4/defs.tex}
		%
\item 
Out of 100 students, two sections of 40 and 60 are formed. If you and your friend are among the 100 students, what is the probability that
\begin{enumerate}
\item you both enter the same section?
\item you both enter the different sections?
\end{enumerate}
\solution
		%\input{ncert/11/16/4/5/defs.tex}
	\item 
The number lock of a suitcase has 4 wheels each labelled with ten digits i.e. from 0 to 9.The lock opens with a sequence of four digits with no repeats.What is the probability of a person getting the right sequence to open the suitcase.
\\
\solution
		%\input{ncert/11/16/4/10/defs.tex}
		%
\item 
Two cards are drawn at random and without replacement from a pack of 52 playing cards. Find the probability that both the cards are black.
\\
\solution
		%\input{ncert/12/13/2/2/defs.tex}
		\item A box of oranges is inspected by examining three randomly selected oranges drawn without replacement. If all the three oranges are good, the box is approved for sale, otherwise, it is rejected. Find the probability that a box containing 15 oranges out of which 12 are good and 3 are bad ones will be approved for sale.
		\label{ncert/12/13/2/3/defs.tex}
		\item Two balls are drawn at random with replacement from a box containing 10 black and 8 red balls. Find the probability that
		\label{ncert/12/13/2/12}
\begin{enumerate}
\item both balls are red.
\item first ball is black and second is red.
\item one of them is black and other is red.
\end{enumerate}

\item In a hostel, 60\% of the students read Hindi newspaper, 40\% read English newspaper and 20\% read both Hindi and English newspapers. A student is selected at random.
		\label{ncert/12/13/2/15}
\begin{enumerate}
\item Find the probability that she reads neither Hindi nor English newspapers.
\item If she reads Hindi newspaper, find the probability that she reads English newspaper.
\item If she reads English newspaper, find the probability that she reads Hindi newspaper.\\
\end{enumerate}
\item The probability of obtaining an even prime number on each die, when a pair of dice is rolled is 
\begin{enumerate}
    \item $0$ 
    
    \item $\frac{1}{3}$ 
    
    \item $\frac{1}{12}$ 
    
    \item $\frac{1}{36}$ 
\end{enumerate}
\solution
		%\input{ncert/12/13/2/17/defs.tex}
	\item A bag contains 4 red and 4 black balls, another bag contains 2 red and 6 black balls. One of the two bags is selected at random and a ball is drawn from the bag which is found to be red. Find the probability that the ball is drawn from the first bag.
\\
\solution
		%\input{ncert/12/13/3/2/main.tex}
  \item
  Cards with numbers 2 to 101 are placed in a box. A card is selected at random.Find the probability that the card has
\begin{enumerate}[label=(\roman*)]
	\item an even number 
	\item a square number
\end{enumerate}
\solution
%\input{exemplar/10/13/3/32/main.tex}
\item
The king, queen and jack of clubs are removed from a deck of 52 playing cards and then well shuffled. Now one card is drawn at random from the remaining cards.  Determine the probability that the card is
\begin{enumerate}[label=(\roman*)]
\item a club
\item 10 of hearts
\end{enumerate}
\solution
%\input{exemplar/10/13/3/29/main.tex}
\item A team of medical students doing their internship have to assist during surgeries
at a city hospital. The probabilities of surgeries rated as very complex, complex,
routine, simple or very simple are respectively, 0.15, 0.20, 0.31, 0.26, .08. Find
the probabilities that a particular surgery will be rated
\begin{enumerate}
	\item complex or very complex;
	\item neither very complex nor very simple;
	\item routine or complex
	\item routine or simple
\end{enumerate}
\solution
%\input{exemplar/11/16/3/8(1)/main.tex}
\item A card is selected from a pack of 52 cards.
\begin{enumerate}[label=(\alph*)]
    \item How many points are there in the sample space?
    \item Calculate the probability that the card is an ace of spades.
    \item Calculate the probability that the card is (i) an ace and (ii) black card.
\end{enumerate}
\solution
%\input{exemplar/11/16/3/4/main2.tex}
\item The probability that a non leap year selected at random will contain 53 sundays.
\\
\solution
%\input{exemplar/10/13/1/19/main.tex}
\item One of the four persons John, Rita, Aslam or Gurpreet will be promoted next
month. Consequently the sample space consists of four elementary outcomes
S = {John promoted, Rita promoted, Aslam promoted, Gurpreet promoted}
You are told that the chances of John’s promotion is same as that of Gurpreet,
Rita’s chances of promotion are twice as likely as Johns. Aslam’s chances are
four times that of John.
\begin{enumerate}
	\item Determine
	\begin{enumerate}
		\item P (John promoted)
		\item P (Rita promoted)
		\item P (Aslam promoted)
		\item P (Gurpreet promoted)
	\end{enumerate}
	\item If A = {John promoted or Gurpreet promoted}, find P (A).
\end{enumerate}
\solution
%\input{exemplar/11/16/3/10/main.tex}
\item A card is drawn from a deck of 52 cards. Find the probability of getting a king or a heart or a red card.\\
\solution
%\input{exemplar/11/16/3/15/main.tex}
\item The probability that a student will pass his examination is 0.73, the probability of
the student getting a compartment is 0.13, and the probability that the student will
either pass or get compartment is 0.96. State True or False.\\
\solution
%\input{exemplar/11/16/3/31/main.tex}
\item A card is selected from a pack of 52 cards\\
\begin{enumerate}[label=(\alph*)]
\item How many points are there in the sample space?
\item Calculate the probability that the cards is an ace of spades.
\item Calculate the probability that the card is (i) an ace (ii)black card.\\
\end{enumerate}
%\input{ncert/11/16/3/4_1/Prob_4.tex}
\item In a non-leap year, the probability of having 53 tuesdays or 53 wednesdays is\\
\solution
%\input{exemplar/11/16/3/18/main.tex}
\item There are 1000 sealed envelopes in a box, 10 of them contain a cash prize of
Rs 100 each, 100 of them contain a cash prize of Rs 50 each and 200 of them
contain a cash prize of Rs 10 each and rest do not contain any cash prize. If they
are well shuffled and an envelope is picked up out, what is the probability that it
contains no cash prize?\\
\solution
%\input{exemplar/10/13/3/34/main.tex}
\item 
A die is thrown and a card is selected at random from a deck of 52 playing cards. The probability of getting an even number on the die and a spade card.\\
\solution
%\input{exemplar/12/13/3/78/main.tex}
\item
If 4-digit numbers greater than 5,000 are randomly formed from the digits 0, 1, 3, 5, and 7, what is the probability of forming a number divisible by 5 when:
\begin{enumerate}
    \item The digits are repeated?
    \item The repetition of digits is not allowed?
\end{enumerate}
\solution
%\input{ncert/11/16/4/9/main.tex}
\item Consider the probability space $\brak{\Omega, \mathcal{G}, P}$ where $\Omega = [0,2]$ and $\mathcal{G} = \cbrak{\phi, \Omega, [0,1], (1,2]}$. Let $X$ and $Y$ be two functions on $\Omega$ defined as
\begin{align*}
    X(\omega) = 
    \begin{cases}
        1 & \text{if }\omega \in [0, 1]\\
        2 & \text{if }\omega \in (1, 2]
    \end{cases}
\end{align*}
and
\begin{align*}
    Y(\omega) = 
    \begin{cases}
        2 & \text{if }\omega \in [0, 1.5]\\
        3 & \text{if }\omega \in (1.5, 2].
    \end{cases}
\end{align*}
Then which one of the following statements is true?
\begin{enumerate}
    \item [(A)] $X$ is a random variable with respect to $\mathcal{G}$, but $Y$ is not a random variable with respect to $\mathcal{G}$.
    \item [(B)] $Y$ is a random variable with respect to $\mathcal{G}$, but $X$ is not a random variable with respect to $\mathcal{G}$.
    \item [(C)] Neither $X$ nor $Y$ is a random variable with respect to $\mathcal{G}$.
    \item [(D)] Both $X$ and $Y$ are random variables with respect to $\mathcal{G}$.
\end{enumerate} \hfill (GATE ST 2023)\\
\solution
%\input{gate/ST/2023/14/main.tex}
	\item  A die is loaded in such a way that each odd number is twice as likely to occur as
each even number. Find $P(G)$, where $G$ is the event that a number greater than
3 occurs on a single roll of the die.
\\
\solution
		%\input{exemplar/11/16/3/5/main.tex}
	\item All the jacks, queens and kings are removed from a deck of 52 playing cards. The remaining cards are well shuffled and then one card is drawn at random. Giving ace a value 1 similar value for other cards, find the probability that the card has a value 
		\begin{enumerate}
			\item 7
			\item greater than 7
			\item less than 7
		\end{enumerate}
		%\input{exemplar/10/13/3/30/main.tex}
  \item A Lot consists of 48 mobile phones of which 42 are good, 3 have only minor defects and 3 have major defects.Varnika will buy a phone if it is good but the trader will only buy a mobile if it has no major defects. One phone is selected at random from the lot. What is the probability that it is
\begin{enumerate}
	\item acceptable to Varnika?
            \item acceptable to the trader?
\end{enumerate}
\solution
	%\input{exemplar/10/13/3/40/main.tex}
 \item A student says that if you throw a die, it will show up 1 or not 1. Therefore, the probability of getting 1 and the probability of getting 'not 1' each is equal to $\frac{1}{2}$. Is this correct? Give reasons.\\
 \solution
        %\input{exemplar/10/13/2/9/main.tex}
   \item Four candidates A, B, C, D have ap-
plied for the assignment to coach a school cricket
team. If A is twice as likely to be selected as B, and
B and C are given about the same chance of being
selected, while C is twice as likely to be selected
as D, what are the probabilities that
\begin{enumerate}
\item C will be selected?
\item A will not be selected?
\end{enumerate}
	%\input{exemplar/11/16/3/9/main.tex}
 \item A bag contain 24 balls of which $x$ balls are red, $2x$ are white and $3x$ are blue. A ball is selected at random, What is the probability that it is
\begin{enumerate}[label=\alph*)]
\item not red ?
\item white ?
\end{enumerate}
%\input{exemplar/10/13/3/41/main.tex}
If the letters of the word ASSASSINATION are arranged at random. Find the Probability that
\begin{enumerate}[label=(\alph*)]
\item Four $S's$ come consecutively in the word
\item Two  $I's$ and two $N's$ come together
\item All $A's$ are not coming together
\item No two $A's$ are coming together
\end{enumerate}
%\input{exemplar/11/16/3/14/main.tex}
	\item One urn contains two black balls (labelled B1 and B2) and one white ball. A
	second urn contains one black ball and two white balls (labelled W1 and W2).
	Suppose the following experiment is performed. One of the two urns is chosen
	at random. Next a ball is randomly chosen from the urn. Then a second ball is
	chosen at random from the same urn without replacing the first ball.
	
	\begin{enumerate}
	\item What is the probability that two black balls are chosen?
	
	\item What is the probability that two balls of opposite colour are chosen?
	\end{enumerate}
	\solution
	%\input{exemplar/11/16/3/12/main1.tex}
\end{enumerate}

		%
\item 
Out of 100 students, two sections of 40 and 60 are formed. If you and your friend are among the 100 students, what is the probability that
\begin{enumerate}
\item you both enter the same section?
\item you both enter the different sections?
\end{enumerate}
\solution
		%\begin{enumerate}[label=\thesection.\arabic*,ref=\thesection.\theenumi]
	\item One card is drawn from a well-shuffled deck of 52 cards. Find the probability of getting
\begin{enumerate}
\item A king of red colour 
\item A face card 
\item A red face card
\item The jack of hearts
\item A spade
\item The queen of diamonds

\end{enumerate}
\solution
		%\input{ncert/10/15/1/14/main.tex}
	\item Five cards—the ten, jack, queen, king and ace of diamonds, are well-shuffled with their face downwards. One card is then picked up at random.
\begin{enumerate}
\item
What is the probability that the card is the queen? 
\item
If the queen is drawn and put aside, what is the probability that the second card picked up is (a) an ace? (b) a queen?\\
\end{enumerate}
\solution
		%\input{ncert/10/15/1/15/defs.tex}
	\item A bag contains $5$ red balls and some blue balls. If the probability of drawing a blue ball is double that if a red ball, determine the number of blue balls in the bag. 
		\\
\solution
		%\input{ncert/10/15/2/3/defs.tex}
	\item A card is selected from a pack of 52 cards.
 \begin{enumerate}[label=(\alph*)] 
                 \item How many points are there in the sample space?
                 \item Calculate the probability that the card is an ace of spades.
                 \item Calculate the probability that the card is (i) an ace and (ii) black card.
 \end{enumerate}
\solution
		%\input{ncert/11/16/3/4/main.tex}
\item Four cards are drawn from a well-shuffled deck of 52 cards. What is the probability of obtaining 3 diamonds and one spade.
\\
\solution
		%\input{ncert/11/16/4/2/defs.tex}
\item In a certain lottery 10,000 tickets are sold and ten equal prizes are awarded. What is the probability of not getting a prize if you buy (a) one ticket (b) two tickets (c) 10 tickets ?	
\\
\solution
		%\input{ncert/11/16/4/4/defs.tex}
		%
\item 
Out of 100 students, two sections of 40 and 60 are formed. If you and your friend are among the 100 students, what is the probability that
\begin{enumerate}
\item you both enter the same section?
\item you both enter the different sections?
\end{enumerate}
\solution
		%\input{ncert/11/16/4/5/defs.tex}
	\item 
The number lock of a suitcase has 4 wheels each labelled with ten digits i.e. from 0 to 9.The lock opens with a sequence of four digits with no repeats.What is the probability of a person getting the right sequence to open the suitcase.
\\
\solution
		%\input{ncert/11/16/4/10/defs.tex}
		%
\item 
Two cards are drawn at random and without replacement from a pack of 52 playing cards. Find the probability that both the cards are black.
\\
\solution
		%\input{ncert/12/13/2/2/defs.tex}
		\item A box of oranges is inspected by examining three randomly selected oranges drawn without replacement. If all the three oranges are good, the box is approved for sale, otherwise, it is rejected. Find the probability that a box containing 15 oranges out of which 12 are good and 3 are bad ones will be approved for sale.
		\label{ncert/12/13/2/3/defs.tex}
		\item Two balls are drawn at random with replacement from a box containing 10 black and 8 red balls. Find the probability that
		\label{ncert/12/13/2/12}
\begin{enumerate}
\item both balls are red.
\item first ball is black and second is red.
\item one of them is black and other is red.
\end{enumerate}

\item In a hostel, 60\% of the students read Hindi newspaper, 40\% read English newspaper and 20\% read both Hindi and English newspapers. A student is selected at random.
		\label{ncert/12/13/2/15}
\begin{enumerate}
\item Find the probability that she reads neither Hindi nor English newspapers.
\item If she reads Hindi newspaper, find the probability that she reads English newspaper.
\item If she reads English newspaper, find the probability that she reads Hindi newspaper.\\
\end{enumerate}
\item The probability of obtaining an even prime number on each die, when a pair of dice is rolled is 
\begin{enumerate}
    \item $0$ 
    
    \item $\frac{1}{3}$ 
    
    \item $\frac{1}{12}$ 
    
    \item $\frac{1}{36}$ 
\end{enumerate}
\solution
		%\input{ncert/12/13/2/17/defs.tex}
	\item A bag contains 4 red and 4 black balls, another bag contains 2 red and 6 black balls. One of the two bags is selected at random and a ball is drawn from the bag which is found to be red. Find the probability that the ball is drawn from the first bag.
\\
\solution
		%\input{ncert/12/13/3/2/main.tex}
  \item
  Cards with numbers 2 to 101 are placed in a box. A card is selected at random.Find the probability that the card has
\begin{enumerate}[label=(\roman*)]
	\item an even number 
	\item a square number
\end{enumerate}
\solution
%\input{exemplar/10/13/3/32/main.tex}
\item
The king, queen and jack of clubs are removed from a deck of 52 playing cards and then well shuffled. Now one card is drawn at random from the remaining cards.  Determine the probability that the card is
\begin{enumerate}[label=(\roman*)]
\item a club
\item 10 of hearts
\end{enumerate}
\solution
%\input{exemplar/10/13/3/29/main.tex}
\item A team of medical students doing their internship have to assist during surgeries
at a city hospital. The probabilities of surgeries rated as very complex, complex,
routine, simple or very simple are respectively, 0.15, 0.20, 0.31, 0.26, .08. Find
the probabilities that a particular surgery will be rated
\begin{enumerate}
	\item complex or very complex;
	\item neither very complex nor very simple;
	\item routine or complex
	\item routine or simple
\end{enumerate}
\solution
%\input{exemplar/11/16/3/8(1)/main.tex}
\item A card is selected from a pack of 52 cards.
\begin{enumerate}[label=(\alph*)]
    \item How many points are there in the sample space?
    \item Calculate the probability that the card is an ace of spades.
    \item Calculate the probability that the card is (i) an ace and (ii) black card.
\end{enumerate}
\solution
%\input{exemplar/11/16/3/4/main2.tex}
\item The probability that a non leap year selected at random will contain 53 sundays.
\\
\solution
%\input{exemplar/10/13/1/19/main.tex}
\item One of the four persons John, Rita, Aslam or Gurpreet will be promoted next
month. Consequently the sample space consists of four elementary outcomes
S = {John promoted, Rita promoted, Aslam promoted, Gurpreet promoted}
You are told that the chances of John’s promotion is same as that of Gurpreet,
Rita’s chances of promotion are twice as likely as Johns. Aslam’s chances are
four times that of John.
\begin{enumerate}
	\item Determine
	\begin{enumerate}
		\item P (John promoted)
		\item P (Rita promoted)
		\item P (Aslam promoted)
		\item P (Gurpreet promoted)
	\end{enumerate}
	\item If A = {John promoted or Gurpreet promoted}, find P (A).
\end{enumerate}
\solution
%\input{exemplar/11/16/3/10/main.tex}
\item A card is drawn from a deck of 52 cards. Find the probability of getting a king or a heart or a red card.\\
\solution
%\input{exemplar/11/16/3/15/main.tex}
\item The probability that a student will pass his examination is 0.73, the probability of
the student getting a compartment is 0.13, and the probability that the student will
either pass or get compartment is 0.96. State True or False.\\
\solution
%\input{exemplar/11/16/3/31/main.tex}
\item A card is selected from a pack of 52 cards\\
\begin{enumerate}[label=(\alph*)]
\item How many points are there in the sample space?
\item Calculate the probability that the cards is an ace of spades.
\item Calculate the probability that the card is (i) an ace (ii)black card.\\
\end{enumerate}
%\input{ncert/11/16/3/4_1/Prob_4.tex}
\item In a non-leap year, the probability of having 53 tuesdays or 53 wednesdays is\\
\solution
%\input{exemplar/11/16/3/18/main.tex}
\item There are 1000 sealed envelopes in a box, 10 of them contain a cash prize of
Rs 100 each, 100 of them contain a cash prize of Rs 50 each and 200 of them
contain a cash prize of Rs 10 each and rest do not contain any cash prize. If they
are well shuffled and an envelope is picked up out, what is the probability that it
contains no cash prize?\\
\solution
%\input{exemplar/10/13/3/34/main.tex}
\item 
A die is thrown and a card is selected at random from a deck of 52 playing cards. The probability of getting an even number on the die and a spade card.\\
\solution
%\input{exemplar/12/13/3/78/main.tex}
\item
If 4-digit numbers greater than 5,000 are randomly formed from the digits 0, 1, 3, 5, and 7, what is the probability of forming a number divisible by 5 when:
\begin{enumerate}
    \item The digits are repeated?
    \item The repetition of digits is not allowed?
\end{enumerate}
\solution
%\input{ncert/11/16/4/9/main.tex}
\item Consider the probability space $\brak{\Omega, \mathcal{G}, P}$ where $\Omega = [0,2]$ and $\mathcal{G} = \cbrak{\phi, \Omega, [0,1], (1,2]}$. Let $X$ and $Y$ be two functions on $\Omega$ defined as
\begin{align*}
    X(\omega) = 
    \begin{cases}
        1 & \text{if }\omega \in [0, 1]\\
        2 & \text{if }\omega \in (1, 2]
    \end{cases}
\end{align*}
and
\begin{align*}
    Y(\omega) = 
    \begin{cases}
        2 & \text{if }\omega \in [0, 1.5]\\
        3 & \text{if }\omega \in (1.5, 2].
    \end{cases}
\end{align*}
Then which one of the following statements is true?
\begin{enumerate}
    \item [(A)] $X$ is a random variable with respect to $\mathcal{G}$, but $Y$ is not a random variable with respect to $\mathcal{G}$.
    \item [(B)] $Y$ is a random variable with respect to $\mathcal{G}$, but $X$ is not a random variable with respect to $\mathcal{G}$.
    \item [(C)] Neither $X$ nor $Y$ is a random variable with respect to $\mathcal{G}$.
    \item [(D)] Both $X$ and $Y$ are random variables with respect to $\mathcal{G}$.
\end{enumerate} \hfill (GATE ST 2023)\\
\solution
%\input{gate/ST/2023/14/main.tex}
	\item  A die is loaded in such a way that each odd number is twice as likely to occur as
each even number. Find $P(G)$, where $G$ is the event that a number greater than
3 occurs on a single roll of the die.
\\
\solution
		%\input{exemplar/11/16/3/5/main.tex}
	\item All the jacks, queens and kings are removed from a deck of 52 playing cards. The remaining cards are well shuffled and then one card is drawn at random. Giving ace a value 1 similar value for other cards, find the probability that the card has a value 
		\begin{enumerate}
			\item 7
			\item greater than 7
			\item less than 7
		\end{enumerate}
		%\input{exemplar/10/13/3/30/main.tex}
  \item A Lot consists of 48 mobile phones of which 42 are good, 3 have only minor defects and 3 have major defects.Varnika will buy a phone if it is good but the trader will only buy a mobile if it has no major defects. One phone is selected at random from the lot. What is the probability that it is
\begin{enumerate}
	\item acceptable to Varnika?
            \item acceptable to the trader?
\end{enumerate}
\solution
	%\input{exemplar/10/13/3/40/main.tex}
 \item A student says that if you throw a die, it will show up 1 or not 1. Therefore, the probability of getting 1 and the probability of getting 'not 1' each is equal to $\frac{1}{2}$. Is this correct? Give reasons.\\
 \solution
        %\input{exemplar/10/13/2/9/main.tex}
   \item Four candidates A, B, C, D have ap-
plied for the assignment to coach a school cricket
team. If A is twice as likely to be selected as B, and
B and C are given about the same chance of being
selected, while C is twice as likely to be selected
as D, what are the probabilities that
\begin{enumerate}
\item C will be selected?
\item A will not be selected?
\end{enumerate}
	%\input{exemplar/11/16/3/9/main.tex}
 \item A bag contain 24 balls of which $x$ balls are red, $2x$ are white and $3x$ are blue. A ball is selected at random, What is the probability that it is
\begin{enumerate}[label=\alph*)]
\item not red ?
\item white ?
\end{enumerate}
%\input{exemplar/10/13/3/41/main.tex}
If the letters of the word ASSASSINATION are arranged at random. Find the Probability that
\begin{enumerate}[label=(\alph*)]
\item Four $S's$ come consecutively in the word
\item Two  $I's$ and two $N's$ come together
\item All $A's$ are not coming together
\item No two $A's$ are coming together
\end{enumerate}
%\input{exemplar/11/16/3/14/main.tex}
	\item One urn contains two black balls (labelled B1 and B2) and one white ball. A
	second urn contains one black ball and two white balls (labelled W1 and W2).
	Suppose the following experiment is performed. One of the two urns is chosen
	at random. Next a ball is randomly chosen from the urn. Then a second ball is
	chosen at random from the same urn without replacing the first ball.
	
	\begin{enumerate}
	\item What is the probability that two black balls are chosen?
	
	\item What is the probability that two balls of opposite colour are chosen?
	\end{enumerate}
	\solution
	%\input{exemplar/11/16/3/12/main1.tex}
\end{enumerate}

	\item 
The number lock of a suitcase has 4 wheels each labelled with ten digits i.e. from 0 to 9.The lock opens with a sequence of four digits with no repeats.What is the probability of a person getting the right sequence to open the suitcase.
\\
\solution
		%\begin{enumerate}[label=\thesection.\arabic*,ref=\thesection.\theenumi]
	\item One card is drawn from a well-shuffled deck of 52 cards. Find the probability of getting
\begin{enumerate}
\item A king of red colour 
\item A face card 
\item A red face card
\item The jack of hearts
\item A spade
\item The queen of diamonds

\end{enumerate}
\solution
		%\input{ncert/10/15/1/14/main.tex}
	\item Five cards—the ten, jack, queen, king and ace of diamonds, are well-shuffled with their face downwards. One card is then picked up at random.
\begin{enumerate}
\item
What is the probability that the card is the queen? 
\item
If the queen is drawn and put aside, what is the probability that the second card picked up is (a) an ace? (b) a queen?\\
\end{enumerate}
\solution
		%\input{ncert/10/15/1/15/defs.tex}
	\item A bag contains $5$ red balls and some blue balls. If the probability of drawing a blue ball is double that if a red ball, determine the number of blue balls in the bag. 
		\\
\solution
		%\input{ncert/10/15/2/3/defs.tex}
	\item A card is selected from a pack of 52 cards.
 \begin{enumerate}[label=(\alph*)] 
                 \item How many points are there in the sample space?
                 \item Calculate the probability that the card is an ace of spades.
                 \item Calculate the probability that the card is (i) an ace and (ii) black card.
 \end{enumerate}
\solution
		%\input{ncert/11/16/3/4/main.tex}
\item Four cards are drawn from a well-shuffled deck of 52 cards. What is the probability of obtaining 3 diamonds and one spade.
\\
\solution
		%\input{ncert/11/16/4/2/defs.tex}
\item In a certain lottery 10,000 tickets are sold and ten equal prizes are awarded. What is the probability of not getting a prize if you buy (a) one ticket (b) two tickets (c) 10 tickets ?	
\\
\solution
		%\input{ncert/11/16/4/4/defs.tex}
		%
\item 
Out of 100 students, two sections of 40 and 60 are formed. If you and your friend are among the 100 students, what is the probability that
\begin{enumerate}
\item you both enter the same section?
\item you both enter the different sections?
\end{enumerate}
\solution
		%\input{ncert/11/16/4/5/defs.tex}
	\item 
The number lock of a suitcase has 4 wheels each labelled with ten digits i.e. from 0 to 9.The lock opens with a sequence of four digits with no repeats.What is the probability of a person getting the right sequence to open the suitcase.
\\
\solution
		%\input{ncert/11/16/4/10/defs.tex}
		%
\item 
Two cards are drawn at random and without replacement from a pack of 52 playing cards. Find the probability that both the cards are black.
\\
\solution
		%\input{ncert/12/13/2/2/defs.tex}
		\item A box of oranges is inspected by examining three randomly selected oranges drawn without replacement. If all the three oranges are good, the box is approved for sale, otherwise, it is rejected. Find the probability that a box containing 15 oranges out of which 12 are good and 3 are bad ones will be approved for sale.
		\label{ncert/12/13/2/3/defs.tex}
		\item Two balls are drawn at random with replacement from a box containing 10 black and 8 red balls. Find the probability that
		\label{ncert/12/13/2/12}
\begin{enumerate}
\item both balls are red.
\item first ball is black and second is red.
\item one of them is black and other is red.
\end{enumerate}

\item In a hostel, 60\% of the students read Hindi newspaper, 40\% read English newspaper and 20\% read both Hindi and English newspapers. A student is selected at random.
		\label{ncert/12/13/2/15}
\begin{enumerate}
\item Find the probability that she reads neither Hindi nor English newspapers.
\item If she reads Hindi newspaper, find the probability that she reads English newspaper.
\item If she reads English newspaper, find the probability that she reads Hindi newspaper.\\
\end{enumerate}
\item The probability of obtaining an even prime number on each die, when a pair of dice is rolled is 
\begin{enumerate}
    \item $0$ 
    
    \item $\frac{1}{3}$ 
    
    \item $\frac{1}{12}$ 
    
    \item $\frac{1}{36}$ 
\end{enumerate}
\solution
		%\input{ncert/12/13/2/17/defs.tex}
	\item A bag contains 4 red and 4 black balls, another bag contains 2 red and 6 black balls. One of the two bags is selected at random and a ball is drawn from the bag which is found to be red. Find the probability that the ball is drawn from the first bag.
\\
\solution
		%\input{ncert/12/13/3/2/main.tex}
  \item
  Cards with numbers 2 to 101 are placed in a box. A card is selected at random.Find the probability that the card has
\begin{enumerate}[label=(\roman*)]
	\item an even number 
	\item a square number
\end{enumerate}
\solution
%\input{exemplar/10/13/3/32/main.tex}
\item
The king, queen and jack of clubs are removed from a deck of 52 playing cards and then well shuffled. Now one card is drawn at random from the remaining cards.  Determine the probability that the card is
\begin{enumerate}[label=(\roman*)]
\item a club
\item 10 of hearts
\end{enumerate}
\solution
%\input{exemplar/10/13/3/29/main.tex}
\item A team of medical students doing their internship have to assist during surgeries
at a city hospital. The probabilities of surgeries rated as very complex, complex,
routine, simple or very simple are respectively, 0.15, 0.20, 0.31, 0.26, .08. Find
the probabilities that a particular surgery will be rated
\begin{enumerate}
	\item complex or very complex;
	\item neither very complex nor very simple;
	\item routine or complex
	\item routine or simple
\end{enumerate}
\solution
%\input{exemplar/11/16/3/8(1)/main.tex}
\item A card is selected from a pack of 52 cards.
\begin{enumerate}[label=(\alph*)]
    \item How many points are there in the sample space?
    \item Calculate the probability that the card is an ace of spades.
    \item Calculate the probability that the card is (i) an ace and (ii) black card.
\end{enumerate}
\solution
%\input{exemplar/11/16/3/4/main2.tex}
\item The probability that a non leap year selected at random will contain 53 sundays.
\\
\solution
%\input{exemplar/10/13/1/19/main.tex}
\item One of the four persons John, Rita, Aslam or Gurpreet will be promoted next
month. Consequently the sample space consists of four elementary outcomes
S = {John promoted, Rita promoted, Aslam promoted, Gurpreet promoted}
You are told that the chances of John’s promotion is same as that of Gurpreet,
Rita’s chances of promotion are twice as likely as Johns. Aslam’s chances are
four times that of John.
\begin{enumerate}
	\item Determine
	\begin{enumerate}
		\item P (John promoted)
		\item P (Rita promoted)
		\item P (Aslam promoted)
		\item P (Gurpreet promoted)
	\end{enumerate}
	\item If A = {John promoted or Gurpreet promoted}, find P (A).
\end{enumerate}
\solution
%\input{exemplar/11/16/3/10/main.tex}
\item A card is drawn from a deck of 52 cards. Find the probability of getting a king or a heart or a red card.\\
\solution
%\input{exemplar/11/16/3/15/main.tex}
\item The probability that a student will pass his examination is 0.73, the probability of
the student getting a compartment is 0.13, and the probability that the student will
either pass or get compartment is 0.96. State True or False.\\
\solution
%\input{exemplar/11/16/3/31/main.tex}
\item A card is selected from a pack of 52 cards\\
\begin{enumerate}[label=(\alph*)]
\item How many points are there in the sample space?
\item Calculate the probability that the cards is an ace of spades.
\item Calculate the probability that the card is (i) an ace (ii)black card.\\
\end{enumerate}
%\input{ncert/11/16/3/4_1/Prob_4.tex}
\item In a non-leap year, the probability of having 53 tuesdays or 53 wednesdays is\\
\solution
%\input{exemplar/11/16/3/18/main.tex}
\item There are 1000 sealed envelopes in a box, 10 of them contain a cash prize of
Rs 100 each, 100 of them contain a cash prize of Rs 50 each and 200 of them
contain a cash prize of Rs 10 each and rest do not contain any cash prize. If they
are well shuffled and an envelope is picked up out, what is the probability that it
contains no cash prize?\\
\solution
%\input{exemplar/10/13/3/34/main.tex}
\item 
A die is thrown and a card is selected at random from a deck of 52 playing cards. The probability of getting an even number on the die and a spade card.\\
\solution
%\input{exemplar/12/13/3/78/main.tex}
\item
If 4-digit numbers greater than 5,000 are randomly formed from the digits 0, 1, 3, 5, and 7, what is the probability of forming a number divisible by 5 when:
\begin{enumerate}
    \item The digits are repeated?
    \item The repetition of digits is not allowed?
\end{enumerate}
\solution
%\input{ncert/11/16/4/9/main.tex}
\item Consider the probability space $\brak{\Omega, \mathcal{G}, P}$ where $\Omega = [0,2]$ and $\mathcal{G} = \cbrak{\phi, \Omega, [0,1], (1,2]}$. Let $X$ and $Y$ be two functions on $\Omega$ defined as
\begin{align*}
    X(\omega) = 
    \begin{cases}
        1 & \text{if }\omega \in [0, 1]\\
        2 & \text{if }\omega \in (1, 2]
    \end{cases}
\end{align*}
and
\begin{align*}
    Y(\omega) = 
    \begin{cases}
        2 & \text{if }\omega \in [0, 1.5]\\
        3 & \text{if }\omega \in (1.5, 2].
    \end{cases}
\end{align*}
Then which one of the following statements is true?
\begin{enumerate}
    \item [(A)] $X$ is a random variable with respect to $\mathcal{G}$, but $Y$ is not a random variable with respect to $\mathcal{G}$.
    \item [(B)] $Y$ is a random variable with respect to $\mathcal{G}$, but $X$ is not a random variable with respect to $\mathcal{G}$.
    \item [(C)] Neither $X$ nor $Y$ is a random variable with respect to $\mathcal{G}$.
    \item [(D)] Both $X$ and $Y$ are random variables with respect to $\mathcal{G}$.
\end{enumerate} \hfill (GATE ST 2023)\\
\solution
%\input{gate/ST/2023/14/main.tex}
	\item  A die is loaded in such a way that each odd number is twice as likely to occur as
each even number. Find $P(G)$, where $G$ is the event that a number greater than
3 occurs on a single roll of the die.
\\
\solution
		%\input{exemplar/11/16/3/5/main.tex}
	\item All the jacks, queens and kings are removed from a deck of 52 playing cards. The remaining cards are well shuffled and then one card is drawn at random. Giving ace a value 1 similar value for other cards, find the probability that the card has a value 
		\begin{enumerate}
			\item 7
			\item greater than 7
			\item less than 7
		\end{enumerate}
		%\input{exemplar/10/13/3/30/main.tex}
  \item A Lot consists of 48 mobile phones of which 42 are good, 3 have only minor defects and 3 have major defects.Varnika will buy a phone if it is good but the trader will only buy a mobile if it has no major defects. One phone is selected at random from the lot. What is the probability that it is
\begin{enumerate}
	\item acceptable to Varnika?
            \item acceptable to the trader?
\end{enumerate}
\solution
	%\input{exemplar/10/13/3/40/main.tex}
 \item A student says that if you throw a die, it will show up 1 or not 1. Therefore, the probability of getting 1 and the probability of getting 'not 1' each is equal to $\frac{1}{2}$. Is this correct? Give reasons.\\
 \solution
        %\input{exemplar/10/13/2/9/main.tex}
   \item Four candidates A, B, C, D have ap-
plied for the assignment to coach a school cricket
team. If A is twice as likely to be selected as B, and
B and C are given about the same chance of being
selected, while C is twice as likely to be selected
as D, what are the probabilities that
\begin{enumerate}
\item C will be selected?
\item A will not be selected?
\end{enumerate}
	%\input{exemplar/11/16/3/9/main.tex}
 \item A bag contain 24 balls of which $x$ balls are red, $2x$ are white and $3x$ are blue. A ball is selected at random, What is the probability that it is
\begin{enumerate}[label=\alph*)]
\item not red ?
\item white ?
\end{enumerate}
%\input{exemplar/10/13/3/41/main.tex}
If the letters of the word ASSASSINATION are arranged at random. Find the Probability that
\begin{enumerate}[label=(\alph*)]
\item Four $S's$ come consecutively in the word
\item Two  $I's$ and two $N's$ come together
\item All $A's$ are not coming together
\item No two $A's$ are coming together
\end{enumerate}
%\input{exemplar/11/16/3/14/main.tex}
	\item One urn contains two black balls (labelled B1 and B2) and one white ball. A
	second urn contains one black ball and two white balls (labelled W1 and W2).
	Suppose the following experiment is performed. One of the two urns is chosen
	at random. Next a ball is randomly chosen from the urn. Then a second ball is
	chosen at random from the same urn without replacing the first ball.
	
	\begin{enumerate}
	\item What is the probability that two black balls are chosen?
	
	\item What is the probability that two balls of opposite colour are chosen?
	\end{enumerate}
	\solution
	%\input{exemplar/11/16/3/12/main1.tex}
\end{enumerate}

		%
\item 
Two cards are drawn at random and without replacement from a pack of 52 playing cards. Find the probability that both the cards are black.
\\
\solution
		%\begin{enumerate}[label=\thesection.\arabic*,ref=\thesection.\theenumi]
	\item One card is drawn from a well-shuffled deck of 52 cards. Find the probability of getting
\begin{enumerate}
\item A king of red colour 
\item A face card 
\item A red face card
\item The jack of hearts
\item A spade
\item The queen of diamonds

\end{enumerate}
\solution
		%\input{ncert/10/15/1/14/main.tex}
	\item Five cards—the ten, jack, queen, king and ace of diamonds, are well-shuffled with their face downwards. One card is then picked up at random.
\begin{enumerate}
\item
What is the probability that the card is the queen? 
\item
If the queen is drawn and put aside, what is the probability that the second card picked up is (a) an ace? (b) a queen?\\
\end{enumerate}
\solution
		%\input{ncert/10/15/1/15/defs.tex}
	\item A bag contains $5$ red balls and some blue balls. If the probability of drawing a blue ball is double that if a red ball, determine the number of blue balls in the bag. 
		\\
\solution
		%\input{ncert/10/15/2/3/defs.tex}
	\item A card is selected from a pack of 52 cards.
 \begin{enumerate}[label=(\alph*)] 
                 \item How many points are there in the sample space?
                 \item Calculate the probability that the card is an ace of spades.
                 \item Calculate the probability that the card is (i) an ace and (ii) black card.
 \end{enumerate}
\solution
		%\input{ncert/11/16/3/4/main.tex}
\item Four cards are drawn from a well-shuffled deck of 52 cards. What is the probability of obtaining 3 diamonds and one spade.
\\
\solution
		%\input{ncert/11/16/4/2/defs.tex}
\item In a certain lottery 10,000 tickets are sold and ten equal prizes are awarded. What is the probability of not getting a prize if you buy (a) one ticket (b) two tickets (c) 10 tickets ?	
\\
\solution
		%\input{ncert/11/16/4/4/defs.tex}
		%
\item 
Out of 100 students, two sections of 40 and 60 are formed. If you and your friend are among the 100 students, what is the probability that
\begin{enumerate}
\item you both enter the same section?
\item you both enter the different sections?
\end{enumerate}
\solution
		%\input{ncert/11/16/4/5/defs.tex}
	\item 
The number lock of a suitcase has 4 wheels each labelled with ten digits i.e. from 0 to 9.The lock opens with a sequence of four digits with no repeats.What is the probability of a person getting the right sequence to open the suitcase.
\\
\solution
		%\input{ncert/11/16/4/10/defs.tex}
		%
\item 
Two cards are drawn at random and without replacement from a pack of 52 playing cards. Find the probability that both the cards are black.
\\
\solution
		%\input{ncert/12/13/2/2/defs.tex}
		\item A box of oranges is inspected by examining three randomly selected oranges drawn without replacement. If all the three oranges are good, the box is approved for sale, otherwise, it is rejected. Find the probability that a box containing 15 oranges out of which 12 are good and 3 are bad ones will be approved for sale.
		\label{ncert/12/13/2/3/defs.tex}
		\item Two balls are drawn at random with replacement from a box containing 10 black and 8 red balls. Find the probability that
		\label{ncert/12/13/2/12}
\begin{enumerate}
\item both balls are red.
\item first ball is black and second is red.
\item one of them is black and other is red.
\end{enumerate}

\item In a hostel, 60\% of the students read Hindi newspaper, 40\% read English newspaper and 20\% read both Hindi and English newspapers. A student is selected at random.
		\label{ncert/12/13/2/15}
\begin{enumerate}
\item Find the probability that she reads neither Hindi nor English newspapers.
\item If she reads Hindi newspaper, find the probability that she reads English newspaper.
\item If she reads English newspaper, find the probability that she reads Hindi newspaper.\\
\end{enumerate}
\item The probability of obtaining an even prime number on each die, when a pair of dice is rolled is 
\begin{enumerate}
    \item $0$ 
    
    \item $\frac{1}{3}$ 
    
    \item $\frac{1}{12}$ 
    
    \item $\frac{1}{36}$ 
\end{enumerate}
\solution
		%\input{ncert/12/13/2/17/defs.tex}
	\item A bag contains 4 red and 4 black balls, another bag contains 2 red and 6 black balls. One of the two bags is selected at random and a ball is drawn from the bag which is found to be red. Find the probability that the ball is drawn from the first bag.
\\
\solution
		%\input{ncert/12/13/3/2/main.tex}
  \item
  Cards with numbers 2 to 101 are placed in a box. A card is selected at random.Find the probability that the card has
\begin{enumerate}[label=(\roman*)]
	\item an even number 
	\item a square number
\end{enumerate}
\solution
%\input{exemplar/10/13/3/32/main.tex}
\item
The king, queen and jack of clubs are removed from a deck of 52 playing cards and then well shuffled. Now one card is drawn at random from the remaining cards.  Determine the probability that the card is
\begin{enumerate}[label=(\roman*)]
\item a club
\item 10 of hearts
\end{enumerate}
\solution
%\input{exemplar/10/13/3/29/main.tex}
\item A team of medical students doing their internship have to assist during surgeries
at a city hospital. The probabilities of surgeries rated as very complex, complex,
routine, simple or very simple are respectively, 0.15, 0.20, 0.31, 0.26, .08. Find
the probabilities that a particular surgery will be rated
\begin{enumerate}
	\item complex or very complex;
	\item neither very complex nor very simple;
	\item routine or complex
	\item routine or simple
\end{enumerate}
\solution
%\input{exemplar/11/16/3/8(1)/main.tex}
\item A card is selected from a pack of 52 cards.
\begin{enumerate}[label=(\alph*)]
    \item How many points are there in the sample space?
    \item Calculate the probability that the card is an ace of spades.
    \item Calculate the probability that the card is (i) an ace and (ii) black card.
\end{enumerate}
\solution
%\input{exemplar/11/16/3/4/main2.tex}
\item The probability that a non leap year selected at random will contain 53 sundays.
\\
\solution
%\input{exemplar/10/13/1/19/main.tex}
\item One of the four persons John, Rita, Aslam or Gurpreet will be promoted next
month. Consequently the sample space consists of four elementary outcomes
S = {John promoted, Rita promoted, Aslam promoted, Gurpreet promoted}
You are told that the chances of John’s promotion is same as that of Gurpreet,
Rita’s chances of promotion are twice as likely as Johns. Aslam’s chances are
four times that of John.
\begin{enumerate}
	\item Determine
	\begin{enumerate}
		\item P (John promoted)
		\item P (Rita promoted)
		\item P (Aslam promoted)
		\item P (Gurpreet promoted)
	\end{enumerate}
	\item If A = {John promoted or Gurpreet promoted}, find P (A).
\end{enumerate}
\solution
%\input{exemplar/11/16/3/10/main.tex}
\item A card is drawn from a deck of 52 cards. Find the probability of getting a king or a heart or a red card.\\
\solution
%\input{exemplar/11/16/3/15/main.tex}
\item The probability that a student will pass his examination is 0.73, the probability of
the student getting a compartment is 0.13, and the probability that the student will
either pass or get compartment is 0.96. State True or False.\\
\solution
%\input{exemplar/11/16/3/31/main.tex}
\item A card is selected from a pack of 52 cards\\
\begin{enumerate}[label=(\alph*)]
\item How many points are there in the sample space?
\item Calculate the probability that the cards is an ace of spades.
\item Calculate the probability that the card is (i) an ace (ii)black card.\\
\end{enumerate}
%\input{ncert/11/16/3/4_1/Prob_4.tex}
\item In a non-leap year, the probability of having 53 tuesdays or 53 wednesdays is\\
\solution
%\input{exemplar/11/16/3/18/main.tex}
\item There are 1000 sealed envelopes in a box, 10 of them contain a cash prize of
Rs 100 each, 100 of them contain a cash prize of Rs 50 each and 200 of them
contain a cash prize of Rs 10 each and rest do not contain any cash prize. If they
are well shuffled and an envelope is picked up out, what is the probability that it
contains no cash prize?\\
\solution
%\input{exemplar/10/13/3/34/main.tex}
\item 
A die is thrown and a card is selected at random from a deck of 52 playing cards. The probability of getting an even number on the die and a spade card.\\
\solution
%\input{exemplar/12/13/3/78/main.tex}
\item
If 4-digit numbers greater than 5,000 are randomly formed from the digits 0, 1, 3, 5, and 7, what is the probability of forming a number divisible by 5 when:
\begin{enumerate}
    \item The digits are repeated?
    \item The repetition of digits is not allowed?
\end{enumerate}
\solution
%\input{ncert/11/16/4/9/main.tex}
\item Consider the probability space $\brak{\Omega, \mathcal{G}, P}$ where $\Omega = [0,2]$ and $\mathcal{G} = \cbrak{\phi, \Omega, [0,1], (1,2]}$. Let $X$ and $Y$ be two functions on $\Omega$ defined as
\begin{align*}
    X(\omega) = 
    \begin{cases}
        1 & \text{if }\omega \in [0, 1]\\
        2 & \text{if }\omega \in (1, 2]
    \end{cases}
\end{align*}
and
\begin{align*}
    Y(\omega) = 
    \begin{cases}
        2 & \text{if }\omega \in [0, 1.5]\\
        3 & \text{if }\omega \in (1.5, 2].
    \end{cases}
\end{align*}
Then which one of the following statements is true?
\begin{enumerate}
    \item [(A)] $X$ is a random variable with respect to $\mathcal{G}$, but $Y$ is not a random variable with respect to $\mathcal{G}$.
    \item [(B)] $Y$ is a random variable with respect to $\mathcal{G}$, but $X$ is not a random variable with respect to $\mathcal{G}$.
    \item [(C)] Neither $X$ nor $Y$ is a random variable with respect to $\mathcal{G}$.
    \item [(D)] Both $X$ and $Y$ are random variables with respect to $\mathcal{G}$.
\end{enumerate} \hfill (GATE ST 2023)\\
\solution
%\input{gate/ST/2023/14/main.tex}
	\item  A die is loaded in such a way that each odd number is twice as likely to occur as
each even number. Find $P(G)$, where $G$ is the event that a number greater than
3 occurs on a single roll of the die.
\\
\solution
		%\input{exemplar/11/16/3/5/main.tex}
	\item All the jacks, queens and kings are removed from a deck of 52 playing cards. The remaining cards are well shuffled and then one card is drawn at random. Giving ace a value 1 similar value for other cards, find the probability that the card has a value 
		\begin{enumerate}
			\item 7
			\item greater than 7
			\item less than 7
		\end{enumerate}
		%\input{exemplar/10/13/3/30/main.tex}
  \item A Lot consists of 48 mobile phones of which 42 are good, 3 have only minor defects and 3 have major defects.Varnika will buy a phone if it is good but the trader will only buy a mobile if it has no major defects. One phone is selected at random from the lot. What is the probability that it is
\begin{enumerate}
	\item acceptable to Varnika?
            \item acceptable to the trader?
\end{enumerate}
\solution
	%\input{exemplar/10/13/3/40/main.tex}
 \item A student says that if you throw a die, it will show up 1 or not 1. Therefore, the probability of getting 1 and the probability of getting 'not 1' each is equal to $\frac{1}{2}$. Is this correct? Give reasons.\\
 \solution
        %\input{exemplar/10/13/2/9/main.tex}
   \item Four candidates A, B, C, D have ap-
plied for the assignment to coach a school cricket
team. If A is twice as likely to be selected as B, and
B and C are given about the same chance of being
selected, while C is twice as likely to be selected
as D, what are the probabilities that
\begin{enumerate}
\item C will be selected?
\item A will not be selected?
\end{enumerate}
	%\input{exemplar/11/16/3/9/main.tex}
 \item A bag contain 24 balls of which $x$ balls are red, $2x$ are white and $3x$ are blue. A ball is selected at random, What is the probability that it is
\begin{enumerate}[label=\alph*)]
\item not red ?
\item white ?
\end{enumerate}
%\input{exemplar/10/13/3/41/main.tex}
If the letters of the word ASSASSINATION are arranged at random. Find the Probability that
\begin{enumerate}[label=(\alph*)]
\item Four $S's$ come consecutively in the word
\item Two  $I's$ and two $N's$ come together
\item All $A's$ are not coming together
\item No two $A's$ are coming together
\end{enumerate}
%\input{exemplar/11/16/3/14/main.tex}
	\item One urn contains two black balls (labelled B1 and B2) and one white ball. A
	second urn contains one black ball and two white balls (labelled W1 and W2).
	Suppose the following experiment is performed. One of the two urns is chosen
	at random. Next a ball is randomly chosen from the urn. Then a second ball is
	chosen at random from the same urn without replacing the first ball.
	
	\begin{enumerate}
	\item What is the probability that two black balls are chosen?
	
	\item What is the probability that two balls of opposite colour are chosen?
	\end{enumerate}
	\solution
	%\input{exemplar/11/16/3/12/main1.tex}
\end{enumerate}

		\item A box of oranges is inspected by examining three randomly selected oranges drawn without replacement. If all the three oranges are good, the box is approved for sale, otherwise, it is rejected. Find the probability that a box containing 15 oranges out of which 12 are good and 3 are bad ones will be approved for sale.
		\label{ncert/12/13/2/3/defs.tex}
		\item Two balls are drawn at random with replacement from a box containing 10 black and 8 red balls. Find the probability that
		\label{ncert/12/13/2/12}
\begin{enumerate}
\item both balls are red.
\item first ball is black and second is red.
\item one of them is black and other is red.
\end{enumerate}

\item In a hostel, 60\% of the students read Hindi newspaper, 40\% read English newspaper and 20\% read both Hindi and English newspapers. A student is selected at random.
		\label{ncert/12/13/2/15}
\begin{enumerate}
\item Find the probability that she reads neither Hindi nor English newspapers.
\item If she reads Hindi newspaper, find the probability that she reads English newspaper.
\item If she reads English newspaper, find the probability that she reads Hindi newspaper.\\
\end{enumerate}
\item The probability of obtaining an even prime number on each die, when a pair of dice is rolled is 
\begin{enumerate}
    \item $0$ 
    
    \item $\frac{1}{3}$ 
    
    \item $\frac{1}{12}$ 
    
    \item $\frac{1}{36}$ 
\end{enumerate}
\solution
		%\begin{enumerate}[label=\thesection.\arabic*,ref=\thesection.\theenumi]
	\item One card is drawn from a well-shuffled deck of 52 cards. Find the probability of getting
\begin{enumerate}
\item A king of red colour 
\item A face card 
\item A red face card
\item The jack of hearts
\item A spade
\item The queen of diamonds

\end{enumerate}
\solution
		%\input{ncert/10/15/1/14/main.tex}
	\item Five cards—the ten, jack, queen, king and ace of diamonds, are well-shuffled with their face downwards. One card is then picked up at random.
\begin{enumerate}
\item
What is the probability that the card is the queen? 
\item
If the queen is drawn and put aside, what is the probability that the second card picked up is (a) an ace? (b) a queen?\\
\end{enumerate}
\solution
		%\input{ncert/10/15/1/15/defs.tex}
	\item A bag contains $5$ red balls and some blue balls. If the probability of drawing a blue ball is double that if a red ball, determine the number of blue balls in the bag. 
		\\
\solution
		%\input{ncert/10/15/2/3/defs.tex}
	\item A card is selected from a pack of 52 cards.
 \begin{enumerate}[label=(\alph*)] 
                 \item How many points are there in the sample space?
                 \item Calculate the probability that the card is an ace of spades.
                 \item Calculate the probability that the card is (i) an ace and (ii) black card.
 \end{enumerate}
\solution
		%\input{ncert/11/16/3/4/main.tex}
\item Four cards are drawn from a well-shuffled deck of 52 cards. What is the probability of obtaining 3 diamonds and one spade.
\\
\solution
		%\input{ncert/11/16/4/2/defs.tex}
\item In a certain lottery 10,000 tickets are sold and ten equal prizes are awarded. What is the probability of not getting a prize if you buy (a) one ticket (b) two tickets (c) 10 tickets ?	
\\
\solution
		%\input{ncert/11/16/4/4/defs.tex}
		%
\item 
Out of 100 students, two sections of 40 and 60 are formed. If you and your friend are among the 100 students, what is the probability that
\begin{enumerate}
\item you both enter the same section?
\item you both enter the different sections?
\end{enumerate}
\solution
		%\input{ncert/11/16/4/5/defs.tex}
	\item 
The number lock of a suitcase has 4 wheels each labelled with ten digits i.e. from 0 to 9.The lock opens with a sequence of four digits with no repeats.What is the probability of a person getting the right sequence to open the suitcase.
\\
\solution
		%\input{ncert/11/16/4/10/defs.tex}
		%
\item 
Two cards are drawn at random and without replacement from a pack of 52 playing cards. Find the probability that both the cards are black.
\\
\solution
		%\input{ncert/12/13/2/2/defs.tex}
		\item A box of oranges is inspected by examining three randomly selected oranges drawn without replacement. If all the three oranges are good, the box is approved for sale, otherwise, it is rejected. Find the probability that a box containing 15 oranges out of which 12 are good and 3 are bad ones will be approved for sale.
		\label{ncert/12/13/2/3/defs.tex}
		\item Two balls are drawn at random with replacement from a box containing 10 black and 8 red balls. Find the probability that
		\label{ncert/12/13/2/12}
\begin{enumerate}
\item both balls are red.
\item first ball is black and second is red.
\item one of them is black and other is red.
\end{enumerate}

\item In a hostel, 60\% of the students read Hindi newspaper, 40\% read English newspaper and 20\% read both Hindi and English newspapers. A student is selected at random.
		\label{ncert/12/13/2/15}
\begin{enumerate}
\item Find the probability that she reads neither Hindi nor English newspapers.
\item If she reads Hindi newspaper, find the probability that she reads English newspaper.
\item If she reads English newspaper, find the probability that she reads Hindi newspaper.\\
\end{enumerate}
\item The probability of obtaining an even prime number on each die, when a pair of dice is rolled is 
\begin{enumerate}
    \item $0$ 
    
    \item $\frac{1}{3}$ 
    
    \item $\frac{1}{12}$ 
    
    \item $\frac{1}{36}$ 
\end{enumerate}
\solution
		%\input{ncert/12/13/2/17/defs.tex}
	\item A bag contains 4 red and 4 black balls, another bag contains 2 red and 6 black balls. One of the two bags is selected at random and a ball is drawn from the bag which is found to be red. Find the probability that the ball is drawn from the first bag.
\\
\solution
		%\input{ncert/12/13/3/2/main.tex}
  \item
  Cards with numbers 2 to 101 are placed in a box. A card is selected at random.Find the probability that the card has
\begin{enumerate}[label=(\roman*)]
	\item an even number 
	\item a square number
\end{enumerate}
\solution
%\input{exemplar/10/13/3/32/main.tex}
\item
The king, queen and jack of clubs are removed from a deck of 52 playing cards and then well shuffled. Now one card is drawn at random from the remaining cards.  Determine the probability that the card is
\begin{enumerate}[label=(\roman*)]
\item a club
\item 10 of hearts
\end{enumerate}
\solution
%\input{exemplar/10/13/3/29/main.tex}
\item A team of medical students doing their internship have to assist during surgeries
at a city hospital. The probabilities of surgeries rated as very complex, complex,
routine, simple or very simple are respectively, 0.15, 0.20, 0.31, 0.26, .08. Find
the probabilities that a particular surgery will be rated
\begin{enumerate}
	\item complex or very complex;
	\item neither very complex nor very simple;
	\item routine or complex
	\item routine or simple
\end{enumerate}
\solution
%\input{exemplar/11/16/3/8(1)/main.tex}
\item A card is selected from a pack of 52 cards.
\begin{enumerate}[label=(\alph*)]
    \item How many points are there in the sample space?
    \item Calculate the probability that the card is an ace of spades.
    \item Calculate the probability that the card is (i) an ace and (ii) black card.
\end{enumerate}
\solution
%\input{exemplar/11/16/3/4/main2.tex}
\item The probability that a non leap year selected at random will contain 53 sundays.
\\
\solution
%\input{exemplar/10/13/1/19/main.tex}
\item One of the four persons John, Rita, Aslam or Gurpreet will be promoted next
month. Consequently the sample space consists of four elementary outcomes
S = {John promoted, Rita promoted, Aslam promoted, Gurpreet promoted}
You are told that the chances of John’s promotion is same as that of Gurpreet,
Rita’s chances of promotion are twice as likely as Johns. Aslam’s chances are
four times that of John.
\begin{enumerate}
	\item Determine
	\begin{enumerate}
		\item P (John promoted)
		\item P (Rita promoted)
		\item P (Aslam promoted)
		\item P (Gurpreet promoted)
	\end{enumerate}
	\item If A = {John promoted or Gurpreet promoted}, find P (A).
\end{enumerate}
\solution
%\input{exemplar/11/16/3/10/main.tex}
\item A card is drawn from a deck of 52 cards. Find the probability of getting a king or a heart or a red card.\\
\solution
%\input{exemplar/11/16/3/15/main.tex}
\item The probability that a student will pass his examination is 0.73, the probability of
the student getting a compartment is 0.13, and the probability that the student will
either pass or get compartment is 0.96. State True or False.\\
\solution
%\input{exemplar/11/16/3/31/main.tex}
\item A card is selected from a pack of 52 cards\\
\begin{enumerate}[label=(\alph*)]
\item How many points are there in the sample space?
\item Calculate the probability that the cards is an ace of spades.
\item Calculate the probability that the card is (i) an ace (ii)black card.\\
\end{enumerate}
%\input{ncert/11/16/3/4_1/Prob_4.tex}
\item In a non-leap year, the probability of having 53 tuesdays or 53 wednesdays is\\
\solution
%\input{exemplar/11/16/3/18/main.tex}
\item There are 1000 sealed envelopes in a box, 10 of them contain a cash prize of
Rs 100 each, 100 of them contain a cash prize of Rs 50 each and 200 of them
contain a cash prize of Rs 10 each and rest do not contain any cash prize. If they
are well shuffled and an envelope is picked up out, what is the probability that it
contains no cash prize?\\
\solution
%\input{exemplar/10/13/3/34/main.tex}
\item 
A die is thrown and a card is selected at random from a deck of 52 playing cards. The probability of getting an even number on the die and a spade card.\\
\solution
%\input{exemplar/12/13/3/78/main.tex}
\item
If 4-digit numbers greater than 5,000 are randomly formed from the digits 0, 1, 3, 5, and 7, what is the probability of forming a number divisible by 5 when:
\begin{enumerate}
    \item The digits are repeated?
    \item The repetition of digits is not allowed?
\end{enumerate}
\solution
%\input{ncert/11/16/4/9/main.tex}
\item Consider the probability space $\brak{\Omega, \mathcal{G}, P}$ where $\Omega = [0,2]$ and $\mathcal{G} = \cbrak{\phi, \Omega, [0,1], (1,2]}$. Let $X$ and $Y$ be two functions on $\Omega$ defined as
\begin{align*}
    X(\omega) = 
    \begin{cases}
        1 & \text{if }\omega \in [0, 1]\\
        2 & \text{if }\omega \in (1, 2]
    \end{cases}
\end{align*}
and
\begin{align*}
    Y(\omega) = 
    \begin{cases}
        2 & \text{if }\omega \in [0, 1.5]\\
        3 & \text{if }\omega \in (1.5, 2].
    \end{cases}
\end{align*}
Then which one of the following statements is true?
\begin{enumerate}
    \item [(A)] $X$ is a random variable with respect to $\mathcal{G}$, but $Y$ is not a random variable with respect to $\mathcal{G}$.
    \item [(B)] $Y$ is a random variable with respect to $\mathcal{G}$, but $X$ is not a random variable with respect to $\mathcal{G}$.
    \item [(C)] Neither $X$ nor $Y$ is a random variable with respect to $\mathcal{G}$.
    \item [(D)] Both $X$ and $Y$ are random variables with respect to $\mathcal{G}$.
\end{enumerate} \hfill (GATE ST 2023)\\
\solution
%\input{gate/ST/2023/14/main.tex}
	\item  A die is loaded in such a way that each odd number is twice as likely to occur as
each even number. Find $P(G)$, where $G$ is the event that a number greater than
3 occurs on a single roll of the die.
\\
\solution
		%\input{exemplar/11/16/3/5/main.tex}
	\item All the jacks, queens and kings are removed from a deck of 52 playing cards. The remaining cards are well shuffled and then one card is drawn at random. Giving ace a value 1 similar value for other cards, find the probability that the card has a value 
		\begin{enumerate}
			\item 7
			\item greater than 7
			\item less than 7
		\end{enumerate}
		%\input{exemplar/10/13/3/30/main.tex}
  \item A Lot consists of 48 mobile phones of which 42 are good, 3 have only minor defects and 3 have major defects.Varnika will buy a phone if it is good but the trader will only buy a mobile if it has no major defects. One phone is selected at random from the lot. What is the probability that it is
\begin{enumerate}
	\item acceptable to Varnika?
            \item acceptable to the trader?
\end{enumerate}
\solution
	%\input{exemplar/10/13/3/40/main.tex}
 \item A student says that if you throw a die, it will show up 1 or not 1. Therefore, the probability of getting 1 and the probability of getting 'not 1' each is equal to $\frac{1}{2}$. Is this correct? Give reasons.\\
 \solution
        %\input{exemplar/10/13/2/9/main.tex}
   \item Four candidates A, B, C, D have ap-
plied for the assignment to coach a school cricket
team. If A is twice as likely to be selected as B, and
B and C are given about the same chance of being
selected, while C is twice as likely to be selected
as D, what are the probabilities that
\begin{enumerate}
\item C will be selected?
\item A will not be selected?
\end{enumerate}
	%\input{exemplar/11/16/3/9/main.tex}
 \item A bag contain 24 balls of which $x$ balls are red, $2x$ are white and $3x$ are blue. A ball is selected at random, What is the probability that it is
\begin{enumerate}[label=\alph*)]
\item not red ?
\item white ?
\end{enumerate}
%\input{exemplar/10/13/3/41/main.tex}
If the letters of the word ASSASSINATION are arranged at random. Find the Probability that
\begin{enumerate}[label=(\alph*)]
\item Four $S's$ come consecutively in the word
\item Two  $I's$ and two $N's$ come together
\item All $A's$ are not coming together
\item No two $A's$ are coming together
\end{enumerate}
%\input{exemplar/11/16/3/14/main.tex}
	\item One urn contains two black balls (labelled B1 and B2) and one white ball. A
	second urn contains one black ball and two white balls (labelled W1 and W2).
	Suppose the following experiment is performed. One of the two urns is chosen
	at random. Next a ball is randomly chosen from the urn. Then a second ball is
	chosen at random from the same urn without replacing the first ball.
	
	\begin{enumerate}
	\item What is the probability that two black balls are chosen?
	
	\item What is the probability that two balls of opposite colour are chosen?
	\end{enumerate}
	\solution
	%\input{exemplar/11/16/3/12/main1.tex}
\end{enumerate}

	\item A bag contains 4 red and 4 black balls, another bag contains 2 red and 6 black balls. One of the two bags is selected at random and a ball is drawn from the bag which is found to be red. Find the probability that the ball is drawn from the first bag.
\\
\solution
		%\begin{table}[H]
	\centering
\begin{tabular}{|c|c|c|}
\hline
Random variable &Value &Definition\\ \hline
\multirow{3}{*}{X} &0 &Slips of Rs 1\\
&1 &Slips of Rs 5\\
&2 &Slips of Rs 13\\ \hline
\multirow{2}{*}{Y} &0 &Box A\\
&1 &Box B\\\hline
\end{tabular}
\caption{}
\label{tab:Distribution}
\end{table}
See \tabref{tab:Distribution}.
\begin{align}
p_{Y}\brak{k}= \begin{cases} 
      \frac{1}{3} & {k=0} \\
      \frac{2}{3 }& {k=1} 
   \end{cases}
   \\
p_{Y|X}\brak{0|0} = \frac{19}{25}\, 
p_{Y|X}\brak{0|1} = \frac{6}{25}\,
p_{Y|X}\brak{1|0} = \frac{45}{50}\,
p_{Y|X}\brak{1|2} = \frac{5}{50}
\end{align}
The desired probability is the probability that a slip drawn at random is marked other than Rs 1,
\begin{align}
&=1-p_X\brak{0}\\
&= p_X(1) + p_X(2)
\end{align}
Using Bayes theorem,
\begin{align}
&= p_Y\brak{0} \times \pr{Y=0 | X=1} + p_Y\brak{1} \times \pr{Y=1|X=2}\\
&=\frac{1}{3} \times \frac{6}{25} + \frac{2}{3} \times \frac{5}{50}\\
&=\frac{11}{75}
\end{align}

\newpage

%\tableofcontents

\bigskip

\renewcommand{\thefigure}{\theenumi}
\renewcommand{\thetable}{\theenumi}
%\renewcommand{\theequation}{\theenumi}

%\begin{abstract}
%%\boldmath
%In this letter, an algorithm for evaluating the exact analytical bit error rate  (BER)  for the piecewise linear (PL) combiner for  multiple relays is presented. Previous results were available only for upto three relays. The algorithm is unique in the sense that  the actual mathematical expressions, that are prohibitively large, need not be explicitly obtained. The diversity gain due to multiple relays is shown through plots of the analytical BER, well supported by simulations. 
%
%\end{abstract}
% IEEEtran.cls defaults to using nonbold math in the Abstract.
% This preserves the distinction between vectors and scalars. However,
% if the journal you are submitting to favors bold math in the abstract,
% then you can use LaTeX's standard command \boldmath at the very start
% of the abstract to achieve this. Many IEEE journals frown on math
% in the abstract anyway.

% Note that keywords are not normally used for peerreview papers.
%\begin{IEEEkeywords}
%Cooperative diversity, decode and forward, piecewise linear
%\end{IEEEkeywords}



% For peer review papers, you can put extra information on the cover
% page as needed:
% \ifCLASSOPTIONpeerreview
% \begin{center} \bfseries EDICS Category: 3-BBND \end{center}
% \fi
%
% For peerreview papers, this IEEEtran command inserts a page break and
% creates the second title. It will be ignored for other modes.
%\IEEEpeerreviewmaketitle




  \item
  Cards with numbers 2 to 101 are placed in a box. A card is selected at random.Find the probability that the card has
\begin{enumerate}[label=(\roman*)]
	\item an even number 
	\item a square number
\end{enumerate}
\solution
%\begin{table}[H]
	\centering
\begin{tabular}{|c|c|c|}
\hline
Random variable &Value &Definition\\ \hline
\multirow{3}{*}{X} &0 &Slips of Rs 1\\
&1 &Slips of Rs 5\\
&2 &Slips of Rs 13\\ \hline
\multirow{2}{*}{Y} &0 &Box A\\
&1 &Box B\\\hline
\end{tabular}
\caption{}
\label{tab:Distribution}
\end{table}
See \tabref{tab:Distribution}.
\begin{align}
p_{Y}\brak{k}= \begin{cases} 
      \frac{1}{3} & {k=0} \\
      \frac{2}{3 }& {k=1} 
   \end{cases}
   \\
p_{Y|X}\brak{0|0} = \frac{19}{25}\, 
p_{Y|X}\brak{0|1} = \frac{6}{25}\,
p_{Y|X}\brak{1|0} = \frac{45}{50}\,
p_{Y|X}\brak{1|2} = \frac{5}{50}
\end{align}
The desired probability is the probability that a slip drawn at random is marked other than Rs 1,
\begin{align}
&=1-p_X\brak{0}\\
&= p_X(1) + p_X(2)
\end{align}
Using Bayes theorem,
\begin{align}
&= p_Y\brak{0} \times \pr{Y=0 | X=1} + p_Y\brak{1} \times \pr{Y=1|X=2}\\
&=\frac{1}{3} \times \frac{6}{25} + \frac{2}{3} \times \frac{5}{50}\\
&=\frac{11}{75}
\end{align}

\newpage

%\tableofcontents

\bigskip

\renewcommand{\thefigure}{\theenumi}
\renewcommand{\thetable}{\theenumi}
%\renewcommand{\theequation}{\theenumi}

%\begin{abstract}
%%\boldmath
%In this letter, an algorithm for evaluating the exact analytical bit error rate  (BER)  for the piecewise linear (PL) combiner for  multiple relays is presented. Previous results were available only for upto three relays. The algorithm is unique in the sense that  the actual mathematical expressions, that are prohibitively large, need not be explicitly obtained. The diversity gain due to multiple relays is shown through plots of the analytical BER, well supported by simulations. 
%
%\end{abstract}
% IEEEtran.cls defaults to using nonbold math in the Abstract.
% This preserves the distinction between vectors and scalars. However,
% if the journal you are submitting to favors bold math in the abstract,
% then you can use LaTeX's standard command \boldmath at the very start
% of the abstract to achieve this. Many IEEE journals frown on math
% in the abstract anyway.

% Note that keywords are not normally used for peerreview papers.
%\begin{IEEEkeywords}
%Cooperative diversity, decode and forward, piecewise linear
%\end{IEEEkeywords}



% For peer review papers, you can put extra information on the cover
% page as needed:
% \ifCLASSOPTIONpeerreview
% \begin{center} \bfseries EDICS Category: 3-BBND \end{center}
% \fi
%
% For peerreview papers, this IEEEtran command inserts a page break and
% creates the second title. It will be ignored for other modes.
%\IEEEpeerreviewmaketitle




\item
The king, queen and jack of clubs are removed from a deck of 52 playing cards and then well shuffled. Now one card is drawn at random from the remaining cards.  Determine the probability that the card is
\begin{enumerate}[label=(\roman*)]
\item a club
\item 10 of hearts
\end{enumerate}
\solution
%\begin{table}[H]
	\centering
\begin{tabular}{|c|c|c|}
\hline
Random variable &Value &Definition\\ \hline
\multirow{3}{*}{X} &0 &Slips of Rs 1\\
&1 &Slips of Rs 5\\
&2 &Slips of Rs 13\\ \hline
\multirow{2}{*}{Y} &0 &Box A\\
&1 &Box B\\\hline
\end{tabular}
\caption{}
\label{tab:Distribution}
\end{table}
See \tabref{tab:Distribution}.
\begin{align}
p_{Y}\brak{k}= \begin{cases} 
      \frac{1}{3} & {k=0} \\
      \frac{2}{3 }& {k=1} 
   \end{cases}
   \\
p_{Y|X}\brak{0|0} = \frac{19}{25}\, 
p_{Y|X}\brak{0|1} = \frac{6}{25}\,
p_{Y|X}\brak{1|0} = \frac{45}{50}\,
p_{Y|X}\brak{1|2} = \frac{5}{50}
\end{align}
The desired probability is the probability that a slip drawn at random is marked other than Rs 1,
\begin{align}
&=1-p_X\brak{0}\\
&= p_X(1) + p_X(2)
\end{align}
Using Bayes theorem,
\begin{align}
&= p_Y\brak{0} \times \pr{Y=0 | X=1} + p_Y\brak{1} \times \pr{Y=1|X=2}\\
&=\frac{1}{3} \times \frac{6}{25} + \frac{2}{3} \times \frac{5}{50}\\
&=\frac{11}{75}
\end{align}

\newpage

%\tableofcontents

\bigskip

\renewcommand{\thefigure}{\theenumi}
\renewcommand{\thetable}{\theenumi}
%\renewcommand{\theequation}{\theenumi}

%\begin{abstract}
%%\boldmath
%In this letter, an algorithm for evaluating the exact analytical bit error rate  (BER)  for the piecewise linear (PL) combiner for  multiple relays is presented. Previous results were available only for upto three relays. The algorithm is unique in the sense that  the actual mathematical expressions, that are prohibitively large, need not be explicitly obtained. The diversity gain due to multiple relays is shown through plots of the analytical BER, well supported by simulations. 
%
%\end{abstract}
% IEEEtran.cls defaults to using nonbold math in the Abstract.
% This preserves the distinction between vectors and scalars. However,
% if the journal you are submitting to favors bold math in the abstract,
% then you can use LaTeX's standard command \boldmath at the very start
% of the abstract to achieve this. Many IEEE journals frown on math
% in the abstract anyway.

% Note that keywords are not normally used for peerreview papers.
%\begin{IEEEkeywords}
%Cooperative diversity, decode and forward, piecewise linear
%\end{IEEEkeywords}



% For peer review papers, you can put extra information on the cover
% page as needed:
% \ifCLASSOPTIONpeerreview
% \begin{center} \bfseries EDICS Category: 3-BBND \end{center}
% \fi
%
% For peerreview papers, this IEEEtran command inserts a page break and
% creates the second title. It will be ignored for other modes.
%\IEEEpeerreviewmaketitle




\item A team of medical students doing their internship have to assist during surgeries
at a city hospital. The probabilities of surgeries rated as very complex, complex,
routine, simple or very simple are respectively, 0.15, 0.20, 0.31, 0.26, .08. Find
the probabilities that a particular surgery will be rated
\begin{enumerate}
	\item complex or very complex;
	\item neither very complex nor very simple;
	\item routine or complex
	\item routine or simple
\end{enumerate}
\solution
%\begin{table}[H]
	\centering
\begin{tabular}{|c|c|c|}
\hline
Random variable &Value &Definition\\ \hline
\multirow{3}{*}{X} &0 &Slips of Rs 1\\
&1 &Slips of Rs 5\\
&2 &Slips of Rs 13\\ \hline
\multirow{2}{*}{Y} &0 &Box A\\
&1 &Box B\\\hline
\end{tabular}
\caption{}
\label{tab:Distribution}
\end{table}
See \tabref{tab:Distribution}.
\begin{align}
p_{Y}\brak{k}= \begin{cases} 
      \frac{1}{3} & {k=0} \\
      \frac{2}{3 }& {k=1} 
   \end{cases}
   \\
p_{Y|X}\brak{0|0} = \frac{19}{25}\, 
p_{Y|X}\brak{0|1} = \frac{6}{25}\,
p_{Y|X}\brak{1|0} = \frac{45}{50}\,
p_{Y|X}\brak{1|2} = \frac{5}{50}
\end{align}
The desired probability is the probability that a slip drawn at random is marked other than Rs 1,
\begin{align}
&=1-p_X\brak{0}\\
&= p_X(1) + p_X(2)
\end{align}
Using Bayes theorem,
\begin{align}
&= p_Y\brak{0} \times \pr{Y=0 | X=1} + p_Y\brak{1} \times \pr{Y=1|X=2}\\
&=\frac{1}{3} \times \frac{6}{25} + \frac{2}{3} \times \frac{5}{50}\\
&=\frac{11}{75}
\end{align}

\newpage

%\tableofcontents

\bigskip

\renewcommand{\thefigure}{\theenumi}
\renewcommand{\thetable}{\theenumi}
%\renewcommand{\theequation}{\theenumi}

%\begin{abstract}
%%\boldmath
%In this letter, an algorithm for evaluating the exact analytical bit error rate  (BER)  for the piecewise linear (PL) combiner for  multiple relays is presented. Previous results were available only for upto three relays. The algorithm is unique in the sense that  the actual mathematical expressions, that are prohibitively large, need not be explicitly obtained. The diversity gain due to multiple relays is shown through plots of the analytical BER, well supported by simulations. 
%
%\end{abstract}
% IEEEtran.cls defaults to using nonbold math in the Abstract.
% This preserves the distinction between vectors and scalars. However,
% if the journal you are submitting to favors bold math in the abstract,
% then you can use LaTeX's standard command \boldmath at the very start
% of the abstract to achieve this. Many IEEE journals frown on math
% in the abstract anyway.

% Note that keywords are not normally used for peerreview papers.
%\begin{IEEEkeywords}
%Cooperative diversity, decode and forward, piecewise linear
%\end{IEEEkeywords}



% For peer review papers, you can put extra information on the cover
% page as needed:
% \ifCLASSOPTIONpeerreview
% \begin{center} \bfseries EDICS Category: 3-BBND \end{center}
% \fi
%
% For peerreview papers, this IEEEtran command inserts a page break and
% creates the second title. It will be ignored for other modes.
%\IEEEpeerreviewmaketitle




\item A card is selected from a pack of 52 cards.
\begin{enumerate}[label=(\alph*)]
    \item How many points are there in the sample space?
    \item Calculate the probability that the card is an ace of spades.
    \item Calculate the probability that the card is (i) an ace and (ii) black card.
\end{enumerate}
\solution
%Let $X$ be an bernoulli rv defined as in \tabref{tab:exemplar/11/16/3/26}.  Then, 
\begin{equation}
    p =
        \frac{4}{11} 
\end{equation}
\begin{table}[H]
	\centering
	\input{exemplar/11/16/3/26/tables/Table2.tex}
	\caption{}
        \label{tab:exemplar/11/16/3/26}
\end{table}

\item The probability that a non leap year selected at random will contain 53 sundays.
\\
\solution
%\begin{table}[H]
	\centering
\begin{tabular}{|c|c|c|}
\hline
Random variable &Value &Definition\\ \hline
\multirow{3}{*}{X} &0 &Slips of Rs 1\\
&1 &Slips of Rs 5\\
&2 &Slips of Rs 13\\ \hline
\multirow{2}{*}{Y} &0 &Box A\\
&1 &Box B\\\hline
\end{tabular}
\caption{}
\label{tab:Distribution}
\end{table}
See \tabref{tab:Distribution}.
\begin{align}
p_{Y}\brak{k}= \begin{cases} 
      \frac{1}{3} & {k=0} \\
      \frac{2}{3 }& {k=1} 
   \end{cases}
   \\
p_{Y|X}\brak{0|0} = \frac{19}{25}\, 
p_{Y|X}\brak{0|1} = \frac{6}{25}\,
p_{Y|X}\brak{1|0} = \frac{45}{50}\,
p_{Y|X}\brak{1|2} = \frac{5}{50}
\end{align}
The desired probability is the probability that a slip drawn at random is marked other than Rs 1,
\begin{align}
&=1-p_X\brak{0}\\
&= p_X(1) + p_X(2)
\end{align}
Using Bayes theorem,
\begin{align}
&= p_Y\brak{0} \times \pr{Y=0 | X=1} + p_Y\brak{1} \times \pr{Y=1|X=2}\\
&=\frac{1}{3} \times \frac{6}{25} + \frac{2}{3} \times \frac{5}{50}\\
&=\frac{11}{75}
\end{align}

\newpage

%\tableofcontents

\bigskip

\renewcommand{\thefigure}{\theenumi}
\renewcommand{\thetable}{\theenumi}
%\renewcommand{\theequation}{\theenumi}

%\begin{abstract}
%%\boldmath
%In this letter, an algorithm for evaluating the exact analytical bit error rate  (BER)  for the piecewise linear (PL) combiner for  multiple relays is presented. Previous results were available only for upto three relays. The algorithm is unique in the sense that  the actual mathematical expressions, that are prohibitively large, need not be explicitly obtained. The diversity gain due to multiple relays is shown through plots of the analytical BER, well supported by simulations. 
%
%\end{abstract}
% IEEEtran.cls defaults to using nonbold math in the Abstract.
% This preserves the distinction between vectors and scalars. However,
% if the journal you are submitting to favors bold math in the abstract,
% then you can use LaTeX's standard command \boldmath at the very start
% of the abstract to achieve this. Many IEEE journals frown on math
% in the abstract anyway.

% Note that keywords are not normally used for peerreview papers.
%\begin{IEEEkeywords}
%Cooperative diversity, decode and forward, piecewise linear
%\end{IEEEkeywords}



% For peer review papers, you can put extra information on the cover
% page as needed:
% \ifCLASSOPTIONpeerreview
% \begin{center} \bfseries EDICS Category: 3-BBND \end{center}
% \fi
%
% For peerreview papers, this IEEEtran command inserts a page break and
% creates the second title. It will be ignored for other modes.
%\IEEEpeerreviewmaketitle




\item One of the four persons John, Rita, Aslam or Gurpreet will be promoted next
month. Consequently the sample space consists of four elementary outcomes
S = {John promoted, Rita promoted, Aslam promoted, Gurpreet promoted}
You are told that the chances of John’s promotion is same as that of Gurpreet,
Rita’s chances of promotion are twice as likely as Johns. Aslam’s chances are
four times that of John.
\begin{enumerate}
	\item Determine
	\begin{enumerate}
		\item P (John promoted)
		\item P (Rita promoted)
		\item P (Aslam promoted)
		\item P (Gurpreet promoted)
	\end{enumerate}
	\item If A = {John promoted or Gurpreet promoted}, find P (A).
\end{enumerate}
\solution
%\begin{table}[H]
	\centering
\begin{tabular}{|c|c|c|}
\hline
Random variable &Value &Definition\\ \hline
\multirow{3}{*}{X} &0 &Slips of Rs 1\\
&1 &Slips of Rs 5\\
&2 &Slips of Rs 13\\ \hline
\multirow{2}{*}{Y} &0 &Box A\\
&1 &Box B\\\hline
\end{tabular}
\caption{}
\label{tab:Distribution}
\end{table}
See \tabref{tab:Distribution}.
\begin{align}
p_{Y}\brak{k}= \begin{cases} 
      \frac{1}{3} & {k=0} \\
      \frac{2}{3 }& {k=1} 
   \end{cases}
   \\
p_{Y|X}\brak{0|0} = \frac{19}{25}\, 
p_{Y|X}\brak{0|1} = \frac{6}{25}\,
p_{Y|X}\brak{1|0} = \frac{45}{50}\,
p_{Y|X}\brak{1|2} = \frac{5}{50}
\end{align}
The desired probability is the probability that a slip drawn at random is marked other than Rs 1,
\begin{align}
&=1-p_X\brak{0}\\
&= p_X(1) + p_X(2)
\end{align}
Using Bayes theorem,
\begin{align}
&= p_Y\brak{0} \times \pr{Y=0 | X=1} + p_Y\brak{1} \times \pr{Y=1|X=2}\\
&=\frac{1}{3} \times \frac{6}{25} + \frac{2}{3} \times \frac{5}{50}\\
&=\frac{11}{75}
\end{align}

\newpage

%\tableofcontents

\bigskip

\renewcommand{\thefigure}{\theenumi}
\renewcommand{\thetable}{\theenumi}
%\renewcommand{\theequation}{\theenumi}

%\begin{abstract}
%%\boldmath
%In this letter, an algorithm for evaluating the exact analytical bit error rate  (BER)  for the piecewise linear (PL) combiner for  multiple relays is presented. Previous results were available only for upto three relays. The algorithm is unique in the sense that  the actual mathematical expressions, that are prohibitively large, need not be explicitly obtained. The diversity gain due to multiple relays is shown through plots of the analytical BER, well supported by simulations. 
%
%\end{abstract}
% IEEEtran.cls defaults to using nonbold math in the Abstract.
% This preserves the distinction between vectors and scalars. However,
% if the journal you are submitting to favors bold math in the abstract,
% then you can use LaTeX's standard command \boldmath at the very start
% of the abstract to achieve this. Many IEEE journals frown on math
% in the abstract anyway.

% Note that keywords are not normally used for peerreview papers.
%\begin{IEEEkeywords}
%Cooperative diversity, decode and forward, piecewise linear
%\end{IEEEkeywords}



% For peer review papers, you can put extra information on the cover
% page as needed:
% \ifCLASSOPTIONpeerreview
% \begin{center} \bfseries EDICS Category: 3-BBND \end{center}
% \fi
%
% For peerreview papers, this IEEEtran command inserts a page break and
% creates the second title. It will be ignored for other modes.
%\IEEEpeerreviewmaketitle




\item A card is drawn from a deck of 52 cards. Find the probability of getting a king or a heart or a red card.\\
\solution
%\begin{table}[H]
	\centering
\begin{tabular}{|c|c|c|}
\hline
Random variable &Value &Definition\\ \hline
\multirow{3}{*}{X} &0 &Slips of Rs 1\\
&1 &Slips of Rs 5\\
&2 &Slips of Rs 13\\ \hline
\multirow{2}{*}{Y} &0 &Box A\\
&1 &Box B\\\hline
\end{tabular}
\caption{}
\label{tab:Distribution}
\end{table}
See \tabref{tab:Distribution}.
\begin{align}
p_{Y}\brak{k}= \begin{cases} 
      \frac{1}{3} & {k=0} \\
      \frac{2}{3 }& {k=1} 
   \end{cases}
   \\
p_{Y|X}\brak{0|0} = \frac{19}{25}\, 
p_{Y|X}\brak{0|1} = \frac{6}{25}\,
p_{Y|X}\brak{1|0} = \frac{45}{50}\,
p_{Y|X}\brak{1|2} = \frac{5}{50}
\end{align}
The desired probability is the probability that a slip drawn at random is marked other than Rs 1,
\begin{align}
&=1-p_X\brak{0}\\
&= p_X(1) + p_X(2)
\end{align}
Using Bayes theorem,
\begin{align}
&= p_Y\brak{0} \times \pr{Y=0 | X=1} + p_Y\brak{1} \times \pr{Y=1|X=2}\\
&=\frac{1}{3} \times \frac{6}{25} + \frac{2}{3} \times \frac{5}{50}\\
&=\frac{11}{75}
\end{align}

\newpage

%\tableofcontents

\bigskip

\renewcommand{\thefigure}{\theenumi}
\renewcommand{\thetable}{\theenumi}
%\renewcommand{\theequation}{\theenumi}

%\begin{abstract}
%%\boldmath
%In this letter, an algorithm for evaluating the exact analytical bit error rate  (BER)  for the piecewise linear (PL) combiner for  multiple relays is presented. Previous results were available only for upto three relays. The algorithm is unique in the sense that  the actual mathematical expressions, that are prohibitively large, need not be explicitly obtained. The diversity gain due to multiple relays is shown through plots of the analytical BER, well supported by simulations. 
%
%\end{abstract}
% IEEEtran.cls defaults to using nonbold math in the Abstract.
% This preserves the distinction between vectors and scalars. However,
% if the journal you are submitting to favors bold math in the abstract,
% then you can use LaTeX's standard command \boldmath at the very start
% of the abstract to achieve this. Many IEEE journals frown on math
% in the abstract anyway.

% Note that keywords are not normally used for peerreview papers.
%\begin{IEEEkeywords}
%Cooperative diversity, decode and forward, piecewise linear
%\end{IEEEkeywords}



% For peer review papers, you can put extra information on the cover
% page as needed:
% \ifCLASSOPTIONpeerreview
% \begin{center} \bfseries EDICS Category: 3-BBND \end{center}
% \fi
%
% For peerreview papers, this IEEEtran command inserts a page break and
% creates the second title. It will be ignored for other modes.
%\IEEEpeerreviewmaketitle




\item The probability that a student will pass his examination is 0.73, the probability of
the student getting a compartment is 0.13, and the probability that the student will
either pass or get compartment is 0.96. State True or False.\\
\solution
%\begin{table}[H]
	\centering
\begin{tabular}{|c|c|c|}
\hline
Random variable &Value &Definition\\ \hline
\multirow{3}{*}{X} &0 &Slips of Rs 1\\
&1 &Slips of Rs 5\\
&2 &Slips of Rs 13\\ \hline
\multirow{2}{*}{Y} &0 &Box A\\
&1 &Box B\\\hline
\end{tabular}
\caption{}
\label{tab:Distribution}
\end{table}
See \tabref{tab:Distribution}.
\begin{align}
p_{Y}\brak{k}= \begin{cases} 
      \frac{1}{3} & {k=0} \\
      \frac{2}{3 }& {k=1} 
   \end{cases}
   \\
p_{Y|X}\brak{0|0} = \frac{19}{25}\, 
p_{Y|X}\brak{0|1} = \frac{6}{25}\,
p_{Y|X}\brak{1|0} = \frac{45}{50}\,
p_{Y|X}\brak{1|2} = \frac{5}{50}
\end{align}
The desired probability is the probability that a slip drawn at random is marked other than Rs 1,
\begin{align}
&=1-p_X\brak{0}\\
&= p_X(1) + p_X(2)
\end{align}
Using Bayes theorem,
\begin{align}
&= p_Y\brak{0} \times \pr{Y=0 | X=1} + p_Y\brak{1} \times \pr{Y=1|X=2}\\
&=\frac{1}{3} \times \frac{6}{25} + \frac{2}{3} \times \frac{5}{50}\\
&=\frac{11}{75}
\end{align}

\newpage

%\tableofcontents

\bigskip

\renewcommand{\thefigure}{\theenumi}
\renewcommand{\thetable}{\theenumi}
%\renewcommand{\theequation}{\theenumi}

%\begin{abstract}
%%\boldmath
%In this letter, an algorithm for evaluating the exact analytical bit error rate  (BER)  for the piecewise linear (PL) combiner for  multiple relays is presented. Previous results were available only for upto three relays. The algorithm is unique in the sense that  the actual mathematical expressions, that are prohibitively large, need not be explicitly obtained. The diversity gain due to multiple relays is shown through plots of the analytical BER, well supported by simulations. 
%
%\end{abstract}
% IEEEtran.cls defaults to using nonbold math in the Abstract.
% This preserves the distinction between vectors and scalars. However,
% if the journal you are submitting to favors bold math in the abstract,
% then you can use LaTeX's standard command \boldmath at the very start
% of the abstract to achieve this. Many IEEE journals frown on math
% in the abstract anyway.

% Note that keywords are not normally used for peerreview papers.
%\begin{IEEEkeywords}
%Cooperative diversity, decode and forward, piecewise linear
%\end{IEEEkeywords}



% For peer review papers, you can put extra information on the cover
% page as needed:
% \ifCLASSOPTIONpeerreview
% \begin{center} \bfseries EDICS Category: 3-BBND \end{center}
% \fi
%
% For peerreview papers, this IEEEtran command inserts a page break and
% creates the second title. It will be ignored for other modes.
%\IEEEpeerreviewmaketitle




\item A card is selected from a pack of 52 cards\\
\begin{enumerate}[label=(\alph*)]
\item How many points are there in the sample space?
\item Calculate the probability that the cards is an ace of spades.
\item Calculate the probability that the card is (i) an ace (ii)black card.\\
\end{enumerate}
%\input{ncert/11/16/3/4_1/Prob_4.tex}
\item In a non-leap year, the probability of having 53 tuesdays or 53 wednesdays is\\
\solution
%A non-leap year has a total of 365 days, and a week has 7 days.\\
So it can be expressed as 
\begin{align}
365\text{days} &=52\times 7+1 \text{day}
\end{align}
$\implies$ 52 tuesdays or wednesdays\\
Random variable X denotes the days of a week
\begin{align}
p_X\brak{k}&=\frac{1}{7}; \quad \brak{1<k<7}
\end{align}
So the probability of extra day being tuesday or wednesday is
\begin{align}
p_X\brak{3}+p_X\brak{4}&=\frac{1}{7}+\frac{1}{7}=\frac{2}{7}
\end{align}



\item There are 1000 sealed envelopes in a box, 10 of them contain a cash prize of
Rs 100 each, 100 of them contain a cash prize of Rs 50 each and 200 of them
contain a cash prize of Rs 10 each and rest do not contain any cash prize. If they
are well shuffled and an envelope is picked up out, what is the probability that it
contains no cash prize?\\
\solution
%\begin{table}[H]
	\centering
\begin{tabular}{|c|c|c|}
\hline
Random variable &Value &Definition\\ \hline
\multirow{3}{*}{X} &0 &Slips of Rs 1\\
&1 &Slips of Rs 5\\
&2 &Slips of Rs 13\\ \hline
\multirow{2}{*}{Y} &0 &Box A\\
&1 &Box B\\\hline
\end{tabular}
\caption{}
\label{tab:Distribution}
\end{table}
See \tabref{tab:Distribution}.
\begin{align}
p_{Y}\brak{k}= \begin{cases} 
      \frac{1}{3} & {k=0} \\
      \frac{2}{3 }& {k=1} 
   \end{cases}
   \\
p_{Y|X}\brak{0|0} = \frac{19}{25}\, 
p_{Y|X}\brak{0|1} = \frac{6}{25}\,
p_{Y|X}\brak{1|0} = \frac{45}{50}\,
p_{Y|X}\brak{1|2} = \frac{5}{50}
\end{align}
The desired probability is the probability that a slip drawn at random is marked other than Rs 1,
\begin{align}
&=1-p_X\brak{0}\\
&= p_X(1) + p_X(2)
\end{align}
Using Bayes theorem,
\begin{align}
&= p_Y\brak{0} \times \pr{Y=0 | X=1} + p_Y\brak{1} \times \pr{Y=1|X=2}\\
&=\frac{1}{3} \times \frac{6}{25} + \frac{2}{3} \times \frac{5}{50}\\
&=\frac{11}{75}
\end{align}

\newpage

%\tableofcontents

\bigskip

\renewcommand{\thefigure}{\theenumi}
\renewcommand{\thetable}{\theenumi}
%\renewcommand{\theequation}{\theenumi}

%\begin{abstract}
%%\boldmath
%In this letter, an algorithm for evaluating the exact analytical bit error rate  (BER)  for the piecewise linear (PL) combiner for  multiple relays is presented. Previous results were available only for upto three relays. The algorithm is unique in the sense that  the actual mathematical expressions, that are prohibitively large, need not be explicitly obtained. The diversity gain due to multiple relays is shown through plots of the analytical BER, well supported by simulations. 
%
%\end{abstract}
% IEEEtran.cls defaults to using nonbold math in the Abstract.
% This preserves the distinction between vectors and scalars. However,
% if the journal you are submitting to favors bold math in the abstract,
% then you can use LaTeX's standard command \boldmath at the very start
% of the abstract to achieve this. Many IEEE journals frown on math
% in the abstract anyway.

% Note that keywords are not normally used for peerreview papers.
%\begin{IEEEkeywords}
%Cooperative diversity, decode and forward, piecewise linear
%\end{IEEEkeywords}



% For peer review papers, you can put extra information on the cover
% page as needed:
% \ifCLASSOPTIONpeerreview
% \begin{center} \bfseries EDICS Category: 3-BBND \end{center}
% \fi
%
% For peerreview papers, this IEEEtran command inserts a page break and
% creates the second title. It will be ignored for other modes.
%\IEEEpeerreviewmaketitle




\item 
A die is thrown and a card is selected at random from a deck of 52 playing cards. The probability of getting an even number on the die and a spade card.\\
\solution
%\begin{table}[H]
	\centering
\begin{tabular}{|c|c|c|}
\hline
Random variable &Value &Definition\\ \hline
\multirow{3}{*}{X} &0 &Slips of Rs 1\\
&1 &Slips of Rs 5\\
&2 &Slips of Rs 13\\ \hline
\multirow{2}{*}{Y} &0 &Box A\\
&1 &Box B\\\hline
\end{tabular}
\caption{}
\label{tab:Distribution}
\end{table}
See \tabref{tab:Distribution}.
\begin{align}
p_{Y}\brak{k}= \begin{cases} 
      \frac{1}{3} & {k=0} \\
      \frac{2}{3 }& {k=1} 
   \end{cases}
   \\
p_{Y|X}\brak{0|0} = \frac{19}{25}\, 
p_{Y|X}\brak{0|1} = \frac{6}{25}\,
p_{Y|X}\brak{1|0} = \frac{45}{50}\,
p_{Y|X}\brak{1|2} = \frac{5}{50}
\end{align}
The desired probability is the probability that a slip drawn at random is marked other than Rs 1,
\begin{align}
&=1-p_X\brak{0}\\
&= p_X(1) + p_X(2)
\end{align}
Using Bayes theorem,
\begin{align}
&= p_Y\brak{0} \times \pr{Y=0 | X=1} + p_Y\brak{1} \times \pr{Y=1|X=2}\\
&=\frac{1}{3} \times \frac{6}{25} + \frac{2}{3} \times \frac{5}{50}\\
&=\frac{11}{75}
\end{align}

\newpage

%\tableofcontents

\bigskip

\renewcommand{\thefigure}{\theenumi}
\renewcommand{\thetable}{\theenumi}
%\renewcommand{\theequation}{\theenumi}

%\begin{abstract}
%%\boldmath
%In this letter, an algorithm for evaluating the exact analytical bit error rate  (BER)  for the piecewise linear (PL) combiner for  multiple relays is presented. Previous results were available only for upto three relays. The algorithm is unique in the sense that  the actual mathematical expressions, that are prohibitively large, need not be explicitly obtained. The diversity gain due to multiple relays is shown through plots of the analytical BER, well supported by simulations. 
%
%\end{abstract}
% IEEEtran.cls defaults to using nonbold math in the Abstract.
% This preserves the distinction between vectors and scalars. However,
% if the journal you are submitting to favors bold math in the abstract,
% then you can use LaTeX's standard command \boldmath at the very start
% of the abstract to achieve this. Many IEEE journals frown on math
% in the abstract anyway.

% Note that keywords are not normally used for peerreview papers.
%\begin{IEEEkeywords}
%Cooperative diversity, decode and forward, piecewise linear
%\end{IEEEkeywords}



% For peer review papers, you can put extra information on the cover
% page as needed:
% \ifCLASSOPTIONpeerreview
% \begin{center} \bfseries EDICS Category: 3-BBND \end{center}
% \fi
%
% For peerreview papers, this IEEEtran command inserts a page break and
% creates the second title. It will be ignored for other modes.
%\IEEEpeerreviewmaketitle




\item
If 4-digit numbers greater than 5,000 are randomly formed from the digits 0, 1, 3, 5, and 7, what is the probability of forming a number divisible by 5 when:
\begin{enumerate}
    \item The digits are repeated?
    \item The repetition of digits is not allowed?
\end{enumerate}
\solution
%\begin{table}[H]
	\centering
\begin{tabular}{|c|c|c|}
\hline
Random variable &Value &Definition\\ \hline
\multirow{3}{*}{X} &0 &Slips of Rs 1\\
&1 &Slips of Rs 5\\
&2 &Slips of Rs 13\\ \hline
\multirow{2}{*}{Y} &0 &Box A\\
&1 &Box B\\\hline
\end{tabular}
\caption{}
\label{tab:Distribution}
\end{table}
See \tabref{tab:Distribution}.
\begin{align}
p_{Y}\brak{k}= \begin{cases} 
      \frac{1}{3} & {k=0} \\
      \frac{2}{3 }& {k=1} 
   \end{cases}
   \\
p_{Y|X}\brak{0|0} = \frac{19}{25}\, 
p_{Y|X}\brak{0|1} = \frac{6}{25}\,
p_{Y|X}\brak{1|0} = \frac{45}{50}\,
p_{Y|X}\brak{1|2} = \frac{5}{50}
\end{align}
The desired probability is the probability that a slip drawn at random is marked other than Rs 1,
\begin{align}
&=1-p_X\brak{0}\\
&= p_X(1) + p_X(2)
\end{align}
Using Bayes theorem,
\begin{align}
&= p_Y\brak{0} \times \pr{Y=0 | X=1} + p_Y\brak{1} \times \pr{Y=1|X=2}\\
&=\frac{1}{3} \times \frac{6}{25} + \frac{2}{3} \times \frac{5}{50}\\
&=\frac{11}{75}
\end{align}

\newpage

%\tableofcontents

\bigskip

\renewcommand{\thefigure}{\theenumi}
\renewcommand{\thetable}{\theenumi}
%\renewcommand{\theequation}{\theenumi}

%\begin{abstract}
%%\boldmath
%In this letter, an algorithm for evaluating the exact analytical bit error rate  (BER)  for the piecewise linear (PL) combiner for  multiple relays is presented. Previous results were available only for upto three relays. The algorithm is unique in the sense that  the actual mathematical expressions, that are prohibitively large, need not be explicitly obtained. The diversity gain due to multiple relays is shown through plots of the analytical BER, well supported by simulations. 
%
%\end{abstract}
% IEEEtran.cls defaults to using nonbold math in the Abstract.
% This preserves the distinction between vectors and scalars. However,
% if the journal you are submitting to favors bold math in the abstract,
% then you can use LaTeX's standard command \boldmath at the very start
% of the abstract to achieve this. Many IEEE journals frown on math
% in the abstract anyway.

% Note that keywords are not normally used for peerreview papers.
%\begin{IEEEkeywords}
%Cooperative diversity, decode and forward, piecewise linear
%\end{IEEEkeywords}



% For peer review papers, you can put extra information on the cover
% page as needed:
% \ifCLASSOPTIONpeerreview
% \begin{center} \bfseries EDICS Category: 3-BBND \end{center}
% \fi
%
% For peerreview papers, this IEEEtran command inserts a page break and
% creates the second title. It will be ignored for other modes.
%\IEEEpeerreviewmaketitle




\item Consider the probability space $\brak{\Omega, \mathcal{G}, P}$ where $\Omega = [0,2]$ and $\mathcal{G} = \cbrak{\phi, \Omega, [0,1], (1,2]}$. Let $X$ and $Y$ be two functions on $\Omega$ defined as
\begin{align*}
    X(\omega) = 
    \begin{cases}
        1 & \text{if }\omega \in [0, 1]\\
        2 & \text{if }\omega \in (1, 2]
    \end{cases}
\end{align*}
and
\begin{align*}
    Y(\omega) = 
    \begin{cases}
        2 & \text{if }\omega \in [0, 1.5]\\
        3 & \text{if }\omega \in (1.5, 2].
    \end{cases}
\end{align*}
Then which one of the following statements is true?
\begin{enumerate}
    \item [(A)] $X$ is a random variable with respect to $\mathcal{G}$, but $Y$ is not a random variable with respect to $\mathcal{G}$.
    \item [(B)] $Y$ is a random variable with respect to $\mathcal{G}$, but $X$ is not a random variable with respect to $\mathcal{G}$.
    \item [(C)] Neither $X$ nor $Y$ is a random variable with respect to $\mathcal{G}$.
    \item [(D)] Both $X$ and $Y$ are random variables with respect to $\mathcal{G}$.
\end{enumerate} \hfill (GATE ST 2023)\\
\solution
%\begin{table}[H]
	\centering
\begin{tabular}{|c|c|c|}
\hline
Random variable &Value &Definition\\ \hline
\multirow{3}{*}{X} &0 &Slips of Rs 1\\
&1 &Slips of Rs 5\\
&2 &Slips of Rs 13\\ \hline
\multirow{2}{*}{Y} &0 &Box A\\
&1 &Box B\\\hline
\end{tabular}
\caption{}
\label{tab:Distribution}
\end{table}
See \tabref{tab:Distribution}.
\begin{align}
p_{Y}\brak{k}= \begin{cases} 
      \frac{1}{3} & {k=0} \\
      \frac{2}{3 }& {k=1} 
   \end{cases}
   \\
p_{Y|X}\brak{0|0} = \frac{19}{25}\, 
p_{Y|X}\brak{0|1} = \frac{6}{25}\,
p_{Y|X}\brak{1|0} = \frac{45}{50}\,
p_{Y|X}\brak{1|2} = \frac{5}{50}
\end{align}
The desired probability is the probability that a slip drawn at random is marked other than Rs 1,
\begin{align}
&=1-p_X\brak{0}\\
&= p_X(1) + p_X(2)
\end{align}
Using Bayes theorem,
\begin{align}
&= p_Y\brak{0} \times \pr{Y=0 | X=1} + p_Y\brak{1} \times \pr{Y=1|X=2}\\
&=\frac{1}{3} \times \frac{6}{25} + \frac{2}{3} \times \frac{5}{50}\\
&=\frac{11}{75}
\end{align}

\newpage

%\tableofcontents

\bigskip

\renewcommand{\thefigure}{\theenumi}
\renewcommand{\thetable}{\theenumi}
%\renewcommand{\theequation}{\theenumi}

%\begin{abstract}
%%\boldmath
%In this letter, an algorithm for evaluating the exact analytical bit error rate  (BER)  for the piecewise linear (PL) combiner for  multiple relays is presented. Previous results were available only for upto three relays. The algorithm is unique in the sense that  the actual mathematical expressions, that are prohibitively large, need not be explicitly obtained. The diversity gain due to multiple relays is shown through plots of the analytical BER, well supported by simulations. 
%
%\end{abstract}
% IEEEtran.cls defaults to using nonbold math in the Abstract.
% This preserves the distinction between vectors and scalars. However,
% if the journal you are submitting to favors bold math in the abstract,
% then you can use LaTeX's standard command \boldmath at the very start
% of the abstract to achieve this. Many IEEE journals frown on math
% in the abstract anyway.

% Note that keywords are not normally used for peerreview papers.
%\begin{IEEEkeywords}
%Cooperative diversity, decode and forward, piecewise linear
%\end{IEEEkeywords}



% For peer review papers, you can put extra information on the cover
% page as needed:
% \ifCLASSOPTIONpeerreview
% \begin{center} \bfseries EDICS Category: 3-BBND \end{center}
% \fi
%
% For peerreview papers, this IEEEtran command inserts a page break and
% creates the second title. It will be ignored for other modes.
%\IEEEpeerreviewmaketitle




	\item  A die is loaded in such a way that each odd number is twice as likely to occur as
each even number. Find $P(G)$, where $G$ is the event that a number greater than
3 occurs on a single roll of the die.
\\
\solution
		%\begin{table}[H]
	\centering
\begin{tabular}{|c|c|c|}
\hline
Random variable &Value &Definition\\ \hline
\multirow{3}{*}{X} &0 &Slips of Rs 1\\
&1 &Slips of Rs 5\\
&2 &Slips of Rs 13\\ \hline
\multirow{2}{*}{Y} &0 &Box A\\
&1 &Box B\\\hline
\end{tabular}
\caption{}
\label{tab:Distribution}
\end{table}
See \tabref{tab:Distribution}.
\begin{align}
p_{Y}\brak{k}= \begin{cases} 
      \frac{1}{3} & {k=0} \\
      \frac{2}{3 }& {k=1} 
   \end{cases}
   \\
p_{Y|X}\brak{0|0} = \frac{19}{25}\, 
p_{Y|X}\brak{0|1} = \frac{6}{25}\,
p_{Y|X}\brak{1|0} = \frac{45}{50}\,
p_{Y|X}\brak{1|2} = \frac{5}{50}
\end{align}
The desired probability is the probability that a slip drawn at random is marked other than Rs 1,
\begin{align}
&=1-p_X\brak{0}\\
&= p_X(1) + p_X(2)
\end{align}
Using Bayes theorem,
\begin{align}
&= p_Y\brak{0} \times \pr{Y=0 | X=1} + p_Y\brak{1} \times \pr{Y=1|X=2}\\
&=\frac{1}{3} \times \frac{6}{25} + \frac{2}{3} \times \frac{5}{50}\\
&=\frac{11}{75}
\end{align}

\newpage

%\tableofcontents

\bigskip

\renewcommand{\thefigure}{\theenumi}
\renewcommand{\thetable}{\theenumi}
%\renewcommand{\theequation}{\theenumi}

%\begin{abstract}
%%\boldmath
%In this letter, an algorithm for evaluating the exact analytical bit error rate  (BER)  for the piecewise linear (PL) combiner for  multiple relays is presented. Previous results were available only for upto three relays. The algorithm is unique in the sense that  the actual mathematical expressions, that are prohibitively large, need not be explicitly obtained. The diversity gain due to multiple relays is shown through plots of the analytical BER, well supported by simulations. 
%
%\end{abstract}
% IEEEtran.cls defaults to using nonbold math in the Abstract.
% This preserves the distinction between vectors and scalars. However,
% if the journal you are submitting to favors bold math in the abstract,
% then you can use LaTeX's standard command \boldmath at the very start
% of the abstract to achieve this. Many IEEE journals frown on math
% in the abstract anyway.

% Note that keywords are not normally used for peerreview papers.
%\begin{IEEEkeywords}
%Cooperative diversity, decode and forward, piecewise linear
%\end{IEEEkeywords}



% For peer review papers, you can put extra information on the cover
% page as needed:
% \ifCLASSOPTIONpeerreview
% \begin{center} \bfseries EDICS Category: 3-BBND \end{center}
% \fi
%
% For peerreview papers, this IEEEtran command inserts a page break and
% creates the second title. It will be ignored for other modes.
%\IEEEpeerreviewmaketitle




	\item All the jacks, queens and kings are removed from a deck of 52 playing cards. The remaining cards are well shuffled and then one card is drawn at random. Giving ace a value 1 similar value for other cards, find the probability that the card has a value 
		\begin{enumerate}
			\item 7
			\item greater than 7
			\item less than 7
		\end{enumerate}
		%Number of cards left after removing all jacks, queens and kings 
\begin{align}
N	= 52 - 4\times 3
	= 40
\end{align}
%\begin{table}[H]
%\def\arraystretch{1.2}
%\begin{tabular}{|c|c|c|}
%\hline
%	\textbf{Parameter} &\textbf{Value} &\textbf{Description}\\ \hline
%	$X$ &1-10 &Represents the value of the card picked \\ \hline
%\end{tabular}
%\end{table}
Let $1 \le X \le 10$ be the value of the card picked.  Then,
\begin{align}
	p_X(k) &= \Pr(X=k)\ \forall\ 1 \leq k \leq 10\\
	&= \frac{4\times 1}{40}\\
	&= \frac{1}{10}\\
	\therefore p_X(k) &= 
	\begin{cases}
		\frac{1}{10} & 1 \leq k \leq 10\\
		0 & \text{otherwise}
	\end{cases}
\end{align}
and
\begin{align}
	F_{X}(k) &= \sum_{m=0}^{k}p_{X}(m) \quad 1 \leq k \leq 10\\
	&= \frac{k}{10}\\
	\therefore F_{X}(k) &= 
	\begin{cases}
		0 & k \leq 0\\
		\frac{k}{10} & 1\leq k \leq 10\\
		1 & k > 10 
	\end{cases}
\end{align}
\begin{enumerate}
	\item Probability that card has value equal to 7 is
		\begin{align}
			 p_{X}(7)
			= \frac{1}{10}
		\end{align}
	\item Probability that card has value greater than 7 is
		\begin{align}
			1 - F_X(7)
			&= 1 - \frac{7}{10}
			\\
			&= \frac{3}{10}
		\end{align}
	\item Probability that card has value less than 7 is
		\begin{align}
			 F_{X}(6)
			=\frac{6}{10}
		\end{align}
\end{enumerate}

  \item A Lot consists of 48 mobile phones of which 42 are good, 3 have only minor defects and 3 have major defects.Varnika will buy a phone if it is good but the trader will only buy a mobile if it has no major defects. One phone is selected at random from the lot. What is the probability that it is
\begin{enumerate}
	\item acceptable to Varnika?
            \item acceptable to the trader?
\end{enumerate}
\solution
	%\begin{table}[H]
	\centering
\begin{tabular}{|c|c|c|}
\hline
Random variable &Value &Definition\\ \hline
\multirow{3}{*}{X} &0 &Slips of Rs 1\\
&1 &Slips of Rs 5\\
&2 &Slips of Rs 13\\ \hline
\multirow{2}{*}{Y} &0 &Box A\\
&1 &Box B\\\hline
\end{tabular}
\caption{}
\label{tab:Distribution}
\end{table}
See \tabref{tab:Distribution}.
\begin{align}
p_{Y}\brak{k}= \begin{cases} 
      \frac{1}{3} & {k=0} \\
      \frac{2}{3 }& {k=1} 
   \end{cases}
   \\
p_{Y|X}\brak{0|0} = \frac{19}{25}\, 
p_{Y|X}\brak{0|1} = \frac{6}{25}\,
p_{Y|X}\brak{1|0} = \frac{45}{50}\,
p_{Y|X}\brak{1|2} = \frac{5}{50}
\end{align}
The desired probability is the probability that a slip drawn at random is marked other than Rs 1,
\begin{align}
&=1-p_X\brak{0}\\
&= p_X(1) + p_X(2)
\end{align}
Using Bayes theorem,
\begin{align}
&= p_Y\brak{0} \times \pr{Y=0 | X=1} + p_Y\brak{1} \times \pr{Y=1|X=2}\\
&=\frac{1}{3} \times \frac{6}{25} + \frac{2}{3} \times \frac{5}{50}\\
&=\frac{11}{75}
\end{align}

\newpage

%\tableofcontents

\bigskip

\renewcommand{\thefigure}{\theenumi}
\renewcommand{\thetable}{\theenumi}
%\renewcommand{\theequation}{\theenumi}

%\begin{abstract}
%%\boldmath
%In this letter, an algorithm for evaluating the exact analytical bit error rate  (BER)  for the piecewise linear (PL) combiner for  multiple relays is presented. Previous results were available only for upto three relays. The algorithm is unique in the sense that  the actual mathematical expressions, that are prohibitively large, need not be explicitly obtained. The diversity gain due to multiple relays is shown through plots of the analytical BER, well supported by simulations. 
%
%\end{abstract}
% IEEEtran.cls defaults to using nonbold math in the Abstract.
% This preserves the distinction between vectors and scalars. However,
% if the journal you are submitting to favors bold math in the abstract,
% then you can use LaTeX's standard command \boldmath at the very start
% of the abstract to achieve this. Many IEEE journals frown on math
% in the abstract anyway.

% Note that keywords are not normally used for peerreview papers.
%\begin{IEEEkeywords}
%Cooperative diversity, decode and forward, piecewise linear
%\end{IEEEkeywords}



% For peer review papers, you can put extra information on the cover
% page as needed:
% \ifCLASSOPTIONpeerreview
% \begin{center} \bfseries EDICS Category: 3-BBND \end{center}
% \fi
%
% For peerreview papers, this IEEEtran command inserts a page break and
% creates the second title. It will be ignored for other modes.
%\IEEEpeerreviewmaketitle




 \item A student says that if you throw a die, it will show up 1 or not 1. Therefore, the probability of getting 1 and the probability of getting 'not 1' each is equal to $\frac{1}{2}$. Is this correct? Give reasons.\\
 \solution
        %\begin{table}[H]
	\centering
\begin{tabular}{|c|c|c|}
\hline
Random variable &Value &Definition\\ \hline
\multirow{3}{*}{X} &0 &Slips of Rs 1\\
&1 &Slips of Rs 5\\
&2 &Slips of Rs 13\\ \hline
\multirow{2}{*}{Y} &0 &Box A\\
&1 &Box B\\\hline
\end{tabular}
\caption{}
\label{tab:Distribution}
\end{table}
See \tabref{tab:Distribution}.
\begin{align}
p_{Y}\brak{k}= \begin{cases} 
      \frac{1}{3} & {k=0} \\
      \frac{2}{3 }& {k=1} 
   \end{cases}
   \\
p_{Y|X}\brak{0|0} = \frac{19}{25}\, 
p_{Y|X}\brak{0|1} = \frac{6}{25}\,
p_{Y|X}\brak{1|0} = \frac{45}{50}\,
p_{Y|X}\brak{1|2} = \frac{5}{50}
\end{align}
The desired probability is the probability that a slip drawn at random is marked other than Rs 1,
\begin{align}
&=1-p_X\brak{0}\\
&= p_X(1) + p_X(2)
\end{align}
Using Bayes theorem,
\begin{align}
&= p_Y\brak{0} \times \pr{Y=0 | X=1} + p_Y\brak{1} \times \pr{Y=1|X=2}\\
&=\frac{1}{3} \times \frac{6}{25} + \frac{2}{3} \times \frac{5}{50}\\
&=\frac{11}{75}
\end{align}

\newpage

%\tableofcontents

\bigskip

\renewcommand{\thefigure}{\theenumi}
\renewcommand{\thetable}{\theenumi}
%\renewcommand{\theequation}{\theenumi}

%\begin{abstract}
%%\boldmath
%In this letter, an algorithm for evaluating the exact analytical bit error rate  (BER)  for the piecewise linear (PL) combiner for  multiple relays is presented. Previous results were available only for upto three relays. The algorithm is unique in the sense that  the actual mathematical expressions, that are prohibitively large, need not be explicitly obtained. The diversity gain due to multiple relays is shown through plots of the analytical BER, well supported by simulations. 
%
%\end{abstract}
% IEEEtran.cls defaults to using nonbold math in the Abstract.
% This preserves the distinction between vectors and scalars. However,
% if the journal you are submitting to favors bold math in the abstract,
% then you can use LaTeX's standard command \boldmath at the very start
% of the abstract to achieve this. Many IEEE journals frown on math
% in the abstract anyway.

% Note that keywords are not normally used for peerreview papers.
%\begin{IEEEkeywords}
%Cooperative diversity, decode and forward, piecewise linear
%\end{IEEEkeywords}



% For peer review papers, you can put extra information on the cover
% page as needed:
% \ifCLASSOPTIONpeerreview
% \begin{center} \bfseries EDICS Category: 3-BBND \end{center}
% \fi
%
% For peerreview papers, this IEEEtran command inserts a page break and
% creates the second title. It will be ignored for other modes.
%\IEEEpeerreviewmaketitle




   \item Four candidates A, B, C, D have ap-
plied for the assignment to coach a school cricket
team. If A is twice as likely to be selected as B, and
B and C are given about the same chance of being
selected, while C is twice as likely to be selected
as D, what are the probabilities that
\begin{enumerate}
\item C will be selected?
\item A will not be selected?
\end{enumerate}
	%\begin{table}[H]
	\centering
\begin{tabular}{|c|c|c|}
\hline
Random variable &Value &Definition\\ \hline
\multirow{3}{*}{X} &0 &Slips of Rs 1\\
&1 &Slips of Rs 5\\
&2 &Slips of Rs 13\\ \hline
\multirow{2}{*}{Y} &0 &Box A\\
&1 &Box B\\\hline
\end{tabular}
\caption{}
\label{tab:Distribution}
\end{table}
See \tabref{tab:Distribution}.
\begin{align}
p_{Y}\brak{k}= \begin{cases} 
      \frac{1}{3} & {k=0} \\
      \frac{2}{3 }& {k=1} 
   \end{cases}
   \\
p_{Y|X}\brak{0|0} = \frac{19}{25}\, 
p_{Y|X}\brak{0|1} = \frac{6}{25}\,
p_{Y|X}\brak{1|0} = \frac{45}{50}\,
p_{Y|X}\brak{1|2} = \frac{5}{50}
\end{align}
The desired probability is the probability that a slip drawn at random is marked other than Rs 1,
\begin{align}
&=1-p_X\brak{0}\\
&= p_X(1) + p_X(2)
\end{align}
Using Bayes theorem,
\begin{align}
&= p_Y\brak{0} \times \pr{Y=0 | X=1} + p_Y\brak{1} \times \pr{Y=1|X=2}\\
&=\frac{1}{3} \times \frac{6}{25} + \frac{2}{3} \times \frac{5}{50}\\
&=\frac{11}{75}
\end{align}

\newpage

%\tableofcontents

\bigskip

\renewcommand{\thefigure}{\theenumi}
\renewcommand{\thetable}{\theenumi}
%\renewcommand{\theequation}{\theenumi}

%\begin{abstract}
%%\boldmath
%In this letter, an algorithm for evaluating the exact analytical bit error rate  (BER)  for the piecewise linear (PL) combiner for  multiple relays is presented. Previous results were available only for upto three relays. The algorithm is unique in the sense that  the actual mathematical expressions, that are prohibitively large, need not be explicitly obtained. The diversity gain due to multiple relays is shown through plots of the analytical BER, well supported by simulations. 
%
%\end{abstract}
% IEEEtran.cls defaults to using nonbold math in the Abstract.
% This preserves the distinction between vectors and scalars. However,
% if the journal you are submitting to favors bold math in the abstract,
% then you can use LaTeX's standard command \boldmath at the very start
% of the abstract to achieve this. Many IEEE journals frown on math
% in the abstract anyway.

% Note that keywords are not normally used for peerreview papers.
%\begin{IEEEkeywords}
%Cooperative diversity, decode and forward, piecewise linear
%\end{IEEEkeywords}



% For peer review papers, you can put extra information on the cover
% page as needed:
% \ifCLASSOPTIONpeerreview
% \begin{center} \bfseries EDICS Category: 3-BBND \end{center}
% \fi
%
% For peerreview papers, this IEEEtran command inserts a page break and
% creates the second title. It will be ignored for other modes.
%\IEEEpeerreviewmaketitle




 \item A bag contain 24 balls of which $x$ balls are red, $2x$ are white and $3x$ are blue. A ball is selected at random, What is the probability that it is
\begin{enumerate}[label=\alph*)]
\item not red ?
\item white ?
\end{enumerate}
%\begin{table}[H]
	\centering
\begin{tabular}{|c|c|c|}
\hline
Random variable &Value &Definition\\ \hline
\multirow{3}{*}{X} &0 &Slips of Rs 1\\
&1 &Slips of Rs 5\\
&2 &Slips of Rs 13\\ \hline
\multirow{2}{*}{Y} &0 &Box A\\
&1 &Box B\\\hline
\end{tabular}
\caption{}
\label{tab:Distribution}
\end{table}
See \tabref{tab:Distribution}.
\begin{align}
p_{Y}\brak{k}= \begin{cases} 
      \frac{1}{3} & {k=0} \\
      \frac{2}{3 }& {k=1} 
   \end{cases}
   \\
p_{Y|X}\brak{0|0} = \frac{19}{25}\, 
p_{Y|X}\brak{0|1} = \frac{6}{25}\,
p_{Y|X}\brak{1|0} = \frac{45}{50}\,
p_{Y|X}\brak{1|2} = \frac{5}{50}
\end{align}
The desired probability is the probability that a slip drawn at random is marked other than Rs 1,
\begin{align}
&=1-p_X\brak{0}\\
&= p_X(1) + p_X(2)
\end{align}
Using Bayes theorem,
\begin{align}
&= p_Y\brak{0} \times \pr{Y=0 | X=1} + p_Y\brak{1} \times \pr{Y=1|X=2}\\
&=\frac{1}{3} \times \frac{6}{25} + \frac{2}{3} \times \frac{5}{50}\\
&=\frac{11}{75}
\end{align}

\newpage

%\tableofcontents

\bigskip

\renewcommand{\thefigure}{\theenumi}
\renewcommand{\thetable}{\theenumi}
%\renewcommand{\theequation}{\theenumi}

%\begin{abstract}
%%\boldmath
%In this letter, an algorithm for evaluating the exact analytical bit error rate  (BER)  for the piecewise linear (PL) combiner for  multiple relays is presented. Previous results were available only for upto three relays. The algorithm is unique in the sense that  the actual mathematical expressions, that are prohibitively large, need not be explicitly obtained. The diversity gain due to multiple relays is shown through plots of the analytical BER, well supported by simulations. 
%
%\end{abstract}
% IEEEtran.cls defaults to using nonbold math in the Abstract.
% This preserves the distinction between vectors and scalars. However,
% if the journal you are submitting to favors bold math in the abstract,
% then you can use LaTeX's standard command \boldmath at the very start
% of the abstract to achieve this. Many IEEE journals frown on math
% in the abstract anyway.

% Note that keywords are not normally used for peerreview papers.
%\begin{IEEEkeywords}
%Cooperative diversity, decode and forward, piecewise linear
%\end{IEEEkeywords}



% For peer review papers, you can put extra information on the cover
% page as needed:
% \ifCLASSOPTIONpeerreview
% \begin{center} \bfseries EDICS Category: 3-BBND \end{center}
% \fi
%
% For peerreview papers, this IEEEtran command inserts a page break and
% creates the second title. It will be ignored for other modes.
%\IEEEpeerreviewmaketitle




If the letters of the word ASSASSINATION are arranged at random. Find the Probability that
\begin{enumerate}[label=(\alph*)]
\item Four $S's$ come consecutively in the word
\item Two  $I's$ and two $N's$ come together
\item All $A's$ are not coming together
\item No two $A's$ are coming together
\end{enumerate}
%\begin{table}[H]
	\centering
\begin{tabular}{|c|c|c|}
\hline
Random variable &Value &Definition\\ \hline
\multirow{3}{*}{X} &0 &Slips of Rs 1\\
&1 &Slips of Rs 5\\
&2 &Slips of Rs 13\\ \hline
\multirow{2}{*}{Y} &0 &Box A\\
&1 &Box B\\\hline
\end{tabular}
\caption{}
\label{tab:Distribution}
\end{table}
See \tabref{tab:Distribution}.
\begin{align}
p_{Y}\brak{k}= \begin{cases} 
      \frac{1}{3} & {k=0} \\
      \frac{2}{3 }& {k=1} 
   \end{cases}
   \\
p_{Y|X}\brak{0|0} = \frac{19}{25}\, 
p_{Y|X}\brak{0|1} = \frac{6}{25}\,
p_{Y|X}\brak{1|0} = \frac{45}{50}\,
p_{Y|X}\brak{1|2} = \frac{5}{50}
\end{align}
The desired probability is the probability that a slip drawn at random is marked other than Rs 1,
\begin{align}
&=1-p_X\brak{0}\\
&= p_X(1) + p_X(2)
\end{align}
Using Bayes theorem,
\begin{align}
&= p_Y\brak{0} \times \pr{Y=0 | X=1} + p_Y\brak{1} \times \pr{Y=1|X=2}\\
&=\frac{1}{3} \times \frac{6}{25} + \frac{2}{3} \times \frac{5}{50}\\
&=\frac{11}{75}
\end{align}

\newpage

%\tableofcontents

\bigskip

\renewcommand{\thefigure}{\theenumi}
\renewcommand{\thetable}{\theenumi}
%\renewcommand{\theequation}{\theenumi}

%\begin{abstract}
%%\boldmath
%In this letter, an algorithm for evaluating the exact analytical bit error rate  (BER)  for the piecewise linear (PL) combiner for  multiple relays is presented. Previous results were available only for upto three relays. The algorithm is unique in the sense that  the actual mathematical expressions, that are prohibitively large, need not be explicitly obtained. The diversity gain due to multiple relays is shown through plots of the analytical BER, well supported by simulations. 
%
%\end{abstract}
% IEEEtran.cls defaults to using nonbold math in the Abstract.
% This preserves the distinction between vectors and scalars. However,
% if the journal you are submitting to favors bold math in the abstract,
% then you can use LaTeX's standard command \boldmath at the very start
% of the abstract to achieve this. Many IEEE journals frown on math
% in the abstract anyway.

% Note that keywords are not normally used for peerreview papers.
%\begin{IEEEkeywords}
%Cooperative diversity, decode and forward, piecewise linear
%\end{IEEEkeywords}



% For peer review papers, you can put extra information on the cover
% page as needed:
% \ifCLASSOPTIONpeerreview
% \begin{center} \bfseries EDICS Category: 3-BBND \end{center}
% \fi
%
% For peerreview papers, this IEEEtran command inserts a page break and
% creates the second title. It will be ignored for other modes.
%\IEEEpeerreviewmaketitle




	\item One urn contains two black balls (labelled B1 and B2) and one white ball. A
	second urn contains one black ball and two white balls (labelled W1 and W2).
	Suppose the following experiment is performed. One of the two urns is chosen
	at random. Next a ball is randomly chosen from the urn. Then a second ball is
	chosen at random from the same urn without replacing the first ball.
	
	\begin{enumerate}
	\item What is the probability that two black balls are chosen?
	
	\item What is the probability that two balls of opposite colour are chosen?
	\end{enumerate}
	\solution
	%\begin{align}
    \label{eq:12.13.6.18.1}
	\because	\pr{A|B} &> \pr{A},\
\frac{\pr{AB}}{\pr{B}} > \pr{A}
\\
    \label{eq:12.13.6.18.2}
	\implies \pr{AB} &> \pr{A}\pr{B}
	\\
	\text{or, } \frac{\pr{AB}}{\pr{A}} &=\pr{B|A} > \pr{A}
\end{align}

\end{enumerate}

	\item A card is selected from a pack of 52 cards.
 \begin{enumerate}[label=(\alph*)] 
                 \item How many points are there in the sample space?
                 \item Calculate the probability that the card is an ace of spades.
                 \item Calculate the probability that the card is (i) an ace and (ii) black card.
 \end{enumerate}
\solution
		%\begin{table}[H]
	\centering
\begin{tabular}{|c|c|c|}
\hline
Random variable &Value &Definition\\ \hline
\multirow{3}{*}{X} &0 &Slips of Rs 1\\
&1 &Slips of Rs 5\\
&2 &Slips of Rs 13\\ \hline
\multirow{2}{*}{Y} &0 &Box A\\
&1 &Box B\\\hline
\end{tabular}
\caption{}
\label{tab:Distribution}
\end{table}
See \tabref{tab:Distribution}.
\begin{align}
p_{Y}\brak{k}= \begin{cases} 
      \frac{1}{3} & {k=0} \\
      \frac{2}{3 }& {k=1} 
   \end{cases}
   \\
p_{Y|X}\brak{0|0} = \frac{19}{25}\, 
p_{Y|X}\brak{0|1} = \frac{6}{25}\,
p_{Y|X}\brak{1|0} = \frac{45}{50}\,
p_{Y|X}\brak{1|2} = \frac{5}{50}
\end{align}
The desired probability is the probability that a slip drawn at random is marked other than Rs 1,
\begin{align}
&=1-p_X\brak{0}\\
&= p_X(1) + p_X(2)
\end{align}
Using Bayes theorem,
\begin{align}
&= p_Y\brak{0} \times \pr{Y=0 | X=1} + p_Y\brak{1} \times \pr{Y=1|X=2}\\
&=\frac{1}{3} \times \frac{6}{25} + \frac{2}{3} \times \frac{5}{50}\\
&=\frac{11}{75}
\end{align}

\newpage

%\tableofcontents

\bigskip

\renewcommand{\thefigure}{\theenumi}
\renewcommand{\thetable}{\theenumi}
%\renewcommand{\theequation}{\theenumi}

%\begin{abstract}
%%\boldmath
%In this letter, an algorithm for evaluating the exact analytical bit error rate  (BER)  for the piecewise linear (PL) combiner for  multiple relays is presented. Previous results were available only for upto three relays. The algorithm is unique in the sense that  the actual mathematical expressions, that are prohibitively large, need not be explicitly obtained. The diversity gain due to multiple relays is shown through plots of the analytical BER, well supported by simulations. 
%
%\end{abstract}
% IEEEtran.cls defaults to using nonbold math in the Abstract.
% This preserves the distinction between vectors and scalars. However,
% if the journal you are submitting to favors bold math in the abstract,
% then you can use LaTeX's standard command \boldmath at the very start
% of the abstract to achieve this. Many IEEE journals frown on math
% in the abstract anyway.

% Note that keywords are not normally used for peerreview papers.
%\begin{IEEEkeywords}
%Cooperative diversity, decode and forward, piecewise linear
%\end{IEEEkeywords}



% For peer review papers, you can put extra information on the cover
% page as needed:
% \ifCLASSOPTIONpeerreview
% \begin{center} \bfseries EDICS Category: 3-BBND \end{center}
% \fi
%
% For peerreview papers, this IEEEtran command inserts a page break and
% creates the second title. It will be ignored for other modes.
%\IEEEpeerreviewmaketitle




\item Four cards are drawn from a well-shuffled deck of 52 cards. What is the probability of obtaining 3 diamonds and one spade.
\\
\solution
		%\begin{enumerate}[label=\thesection.\arabic*,ref=\thesection.\theenumi]
	\item One card is drawn from a well-shuffled deck of 52 cards. Find the probability of getting
\begin{enumerate}
\item A king of red colour 
\item A face card 
\item A red face card
\item The jack of hearts
\item A spade
\item The queen of diamonds

\end{enumerate}
\solution
		%\begin{table}[H]
	\centering
\begin{tabular}{|c|c|c|}
\hline
Random variable &Value &Definition\\ \hline
\multirow{3}{*}{X} &0 &Slips of Rs 1\\
&1 &Slips of Rs 5\\
&2 &Slips of Rs 13\\ \hline
\multirow{2}{*}{Y} &0 &Box A\\
&1 &Box B\\\hline
\end{tabular}
\caption{}
\label{tab:Distribution}
\end{table}
See \tabref{tab:Distribution}.
\begin{align}
p_{Y}\brak{k}= \begin{cases} 
      \frac{1}{3} & {k=0} \\
      \frac{2}{3 }& {k=1} 
   \end{cases}
   \\
p_{Y|X}\brak{0|0} = \frac{19}{25}\, 
p_{Y|X}\brak{0|1} = \frac{6}{25}\,
p_{Y|X}\brak{1|0} = \frac{45}{50}\,
p_{Y|X}\brak{1|2} = \frac{5}{50}
\end{align}
The desired probability is the probability that a slip drawn at random is marked other than Rs 1,
\begin{align}
&=1-p_X\brak{0}\\
&= p_X(1) + p_X(2)
\end{align}
Using Bayes theorem,
\begin{align}
&= p_Y\brak{0} \times \pr{Y=0 | X=1} + p_Y\brak{1} \times \pr{Y=1|X=2}\\
&=\frac{1}{3} \times \frac{6}{25} + \frac{2}{3} \times \frac{5}{50}\\
&=\frac{11}{75}
\end{align}

\newpage

%\tableofcontents

\bigskip

\renewcommand{\thefigure}{\theenumi}
\renewcommand{\thetable}{\theenumi}
%\renewcommand{\theequation}{\theenumi}

%\begin{abstract}
%%\boldmath
%In this letter, an algorithm for evaluating the exact analytical bit error rate  (BER)  for the piecewise linear (PL) combiner for  multiple relays is presented. Previous results were available only for upto three relays. The algorithm is unique in the sense that  the actual mathematical expressions, that are prohibitively large, need not be explicitly obtained. The diversity gain due to multiple relays is shown through plots of the analytical BER, well supported by simulations. 
%
%\end{abstract}
% IEEEtran.cls defaults to using nonbold math in the Abstract.
% This preserves the distinction between vectors and scalars. However,
% if the journal you are submitting to favors bold math in the abstract,
% then you can use LaTeX's standard command \boldmath at the very start
% of the abstract to achieve this. Many IEEE journals frown on math
% in the abstract anyway.

% Note that keywords are not normally used for peerreview papers.
%\begin{IEEEkeywords}
%Cooperative diversity, decode and forward, piecewise linear
%\end{IEEEkeywords}



% For peer review papers, you can put extra information on the cover
% page as needed:
% \ifCLASSOPTIONpeerreview
% \begin{center} \bfseries EDICS Category: 3-BBND \end{center}
% \fi
%
% For peerreview papers, this IEEEtran command inserts a page break and
% creates the second title. It will be ignored for other modes.
%\IEEEpeerreviewmaketitle




	\item Five cards—the ten, jack, queen, king and ace of diamonds, are well-shuffled with their face downwards. One card is then picked up at random.
\begin{enumerate}
\item
What is the probability that the card is the queen? 
\item
If the queen is drawn and put aside, what is the probability that the second card picked up is (a) an ace? (b) a queen?\\
\end{enumerate}
\solution
		%\begin{enumerate}[label=\thesection.\arabic*,ref=\thesection.\theenumi]
	\item One card is drawn from a well-shuffled deck of 52 cards. Find the probability of getting
\begin{enumerate}
\item A king of red colour 
\item A face card 
\item A red face card
\item The jack of hearts
\item A spade
\item The queen of diamonds

\end{enumerate}
\solution
		%\input{ncert/10/15/1/14/main.tex}
	\item Five cards—the ten, jack, queen, king and ace of diamonds, are well-shuffled with their face downwards. One card is then picked up at random.
\begin{enumerate}
\item
What is the probability that the card is the queen? 
\item
If the queen is drawn and put aside, what is the probability that the second card picked up is (a) an ace? (b) a queen?\\
\end{enumerate}
\solution
		%\input{ncert/10/15/1/15/defs.tex}
	\item A bag contains $5$ red balls and some blue balls. If the probability of drawing a blue ball is double that if a red ball, determine the number of blue balls in the bag. 
		\\
\solution
		%\input{ncert/10/15/2/3/defs.tex}
	\item A card is selected from a pack of 52 cards.
 \begin{enumerate}[label=(\alph*)] 
                 \item How many points are there in the sample space?
                 \item Calculate the probability that the card is an ace of spades.
                 \item Calculate the probability that the card is (i) an ace and (ii) black card.
 \end{enumerate}
\solution
		%\input{ncert/11/16/3/4/main.tex}
\item Four cards are drawn from a well-shuffled deck of 52 cards. What is the probability of obtaining 3 diamonds and one spade.
\\
\solution
		%\input{ncert/11/16/4/2/defs.tex}
\item In a certain lottery 10,000 tickets are sold and ten equal prizes are awarded. What is the probability of not getting a prize if you buy (a) one ticket (b) two tickets (c) 10 tickets ?	
\\
\solution
		%\input{ncert/11/16/4/4/defs.tex}
		%
\item 
Out of 100 students, two sections of 40 and 60 are formed. If you and your friend are among the 100 students, what is the probability that
\begin{enumerate}
\item you both enter the same section?
\item you both enter the different sections?
\end{enumerate}
\solution
		%\input{ncert/11/16/4/5/defs.tex}
	\item 
The number lock of a suitcase has 4 wheels each labelled with ten digits i.e. from 0 to 9.The lock opens with a sequence of four digits with no repeats.What is the probability of a person getting the right sequence to open the suitcase.
\\
\solution
		%\input{ncert/11/16/4/10/defs.tex}
		%
\item 
Two cards are drawn at random and without replacement from a pack of 52 playing cards. Find the probability that both the cards are black.
\\
\solution
		%\input{ncert/12/13/2/2/defs.tex}
		\item A box of oranges is inspected by examining three randomly selected oranges drawn without replacement. If all the three oranges are good, the box is approved for sale, otherwise, it is rejected. Find the probability that a box containing 15 oranges out of which 12 are good and 3 are bad ones will be approved for sale.
		\label{ncert/12/13/2/3/defs.tex}
		\item Two balls are drawn at random with replacement from a box containing 10 black and 8 red balls. Find the probability that
		\label{ncert/12/13/2/12}
\begin{enumerate}
\item both balls are red.
\item first ball is black and second is red.
\item one of them is black and other is red.
\end{enumerate}

\item In a hostel, 60\% of the students read Hindi newspaper, 40\% read English newspaper and 20\% read both Hindi and English newspapers. A student is selected at random.
		\label{ncert/12/13/2/15}
\begin{enumerate}
\item Find the probability that she reads neither Hindi nor English newspapers.
\item If she reads Hindi newspaper, find the probability that she reads English newspaper.
\item If she reads English newspaper, find the probability that she reads Hindi newspaper.\\
\end{enumerate}
\item The probability of obtaining an even prime number on each die, when a pair of dice is rolled is 
\begin{enumerate}
    \item $0$ 
    
    \item $\frac{1}{3}$ 
    
    \item $\frac{1}{12}$ 
    
    \item $\frac{1}{36}$ 
\end{enumerate}
\solution
		%\input{ncert/12/13/2/17/defs.tex}
	\item A bag contains 4 red and 4 black balls, another bag contains 2 red and 6 black balls. One of the two bags is selected at random and a ball is drawn from the bag which is found to be red. Find the probability that the ball is drawn from the first bag.
\\
\solution
		%\input{ncert/12/13/3/2/main.tex}
  \item
  Cards with numbers 2 to 101 are placed in a box. A card is selected at random.Find the probability that the card has
\begin{enumerate}[label=(\roman*)]
	\item an even number 
	\item a square number
\end{enumerate}
\solution
%\input{exemplar/10/13/3/32/main.tex}
\item
The king, queen and jack of clubs are removed from a deck of 52 playing cards and then well shuffled. Now one card is drawn at random from the remaining cards.  Determine the probability that the card is
\begin{enumerate}[label=(\roman*)]
\item a club
\item 10 of hearts
\end{enumerate}
\solution
%\input{exemplar/10/13/3/29/main.tex}
\item A team of medical students doing their internship have to assist during surgeries
at a city hospital. The probabilities of surgeries rated as very complex, complex,
routine, simple or very simple are respectively, 0.15, 0.20, 0.31, 0.26, .08. Find
the probabilities that a particular surgery will be rated
\begin{enumerate}
	\item complex or very complex;
	\item neither very complex nor very simple;
	\item routine or complex
	\item routine or simple
\end{enumerate}
\solution
%\input{exemplar/11/16/3/8(1)/main.tex}
\item A card is selected from a pack of 52 cards.
\begin{enumerate}[label=(\alph*)]
    \item How many points are there in the sample space?
    \item Calculate the probability that the card is an ace of spades.
    \item Calculate the probability that the card is (i) an ace and (ii) black card.
\end{enumerate}
\solution
%\input{exemplar/11/16/3/4/main2.tex}
\item The probability that a non leap year selected at random will contain 53 sundays.
\\
\solution
%\input{exemplar/10/13/1/19/main.tex}
\item One of the four persons John, Rita, Aslam or Gurpreet will be promoted next
month. Consequently the sample space consists of four elementary outcomes
S = {John promoted, Rita promoted, Aslam promoted, Gurpreet promoted}
You are told that the chances of John’s promotion is same as that of Gurpreet,
Rita’s chances of promotion are twice as likely as Johns. Aslam’s chances are
four times that of John.
\begin{enumerate}
	\item Determine
	\begin{enumerate}
		\item P (John promoted)
		\item P (Rita promoted)
		\item P (Aslam promoted)
		\item P (Gurpreet promoted)
	\end{enumerate}
	\item If A = {John promoted or Gurpreet promoted}, find P (A).
\end{enumerate}
\solution
%\input{exemplar/11/16/3/10/main.tex}
\item A card is drawn from a deck of 52 cards. Find the probability of getting a king or a heart or a red card.\\
\solution
%\input{exemplar/11/16/3/15/main.tex}
\item The probability that a student will pass his examination is 0.73, the probability of
the student getting a compartment is 0.13, and the probability that the student will
either pass or get compartment is 0.96. State True or False.\\
\solution
%\input{exemplar/11/16/3/31/main.tex}
\item A card is selected from a pack of 52 cards\\
\begin{enumerate}[label=(\alph*)]
\item How many points are there in the sample space?
\item Calculate the probability that the cards is an ace of spades.
\item Calculate the probability that the card is (i) an ace (ii)black card.\\
\end{enumerate}
%\input{ncert/11/16/3/4_1/Prob_4.tex}
\item In a non-leap year, the probability of having 53 tuesdays or 53 wednesdays is\\
\solution
%\input{exemplar/11/16/3/18/main.tex}
\item There are 1000 sealed envelopes in a box, 10 of them contain a cash prize of
Rs 100 each, 100 of them contain a cash prize of Rs 50 each and 200 of them
contain a cash prize of Rs 10 each and rest do not contain any cash prize. If they
are well shuffled and an envelope is picked up out, what is the probability that it
contains no cash prize?\\
\solution
%\input{exemplar/10/13/3/34/main.tex}
\item 
A die is thrown and a card is selected at random from a deck of 52 playing cards. The probability of getting an even number on the die and a spade card.\\
\solution
%\input{exemplar/12/13/3/78/main.tex}
\item
If 4-digit numbers greater than 5,000 are randomly formed from the digits 0, 1, 3, 5, and 7, what is the probability of forming a number divisible by 5 when:
\begin{enumerate}
    \item The digits are repeated?
    \item The repetition of digits is not allowed?
\end{enumerate}
\solution
%\input{ncert/11/16/4/9/main.tex}
\item Consider the probability space $\brak{\Omega, \mathcal{G}, P}$ where $\Omega = [0,2]$ and $\mathcal{G} = \cbrak{\phi, \Omega, [0,1], (1,2]}$. Let $X$ and $Y$ be two functions on $\Omega$ defined as
\begin{align*}
    X(\omega) = 
    \begin{cases}
        1 & \text{if }\omega \in [0, 1]\\
        2 & \text{if }\omega \in (1, 2]
    \end{cases}
\end{align*}
and
\begin{align*}
    Y(\omega) = 
    \begin{cases}
        2 & \text{if }\omega \in [0, 1.5]\\
        3 & \text{if }\omega \in (1.5, 2].
    \end{cases}
\end{align*}
Then which one of the following statements is true?
\begin{enumerate}
    \item [(A)] $X$ is a random variable with respect to $\mathcal{G}$, but $Y$ is not a random variable with respect to $\mathcal{G}$.
    \item [(B)] $Y$ is a random variable with respect to $\mathcal{G}$, but $X$ is not a random variable with respect to $\mathcal{G}$.
    \item [(C)] Neither $X$ nor $Y$ is a random variable with respect to $\mathcal{G}$.
    \item [(D)] Both $X$ and $Y$ are random variables with respect to $\mathcal{G}$.
\end{enumerate} \hfill (GATE ST 2023)\\
\solution
%\input{gate/ST/2023/14/main.tex}
	\item  A die is loaded in such a way that each odd number is twice as likely to occur as
each even number. Find $P(G)$, where $G$ is the event that a number greater than
3 occurs on a single roll of the die.
\\
\solution
		%\input{exemplar/11/16/3/5/main.tex}
	\item All the jacks, queens and kings are removed from a deck of 52 playing cards. The remaining cards are well shuffled and then one card is drawn at random. Giving ace a value 1 similar value for other cards, find the probability that the card has a value 
		\begin{enumerate}
			\item 7
			\item greater than 7
			\item less than 7
		\end{enumerate}
		%\input{exemplar/10/13/3/30/main.tex}
  \item A Lot consists of 48 mobile phones of which 42 are good, 3 have only minor defects and 3 have major defects.Varnika will buy a phone if it is good but the trader will only buy a mobile if it has no major defects. One phone is selected at random from the lot. What is the probability that it is
\begin{enumerate}
	\item acceptable to Varnika?
            \item acceptable to the trader?
\end{enumerate}
\solution
	%\input{exemplar/10/13/3/40/main.tex}
 \item A student says that if you throw a die, it will show up 1 or not 1. Therefore, the probability of getting 1 and the probability of getting 'not 1' each is equal to $\frac{1}{2}$. Is this correct? Give reasons.\\
 \solution
        %\input{exemplar/10/13/2/9/main.tex}
   \item Four candidates A, B, C, D have ap-
plied for the assignment to coach a school cricket
team. If A is twice as likely to be selected as B, and
B and C are given about the same chance of being
selected, while C is twice as likely to be selected
as D, what are the probabilities that
\begin{enumerate}
\item C will be selected?
\item A will not be selected?
\end{enumerate}
	%\input{exemplar/11/16/3/9/main.tex}
 \item A bag contain 24 balls of which $x$ balls are red, $2x$ are white and $3x$ are blue. A ball is selected at random, What is the probability that it is
\begin{enumerate}[label=\alph*)]
\item not red ?
\item white ?
\end{enumerate}
%\input{exemplar/10/13/3/41/main.tex}
If the letters of the word ASSASSINATION are arranged at random. Find the Probability that
\begin{enumerate}[label=(\alph*)]
\item Four $S's$ come consecutively in the word
\item Two  $I's$ and two $N's$ come together
\item All $A's$ are not coming together
\item No two $A's$ are coming together
\end{enumerate}
%\input{exemplar/11/16/3/14/main.tex}
	\item One urn contains two black balls (labelled B1 and B2) and one white ball. A
	second urn contains one black ball and two white balls (labelled W1 and W2).
	Suppose the following experiment is performed. One of the two urns is chosen
	at random. Next a ball is randomly chosen from the urn. Then a second ball is
	chosen at random from the same urn without replacing the first ball.
	
	\begin{enumerate}
	\item What is the probability that two black balls are chosen?
	
	\item What is the probability that two balls of opposite colour are chosen?
	\end{enumerate}
	\solution
	%\input{exemplar/11/16/3/12/main1.tex}
\end{enumerate}

	\item A bag contains $5$ red balls and some blue balls. If the probability of drawing a blue ball is double that if a red ball, determine the number of blue balls in the bag. 
		\\
\solution
		%\begin{enumerate}[label=\thesection.\arabic*,ref=\thesection.\theenumi]
	\item One card is drawn from a well-shuffled deck of 52 cards. Find the probability of getting
\begin{enumerate}
\item A king of red colour 
\item A face card 
\item A red face card
\item The jack of hearts
\item A spade
\item The queen of diamonds

\end{enumerate}
\solution
		%\input{ncert/10/15/1/14/main.tex}
	\item Five cards—the ten, jack, queen, king and ace of diamonds, are well-shuffled with their face downwards. One card is then picked up at random.
\begin{enumerate}
\item
What is the probability that the card is the queen? 
\item
If the queen is drawn and put aside, what is the probability that the second card picked up is (a) an ace? (b) a queen?\\
\end{enumerate}
\solution
		%\input{ncert/10/15/1/15/defs.tex}
	\item A bag contains $5$ red balls and some blue balls. If the probability of drawing a blue ball is double that if a red ball, determine the number of blue balls in the bag. 
		\\
\solution
		%\input{ncert/10/15/2/3/defs.tex}
	\item A card is selected from a pack of 52 cards.
 \begin{enumerate}[label=(\alph*)] 
                 \item How many points are there in the sample space?
                 \item Calculate the probability that the card is an ace of spades.
                 \item Calculate the probability that the card is (i) an ace and (ii) black card.
 \end{enumerate}
\solution
		%\input{ncert/11/16/3/4/main.tex}
\item Four cards are drawn from a well-shuffled deck of 52 cards. What is the probability of obtaining 3 diamonds and one spade.
\\
\solution
		%\input{ncert/11/16/4/2/defs.tex}
\item In a certain lottery 10,000 tickets are sold and ten equal prizes are awarded. What is the probability of not getting a prize if you buy (a) one ticket (b) two tickets (c) 10 tickets ?	
\\
\solution
		%\input{ncert/11/16/4/4/defs.tex}
		%
\item 
Out of 100 students, two sections of 40 and 60 are formed. If you and your friend are among the 100 students, what is the probability that
\begin{enumerate}
\item you both enter the same section?
\item you both enter the different sections?
\end{enumerate}
\solution
		%\input{ncert/11/16/4/5/defs.tex}
	\item 
The number lock of a suitcase has 4 wheels each labelled with ten digits i.e. from 0 to 9.The lock opens with a sequence of four digits with no repeats.What is the probability of a person getting the right sequence to open the suitcase.
\\
\solution
		%\input{ncert/11/16/4/10/defs.tex}
		%
\item 
Two cards are drawn at random and without replacement from a pack of 52 playing cards. Find the probability that both the cards are black.
\\
\solution
		%\input{ncert/12/13/2/2/defs.tex}
		\item A box of oranges is inspected by examining three randomly selected oranges drawn without replacement. If all the three oranges are good, the box is approved for sale, otherwise, it is rejected. Find the probability that a box containing 15 oranges out of which 12 are good and 3 are bad ones will be approved for sale.
		\label{ncert/12/13/2/3/defs.tex}
		\item Two balls are drawn at random with replacement from a box containing 10 black and 8 red balls. Find the probability that
		\label{ncert/12/13/2/12}
\begin{enumerate}
\item both balls are red.
\item first ball is black and second is red.
\item one of them is black and other is red.
\end{enumerate}

\item In a hostel, 60\% of the students read Hindi newspaper, 40\% read English newspaper and 20\% read both Hindi and English newspapers. A student is selected at random.
		\label{ncert/12/13/2/15}
\begin{enumerate}
\item Find the probability that she reads neither Hindi nor English newspapers.
\item If she reads Hindi newspaper, find the probability that she reads English newspaper.
\item If she reads English newspaper, find the probability that she reads Hindi newspaper.\\
\end{enumerate}
\item The probability of obtaining an even prime number on each die, when a pair of dice is rolled is 
\begin{enumerate}
    \item $0$ 
    
    \item $\frac{1}{3}$ 
    
    \item $\frac{1}{12}$ 
    
    \item $\frac{1}{36}$ 
\end{enumerate}
\solution
		%\input{ncert/12/13/2/17/defs.tex}
	\item A bag contains 4 red and 4 black balls, another bag contains 2 red and 6 black balls. One of the two bags is selected at random and a ball is drawn from the bag which is found to be red. Find the probability that the ball is drawn from the first bag.
\\
\solution
		%\input{ncert/12/13/3/2/main.tex}
  \item
  Cards with numbers 2 to 101 are placed in a box. A card is selected at random.Find the probability that the card has
\begin{enumerate}[label=(\roman*)]
	\item an even number 
	\item a square number
\end{enumerate}
\solution
%\input{exemplar/10/13/3/32/main.tex}
\item
The king, queen and jack of clubs are removed from a deck of 52 playing cards and then well shuffled. Now one card is drawn at random from the remaining cards.  Determine the probability that the card is
\begin{enumerate}[label=(\roman*)]
\item a club
\item 10 of hearts
\end{enumerate}
\solution
%\input{exemplar/10/13/3/29/main.tex}
\item A team of medical students doing their internship have to assist during surgeries
at a city hospital. The probabilities of surgeries rated as very complex, complex,
routine, simple or very simple are respectively, 0.15, 0.20, 0.31, 0.26, .08. Find
the probabilities that a particular surgery will be rated
\begin{enumerate}
	\item complex or very complex;
	\item neither very complex nor very simple;
	\item routine or complex
	\item routine or simple
\end{enumerate}
\solution
%\input{exemplar/11/16/3/8(1)/main.tex}
\item A card is selected from a pack of 52 cards.
\begin{enumerate}[label=(\alph*)]
    \item How many points are there in the sample space?
    \item Calculate the probability that the card is an ace of spades.
    \item Calculate the probability that the card is (i) an ace and (ii) black card.
\end{enumerate}
\solution
%\input{exemplar/11/16/3/4/main2.tex}
\item The probability that a non leap year selected at random will contain 53 sundays.
\\
\solution
%\input{exemplar/10/13/1/19/main.tex}
\item One of the four persons John, Rita, Aslam or Gurpreet will be promoted next
month. Consequently the sample space consists of four elementary outcomes
S = {John promoted, Rita promoted, Aslam promoted, Gurpreet promoted}
You are told that the chances of John’s promotion is same as that of Gurpreet,
Rita’s chances of promotion are twice as likely as Johns. Aslam’s chances are
four times that of John.
\begin{enumerate}
	\item Determine
	\begin{enumerate}
		\item P (John promoted)
		\item P (Rita promoted)
		\item P (Aslam promoted)
		\item P (Gurpreet promoted)
	\end{enumerate}
	\item If A = {John promoted or Gurpreet promoted}, find P (A).
\end{enumerate}
\solution
%\input{exemplar/11/16/3/10/main.tex}
\item A card is drawn from a deck of 52 cards. Find the probability of getting a king or a heart or a red card.\\
\solution
%\input{exemplar/11/16/3/15/main.tex}
\item The probability that a student will pass his examination is 0.73, the probability of
the student getting a compartment is 0.13, and the probability that the student will
either pass or get compartment is 0.96. State True or False.\\
\solution
%\input{exemplar/11/16/3/31/main.tex}
\item A card is selected from a pack of 52 cards\\
\begin{enumerate}[label=(\alph*)]
\item How many points are there in the sample space?
\item Calculate the probability that the cards is an ace of spades.
\item Calculate the probability that the card is (i) an ace (ii)black card.\\
\end{enumerate}
%\input{ncert/11/16/3/4_1/Prob_4.tex}
\item In a non-leap year, the probability of having 53 tuesdays or 53 wednesdays is\\
\solution
%\input{exemplar/11/16/3/18/main.tex}
\item There are 1000 sealed envelopes in a box, 10 of them contain a cash prize of
Rs 100 each, 100 of them contain a cash prize of Rs 50 each and 200 of them
contain a cash prize of Rs 10 each and rest do not contain any cash prize. If they
are well shuffled and an envelope is picked up out, what is the probability that it
contains no cash prize?\\
\solution
%\input{exemplar/10/13/3/34/main.tex}
\item 
A die is thrown and a card is selected at random from a deck of 52 playing cards. The probability of getting an even number on the die and a spade card.\\
\solution
%\input{exemplar/12/13/3/78/main.tex}
\item
If 4-digit numbers greater than 5,000 are randomly formed from the digits 0, 1, 3, 5, and 7, what is the probability of forming a number divisible by 5 when:
\begin{enumerate}
    \item The digits are repeated?
    \item The repetition of digits is not allowed?
\end{enumerate}
\solution
%\input{ncert/11/16/4/9/main.tex}
\item Consider the probability space $\brak{\Omega, \mathcal{G}, P}$ where $\Omega = [0,2]$ and $\mathcal{G} = \cbrak{\phi, \Omega, [0,1], (1,2]}$. Let $X$ and $Y$ be two functions on $\Omega$ defined as
\begin{align*}
    X(\omega) = 
    \begin{cases}
        1 & \text{if }\omega \in [0, 1]\\
        2 & \text{if }\omega \in (1, 2]
    \end{cases}
\end{align*}
and
\begin{align*}
    Y(\omega) = 
    \begin{cases}
        2 & \text{if }\omega \in [0, 1.5]\\
        3 & \text{if }\omega \in (1.5, 2].
    \end{cases}
\end{align*}
Then which one of the following statements is true?
\begin{enumerate}
    \item [(A)] $X$ is a random variable with respect to $\mathcal{G}$, but $Y$ is not a random variable with respect to $\mathcal{G}$.
    \item [(B)] $Y$ is a random variable with respect to $\mathcal{G}$, but $X$ is not a random variable with respect to $\mathcal{G}$.
    \item [(C)] Neither $X$ nor $Y$ is a random variable with respect to $\mathcal{G}$.
    \item [(D)] Both $X$ and $Y$ are random variables with respect to $\mathcal{G}$.
\end{enumerate} \hfill (GATE ST 2023)\\
\solution
%\input{gate/ST/2023/14/main.tex}
	\item  A die is loaded in such a way that each odd number is twice as likely to occur as
each even number. Find $P(G)$, where $G$ is the event that a number greater than
3 occurs on a single roll of the die.
\\
\solution
		%\input{exemplar/11/16/3/5/main.tex}
	\item All the jacks, queens and kings are removed from a deck of 52 playing cards. The remaining cards are well shuffled and then one card is drawn at random. Giving ace a value 1 similar value for other cards, find the probability that the card has a value 
		\begin{enumerate}
			\item 7
			\item greater than 7
			\item less than 7
		\end{enumerate}
		%\input{exemplar/10/13/3/30/main.tex}
  \item A Lot consists of 48 mobile phones of which 42 are good, 3 have only minor defects and 3 have major defects.Varnika will buy a phone if it is good but the trader will only buy a mobile if it has no major defects. One phone is selected at random from the lot. What is the probability that it is
\begin{enumerate}
	\item acceptable to Varnika?
            \item acceptable to the trader?
\end{enumerate}
\solution
	%\input{exemplar/10/13/3/40/main.tex}
 \item A student says that if you throw a die, it will show up 1 or not 1. Therefore, the probability of getting 1 and the probability of getting 'not 1' each is equal to $\frac{1}{2}$. Is this correct? Give reasons.\\
 \solution
        %\input{exemplar/10/13/2/9/main.tex}
   \item Four candidates A, B, C, D have ap-
plied for the assignment to coach a school cricket
team. If A is twice as likely to be selected as B, and
B and C are given about the same chance of being
selected, while C is twice as likely to be selected
as D, what are the probabilities that
\begin{enumerate}
\item C will be selected?
\item A will not be selected?
\end{enumerate}
	%\input{exemplar/11/16/3/9/main.tex}
 \item A bag contain 24 balls of which $x$ balls are red, $2x$ are white and $3x$ are blue. A ball is selected at random, What is the probability that it is
\begin{enumerate}[label=\alph*)]
\item not red ?
\item white ?
\end{enumerate}
%\input{exemplar/10/13/3/41/main.tex}
If the letters of the word ASSASSINATION are arranged at random. Find the Probability that
\begin{enumerate}[label=(\alph*)]
\item Four $S's$ come consecutively in the word
\item Two  $I's$ and two $N's$ come together
\item All $A's$ are not coming together
\item No two $A's$ are coming together
\end{enumerate}
%\input{exemplar/11/16/3/14/main.tex}
	\item One urn contains two black balls (labelled B1 and B2) and one white ball. A
	second urn contains one black ball and two white balls (labelled W1 and W2).
	Suppose the following experiment is performed. One of the two urns is chosen
	at random. Next a ball is randomly chosen from the urn. Then a second ball is
	chosen at random from the same urn without replacing the first ball.
	
	\begin{enumerate}
	\item What is the probability that two black balls are chosen?
	
	\item What is the probability that two balls of opposite colour are chosen?
	\end{enumerate}
	\solution
	%\input{exemplar/11/16/3/12/main1.tex}
\end{enumerate}

	\item A card is selected from a pack of 52 cards.
 \begin{enumerate}[label=(\alph*)] 
                 \item How many points are there in the sample space?
                 \item Calculate the probability that the card is an ace of spades.
                 \item Calculate the probability that the card is (i) an ace and (ii) black card.
 \end{enumerate}
\solution
		%\begin{table}[H]
	\centering
\begin{tabular}{|c|c|c|}
\hline
Random variable &Value &Definition\\ \hline
\multirow{3}{*}{X} &0 &Slips of Rs 1\\
&1 &Slips of Rs 5\\
&2 &Slips of Rs 13\\ \hline
\multirow{2}{*}{Y} &0 &Box A\\
&1 &Box B\\\hline
\end{tabular}
\caption{}
\label{tab:Distribution}
\end{table}
See \tabref{tab:Distribution}.
\begin{align}
p_{Y}\brak{k}= \begin{cases} 
      \frac{1}{3} & {k=0} \\
      \frac{2}{3 }& {k=1} 
   \end{cases}
   \\
p_{Y|X}\brak{0|0} = \frac{19}{25}\, 
p_{Y|X}\brak{0|1} = \frac{6}{25}\,
p_{Y|X}\brak{1|0} = \frac{45}{50}\,
p_{Y|X}\brak{1|2} = \frac{5}{50}
\end{align}
The desired probability is the probability that a slip drawn at random is marked other than Rs 1,
\begin{align}
&=1-p_X\brak{0}\\
&= p_X(1) + p_X(2)
\end{align}
Using Bayes theorem,
\begin{align}
&= p_Y\brak{0} \times \pr{Y=0 | X=1} + p_Y\brak{1} \times \pr{Y=1|X=2}\\
&=\frac{1}{3} \times \frac{6}{25} + \frac{2}{3} \times \frac{5}{50}\\
&=\frac{11}{75}
\end{align}

\newpage

%\tableofcontents

\bigskip

\renewcommand{\thefigure}{\theenumi}
\renewcommand{\thetable}{\theenumi}
%\renewcommand{\theequation}{\theenumi}

%\begin{abstract}
%%\boldmath
%In this letter, an algorithm for evaluating the exact analytical bit error rate  (BER)  for the piecewise linear (PL) combiner for  multiple relays is presented. Previous results were available only for upto three relays. The algorithm is unique in the sense that  the actual mathematical expressions, that are prohibitively large, need not be explicitly obtained. The diversity gain due to multiple relays is shown through plots of the analytical BER, well supported by simulations. 
%
%\end{abstract}
% IEEEtran.cls defaults to using nonbold math in the Abstract.
% This preserves the distinction between vectors and scalars. However,
% if the journal you are submitting to favors bold math in the abstract,
% then you can use LaTeX's standard command \boldmath at the very start
% of the abstract to achieve this. Many IEEE journals frown on math
% in the abstract anyway.

% Note that keywords are not normally used for peerreview papers.
%\begin{IEEEkeywords}
%Cooperative diversity, decode and forward, piecewise linear
%\end{IEEEkeywords}



% For peer review papers, you can put extra information on the cover
% page as needed:
% \ifCLASSOPTIONpeerreview
% \begin{center} \bfseries EDICS Category: 3-BBND \end{center}
% \fi
%
% For peerreview papers, this IEEEtran command inserts a page break and
% creates the second title. It will be ignored for other modes.
%\IEEEpeerreviewmaketitle




\item Four cards are drawn from a well-shuffled deck of 52 cards. What is the probability of obtaining 3 diamonds and one spade.
\\
\solution
		%\begin{enumerate}[label=\thesection.\arabic*,ref=\thesection.\theenumi]
	\item One card is drawn from a well-shuffled deck of 52 cards. Find the probability of getting
\begin{enumerate}
\item A king of red colour 
\item A face card 
\item A red face card
\item The jack of hearts
\item A spade
\item The queen of diamonds

\end{enumerate}
\solution
		%\input{ncert/10/15/1/14/main.tex}
	\item Five cards—the ten, jack, queen, king and ace of diamonds, are well-shuffled with their face downwards. One card is then picked up at random.
\begin{enumerate}
\item
What is the probability that the card is the queen? 
\item
If the queen is drawn and put aside, what is the probability that the second card picked up is (a) an ace? (b) a queen?\\
\end{enumerate}
\solution
		%\input{ncert/10/15/1/15/defs.tex}
	\item A bag contains $5$ red balls and some blue balls. If the probability of drawing a blue ball is double that if a red ball, determine the number of blue balls in the bag. 
		\\
\solution
		%\input{ncert/10/15/2/3/defs.tex}
	\item A card is selected from a pack of 52 cards.
 \begin{enumerate}[label=(\alph*)] 
                 \item How many points are there in the sample space?
                 \item Calculate the probability that the card is an ace of spades.
                 \item Calculate the probability that the card is (i) an ace and (ii) black card.
 \end{enumerate}
\solution
		%\input{ncert/11/16/3/4/main.tex}
\item Four cards are drawn from a well-shuffled deck of 52 cards. What is the probability of obtaining 3 diamonds and one spade.
\\
\solution
		%\input{ncert/11/16/4/2/defs.tex}
\item In a certain lottery 10,000 tickets are sold and ten equal prizes are awarded. What is the probability of not getting a prize if you buy (a) one ticket (b) two tickets (c) 10 tickets ?	
\\
\solution
		%\input{ncert/11/16/4/4/defs.tex}
		%
\item 
Out of 100 students, two sections of 40 and 60 are formed. If you and your friend are among the 100 students, what is the probability that
\begin{enumerate}
\item you both enter the same section?
\item you both enter the different sections?
\end{enumerate}
\solution
		%\input{ncert/11/16/4/5/defs.tex}
	\item 
The number lock of a suitcase has 4 wheels each labelled with ten digits i.e. from 0 to 9.The lock opens with a sequence of four digits with no repeats.What is the probability of a person getting the right sequence to open the suitcase.
\\
\solution
		%\input{ncert/11/16/4/10/defs.tex}
		%
\item 
Two cards are drawn at random and without replacement from a pack of 52 playing cards. Find the probability that both the cards are black.
\\
\solution
		%\input{ncert/12/13/2/2/defs.tex}
		\item A box of oranges is inspected by examining three randomly selected oranges drawn without replacement. If all the three oranges are good, the box is approved for sale, otherwise, it is rejected. Find the probability that a box containing 15 oranges out of which 12 are good and 3 are bad ones will be approved for sale.
		\label{ncert/12/13/2/3/defs.tex}
		\item Two balls are drawn at random with replacement from a box containing 10 black and 8 red balls. Find the probability that
		\label{ncert/12/13/2/12}
\begin{enumerate}
\item both balls are red.
\item first ball is black and second is red.
\item one of them is black and other is red.
\end{enumerate}

\item In a hostel, 60\% of the students read Hindi newspaper, 40\% read English newspaper and 20\% read both Hindi and English newspapers. A student is selected at random.
		\label{ncert/12/13/2/15}
\begin{enumerate}
\item Find the probability that she reads neither Hindi nor English newspapers.
\item If she reads Hindi newspaper, find the probability that she reads English newspaper.
\item If she reads English newspaper, find the probability that she reads Hindi newspaper.\\
\end{enumerate}
\item The probability of obtaining an even prime number on each die, when a pair of dice is rolled is 
\begin{enumerate}
    \item $0$ 
    
    \item $\frac{1}{3}$ 
    
    \item $\frac{1}{12}$ 
    
    \item $\frac{1}{36}$ 
\end{enumerate}
\solution
		%\input{ncert/12/13/2/17/defs.tex}
	\item A bag contains 4 red and 4 black balls, another bag contains 2 red and 6 black balls. One of the two bags is selected at random and a ball is drawn from the bag which is found to be red. Find the probability that the ball is drawn from the first bag.
\\
\solution
		%\input{ncert/12/13/3/2/main.tex}
  \item
  Cards with numbers 2 to 101 are placed in a box. A card is selected at random.Find the probability that the card has
\begin{enumerate}[label=(\roman*)]
	\item an even number 
	\item a square number
\end{enumerate}
\solution
%\input{exemplar/10/13/3/32/main.tex}
\item
The king, queen and jack of clubs are removed from a deck of 52 playing cards and then well shuffled. Now one card is drawn at random from the remaining cards.  Determine the probability that the card is
\begin{enumerate}[label=(\roman*)]
\item a club
\item 10 of hearts
\end{enumerate}
\solution
%\input{exemplar/10/13/3/29/main.tex}
\item A team of medical students doing their internship have to assist during surgeries
at a city hospital. The probabilities of surgeries rated as very complex, complex,
routine, simple or very simple are respectively, 0.15, 0.20, 0.31, 0.26, .08. Find
the probabilities that a particular surgery will be rated
\begin{enumerate}
	\item complex or very complex;
	\item neither very complex nor very simple;
	\item routine or complex
	\item routine or simple
\end{enumerate}
\solution
%\input{exemplar/11/16/3/8(1)/main.tex}
\item A card is selected from a pack of 52 cards.
\begin{enumerate}[label=(\alph*)]
    \item How many points are there in the sample space?
    \item Calculate the probability that the card is an ace of spades.
    \item Calculate the probability that the card is (i) an ace and (ii) black card.
\end{enumerate}
\solution
%\input{exemplar/11/16/3/4/main2.tex}
\item The probability that a non leap year selected at random will contain 53 sundays.
\\
\solution
%\input{exemplar/10/13/1/19/main.tex}
\item One of the four persons John, Rita, Aslam or Gurpreet will be promoted next
month. Consequently the sample space consists of four elementary outcomes
S = {John promoted, Rita promoted, Aslam promoted, Gurpreet promoted}
You are told that the chances of John’s promotion is same as that of Gurpreet,
Rita’s chances of promotion are twice as likely as Johns. Aslam’s chances are
four times that of John.
\begin{enumerate}
	\item Determine
	\begin{enumerate}
		\item P (John promoted)
		\item P (Rita promoted)
		\item P (Aslam promoted)
		\item P (Gurpreet promoted)
	\end{enumerate}
	\item If A = {John promoted or Gurpreet promoted}, find P (A).
\end{enumerate}
\solution
%\input{exemplar/11/16/3/10/main.tex}
\item A card is drawn from a deck of 52 cards. Find the probability of getting a king or a heart or a red card.\\
\solution
%\input{exemplar/11/16/3/15/main.tex}
\item The probability that a student will pass his examination is 0.73, the probability of
the student getting a compartment is 0.13, and the probability that the student will
either pass or get compartment is 0.96. State True or False.\\
\solution
%\input{exemplar/11/16/3/31/main.tex}
\item A card is selected from a pack of 52 cards\\
\begin{enumerate}[label=(\alph*)]
\item How many points are there in the sample space?
\item Calculate the probability that the cards is an ace of spades.
\item Calculate the probability that the card is (i) an ace (ii)black card.\\
\end{enumerate}
%\input{ncert/11/16/3/4_1/Prob_4.tex}
\item In a non-leap year, the probability of having 53 tuesdays or 53 wednesdays is\\
\solution
%\input{exemplar/11/16/3/18/main.tex}
\item There are 1000 sealed envelopes in a box, 10 of them contain a cash prize of
Rs 100 each, 100 of them contain a cash prize of Rs 50 each and 200 of them
contain a cash prize of Rs 10 each and rest do not contain any cash prize. If they
are well shuffled and an envelope is picked up out, what is the probability that it
contains no cash prize?\\
\solution
%\input{exemplar/10/13/3/34/main.tex}
\item 
A die is thrown and a card is selected at random from a deck of 52 playing cards. The probability of getting an even number on the die and a spade card.\\
\solution
%\input{exemplar/12/13/3/78/main.tex}
\item
If 4-digit numbers greater than 5,000 are randomly formed from the digits 0, 1, 3, 5, and 7, what is the probability of forming a number divisible by 5 when:
\begin{enumerate}
    \item The digits are repeated?
    \item The repetition of digits is not allowed?
\end{enumerate}
\solution
%\input{ncert/11/16/4/9/main.tex}
\item Consider the probability space $\brak{\Omega, \mathcal{G}, P}$ where $\Omega = [0,2]$ and $\mathcal{G} = \cbrak{\phi, \Omega, [0,1], (1,2]}$. Let $X$ and $Y$ be two functions on $\Omega$ defined as
\begin{align*}
    X(\omega) = 
    \begin{cases}
        1 & \text{if }\omega \in [0, 1]\\
        2 & \text{if }\omega \in (1, 2]
    \end{cases}
\end{align*}
and
\begin{align*}
    Y(\omega) = 
    \begin{cases}
        2 & \text{if }\omega \in [0, 1.5]\\
        3 & \text{if }\omega \in (1.5, 2].
    \end{cases}
\end{align*}
Then which one of the following statements is true?
\begin{enumerate}
    \item [(A)] $X$ is a random variable with respect to $\mathcal{G}$, but $Y$ is not a random variable with respect to $\mathcal{G}$.
    \item [(B)] $Y$ is a random variable with respect to $\mathcal{G}$, but $X$ is not a random variable with respect to $\mathcal{G}$.
    \item [(C)] Neither $X$ nor $Y$ is a random variable with respect to $\mathcal{G}$.
    \item [(D)] Both $X$ and $Y$ are random variables with respect to $\mathcal{G}$.
\end{enumerate} \hfill (GATE ST 2023)\\
\solution
%\input{gate/ST/2023/14/main.tex}
	\item  A die is loaded in such a way that each odd number is twice as likely to occur as
each even number. Find $P(G)$, where $G$ is the event that a number greater than
3 occurs on a single roll of the die.
\\
\solution
		%\input{exemplar/11/16/3/5/main.tex}
	\item All the jacks, queens and kings are removed from a deck of 52 playing cards. The remaining cards are well shuffled and then one card is drawn at random. Giving ace a value 1 similar value for other cards, find the probability that the card has a value 
		\begin{enumerate}
			\item 7
			\item greater than 7
			\item less than 7
		\end{enumerate}
		%\input{exemplar/10/13/3/30/main.tex}
  \item A Lot consists of 48 mobile phones of which 42 are good, 3 have only minor defects and 3 have major defects.Varnika will buy a phone if it is good but the trader will only buy a mobile if it has no major defects. One phone is selected at random from the lot. What is the probability that it is
\begin{enumerate}
	\item acceptable to Varnika?
            \item acceptable to the trader?
\end{enumerate}
\solution
	%\input{exemplar/10/13/3/40/main.tex}
 \item A student says that if you throw a die, it will show up 1 or not 1. Therefore, the probability of getting 1 and the probability of getting 'not 1' each is equal to $\frac{1}{2}$. Is this correct? Give reasons.\\
 \solution
        %\input{exemplar/10/13/2/9/main.tex}
   \item Four candidates A, B, C, D have ap-
plied for the assignment to coach a school cricket
team. If A is twice as likely to be selected as B, and
B and C are given about the same chance of being
selected, while C is twice as likely to be selected
as D, what are the probabilities that
\begin{enumerate}
\item C will be selected?
\item A will not be selected?
\end{enumerate}
	%\input{exemplar/11/16/3/9/main.tex}
 \item A bag contain 24 balls of which $x$ balls are red, $2x$ are white and $3x$ are blue. A ball is selected at random, What is the probability that it is
\begin{enumerate}[label=\alph*)]
\item not red ?
\item white ?
\end{enumerate}
%\input{exemplar/10/13/3/41/main.tex}
If the letters of the word ASSASSINATION are arranged at random. Find the Probability that
\begin{enumerate}[label=(\alph*)]
\item Four $S's$ come consecutively in the word
\item Two  $I's$ and two $N's$ come together
\item All $A's$ are not coming together
\item No two $A's$ are coming together
\end{enumerate}
%\input{exemplar/11/16/3/14/main.tex}
	\item One urn contains two black balls (labelled B1 and B2) and one white ball. A
	second urn contains one black ball and two white balls (labelled W1 and W2).
	Suppose the following experiment is performed. One of the two urns is chosen
	at random. Next a ball is randomly chosen from the urn. Then a second ball is
	chosen at random from the same urn without replacing the first ball.
	
	\begin{enumerate}
	\item What is the probability that two black balls are chosen?
	
	\item What is the probability that two balls of opposite colour are chosen?
	\end{enumerate}
	\solution
	%\input{exemplar/11/16/3/12/main1.tex}
\end{enumerate}

\item In a certain lottery 10,000 tickets are sold and ten equal prizes are awarded. What is the probability of not getting a prize if you buy (a) one ticket (b) two tickets (c) 10 tickets ?	
\\
\solution
		%\begin{enumerate}[label=\thesection.\arabic*,ref=\thesection.\theenumi]
	\item One card is drawn from a well-shuffled deck of 52 cards. Find the probability of getting
\begin{enumerate}
\item A king of red colour 
\item A face card 
\item A red face card
\item The jack of hearts
\item A spade
\item The queen of diamonds

\end{enumerate}
\solution
		%\input{ncert/10/15/1/14/main.tex}
	\item Five cards—the ten, jack, queen, king and ace of diamonds, are well-shuffled with their face downwards. One card is then picked up at random.
\begin{enumerate}
\item
What is the probability that the card is the queen? 
\item
If the queen is drawn and put aside, what is the probability that the second card picked up is (a) an ace? (b) a queen?\\
\end{enumerate}
\solution
		%\input{ncert/10/15/1/15/defs.tex}
	\item A bag contains $5$ red balls and some blue balls. If the probability of drawing a blue ball is double that if a red ball, determine the number of blue balls in the bag. 
		\\
\solution
		%\input{ncert/10/15/2/3/defs.tex}
	\item A card is selected from a pack of 52 cards.
 \begin{enumerate}[label=(\alph*)] 
                 \item How many points are there in the sample space?
                 \item Calculate the probability that the card is an ace of spades.
                 \item Calculate the probability that the card is (i) an ace and (ii) black card.
 \end{enumerate}
\solution
		%\input{ncert/11/16/3/4/main.tex}
\item Four cards are drawn from a well-shuffled deck of 52 cards. What is the probability of obtaining 3 diamonds and one spade.
\\
\solution
		%\input{ncert/11/16/4/2/defs.tex}
\item In a certain lottery 10,000 tickets are sold and ten equal prizes are awarded. What is the probability of not getting a prize if you buy (a) one ticket (b) two tickets (c) 10 tickets ?	
\\
\solution
		%\input{ncert/11/16/4/4/defs.tex}
		%
\item 
Out of 100 students, two sections of 40 and 60 are formed. If you and your friend are among the 100 students, what is the probability that
\begin{enumerate}
\item you both enter the same section?
\item you both enter the different sections?
\end{enumerate}
\solution
		%\input{ncert/11/16/4/5/defs.tex}
	\item 
The number lock of a suitcase has 4 wheels each labelled with ten digits i.e. from 0 to 9.The lock opens with a sequence of four digits with no repeats.What is the probability of a person getting the right sequence to open the suitcase.
\\
\solution
		%\input{ncert/11/16/4/10/defs.tex}
		%
\item 
Two cards are drawn at random and without replacement from a pack of 52 playing cards. Find the probability that both the cards are black.
\\
\solution
		%\input{ncert/12/13/2/2/defs.tex}
		\item A box of oranges is inspected by examining three randomly selected oranges drawn without replacement. If all the three oranges are good, the box is approved for sale, otherwise, it is rejected. Find the probability that a box containing 15 oranges out of which 12 are good and 3 are bad ones will be approved for sale.
		\label{ncert/12/13/2/3/defs.tex}
		\item Two balls are drawn at random with replacement from a box containing 10 black and 8 red balls. Find the probability that
		\label{ncert/12/13/2/12}
\begin{enumerate}
\item both balls are red.
\item first ball is black and second is red.
\item one of them is black and other is red.
\end{enumerate}

\item In a hostel, 60\% of the students read Hindi newspaper, 40\% read English newspaper and 20\% read both Hindi and English newspapers. A student is selected at random.
		\label{ncert/12/13/2/15}
\begin{enumerate}
\item Find the probability that she reads neither Hindi nor English newspapers.
\item If she reads Hindi newspaper, find the probability that she reads English newspaper.
\item If she reads English newspaper, find the probability that she reads Hindi newspaper.\\
\end{enumerate}
\item The probability of obtaining an even prime number on each die, when a pair of dice is rolled is 
\begin{enumerate}
    \item $0$ 
    
    \item $\frac{1}{3}$ 
    
    \item $\frac{1}{12}$ 
    
    \item $\frac{1}{36}$ 
\end{enumerate}
\solution
		%\input{ncert/12/13/2/17/defs.tex}
	\item A bag contains 4 red and 4 black balls, another bag contains 2 red and 6 black balls. One of the two bags is selected at random and a ball is drawn from the bag which is found to be red. Find the probability that the ball is drawn from the first bag.
\\
\solution
		%\input{ncert/12/13/3/2/main.tex}
  \item
  Cards with numbers 2 to 101 are placed in a box. A card is selected at random.Find the probability that the card has
\begin{enumerate}[label=(\roman*)]
	\item an even number 
	\item a square number
\end{enumerate}
\solution
%\input{exemplar/10/13/3/32/main.tex}
\item
The king, queen and jack of clubs are removed from a deck of 52 playing cards and then well shuffled. Now one card is drawn at random from the remaining cards.  Determine the probability that the card is
\begin{enumerate}[label=(\roman*)]
\item a club
\item 10 of hearts
\end{enumerate}
\solution
%\input{exemplar/10/13/3/29/main.tex}
\item A team of medical students doing their internship have to assist during surgeries
at a city hospital. The probabilities of surgeries rated as very complex, complex,
routine, simple or very simple are respectively, 0.15, 0.20, 0.31, 0.26, .08. Find
the probabilities that a particular surgery will be rated
\begin{enumerate}
	\item complex or very complex;
	\item neither very complex nor very simple;
	\item routine or complex
	\item routine or simple
\end{enumerate}
\solution
%\input{exemplar/11/16/3/8(1)/main.tex}
\item A card is selected from a pack of 52 cards.
\begin{enumerate}[label=(\alph*)]
    \item How many points are there in the sample space?
    \item Calculate the probability that the card is an ace of spades.
    \item Calculate the probability that the card is (i) an ace and (ii) black card.
\end{enumerate}
\solution
%\input{exemplar/11/16/3/4/main2.tex}
\item The probability that a non leap year selected at random will contain 53 sundays.
\\
\solution
%\input{exemplar/10/13/1/19/main.tex}
\item One of the four persons John, Rita, Aslam or Gurpreet will be promoted next
month. Consequently the sample space consists of four elementary outcomes
S = {John promoted, Rita promoted, Aslam promoted, Gurpreet promoted}
You are told that the chances of John’s promotion is same as that of Gurpreet,
Rita’s chances of promotion are twice as likely as Johns. Aslam’s chances are
four times that of John.
\begin{enumerate}
	\item Determine
	\begin{enumerate}
		\item P (John promoted)
		\item P (Rita promoted)
		\item P (Aslam promoted)
		\item P (Gurpreet promoted)
	\end{enumerate}
	\item If A = {John promoted or Gurpreet promoted}, find P (A).
\end{enumerate}
\solution
%\input{exemplar/11/16/3/10/main.tex}
\item A card is drawn from a deck of 52 cards. Find the probability of getting a king or a heart or a red card.\\
\solution
%\input{exemplar/11/16/3/15/main.tex}
\item The probability that a student will pass his examination is 0.73, the probability of
the student getting a compartment is 0.13, and the probability that the student will
either pass or get compartment is 0.96. State True or False.\\
\solution
%\input{exemplar/11/16/3/31/main.tex}
\item A card is selected from a pack of 52 cards\\
\begin{enumerate}[label=(\alph*)]
\item How many points are there in the sample space?
\item Calculate the probability that the cards is an ace of spades.
\item Calculate the probability that the card is (i) an ace (ii)black card.\\
\end{enumerate}
%\input{ncert/11/16/3/4_1/Prob_4.tex}
\item In a non-leap year, the probability of having 53 tuesdays or 53 wednesdays is\\
\solution
%\input{exemplar/11/16/3/18/main.tex}
\item There are 1000 sealed envelopes in a box, 10 of them contain a cash prize of
Rs 100 each, 100 of them contain a cash prize of Rs 50 each and 200 of them
contain a cash prize of Rs 10 each and rest do not contain any cash prize. If they
are well shuffled and an envelope is picked up out, what is the probability that it
contains no cash prize?\\
\solution
%\input{exemplar/10/13/3/34/main.tex}
\item 
A die is thrown and a card is selected at random from a deck of 52 playing cards. The probability of getting an even number on the die and a spade card.\\
\solution
%\input{exemplar/12/13/3/78/main.tex}
\item
If 4-digit numbers greater than 5,000 are randomly formed from the digits 0, 1, 3, 5, and 7, what is the probability of forming a number divisible by 5 when:
\begin{enumerate}
    \item The digits are repeated?
    \item The repetition of digits is not allowed?
\end{enumerate}
\solution
%\input{ncert/11/16/4/9/main.tex}
\item Consider the probability space $\brak{\Omega, \mathcal{G}, P}$ where $\Omega = [0,2]$ and $\mathcal{G} = \cbrak{\phi, \Omega, [0,1], (1,2]}$. Let $X$ and $Y$ be two functions on $\Omega$ defined as
\begin{align*}
    X(\omega) = 
    \begin{cases}
        1 & \text{if }\omega \in [0, 1]\\
        2 & \text{if }\omega \in (1, 2]
    \end{cases}
\end{align*}
and
\begin{align*}
    Y(\omega) = 
    \begin{cases}
        2 & \text{if }\omega \in [0, 1.5]\\
        3 & \text{if }\omega \in (1.5, 2].
    \end{cases}
\end{align*}
Then which one of the following statements is true?
\begin{enumerate}
    \item [(A)] $X$ is a random variable with respect to $\mathcal{G}$, but $Y$ is not a random variable with respect to $\mathcal{G}$.
    \item [(B)] $Y$ is a random variable with respect to $\mathcal{G}$, but $X$ is not a random variable with respect to $\mathcal{G}$.
    \item [(C)] Neither $X$ nor $Y$ is a random variable with respect to $\mathcal{G}$.
    \item [(D)] Both $X$ and $Y$ are random variables with respect to $\mathcal{G}$.
\end{enumerate} \hfill (GATE ST 2023)\\
\solution
%\input{gate/ST/2023/14/main.tex}
	\item  A die is loaded in such a way that each odd number is twice as likely to occur as
each even number. Find $P(G)$, where $G$ is the event that a number greater than
3 occurs on a single roll of the die.
\\
\solution
		%\input{exemplar/11/16/3/5/main.tex}
	\item All the jacks, queens and kings are removed from a deck of 52 playing cards. The remaining cards are well shuffled and then one card is drawn at random. Giving ace a value 1 similar value for other cards, find the probability that the card has a value 
		\begin{enumerate}
			\item 7
			\item greater than 7
			\item less than 7
		\end{enumerate}
		%\input{exemplar/10/13/3/30/main.tex}
  \item A Lot consists of 48 mobile phones of which 42 are good, 3 have only minor defects and 3 have major defects.Varnika will buy a phone if it is good but the trader will only buy a mobile if it has no major defects. One phone is selected at random from the lot. What is the probability that it is
\begin{enumerate}
	\item acceptable to Varnika?
            \item acceptable to the trader?
\end{enumerate}
\solution
	%\input{exemplar/10/13/3/40/main.tex}
 \item A student says that if you throw a die, it will show up 1 or not 1. Therefore, the probability of getting 1 and the probability of getting 'not 1' each is equal to $\frac{1}{2}$. Is this correct? Give reasons.\\
 \solution
        %\input{exemplar/10/13/2/9/main.tex}
   \item Four candidates A, B, C, D have ap-
plied for the assignment to coach a school cricket
team. If A is twice as likely to be selected as B, and
B and C are given about the same chance of being
selected, while C is twice as likely to be selected
as D, what are the probabilities that
\begin{enumerate}
\item C will be selected?
\item A will not be selected?
\end{enumerate}
	%\input{exemplar/11/16/3/9/main.tex}
 \item A bag contain 24 balls of which $x$ balls are red, $2x$ are white and $3x$ are blue. A ball is selected at random, What is the probability that it is
\begin{enumerate}[label=\alph*)]
\item not red ?
\item white ?
\end{enumerate}
%\input{exemplar/10/13/3/41/main.tex}
If the letters of the word ASSASSINATION are arranged at random. Find the Probability that
\begin{enumerate}[label=(\alph*)]
\item Four $S's$ come consecutively in the word
\item Two  $I's$ and two $N's$ come together
\item All $A's$ are not coming together
\item No two $A's$ are coming together
\end{enumerate}
%\input{exemplar/11/16/3/14/main.tex}
	\item One urn contains two black balls (labelled B1 and B2) and one white ball. A
	second urn contains one black ball and two white balls (labelled W1 and W2).
	Suppose the following experiment is performed. One of the two urns is chosen
	at random. Next a ball is randomly chosen from the urn. Then a second ball is
	chosen at random from the same urn without replacing the first ball.
	
	\begin{enumerate}
	\item What is the probability that two black balls are chosen?
	
	\item What is the probability that two balls of opposite colour are chosen?
	\end{enumerate}
	\solution
	%\input{exemplar/11/16/3/12/main1.tex}
\end{enumerate}

		%
\item 
Out of 100 students, two sections of 40 and 60 are formed. If you and your friend are among the 100 students, what is the probability that
\begin{enumerate}
\item you both enter the same section?
\item you both enter the different sections?
\end{enumerate}
\solution
		%\begin{enumerate}[label=\thesection.\arabic*,ref=\thesection.\theenumi]
	\item One card is drawn from a well-shuffled deck of 52 cards. Find the probability of getting
\begin{enumerate}
\item A king of red colour 
\item A face card 
\item A red face card
\item The jack of hearts
\item A spade
\item The queen of diamonds

\end{enumerate}
\solution
		%\input{ncert/10/15/1/14/main.tex}
	\item Five cards—the ten, jack, queen, king and ace of diamonds, are well-shuffled with their face downwards. One card is then picked up at random.
\begin{enumerate}
\item
What is the probability that the card is the queen? 
\item
If the queen is drawn and put aside, what is the probability that the second card picked up is (a) an ace? (b) a queen?\\
\end{enumerate}
\solution
		%\input{ncert/10/15/1/15/defs.tex}
	\item A bag contains $5$ red balls and some blue balls. If the probability of drawing a blue ball is double that if a red ball, determine the number of blue balls in the bag. 
		\\
\solution
		%\input{ncert/10/15/2/3/defs.tex}
	\item A card is selected from a pack of 52 cards.
 \begin{enumerate}[label=(\alph*)] 
                 \item How many points are there in the sample space?
                 \item Calculate the probability that the card is an ace of spades.
                 \item Calculate the probability that the card is (i) an ace and (ii) black card.
 \end{enumerate}
\solution
		%\input{ncert/11/16/3/4/main.tex}
\item Four cards are drawn from a well-shuffled deck of 52 cards. What is the probability of obtaining 3 diamonds and one spade.
\\
\solution
		%\input{ncert/11/16/4/2/defs.tex}
\item In a certain lottery 10,000 tickets are sold and ten equal prizes are awarded. What is the probability of not getting a prize if you buy (a) one ticket (b) two tickets (c) 10 tickets ?	
\\
\solution
		%\input{ncert/11/16/4/4/defs.tex}
		%
\item 
Out of 100 students, two sections of 40 and 60 are formed. If you and your friend are among the 100 students, what is the probability that
\begin{enumerate}
\item you both enter the same section?
\item you both enter the different sections?
\end{enumerate}
\solution
		%\input{ncert/11/16/4/5/defs.tex}
	\item 
The number lock of a suitcase has 4 wheels each labelled with ten digits i.e. from 0 to 9.The lock opens with a sequence of four digits with no repeats.What is the probability of a person getting the right sequence to open the suitcase.
\\
\solution
		%\input{ncert/11/16/4/10/defs.tex}
		%
\item 
Two cards are drawn at random and without replacement from a pack of 52 playing cards. Find the probability that both the cards are black.
\\
\solution
		%\input{ncert/12/13/2/2/defs.tex}
		\item A box of oranges is inspected by examining three randomly selected oranges drawn without replacement. If all the three oranges are good, the box is approved for sale, otherwise, it is rejected. Find the probability that a box containing 15 oranges out of which 12 are good and 3 are bad ones will be approved for sale.
		\label{ncert/12/13/2/3/defs.tex}
		\item Two balls are drawn at random with replacement from a box containing 10 black and 8 red balls. Find the probability that
		\label{ncert/12/13/2/12}
\begin{enumerate}
\item both balls are red.
\item first ball is black and second is red.
\item one of them is black and other is red.
\end{enumerate}

\item In a hostel, 60\% of the students read Hindi newspaper, 40\% read English newspaper and 20\% read both Hindi and English newspapers. A student is selected at random.
		\label{ncert/12/13/2/15}
\begin{enumerate}
\item Find the probability that she reads neither Hindi nor English newspapers.
\item If she reads Hindi newspaper, find the probability that she reads English newspaper.
\item If she reads English newspaper, find the probability that she reads Hindi newspaper.\\
\end{enumerate}
\item The probability of obtaining an even prime number on each die, when a pair of dice is rolled is 
\begin{enumerate}
    \item $0$ 
    
    \item $\frac{1}{3}$ 
    
    \item $\frac{1}{12}$ 
    
    \item $\frac{1}{36}$ 
\end{enumerate}
\solution
		%\input{ncert/12/13/2/17/defs.tex}
	\item A bag contains 4 red and 4 black balls, another bag contains 2 red and 6 black balls. One of the two bags is selected at random and a ball is drawn from the bag which is found to be red. Find the probability that the ball is drawn from the first bag.
\\
\solution
		%\input{ncert/12/13/3/2/main.tex}
  \item
  Cards with numbers 2 to 101 are placed in a box. A card is selected at random.Find the probability that the card has
\begin{enumerate}[label=(\roman*)]
	\item an even number 
	\item a square number
\end{enumerate}
\solution
%\input{exemplar/10/13/3/32/main.tex}
\item
The king, queen and jack of clubs are removed from a deck of 52 playing cards and then well shuffled. Now one card is drawn at random from the remaining cards.  Determine the probability that the card is
\begin{enumerate}[label=(\roman*)]
\item a club
\item 10 of hearts
\end{enumerate}
\solution
%\input{exemplar/10/13/3/29/main.tex}
\item A team of medical students doing their internship have to assist during surgeries
at a city hospital. The probabilities of surgeries rated as very complex, complex,
routine, simple or very simple are respectively, 0.15, 0.20, 0.31, 0.26, .08. Find
the probabilities that a particular surgery will be rated
\begin{enumerate}
	\item complex or very complex;
	\item neither very complex nor very simple;
	\item routine or complex
	\item routine or simple
\end{enumerate}
\solution
%\input{exemplar/11/16/3/8(1)/main.tex}
\item A card is selected from a pack of 52 cards.
\begin{enumerate}[label=(\alph*)]
    \item How many points are there in the sample space?
    \item Calculate the probability that the card is an ace of spades.
    \item Calculate the probability that the card is (i) an ace and (ii) black card.
\end{enumerate}
\solution
%\input{exemplar/11/16/3/4/main2.tex}
\item The probability that a non leap year selected at random will contain 53 sundays.
\\
\solution
%\input{exemplar/10/13/1/19/main.tex}
\item One of the four persons John, Rita, Aslam or Gurpreet will be promoted next
month. Consequently the sample space consists of four elementary outcomes
S = {John promoted, Rita promoted, Aslam promoted, Gurpreet promoted}
You are told that the chances of John’s promotion is same as that of Gurpreet,
Rita’s chances of promotion are twice as likely as Johns. Aslam’s chances are
four times that of John.
\begin{enumerate}
	\item Determine
	\begin{enumerate}
		\item P (John promoted)
		\item P (Rita promoted)
		\item P (Aslam promoted)
		\item P (Gurpreet promoted)
	\end{enumerate}
	\item If A = {John promoted or Gurpreet promoted}, find P (A).
\end{enumerate}
\solution
%\input{exemplar/11/16/3/10/main.tex}
\item A card is drawn from a deck of 52 cards. Find the probability of getting a king or a heart or a red card.\\
\solution
%\input{exemplar/11/16/3/15/main.tex}
\item The probability that a student will pass his examination is 0.73, the probability of
the student getting a compartment is 0.13, and the probability that the student will
either pass or get compartment is 0.96. State True or False.\\
\solution
%\input{exemplar/11/16/3/31/main.tex}
\item A card is selected from a pack of 52 cards\\
\begin{enumerate}[label=(\alph*)]
\item How many points are there in the sample space?
\item Calculate the probability that the cards is an ace of spades.
\item Calculate the probability that the card is (i) an ace (ii)black card.\\
\end{enumerate}
%\input{ncert/11/16/3/4_1/Prob_4.tex}
\item In a non-leap year, the probability of having 53 tuesdays or 53 wednesdays is\\
\solution
%\input{exemplar/11/16/3/18/main.tex}
\item There are 1000 sealed envelopes in a box, 10 of them contain a cash prize of
Rs 100 each, 100 of them contain a cash prize of Rs 50 each and 200 of them
contain a cash prize of Rs 10 each and rest do not contain any cash prize. If they
are well shuffled and an envelope is picked up out, what is the probability that it
contains no cash prize?\\
\solution
%\input{exemplar/10/13/3/34/main.tex}
\item 
A die is thrown and a card is selected at random from a deck of 52 playing cards. The probability of getting an even number on the die and a spade card.\\
\solution
%\input{exemplar/12/13/3/78/main.tex}
\item
If 4-digit numbers greater than 5,000 are randomly formed from the digits 0, 1, 3, 5, and 7, what is the probability of forming a number divisible by 5 when:
\begin{enumerate}
    \item The digits are repeated?
    \item The repetition of digits is not allowed?
\end{enumerate}
\solution
%\input{ncert/11/16/4/9/main.tex}
\item Consider the probability space $\brak{\Omega, \mathcal{G}, P}$ where $\Omega = [0,2]$ and $\mathcal{G} = \cbrak{\phi, \Omega, [0,1], (1,2]}$. Let $X$ and $Y$ be two functions on $\Omega$ defined as
\begin{align*}
    X(\omega) = 
    \begin{cases}
        1 & \text{if }\omega \in [0, 1]\\
        2 & \text{if }\omega \in (1, 2]
    \end{cases}
\end{align*}
and
\begin{align*}
    Y(\omega) = 
    \begin{cases}
        2 & \text{if }\omega \in [0, 1.5]\\
        3 & \text{if }\omega \in (1.5, 2].
    \end{cases}
\end{align*}
Then which one of the following statements is true?
\begin{enumerate}
    \item [(A)] $X$ is a random variable with respect to $\mathcal{G}$, but $Y$ is not a random variable with respect to $\mathcal{G}$.
    \item [(B)] $Y$ is a random variable with respect to $\mathcal{G}$, but $X$ is not a random variable with respect to $\mathcal{G}$.
    \item [(C)] Neither $X$ nor $Y$ is a random variable with respect to $\mathcal{G}$.
    \item [(D)] Both $X$ and $Y$ are random variables with respect to $\mathcal{G}$.
\end{enumerate} \hfill (GATE ST 2023)\\
\solution
%\input{gate/ST/2023/14/main.tex}
	\item  A die is loaded in such a way that each odd number is twice as likely to occur as
each even number. Find $P(G)$, where $G$ is the event that a number greater than
3 occurs on a single roll of the die.
\\
\solution
		%\input{exemplar/11/16/3/5/main.tex}
	\item All the jacks, queens and kings are removed from a deck of 52 playing cards. The remaining cards are well shuffled and then one card is drawn at random. Giving ace a value 1 similar value for other cards, find the probability that the card has a value 
		\begin{enumerate}
			\item 7
			\item greater than 7
			\item less than 7
		\end{enumerate}
		%\input{exemplar/10/13/3/30/main.tex}
  \item A Lot consists of 48 mobile phones of which 42 are good, 3 have only minor defects and 3 have major defects.Varnika will buy a phone if it is good but the trader will only buy a mobile if it has no major defects. One phone is selected at random from the lot. What is the probability that it is
\begin{enumerate}
	\item acceptable to Varnika?
            \item acceptable to the trader?
\end{enumerate}
\solution
	%\input{exemplar/10/13/3/40/main.tex}
 \item A student says that if you throw a die, it will show up 1 or not 1. Therefore, the probability of getting 1 and the probability of getting 'not 1' each is equal to $\frac{1}{2}$. Is this correct? Give reasons.\\
 \solution
        %\input{exemplar/10/13/2/9/main.tex}
   \item Four candidates A, B, C, D have ap-
plied for the assignment to coach a school cricket
team. If A is twice as likely to be selected as B, and
B and C are given about the same chance of being
selected, while C is twice as likely to be selected
as D, what are the probabilities that
\begin{enumerate}
\item C will be selected?
\item A will not be selected?
\end{enumerate}
	%\input{exemplar/11/16/3/9/main.tex}
 \item A bag contain 24 balls of which $x$ balls are red, $2x$ are white and $3x$ are blue. A ball is selected at random, What is the probability that it is
\begin{enumerate}[label=\alph*)]
\item not red ?
\item white ?
\end{enumerate}
%\input{exemplar/10/13/3/41/main.tex}
If the letters of the word ASSASSINATION are arranged at random. Find the Probability that
\begin{enumerate}[label=(\alph*)]
\item Four $S's$ come consecutively in the word
\item Two  $I's$ and two $N's$ come together
\item All $A's$ are not coming together
\item No two $A's$ are coming together
\end{enumerate}
%\input{exemplar/11/16/3/14/main.tex}
	\item One urn contains two black balls (labelled B1 and B2) and one white ball. A
	second urn contains one black ball and two white balls (labelled W1 and W2).
	Suppose the following experiment is performed. One of the two urns is chosen
	at random. Next a ball is randomly chosen from the urn. Then a second ball is
	chosen at random from the same urn without replacing the first ball.
	
	\begin{enumerate}
	\item What is the probability that two black balls are chosen?
	
	\item What is the probability that two balls of opposite colour are chosen?
	\end{enumerate}
	\solution
	%\input{exemplar/11/16/3/12/main1.tex}
\end{enumerate}

	\item 
The number lock of a suitcase has 4 wheels each labelled with ten digits i.e. from 0 to 9.The lock opens with a sequence of four digits with no repeats.What is the probability of a person getting the right sequence to open the suitcase.
\\
\solution
		%\begin{enumerate}[label=\thesection.\arabic*,ref=\thesection.\theenumi]
	\item One card is drawn from a well-shuffled deck of 52 cards. Find the probability of getting
\begin{enumerate}
\item A king of red colour 
\item A face card 
\item A red face card
\item The jack of hearts
\item A spade
\item The queen of diamonds

\end{enumerate}
\solution
		%\input{ncert/10/15/1/14/main.tex}
	\item Five cards—the ten, jack, queen, king and ace of diamonds, are well-shuffled with their face downwards. One card is then picked up at random.
\begin{enumerate}
\item
What is the probability that the card is the queen? 
\item
If the queen is drawn and put aside, what is the probability that the second card picked up is (a) an ace? (b) a queen?\\
\end{enumerate}
\solution
		%\input{ncert/10/15/1/15/defs.tex}
	\item A bag contains $5$ red balls and some blue balls. If the probability of drawing a blue ball is double that if a red ball, determine the number of blue balls in the bag. 
		\\
\solution
		%\input{ncert/10/15/2/3/defs.tex}
	\item A card is selected from a pack of 52 cards.
 \begin{enumerate}[label=(\alph*)] 
                 \item How many points are there in the sample space?
                 \item Calculate the probability that the card is an ace of spades.
                 \item Calculate the probability that the card is (i) an ace and (ii) black card.
 \end{enumerate}
\solution
		%\input{ncert/11/16/3/4/main.tex}
\item Four cards are drawn from a well-shuffled deck of 52 cards. What is the probability of obtaining 3 diamonds and one spade.
\\
\solution
		%\input{ncert/11/16/4/2/defs.tex}
\item In a certain lottery 10,000 tickets are sold and ten equal prizes are awarded. What is the probability of not getting a prize if you buy (a) one ticket (b) two tickets (c) 10 tickets ?	
\\
\solution
		%\input{ncert/11/16/4/4/defs.tex}
		%
\item 
Out of 100 students, two sections of 40 and 60 are formed. If you and your friend are among the 100 students, what is the probability that
\begin{enumerate}
\item you both enter the same section?
\item you both enter the different sections?
\end{enumerate}
\solution
		%\input{ncert/11/16/4/5/defs.tex}
	\item 
The number lock of a suitcase has 4 wheels each labelled with ten digits i.e. from 0 to 9.The lock opens with a sequence of four digits with no repeats.What is the probability of a person getting the right sequence to open the suitcase.
\\
\solution
		%\input{ncert/11/16/4/10/defs.tex}
		%
\item 
Two cards are drawn at random and without replacement from a pack of 52 playing cards. Find the probability that both the cards are black.
\\
\solution
		%\input{ncert/12/13/2/2/defs.tex}
		\item A box of oranges is inspected by examining three randomly selected oranges drawn without replacement. If all the three oranges are good, the box is approved for sale, otherwise, it is rejected. Find the probability that a box containing 15 oranges out of which 12 are good and 3 are bad ones will be approved for sale.
		\label{ncert/12/13/2/3/defs.tex}
		\item Two balls are drawn at random with replacement from a box containing 10 black and 8 red balls. Find the probability that
		\label{ncert/12/13/2/12}
\begin{enumerate}
\item both balls are red.
\item first ball is black and second is red.
\item one of them is black and other is red.
\end{enumerate}

\item In a hostel, 60\% of the students read Hindi newspaper, 40\% read English newspaper and 20\% read both Hindi and English newspapers. A student is selected at random.
		\label{ncert/12/13/2/15}
\begin{enumerate}
\item Find the probability that she reads neither Hindi nor English newspapers.
\item If she reads Hindi newspaper, find the probability that she reads English newspaper.
\item If she reads English newspaper, find the probability that she reads Hindi newspaper.\\
\end{enumerate}
\item The probability of obtaining an even prime number on each die, when a pair of dice is rolled is 
\begin{enumerate}
    \item $0$ 
    
    \item $\frac{1}{3}$ 
    
    \item $\frac{1}{12}$ 
    
    \item $\frac{1}{36}$ 
\end{enumerate}
\solution
		%\input{ncert/12/13/2/17/defs.tex}
	\item A bag contains 4 red and 4 black balls, another bag contains 2 red and 6 black balls. One of the two bags is selected at random and a ball is drawn from the bag which is found to be red. Find the probability that the ball is drawn from the first bag.
\\
\solution
		%\input{ncert/12/13/3/2/main.tex}
  \item
  Cards with numbers 2 to 101 are placed in a box. A card is selected at random.Find the probability that the card has
\begin{enumerate}[label=(\roman*)]
	\item an even number 
	\item a square number
\end{enumerate}
\solution
%\input{exemplar/10/13/3/32/main.tex}
\item
The king, queen and jack of clubs are removed from a deck of 52 playing cards and then well shuffled. Now one card is drawn at random from the remaining cards.  Determine the probability that the card is
\begin{enumerate}[label=(\roman*)]
\item a club
\item 10 of hearts
\end{enumerate}
\solution
%\input{exemplar/10/13/3/29/main.tex}
\item A team of medical students doing their internship have to assist during surgeries
at a city hospital. The probabilities of surgeries rated as very complex, complex,
routine, simple or very simple are respectively, 0.15, 0.20, 0.31, 0.26, .08. Find
the probabilities that a particular surgery will be rated
\begin{enumerate}
	\item complex or very complex;
	\item neither very complex nor very simple;
	\item routine or complex
	\item routine or simple
\end{enumerate}
\solution
%\input{exemplar/11/16/3/8(1)/main.tex}
\item A card is selected from a pack of 52 cards.
\begin{enumerate}[label=(\alph*)]
    \item How many points are there in the sample space?
    \item Calculate the probability that the card is an ace of spades.
    \item Calculate the probability that the card is (i) an ace and (ii) black card.
\end{enumerate}
\solution
%\input{exemplar/11/16/3/4/main2.tex}
\item The probability that a non leap year selected at random will contain 53 sundays.
\\
\solution
%\input{exemplar/10/13/1/19/main.tex}
\item One of the four persons John, Rita, Aslam or Gurpreet will be promoted next
month. Consequently the sample space consists of four elementary outcomes
S = {John promoted, Rita promoted, Aslam promoted, Gurpreet promoted}
You are told that the chances of John’s promotion is same as that of Gurpreet,
Rita’s chances of promotion are twice as likely as Johns. Aslam’s chances are
four times that of John.
\begin{enumerate}
	\item Determine
	\begin{enumerate}
		\item P (John promoted)
		\item P (Rita promoted)
		\item P (Aslam promoted)
		\item P (Gurpreet promoted)
	\end{enumerate}
	\item If A = {John promoted or Gurpreet promoted}, find P (A).
\end{enumerate}
\solution
%\input{exemplar/11/16/3/10/main.tex}
\item A card is drawn from a deck of 52 cards. Find the probability of getting a king or a heart or a red card.\\
\solution
%\input{exemplar/11/16/3/15/main.tex}
\item The probability that a student will pass his examination is 0.73, the probability of
the student getting a compartment is 0.13, and the probability that the student will
either pass or get compartment is 0.96. State True or False.\\
\solution
%\input{exemplar/11/16/3/31/main.tex}
\item A card is selected from a pack of 52 cards\\
\begin{enumerate}[label=(\alph*)]
\item How many points are there in the sample space?
\item Calculate the probability that the cards is an ace of spades.
\item Calculate the probability that the card is (i) an ace (ii)black card.\\
\end{enumerate}
%\input{ncert/11/16/3/4_1/Prob_4.tex}
\item In a non-leap year, the probability of having 53 tuesdays or 53 wednesdays is\\
\solution
%\input{exemplar/11/16/3/18/main.tex}
\item There are 1000 sealed envelopes in a box, 10 of them contain a cash prize of
Rs 100 each, 100 of them contain a cash prize of Rs 50 each and 200 of them
contain a cash prize of Rs 10 each and rest do not contain any cash prize. If they
are well shuffled and an envelope is picked up out, what is the probability that it
contains no cash prize?\\
\solution
%\input{exemplar/10/13/3/34/main.tex}
\item 
A die is thrown and a card is selected at random from a deck of 52 playing cards. The probability of getting an even number on the die and a spade card.\\
\solution
%\input{exemplar/12/13/3/78/main.tex}
\item
If 4-digit numbers greater than 5,000 are randomly formed from the digits 0, 1, 3, 5, and 7, what is the probability of forming a number divisible by 5 when:
\begin{enumerate}
    \item The digits are repeated?
    \item The repetition of digits is not allowed?
\end{enumerate}
\solution
%\input{ncert/11/16/4/9/main.tex}
\item Consider the probability space $\brak{\Omega, \mathcal{G}, P}$ where $\Omega = [0,2]$ and $\mathcal{G} = \cbrak{\phi, \Omega, [0,1], (1,2]}$. Let $X$ and $Y$ be two functions on $\Omega$ defined as
\begin{align*}
    X(\omega) = 
    \begin{cases}
        1 & \text{if }\omega \in [0, 1]\\
        2 & \text{if }\omega \in (1, 2]
    \end{cases}
\end{align*}
and
\begin{align*}
    Y(\omega) = 
    \begin{cases}
        2 & \text{if }\omega \in [0, 1.5]\\
        3 & \text{if }\omega \in (1.5, 2].
    \end{cases}
\end{align*}
Then which one of the following statements is true?
\begin{enumerate}
    \item [(A)] $X$ is a random variable with respect to $\mathcal{G}$, but $Y$ is not a random variable with respect to $\mathcal{G}$.
    \item [(B)] $Y$ is a random variable with respect to $\mathcal{G}$, but $X$ is not a random variable with respect to $\mathcal{G}$.
    \item [(C)] Neither $X$ nor $Y$ is a random variable with respect to $\mathcal{G}$.
    \item [(D)] Both $X$ and $Y$ are random variables with respect to $\mathcal{G}$.
\end{enumerate} \hfill (GATE ST 2023)\\
\solution
%\input{gate/ST/2023/14/main.tex}
	\item  A die is loaded in such a way that each odd number is twice as likely to occur as
each even number. Find $P(G)$, where $G$ is the event that a number greater than
3 occurs on a single roll of the die.
\\
\solution
		%\input{exemplar/11/16/3/5/main.tex}
	\item All the jacks, queens and kings are removed from a deck of 52 playing cards. The remaining cards are well shuffled and then one card is drawn at random. Giving ace a value 1 similar value for other cards, find the probability that the card has a value 
		\begin{enumerate}
			\item 7
			\item greater than 7
			\item less than 7
		\end{enumerate}
		%\input{exemplar/10/13/3/30/main.tex}
  \item A Lot consists of 48 mobile phones of which 42 are good, 3 have only minor defects and 3 have major defects.Varnika will buy a phone if it is good but the trader will only buy a mobile if it has no major defects. One phone is selected at random from the lot. What is the probability that it is
\begin{enumerate}
	\item acceptable to Varnika?
            \item acceptable to the trader?
\end{enumerate}
\solution
	%\input{exemplar/10/13/3/40/main.tex}
 \item A student says that if you throw a die, it will show up 1 or not 1. Therefore, the probability of getting 1 and the probability of getting 'not 1' each is equal to $\frac{1}{2}$. Is this correct? Give reasons.\\
 \solution
        %\input{exemplar/10/13/2/9/main.tex}
   \item Four candidates A, B, C, D have ap-
plied for the assignment to coach a school cricket
team. If A is twice as likely to be selected as B, and
B and C are given about the same chance of being
selected, while C is twice as likely to be selected
as D, what are the probabilities that
\begin{enumerate}
\item C will be selected?
\item A will not be selected?
\end{enumerate}
	%\input{exemplar/11/16/3/9/main.tex}
 \item A bag contain 24 balls of which $x$ balls are red, $2x$ are white and $3x$ are blue. A ball is selected at random, What is the probability that it is
\begin{enumerate}[label=\alph*)]
\item not red ?
\item white ?
\end{enumerate}
%\input{exemplar/10/13/3/41/main.tex}
If the letters of the word ASSASSINATION are arranged at random. Find the Probability that
\begin{enumerate}[label=(\alph*)]
\item Four $S's$ come consecutively in the word
\item Two  $I's$ and two $N's$ come together
\item All $A's$ are not coming together
\item No two $A's$ are coming together
\end{enumerate}
%\input{exemplar/11/16/3/14/main.tex}
	\item One urn contains two black balls (labelled B1 and B2) and one white ball. A
	second urn contains one black ball and two white balls (labelled W1 and W2).
	Suppose the following experiment is performed. One of the two urns is chosen
	at random. Next a ball is randomly chosen from the urn. Then a second ball is
	chosen at random from the same urn without replacing the first ball.
	
	\begin{enumerate}
	\item What is the probability that two black balls are chosen?
	
	\item What is the probability that two balls of opposite colour are chosen?
	\end{enumerate}
	\solution
	%\input{exemplar/11/16/3/12/main1.tex}
\end{enumerate}

		%
\item 
Two cards are drawn at random and without replacement from a pack of 52 playing cards. Find the probability that both the cards are black.
\\
\solution
		%\begin{enumerate}[label=\thesection.\arabic*,ref=\thesection.\theenumi]
	\item One card is drawn from a well-shuffled deck of 52 cards. Find the probability of getting
\begin{enumerate}
\item A king of red colour 
\item A face card 
\item A red face card
\item The jack of hearts
\item A spade
\item The queen of diamonds

\end{enumerate}
\solution
		%\input{ncert/10/15/1/14/main.tex}
	\item Five cards—the ten, jack, queen, king and ace of diamonds, are well-shuffled with their face downwards. One card is then picked up at random.
\begin{enumerate}
\item
What is the probability that the card is the queen? 
\item
If the queen is drawn and put aside, what is the probability that the second card picked up is (a) an ace? (b) a queen?\\
\end{enumerate}
\solution
		%\input{ncert/10/15/1/15/defs.tex}
	\item A bag contains $5$ red balls and some blue balls. If the probability of drawing a blue ball is double that if a red ball, determine the number of blue balls in the bag. 
		\\
\solution
		%\input{ncert/10/15/2/3/defs.tex}
	\item A card is selected from a pack of 52 cards.
 \begin{enumerate}[label=(\alph*)] 
                 \item How many points are there in the sample space?
                 \item Calculate the probability that the card is an ace of spades.
                 \item Calculate the probability that the card is (i) an ace and (ii) black card.
 \end{enumerate}
\solution
		%\input{ncert/11/16/3/4/main.tex}
\item Four cards are drawn from a well-shuffled deck of 52 cards. What is the probability of obtaining 3 diamonds and one spade.
\\
\solution
		%\input{ncert/11/16/4/2/defs.tex}
\item In a certain lottery 10,000 tickets are sold and ten equal prizes are awarded. What is the probability of not getting a prize if you buy (a) one ticket (b) two tickets (c) 10 tickets ?	
\\
\solution
		%\input{ncert/11/16/4/4/defs.tex}
		%
\item 
Out of 100 students, two sections of 40 and 60 are formed. If you and your friend are among the 100 students, what is the probability that
\begin{enumerate}
\item you both enter the same section?
\item you both enter the different sections?
\end{enumerate}
\solution
		%\input{ncert/11/16/4/5/defs.tex}
	\item 
The number lock of a suitcase has 4 wheels each labelled with ten digits i.e. from 0 to 9.The lock opens with a sequence of four digits with no repeats.What is the probability of a person getting the right sequence to open the suitcase.
\\
\solution
		%\input{ncert/11/16/4/10/defs.tex}
		%
\item 
Two cards are drawn at random and without replacement from a pack of 52 playing cards. Find the probability that both the cards are black.
\\
\solution
		%\input{ncert/12/13/2/2/defs.tex}
		\item A box of oranges is inspected by examining three randomly selected oranges drawn without replacement. If all the three oranges are good, the box is approved for sale, otherwise, it is rejected. Find the probability that a box containing 15 oranges out of which 12 are good and 3 are bad ones will be approved for sale.
		\label{ncert/12/13/2/3/defs.tex}
		\item Two balls are drawn at random with replacement from a box containing 10 black and 8 red balls. Find the probability that
		\label{ncert/12/13/2/12}
\begin{enumerate}
\item both balls are red.
\item first ball is black and second is red.
\item one of them is black and other is red.
\end{enumerate}

\item In a hostel, 60\% of the students read Hindi newspaper, 40\% read English newspaper and 20\% read both Hindi and English newspapers. A student is selected at random.
		\label{ncert/12/13/2/15}
\begin{enumerate}
\item Find the probability that she reads neither Hindi nor English newspapers.
\item If she reads Hindi newspaper, find the probability that she reads English newspaper.
\item If she reads English newspaper, find the probability that she reads Hindi newspaper.\\
\end{enumerate}
\item The probability of obtaining an even prime number on each die, when a pair of dice is rolled is 
\begin{enumerate}
    \item $0$ 
    
    \item $\frac{1}{3}$ 
    
    \item $\frac{1}{12}$ 
    
    \item $\frac{1}{36}$ 
\end{enumerate}
\solution
		%\input{ncert/12/13/2/17/defs.tex}
	\item A bag contains 4 red and 4 black balls, another bag contains 2 red and 6 black balls. One of the two bags is selected at random and a ball is drawn from the bag which is found to be red. Find the probability that the ball is drawn from the first bag.
\\
\solution
		%\input{ncert/12/13/3/2/main.tex}
  \item
  Cards with numbers 2 to 101 are placed in a box. A card is selected at random.Find the probability that the card has
\begin{enumerate}[label=(\roman*)]
	\item an even number 
	\item a square number
\end{enumerate}
\solution
%\input{exemplar/10/13/3/32/main.tex}
\item
The king, queen and jack of clubs are removed from a deck of 52 playing cards and then well shuffled. Now one card is drawn at random from the remaining cards.  Determine the probability that the card is
\begin{enumerate}[label=(\roman*)]
\item a club
\item 10 of hearts
\end{enumerate}
\solution
%\input{exemplar/10/13/3/29/main.tex}
\item A team of medical students doing their internship have to assist during surgeries
at a city hospital. The probabilities of surgeries rated as very complex, complex,
routine, simple or very simple are respectively, 0.15, 0.20, 0.31, 0.26, .08. Find
the probabilities that a particular surgery will be rated
\begin{enumerate}
	\item complex or very complex;
	\item neither very complex nor very simple;
	\item routine or complex
	\item routine or simple
\end{enumerate}
\solution
%\input{exemplar/11/16/3/8(1)/main.tex}
\item A card is selected from a pack of 52 cards.
\begin{enumerate}[label=(\alph*)]
    \item How many points are there in the sample space?
    \item Calculate the probability that the card is an ace of spades.
    \item Calculate the probability that the card is (i) an ace and (ii) black card.
\end{enumerate}
\solution
%\input{exemplar/11/16/3/4/main2.tex}
\item The probability that a non leap year selected at random will contain 53 sundays.
\\
\solution
%\input{exemplar/10/13/1/19/main.tex}
\item One of the four persons John, Rita, Aslam or Gurpreet will be promoted next
month. Consequently the sample space consists of four elementary outcomes
S = {John promoted, Rita promoted, Aslam promoted, Gurpreet promoted}
You are told that the chances of John’s promotion is same as that of Gurpreet,
Rita’s chances of promotion are twice as likely as Johns. Aslam’s chances are
four times that of John.
\begin{enumerate}
	\item Determine
	\begin{enumerate}
		\item P (John promoted)
		\item P (Rita promoted)
		\item P (Aslam promoted)
		\item P (Gurpreet promoted)
	\end{enumerate}
	\item If A = {John promoted or Gurpreet promoted}, find P (A).
\end{enumerate}
\solution
%\input{exemplar/11/16/3/10/main.tex}
\item A card is drawn from a deck of 52 cards. Find the probability of getting a king or a heart or a red card.\\
\solution
%\input{exemplar/11/16/3/15/main.tex}
\item The probability that a student will pass his examination is 0.73, the probability of
the student getting a compartment is 0.13, and the probability that the student will
either pass or get compartment is 0.96. State True or False.\\
\solution
%\input{exemplar/11/16/3/31/main.tex}
\item A card is selected from a pack of 52 cards\\
\begin{enumerate}[label=(\alph*)]
\item How many points are there in the sample space?
\item Calculate the probability that the cards is an ace of spades.
\item Calculate the probability that the card is (i) an ace (ii)black card.\\
\end{enumerate}
%\input{ncert/11/16/3/4_1/Prob_4.tex}
\item In a non-leap year, the probability of having 53 tuesdays or 53 wednesdays is\\
\solution
%\input{exemplar/11/16/3/18/main.tex}
\item There are 1000 sealed envelopes in a box, 10 of them contain a cash prize of
Rs 100 each, 100 of them contain a cash prize of Rs 50 each and 200 of them
contain a cash prize of Rs 10 each and rest do not contain any cash prize. If they
are well shuffled and an envelope is picked up out, what is the probability that it
contains no cash prize?\\
\solution
%\input{exemplar/10/13/3/34/main.tex}
\item 
A die is thrown and a card is selected at random from a deck of 52 playing cards. The probability of getting an even number on the die and a spade card.\\
\solution
%\input{exemplar/12/13/3/78/main.tex}
\item
If 4-digit numbers greater than 5,000 are randomly formed from the digits 0, 1, 3, 5, and 7, what is the probability of forming a number divisible by 5 when:
\begin{enumerate}
    \item The digits are repeated?
    \item The repetition of digits is not allowed?
\end{enumerate}
\solution
%\input{ncert/11/16/4/9/main.tex}
\item Consider the probability space $\brak{\Omega, \mathcal{G}, P}$ where $\Omega = [0,2]$ and $\mathcal{G} = \cbrak{\phi, \Omega, [0,1], (1,2]}$. Let $X$ and $Y$ be two functions on $\Omega$ defined as
\begin{align*}
    X(\omega) = 
    \begin{cases}
        1 & \text{if }\omega \in [0, 1]\\
        2 & \text{if }\omega \in (1, 2]
    \end{cases}
\end{align*}
and
\begin{align*}
    Y(\omega) = 
    \begin{cases}
        2 & \text{if }\omega \in [0, 1.5]\\
        3 & \text{if }\omega \in (1.5, 2].
    \end{cases}
\end{align*}
Then which one of the following statements is true?
\begin{enumerate}
    \item [(A)] $X$ is a random variable with respect to $\mathcal{G}$, but $Y$ is not a random variable with respect to $\mathcal{G}$.
    \item [(B)] $Y$ is a random variable with respect to $\mathcal{G}$, but $X$ is not a random variable with respect to $\mathcal{G}$.
    \item [(C)] Neither $X$ nor $Y$ is a random variable with respect to $\mathcal{G}$.
    \item [(D)] Both $X$ and $Y$ are random variables with respect to $\mathcal{G}$.
\end{enumerate} \hfill (GATE ST 2023)\\
\solution
%\input{gate/ST/2023/14/main.tex}
	\item  A die is loaded in such a way that each odd number is twice as likely to occur as
each even number. Find $P(G)$, where $G$ is the event that a number greater than
3 occurs on a single roll of the die.
\\
\solution
		%\input{exemplar/11/16/3/5/main.tex}
	\item All the jacks, queens and kings are removed from a deck of 52 playing cards. The remaining cards are well shuffled and then one card is drawn at random. Giving ace a value 1 similar value for other cards, find the probability that the card has a value 
		\begin{enumerate}
			\item 7
			\item greater than 7
			\item less than 7
		\end{enumerate}
		%\input{exemplar/10/13/3/30/main.tex}
  \item A Lot consists of 48 mobile phones of which 42 are good, 3 have only minor defects and 3 have major defects.Varnika will buy a phone if it is good but the trader will only buy a mobile if it has no major defects. One phone is selected at random from the lot. What is the probability that it is
\begin{enumerate}
	\item acceptable to Varnika?
            \item acceptable to the trader?
\end{enumerate}
\solution
	%\input{exemplar/10/13/3/40/main.tex}
 \item A student says that if you throw a die, it will show up 1 or not 1. Therefore, the probability of getting 1 and the probability of getting 'not 1' each is equal to $\frac{1}{2}$. Is this correct? Give reasons.\\
 \solution
        %\input{exemplar/10/13/2/9/main.tex}
   \item Four candidates A, B, C, D have ap-
plied for the assignment to coach a school cricket
team. If A is twice as likely to be selected as B, and
B and C are given about the same chance of being
selected, while C is twice as likely to be selected
as D, what are the probabilities that
\begin{enumerate}
\item C will be selected?
\item A will not be selected?
\end{enumerate}
	%\input{exemplar/11/16/3/9/main.tex}
 \item A bag contain 24 balls of which $x$ balls are red, $2x$ are white and $3x$ are blue. A ball is selected at random, What is the probability that it is
\begin{enumerate}[label=\alph*)]
\item not red ?
\item white ?
\end{enumerate}
%\input{exemplar/10/13/3/41/main.tex}
If the letters of the word ASSASSINATION are arranged at random. Find the Probability that
\begin{enumerate}[label=(\alph*)]
\item Four $S's$ come consecutively in the word
\item Two  $I's$ and two $N's$ come together
\item All $A's$ are not coming together
\item No two $A's$ are coming together
\end{enumerate}
%\input{exemplar/11/16/3/14/main.tex}
	\item One urn contains two black balls (labelled B1 and B2) and one white ball. A
	second urn contains one black ball and two white balls (labelled W1 and W2).
	Suppose the following experiment is performed. One of the two urns is chosen
	at random. Next a ball is randomly chosen from the urn. Then a second ball is
	chosen at random from the same urn without replacing the first ball.
	
	\begin{enumerate}
	\item What is the probability that two black balls are chosen?
	
	\item What is the probability that two balls of opposite colour are chosen?
	\end{enumerate}
	\solution
	%\input{exemplar/11/16/3/12/main1.tex}
\end{enumerate}

		\item A box of oranges is inspected by examining three randomly selected oranges drawn without replacement. If all the three oranges are good, the box is approved for sale, otherwise, it is rejected. Find the probability that a box containing 15 oranges out of which 12 are good and 3 are bad ones will be approved for sale.
		\label{ncert/12/13/2/3/defs.tex}
		\item Two balls are drawn at random with replacement from a box containing 10 black and 8 red balls. Find the probability that
		\label{ncert/12/13/2/12}
\begin{enumerate}
\item both balls are red.
\item first ball is black and second is red.
\item one of them is black and other is red.
\end{enumerate}

\item In a hostel, 60\% of the students read Hindi newspaper, 40\% read English newspaper and 20\% read both Hindi and English newspapers. A student is selected at random.
		\label{ncert/12/13/2/15}
\begin{enumerate}
\item Find the probability that she reads neither Hindi nor English newspapers.
\item If she reads Hindi newspaper, find the probability that she reads English newspaper.
\item If she reads English newspaper, find the probability that she reads Hindi newspaper.\\
\end{enumerate}
\item The probability of obtaining an even prime number on each die, when a pair of dice is rolled is 
\begin{enumerate}
    \item $0$ 
    
    \item $\frac{1}{3}$ 
    
    \item $\frac{1}{12}$ 
    
    \item $\frac{1}{36}$ 
\end{enumerate}
\solution
		%\begin{enumerate}[label=\thesection.\arabic*,ref=\thesection.\theenumi]
	\item One card is drawn from a well-shuffled deck of 52 cards. Find the probability of getting
\begin{enumerate}
\item A king of red colour 
\item A face card 
\item A red face card
\item The jack of hearts
\item A spade
\item The queen of diamonds

\end{enumerate}
\solution
		%\input{ncert/10/15/1/14/main.tex}
	\item Five cards—the ten, jack, queen, king and ace of diamonds, are well-shuffled with their face downwards. One card is then picked up at random.
\begin{enumerate}
\item
What is the probability that the card is the queen? 
\item
If the queen is drawn and put aside, what is the probability that the second card picked up is (a) an ace? (b) a queen?\\
\end{enumerate}
\solution
		%\input{ncert/10/15/1/15/defs.tex}
	\item A bag contains $5$ red balls and some blue balls. If the probability of drawing a blue ball is double that if a red ball, determine the number of blue balls in the bag. 
		\\
\solution
		%\input{ncert/10/15/2/3/defs.tex}
	\item A card is selected from a pack of 52 cards.
 \begin{enumerate}[label=(\alph*)] 
                 \item How many points are there in the sample space?
                 \item Calculate the probability that the card is an ace of spades.
                 \item Calculate the probability that the card is (i) an ace and (ii) black card.
 \end{enumerate}
\solution
		%\input{ncert/11/16/3/4/main.tex}
\item Four cards are drawn from a well-shuffled deck of 52 cards. What is the probability of obtaining 3 diamonds and one spade.
\\
\solution
		%\input{ncert/11/16/4/2/defs.tex}
\item In a certain lottery 10,000 tickets are sold and ten equal prizes are awarded. What is the probability of not getting a prize if you buy (a) one ticket (b) two tickets (c) 10 tickets ?	
\\
\solution
		%\input{ncert/11/16/4/4/defs.tex}
		%
\item 
Out of 100 students, two sections of 40 and 60 are formed. If you and your friend are among the 100 students, what is the probability that
\begin{enumerate}
\item you both enter the same section?
\item you both enter the different sections?
\end{enumerate}
\solution
		%\input{ncert/11/16/4/5/defs.tex}
	\item 
The number lock of a suitcase has 4 wheels each labelled with ten digits i.e. from 0 to 9.The lock opens with a sequence of four digits with no repeats.What is the probability of a person getting the right sequence to open the suitcase.
\\
\solution
		%\input{ncert/11/16/4/10/defs.tex}
		%
\item 
Two cards are drawn at random and without replacement from a pack of 52 playing cards. Find the probability that both the cards are black.
\\
\solution
		%\input{ncert/12/13/2/2/defs.tex}
		\item A box of oranges is inspected by examining three randomly selected oranges drawn without replacement. If all the three oranges are good, the box is approved for sale, otherwise, it is rejected. Find the probability that a box containing 15 oranges out of which 12 are good and 3 are bad ones will be approved for sale.
		\label{ncert/12/13/2/3/defs.tex}
		\item Two balls are drawn at random with replacement from a box containing 10 black and 8 red balls. Find the probability that
		\label{ncert/12/13/2/12}
\begin{enumerate}
\item both balls are red.
\item first ball is black and second is red.
\item one of them is black and other is red.
\end{enumerate}

\item In a hostel, 60\% of the students read Hindi newspaper, 40\% read English newspaper and 20\% read both Hindi and English newspapers. A student is selected at random.
		\label{ncert/12/13/2/15}
\begin{enumerate}
\item Find the probability that she reads neither Hindi nor English newspapers.
\item If she reads Hindi newspaper, find the probability that she reads English newspaper.
\item If she reads English newspaper, find the probability that she reads Hindi newspaper.\\
\end{enumerate}
\item The probability of obtaining an even prime number on each die, when a pair of dice is rolled is 
\begin{enumerate}
    \item $0$ 
    
    \item $\frac{1}{3}$ 
    
    \item $\frac{1}{12}$ 
    
    \item $\frac{1}{36}$ 
\end{enumerate}
\solution
		%\input{ncert/12/13/2/17/defs.tex}
	\item A bag contains 4 red and 4 black balls, another bag contains 2 red and 6 black balls. One of the two bags is selected at random and a ball is drawn from the bag which is found to be red. Find the probability that the ball is drawn from the first bag.
\\
\solution
		%\input{ncert/12/13/3/2/main.tex}
  \item
  Cards with numbers 2 to 101 are placed in a box. A card is selected at random.Find the probability that the card has
\begin{enumerate}[label=(\roman*)]
	\item an even number 
	\item a square number
\end{enumerate}
\solution
%\input{exemplar/10/13/3/32/main.tex}
\item
The king, queen and jack of clubs are removed from a deck of 52 playing cards and then well shuffled. Now one card is drawn at random from the remaining cards.  Determine the probability that the card is
\begin{enumerate}[label=(\roman*)]
\item a club
\item 10 of hearts
\end{enumerate}
\solution
%\input{exemplar/10/13/3/29/main.tex}
\item A team of medical students doing their internship have to assist during surgeries
at a city hospital. The probabilities of surgeries rated as very complex, complex,
routine, simple or very simple are respectively, 0.15, 0.20, 0.31, 0.26, .08. Find
the probabilities that a particular surgery will be rated
\begin{enumerate}
	\item complex or very complex;
	\item neither very complex nor very simple;
	\item routine or complex
	\item routine or simple
\end{enumerate}
\solution
%\input{exemplar/11/16/3/8(1)/main.tex}
\item A card is selected from a pack of 52 cards.
\begin{enumerate}[label=(\alph*)]
    \item How many points are there in the sample space?
    \item Calculate the probability that the card is an ace of spades.
    \item Calculate the probability that the card is (i) an ace and (ii) black card.
\end{enumerate}
\solution
%\input{exemplar/11/16/3/4/main2.tex}
\item The probability that a non leap year selected at random will contain 53 sundays.
\\
\solution
%\input{exemplar/10/13/1/19/main.tex}
\item One of the four persons John, Rita, Aslam or Gurpreet will be promoted next
month. Consequently the sample space consists of four elementary outcomes
S = {John promoted, Rita promoted, Aslam promoted, Gurpreet promoted}
You are told that the chances of John’s promotion is same as that of Gurpreet,
Rita’s chances of promotion are twice as likely as Johns. Aslam’s chances are
four times that of John.
\begin{enumerate}
	\item Determine
	\begin{enumerate}
		\item P (John promoted)
		\item P (Rita promoted)
		\item P (Aslam promoted)
		\item P (Gurpreet promoted)
	\end{enumerate}
	\item If A = {John promoted or Gurpreet promoted}, find P (A).
\end{enumerate}
\solution
%\input{exemplar/11/16/3/10/main.tex}
\item A card is drawn from a deck of 52 cards. Find the probability of getting a king or a heart or a red card.\\
\solution
%\input{exemplar/11/16/3/15/main.tex}
\item The probability that a student will pass his examination is 0.73, the probability of
the student getting a compartment is 0.13, and the probability that the student will
either pass or get compartment is 0.96. State True or False.\\
\solution
%\input{exemplar/11/16/3/31/main.tex}
\item A card is selected from a pack of 52 cards\\
\begin{enumerate}[label=(\alph*)]
\item How many points are there in the sample space?
\item Calculate the probability that the cards is an ace of spades.
\item Calculate the probability that the card is (i) an ace (ii)black card.\\
\end{enumerate}
%\input{ncert/11/16/3/4_1/Prob_4.tex}
\item In a non-leap year, the probability of having 53 tuesdays or 53 wednesdays is\\
\solution
%\input{exemplar/11/16/3/18/main.tex}
\item There are 1000 sealed envelopes in a box, 10 of them contain a cash prize of
Rs 100 each, 100 of them contain a cash prize of Rs 50 each and 200 of them
contain a cash prize of Rs 10 each and rest do not contain any cash prize. If they
are well shuffled and an envelope is picked up out, what is the probability that it
contains no cash prize?\\
\solution
%\input{exemplar/10/13/3/34/main.tex}
\item 
A die is thrown and a card is selected at random from a deck of 52 playing cards. The probability of getting an even number on the die and a spade card.\\
\solution
%\input{exemplar/12/13/3/78/main.tex}
\item
If 4-digit numbers greater than 5,000 are randomly formed from the digits 0, 1, 3, 5, and 7, what is the probability of forming a number divisible by 5 when:
\begin{enumerate}
    \item The digits are repeated?
    \item The repetition of digits is not allowed?
\end{enumerate}
\solution
%\input{ncert/11/16/4/9/main.tex}
\item Consider the probability space $\brak{\Omega, \mathcal{G}, P}$ where $\Omega = [0,2]$ and $\mathcal{G} = \cbrak{\phi, \Omega, [0,1], (1,2]}$. Let $X$ and $Y$ be two functions on $\Omega$ defined as
\begin{align*}
    X(\omega) = 
    \begin{cases}
        1 & \text{if }\omega \in [0, 1]\\
        2 & \text{if }\omega \in (1, 2]
    \end{cases}
\end{align*}
and
\begin{align*}
    Y(\omega) = 
    \begin{cases}
        2 & \text{if }\omega \in [0, 1.5]\\
        3 & \text{if }\omega \in (1.5, 2].
    \end{cases}
\end{align*}
Then which one of the following statements is true?
\begin{enumerate}
    \item [(A)] $X$ is a random variable with respect to $\mathcal{G}$, but $Y$ is not a random variable with respect to $\mathcal{G}$.
    \item [(B)] $Y$ is a random variable with respect to $\mathcal{G}$, but $X$ is not a random variable with respect to $\mathcal{G}$.
    \item [(C)] Neither $X$ nor $Y$ is a random variable with respect to $\mathcal{G}$.
    \item [(D)] Both $X$ and $Y$ are random variables with respect to $\mathcal{G}$.
\end{enumerate} \hfill (GATE ST 2023)\\
\solution
%\input{gate/ST/2023/14/main.tex}
	\item  A die is loaded in such a way that each odd number is twice as likely to occur as
each even number. Find $P(G)$, where $G$ is the event that a number greater than
3 occurs on a single roll of the die.
\\
\solution
		%\input{exemplar/11/16/3/5/main.tex}
	\item All the jacks, queens and kings are removed from a deck of 52 playing cards. The remaining cards are well shuffled and then one card is drawn at random. Giving ace a value 1 similar value for other cards, find the probability that the card has a value 
		\begin{enumerate}
			\item 7
			\item greater than 7
			\item less than 7
		\end{enumerate}
		%\input{exemplar/10/13/3/30/main.tex}
  \item A Lot consists of 48 mobile phones of which 42 are good, 3 have only minor defects and 3 have major defects.Varnika will buy a phone if it is good but the trader will only buy a mobile if it has no major defects. One phone is selected at random from the lot. What is the probability that it is
\begin{enumerate}
	\item acceptable to Varnika?
            \item acceptable to the trader?
\end{enumerate}
\solution
	%\input{exemplar/10/13/3/40/main.tex}
 \item A student says that if you throw a die, it will show up 1 or not 1. Therefore, the probability of getting 1 and the probability of getting 'not 1' each is equal to $\frac{1}{2}$. Is this correct? Give reasons.\\
 \solution
        %\input{exemplar/10/13/2/9/main.tex}
   \item Four candidates A, B, C, D have ap-
plied for the assignment to coach a school cricket
team. If A is twice as likely to be selected as B, and
B and C are given about the same chance of being
selected, while C is twice as likely to be selected
as D, what are the probabilities that
\begin{enumerate}
\item C will be selected?
\item A will not be selected?
\end{enumerate}
	%\input{exemplar/11/16/3/9/main.tex}
 \item A bag contain 24 balls of which $x$ balls are red, $2x$ are white and $3x$ are blue. A ball is selected at random, What is the probability that it is
\begin{enumerate}[label=\alph*)]
\item not red ?
\item white ?
\end{enumerate}
%\input{exemplar/10/13/3/41/main.tex}
If the letters of the word ASSASSINATION are arranged at random. Find the Probability that
\begin{enumerate}[label=(\alph*)]
\item Four $S's$ come consecutively in the word
\item Two  $I's$ and two $N's$ come together
\item All $A's$ are not coming together
\item No two $A's$ are coming together
\end{enumerate}
%\input{exemplar/11/16/3/14/main.tex}
	\item One urn contains two black balls (labelled B1 and B2) and one white ball. A
	second urn contains one black ball and two white balls (labelled W1 and W2).
	Suppose the following experiment is performed. One of the two urns is chosen
	at random. Next a ball is randomly chosen from the urn. Then a second ball is
	chosen at random from the same urn without replacing the first ball.
	
	\begin{enumerate}
	\item What is the probability that two black balls are chosen?
	
	\item What is the probability that two balls of opposite colour are chosen?
	\end{enumerate}
	\solution
	%\input{exemplar/11/16/3/12/main1.tex}
\end{enumerate}

	\item A bag contains 4 red and 4 black balls, another bag contains 2 red and 6 black balls. One of the two bags is selected at random and a ball is drawn from the bag which is found to be red. Find the probability that the ball is drawn from the first bag.
\\
\solution
		%\begin{table}[H]
	\centering
\begin{tabular}{|c|c|c|}
\hline
Random variable &Value &Definition\\ \hline
\multirow{3}{*}{X} &0 &Slips of Rs 1\\
&1 &Slips of Rs 5\\
&2 &Slips of Rs 13\\ \hline
\multirow{2}{*}{Y} &0 &Box A\\
&1 &Box B\\\hline
\end{tabular}
\caption{}
\label{tab:Distribution}
\end{table}
See \tabref{tab:Distribution}.
\begin{align}
p_{Y}\brak{k}= \begin{cases} 
      \frac{1}{3} & {k=0} \\
      \frac{2}{3 }& {k=1} 
   \end{cases}
   \\
p_{Y|X}\brak{0|0} = \frac{19}{25}\, 
p_{Y|X}\brak{0|1} = \frac{6}{25}\,
p_{Y|X}\brak{1|0} = \frac{45}{50}\,
p_{Y|X}\brak{1|2} = \frac{5}{50}
\end{align}
The desired probability is the probability that a slip drawn at random is marked other than Rs 1,
\begin{align}
&=1-p_X\brak{0}\\
&= p_X(1) + p_X(2)
\end{align}
Using Bayes theorem,
\begin{align}
&= p_Y\brak{0} \times \pr{Y=0 | X=1} + p_Y\brak{1} \times \pr{Y=1|X=2}\\
&=\frac{1}{3} \times \frac{6}{25} + \frac{2}{3} \times \frac{5}{50}\\
&=\frac{11}{75}
\end{align}

\newpage

%\tableofcontents

\bigskip

\renewcommand{\thefigure}{\theenumi}
\renewcommand{\thetable}{\theenumi}
%\renewcommand{\theequation}{\theenumi}

%\begin{abstract}
%%\boldmath
%In this letter, an algorithm for evaluating the exact analytical bit error rate  (BER)  for the piecewise linear (PL) combiner for  multiple relays is presented. Previous results were available only for upto three relays. The algorithm is unique in the sense that  the actual mathematical expressions, that are prohibitively large, need not be explicitly obtained. The diversity gain due to multiple relays is shown through plots of the analytical BER, well supported by simulations. 
%
%\end{abstract}
% IEEEtran.cls defaults to using nonbold math in the Abstract.
% This preserves the distinction between vectors and scalars. However,
% if the journal you are submitting to favors bold math in the abstract,
% then you can use LaTeX's standard command \boldmath at the very start
% of the abstract to achieve this. Many IEEE journals frown on math
% in the abstract anyway.

% Note that keywords are not normally used for peerreview papers.
%\begin{IEEEkeywords}
%Cooperative diversity, decode and forward, piecewise linear
%\end{IEEEkeywords}



% For peer review papers, you can put extra information on the cover
% page as needed:
% \ifCLASSOPTIONpeerreview
% \begin{center} \bfseries EDICS Category: 3-BBND \end{center}
% \fi
%
% For peerreview papers, this IEEEtran command inserts a page break and
% creates the second title. It will be ignored for other modes.
%\IEEEpeerreviewmaketitle




  \item
  Cards with numbers 2 to 101 are placed in a box. A card is selected at random.Find the probability that the card has
\begin{enumerate}[label=(\roman*)]
	\item an even number 
	\item a square number
\end{enumerate}
\solution
%\begin{table}[H]
	\centering
\begin{tabular}{|c|c|c|}
\hline
Random variable &Value &Definition\\ \hline
\multirow{3}{*}{X} &0 &Slips of Rs 1\\
&1 &Slips of Rs 5\\
&2 &Slips of Rs 13\\ \hline
\multirow{2}{*}{Y} &0 &Box A\\
&1 &Box B\\\hline
\end{tabular}
\caption{}
\label{tab:Distribution}
\end{table}
See \tabref{tab:Distribution}.
\begin{align}
p_{Y}\brak{k}= \begin{cases} 
      \frac{1}{3} & {k=0} \\
      \frac{2}{3 }& {k=1} 
   \end{cases}
   \\
p_{Y|X}\brak{0|0} = \frac{19}{25}\, 
p_{Y|X}\brak{0|1} = \frac{6}{25}\,
p_{Y|X}\brak{1|0} = \frac{45}{50}\,
p_{Y|X}\brak{1|2} = \frac{5}{50}
\end{align}
The desired probability is the probability that a slip drawn at random is marked other than Rs 1,
\begin{align}
&=1-p_X\brak{0}\\
&= p_X(1) + p_X(2)
\end{align}
Using Bayes theorem,
\begin{align}
&= p_Y\brak{0} \times \pr{Y=0 | X=1} + p_Y\brak{1} \times \pr{Y=1|X=2}\\
&=\frac{1}{3} \times \frac{6}{25} + \frac{2}{3} \times \frac{5}{50}\\
&=\frac{11}{75}
\end{align}

\newpage

%\tableofcontents

\bigskip

\renewcommand{\thefigure}{\theenumi}
\renewcommand{\thetable}{\theenumi}
%\renewcommand{\theequation}{\theenumi}

%\begin{abstract}
%%\boldmath
%In this letter, an algorithm for evaluating the exact analytical bit error rate  (BER)  for the piecewise linear (PL) combiner for  multiple relays is presented. Previous results were available only for upto three relays. The algorithm is unique in the sense that  the actual mathematical expressions, that are prohibitively large, need not be explicitly obtained. The diversity gain due to multiple relays is shown through plots of the analytical BER, well supported by simulations. 
%
%\end{abstract}
% IEEEtran.cls defaults to using nonbold math in the Abstract.
% This preserves the distinction between vectors and scalars. However,
% if the journal you are submitting to favors bold math in the abstract,
% then you can use LaTeX's standard command \boldmath at the very start
% of the abstract to achieve this. Many IEEE journals frown on math
% in the abstract anyway.

% Note that keywords are not normally used for peerreview papers.
%\begin{IEEEkeywords}
%Cooperative diversity, decode and forward, piecewise linear
%\end{IEEEkeywords}



% For peer review papers, you can put extra information on the cover
% page as needed:
% \ifCLASSOPTIONpeerreview
% \begin{center} \bfseries EDICS Category: 3-BBND \end{center}
% \fi
%
% For peerreview papers, this IEEEtran command inserts a page break and
% creates the second title. It will be ignored for other modes.
%\IEEEpeerreviewmaketitle




\item
The king, queen and jack of clubs are removed from a deck of 52 playing cards and then well shuffled. Now one card is drawn at random from the remaining cards.  Determine the probability that the card is
\begin{enumerate}[label=(\roman*)]
\item a club
\item 10 of hearts
\end{enumerate}
\solution
%\begin{table}[H]
	\centering
\begin{tabular}{|c|c|c|}
\hline
Random variable &Value &Definition\\ \hline
\multirow{3}{*}{X} &0 &Slips of Rs 1\\
&1 &Slips of Rs 5\\
&2 &Slips of Rs 13\\ \hline
\multirow{2}{*}{Y} &0 &Box A\\
&1 &Box B\\\hline
\end{tabular}
\caption{}
\label{tab:Distribution}
\end{table}
See \tabref{tab:Distribution}.
\begin{align}
p_{Y}\brak{k}= \begin{cases} 
      \frac{1}{3} & {k=0} \\
      \frac{2}{3 }& {k=1} 
   \end{cases}
   \\
p_{Y|X}\brak{0|0} = \frac{19}{25}\, 
p_{Y|X}\brak{0|1} = \frac{6}{25}\,
p_{Y|X}\brak{1|0} = \frac{45}{50}\,
p_{Y|X}\brak{1|2} = \frac{5}{50}
\end{align}
The desired probability is the probability that a slip drawn at random is marked other than Rs 1,
\begin{align}
&=1-p_X\brak{0}\\
&= p_X(1) + p_X(2)
\end{align}
Using Bayes theorem,
\begin{align}
&= p_Y\brak{0} \times \pr{Y=0 | X=1} + p_Y\brak{1} \times \pr{Y=1|X=2}\\
&=\frac{1}{3} \times \frac{6}{25} + \frac{2}{3} \times \frac{5}{50}\\
&=\frac{11}{75}
\end{align}

\newpage

%\tableofcontents

\bigskip

\renewcommand{\thefigure}{\theenumi}
\renewcommand{\thetable}{\theenumi}
%\renewcommand{\theequation}{\theenumi}

%\begin{abstract}
%%\boldmath
%In this letter, an algorithm for evaluating the exact analytical bit error rate  (BER)  for the piecewise linear (PL) combiner for  multiple relays is presented. Previous results were available only for upto three relays. The algorithm is unique in the sense that  the actual mathematical expressions, that are prohibitively large, need not be explicitly obtained. The diversity gain due to multiple relays is shown through plots of the analytical BER, well supported by simulations. 
%
%\end{abstract}
% IEEEtran.cls defaults to using nonbold math in the Abstract.
% This preserves the distinction between vectors and scalars. However,
% if the journal you are submitting to favors bold math in the abstract,
% then you can use LaTeX's standard command \boldmath at the very start
% of the abstract to achieve this. Many IEEE journals frown on math
% in the abstract anyway.

% Note that keywords are not normally used for peerreview papers.
%\begin{IEEEkeywords}
%Cooperative diversity, decode and forward, piecewise linear
%\end{IEEEkeywords}



% For peer review papers, you can put extra information on the cover
% page as needed:
% \ifCLASSOPTIONpeerreview
% \begin{center} \bfseries EDICS Category: 3-BBND \end{center}
% \fi
%
% For peerreview papers, this IEEEtran command inserts a page break and
% creates the second title. It will be ignored for other modes.
%\IEEEpeerreviewmaketitle




\item A team of medical students doing their internship have to assist during surgeries
at a city hospital. The probabilities of surgeries rated as very complex, complex,
routine, simple or very simple are respectively, 0.15, 0.20, 0.31, 0.26, .08. Find
the probabilities that a particular surgery will be rated
\begin{enumerate}
	\item complex or very complex;
	\item neither very complex nor very simple;
	\item routine or complex
	\item routine or simple
\end{enumerate}
\solution
%\begin{table}[H]
	\centering
\begin{tabular}{|c|c|c|}
\hline
Random variable &Value &Definition\\ \hline
\multirow{3}{*}{X} &0 &Slips of Rs 1\\
&1 &Slips of Rs 5\\
&2 &Slips of Rs 13\\ \hline
\multirow{2}{*}{Y} &0 &Box A\\
&1 &Box B\\\hline
\end{tabular}
\caption{}
\label{tab:Distribution}
\end{table}
See \tabref{tab:Distribution}.
\begin{align}
p_{Y}\brak{k}= \begin{cases} 
      \frac{1}{3} & {k=0} \\
      \frac{2}{3 }& {k=1} 
   \end{cases}
   \\
p_{Y|X}\brak{0|0} = \frac{19}{25}\, 
p_{Y|X}\brak{0|1} = \frac{6}{25}\,
p_{Y|X}\brak{1|0} = \frac{45}{50}\,
p_{Y|X}\brak{1|2} = \frac{5}{50}
\end{align}
The desired probability is the probability that a slip drawn at random is marked other than Rs 1,
\begin{align}
&=1-p_X\brak{0}\\
&= p_X(1) + p_X(2)
\end{align}
Using Bayes theorem,
\begin{align}
&= p_Y\brak{0} \times \pr{Y=0 | X=1} + p_Y\brak{1} \times \pr{Y=1|X=2}\\
&=\frac{1}{3} \times \frac{6}{25} + \frac{2}{3} \times \frac{5}{50}\\
&=\frac{11}{75}
\end{align}

\newpage

%\tableofcontents

\bigskip

\renewcommand{\thefigure}{\theenumi}
\renewcommand{\thetable}{\theenumi}
%\renewcommand{\theequation}{\theenumi}

%\begin{abstract}
%%\boldmath
%In this letter, an algorithm for evaluating the exact analytical bit error rate  (BER)  for the piecewise linear (PL) combiner for  multiple relays is presented. Previous results were available only for upto three relays. The algorithm is unique in the sense that  the actual mathematical expressions, that are prohibitively large, need not be explicitly obtained. The diversity gain due to multiple relays is shown through plots of the analytical BER, well supported by simulations. 
%
%\end{abstract}
% IEEEtran.cls defaults to using nonbold math in the Abstract.
% This preserves the distinction between vectors and scalars. However,
% if the journal you are submitting to favors bold math in the abstract,
% then you can use LaTeX's standard command \boldmath at the very start
% of the abstract to achieve this. Many IEEE journals frown on math
% in the abstract anyway.

% Note that keywords are not normally used for peerreview papers.
%\begin{IEEEkeywords}
%Cooperative diversity, decode and forward, piecewise linear
%\end{IEEEkeywords}



% For peer review papers, you can put extra information on the cover
% page as needed:
% \ifCLASSOPTIONpeerreview
% \begin{center} \bfseries EDICS Category: 3-BBND \end{center}
% \fi
%
% For peerreview papers, this IEEEtran command inserts a page break and
% creates the second title. It will be ignored for other modes.
%\IEEEpeerreviewmaketitle




\item A card is selected from a pack of 52 cards.
\begin{enumerate}[label=(\alph*)]
    \item How many points are there in the sample space?
    \item Calculate the probability that the card is an ace of spades.
    \item Calculate the probability that the card is (i) an ace and (ii) black card.
\end{enumerate}
\solution
%Let $X$ be an bernoulli rv defined as in \tabref{tab:exemplar/11/16/3/26}.  Then, 
\begin{equation}
    p =
        \frac{4}{11} 
\end{equation}
\begin{table}[H]
	\centering
	\input{exemplar/11/16/3/26/tables/Table2.tex}
	\caption{}
        \label{tab:exemplar/11/16/3/26}
\end{table}

\item The probability that a non leap year selected at random will contain 53 sundays.
\\
\solution
%\begin{table}[H]
	\centering
\begin{tabular}{|c|c|c|}
\hline
Random variable &Value &Definition\\ \hline
\multirow{3}{*}{X} &0 &Slips of Rs 1\\
&1 &Slips of Rs 5\\
&2 &Slips of Rs 13\\ \hline
\multirow{2}{*}{Y} &0 &Box A\\
&1 &Box B\\\hline
\end{tabular}
\caption{}
\label{tab:Distribution}
\end{table}
See \tabref{tab:Distribution}.
\begin{align}
p_{Y}\brak{k}= \begin{cases} 
      \frac{1}{3} & {k=0} \\
      \frac{2}{3 }& {k=1} 
   \end{cases}
   \\
p_{Y|X}\brak{0|0} = \frac{19}{25}\, 
p_{Y|X}\brak{0|1} = \frac{6}{25}\,
p_{Y|X}\brak{1|0} = \frac{45}{50}\,
p_{Y|X}\brak{1|2} = \frac{5}{50}
\end{align}
The desired probability is the probability that a slip drawn at random is marked other than Rs 1,
\begin{align}
&=1-p_X\brak{0}\\
&= p_X(1) + p_X(2)
\end{align}
Using Bayes theorem,
\begin{align}
&= p_Y\brak{0} \times \pr{Y=0 | X=1} + p_Y\brak{1} \times \pr{Y=1|X=2}\\
&=\frac{1}{3} \times \frac{6}{25} + \frac{2}{3} \times \frac{5}{50}\\
&=\frac{11}{75}
\end{align}

\newpage

%\tableofcontents

\bigskip

\renewcommand{\thefigure}{\theenumi}
\renewcommand{\thetable}{\theenumi}
%\renewcommand{\theequation}{\theenumi}

%\begin{abstract}
%%\boldmath
%In this letter, an algorithm for evaluating the exact analytical bit error rate  (BER)  for the piecewise linear (PL) combiner for  multiple relays is presented. Previous results were available only for upto three relays. The algorithm is unique in the sense that  the actual mathematical expressions, that are prohibitively large, need not be explicitly obtained. The diversity gain due to multiple relays is shown through plots of the analytical BER, well supported by simulations. 
%
%\end{abstract}
% IEEEtran.cls defaults to using nonbold math in the Abstract.
% This preserves the distinction between vectors and scalars. However,
% if the journal you are submitting to favors bold math in the abstract,
% then you can use LaTeX's standard command \boldmath at the very start
% of the abstract to achieve this. Many IEEE journals frown on math
% in the abstract anyway.

% Note that keywords are not normally used for peerreview papers.
%\begin{IEEEkeywords}
%Cooperative diversity, decode and forward, piecewise linear
%\end{IEEEkeywords}



% For peer review papers, you can put extra information on the cover
% page as needed:
% \ifCLASSOPTIONpeerreview
% \begin{center} \bfseries EDICS Category: 3-BBND \end{center}
% \fi
%
% For peerreview papers, this IEEEtran command inserts a page break and
% creates the second title. It will be ignored for other modes.
%\IEEEpeerreviewmaketitle




\item One of the four persons John, Rita, Aslam or Gurpreet will be promoted next
month. Consequently the sample space consists of four elementary outcomes
S = {John promoted, Rita promoted, Aslam promoted, Gurpreet promoted}
You are told that the chances of John’s promotion is same as that of Gurpreet,
Rita’s chances of promotion are twice as likely as Johns. Aslam’s chances are
four times that of John.
\begin{enumerate}
	\item Determine
	\begin{enumerate}
		\item P (John promoted)
		\item P (Rita promoted)
		\item P (Aslam promoted)
		\item P (Gurpreet promoted)
	\end{enumerate}
	\item If A = {John promoted or Gurpreet promoted}, find P (A).
\end{enumerate}
\solution
%\begin{table}[H]
	\centering
\begin{tabular}{|c|c|c|}
\hline
Random variable &Value &Definition\\ \hline
\multirow{3}{*}{X} &0 &Slips of Rs 1\\
&1 &Slips of Rs 5\\
&2 &Slips of Rs 13\\ \hline
\multirow{2}{*}{Y} &0 &Box A\\
&1 &Box B\\\hline
\end{tabular}
\caption{}
\label{tab:Distribution}
\end{table}
See \tabref{tab:Distribution}.
\begin{align}
p_{Y}\brak{k}= \begin{cases} 
      \frac{1}{3} & {k=0} \\
      \frac{2}{3 }& {k=1} 
   \end{cases}
   \\
p_{Y|X}\brak{0|0} = \frac{19}{25}\, 
p_{Y|X}\brak{0|1} = \frac{6}{25}\,
p_{Y|X}\brak{1|0} = \frac{45}{50}\,
p_{Y|X}\brak{1|2} = \frac{5}{50}
\end{align}
The desired probability is the probability that a slip drawn at random is marked other than Rs 1,
\begin{align}
&=1-p_X\brak{0}\\
&= p_X(1) + p_X(2)
\end{align}
Using Bayes theorem,
\begin{align}
&= p_Y\brak{0} \times \pr{Y=0 | X=1} + p_Y\brak{1} \times \pr{Y=1|X=2}\\
&=\frac{1}{3} \times \frac{6}{25} + \frac{2}{3} \times \frac{5}{50}\\
&=\frac{11}{75}
\end{align}

\newpage

%\tableofcontents

\bigskip

\renewcommand{\thefigure}{\theenumi}
\renewcommand{\thetable}{\theenumi}
%\renewcommand{\theequation}{\theenumi}

%\begin{abstract}
%%\boldmath
%In this letter, an algorithm for evaluating the exact analytical bit error rate  (BER)  for the piecewise linear (PL) combiner for  multiple relays is presented. Previous results were available only for upto three relays. The algorithm is unique in the sense that  the actual mathematical expressions, that are prohibitively large, need not be explicitly obtained. The diversity gain due to multiple relays is shown through plots of the analytical BER, well supported by simulations. 
%
%\end{abstract}
% IEEEtran.cls defaults to using nonbold math in the Abstract.
% This preserves the distinction between vectors and scalars. However,
% if the journal you are submitting to favors bold math in the abstract,
% then you can use LaTeX's standard command \boldmath at the very start
% of the abstract to achieve this. Many IEEE journals frown on math
% in the abstract anyway.

% Note that keywords are not normally used for peerreview papers.
%\begin{IEEEkeywords}
%Cooperative diversity, decode and forward, piecewise linear
%\end{IEEEkeywords}



% For peer review papers, you can put extra information on the cover
% page as needed:
% \ifCLASSOPTIONpeerreview
% \begin{center} \bfseries EDICS Category: 3-BBND \end{center}
% \fi
%
% For peerreview papers, this IEEEtran command inserts a page break and
% creates the second title. It will be ignored for other modes.
%\IEEEpeerreviewmaketitle




\item A card is drawn from a deck of 52 cards. Find the probability of getting a king or a heart or a red card.\\
\solution
%\begin{table}[H]
	\centering
\begin{tabular}{|c|c|c|}
\hline
Random variable &Value &Definition\\ \hline
\multirow{3}{*}{X} &0 &Slips of Rs 1\\
&1 &Slips of Rs 5\\
&2 &Slips of Rs 13\\ \hline
\multirow{2}{*}{Y} &0 &Box A\\
&1 &Box B\\\hline
\end{tabular}
\caption{}
\label{tab:Distribution}
\end{table}
See \tabref{tab:Distribution}.
\begin{align}
p_{Y}\brak{k}= \begin{cases} 
      \frac{1}{3} & {k=0} \\
      \frac{2}{3 }& {k=1} 
   \end{cases}
   \\
p_{Y|X}\brak{0|0} = \frac{19}{25}\, 
p_{Y|X}\brak{0|1} = \frac{6}{25}\,
p_{Y|X}\brak{1|0} = \frac{45}{50}\,
p_{Y|X}\brak{1|2} = \frac{5}{50}
\end{align}
The desired probability is the probability that a slip drawn at random is marked other than Rs 1,
\begin{align}
&=1-p_X\brak{0}\\
&= p_X(1) + p_X(2)
\end{align}
Using Bayes theorem,
\begin{align}
&= p_Y\brak{0} \times \pr{Y=0 | X=1} + p_Y\brak{1} \times \pr{Y=1|X=2}\\
&=\frac{1}{3} \times \frac{6}{25} + \frac{2}{3} \times \frac{5}{50}\\
&=\frac{11}{75}
\end{align}

\newpage

%\tableofcontents

\bigskip

\renewcommand{\thefigure}{\theenumi}
\renewcommand{\thetable}{\theenumi}
%\renewcommand{\theequation}{\theenumi}

%\begin{abstract}
%%\boldmath
%In this letter, an algorithm for evaluating the exact analytical bit error rate  (BER)  for the piecewise linear (PL) combiner for  multiple relays is presented. Previous results were available only for upto three relays. The algorithm is unique in the sense that  the actual mathematical expressions, that are prohibitively large, need not be explicitly obtained. The diversity gain due to multiple relays is shown through plots of the analytical BER, well supported by simulations. 
%
%\end{abstract}
% IEEEtran.cls defaults to using nonbold math in the Abstract.
% This preserves the distinction between vectors and scalars. However,
% if the journal you are submitting to favors bold math in the abstract,
% then you can use LaTeX's standard command \boldmath at the very start
% of the abstract to achieve this. Many IEEE journals frown on math
% in the abstract anyway.

% Note that keywords are not normally used for peerreview papers.
%\begin{IEEEkeywords}
%Cooperative diversity, decode and forward, piecewise linear
%\end{IEEEkeywords}



% For peer review papers, you can put extra information on the cover
% page as needed:
% \ifCLASSOPTIONpeerreview
% \begin{center} \bfseries EDICS Category: 3-BBND \end{center}
% \fi
%
% For peerreview papers, this IEEEtran command inserts a page break and
% creates the second title. It will be ignored for other modes.
%\IEEEpeerreviewmaketitle




\item The probability that a student will pass his examination is 0.73, the probability of
the student getting a compartment is 0.13, and the probability that the student will
either pass or get compartment is 0.96. State True or False.\\
\solution
%\begin{table}[H]
	\centering
\begin{tabular}{|c|c|c|}
\hline
Random variable &Value &Definition\\ \hline
\multirow{3}{*}{X} &0 &Slips of Rs 1\\
&1 &Slips of Rs 5\\
&2 &Slips of Rs 13\\ \hline
\multirow{2}{*}{Y} &0 &Box A\\
&1 &Box B\\\hline
\end{tabular}
\caption{}
\label{tab:Distribution}
\end{table}
See \tabref{tab:Distribution}.
\begin{align}
p_{Y}\brak{k}= \begin{cases} 
      \frac{1}{3} & {k=0} \\
      \frac{2}{3 }& {k=1} 
   \end{cases}
   \\
p_{Y|X}\brak{0|0} = \frac{19}{25}\, 
p_{Y|X}\brak{0|1} = \frac{6}{25}\,
p_{Y|X}\brak{1|0} = \frac{45}{50}\,
p_{Y|X}\brak{1|2} = \frac{5}{50}
\end{align}
The desired probability is the probability that a slip drawn at random is marked other than Rs 1,
\begin{align}
&=1-p_X\brak{0}\\
&= p_X(1) + p_X(2)
\end{align}
Using Bayes theorem,
\begin{align}
&= p_Y\brak{0} \times \pr{Y=0 | X=1} + p_Y\brak{1} \times \pr{Y=1|X=2}\\
&=\frac{1}{3} \times \frac{6}{25} + \frac{2}{3} \times \frac{5}{50}\\
&=\frac{11}{75}
\end{align}

\newpage

%\tableofcontents

\bigskip

\renewcommand{\thefigure}{\theenumi}
\renewcommand{\thetable}{\theenumi}
%\renewcommand{\theequation}{\theenumi}

%\begin{abstract}
%%\boldmath
%In this letter, an algorithm for evaluating the exact analytical bit error rate  (BER)  for the piecewise linear (PL) combiner for  multiple relays is presented. Previous results were available only for upto three relays. The algorithm is unique in the sense that  the actual mathematical expressions, that are prohibitively large, need not be explicitly obtained. The diversity gain due to multiple relays is shown through plots of the analytical BER, well supported by simulations. 
%
%\end{abstract}
% IEEEtran.cls defaults to using nonbold math in the Abstract.
% This preserves the distinction between vectors and scalars. However,
% if the journal you are submitting to favors bold math in the abstract,
% then you can use LaTeX's standard command \boldmath at the very start
% of the abstract to achieve this. Many IEEE journals frown on math
% in the abstract anyway.

% Note that keywords are not normally used for peerreview papers.
%\begin{IEEEkeywords}
%Cooperative diversity, decode and forward, piecewise linear
%\end{IEEEkeywords}



% For peer review papers, you can put extra information on the cover
% page as needed:
% \ifCLASSOPTIONpeerreview
% \begin{center} \bfseries EDICS Category: 3-BBND \end{center}
% \fi
%
% For peerreview papers, this IEEEtran command inserts a page break and
% creates the second title. It will be ignored for other modes.
%\IEEEpeerreviewmaketitle




\item A card is selected from a pack of 52 cards\\
\begin{enumerate}[label=(\alph*)]
\item How many points are there in the sample space?
\item Calculate the probability that the cards is an ace of spades.
\item Calculate the probability that the card is (i) an ace (ii)black card.\\
\end{enumerate}
%\input{ncert/11/16/3/4_1/Prob_4.tex}
\item In a non-leap year, the probability of having 53 tuesdays or 53 wednesdays is\\
\solution
%A non-leap year has a total of 365 days, and a week has 7 days.\\
So it can be expressed as 
\begin{align}
365\text{days} &=52\times 7+1 \text{day}
\end{align}
$\implies$ 52 tuesdays or wednesdays\\
Random variable X denotes the days of a week
\begin{align}
p_X\brak{k}&=\frac{1}{7}; \quad \brak{1<k<7}
\end{align}
So the probability of extra day being tuesday or wednesday is
\begin{align}
p_X\brak{3}+p_X\brak{4}&=\frac{1}{7}+\frac{1}{7}=\frac{2}{7}
\end{align}



\item There are 1000 sealed envelopes in a box, 10 of them contain a cash prize of
Rs 100 each, 100 of them contain a cash prize of Rs 50 each and 200 of them
contain a cash prize of Rs 10 each and rest do not contain any cash prize. If they
are well shuffled and an envelope is picked up out, what is the probability that it
contains no cash prize?\\
\solution
%\begin{table}[H]
	\centering
\begin{tabular}{|c|c|c|}
\hline
Random variable &Value &Definition\\ \hline
\multirow{3}{*}{X} &0 &Slips of Rs 1\\
&1 &Slips of Rs 5\\
&2 &Slips of Rs 13\\ \hline
\multirow{2}{*}{Y} &0 &Box A\\
&1 &Box B\\\hline
\end{tabular}
\caption{}
\label{tab:Distribution}
\end{table}
See \tabref{tab:Distribution}.
\begin{align}
p_{Y}\brak{k}= \begin{cases} 
      \frac{1}{3} & {k=0} \\
      \frac{2}{3 }& {k=1} 
   \end{cases}
   \\
p_{Y|X}\brak{0|0} = \frac{19}{25}\, 
p_{Y|X}\brak{0|1} = \frac{6}{25}\,
p_{Y|X}\brak{1|0} = \frac{45}{50}\,
p_{Y|X}\brak{1|2} = \frac{5}{50}
\end{align}
The desired probability is the probability that a slip drawn at random is marked other than Rs 1,
\begin{align}
&=1-p_X\brak{0}\\
&= p_X(1) + p_X(2)
\end{align}
Using Bayes theorem,
\begin{align}
&= p_Y\brak{0} \times \pr{Y=0 | X=1} + p_Y\brak{1} \times \pr{Y=1|X=2}\\
&=\frac{1}{3} \times \frac{6}{25} + \frac{2}{3} \times \frac{5}{50}\\
&=\frac{11}{75}
\end{align}

\newpage

%\tableofcontents

\bigskip

\renewcommand{\thefigure}{\theenumi}
\renewcommand{\thetable}{\theenumi}
%\renewcommand{\theequation}{\theenumi}

%\begin{abstract}
%%\boldmath
%In this letter, an algorithm for evaluating the exact analytical bit error rate  (BER)  for the piecewise linear (PL) combiner for  multiple relays is presented. Previous results were available only for upto three relays. The algorithm is unique in the sense that  the actual mathematical expressions, that are prohibitively large, need not be explicitly obtained. The diversity gain due to multiple relays is shown through plots of the analytical BER, well supported by simulations. 
%
%\end{abstract}
% IEEEtran.cls defaults to using nonbold math in the Abstract.
% This preserves the distinction between vectors and scalars. However,
% if the journal you are submitting to favors bold math in the abstract,
% then you can use LaTeX's standard command \boldmath at the very start
% of the abstract to achieve this. Many IEEE journals frown on math
% in the abstract anyway.

% Note that keywords are not normally used for peerreview papers.
%\begin{IEEEkeywords}
%Cooperative diversity, decode and forward, piecewise linear
%\end{IEEEkeywords}



% For peer review papers, you can put extra information on the cover
% page as needed:
% \ifCLASSOPTIONpeerreview
% \begin{center} \bfseries EDICS Category: 3-BBND \end{center}
% \fi
%
% For peerreview papers, this IEEEtran command inserts a page break and
% creates the second title. It will be ignored for other modes.
%\IEEEpeerreviewmaketitle




\item 
A die is thrown and a card is selected at random from a deck of 52 playing cards. The probability of getting an even number on the die and a spade card.\\
\solution
%\begin{table}[H]
	\centering
\begin{tabular}{|c|c|c|}
\hline
Random variable &Value &Definition\\ \hline
\multirow{3}{*}{X} &0 &Slips of Rs 1\\
&1 &Slips of Rs 5\\
&2 &Slips of Rs 13\\ \hline
\multirow{2}{*}{Y} &0 &Box A\\
&1 &Box B\\\hline
\end{tabular}
\caption{}
\label{tab:Distribution}
\end{table}
See \tabref{tab:Distribution}.
\begin{align}
p_{Y}\brak{k}= \begin{cases} 
      \frac{1}{3} & {k=0} \\
      \frac{2}{3 }& {k=1} 
   \end{cases}
   \\
p_{Y|X}\brak{0|0} = \frac{19}{25}\, 
p_{Y|X}\brak{0|1} = \frac{6}{25}\,
p_{Y|X}\brak{1|0} = \frac{45}{50}\,
p_{Y|X}\brak{1|2} = \frac{5}{50}
\end{align}
The desired probability is the probability that a slip drawn at random is marked other than Rs 1,
\begin{align}
&=1-p_X\brak{0}\\
&= p_X(1) + p_X(2)
\end{align}
Using Bayes theorem,
\begin{align}
&= p_Y\brak{0} \times \pr{Y=0 | X=1} + p_Y\brak{1} \times \pr{Y=1|X=2}\\
&=\frac{1}{3} \times \frac{6}{25} + \frac{2}{3} \times \frac{5}{50}\\
&=\frac{11}{75}
\end{align}

\newpage

%\tableofcontents

\bigskip

\renewcommand{\thefigure}{\theenumi}
\renewcommand{\thetable}{\theenumi}
%\renewcommand{\theequation}{\theenumi}

%\begin{abstract}
%%\boldmath
%In this letter, an algorithm for evaluating the exact analytical bit error rate  (BER)  for the piecewise linear (PL) combiner for  multiple relays is presented. Previous results were available only for upto three relays. The algorithm is unique in the sense that  the actual mathematical expressions, that are prohibitively large, need not be explicitly obtained. The diversity gain due to multiple relays is shown through plots of the analytical BER, well supported by simulations. 
%
%\end{abstract}
% IEEEtran.cls defaults to using nonbold math in the Abstract.
% This preserves the distinction between vectors and scalars. However,
% if the journal you are submitting to favors bold math in the abstract,
% then you can use LaTeX's standard command \boldmath at the very start
% of the abstract to achieve this. Many IEEE journals frown on math
% in the abstract anyway.

% Note that keywords are not normally used for peerreview papers.
%\begin{IEEEkeywords}
%Cooperative diversity, decode and forward, piecewise linear
%\end{IEEEkeywords}



% For peer review papers, you can put extra information on the cover
% page as needed:
% \ifCLASSOPTIONpeerreview
% \begin{center} \bfseries EDICS Category: 3-BBND \end{center}
% \fi
%
% For peerreview papers, this IEEEtran command inserts a page break and
% creates the second title. It will be ignored for other modes.
%\IEEEpeerreviewmaketitle




\item
If 4-digit numbers greater than 5,000 are randomly formed from the digits 0, 1, 3, 5, and 7, what is the probability of forming a number divisible by 5 when:
\begin{enumerate}
    \item The digits are repeated?
    \item The repetition of digits is not allowed?
\end{enumerate}
\solution
%\begin{table}[H]
	\centering
\begin{tabular}{|c|c|c|}
\hline
Random variable &Value &Definition\\ \hline
\multirow{3}{*}{X} &0 &Slips of Rs 1\\
&1 &Slips of Rs 5\\
&2 &Slips of Rs 13\\ \hline
\multirow{2}{*}{Y} &0 &Box A\\
&1 &Box B\\\hline
\end{tabular}
\caption{}
\label{tab:Distribution}
\end{table}
See \tabref{tab:Distribution}.
\begin{align}
p_{Y}\brak{k}= \begin{cases} 
      \frac{1}{3} & {k=0} \\
      \frac{2}{3 }& {k=1} 
   \end{cases}
   \\
p_{Y|X}\brak{0|0} = \frac{19}{25}\, 
p_{Y|X}\brak{0|1} = \frac{6}{25}\,
p_{Y|X}\brak{1|0} = \frac{45}{50}\,
p_{Y|X}\brak{1|2} = \frac{5}{50}
\end{align}
The desired probability is the probability that a slip drawn at random is marked other than Rs 1,
\begin{align}
&=1-p_X\brak{0}\\
&= p_X(1) + p_X(2)
\end{align}
Using Bayes theorem,
\begin{align}
&= p_Y\brak{0} \times \pr{Y=0 | X=1} + p_Y\brak{1} \times \pr{Y=1|X=2}\\
&=\frac{1}{3} \times \frac{6}{25} + \frac{2}{3} \times \frac{5}{50}\\
&=\frac{11}{75}
\end{align}

\newpage

%\tableofcontents

\bigskip

\renewcommand{\thefigure}{\theenumi}
\renewcommand{\thetable}{\theenumi}
%\renewcommand{\theequation}{\theenumi}

%\begin{abstract}
%%\boldmath
%In this letter, an algorithm for evaluating the exact analytical bit error rate  (BER)  for the piecewise linear (PL) combiner for  multiple relays is presented. Previous results were available only for upto three relays. The algorithm is unique in the sense that  the actual mathematical expressions, that are prohibitively large, need not be explicitly obtained. The diversity gain due to multiple relays is shown through plots of the analytical BER, well supported by simulations. 
%
%\end{abstract}
% IEEEtran.cls defaults to using nonbold math in the Abstract.
% This preserves the distinction between vectors and scalars. However,
% if the journal you are submitting to favors bold math in the abstract,
% then you can use LaTeX's standard command \boldmath at the very start
% of the abstract to achieve this. Many IEEE journals frown on math
% in the abstract anyway.

% Note that keywords are not normally used for peerreview papers.
%\begin{IEEEkeywords}
%Cooperative diversity, decode and forward, piecewise linear
%\end{IEEEkeywords}



% For peer review papers, you can put extra information on the cover
% page as needed:
% \ifCLASSOPTIONpeerreview
% \begin{center} \bfseries EDICS Category: 3-BBND \end{center}
% \fi
%
% For peerreview papers, this IEEEtran command inserts a page break and
% creates the second title. It will be ignored for other modes.
%\IEEEpeerreviewmaketitle




\item Consider the probability space $\brak{\Omega, \mathcal{G}, P}$ where $\Omega = [0,2]$ and $\mathcal{G} = \cbrak{\phi, \Omega, [0,1], (1,2]}$. Let $X$ and $Y$ be two functions on $\Omega$ defined as
\begin{align*}
    X(\omega) = 
    \begin{cases}
        1 & \text{if }\omega \in [0, 1]\\
        2 & \text{if }\omega \in (1, 2]
    \end{cases}
\end{align*}
and
\begin{align*}
    Y(\omega) = 
    \begin{cases}
        2 & \text{if }\omega \in [0, 1.5]\\
        3 & \text{if }\omega \in (1.5, 2].
    \end{cases}
\end{align*}
Then which one of the following statements is true?
\begin{enumerate}
    \item [(A)] $X$ is a random variable with respect to $\mathcal{G}$, but $Y$ is not a random variable with respect to $\mathcal{G}$.
    \item [(B)] $Y$ is a random variable with respect to $\mathcal{G}$, but $X$ is not a random variable with respect to $\mathcal{G}$.
    \item [(C)] Neither $X$ nor $Y$ is a random variable with respect to $\mathcal{G}$.
    \item [(D)] Both $X$ and $Y$ are random variables with respect to $\mathcal{G}$.
\end{enumerate} \hfill (GATE ST 2023)\\
\solution
%\begin{table}[H]
	\centering
\begin{tabular}{|c|c|c|}
\hline
Random variable &Value &Definition\\ \hline
\multirow{3}{*}{X} &0 &Slips of Rs 1\\
&1 &Slips of Rs 5\\
&2 &Slips of Rs 13\\ \hline
\multirow{2}{*}{Y} &0 &Box A\\
&1 &Box B\\\hline
\end{tabular}
\caption{}
\label{tab:Distribution}
\end{table}
See \tabref{tab:Distribution}.
\begin{align}
p_{Y}\brak{k}= \begin{cases} 
      \frac{1}{3} & {k=0} \\
      \frac{2}{3 }& {k=1} 
   \end{cases}
   \\
p_{Y|X}\brak{0|0} = \frac{19}{25}\, 
p_{Y|X}\brak{0|1} = \frac{6}{25}\,
p_{Y|X}\brak{1|0} = \frac{45}{50}\,
p_{Y|X}\brak{1|2} = \frac{5}{50}
\end{align}
The desired probability is the probability that a slip drawn at random is marked other than Rs 1,
\begin{align}
&=1-p_X\brak{0}\\
&= p_X(1) + p_X(2)
\end{align}
Using Bayes theorem,
\begin{align}
&= p_Y\brak{0} \times \pr{Y=0 | X=1} + p_Y\brak{1} \times \pr{Y=1|X=2}\\
&=\frac{1}{3} \times \frac{6}{25} + \frac{2}{3} \times \frac{5}{50}\\
&=\frac{11}{75}
\end{align}

\newpage

%\tableofcontents

\bigskip

\renewcommand{\thefigure}{\theenumi}
\renewcommand{\thetable}{\theenumi}
%\renewcommand{\theequation}{\theenumi}

%\begin{abstract}
%%\boldmath
%In this letter, an algorithm for evaluating the exact analytical bit error rate  (BER)  for the piecewise linear (PL) combiner for  multiple relays is presented. Previous results were available only for upto three relays. The algorithm is unique in the sense that  the actual mathematical expressions, that are prohibitively large, need not be explicitly obtained. The diversity gain due to multiple relays is shown through plots of the analytical BER, well supported by simulations. 
%
%\end{abstract}
% IEEEtran.cls defaults to using nonbold math in the Abstract.
% This preserves the distinction between vectors and scalars. However,
% if the journal you are submitting to favors bold math in the abstract,
% then you can use LaTeX's standard command \boldmath at the very start
% of the abstract to achieve this. Many IEEE journals frown on math
% in the abstract anyway.

% Note that keywords are not normally used for peerreview papers.
%\begin{IEEEkeywords}
%Cooperative diversity, decode and forward, piecewise linear
%\end{IEEEkeywords}



% For peer review papers, you can put extra information on the cover
% page as needed:
% \ifCLASSOPTIONpeerreview
% \begin{center} \bfseries EDICS Category: 3-BBND \end{center}
% \fi
%
% For peerreview papers, this IEEEtran command inserts a page break and
% creates the second title. It will be ignored for other modes.
%\IEEEpeerreviewmaketitle




	\item  A die is loaded in such a way that each odd number is twice as likely to occur as
each even number. Find $P(G)$, where $G$ is the event that a number greater than
3 occurs on a single roll of the die.
\\
\solution
		%\begin{table}[H]
	\centering
\begin{tabular}{|c|c|c|}
\hline
Random variable &Value &Definition\\ \hline
\multirow{3}{*}{X} &0 &Slips of Rs 1\\
&1 &Slips of Rs 5\\
&2 &Slips of Rs 13\\ \hline
\multirow{2}{*}{Y} &0 &Box A\\
&1 &Box B\\\hline
\end{tabular}
\caption{}
\label{tab:Distribution}
\end{table}
See \tabref{tab:Distribution}.
\begin{align}
p_{Y}\brak{k}= \begin{cases} 
      \frac{1}{3} & {k=0} \\
      \frac{2}{3 }& {k=1} 
   \end{cases}
   \\
p_{Y|X}\brak{0|0} = \frac{19}{25}\, 
p_{Y|X}\brak{0|1} = \frac{6}{25}\,
p_{Y|X}\brak{1|0} = \frac{45}{50}\,
p_{Y|X}\brak{1|2} = \frac{5}{50}
\end{align}
The desired probability is the probability that a slip drawn at random is marked other than Rs 1,
\begin{align}
&=1-p_X\brak{0}\\
&= p_X(1) + p_X(2)
\end{align}
Using Bayes theorem,
\begin{align}
&= p_Y\brak{0} \times \pr{Y=0 | X=1} + p_Y\brak{1} \times \pr{Y=1|X=2}\\
&=\frac{1}{3} \times \frac{6}{25} + \frac{2}{3} \times \frac{5}{50}\\
&=\frac{11}{75}
\end{align}

\newpage

%\tableofcontents

\bigskip

\renewcommand{\thefigure}{\theenumi}
\renewcommand{\thetable}{\theenumi}
%\renewcommand{\theequation}{\theenumi}

%\begin{abstract}
%%\boldmath
%In this letter, an algorithm for evaluating the exact analytical bit error rate  (BER)  for the piecewise linear (PL) combiner for  multiple relays is presented. Previous results were available only for upto three relays. The algorithm is unique in the sense that  the actual mathematical expressions, that are prohibitively large, need not be explicitly obtained. The diversity gain due to multiple relays is shown through plots of the analytical BER, well supported by simulations. 
%
%\end{abstract}
% IEEEtran.cls defaults to using nonbold math in the Abstract.
% This preserves the distinction between vectors and scalars. However,
% if the journal you are submitting to favors bold math in the abstract,
% then you can use LaTeX's standard command \boldmath at the very start
% of the abstract to achieve this. Many IEEE journals frown on math
% in the abstract anyway.

% Note that keywords are not normally used for peerreview papers.
%\begin{IEEEkeywords}
%Cooperative diversity, decode and forward, piecewise linear
%\end{IEEEkeywords}



% For peer review papers, you can put extra information on the cover
% page as needed:
% \ifCLASSOPTIONpeerreview
% \begin{center} \bfseries EDICS Category: 3-BBND \end{center}
% \fi
%
% For peerreview papers, this IEEEtran command inserts a page break and
% creates the second title. It will be ignored for other modes.
%\IEEEpeerreviewmaketitle




	\item All the jacks, queens and kings are removed from a deck of 52 playing cards. The remaining cards are well shuffled and then one card is drawn at random. Giving ace a value 1 similar value for other cards, find the probability that the card has a value 
		\begin{enumerate}
			\item 7
			\item greater than 7
			\item less than 7
		\end{enumerate}
		%Number of cards left after removing all jacks, queens and kings 
\begin{align}
N	= 52 - 4\times 3
	= 40
\end{align}
%\begin{table}[H]
%\def\arraystretch{1.2}
%\begin{tabular}{|c|c|c|}
%\hline
%	\textbf{Parameter} &\textbf{Value} &\textbf{Description}\\ \hline
%	$X$ &1-10 &Represents the value of the card picked \\ \hline
%\end{tabular}
%\end{table}
Let $1 \le X \le 10$ be the value of the card picked.  Then,
\begin{align}
	p_X(k) &= \Pr(X=k)\ \forall\ 1 \leq k \leq 10\\
	&= \frac{4\times 1}{40}\\
	&= \frac{1}{10}\\
	\therefore p_X(k) &= 
	\begin{cases}
		\frac{1}{10} & 1 \leq k \leq 10\\
		0 & \text{otherwise}
	\end{cases}
\end{align}
and
\begin{align}
	F_{X}(k) &= \sum_{m=0}^{k}p_{X}(m) \quad 1 \leq k \leq 10\\
	&= \frac{k}{10}\\
	\therefore F_{X}(k) &= 
	\begin{cases}
		0 & k \leq 0\\
		\frac{k}{10} & 1\leq k \leq 10\\
		1 & k > 10 
	\end{cases}
\end{align}
\begin{enumerate}
	\item Probability that card has value equal to 7 is
		\begin{align}
			 p_{X}(7)
			= \frac{1}{10}
		\end{align}
	\item Probability that card has value greater than 7 is
		\begin{align}
			1 - F_X(7)
			&= 1 - \frac{7}{10}
			\\
			&= \frac{3}{10}
		\end{align}
	\item Probability that card has value less than 7 is
		\begin{align}
			 F_{X}(6)
			=\frac{6}{10}
		\end{align}
\end{enumerate}

  \item A Lot consists of 48 mobile phones of which 42 are good, 3 have only minor defects and 3 have major defects.Varnika will buy a phone if it is good but the trader will only buy a mobile if it has no major defects. One phone is selected at random from the lot. What is the probability that it is
\begin{enumerate}
	\item acceptable to Varnika?
            \item acceptable to the trader?
\end{enumerate}
\solution
	%\begin{table}[H]
	\centering
\begin{tabular}{|c|c|c|}
\hline
Random variable &Value &Definition\\ \hline
\multirow{3}{*}{X} &0 &Slips of Rs 1\\
&1 &Slips of Rs 5\\
&2 &Slips of Rs 13\\ \hline
\multirow{2}{*}{Y} &0 &Box A\\
&1 &Box B\\\hline
\end{tabular}
\caption{}
\label{tab:Distribution}
\end{table}
See \tabref{tab:Distribution}.
\begin{align}
p_{Y}\brak{k}= \begin{cases} 
      \frac{1}{3} & {k=0} \\
      \frac{2}{3 }& {k=1} 
   \end{cases}
   \\
p_{Y|X}\brak{0|0} = \frac{19}{25}\, 
p_{Y|X}\brak{0|1} = \frac{6}{25}\,
p_{Y|X}\brak{1|0} = \frac{45}{50}\,
p_{Y|X}\brak{1|2} = \frac{5}{50}
\end{align}
The desired probability is the probability that a slip drawn at random is marked other than Rs 1,
\begin{align}
&=1-p_X\brak{0}\\
&= p_X(1) + p_X(2)
\end{align}
Using Bayes theorem,
\begin{align}
&= p_Y\brak{0} \times \pr{Y=0 | X=1} + p_Y\brak{1} \times \pr{Y=1|X=2}\\
&=\frac{1}{3} \times \frac{6}{25} + \frac{2}{3} \times \frac{5}{50}\\
&=\frac{11}{75}
\end{align}

\newpage

%\tableofcontents

\bigskip

\renewcommand{\thefigure}{\theenumi}
\renewcommand{\thetable}{\theenumi}
%\renewcommand{\theequation}{\theenumi}

%\begin{abstract}
%%\boldmath
%In this letter, an algorithm for evaluating the exact analytical bit error rate  (BER)  for the piecewise linear (PL) combiner for  multiple relays is presented. Previous results were available only for upto three relays. The algorithm is unique in the sense that  the actual mathematical expressions, that are prohibitively large, need not be explicitly obtained. The diversity gain due to multiple relays is shown through plots of the analytical BER, well supported by simulations. 
%
%\end{abstract}
% IEEEtran.cls defaults to using nonbold math in the Abstract.
% This preserves the distinction between vectors and scalars. However,
% if the journal you are submitting to favors bold math in the abstract,
% then you can use LaTeX's standard command \boldmath at the very start
% of the abstract to achieve this. Many IEEE journals frown on math
% in the abstract anyway.

% Note that keywords are not normally used for peerreview papers.
%\begin{IEEEkeywords}
%Cooperative diversity, decode and forward, piecewise linear
%\end{IEEEkeywords}



% For peer review papers, you can put extra information on the cover
% page as needed:
% \ifCLASSOPTIONpeerreview
% \begin{center} \bfseries EDICS Category: 3-BBND \end{center}
% \fi
%
% For peerreview papers, this IEEEtran command inserts a page break and
% creates the second title. It will be ignored for other modes.
%\IEEEpeerreviewmaketitle




 \item A student says that if you throw a die, it will show up 1 or not 1. Therefore, the probability of getting 1 and the probability of getting 'not 1' each is equal to $\frac{1}{2}$. Is this correct? Give reasons.\\
 \solution
        %\begin{table}[H]
	\centering
\begin{tabular}{|c|c|c|}
\hline
Random variable &Value &Definition\\ \hline
\multirow{3}{*}{X} &0 &Slips of Rs 1\\
&1 &Slips of Rs 5\\
&2 &Slips of Rs 13\\ \hline
\multirow{2}{*}{Y} &0 &Box A\\
&1 &Box B\\\hline
\end{tabular}
\caption{}
\label{tab:Distribution}
\end{table}
See \tabref{tab:Distribution}.
\begin{align}
p_{Y}\brak{k}= \begin{cases} 
      \frac{1}{3} & {k=0} \\
      \frac{2}{3 }& {k=1} 
   \end{cases}
   \\
p_{Y|X}\brak{0|0} = \frac{19}{25}\, 
p_{Y|X}\brak{0|1} = \frac{6}{25}\,
p_{Y|X}\brak{1|0} = \frac{45}{50}\,
p_{Y|X}\brak{1|2} = \frac{5}{50}
\end{align}
The desired probability is the probability that a slip drawn at random is marked other than Rs 1,
\begin{align}
&=1-p_X\brak{0}\\
&= p_X(1) + p_X(2)
\end{align}
Using Bayes theorem,
\begin{align}
&= p_Y\brak{0} \times \pr{Y=0 | X=1} + p_Y\brak{1} \times \pr{Y=1|X=2}\\
&=\frac{1}{3} \times \frac{6}{25} + \frac{2}{3} \times \frac{5}{50}\\
&=\frac{11}{75}
\end{align}

\newpage

%\tableofcontents

\bigskip

\renewcommand{\thefigure}{\theenumi}
\renewcommand{\thetable}{\theenumi}
%\renewcommand{\theequation}{\theenumi}

%\begin{abstract}
%%\boldmath
%In this letter, an algorithm for evaluating the exact analytical bit error rate  (BER)  for the piecewise linear (PL) combiner for  multiple relays is presented. Previous results were available only for upto three relays. The algorithm is unique in the sense that  the actual mathematical expressions, that are prohibitively large, need not be explicitly obtained. The diversity gain due to multiple relays is shown through plots of the analytical BER, well supported by simulations. 
%
%\end{abstract}
% IEEEtran.cls defaults to using nonbold math in the Abstract.
% This preserves the distinction between vectors and scalars. However,
% if the journal you are submitting to favors bold math in the abstract,
% then you can use LaTeX's standard command \boldmath at the very start
% of the abstract to achieve this. Many IEEE journals frown on math
% in the abstract anyway.

% Note that keywords are not normally used for peerreview papers.
%\begin{IEEEkeywords}
%Cooperative diversity, decode and forward, piecewise linear
%\end{IEEEkeywords}



% For peer review papers, you can put extra information on the cover
% page as needed:
% \ifCLASSOPTIONpeerreview
% \begin{center} \bfseries EDICS Category: 3-BBND \end{center}
% \fi
%
% For peerreview papers, this IEEEtran command inserts a page break and
% creates the second title. It will be ignored for other modes.
%\IEEEpeerreviewmaketitle




   \item Four candidates A, B, C, D have ap-
plied for the assignment to coach a school cricket
team. If A is twice as likely to be selected as B, and
B and C are given about the same chance of being
selected, while C is twice as likely to be selected
as D, what are the probabilities that
\begin{enumerate}
\item C will be selected?
\item A will not be selected?
\end{enumerate}
	%\begin{table}[H]
	\centering
\begin{tabular}{|c|c|c|}
\hline
Random variable &Value &Definition\\ \hline
\multirow{3}{*}{X} &0 &Slips of Rs 1\\
&1 &Slips of Rs 5\\
&2 &Slips of Rs 13\\ \hline
\multirow{2}{*}{Y} &0 &Box A\\
&1 &Box B\\\hline
\end{tabular}
\caption{}
\label{tab:Distribution}
\end{table}
See \tabref{tab:Distribution}.
\begin{align}
p_{Y}\brak{k}= \begin{cases} 
      \frac{1}{3} & {k=0} \\
      \frac{2}{3 }& {k=1} 
   \end{cases}
   \\
p_{Y|X}\brak{0|0} = \frac{19}{25}\, 
p_{Y|X}\brak{0|1} = \frac{6}{25}\,
p_{Y|X}\brak{1|0} = \frac{45}{50}\,
p_{Y|X}\brak{1|2} = \frac{5}{50}
\end{align}
The desired probability is the probability that a slip drawn at random is marked other than Rs 1,
\begin{align}
&=1-p_X\brak{0}\\
&= p_X(1) + p_X(2)
\end{align}
Using Bayes theorem,
\begin{align}
&= p_Y\brak{0} \times \pr{Y=0 | X=1} + p_Y\brak{1} \times \pr{Y=1|X=2}\\
&=\frac{1}{3} \times \frac{6}{25} + \frac{2}{3} \times \frac{5}{50}\\
&=\frac{11}{75}
\end{align}

\newpage

%\tableofcontents

\bigskip

\renewcommand{\thefigure}{\theenumi}
\renewcommand{\thetable}{\theenumi}
%\renewcommand{\theequation}{\theenumi}

%\begin{abstract}
%%\boldmath
%In this letter, an algorithm for evaluating the exact analytical bit error rate  (BER)  for the piecewise linear (PL) combiner for  multiple relays is presented. Previous results were available only for upto three relays. The algorithm is unique in the sense that  the actual mathematical expressions, that are prohibitively large, need not be explicitly obtained. The diversity gain due to multiple relays is shown through plots of the analytical BER, well supported by simulations. 
%
%\end{abstract}
% IEEEtran.cls defaults to using nonbold math in the Abstract.
% This preserves the distinction between vectors and scalars. However,
% if the journal you are submitting to favors bold math in the abstract,
% then you can use LaTeX's standard command \boldmath at the very start
% of the abstract to achieve this. Many IEEE journals frown on math
% in the abstract anyway.

% Note that keywords are not normally used for peerreview papers.
%\begin{IEEEkeywords}
%Cooperative diversity, decode and forward, piecewise linear
%\end{IEEEkeywords}



% For peer review papers, you can put extra information on the cover
% page as needed:
% \ifCLASSOPTIONpeerreview
% \begin{center} \bfseries EDICS Category: 3-BBND \end{center}
% \fi
%
% For peerreview papers, this IEEEtran command inserts a page break and
% creates the second title. It will be ignored for other modes.
%\IEEEpeerreviewmaketitle




 \item A bag contain 24 balls of which $x$ balls are red, $2x$ are white and $3x$ are blue. A ball is selected at random, What is the probability that it is
\begin{enumerate}[label=\alph*)]
\item not red ?
\item white ?
\end{enumerate}
%\begin{table}[H]
	\centering
\begin{tabular}{|c|c|c|}
\hline
Random variable &Value &Definition\\ \hline
\multirow{3}{*}{X} &0 &Slips of Rs 1\\
&1 &Slips of Rs 5\\
&2 &Slips of Rs 13\\ \hline
\multirow{2}{*}{Y} &0 &Box A\\
&1 &Box B\\\hline
\end{tabular}
\caption{}
\label{tab:Distribution}
\end{table}
See \tabref{tab:Distribution}.
\begin{align}
p_{Y}\brak{k}= \begin{cases} 
      \frac{1}{3} & {k=0} \\
      \frac{2}{3 }& {k=1} 
   \end{cases}
   \\
p_{Y|X}\brak{0|0} = \frac{19}{25}\, 
p_{Y|X}\brak{0|1} = \frac{6}{25}\,
p_{Y|X}\brak{1|0} = \frac{45}{50}\,
p_{Y|X}\brak{1|2} = \frac{5}{50}
\end{align}
The desired probability is the probability that a slip drawn at random is marked other than Rs 1,
\begin{align}
&=1-p_X\brak{0}\\
&= p_X(1) + p_X(2)
\end{align}
Using Bayes theorem,
\begin{align}
&= p_Y\brak{0} \times \pr{Y=0 | X=1} + p_Y\brak{1} \times \pr{Y=1|X=2}\\
&=\frac{1}{3} \times \frac{6}{25} + \frac{2}{3} \times \frac{5}{50}\\
&=\frac{11}{75}
\end{align}

\newpage

%\tableofcontents

\bigskip

\renewcommand{\thefigure}{\theenumi}
\renewcommand{\thetable}{\theenumi}
%\renewcommand{\theequation}{\theenumi}

%\begin{abstract}
%%\boldmath
%In this letter, an algorithm for evaluating the exact analytical bit error rate  (BER)  for the piecewise linear (PL) combiner for  multiple relays is presented. Previous results were available only for upto three relays. The algorithm is unique in the sense that  the actual mathematical expressions, that are prohibitively large, need not be explicitly obtained. The diversity gain due to multiple relays is shown through plots of the analytical BER, well supported by simulations. 
%
%\end{abstract}
% IEEEtran.cls defaults to using nonbold math in the Abstract.
% This preserves the distinction between vectors and scalars. However,
% if the journal you are submitting to favors bold math in the abstract,
% then you can use LaTeX's standard command \boldmath at the very start
% of the abstract to achieve this. Many IEEE journals frown on math
% in the abstract anyway.

% Note that keywords are not normally used for peerreview papers.
%\begin{IEEEkeywords}
%Cooperative diversity, decode and forward, piecewise linear
%\end{IEEEkeywords}



% For peer review papers, you can put extra information on the cover
% page as needed:
% \ifCLASSOPTIONpeerreview
% \begin{center} \bfseries EDICS Category: 3-BBND \end{center}
% \fi
%
% For peerreview papers, this IEEEtran command inserts a page break and
% creates the second title. It will be ignored for other modes.
%\IEEEpeerreviewmaketitle




If the letters of the word ASSASSINATION are arranged at random. Find the Probability that
\begin{enumerate}[label=(\alph*)]
\item Four $S's$ come consecutively in the word
\item Two  $I's$ and two $N's$ come together
\item All $A's$ are not coming together
\item No two $A's$ are coming together
\end{enumerate}
%\begin{table}[H]
	\centering
\begin{tabular}{|c|c|c|}
\hline
Random variable &Value &Definition\\ \hline
\multirow{3}{*}{X} &0 &Slips of Rs 1\\
&1 &Slips of Rs 5\\
&2 &Slips of Rs 13\\ \hline
\multirow{2}{*}{Y} &0 &Box A\\
&1 &Box B\\\hline
\end{tabular}
\caption{}
\label{tab:Distribution}
\end{table}
See \tabref{tab:Distribution}.
\begin{align}
p_{Y}\brak{k}= \begin{cases} 
      \frac{1}{3} & {k=0} \\
      \frac{2}{3 }& {k=1} 
   \end{cases}
   \\
p_{Y|X}\brak{0|0} = \frac{19}{25}\, 
p_{Y|X}\brak{0|1} = \frac{6}{25}\,
p_{Y|X}\brak{1|0} = \frac{45}{50}\,
p_{Y|X}\brak{1|2} = \frac{5}{50}
\end{align}
The desired probability is the probability that a slip drawn at random is marked other than Rs 1,
\begin{align}
&=1-p_X\brak{0}\\
&= p_X(1) + p_X(2)
\end{align}
Using Bayes theorem,
\begin{align}
&= p_Y\brak{0} \times \pr{Y=0 | X=1} + p_Y\brak{1} \times \pr{Y=1|X=2}\\
&=\frac{1}{3} \times \frac{6}{25} + \frac{2}{3} \times \frac{5}{50}\\
&=\frac{11}{75}
\end{align}

\newpage

%\tableofcontents

\bigskip

\renewcommand{\thefigure}{\theenumi}
\renewcommand{\thetable}{\theenumi}
%\renewcommand{\theequation}{\theenumi}

%\begin{abstract}
%%\boldmath
%In this letter, an algorithm for evaluating the exact analytical bit error rate  (BER)  for the piecewise linear (PL) combiner for  multiple relays is presented. Previous results were available only for upto three relays. The algorithm is unique in the sense that  the actual mathematical expressions, that are prohibitively large, need not be explicitly obtained. The diversity gain due to multiple relays is shown through plots of the analytical BER, well supported by simulations. 
%
%\end{abstract}
% IEEEtran.cls defaults to using nonbold math in the Abstract.
% This preserves the distinction between vectors and scalars. However,
% if the journal you are submitting to favors bold math in the abstract,
% then you can use LaTeX's standard command \boldmath at the very start
% of the abstract to achieve this. Many IEEE journals frown on math
% in the abstract anyway.

% Note that keywords are not normally used for peerreview papers.
%\begin{IEEEkeywords}
%Cooperative diversity, decode and forward, piecewise linear
%\end{IEEEkeywords}



% For peer review papers, you can put extra information on the cover
% page as needed:
% \ifCLASSOPTIONpeerreview
% \begin{center} \bfseries EDICS Category: 3-BBND \end{center}
% \fi
%
% For peerreview papers, this IEEEtran command inserts a page break and
% creates the second title. It will be ignored for other modes.
%\IEEEpeerreviewmaketitle




	\item One urn contains two black balls (labelled B1 and B2) and one white ball. A
	second urn contains one black ball and two white balls (labelled W1 and W2).
	Suppose the following experiment is performed. One of the two urns is chosen
	at random. Next a ball is randomly chosen from the urn. Then a second ball is
	chosen at random from the same urn without replacing the first ball.
	
	\begin{enumerate}
	\item What is the probability that two black balls are chosen?
	
	\item What is the probability that two balls of opposite colour are chosen?
	\end{enumerate}
	\solution
	%\begin{align}
    \label{eq:12.13.6.18.1}
	\because	\pr{A|B} &> \pr{A},\
\frac{\pr{AB}}{\pr{B}} > \pr{A}
\\
    \label{eq:12.13.6.18.2}
	\implies \pr{AB} &> \pr{A}\pr{B}
	\\
	\text{or, } \frac{\pr{AB}}{\pr{A}} &=\pr{B|A} > \pr{A}
\end{align}

\end{enumerate}

\item In a certain lottery 10,000 tickets are sold and ten equal prizes are awarded. What is the probability of not getting a prize if you buy (a) one ticket (b) two tickets (c) 10 tickets ?	
\\
\solution
		%\begin{enumerate}[label=\thesection.\arabic*,ref=\thesection.\theenumi]
	\item One card is drawn from a well-shuffled deck of 52 cards. Find the probability of getting
\begin{enumerate}
\item A king of red colour 
\item A face card 
\item A red face card
\item The jack of hearts
\item A spade
\item The queen of diamonds

\end{enumerate}
\solution
		%\begin{table}[H]
	\centering
\begin{tabular}{|c|c|c|}
\hline
Random variable &Value &Definition\\ \hline
\multirow{3}{*}{X} &0 &Slips of Rs 1\\
&1 &Slips of Rs 5\\
&2 &Slips of Rs 13\\ \hline
\multirow{2}{*}{Y} &0 &Box A\\
&1 &Box B\\\hline
\end{tabular}
\caption{}
\label{tab:Distribution}
\end{table}
See \tabref{tab:Distribution}.
\begin{align}
p_{Y}\brak{k}= \begin{cases} 
      \frac{1}{3} & {k=0} \\
      \frac{2}{3 }& {k=1} 
   \end{cases}
   \\
p_{Y|X}\brak{0|0} = \frac{19}{25}\, 
p_{Y|X}\brak{0|1} = \frac{6}{25}\,
p_{Y|X}\brak{1|0} = \frac{45}{50}\,
p_{Y|X}\brak{1|2} = \frac{5}{50}
\end{align}
The desired probability is the probability that a slip drawn at random is marked other than Rs 1,
\begin{align}
&=1-p_X\brak{0}\\
&= p_X(1) + p_X(2)
\end{align}
Using Bayes theorem,
\begin{align}
&= p_Y\brak{0} \times \pr{Y=0 | X=1} + p_Y\brak{1} \times \pr{Y=1|X=2}\\
&=\frac{1}{3} \times \frac{6}{25} + \frac{2}{3} \times \frac{5}{50}\\
&=\frac{11}{75}
\end{align}

\newpage

%\tableofcontents

\bigskip

\renewcommand{\thefigure}{\theenumi}
\renewcommand{\thetable}{\theenumi}
%\renewcommand{\theequation}{\theenumi}

%\begin{abstract}
%%\boldmath
%In this letter, an algorithm for evaluating the exact analytical bit error rate  (BER)  for the piecewise linear (PL) combiner for  multiple relays is presented. Previous results were available only for upto three relays. The algorithm is unique in the sense that  the actual mathematical expressions, that are prohibitively large, need not be explicitly obtained. The diversity gain due to multiple relays is shown through plots of the analytical BER, well supported by simulations. 
%
%\end{abstract}
% IEEEtran.cls defaults to using nonbold math in the Abstract.
% This preserves the distinction between vectors and scalars. However,
% if the journal you are submitting to favors bold math in the abstract,
% then you can use LaTeX's standard command \boldmath at the very start
% of the abstract to achieve this. Many IEEE journals frown on math
% in the abstract anyway.

% Note that keywords are not normally used for peerreview papers.
%\begin{IEEEkeywords}
%Cooperative diversity, decode and forward, piecewise linear
%\end{IEEEkeywords}



% For peer review papers, you can put extra information on the cover
% page as needed:
% \ifCLASSOPTIONpeerreview
% \begin{center} \bfseries EDICS Category: 3-BBND \end{center}
% \fi
%
% For peerreview papers, this IEEEtran command inserts a page break and
% creates the second title. It will be ignored for other modes.
%\IEEEpeerreviewmaketitle




	\item Five cards—the ten, jack, queen, king and ace of diamonds, are well-shuffled with their face downwards. One card is then picked up at random.
\begin{enumerate}
\item
What is the probability that the card is the queen? 
\item
If the queen is drawn and put aside, what is the probability that the second card picked up is (a) an ace? (b) a queen?\\
\end{enumerate}
\solution
		%\begin{enumerate}[label=\thesection.\arabic*,ref=\thesection.\theenumi]
	\item One card is drawn from a well-shuffled deck of 52 cards. Find the probability of getting
\begin{enumerate}
\item A king of red colour 
\item A face card 
\item A red face card
\item The jack of hearts
\item A spade
\item The queen of diamonds

\end{enumerate}
\solution
		%\input{ncert/10/15/1/14/main.tex}
	\item Five cards—the ten, jack, queen, king and ace of diamonds, are well-shuffled with their face downwards. One card is then picked up at random.
\begin{enumerate}
\item
What is the probability that the card is the queen? 
\item
If the queen is drawn and put aside, what is the probability that the second card picked up is (a) an ace? (b) a queen?\\
\end{enumerate}
\solution
		%\input{ncert/10/15/1/15/defs.tex}
	\item A bag contains $5$ red balls and some blue balls. If the probability of drawing a blue ball is double that if a red ball, determine the number of blue balls in the bag. 
		\\
\solution
		%\input{ncert/10/15/2/3/defs.tex}
	\item A card is selected from a pack of 52 cards.
 \begin{enumerate}[label=(\alph*)] 
                 \item How many points are there in the sample space?
                 \item Calculate the probability that the card is an ace of spades.
                 \item Calculate the probability that the card is (i) an ace and (ii) black card.
 \end{enumerate}
\solution
		%\input{ncert/11/16/3/4/main.tex}
\item Four cards are drawn from a well-shuffled deck of 52 cards. What is the probability of obtaining 3 diamonds and one spade.
\\
\solution
		%\input{ncert/11/16/4/2/defs.tex}
\item In a certain lottery 10,000 tickets are sold and ten equal prizes are awarded. What is the probability of not getting a prize if you buy (a) one ticket (b) two tickets (c) 10 tickets ?	
\\
\solution
		%\input{ncert/11/16/4/4/defs.tex}
		%
\item 
Out of 100 students, two sections of 40 and 60 are formed. If you and your friend are among the 100 students, what is the probability that
\begin{enumerate}
\item you both enter the same section?
\item you both enter the different sections?
\end{enumerate}
\solution
		%\input{ncert/11/16/4/5/defs.tex}
	\item 
The number lock of a suitcase has 4 wheels each labelled with ten digits i.e. from 0 to 9.The lock opens with a sequence of four digits with no repeats.What is the probability of a person getting the right sequence to open the suitcase.
\\
\solution
		%\input{ncert/11/16/4/10/defs.tex}
		%
\item 
Two cards are drawn at random and without replacement from a pack of 52 playing cards. Find the probability that both the cards are black.
\\
\solution
		%\input{ncert/12/13/2/2/defs.tex}
		\item A box of oranges is inspected by examining three randomly selected oranges drawn without replacement. If all the three oranges are good, the box is approved for sale, otherwise, it is rejected. Find the probability that a box containing 15 oranges out of which 12 are good and 3 are bad ones will be approved for sale.
		\label{ncert/12/13/2/3/defs.tex}
		\item Two balls are drawn at random with replacement from a box containing 10 black and 8 red balls. Find the probability that
		\label{ncert/12/13/2/12}
\begin{enumerate}
\item both balls are red.
\item first ball is black and second is red.
\item one of them is black and other is red.
\end{enumerate}

\item In a hostel, 60\% of the students read Hindi newspaper, 40\% read English newspaper and 20\% read both Hindi and English newspapers. A student is selected at random.
		\label{ncert/12/13/2/15}
\begin{enumerate}
\item Find the probability that she reads neither Hindi nor English newspapers.
\item If she reads Hindi newspaper, find the probability that she reads English newspaper.
\item If she reads English newspaper, find the probability that she reads Hindi newspaper.\\
\end{enumerate}
\item The probability of obtaining an even prime number on each die, when a pair of dice is rolled is 
\begin{enumerate}
    \item $0$ 
    
    \item $\frac{1}{3}$ 
    
    \item $\frac{1}{12}$ 
    
    \item $\frac{1}{36}$ 
\end{enumerate}
\solution
		%\input{ncert/12/13/2/17/defs.tex}
	\item A bag contains 4 red and 4 black balls, another bag contains 2 red and 6 black balls. One of the two bags is selected at random and a ball is drawn from the bag which is found to be red. Find the probability that the ball is drawn from the first bag.
\\
\solution
		%\input{ncert/12/13/3/2/main.tex}
  \item
  Cards with numbers 2 to 101 are placed in a box. A card is selected at random.Find the probability that the card has
\begin{enumerate}[label=(\roman*)]
	\item an even number 
	\item a square number
\end{enumerate}
\solution
%\input{exemplar/10/13/3/32/main.tex}
\item
The king, queen and jack of clubs are removed from a deck of 52 playing cards and then well shuffled. Now one card is drawn at random from the remaining cards.  Determine the probability that the card is
\begin{enumerate}[label=(\roman*)]
\item a club
\item 10 of hearts
\end{enumerate}
\solution
%\input{exemplar/10/13/3/29/main.tex}
\item A team of medical students doing their internship have to assist during surgeries
at a city hospital. The probabilities of surgeries rated as very complex, complex,
routine, simple or very simple are respectively, 0.15, 0.20, 0.31, 0.26, .08. Find
the probabilities that a particular surgery will be rated
\begin{enumerate}
	\item complex or very complex;
	\item neither very complex nor very simple;
	\item routine or complex
	\item routine or simple
\end{enumerate}
\solution
%\input{exemplar/11/16/3/8(1)/main.tex}
\item A card is selected from a pack of 52 cards.
\begin{enumerate}[label=(\alph*)]
    \item How many points are there in the sample space?
    \item Calculate the probability that the card is an ace of spades.
    \item Calculate the probability that the card is (i) an ace and (ii) black card.
\end{enumerate}
\solution
%\input{exemplar/11/16/3/4/main2.tex}
\item The probability that a non leap year selected at random will contain 53 sundays.
\\
\solution
%\input{exemplar/10/13/1/19/main.tex}
\item One of the four persons John, Rita, Aslam or Gurpreet will be promoted next
month. Consequently the sample space consists of four elementary outcomes
S = {John promoted, Rita promoted, Aslam promoted, Gurpreet promoted}
You are told that the chances of John’s promotion is same as that of Gurpreet,
Rita’s chances of promotion are twice as likely as Johns. Aslam’s chances are
four times that of John.
\begin{enumerate}
	\item Determine
	\begin{enumerate}
		\item P (John promoted)
		\item P (Rita promoted)
		\item P (Aslam promoted)
		\item P (Gurpreet promoted)
	\end{enumerate}
	\item If A = {John promoted or Gurpreet promoted}, find P (A).
\end{enumerate}
\solution
%\input{exemplar/11/16/3/10/main.tex}
\item A card is drawn from a deck of 52 cards. Find the probability of getting a king or a heart or a red card.\\
\solution
%\input{exemplar/11/16/3/15/main.tex}
\item The probability that a student will pass his examination is 0.73, the probability of
the student getting a compartment is 0.13, and the probability that the student will
either pass or get compartment is 0.96. State True or False.\\
\solution
%\input{exemplar/11/16/3/31/main.tex}
\item A card is selected from a pack of 52 cards\\
\begin{enumerate}[label=(\alph*)]
\item How many points are there in the sample space?
\item Calculate the probability that the cards is an ace of spades.
\item Calculate the probability that the card is (i) an ace (ii)black card.\\
\end{enumerate}
%\input{ncert/11/16/3/4_1/Prob_4.tex}
\item In a non-leap year, the probability of having 53 tuesdays or 53 wednesdays is\\
\solution
%\input{exemplar/11/16/3/18/main.tex}
\item There are 1000 sealed envelopes in a box, 10 of them contain a cash prize of
Rs 100 each, 100 of them contain a cash prize of Rs 50 each and 200 of them
contain a cash prize of Rs 10 each and rest do not contain any cash prize. If they
are well shuffled and an envelope is picked up out, what is the probability that it
contains no cash prize?\\
\solution
%\input{exemplar/10/13/3/34/main.tex}
\item 
A die is thrown and a card is selected at random from a deck of 52 playing cards. The probability of getting an even number on the die and a spade card.\\
\solution
%\input{exemplar/12/13/3/78/main.tex}
\item
If 4-digit numbers greater than 5,000 are randomly formed from the digits 0, 1, 3, 5, and 7, what is the probability of forming a number divisible by 5 when:
\begin{enumerate}
    \item The digits are repeated?
    \item The repetition of digits is not allowed?
\end{enumerate}
\solution
%\input{ncert/11/16/4/9/main.tex}
\item Consider the probability space $\brak{\Omega, \mathcal{G}, P}$ where $\Omega = [0,2]$ and $\mathcal{G} = \cbrak{\phi, \Omega, [0,1], (1,2]}$. Let $X$ and $Y$ be two functions on $\Omega$ defined as
\begin{align*}
    X(\omega) = 
    \begin{cases}
        1 & \text{if }\omega \in [0, 1]\\
        2 & \text{if }\omega \in (1, 2]
    \end{cases}
\end{align*}
and
\begin{align*}
    Y(\omega) = 
    \begin{cases}
        2 & \text{if }\omega \in [0, 1.5]\\
        3 & \text{if }\omega \in (1.5, 2].
    \end{cases}
\end{align*}
Then which one of the following statements is true?
\begin{enumerate}
    \item [(A)] $X$ is a random variable with respect to $\mathcal{G}$, but $Y$ is not a random variable with respect to $\mathcal{G}$.
    \item [(B)] $Y$ is a random variable with respect to $\mathcal{G}$, but $X$ is not a random variable with respect to $\mathcal{G}$.
    \item [(C)] Neither $X$ nor $Y$ is a random variable with respect to $\mathcal{G}$.
    \item [(D)] Both $X$ and $Y$ are random variables with respect to $\mathcal{G}$.
\end{enumerate} \hfill (GATE ST 2023)\\
\solution
%\input{gate/ST/2023/14/main.tex}
	\item  A die is loaded in such a way that each odd number is twice as likely to occur as
each even number. Find $P(G)$, where $G$ is the event that a number greater than
3 occurs on a single roll of the die.
\\
\solution
		%\input{exemplar/11/16/3/5/main.tex}
	\item All the jacks, queens and kings are removed from a deck of 52 playing cards. The remaining cards are well shuffled and then one card is drawn at random. Giving ace a value 1 similar value for other cards, find the probability that the card has a value 
		\begin{enumerate}
			\item 7
			\item greater than 7
			\item less than 7
		\end{enumerate}
		%\input{exemplar/10/13/3/30/main.tex}
  \item A Lot consists of 48 mobile phones of which 42 are good, 3 have only minor defects and 3 have major defects.Varnika will buy a phone if it is good but the trader will only buy a mobile if it has no major defects. One phone is selected at random from the lot. What is the probability that it is
\begin{enumerate}
	\item acceptable to Varnika?
            \item acceptable to the trader?
\end{enumerate}
\solution
	%\input{exemplar/10/13/3/40/main.tex}
 \item A student says that if you throw a die, it will show up 1 or not 1. Therefore, the probability of getting 1 and the probability of getting 'not 1' each is equal to $\frac{1}{2}$. Is this correct? Give reasons.\\
 \solution
        %\input{exemplar/10/13/2/9/main.tex}
   \item Four candidates A, B, C, D have ap-
plied for the assignment to coach a school cricket
team. If A is twice as likely to be selected as B, and
B and C are given about the same chance of being
selected, while C is twice as likely to be selected
as D, what are the probabilities that
\begin{enumerate}
\item C will be selected?
\item A will not be selected?
\end{enumerate}
	%\input{exemplar/11/16/3/9/main.tex}
 \item A bag contain 24 balls of which $x$ balls are red, $2x$ are white and $3x$ are blue. A ball is selected at random, What is the probability that it is
\begin{enumerate}[label=\alph*)]
\item not red ?
\item white ?
\end{enumerate}
%\input{exemplar/10/13/3/41/main.tex}
If the letters of the word ASSASSINATION are arranged at random. Find the Probability that
\begin{enumerate}[label=(\alph*)]
\item Four $S's$ come consecutively in the word
\item Two  $I's$ and two $N's$ come together
\item All $A's$ are not coming together
\item No two $A's$ are coming together
\end{enumerate}
%\input{exemplar/11/16/3/14/main.tex}
	\item One urn contains two black balls (labelled B1 and B2) and one white ball. A
	second urn contains one black ball and two white balls (labelled W1 and W2).
	Suppose the following experiment is performed. One of the two urns is chosen
	at random. Next a ball is randomly chosen from the urn. Then a second ball is
	chosen at random from the same urn without replacing the first ball.
	
	\begin{enumerate}
	\item What is the probability that two black balls are chosen?
	
	\item What is the probability that two balls of opposite colour are chosen?
	\end{enumerate}
	\solution
	%\input{exemplar/11/16/3/12/main1.tex}
\end{enumerate}

	\item A bag contains $5$ red balls and some blue balls. If the probability of drawing a blue ball is double that if a red ball, determine the number of blue balls in the bag. 
		\\
\solution
		%\begin{enumerate}[label=\thesection.\arabic*,ref=\thesection.\theenumi]
	\item One card is drawn from a well-shuffled deck of 52 cards. Find the probability of getting
\begin{enumerate}
\item A king of red colour 
\item A face card 
\item A red face card
\item The jack of hearts
\item A spade
\item The queen of diamonds

\end{enumerate}
\solution
		%\input{ncert/10/15/1/14/main.tex}
	\item Five cards—the ten, jack, queen, king and ace of diamonds, are well-shuffled with their face downwards. One card is then picked up at random.
\begin{enumerate}
\item
What is the probability that the card is the queen? 
\item
If the queen is drawn and put aside, what is the probability that the second card picked up is (a) an ace? (b) a queen?\\
\end{enumerate}
\solution
		%\input{ncert/10/15/1/15/defs.tex}
	\item A bag contains $5$ red balls and some blue balls. If the probability of drawing a blue ball is double that if a red ball, determine the number of blue balls in the bag. 
		\\
\solution
		%\input{ncert/10/15/2/3/defs.tex}
	\item A card is selected from a pack of 52 cards.
 \begin{enumerate}[label=(\alph*)] 
                 \item How many points are there in the sample space?
                 \item Calculate the probability that the card is an ace of spades.
                 \item Calculate the probability that the card is (i) an ace and (ii) black card.
 \end{enumerate}
\solution
		%\input{ncert/11/16/3/4/main.tex}
\item Four cards are drawn from a well-shuffled deck of 52 cards. What is the probability of obtaining 3 diamonds and one spade.
\\
\solution
		%\input{ncert/11/16/4/2/defs.tex}
\item In a certain lottery 10,000 tickets are sold and ten equal prizes are awarded. What is the probability of not getting a prize if you buy (a) one ticket (b) two tickets (c) 10 tickets ?	
\\
\solution
		%\input{ncert/11/16/4/4/defs.tex}
		%
\item 
Out of 100 students, two sections of 40 and 60 are formed. If you and your friend are among the 100 students, what is the probability that
\begin{enumerate}
\item you both enter the same section?
\item you both enter the different sections?
\end{enumerate}
\solution
		%\input{ncert/11/16/4/5/defs.tex}
	\item 
The number lock of a suitcase has 4 wheels each labelled with ten digits i.e. from 0 to 9.The lock opens with a sequence of four digits with no repeats.What is the probability of a person getting the right sequence to open the suitcase.
\\
\solution
		%\input{ncert/11/16/4/10/defs.tex}
		%
\item 
Two cards are drawn at random and without replacement from a pack of 52 playing cards. Find the probability that both the cards are black.
\\
\solution
		%\input{ncert/12/13/2/2/defs.tex}
		\item A box of oranges is inspected by examining three randomly selected oranges drawn without replacement. If all the three oranges are good, the box is approved for sale, otherwise, it is rejected. Find the probability that a box containing 15 oranges out of which 12 are good and 3 are bad ones will be approved for sale.
		\label{ncert/12/13/2/3/defs.tex}
		\item Two balls are drawn at random with replacement from a box containing 10 black and 8 red balls. Find the probability that
		\label{ncert/12/13/2/12}
\begin{enumerate}
\item both balls are red.
\item first ball is black and second is red.
\item one of them is black and other is red.
\end{enumerate}

\item In a hostel, 60\% of the students read Hindi newspaper, 40\% read English newspaper and 20\% read both Hindi and English newspapers. A student is selected at random.
		\label{ncert/12/13/2/15}
\begin{enumerate}
\item Find the probability that she reads neither Hindi nor English newspapers.
\item If she reads Hindi newspaper, find the probability that she reads English newspaper.
\item If she reads English newspaper, find the probability that she reads Hindi newspaper.\\
\end{enumerate}
\item The probability of obtaining an even prime number on each die, when a pair of dice is rolled is 
\begin{enumerate}
    \item $0$ 
    
    \item $\frac{1}{3}$ 
    
    \item $\frac{1}{12}$ 
    
    \item $\frac{1}{36}$ 
\end{enumerate}
\solution
		%\input{ncert/12/13/2/17/defs.tex}
	\item A bag contains 4 red and 4 black balls, another bag contains 2 red and 6 black balls. One of the two bags is selected at random and a ball is drawn from the bag which is found to be red. Find the probability that the ball is drawn from the first bag.
\\
\solution
		%\input{ncert/12/13/3/2/main.tex}
  \item
  Cards with numbers 2 to 101 are placed in a box. A card is selected at random.Find the probability that the card has
\begin{enumerate}[label=(\roman*)]
	\item an even number 
	\item a square number
\end{enumerate}
\solution
%\input{exemplar/10/13/3/32/main.tex}
\item
The king, queen and jack of clubs are removed from a deck of 52 playing cards and then well shuffled. Now one card is drawn at random from the remaining cards.  Determine the probability that the card is
\begin{enumerate}[label=(\roman*)]
\item a club
\item 10 of hearts
\end{enumerate}
\solution
%\input{exemplar/10/13/3/29/main.tex}
\item A team of medical students doing their internship have to assist during surgeries
at a city hospital. The probabilities of surgeries rated as very complex, complex,
routine, simple or very simple are respectively, 0.15, 0.20, 0.31, 0.26, .08. Find
the probabilities that a particular surgery will be rated
\begin{enumerate}
	\item complex or very complex;
	\item neither very complex nor very simple;
	\item routine or complex
	\item routine or simple
\end{enumerate}
\solution
%\input{exemplar/11/16/3/8(1)/main.tex}
\item A card is selected from a pack of 52 cards.
\begin{enumerate}[label=(\alph*)]
    \item How many points are there in the sample space?
    \item Calculate the probability that the card is an ace of spades.
    \item Calculate the probability that the card is (i) an ace and (ii) black card.
\end{enumerate}
\solution
%\input{exemplar/11/16/3/4/main2.tex}
\item The probability that a non leap year selected at random will contain 53 sundays.
\\
\solution
%\input{exemplar/10/13/1/19/main.tex}
\item One of the four persons John, Rita, Aslam or Gurpreet will be promoted next
month. Consequently the sample space consists of four elementary outcomes
S = {John promoted, Rita promoted, Aslam promoted, Gurpreet promoted}
You are told that the chances of John’s promotion is same as that of Gurpreet,
Rita’s chances of promotion are twice as likely as Johns. Aslam’s chances are
four times that of John.
\begin{enumerate}
	\item Determine
	\begin{enumerate}
		\item P (John promoted)
		\item P (Rita promoted)
		\item P (Aslam promoted)
		\item P (Gurpreet promoted)
	\end{enumerate}
	\item If A = {John promoted or Gurpreet promoted}, find P (A).
\end{enumerate}
\solution
%\input{exemplar/11/16/3/10/main.tex}
\item A card is drawn from a deck of 52 cards. Find the probability of getting a king or a heart or a red card.\\
\solution
%\input{exemplar/11/16/3/15/main.tex}
\item The probability that a student will pass his examination is 0.73, the probability of
the student getting a compartment is 0.13, and the probability that the student will
either pass or get compartment is 0.96. State True or False.\\
\solution
%\input{exemplar/11/16/3/31/main.tex}
\item A card is selected from a pack of 52 cards\\
\begin{enumerate}[label=(\alph*)]
\item How many points are there in the sample space?
\item Calculate the probability that the cards is an ace of spades.
\item Calculate the probability that the card is (i) an ace (ii)black card.\\
\end{enumerate}
%\input{ncert/11/16/3/4_1/Prob_4.tex}
\item In a non-leap year, the probability of having 53 tuesdays or 53 wednesdays is\\
\solution
%\input{exemplar/11/16/3/18/main.tex}
\item There are 1000 sealed envelopes in a box, 10 of them contain a cash prize of
Rs 100 each, 100 of them contain a cash prize of Rs 50 each and 200 of them
contain a cash prize of Rs 10 each and rest do not contain any cash prize. If they
are well shuffled and an envelope is picked up out, what is the probability that it
contains no cash prize?\\
\solution
%\input{exemplar/10/13/3/34/main.tex}
\item 
A die is thrown and a card is selected at random from a deck of 52 playing cards. The probability of getting an even number on the die and a spade card.\\
\solution
%\input{exemplar/12/13/3/78/main.tex}
\item
If 4-digit numbers greater than 5,000 are randomly formed from the digits 0, 1, 3, 5, and 7, what is the probability of forming a number divisible by 5 when:
\begin{enumerate}
    \item The digits are repeated?
    \item The repetition of digits is not allowed?
\end{enumerate}
\solution
%\input{ncert/11/16/4/9/main.tex}
\item Consider the probability space $\brak{\Omega, \mathcal{G}, P}$ where $\Omega = [0,2]$ and $\mathcal{G} = \cbrak{\phi, \Omega, [0,1], (1,2]}$. Let $X$ and $Y$ be two functions on $\Omega$ defined as
\begin{align*}
    X(\omega) = 
    \begin{cases}
        1 & \text{if }\omega \in [0, 1]\\
        2 & \text{if }\omega \in (1, 2]
    \end{cases}
\end{align*}
and
\begin{align*}
    Y(\omega) = 
    \begin{cases}
        2 & \text{if }\omega \in [0, 1.5]\\
        3 & \text{if }\omega \in (1.5, 2].
    \end{cases}
\end{align*}
Then which one of the following statements is true?
\begin{enumerate}
    \item [(A)] $X$ is a random variable with respect to $\mathcal{G}$, but $Y$ is not a random variable with respect to $\mathcal{G}$.
    \item [(B)] $Y$ is a random variable with respect to $\mathcal{G}$, but $X$ is not a random variable with respect to $\mathcal{G}$.
    \item [(C)] Neither $X$ nor $Y$ is a random variable with respect to $\mathcal{G}$.
    \item [(D)] Both $X$ and $Y$ are random variables with respect to $\mathcal{G}$.
\end{enumerate} \hfill (GATE ST 2023)\\
\solution
%\input{gate/ST/2023/14/main.tex}
	\item  A die is loaded in such a way that each odd number is twice as likely to occur as
each even number. Find $P(G)$, where $G$ is the event that a number greater than
3 occurs on a single roll of the die.
\\
\solution
		%\input{exemplar/11/16/3/5/main.tex}
	\item All the jacks, queens and kings are removed from a deck of 52 playing cards. The remaining cards are well shuffled and then one card is drawn at random. Giving ace a value 1 similar value for other cards, find the probability that the card has a value 
		\begin{enumerate}
			\item 7
			\item greater than 7
			\item less than 7
		\end{enumerate}
		%\input{exemplar/10/13/3/30/main.tex}
  \item A Lot consists of 48 mobile phones of which 42 are good, 3 have only minor defects and 3 have major defects.Varnika will buy a phone if it is good but the trader will only buy a mobile if it has no major defects. One phone is selected at random from the lot. What is the probability that it is
\begin{enumerate}
	\item acceptable to Varnika?
            \item acceptable to the trader?
\end{enumerate}
\solution
	%\input{exemplar/10/13/3/40/main.tex}
 \item A student says that if you throw a die, it will show up 1 or not 1. Therefore, the probability of getting 1 and the probability of getting 'not 1' each is equal to $\frac{1}{2}$. Is this correct? Give reasons.\\
 \solution
        %\input{exemplar/10/13/2/9/main.tex}
   \item Four candidates A, B, C, D have ap-
plied for the assignment to coach a school cricket
team. If A is twice as likely to be selected as B, and
B and C are given about the same chance of being
selected, while C is twice as likely to be selected
as D, what are the probabilities that
\begin{enumerate}
\item C will be selected?
\item A will not be selected?
\end{enumerate}
	%\input{exemplar/11/16/3/9/main.tex}
 \item A bag contain 24 balls of which $x$ balls are red, $2x$ are white and $3x$ are blue. A ball is selected at random, What is the probability that it is
\begin{enumerate}[label=\alph*)]
\item not red ?
\item white ?
\end{enumerate}
%\input{exemplar/10/13/3/41/main.tex}
If the letters of the word ASSASSINATION are arranged at random. Find the Probability that
\begin{enumerate}[label=(\alph*)]
\item Four $S's$ come consecutively in the word
\item Two  $I's$ and two $N's$ come together
\item All $A's$ are not coming together
\item No two $A's$ are coming together
\end{enumerate}
%\input{exemplar/11/16/3/14/main.tex}
	\item One urn contains two black balls (labelled B1 and B2) and one white ball. A
	second urn contains one black ball and two white balls (labelled W1 and W2).
	Suppose the following experiment is performed. One of the two urns is chosen
	at random. Next a ball is randomly chosen from the urn. Then a second ball is
	chosen at random from the same urn without replacing the first ball.
	
	\begin{enumerate}
	\item What is the probability that two black balls are chosen?
	
	\item What is the probability that two balls of opposite colour are chosen?
	\end{enumerate}
	\solution
	%\input{exemplar/11/16/3/12/main1.tex}
\end{enumerate}

	\item A card is selected from a pack of 52 cards.
 \begin{enumerate}[label=(\alph*)] 
                 \item How many points are there in the sample space?
                 \item Calculate the probability that the card is an ace of spades.
                 \item Calculate the probability that the card is (i) an ace and (ii) black card.
 \end{enumerate}
\solution
		%\begin{table}[H]
	\centering
\begin{tabular}{|c|c|c|}
\hline
Random variable &Value &Definition\\ \hline
\multirow{3}{*}{X} &0 &Slips of Rs 1\\
&1 &Slips of Rs 5\\
&2 &Slips of Rs 13\\ \hline
\multirow{2}{*}{Y} &0 &Box A\\
&1 &Box B\\\hline
\end{tabular}
\caption{}
\label{tab:Distribution}
\end{table}
See \tabref{tab:Distribution}.
\begin{align}
p_{Y}\brak{k}= \begin{cases} 
      \frac{1}{3} & {k=0} \\
      \frac{2}{3 }& {k=1} 
   \end{cases}
   \\
p_{Y|X}\brak{0|0} = \frac{19}{25}\, 
p_{Y|X}\brak{0|1} = \frac{6}{25}\,
p_{Y|X}\brak{1|0} = \frac{45}{50}\,
p_{Y|X}\brak{1|2} = \frac{5}{50}
\end{align}
The desired probability is the probability that a slip drawn at random is marked other than Rs 1,
\begin{align}
&=1-p_X\brak{0}\\
&= p_X(1) + p_X(2)
\end{align}
Using Bayes theorem,
\begin{align}
&= p_Y\brak{0} \times \pr{Y=0 | X=1} + p_Y\brak{1} \times \pr{Y=1|X=2}\\
&=\frac{1}{3} \times \frac{6}{25} + \frac{2}{3} \times \frac{5}{50}\\
&=\frac{11}{75}
\end{align}

\newpage

%\tableofcontents

\bigskip

\renewcommand{\thefigure}{\theenumi}
\renewcommand{\thetable}{\theenumi}
%\renewcommand{\theequation}{\theenumi}

%\begin{abstract}
%%\boldmath
%In this letter, an algorithm for evaluating the exact analytical bit error rate  (BER)  for the piecewise linear (PL) combiner for  multiple relays is presented. Previous results were available only for upto three relays. The algorithm is unique in the sense that  the actual mathematical expressions, that are prohibitively large, need not be explicitly obtained. The diversity gain due to multiple relays is shown through plots of the analytical BER, well supported by simulations. 
%
%\end{abstract}
% IEEEtran.cls defaults to using nonbold math in the Abstract.
% This preserves the distinction between vectors and scalars. However,
% if the journal you are submitting to favors bold math in the abstract,
% then you can use LaTeX's standard command \boldmath at the very start
% of the abstract to achieve this. Many IEEE journals frown on math
% in the abstract anyway.

% Note that keywords are not normally used for peerreview papers.
%\begin{IEEEkeywords}
%Cooperative diversity, decode and forward, piecewise linear
%\end{IEEEkeywords}



% For peer review papers, you can put extra information on the cover
% page as needed:
% \ifCLASSOPTIONpeerreview
% \begin{center} \bfseries EDICS Category: 3-BBND \end{center}
% \fi
%
% For peerreview papers, this IEEEtran command inserts a page break and
% creates the second title. It will be ignored for other modes.
%\IEEEpeerreviewmaketitle




\item Four cards are drawn from a well-shuffled deck of 52 cards. What is the probability of obtaining 3 diamonds and one spade.
\\
\solution
		%\begin{enumerate}[label=\thesection.\arabic*,ref=\thesection.\theenumi]
	\item One card is drawn from a well-shuffled deck of 52 cards. Find the probability of getting
\begin{enumerate}
\item A king of red colour 
\item A face card 
\item A red face card
\item The jack of hearts
\item A spade
\item The queen of diamonds

\end{enumerate}
\solution
		%\input{ncert/10/15/1/14/main.tex}
	\item Five cards—the ten, jack, queen, king and ace of diamonds, are well-shuffled with their face downwards. One card is then picked up at random.
\begin{enumerate}
\item
What is the probability that the card is the queen? 
\item
If the queen is drawn and put aside, what is the probability that the second card picked up is (a) an ace? (b) a queen?\\
\end{enumerate}
\solution
		%\input{ncert/10/15/1/15/defs.tex}
	\item A bag contains $5$ red balls and some blue balls. If the probability of drawing a blue ball is double that if a red ball, determine the number of blue balls in the bag. 
		\\
\solution
		%\input{ncert/10/15/2/3/defs.tex}
	\item A card is selected from a pack of 52 cards.
 \begin{enumerate}[label=(\alph*)] 
                 \item How many points are there in the sample space?
                 \item Calculate the probability that the card is an ace of spades.
                 \item Calculate the probability that the card is (i) an ace and (ii) black card.
 \end{enumerate}
\solution
		%\input{ncert/11/16/3/4/main.tex}
\item Four cards are drawn from a well-shuffled deck of 52 cards. What is the probability of obtaining 3 diamonds and one spade.
\\
\solution
		%\input{ncert/11/16/4/2/defs.tex}
\item In a certain lottery 10,000 tickets are sold and ten equal prizes are awarded. What is the probability of not getting a prize if you buy (a) one ticket (b) two tickets (c) 10 tickets ?	
\\
\solution
		%\input{ncert/11/16/4/4/defs.tex}
		%
\item 
Out of 100 students, two sections of 40 and 60 are formed. If you and your friend are among the 100 students, what is the probability that
\begin{enumerate}
\item you both enter the same section?
\item you both enter the different sections?
\end{enumerate}
\solution
		%\input{ncert/11/16/4/5/defs.tex}
	\item 
The number lock of a suitcase has 4 wheels each labelled with ten digits i.e. from 0 to 9.The lock opens with a sequence of four digits with no repeats.What is the probability of a person getting the right sequence to open the suitcase.
\\
\solution
		%\input{ncert/11/16/4/10/defs.tex}
		%
\item 
Two cards are drawn at random and without replacement from a pack of 52 playing cards. Find the probability that both the cards are black.
\\
\solution
		%\input{ncert/12/13/2/2/defs.tex}
		\item A box of oranges is inspected by examining three randomly selected oranges drawn without replacement. If all the three oranges are good, the box is approved for sale, otherwise, it is rejected. Find the probability that a box containing 15 oranges out of which 12 are good and 3 are bad ones will be approved for sale.
		\label{ncert/12/13/2/3/defs.tex}
		\item Two balls are drawn at random with replacement from a box containing 10 black and 8 red balls. Find the probability that
		\label{ncert/12/13/2/12}
\begin{enumerate}
\item both balls are red.
\item first ball is black and second is red.
\item one of them is black and other is red.
\end{enumerate}

\item In a hostel, 60\% of the students read Hindi newspaper, 40\% read English newspaper and 20\% read both Hindi and English newspapers. A student is selected at random.
		\label{ncert/12/13/2/15}
\begin{enumerate}
\item Find the probability that she reads neither Hindi nor English newspapers.
\item If she reads Hindi newspaper, find the probability that she reads English newspaper.
\item If she reads English newspaper, find the probability that she reads Hindi newspaper.\\
\end{enumerate}
\item The probability of obtaining an even prime number on each die, when a pair of dice is rolled is 
\begin{enumerate}
    \item $0$ 
    
    \item $\frac{1}{3}$ 
    
    \item $\frac{1}{12}$ 
    
    \item $\frac{1}{36}$ 
\end{enumerate}
\solution
		%\input{ncert/12/13/2/17/defs.tex}
	\item A bag contains 4 red and 4 black balls, another bag contains 2 red and 6 black balls. One of the two bags is selected at random and a ball is drawn from the bag which is found to be red. Find the probability that the ball is drawn from the first bag.
\\
\solution
		%\input{ncert/12/13/3/2/main.tex}
  \item
  Cards with numbers 2 to 101 are placed in a box. A card is selected at random.Find the probability that the card has
\begin{enumerate}[label=(\roman*)]
	\item an even number 
	\item a square number
\end{enumerate}
\solution
%\input{exemplar/10/13/3/32/main.tex}
\item
The king, queen and jack of clubs are removed from a deck of 52 playing cards and then well shuffled. Now one card is drawn at random from the remaining cards.  Determine the probability that the card is
\begin{enumerate}[label=(\roman*)]
\item a club
\item 10 of hearts
\end{enumerate}
\solution
%\input{exemplar/10/13/3/29/main.tex}
\item A team of medical students doing their internship have to assist during surgeries
at a city hospital. The probabilities of surgeries rated as very complex, complex,
routine, simple or very simple are respectively, 0.15, 0.20, 0.31, 0.26, .08. Find
the probabilities that a particular surgery will be rated
\begin{enumerate}
	\item complex or very complex;
	\item neither very complex nor very simple;
	\item routine or complex
	\item routine or simple
\end{enumerate}
\solution
%\input{exemplar/11/16/3/8(1)/main.tex}
\item A card is selected from a pack of 52 cards.
\begin{enumerate}[label=(\alph*)]
    \item How many points are there in the sample space?
    \item Calculate the probability that the card is an ace of spades.
    \item Calculate the probability that the card is (i) an ace and (ii) black card.
\end{enumerate}
\solution
%\input{exemplar/11/16/3/4/main2.tex}
\item The probability that a non leap year selected at random will contain 53 sundays.
\\
\solution
%\input{exemplar/10/13/1/19/main.tex}
\item One of the four persons John, Rita, Aslam or Gurpreet will be promoted next
month. Consequently the sample space consists of four elementary outcomes
S = {John promoted, Rita promoted, Aslam promoted, Gurpreet promoted}
You are told that the chances of John’s promotion is same as that of Gurpreet,
Rita’s chances of promotion are twice as likely as Johns. Aslam’s chances are
four times that of John.
\begin{enumerate}
	\item Determine
	\begin{enumerate}
		\item P (John promoted)
		\item P (Rita promoted)
		\item P (Aslam promoted)
		\item P (Gurpreet promoted)
	\end{enumerate}
	\item If A = {John promoted or Gurpreet promoted}, find P (A).
\end{enumerate}
\solution
%\input{exemplar/11/16/3/10/main.tex}
\item A card is drawn from a deck of 52 cards. Find the probability of getting a king or a heart or a red card.\\
\solution
%\input{exemplar/11/16/3/15/main.tex}
\item The probability that a student will pass his examination is 0.73, the probability of
the student getting a compartment is 0.13, and the probability that the student will
either pass or get compartment is 0.96. State True or False.\\
\solution
%\input{exemplar/11/16/3/31/main.tex}
\item A card is selected from a pack of 52 cards\\
\begin{enumerate}[label=(\alph*)]
\item How many points are there in the sample space?
\item Calculate the probability that the cards is an ace of spades.
\item Calculate the probability that the card is (i) an ace (ii)black card.\\
\end{enumerate}
%\input{ncert/11/16/3/4_1/Prob_4.tex}
\item In a non-leap year, the probability of having 53 tuesdays or 53 wednesdays is\\
\solution
%\input{exemplar/11/16/3/18/main.tex}
\item There are 1000 sealed envelopes in a box, 10 of them contain a cash prize of
Rs 100 each, 100 of them contain a cash prize of Rs 50 each and 200 of them
contain a cash prize of Rs 10 each and rest do not contain any cash prize. If they
are well shuffled and an envelope is picked up out, what is the probability that it
contains no cash prize?\\
\solution
%\input{exemplar/10/13/3/34/main.tex}
\item 
A die is thrown and a card is selected at random from a deck of 52 playing cards. The probability of getting an even number on the die and a spade card.\\
\solution
%\input{exemplar/12/13/3/78/main.tex}
\item
If 4-digit numbers greater than 5,000 are randomly formed from the digits 0, 1, 3, 5, and 7, what is the probability of forming a number divisible by 5 when:
\begin{enumerate}
    \item The digits are repeated?
    \item The repetition of digits is not allowed?
\end{enumerate}
\solution
%\input{ncert/11/16/4/9/main.tex}
\item Consider the probability space $\brak{\Omega, \mathcal{G}, P}$ where $\Omega = [0,2]$ and $\mathcal{G} = \cbrak{\phi, \Omega, [0,1], (1,2]}$. Let $X$ and $Y$ be two functions on $\Omega$ defined as
\begin{align*}
    X(\omega) = 
    \begin{cases}
        1 & \text{if }\omega \in [0, 1]\\
        2 & \text{if }\omega \in (1, 2]
    \end{cases}
\end{align*}
and
\begin{align*}
    Y(\omega) = 
    \begin{cases}
        2 & \text{if }\omega \in [0, 1.5]\\
        3 & \text{if }\omega \in (1.5, 2].
    \end{cases}
\end{align*}
Then which one of the following statements is true?
\begin{enumerate}
    \item [(A)] $X$ is a random variable with respect to $\mathcal{G}$, but $Y$ is not a random variable with respect to $\mathcal{G}$.
    \item [(B)] $Y$ is a random variable with respect to $\mathcal{G}$, but $X$ is not a random variable with respect to $\mathcal{G}$.
    \item [(C)] Neither $X$ nor $Y$ is a random variable with respect to $\mathcal{G}$.
    \item [(D)] Both $X$ and $Y$ are random variables with respect to $\mathcal{G}$.
\end{enumerate} \hfill (GATE ST 2023)\\
\solution
%\input{gate/ST/2023/14/main.tex}
	\item  A die is loaded in such a way that each odd number is twice as likely to occur as
each even number. Find $P(G)$, where $G$ is the event that a number greater than
3 occurs on a single roll of the die.
\\
\solution
		%\input{exemplar/11/16/3/5/main.tex}
	\item All the jacks, queens and kings are removed from a deck of 52 playing cards. The remaining cards are well shuffled and then one card is drawn at random. Giving ace a value 1 similar value for other cards, find the probability that the card has a value 
		\begin{enumerate}
			\item 7
			\item greater than 7
			\item less than 7
		\end{enumerate}
		%\input{exemplar/10/13/3/30/main.tex}
  \item A Lot consists of 48 mobile phones of which 42 are good, 3 have only minor defects and 3 have major defects.Varnika will buy a phone if it is good but the trader will only buy a mobile if it has no major defects. One phone is selected at random from the lot. What is the probability that it is
\begin{enumerate}
	\item acceptable to Varnika?
            \item acceptable to the trader?
\end{enumerate}
\solution
	%\input{exemplar/10/13/3/40/main.tex}
 \item A student says that if you throw a die, it will show up 1 or not 1. Therefore, the probability of getting 1 and the probability of getting 'not 1' each is equal to $\frac{1}{2}$. Is this correct? Give reasons.\\
 \solution
        %\input{exemplar/10/13/2/9/main.tex}
   \item Four candidates A, B, C, D have ap-
plied for the assignment to coach a school cricket
team. If A is twice as likely to be selected as B, and
B and C are given about the same chance of being
selected, while C is twice as likely to be selected
as D, what are the probabilities that
\begin{enumerate}
\item C will be selected?
\item A will not be selected?
\end{enumerate}
	%\input{exemplar/11/16/3/9/main.tex}
 \item A bag contain 24 balls of which $x$ balls are red, $2x$ are white and $3x$ are blue. A ball is selected at random, What is the probability that it is
\begin{enumerate}[label=\alph*)]
\item not red ?
\item white ?
\end{enumerate}
%\input{exemplar/10/13/3/41/main.tex}
If the letters of the word ASSASSINATION are arranged at random. Find the Probability that
\begin{enumerate}[label=(\alph*)]
\item Four $S's$ come consecutively in the word
\item Two  $I's$ and two $N's$ come together
\item All $A's$ are not coming together
\item No two $A's$ are coming together
\end{enumerate}
%\input{exemplar/11/16/3/14/main.tex}
	\item One urn contains two black balls (labelled B1 and B2) and one white ball. A
	second urn contains one black ball and two white balls (labelled W1 and W2).
	Suppose the following experiment is performed. One of the two urns is chosen
	at random. Next a ball is randomly chosen from the urn. Then a second ball is
	chosen at random from the same urn without replacing the first ball.
	
	\begin{enumerate}
	\item What is the probability that two black balls are chosen?
	
	\item What is the probability that two balls of opposite colour are chosen?
	\end{enumerate}
	\solution
	%\input{exemplar/11/16/3/12/main1.tex}
\end{enumerate}

\item In a certain lottery 10,000 tickets are sold and ten equal prizes are awarded. What is the probability of not getting a prize if you buy (a) one ticket (b) two tickets (c) 10 tickets ?	
\\
\solution
		%\begin{enumerate}[label=\thesection.\arabic*,ref=\thesection.\theenumi]
	\item One card is drawn from a well-shuffled deck of 52 cards. Find the probability of getting
\begin{enumerate}
\item A king of red colour 
\item A face card 
\item A red face card
\item The jack of hearts
\item A spade
\item The queen of diamonds

\end{enumerate}
\solution
		%\input{ncert/10/15/1/14/main.tex}
	\item Five cards—the ten, jack, queen, king and ace of diamonds, are well-shuffled with their face downwards. One card is then picked up at random.
\begin{enumerate}
\item
What is the probability that the card is the queen? 
\item
If the queen is drawn and put aside, what is the probability that the second card picked up is (a) an ace? (b) a queen?\\
\end{enumerate}
\solution
		%\input{ncert/10/15/1/15/defs.tex}
	\item A bag contains $5$ red balls and some blue balls. If the probability of drawing a blue ball is double that if a red ball, determine the number of blue balls in the bag. 
		\\
\solution
		%\input{ncert/10/15/2/3/defs.tex}
	\item A card is selected from a pack of 52 cards.
 \begin{enumerate}[label=(\alph*)] 
                 \item How many points are there in the sample space?
                 \item Calculate the probability that the card is an ace of spades.
                 \item Calculate the probability that the card is (i) an ace and (ii) black card.
 \end{enumerate}
\solution
		%\input{ncert/11/16/3/4/main.tex}
\item Four cards are drawn from a well-shuffled deck of 52 cards. What is the probability of obtaining 3 diamonds and one spade.
\\
\solution
		%\input{ncert/11/16/4/2/defs.tex}
\item In a certain lottery 10,000 tickets are sold and ten equal prizes are awarded. What is the probability of not getting a prize if you buy (a) one ticket (b) two tickets (c) 10 tickets ?	
\\
\solution
		%\input{ncert/11/16/4/4/defs.tex}
		%
\item 
Out of 100 students, two sections of 40 and 60 are formed. If you and your friend are among the 100 students, what is the probability that
\begin{enumerate}
\item you both enter the same section?
\item you both enter the different sections?
\end{enumerate}
\solution
		%\input{ncert/11/16/4/5/defs.tex}
	\item 
The number lock of a suitcase has 4 wheels each labelled with ten digits i.e. from 0 to 9.The lock opens with a sequence of four digits with no repeats.What is the probability of a person getting the right sequence to open the suitcase.
\\
\solution
		%\input{ncert/11/16/4/10/defs.tex}
		%
\item 
Two cards are drawn at random and without replacement from a pack of 52 playing cards. Find the probability that both the cards are black.
\\
\solution
		%\input{ncert/12/13/2/2/defs.tex}
		\item A box of oranges is inspected by examining three randomly selected oranges drawn without replacement. If all the three oranges are good, the box is approved for sale, otherwise, it is rejected. Find the probability that a box containing 15 oranges out of which 12 are good and 3 are bad ones will be approved for sale.
		\label{ncert/12/13/2/3/defs.tex}
		\item Two balls are drawn at random with replacement from a box containing 10 black and 8 red balls. Find the probability that
		\label{ncert/12/13/2/12}
\begin{enumerate}
\item both balls are red.
\item first ball is black and second is red.
\item one of them is black and other is red.
\end{enumerate}

\item In a hostel, 60\% of the students read Hindi newspaper, 40\% read English newspaper and 20\% read both Hindi and English newspapers. A student is selected at random.
		\label{ncert/12/13/2/15}
\begin{enumerate}
\item Find the probability that she reads neither Hindi nor English newspapers.
\item If she reads Hindi newspaper, find the probability that she reads English newspaper.
\item If she reads English newspaper, find the probability that she reads Hindi newspaper.\\
\end{enumerate}
\item The probability of obtaining an even prime number on each die, when a pair of dice is rolled is 
\begin{enumerate}
    \item $0$ 
    
    \item $\frac{1}{3}$ 
    
    \item $\frac{1}{12}$ 
    
    \item $\frac{1}{36}$ 
\end{enumerate}
\solution
		%\input{ncert/12/13/2/17/defs.tex}
	\item A bag contains 4 red and 4 black balls, another bag contains 2 red and 6 black balls. One of the two bags is selected at random and a ball is drawn from the bag which is found to be red. Find the probability that the ball is drawn from the first bag.
\\
\solution
		%\input{ncert/12/13/3/2/main.tex}
  \item
  Cards with numbers 2 to 101 are placed in a box. A card is selected at random.Find the probability that the card has
\begin{enumerate}[label=(\roman*)]
	\item an even number 
	\item a square number
\end{enumerate}
\solution
%\input{exemplar/10/13/3/32/main.tex}
\item
The king, queen and jack of clubs are removed from a deck of 52 playing cards and then well shuffled. Now one card is drawn at random from the remaining cards.  Determine the probability that the card is
\begin{enumerate}[label=(\roman*)]
\item a club
\item 10 of hearts
\end{enumerate}
\solution
%\input{exemplar/10/13/3/29/main.tex}
\item A team of medical students doing their internship have to assist during surgeries
at a city hospital. The probabilities of surgeries rated as very complex, complex,
routine, simple or very simple are respectively, 0.15, 0.20, 0.31, 0.26, .08. Find
the probabilities that a particular surgery will be rated
\begin{enumerate}
	\item complex or very complex;
	\item neither very complex nor very simple;
	\item routine or complex
	\item routine or simple
\end{enumerate}
\solution
%\input{exemplar/11/16/3/8(1)/main.tex}
\item A card is selected from a pack of 52 cards.
\begin{enumerate}[label=(\alph*)]
    \item How many points are there in the sample space?
    \item Calculate the probability that the card is an ace of spades.
    \item Calculate the probability that the card is (i) an ace and (ii) black card.
\end{enumerate}
\solution
%\input{exemplar/11/16/3/4/main2.tex}
\item The probability that a non leap year selected at random will contain 53 sundays.
\\
\solution
%\input{exemplar/10/13/1/19/main.tex}
\item One of the four persons John, Rita, Aslam or Gurpreet will be promoted next
month. Consequently the sample space consists of four elementary outcomes
S = {John promoted, Rita promoted, Aslam promoted, Gurpreet promoted}
You are told that the chances of John’s promotion is same as that of Gurpreet,
Rita’s chances of promotion are twice as likely as Johns. Aslam’s chances are
four times that of John.
\begin{enumerate}
	\item Determine
	\begin{enumerate}
		\item P (John promoted)
		\item P (Rita promoted)
		\item P (Aslam promoted)
		\item P (Gurpreet promoted)
	\end{enumerate}
	\item If A = {John promoted or Gurpreet promoted}, find P (A).
\end{enumerate}
\solution
%\input{exemplar/11/16/3/10/main.tex}
\item A card is drawn from a deck of 52 cards. Find the probability of getting a king or a heart or a red card.\\
\solution
%\input{exemplar/11/16/3/15/main.tex}
\item The probability that a student will pass his examination is 0.73, the probability of
the student getting a compartment is 0.13, and the probability that the student will
either pass or get compartment is 0.96. State True or False.\\
\solution
%\input{exemplar/11/16/3/31/main.tex}
\item A card is selected from a pack of 52 cards\\
\begin{enumerate}[label=(\alph*)]
\item How many points are there in the sample space?
\item Calculate the probability that the cards is an ace of spades.
\item Calculate the probability that the card is (i) an ace (ii)black card.\\
\end{enumerate}
%\input{ncert/11/16/3/4_1/Prob_4.tex}
\item In a non-leap year, the probability of having 53 tuesdays or 53 wednesdays is\\
\solution
%\input{exemplar/11/16/3/18/main.tex}
\item There are 1000 sealed envelopes in a box, 10 of them contain a cash prize of
Rs 100 each, 100 of them contain a cash prize of Rs 50 each and 200 of them
contain a cash prize of Rs 10 each and rest do not contain any cash prize. If they
are well shuffled and an envelope is picked up out, what is the probability that it
contains no cash prize?\\
\solution
%\input{exemplar/10/13/3/34/main.tex}
\item 
A die is thrown and a card is selected at random from a deck of 52 playing cards. The probability of getting an even number on the die and a spade card.\\
\solution
%\input{exemplar/12/13/3/78/main.tex}
\item
If 4-digit numbers greater than 5,000 are randomly formed from the digits 0, 1, 3, 5, and 7, what is the probability of forming a number divisible by 5 when:
\begin{enumerate}
    \item The digits are repeated?
    \item The repetition of digits is not allowed?
\end{enumerate}
\solution
%\input{ncert/11/16/4/9/main.tex}
\item Consider the probability space $\brak{\Omega, \mathcal{G}, P}$ where $\Omega = [0,2]$ and $\mathcal{G} = \cbrak{\phi, \Omega, [0,1], (1,2]}$. Let $X$ and $Y$ be two functions on $\Omega$ defined as
\begin{align*}
    X(\omega) = 
    \begin{cases}
        1 & \text{if }\omega \in [0, 1]\\
        2 & \text{if }\omega \in (1, 2]
    \end{cases}
\end{align*}
and
\begin{align*}
    Y(\omega) = 
    \begin{cases}
        2 & \text{if }\omega \in [0, 1.5]\\
        3 & \text{if }\omega \in (1.5, 2].
    \end{cases}
\end{align*}
Then which one of the following statements is true?
\begin{enumerate}
    \item [(A)] $X$ is a random variable with respect to $\mathcal{G}$, but $Y$ is not a random variable with respect to $\mathcal{G}$.
    \item [(B)] $Y$ is a random variable with respect to $\mathcal{G}$, but $X$ is not a random variable with respect to $\mathcal{G}$.
    \item [(C)] Neither $X$ nor $Y$ is a random variable with respect to $\mathcal{G}$.
    \item [(D)] Both $X$ and $Y$ are random variables with respect to $\mathcal{G}$.
\end{enumerate} \hfill (GATE ST 2023)\\
\solution
%\input{gate/ST/2023/14/main.tex}
	\item  A die is loaded in such a way that each odd number is twice as likely to occur as
each even number. Find $P(G)$, where $G$ is the event that a number greater than
3 occurs on a single roll of the die.
\\
\solution
		%\input{exemplar/11/16/3/5/main.tex}
	\item All the jacks, queens and kings are removed from a deck of 52 playing cards. The remaining cards are well shuffled and then one card is drawn at random. Giving ace a value 1 similar value for other cards, find the probability that the card has a value 
		\begin{enumerate}
			\item 7
			\item greater than 7
			\item less than 7
		\end{enumerate}
		%\input{exemplar/10/13/3/30/main.tex}
  \item A Lot consists of 48 mobile phones of which 42 are good, 3 have only minor defects and 3 have major defects.Varnika will buy a phone if it is good but the trader will only buy a mobile if it has no major defects. One phone is selected at random from the lot. What is the probability that it is
\begin{enumerate}
	\item acceptable to Varnika?
            \item acceptable to the trader?
\end{enumerate}
\solution
	%\input{exemplar/10/13/3/40/main.tex}
 \item A student says that if you throw a die, it will show up 1 or not 1. Therefore, the probability of getting 1 and the probability of getting 'not 1' each is equal to $\frac{1}{2}$. Is this correct? Give reasons.\\
 \solution
        %\input{exemplar/10/13/2/9/main.tex}
   \item Four candidates A, B, C, D have ap-
plied for the assignment to coach a school cricket
team. If A is twice as likely to be selected as B, and
B and C are given about the same chance of being
selected, while C is twice as likely to be selected
as D, what are the probabilities that
\begin{enumerate}
\item C will be selected?
\item A will not be selected?
\end{enumerate}
	%\input{exemplar/11/16/3/9/main.tex}
 \item A bag contain 24 balls of which $x$ balls are red, $2x$ are white and $3x$ are blue. A ball is selected at random, What is the probability that it is
\begin{enumerate}[label=\alph*)]
\item not red ?
\item white ?
\end{enumerate}
%\input{exemplar/10/13/3/41/main.tex}
If the letters of the word ASSASSINATION are arranged at random. Find the Probability that
\begin{enumerate}[label=(\alph*)]
\item Four $S's$ come consecutively in the word
\item Two  $I's$ and two $N's$ come together
\item All $A's$ are not coming together
\item No two $A's$ are coming together
\end{enumerate}
%\input{exemplar/11/16/3/14/main.tex}
	\item One urn contains two black balls (labelled B1 and B2) and one white ball. A
	second urn contains one black ball and two white balls (labelled W1 and W2).
	Suppose the following experiment is performed. One of the two urns is chosen
	at random. Next a ball is randomly chosen from the urn. Then a second ball is
	chosen at random from the same urn without replacing the first ball.
	
	\begin{enumerate}
	\item What is the probability that two black balls are chosen?
	
	\item What is the probability that two balls of opposite colour are chosen?
	\end{enumerate}
	\solution
	%\input{exemplar/11/16/3/12/main1.tex}
\end{enumerate}

		%
\item 
Out of 100 students, two sections of 40 and 60 are formed. If you and your friend are among the 100 students, what is the probability that
\begin{enumerate}
\item you both enter the same section?
\item you both enter the different sections?
\end{enumerate}
\solution
		%\begin{enumerate}[label=\thesection.\arabic*,ref=\thesection.\theenumi]
	\item One card is drawn from a well-shuffled deck of 52 cards. Find the probability of getting
\begin{enumerate}
\item A king of red colour 
\item A face card 
\item A red face card
\item The jack of hearts
\item A spade
\item The queen of diamonds

\end{enumerate}
\solution
		%\input{ncert/10/15/1/14/main.tex}
	\item Five cards—the ten, jack, queen, king and ace of diamonds, are well-shuffled with their face downwards. One card is then picked up at random.
\begin{enumerate}
\item
What is the probability that the card is the queen? 
\item
If the queen is drawn and put aside, what is the probability that the second card picked up is (a) an ace? (b) a queen?\\
\end{enumerate}
\solution
		%\input{ncert/10/15/1/15/defs.tex}
	\item A bag contains $5$ red balls and some blue balls. If the probability of drawing a blue ball is double that if a red ball, determine the number of blue balls in the bag. 
		\\
\solution
		%\input{ncert/10/15/2/3/defs.tex}
	\item A card is selected from a pack of 52 cards.
 \begin{enumerate}[label=(\alph*)] 
                 \item How many points are there in the sample space?
                 \item Calculate the probability that the card is an ace of spades.
                 \item Calculate the probability that the card is (i) an ace and (ii) black card.
 \end{enumerate}
\solution
		%\input{ncert/11/16/3/4/main.tex}
\item Four cards are drawn from a well-shuffled deck of 52 cards. What is the probability of obtaining 3 diamonds and one spade.
\\
\solution
		%\input{ncert/11/16/4/2/defs.tex}
\item In a certain lottery 10,000 tickets are sold and ten equal prizes are awarded. What is the probability of not getting a prize if you buy (a) one ticket (b) two tickets (c) 10 tickets ?	
\\
\solution
		%\input{ncert/11/16/4/4/defs.tex}
		%
\item 
Out of 100 students, two sections of 40 and 60 are formed. If you and your friend are among the 100 students, what is the probability that
\begin{enumerate}
\item you both enter the same section?
\item you both enter the different sections?
\end{enumerate}
\solution
		%\input{ncert/11/16/4/5/defs.tex}
	\item 
The number lock of a suitcase has 4 wheels each labelled with ten digits i.e. from 0 to 9.The lock opens with a sequence of four digits with no repeats.What is the probability of a person getting the right sequence to open the suitcase.
\\
\solution
		%\input{ncert/11/16/4/10/defs.tex}
		%
\item 
Two cards are drawn at random and without replacement from a pack of 52 playing cards. Find the probability that both the cards are black.
\\
\solution
		%\input{ncert/12/13/2/2/defs.tex}
		\item A box of oranges is inspected by examining three randomly selected oranges drawn without replacement. If all the three oranges are good, the box is approved for sale, otherwise, it is rejected. Find the probability that a box containing 15 oranges out of which 12 are good and 3 are bad ones will be approved for sale.
		\label{ncert/12/13/2/3/defs.tex}
		\item Two balls are drawn at random with replacement from a box containing 10 black and 8 red balls. Find the probability that
		\label{ncert/12/13/2/12}
\begin{enumerate}
\item both balls are red.
\item first ball is black and second is red.
\item one of them is black and other is red.
\end{enumerate}

\item In a hostel, 60\% of the students read Hindi newspaper, 40\% read English newspaper and 20\% read both Hindi and English newspapers. A student is selected at random.
		\label{ncert/12/13/2/15}
\begin{enumerate}
\item Find the probability that she reads neither Hindi nor English newspapers.
\item If she reads Hindi newspaper, find the probability that she reads English newspaper.
\item If she reads English newspaper, find the probability that she reads Hindi newspaper.\\
\end{enumerate}
\item The probability of obtaining an even prime number on each die, when a pair of dice is rolled is 
\begin{enumerate}
    \item $0$ 
    
    \item $\frac{1}{3}$ 
    
    \item $\frac{1}{12}$ 
    
    \item $\frac{1}{36}$ 
\end{enumerate}
\solution
		%\input{ncert/12/13/2/17/defs.tex}
	\item A bag contains 4 red and 4 black balls, another bag contains 2 red and 6 black balls. One of the two bags is selected at random and a ball is drawn from the bag which is found to be red. Find the probability that the ball is drawn from the first bag.
\\
\solution
		%\input{ncert/12/13/3/2/main.tex}
  \item
  Cards with numbers 2 to 101 are placed in a box. A card is selected at random.Find the probability that the card has
\begin{enumerate}[label=(\roman*)]
	\item an even number 
	\item a square number
\end{enumerate}
\solution
%\input{exemplar/10/13/3/32/main.tex}
\item
The king, queen and jack of clubs are removed from a deck of 52 playing cards and then well shuffled. Now one card is drawn at random from the remaining cards.  Determine the probability that the card is
\begin{enumerate}[label=(\roman*)]
\item a club
\item 10 of hearts
\end{enumerate}
\solution
%\input{exemplar/10/13/3/29/main.tex}
\item A team of medical students doing their internship have to assist during surgeries
at a city hospital. The probabilities of surgeries rated as very complex, complex,
routine, simple or very simple are respectively, 0.15, 0.20, 0.31, 0.26, .08. Find
the probabilities that a particular surgery will be rated
\begin{enumerate}
	\item complex or very complex;
	\item neither very complex nor very simple;
	\item routine or complex
	\item routine or simple
\end{enumerate}
\solution
%\input{exemplar/11/16/3/8(1)/main.tex}
\item A card is selected from a pack of 52 cards.
\begin{enumerate}[label=(\alph*)]
    \item How many points are there in the sample space?
    \item Calculate the probability that the card is an ace of spades.
    \item Calculate the probability that the card is (i) an ace and (ii) black card.
\end{enumerate}
\solution
%\input{exemplar/11/16/3/4/main2.tex}
\item The probability that a non leap year selected at random will contain 53 sundays.
\\
\solution
%\input{exemplar/10/13/1/19/main.tex}
\item One of the four persons John, Rita, Aslam or Gurpreet will be promoted next
month. Consequently the sample space consists of four elementary outcomes
S = {John promoted, Rita promoted, Aslam promoted, Gurpreet promoted}
You are told that the chances of John’s promotion is same as that of Gurpreet,
Rita’s chances of promotion are twice as likely as Johns. Aslam’s chances are
four times that of John.
\begin{enumerate}
	\item Determine
	\begin{enumerate}
		\item P (John promoted)
		\item P (Rita promoted)
		\item P (Aslam promoted)
		\item P (Gurpreet promoted)
	\end{enumerate}
	\item If A = {John promoted or Gurpreet promoted}, find P (A).
\end{enumerate}
\solution
%\input{exemplar/11/16/3/10/main.tex}
\item A card is drawn from a deck of 52 cards. Find the probability of getting a king or a heart or a red card.\\
\solution
%\input{exemplar/11/16/3/15/main.tex}
\item The probability that a student will pass his examination is 0.73, the probability of
the student getting a compartment is 0.13, and the probability that the student will
either pass or get compartment is 0.96. State True or False.\\
\solution
%\input{exemplar/11/16/3/31/main.tex}
\item A card is selected from a pack of 52 cards\\
\begin{enumerate}[label=(\alph*)]
\item How many points are there in the sample space?
\item Calculate the probability that the cards is an ace of spades.
\item Calculate the probability that the card is (i) an ace (ii)black card.\\
\end{enumerate}
%\input{ncert/11/16/3/4_1/Prob_4.tex}
\item In a non-leap year, the probability of having 53 tuesdays or 53 wednesdays is\\
\solution
%\input{exemplar/11/16/3/18/main.tex}
\item There are 1000 sealed envelopes in a box, 10 of them contain a cash prize of
Rs 100 each, 100 of them contain a cash prize of Rs 50 each and 200 of them
contain a cash prize of Rs 10 each and rest do not contain any cash prize. If they
are well shuffled and an envelope is picked up out, what is the probability that it
contains no cash prize?\\
\solution
%\input{exemplar/10/13/3/34/main.tex}
\item 
A die is thrown and a card is selected at random from a deck of 52 playing cards. The probability of getting an even number on the die and a spade card.\\
\solution
%\input{exemplar/12/13/3/78/main.tex}
\item
If 4-digit numbers greater than 5,000 are randomly formed from the digits 0, 1, 3, 5, and 7, what is the probability of forming a number divisible by 5 when:
\begin{enumerate}
    \item The digits are repeated?
    \item The repetition of digits is not allowed?
\end{enumerate}
\solution
%\input{ncert/11/16/4/9/main.tex}
\item Consider the probability space $\brak{\Omega, \mathcal{G}, P}$ where $\Omega = [0,2]$ and $\mathcal{G} = \cbrak{\phi, \Omega, [0,1], (1,2]}$. Let $X$ and $Y$ be two functions on $\Omega$ defined as
\begin{align*}
    X(\omega) = 
    \begin{cases}
        1 & \text{if }\omega \in [0, 1]\\
        2 & \text{if }\omega \in (1, 2]
    \end{cases}
\end{align*}
and
\begin{align*}
    Y(\omega) = 
    \begin{cases}
        2 & \text{if }\omega \in [0, 1.5]\\
        3 & \text{if }\omega \in (1.5, 2].
    \end{cases}
\end{align*}
Then which one of the following statements is true?
\begin{enumerate}
    \item [(A)] $X$ is a random variable with respect to $\mathcal{G}$, but $Y$ is not a random variable with respect to $\mathcal{G}$.
    \item [(B)] $Y$ is a random variable with respect to $\mathcal{G}$, but $X$ is not a random variable with respect to $\mathcal{G}$.
    \item [(C)] Neither $X$ nor $Y$ is a random variable with respect to $\mathcal{G}$.
    \item [(D)] Both $X$ and $Y$ are random variables with respect to $\mathcal{G}$.
\end{enumerate} \hfill (GATE ST 2023)\\
\solution
%\input{gate/ST/2023/14/main.tex}
	\item  A die is loaded in such a way that each odd number is twice as likely to occur as
each even number. Find $P(G)$, where $G$ is the event that a number greater than
3 occurs on a single roll of the die.
\\
\solution
		%\input{exemplar/11/16/3/5/main.tex}
	\item All the jacks, queens and kings are removed from a deck of 52 playing cards. The remaining cards are well shuffled and then one card is drawn at random. Giving ace a value 1 similar value for other cards, find the probability that the card has a value 
		\begin{enumerate}
			\item 7
			\item greater than 7
			\item less than 7
		\end{enumerate}
		%\input{exemplar/10/13/3/30/main.tex}
  \item A Lot consists of 48 mobile phones of which 42 are good, 3 have only minor defects and 3 have major defects.Varnika will buy a phone if it is good but the trader will only buy a mobile if it has no major defects. One phone is selected at random from the lot. What is the probability that it is
\begin{enumerate}
	\item acceptable to Varnika?
            \item acceptable to the trader?
\end{enumerate}
\solution
	%\input{exemplar/10/13/3/40/main.tex}
 \item A student says that if you throw a die, it will show up 1 or not 1. Therefore, the probability of getting 1 and the probability of getting 'not 1' each is equal to $\frac{1}{2}$. Is this correct? Give reasons.\\
 \solution
        %\input{exemplar/10/13/2/9/main.tex}
   \item Four candidates A, B, C, D have ap-
plied for the assignment to coach a school cricket
team. If A is twice as likely to be selected as B, and
B and C are given about the same chance of being
selected, while C is twice as likely to be selected
as D, what are the probabilities that
\begin{enumerate}
\item C will be selected?
\item A will not be selected?
\end{enumerate}
	%\input{exemplar/11/16/3/9/main.tex}
 \item A bag contain 24 balls of which $x$ balls are red, $2x$ are white and $3x$ are blue. A ball is selected at random, What is the probability that it is
\begin{enumerate}[label=\alph*)]
\item not red ?
\item white ?
\end{enumerate}
%\input{exemplar/10/13/3/41/main.tex}
If the letters of the word ASSASSINATION are arranged at random. Find the Probability that
\begin{enumerate}[label=(\alph*)]
\item Four $S's$ come consecutively in the word
\item Two  $I's$ and two $N's$ come together
\item All $A's$ are not coming together
\item No two $A's$ are coming together
\end{enumerate}
%\input{exemplar/11/16/3/14/main.tex}
	\item One urn contains two black balls (labelled B1 and B2) and one white ball. A
	second urn contains one black ball and two white balls (labelled W1 and W2).
	Suppose the following experiment is performed. One of the two urns is chosen
	at random. Next a ball is randomly chosen from the urn. Then a second ball is
	chosen at random from the same urn without replacing the first ball.
	
	\begin{enumerate}
	\item What is the probability that two black balls are chosen?
	
	\item What is the probability that two balls of opposite colour are chosen?
	\end{enumerate}
	\solution
	%\input{exemplar/11/16/3/12/main1.tex}
\end{enumerate}

	\item 
The number lock of a suitcase has 4 wheels each labelled with ten digits i.e. from 0 to 9.The lock opens with a sequence of four digits with no repeats.What is the probability of a person getting the right sequence to open the suitcase.
\\
\solution
		%\begin{enumerate}[label=\thesection.\arabic*,ref=\thesection.\theenumi]
	\item One card is drawn from a well-shuffled deck of 52 cards. Find the probability of getting
\begin{enumerate}
\item A king of red colour 
\item A face card 
\item A red face card
\item The jack of hearts
\item A spade
\item The queen of diamonds

\end{enumerate}
\solution
		%\input{ncert/10/15/1/14/main.tex}
	\item Five cards—the ten, jack, queen, king and ace of diamonds, are well-shuffled with their face downwards. One card is then picked up at random.
\begin{enumerate}
\item
What is the probability that the card is the queen? 
\item
If the queen is drawn and put aside, what is the probability that the second card picked up is (a) an ace? (b) a queen?\\
\end{enumerate}
\solution
		%\input{ncert/10/15/1/15/defs.tex}
	\item A bag contains $5$ red balls and some blue balls. If the probability of drawing a blue ball is double that if a red ball, determine the number of blue balls in the bag. 
		\\
\solution
		%\input{ncert/10/15/2/3/defs.tex}
	\item A card is selected from a pack of 52 cards.
 \begin{enumerate}[label=(\alph*)] 
                 \item How many points are there in the sample space?
                 \item Calculate the probability that the card is an ace of spades.
                 \item Calculate the probability that the card is (i) an ace and (ii) black card.
 \end{enumerate}
\solution
		%\input{ncert/11/16/3/4/main.tex}
\item Four cards are drawn from a well-shuffled deck of 52 cards. What is the probability of obtaining 3 diamonds and one spade.
\\
\solution
		%\input{ncert/11/16/4/2/defs.tex}
\item In a certain lottery 10,000 tickets are sold and ten equal prizes are awarded. What is the probability of not getting a prize if you buy (a) one ticket (b) two tickets (c) 10 tickets ?	
\\
\solution
		%\input{ncert/11/16/4/4/defs.tex}
		%
\item 
Out of 100 students, two sections of 40 and 60 are formed. If you and your friend are among the 100 students, what is the probability that
\begin{enumerate}
\item you both enter the same section?
\item you both enter the different sections?
\end{enumerate}
\solution
		%\input{ncert/11/16/4/5/defs.tex}
	\item 
The number lock of a suitcase has 4 wheels each labelled with ten digits i.e. from 0 to 9.The lock opens with a sequence of four digits with no repeats.What is the probability of a person getting the right sequence to open the suitcase.
\\
\solution
		%\input{ncert/11/16/4/10/defs.tex}
		%
\item 
Two cards are drawn at random and without replacement from a pack of 52 playing cards. Find the probability that both the cards are black.
\\
\solution
		%\input{ncert/12/13/2/2/defs.tex}
		\item A box of oranges is inspected by examining three randomly selected oranges drawn without replacement. If all the three oranges are good, the box is approved for sale, otherwise, it is rejected. Find the probability that a box containing 15 oranges out of which 12 are good and 3 are bad ones will be approved for sale.
		\label{ncert/12/13/2/3/defs.tex}
		\item Two balls are drawn at random with replacement from a box containing 10 black and 8 red balls. Find the probability that
		\label{ncert/12/13/2/12}
\begin{enumerate}
\item both balls are red.
\item first ball is black and second is red.
\item one of them is black and other is red.
\end{enumerate}

\item In a hostel, 60\% of the students read Hindi newspaper, 40\% read English newspaper and 20\% read both Hindi and English newspapers. A student is selected at random.
		\label{ncert/12/13/2/15}
\begin{enumerate}
\item Find the probability that she reads neither Hindi nor English newspapers.
\item If she reads Hindi newspaper, find the probability that she reads English newspaper.
\item If she reads English newspaper, find the probability that she reads Hindi newspaper.\\
\end{enumerate}
\item The probability of obtaining an even prime number on each die, when a pair of dice is rolled is 
\begin{enumerate}
    \item $0$ 
    
    \item $\frac{1}{3}$ 
    
    \item $\frac{1}{12}$ 
    
    \item $\frac{1}{36}$ 
\end{enumerate}
\solution
		%\input{ncert/12/13/2/17/defs.tex}
	\item A bag contains 4 red and 4 black balls, another bag contains 2 red and 6 black balls. One of the two bags is selected at random and a ball is drawn from the bag which is found to be red. Find the probability that the ball is drawn from the first bag.
\\
\solution
		%\input{ncert/12/13/3/2/main.tex}
  \item
  Cards with numbers 2 to 101 are placed in a box. A card is selected at random.Find the probability that the card has
\begin{enumerate}[label=(\roman*)]
	\item an even number 
	\item a square number
\end{enumerate}
\solution
%\input{exemplar/10/13/3/32/main.tex}
\item
The king, queen and jack of clubs are removed from a deck of 52 playing cards and then well shuffled. Now one card is drawn at random from the remaining cards.  Determine the probability that the card is
\begin{enumerate}[label=(\roman*)]
\item a club
\item 10 of hearts
\end{enumerate}
\solution
%\input{exemplar/10/13/3/29/main.tex}
\item A team of medical students doing their internship have to assist during surgeries
at a city hospital. The probabilities of surgeries rated as very complex, complex,
routine, simple or very simple are respectively, 0.15, 0.20, 0.31, 0.26, .08. Find
the probabilities that a particular surgery will be rated
\begin{enumerate}
	\item complex or very complex;
	\item neither very complex nor very simple;
	\item routine or complex
	\item routine or simple
\end{enumerate}
\solution
%\input{exemplar/11/16/3/8(1)/main.tex}
\item A card is selected from a pack of 52 cards.
\begin{enumerate}[label=(\alph*)]
    \item How many points are there in the sample space?
    \item Calculate the probability that the card is an ace of spades.
    \item Calculate the probability that the card is (i) an ace and (ii) black card.
\end{enumerate}
\solution
%\input{exemplar/11/16/3/4/main2.tex}
\item The probability that a non leap year selected at random will contain 53 sundays.
\\
\solution
%\input{exemplar/10/13/1/19/main.tex}
\item One of the four persons John, Rita, Aslam or Gurpreet will be promoted next
month. Consequently the sample space consists of four elementary outcomes
S = {John promoted, Rita promoted, Aslam promoted, Gurpreet promoted}
You are told that the chances of John’s promotion is same as that of Gurpreet,
Rita’s chances of promotion are twice as likely as Johns. Aslam’s chances are
four times that of John.
\begin{enumerate}
	\item Determine
	\begin{enumerate}
		\item P (John promoted)
		\item P (Rita promoted)
		\item P (Aslam promoted)
		\item P (Gurpreet promoted)
	\end{enumerate}
	\item If A = {John promoted or Gurpreet promoted}, find P (A).
\end{enumerate}
\solution
%\input{exemplar/11/16/3/10/main.tex}
\item A card is drawn from a deck of 52 cards. Find the probability of getting a king or a heart or a red card.\\
\solution
%\input{exemplar/11/16/3/15/main.tex}
\item The probability that a student will pass his examination is 0.73, the probability of
the student getting a compartment is 0.13, and the probability that the student will
either pass or get compartment is 0.96. State True or False.\\
\solution
%\input{exemplar/11/16/3/31/main.tex}
\item A card is selected from a pack of 52 cards\\
\begin{enumerate}[label=(\alph*)]
\item How many points are there in the sample space?
\item Calculate the probability that the cards is an ace of spades.
\item Calculate the probability that the card is (i) an ace (ii)black card.\\
\end{enumerate}
%\input{ncert/11/16/3/4_1/Prob_4.tex}
\item In a non-leap year, the probability of having 53 tuesdays or 53 wednesdays is\\
\solution
%\input{exemplar/11/16/3/18/main.tex}
\item There are 1000 sealed envelopes in a box, 10 of them contain a cash prize of
Rs 100 each, 100 of them contain a cash prize of Rs 50 each and 200 of them
contain a cash prize of Rs 10 each and rest do not contain any cash prize. If they
are well shuffled and an envelope is picked up out, what is the probability that it
contains no cash prize?\\
\solution
%\input{exemplar/10/13/3/34/main.tex}
\item 
A die is thrown and a card is selected at random from a deck of 52 playing cards. The probability of getting an even number on the die and a spade card.\\
\solution
%\input{exemplar/12/13/3/78/main.tex}
\item
If 4-digit numbers greater than 5,000 are randomly formed from the digits 0, 1, 3, 5, and 7, what is the probability of forming a number divisible by 5 when:
\begin{enumerate}
    \item The digits are repeated?
    \item The repetition of digits is not allowed?
\end{enumerate}
\solution
%\input{ncert/11/16/4/9/main.tex}
\item Consider the probability space $\brak{\Omega, \mathcal{G}, P}$ where $\Omega = [0,2]$ and $\mathcal{G} = \cbrak{\phi, \Omega, [0,1], (1,2]}$. Let $X$ and $Y$ be two functions on $\Omega$ defined as
\begin{align*}
    X(\omega) = 
    \begin{cases}
        1 & \text{if }\omega \in [0, 1]\\
        2 & \text{if }\omega \in (1, 2]
    \end{cases}
\end{align*}
and
\begin{align*}
    Y(\omega) = 
    \begin{cases}
        2 & \text{if }\omega \in [0, 1.5]\\
        3 & \text{if }\omega \in (1.5, 2].
    \end{cases}
\end{align*}
Then which one of the following statements is true?
\begin{enumerate}
    \item [(A)] $X$ is a random variable with respect to $\mathcal{G}$, but $Y$ is not a random variable with respect to $\mathcal{G}$.
    \item [(B)] $Y$ is a random variable with respect to $\mathcal{G}$, but $X$ is not a random variable with respect to $\mathcal{G}$.
    \item [(C)] Neither $X$ nor $Y$ is a random variable with respect to $\mathcal{G}$.
    \item [(D)] Both $X$ and $Y$ are random variables with respect to $\mathcal{G}$.
\end{enumerate} \hfill (GATE ST 2023)\\
\solution
%\input{gate/ST/2023/14/main.tex}
	\item  A die is loaded in such a way that each odd number is twice as likely to occur as
each even number. Find $P(G)$, where $G$ is the event that a number greater than
3 occurs on a single roll of the die.
\\
\solution
		%\input{exemplar/11/16/3/5/main.tex}
	\item All the jacks, queens and kings are removed from a deck of 52 playing cards. The remaining cards are well shuffled and then one card is drawn at random. Giving ace a value 1 similar value for other cards, find the probability that the card has a value 
		\begin{enumerate}
			\item 7
			\item greater than 7
			\item less than 7
		\end{enumerate}
		%\input{exemplar/10/13/3/30/main.tex}
  \item A Lot consists of 48 mobile phones of which 42 are good, 3 have only minor defects and 3 have major defects.Varnika will buy a phone if it is good but the trader will only buy a mobile if it has no major defects. One phone is selected at random from the lot. What is the probability that it is
\begin{enumerate}
	\item acceptable to Varnika?
            \item acceptable to the trader?
\end{enumerate}
\solution
	%\input{exemplar/10/13/3/40/main.tex}
 \item A student says that if you throw a die, it will show up 1 or not 1. Therefore, the probability of getting 1 and the probability of getting 'not 1' each is equal to $\frac{1}{2}$. Is this correct? Give reasons.\\
 \solution
        %\input{exemplar/10/13/2/9/main.tex}
   \item Four candidates A, B, C, D have ap-
plied for the assignment to coach a school cricket
team. If A is twice as likely to be selected as B, and
B and C are given about the same chance of being
selected, while C is twice as likely to be selected
as D, what are the probabilities that
\begin{enumerate}
\item C will be selected?
\item A will not be selected?
\end{enumerate}
	%\input{exemplar/11/16/3/9/main.tex}
 \item A bag contain 24 balls of which $x$ balls are red, $2x$ are white and $3x$ are blue. A ball is selected at random, What is the probability that it is
\begin{enumerate}[label=\alph*)]
\item not red ?
\item white ?
\end{enumerate}
%\input{exemplar/10/13/3/41/main.tex}
If the letters of the word ASSASSINATION are arranged at random. Find the Probability that
\begin{enumerate}[label=(\alph*)]
\item Four $S's$ come consecutively in the word
\item Two  $I's$ and two $N's$ come together
\item All $A's$ are not coming together
\item No two $A's$ are coming together
\end{enumerate}
%\input{exemplar/11/16/3/14/main.tex}
	\item One urn contains two black balls (labelled B1 and B2) and one white ball. A
	second urn contains one black ball and two white balls (labelled W1 and W2).
	Suppose the following experiment is performed. One of the two urns is chosen
	at random. Next a ball is randomly chosen from the urn. Then a second ball is
	chosen at random from the same urn without replacing the first ball.
	
	\begin{enumerate}
	\item What is the probability that two black balls are chosen?
	
	\item What is the probability that two balls of opposite colour are chosen?
	\end{enumerate}
	\solution
	%\input{exemplar/11/16/3/12/main1.tex}
\end{enumerate}

		%
\item 
Two cards are drawn at random and without replacement from a pack of 52 playing cards. Find the probability that both the cards are black.
\\
\solution
		%\begin{enumerate}[label=\thesection.\arabic*,ref=\thesection.\theenumi]
	\item One card is drawn from a well-shuffled deck of 52 cards. Find the probability of getting
\begin{enumerate}
\item A king of red colour 
\item A face card 
\item A red face card
\item The jack of hearts
\item A spade
\item The queen of diamonds

\end{enumerate}
\solution
		%\input{ncert/10/15/1/14/main.tex}
	\item Five cards—the ten, jack, queen, king and ace of diamonds, are well-shuffled with their face downwards. One card is then picked up at random.
\begin{enumerate}
\item
What is the probability that the card is the queen? 
\item
If the queen is drawn and put aside, what is the probability that the second card picked up is (a) an ace? (b) a queen?\\
\end{enumerate}
\solution
		%\input{ncert/10/15/1/15/defs.tex}
	\item A bag contains $5$ red balls and some blue balls. If the probability of drawing a blue ball is double that if a red ball, determine the number of blue balls in the bag. 
		\\
\solution
		%\input{ncert/10/15/2/3/defs.tex}
	\item A card is selected from a pack of 52 cards.
 \begin{enumerate}[label=(\alph*)] 
                 \item How many points are there in the sample space?
                 \item Calculate the probability that the card is an ace of spades.
                 \item Calculate the probability that the card is (i) an ace and (ii) black card.
 \end{enumerate}
\solution
		%\input{ncert/11/16/3/4/main.tex}
\item Four cards are drawn from a well-shuffled deck of 52 cards. What is the probability of obtaining 3 diamonds and one spade.
\\
\solution
		%\input{ncert/11/16/4/2/defs.tex}
\item In a certain lottery 10,000 tickets are sold and ten equal prizes are awarded. What is the probability of not getting a prize if you buy (a) one ticket (b) two tickets (c) 10 tickets ?	
\\
\solution
		%\input{ncert/11/16/4/4/defs.tex}
		%
\item 
Out of 100 students, two sections of 40 and 60 are formed. If you and your friend are among the 100 students, what is the probability that
\begin{enumerate}
\item you both enter the same section?
\item you both enter the different sections?
\end{enumerate}
\solution
		%\input{ncert/11/16/4/5/defs.tex}
	\item 
The number lock of a suitcase has 4 wheels each labelled with ten digits i.e. from 0 to 9.The lock opens with a sequence of four digits with no repeats.What is the probability of a person getting the right sequence to open the suitcase.
\\
\solution
		%\input{ncert/11/16/4/10/defs.tex}
		%
\item 
Two cards are drawn at random and without replacement from a pack of 52 playing cards. Find the probability that both the cards are black.
\\
\solution
		%\input{ncert/12/13/2/2/defs.tex}
		\item A box of oranges is inspected by examining three randomly selected oranges drawn without replacement. If all the three oranges are good, the box is approved for sale, otherwise, it is rejected. Find the probability that a box containing 15 oranges out of which 12 are good and 3 are bad ones will be approved for sale.
		\label{ncert/12/13/2/3/defs.tex}
		\item Two balls are drawn at random with replacement from a box containing 10 black and 8 red balls. Find the probability that
		\label{ncert/12/13/2/12}
\begin{enumerate}
\item both balls are red.
\item first ball is black and second is red.
\item one of them is black and other is red.
\end{enumerate}

\item In a hostel, 60\% of the students read Hindi newspaper, 40\% read English newspaper and 20\% read both Hindi and English newspapers. A student is selected at random.
		\label{ncert/12/13/2/15}
\begin{enumerate}
\item Find the probability that she reads neither Hindi nor English newspapers.
\item If she reads Hindi newspaper, find the probability that she reads English newspaper.
\item If she reads English newspaper, find the probability that she reads Hindi newspaper.\\
\end{enumerate}
\item The probability of obtaining an even prime number on each die, when a pair of dice is rolled is 
\begin{enumerate}
    \item $0$ 
    
    \item $\frac{1}{3}$ 
    
    \item $\frac{1}{12}$ 
    
    \item $\frac{1}{36}$ 
\end{enumerate}
\solution
		%\input{ncert/12/13/2/17/defs.tex}
	\item A bag contains 4 red and 4 black balls, another bag contains 2 red and 6 black balls. One of the two bags is selected at random and a ball is drawn from the bag which is found to be red. Find the probability that the ball is drawn from the first bag.
\\
\solution
		%\input{ncert/12/13/3/2/main.tex}
  \item
  Cards with numbers 2 to 101 are placed in a box. A card is selected at random.Find the probability that the card has
\begin{enumerate}[label=(\roman*)]
	\item an even number 
	\item a square number
\end{enumerate}
\solution
%\input{exemplar/10/13/3/32/main.tex}
\item
The king, queen and jack of clubs are removed from a deck of 52 playing cards and then well shuffled. Now one card is drawn at random from the remaining cards.  Determine the probability that the card is
\begin{enumerate}[label=(\roman*)]
\item a club
\item 10 of hearts
\end{enumerate}
\solution
%\input{exemplar/10/13/3/29/main.tex}
\item A team of medical students doing their internship have to assist during surgeries
at a city hospital. The probabilities of surgeries rated as very complex, complex,
routine, simple or very simple are respectively, 0.15, 0.20, 0.31, 0.26, .08. Find
the probabilities that a particular surgery will be rated
\begin{enumerate}
	\item complex or very complex;
	\item neither very complex nor very simple;
	\item routine or complex
	\item routine or simple
\end{enumerate}
\solution
%\input{exemplar/11/16/3/8(1)/main.tex}
\item A card is selected from a pack of 52 cards.
\begin{enumerate}[label=(\alph*)]
    \item How many points are there in the sample space?
    \item Calculate the probability that the card is an ace of spades.
    \item Calculate the probability that the card is (i) an ace and (ii) black card.
\end{enumerate}
\solution
%\input{exemplar/11/16/3/4/main2.tex}
\item The probability that a non leap year selected at random will contain 53 sundays.
\\
\solution
%\input{exemplar/10/13/1/19/main.tex}
\item One of the four persons John, Rita, Aslam or Gurpreet will be promoted next
month. Consequently the sample space consists of four elementary outcomes
S = {John promoted, Rita promoted, Aslam promoted, Gurpreet promoted}
You are told that the chances of John’s promotion is same as that of Gurpreet,
Rita’s chances of promotion are twice as likely as Johns. Aslam’s chances are
four times that of John.
\begin{enumerate}
	\item Determine
	\begin{enumerate}
		\item P (John promoted)
		\item P (Rita promoted)
		\item P (Aslam promoted)
		\item P (Gurpreet promoted)
	\end{enumerate}
	\item If A = {John promoted or Gurpreet promoted}, find P (A).
\end{enumerate}
\solution
%\input{exemplar/11/16/3/10/main.tex}
\item A card is drawn from a deck of 52 cards. Find the probability of getting a king or a heart or a red card.\\
\solution
%\input{exemplar/11/16/3/15/main.tex}
\item The probability that a student will pass his examination is 0.73, the probability of
the student getting a compartment is 0.13, and the probability that the student will
either pass or get compartment is 0.96. State True or False.\\
\solution
%\input{exemplar/11/16/3/31/main.tex}
\item A card is selected from a pack of 52 cards\\
\begin{enumerate}[label=(\alph*)]
\item How many points are there in the sample space?
\item Calculate the probability that the cards is an ace of spades.
\item Calculate the probability that the card is (i) an ace (ii)black card.\\
\end{enumerate}
%\input{ncert/11/16/3/4_1/Prob_4.tex}
\item In a non-leap year, the probability of having 53 tuesdays or 53 wednesdays is\\
\solution
%\input{exemplar/11/16/3/18/main.tex}
\item There are 1000 sealed envelopes in a box, 10 of them contain a cash prize of
Rs 100 each, 100 of them contain a cash prize of Rs 50 each and 200 of them
contain a cash prize of Rs 10 each and rest do not contain any cash prize. If they
are well shuffled and an envelope is picked up out, what is the probability that it
contains no cash prize?\\
\solution
%\input{exemplar/10/13/3/34/main.tex}
\item 
A die is thrown and a card is selected at random from a deck of 52 playing cards. The probability of getting an even number on the die and a spade card.\\
\solution
%\input{exemplar/12/13/3/78/main.tex}
\item
If 4-digit numbers greater than 5,000 are randomly formed from the digits 0, 1, 3, 5, and 7, what is the probability of forming a number divisible by 5 when:
\begin{enumerate}
    \item The digits are repeated?
    \item The repetition of digits is not allowed?
\end{enumerate}
\solution
%\input{ncert/11/16/4/9/main.tex}
\item Consider the probability space $\brak{\Omega, \mathcal{G}, P}$ where $\Omega = [0,2]$ and $\mathcal{G} = \cbrak{\phi, \Omega, [0,1], (1,2]}$. Let $X$ and $Y$ be two functions on $\Omega$ defined as
\begin{align*}
    X(\omega) = 
    \begin{cases}
        1 & \text{if }\omega \in [0, 1]\\
        2 & \text{if }\omega \in (1, 2]
    \end{cases}
\end{align*}
and
\begin{align*}
    Y(\omega) = 
    \begin{cases}
        2 & \text{if }\omega \in [0, 1.5]\\
        3 & \text{if }\omega \in (1.5, 2].
    \end{cases}
\end{align*}
Then which one of the following statements is true?
\begin{enumerate}
    \item [(A)] $X$ is a random variable with respect to $\mathcal{G}$, but $Y$ is not a random variable with respect to $\mathcal{G}$.
    \item [(B)] $Y$ is a random variable with respect to $\mathcal{G}$, but $X$ is not a random variable with respect to $\mathcal{G}$.
    \item [(C)] Neither $X$ nor $Y$ is a random variable with respect to $\mathcal{G}$.
    \item [(D)] Both $X$ and $Y$ are random variables with respect to $\mathcal{G}$.
\end{enumerate} \hfill (GATE ST 2023)\\
\solution
%\input{gate/ST/2023/14/main.tex}
	\item  A die is loaded in such a way that each odd number is twice as likely to occur as
each even number. Find $P(G)$, where $G$ is the event that a number greater than
3 occurs on a single roll of the die.
\\
\solution
		%\input{exemplar/11/16/3/5/main.tex}
	\item All the jacks, queens and kings are removed from a deck of 52 playing cards. The remaining cards are well shuffled and then one card is drawn at random. Giving ace a value 1 similar value for other cards, find the probability that the card has a value 
		\begin{enumerate}
			\item 7
			\item greater than 7
			\item less than 7
		\end{enumerate}
		%\input{exemplar/10/13/3/30/main.tex}
  \item A Lot consists of 48 mobile phones of which 42 are good, 3 have only minor defects and 3 have major defects.Varnika will buy a phone if it is good but the trader will only buy a mobile if it has no major defects. One phone is selected at random from the lot. What is the probability that it is
\begin{enumerate}
	\item acceptable to Varnika?
            \item acceptable to the trader?
\end{enumerate}
\solution
	%\input{exemplar/10/13/3/40/main.tex}
 \item A student says that if you throw a die, it will show up 1 or not 1. Therefore, the probability of getting 1 and the probability of getting 'not 1' each is equal to $\frac{1}{2}$. Is this correct? Give reasons.\\
 \solution
        %\input{exemplar/10/13/2/9/main.tex}
   \item Four candidates A, B, C, D have ap-
plied for the assignment to coach a school cricket
team. If A is twice as likely to be selected as B, and
B and C are given about the same chance of being
selected, while C is twice as likely to be selected
as D, what are the probabilities that
\begin{enumerate}
\item C will be selected?
\item A will not be selected?
\end{enumerate}
	%\input{exemplar/11/16/3/9/main.tex}
 \item A bag contain 24 balls of which $x$ balls are red, $2x$ are white and $3x$ are blue. A ball is selected at random, What is the probability that it is
\begin{enumerate}[label=\alph*)]
\item not red ?
\item white ?
\end{enumerate}
%\input{exemplar/10/13/3/41/main.tex}
If the letters of the word ASSASSINATION are arranged at random. Find the Probability that
\begin{enumerate}[label=(\alph*)]
\item Four $S's$ come consecutively in the word
\item Two  $I's$ and two $N's$ come together
\item All $A's$ are not coming together
\item No two $A's$ are coming together
\end{enumerate}
%\input{exemplar/11/16/3/14/main.tex}
	\item One urn contains two black balls (labelled B1 and B2) and one white ball. A
	second urn contains one black ball and two white balls (labelled W1 and W2).
	Suppose the following experiment is performed. One of the two urns is chosen
	at random. Next a ball is randomly chosen from the urn. Then a second ball is
	chosen at random from the same urn without replacing the first ball.
	
	\begin{enumerate}
	\item What is the probability that two black balls are chosen?
	
	\item What is the probability that two balls of opposite colour are chosen?
	\end{enumerate}
	\solution
	%\input{exemplar/11/16/3/12/main1.tex}
\end{enumerate}

		\item A box of oranges is inspected by examining three randomly selected oranges drawn without replacement. If all the three oranges are good, the box is approved for sale, otherwise, it is rejected. Find the probability that a box containing 15 oranges out of which 12 are good and 3 are bad ones will be approved for sale.
		\label{ncert/12/13/2/3/defs.tex}
		\item Two balls are drawn at random with replacement from a box containing 10 black and 8 red balls. Find the probability that
		\label{ncert/12/13/2/12}
\begin{enumerate}
\item both balls are red.
\item first ball is black and second is red.
\item one of them is black and other is red.
\end{enumerate}

\item In a hostel, 60\% of the students read Hindi newspaper, 40\% read English newspaper and 20\% read both Hindi and English newspapers. A student is selected at random.
		\label{ncert/12/13/2/15}
\begin{enumerate}
\item Find the probability that she reads neither Hindi nor English newspapers.
\item If she reads Hindi newspaper, find the probability that she reads English newspaper.
\item If she reads English newspaper, find the probability that she reads Hindi newspaper.\\
\end{enumerate}
\item The probability of obtaining an even prime number on each die, when a pair of dice is rolled is 
\begin{enumerate}
    \item $0$ 
    
    \item $\frac{1}{3}$ 
    
    \item $\frac{1}{12}$ 
    
    \item $\frac{1}{36}$ 
\end{enumerate}
\solution
		%\begin{enumerate}[label=\thesection.\arabic*,ref=\thesection.\theenumi]
	\item One card is drawn from a well-shuffled deck of 52 cards. Find the probability of getting
\begin{enumerate}
\item A king of red colour 
\item A face card 
\item A red face card
\item The jack of hearts
\item A spade
\item The queen of diamonds

\end{enumerate}
\solution
		%\input{ncert/10/15/1/14/main.tex}
	\item Five cards—the ten, jack, queen, king and ace of diamonds, are well-shuffled with their face downwards. One card is then picked up at random.
\begin{enumerate}
\item
What is the probability that the card is the queen? 
\item
If the queen is drawn and put aside, what is the probability that the second card picked up is (a) an ace? (b) a queen?\\
\end{enumerate}
\solution
		%\input{ncert/10/15/1/15/defs.tex}
	\item A bag contains $5$ red balls and some blue balls. If the probability of drawing a blue ball is double that if a red ball, determine the number of blue balls in the bag. 
		\\
\solution
		%\input{ncert/10/15/2/3/defs.tex}
	\item A card is selected from a pack of 52 cards.
 \begin{enumerate}[label=(\alph*)] 
                 \item How many points are there in the sample space?
                 \item Calculate the probability that the card is an ace of spades.
                 \item Calculate the probability that the card is (i) an ace and (ii) black card.
 \end{enumerate}
\solution
		%\input{ncert/11/16/3/4/main.tex}
\item Four cards are drawn from a well-shuffled deck of 52 cards. What is the probability of obtaining 3 diamonds and one spade.
\\
\solution
		%\input{ncert/11/16/4/2/defs.tex}
\item In a certain lottery 10,000 tickets are sold and ten equal prizes are awarded. What is the probability of not getting a prize if you buy (a) one ticket (b) two tickets (c) 10 tickets ?	
\\
\solution
		%\input{ncert/11/16/4/4/defs.tex}
		%
\item 
Out of 100 students, two sections of 40 and 60 are formed. If you and your friend are among the 100 students, what is the probability that
\begin{enumerate}
\item you both enter the same section?
\item you both enter the different sections?
\end{enumerate}
\solution
		%\input{ncert/11/16/4/5/defs.tex}
	\item 
The number lock of a suitcase has 4 wheels each labelled with ten digits i.e. from 0 to 9.The lock opens with a sequence of four digits with no repeats.What is the probability of a person getting the right sequence to open the suitcase.
\\
\solution
		%\input{ncert/11/16/4/10/defs.tex}
		%
\item 
Two cards are drawn at random and without replacement from a pack of 52 playing cards. Find the probability that both the cards are black.
\\
\solution
		%\input{ncert/12/13/2/2/defs.tex}
		\item A box of oranges is inspected by examining three randomly selected oranges drawn without replacement. If all the three oranges are good, the box is approved for sale, otherwise, it is rejected. Find the probability that a box containing 15 oranges out of which 12 are good and 3 are bad ones will be approved for sale.
		\label{ncert/12/13/2/3/defs.tex}
		\item Two balls are drawn at random with replacement from a box containing 10 black and 8 red balls. Find the probability that
		\label{ncert/12/13/2/12}
\begin{enumerate}
\item both balls are red.
\item first ball is black and second is red.
\item one of them is black and other is red.
\end{enumerate}

\item In a hostel, 60\% of the students read Hindi newspaper, 40\% read English newspaper and 20\% read both Hindi and English newspapers. A student is selected at random.
		\label{ncert/12/13/2/15}
\begin{enumerate}
\item Find the probability that she reads neither Hindi nor English newspapers.
\item If she reads Hindi newspaper, find the probability that she reads English newspaper.
\item If she reads English newspaper, find the probability that she reads Hindi newspaper.\\
\end{enumerate}
\item The probability of obtaining an even prime number on each die, when a pair of dice is rolled is 
\begin{enumerate}
    \item $0$ 
    
    \item $\frac{1}{3}$ 
    
    \item $\frac{1}{12}$ 
    
    \item $\frac{1}{36}$ 
\end{enumerate}
\solution
		%\input{ncert/12/13/2/17/defs.tex}
	\item A bag contains 4 red and 4 black balls, another bag contains 2 red and 6 black balls. One of the two bags is selected at random and a ball is drawn from the bag which is found to be red. Find the probability that the ball is drawn from the first bag.
\\
\solution
		%\input{ncert/12/13/3/2/main.tex}
  \item
  Cards with numbers 2 to 101 are placed in a box. A card is selected at random.Find the probability that the card has
\begin{enumerate}[label=(\roman*)]
	\item an even number 
	\item a square number
\end{enumerate}
\solution
%\input{exemplar/10/13/3/32/main.tex}
\item
The king, queen and jack of clubs are removed from a deck of 52 playing cards and then well shuffled. Now one card is drawn at random from the remaining cards.  Determine the probability that the card is
\begin{enumerate}[label=(\roman*)]
\item a club
\item 10 of hearts
\end{enumerate}
\solution
%\input{exemplar/10/13/3/29/main.tex}
\item A team of medical students doing their internship have to assist during surgeries
at a city hospital. The probabilities of surgeries rated as very complex, complex,
routine, simple or very simple are respectively, 0.15, 0.20, 0.31, 0.26, .08. Find
the probabilities that a particular surgery will be rated
\begin{enumerate}
	\item complex or very complex;
	\item neither very complex nor very simple;
	\item routine or complex
	\item routine or simple
\end{enumerate}
\solution
%\input{exemplar/11/16/3/8(1)/main.tex}
\item A card is selected from a pack of 52 cards.
\begin{enumerate}[label=(\alph*)]
    \item How many points are there in the sample space?
    \item Calculate the probability that the card is an ace of spades.
    \item Calculate the probability that the card is (i) an ace and (ii) black card.
\end{enumerate}
\solution
%\input{exemplar/11/16/3/4/main2.tex}
\item The probability that a non leap year selected at random will contain 53 sundays.
\\
\solution
%\input{exemplar/10/13/1/19/main.tex}
\item One of the four persons John, Rita, Aslam or Gurpreet will be promoted next
month. Consequently the sample space consists of four elementary outcomes
S = {John promoted, Rita promoted, Aslam promoted, Gurpreet promoted}
You are told that the chances of John’s promotion is same as that of Gurpreet,
Rita’s chances of promotion are twice as likely as Johns. Aslam’s chances are
four times that of John.
\begin{enumerate}
	\item Determine
	\begin{enumerate}
		\item P (John promoted)
		\item P (Rita promoted)
		\item P (Aslam promoted)
		\item P (Gurpreet promoted)
	\end{enumerate}
	\item If A = {John promoted or Gurpreet promoted}, find P (A).
\end{enumerate}
\solution
%\input{exemplar/11/16/3/10/main.tex}
\item A card is drawn from a deck of 52 cards. Find the probability of getting a king or a heart or a red card.\\
\solution
%\input{exemplar/11/16/3/15/main.tex}
\item The probability that a student will pass his examination is 0.73, the probability of
the student getting a compartment is 0.13, and the probability that the student will
either pass or get compartment is 0.96. State True or False.\\
\solution
%\input{exemplar/11/16/3/31/main.tex}
\item A card is selected from a pack of 52 cards\\
\begin{enumerate}[label=(\alph*)]
\item How many points are there in the sample space?
\item Calculate the probability that the cards is an ace of spades.
\item Calculate the probability that the card is (i) an ace (ii)black card.\\
\end{enumerate}
%\input{ncert/11/16/3/4_1/Prob_4.tex}
\item In a non-leap year, the probability of having 53 tuesdays or 53 wednesdays is\\
\solution
%\input{exemplar/11/16/3/18/main.tex}
\item There are 1000 sealed envelopes in a box, 10 of them contain a cash prize of
Rs 100 each, 100 of them contain a cash prize of Rs 50 each and 200 of them
contain a cash prize of Rs 10 each and rest do not contain any cash prize. If they
are well shuffled and an envelope is picked up out, what is the probability that it
contains no cash prize?\\
\solution
%\input{exemplar/10/13/3/34/main.tex}
\item 
A die is thrown and a card is selected at random from a deck of 52 playing cards. The probability of getting an even number on the die and a spade card.\\
\solution
%\input{exemplar/12/13/3/78/main.tex}
\item
If 4-digit numbers greater than 5,000 are randomly formed from the digits 0, 1, 3, 5, and 7, what is the probability of forming a number divisible by 5 when:
\begin{enumerate}
    \item The digits are repeated?
    \item The repetition of digits is not allowed?
\end{enumerate}
\solution
%\input{ncert/11/16/4/9/main.tex}
\item Consider the probability space $\brak{\Omega, \mathcal{G}, P}$ where $\Omega = [0,2]$ and $\mathcal{G} = \cbrak{\phi, \Omega, [0,1], (1,2]}$. Let $X$ and $Y$ be two functions on $\Omega$ defined as
\begin{align*}
    X(\omega) = 
    \begin{cases}
        1 & \text{if }\omega \in [0, 1]\\
        2 & \text{if }\omega \in (1, 2]
    \end{cases}
\end{align*}
and
\begin{align*}
    Y(\omega) = 
    \begin{cases}
        2 & \text{if }\omega \in [0, 1.5]\\
        3 & \text{if }\omega \in (1.5, 2].
    \end{cases}
\end{align*}
Then which one of the following statements is true?
\begin{enumerate}
    \item [(A)] $X$ is a random variable with respect to $\mathcal{G}$, but $Y$ is not a random variable with respect to $\mathcal{G}$.
    \item [(B)] $Y$ is a random variable with respect to $\mathcal{G}$, but $X$ is not a random variable with respect to $\mathcal{G}$.
    \item [(C)] Neither $X$ nor $Y$ is a random variable with respect to $\mathcal{G}$.
    \item [(D)] Both $X$ and $Y$ are random variables with respect to $\mathcal{G}$.
\end{enumerate} \hfill (GATE ST 2023)\\
\solution
%\input{gate/ST/2023/14/main.tex}
	\item  A die is loaded in such a way that each odd number is twice as likely to occur as
each even number. Find $P(G)$, where $G$ is the event that a number greater than
3 occurs on a single roll of the die.
\\
\solution
		%\input{exemplar/11/16/3/5/main.tex}
	\item All the jacks, queens and kings are removed from a deck of 52 playing cards. The remaining cards are well shuffled and then one card is drawn at random. Giving ace a value 1 similar value for other cards, find the probability that the card has a value 
		\begin{enumerate}
			\item 7
			\item greater than 7
			\item less than 7
		\end{enumerate}
		%\input{exemplar/10/13/3/30/main.tex}
  \item A Lot consists of 48 mobile phones of which 42 are good, 3 have only minor defects and 3 have major defects.Varnika will buy a phone if it is good but the trader will only buy a mobile if it has no major defects. One phone is selected at random from the lot. What is the probability that it is
\begin{enumerate}
	\item acceptable to Varnika?
            \item acceptable to the trader?
\end{enumerate}
\solution
	%\input{exemplar/10/13/3/40/main.tex}
 \item A student says that if you throw a die, it will show up 1 or not 1. Therefore, the probability of getting 1 and the probability of getting 'not 1' each is equal to $\frac{1}{2}$. Is this correct? Give reasons.\\
 \solution
        %\input{exemplar/10/13/2/9/main.tex}
   \item Four candidates A, B, C, D have ap-
plied for the assignment to coach a school cricket
team. If A is twice as likely to be selected as B, and
B and C are given about the same chance of being
selected, while C is twice as likely to be selected
as D, what are the probabilities that
\begin{enumerate}
\item C will be selected?
\item A will not be selected?
\end{enumerate}
	%\input{exemplar/11/16/3/9/main.tex}
 \item A bag contain 24 balls of which $x$ balls are red, $2x$ are white and $3x$ are blue. A ball is selected at random, What is the probability that it is
\begin{enumerate}[label=\alph*)]
\item not red ?
\item white ?
\end{enumerate}
%\input{exemplar/10/13/3/41/main.tex}
If the letters of the word ASSASSINATION are arranged at random. Find the Probability that
\begin{enumerate}[label=(\alph*)]
\item Four $S's$ come consecutively in the word
\item Two  $I's$ and two $N's$ come together
\item All $A's$ are not coming together
\item No two $A's$ are coming together
\end{enumerate}
%\input{exemplar/11/16/3/14/main.tex}
	\item One urn contains two black balls (labelled B1 and B2) and one white ball. A
	second urn contains one black ball and two white balls (labelled W1 and W2).
	Suppose the following experiment is performed. One of the two urns is chosen
	at random. Next a ball is randomly chosen from the urn. Then a second ball is
	chosen at random from the same urn without replacing the first ball.
	
	\begin{enumerate}
	\item What is the probability that two black balls are chosen?
	
	\item What is the probability that two balls of opposite colour are chosen?
	\end{enumerate}
	\solution
	%\input{exemplar/11/16/3/12/main1.tex}
\end{enumerate}

	\item A bag contains 4 red and 4 black balls, another bag contains 2 red and 6 black balls. One of the two bags is selected at random and a ball is drawn from the bag which is found to be red. Find the probability that the ball is drawn from the first bag.
\\
\solution
		%\begin{table}[H]
	\centering
\begin{tabular}{|c|c|c|}
\hline
Random variable &Value &Definition\\ \hline
\multirow{3}{*}{X} &0 &Slips of Rs 1\\
&1 &Slips of Rs 5\\
&2 &Slips of Rs 13\\ \hline
\multirow{2}{*}{Y} &0 &Box A\\
&1 &Box B\\\hline
\end{tabular}
\caption{}
\label{tab:Distribution}
\end{table}
See \tabref{tab:Distribution}.
\begin{align}
p_{Y}\brak{k}= \begin{cases} 
      \frac{1}{3} & {k=0} \\
      \frac{2}{3 }& {k=1} 
   \end{cases}
   \\
p_{Y|X}\brak{0|0} = \frac{19}{25}\, 
p_{Y|X}\brak{0|1} = \frac{6}{25}\,
p_{Y|X}\brak{1|0} = \frac{45}{50}\,
p_{Y|X}\brak{1|2} = \frac{5}{50}
\end{align}
The desired probability is the probability that a slip drawn at random is marked other than Rs 1,
\begin{align}
&=1-p_X\brak{0}\\
&= p_X(1) + p_X(2)
\end{align}
Using Bayes theorem,
\begin{align}
&= p_Y\brak{0} \times \pr{Y=0 | X=1} + p_Y\brak{1} \times \pr{Y=1|X=2}\\
&=\frac{1}{3} \times \frac{6}{25} + \frac{2}{3} \times \frac{5}{50}\\
&=\frac{11}{75}
\end{align}

\newpage

%\tableofcontents

\bigskip

\renewcommand{\thefigure}{\theenumi}
\renewcommand{\thetable}{\theenumi}
%\renewcommand{\theequation}{\theenumi}

%\begin{abstract}
%%\boldmath
%In this letter, an algorithm for evaluating the exact analytical bit error rate  (BER)  for the piecewise linear (PL) combiner for  multiple relays is presented. Previous results were available only for upto three relays. The algorithm is unique in the sense that  the actual mathematical expressions, that are prohibitively large, need not be explicitly obtained. The diversity gain due to multiple relays is shown through plots of the analytical BER, well supported by simulations. 
%
%\end{abstract}
% IEEEtran.cls defaults to using nonbold math in the Abstract.
% This preserves the distinction between vectors and scalars. However,
% if the journal you are submitting to favors bold math in the abstract,
% then you can use LaTeX's standard command \boldmath at the very start
% of the abstract to achieve this. Many IEEE journals frown on math
% in the abstract anyway.

% Note that keywords are not normally used for peerreview papers.
%\begin{IEEEkeywords}
%Cooperative diversity, decode and forward, piecewise linear
%\end{IEEEkeywords}



% For peer review papers, you can put extra information on the cover
% page as needed:
% \ifCLASSOPTIONpeerreview
% \begin{center} \bfseries EDICS Category: 3-BBND \end{center}
% \fi
%
% For peerreview papers, this IEEEtran command inserts a page break and
% creates the second title. It will be ignored for other modes.
%\IEEEpeerreviewmaketitle




  \item
  Cards with numbers 2 to 101 are placed in a box. A card is selected at random.Find the probability that the card has
\begin{enumerate}[label=(\roman*)]
	\item an even number 
	\item a square number
\end{enumerate}
\solution
%\begin{table}[H]
	\centering
\begin{tabular}{|c|c|c|}
\hline
Random variable &Value &Definition\\ \hline
\multirow{3}{*}{X} &0 &Slips of Rs 1\\
&1 &Slips of Rs 5\\
&2 &Slips of Rs 13\\ \hline
\multirow{2}{*}{Y} &0 &Box A\\
&1 &Box B\\\hline
\end{tabular}
\caption{}
\label{tab:Distribution}
\end{table}
See \tabref{tab:Distribution}.
\begin{align}
p_{Y}\brak{k}= \begin{cases} 
      \frac{1}{3} & {k=0} \\
      \frac{2}{3 }& {k=1} 
   \end{cases}
   \\
p_{Y|X}\brak{0|0} = \frac{19}{25}\, 
p_{Y|X}\brak{0|1} = \frac{6}{25}\,
p_{Y|X}\brak{1|0} = \frac{45}{50}\,
p_{Y|X}\brak{1|2} = \frac{5}{50}
\end{align}
The desired probability is the probability that a slip drawn at random is marked other than Rs 1,
\begin{align}
&=1-p_X\brak{0}\\
&= p_X(1) + p_X(2)
\end{align}
Using Bayes theorem,
\begin{align}
&= p_Y\brak{0} \times \pr{Y=0 | X=1} + p_Y\brak{1} \times \pr{Y=1|X=2}\\
&=\frac{1}{3} \times \frac{6}{25} + \frac{2}{3} \times \frac{5}{50}\\
&=\frac{11}{75}
\end{align}

\newpage

%\tableofcontents

\bigskip

\renewcommand{\thefigure}{\theenumi}
\renewcommand{\thetable}{\theenumi}
%\renewcommand{\theequation}{\theenumi}

%\begin{abstract}
%%\boldmath
%In this letter, an algorithm for evaluating the exact analytical bit error rate  (BER)  for the piecewise linear (PL) combiner for  multiple relays is presented. Previous results were available only for upto three relays. The algorithm is unique in the sense that  the actual mathematical expressions, that are prohibitively large, need not be explicitly obtained. The diversity gain due to multiple relays is shown through plots of the analytical BER, well supported by simulations. 
%
%\end{abstract}
% IEEEtran.cls defaults to using nonbold math in the Abstract.
% This preserves the distinction between vectors and scalars. However,
% if the journal you are submitting to favors bold math in the abstract,
% then you can use LaTeX's standard command \boldmath at the very start
% of the abstract to achieve this. Many IEEE journals frown on math
% in the abstract anyway.

% Note that keywords are not normally used for peerreview papers.
%\begin{IEEEkeywords}
%Cooperative diversity, decode and forward, piecewise linear
%\end{IEEEkeywords}



% For peer review papers, you can put extra information on the cover
% page as needed:
% \ifCLASSOPTIONpeerreview
% \begin{center} \bfseries EDICS Category: 3-BBND \end{center}
% \fi
%
% For peerreview papers, this IEEEtran command inserts a page break and
% creates the second title. It will be ignored for other modes.
%\IEEEpeerreviewmaketitle




\item
The king, queen and jack of clubs are removed from a deck of 52 playing cards and then well shuffled. Now one card is drawn at random from the remaining cards.  Determine the probability that the card is
\begin{enumerate}[label=(\roman*)]
\item a club
\item 10 of hearts
\end{enumerate}
\solution
%\begin{table}[H]
	\centering
\begin{tabular}{|c|c|c|}
\hline
Random variable &Value &Definition\\ \hline
\multirow{3}{*}{X} &0 &Slips of Rs 1\\
&1 &Slips of Rs 5\\
&2 &Slips of Rs 13\\ \hline
\multirow{2}{*}{Y} &0 &Box A\\
&1 &Box B\\\hline
\end{tabular}
\caption{}
\label{tab:Distribution}
\end{table}
See \tabref{tab:Distribution}.
\begin{align}
p_{Y}\brak{k}= \begin{cases} 
      \frac{1}{3} & {k=0} \\
      \frac{2}{3 }& {k=1} 
   \end{cases}
   \\
p_{Y|X}\brak{0|0} = \frac{19}{25}\, 
p_{Y|X}\brak{0|1} = \frac{6}{25}\,
p_{Y|X}\brak{1|0} = \frac{45}{50}\,
p_{Y|X}\brak{1|2} = \frac{5}{50}
\end{align}
The desired probability is the probability that a slip drawn at random is marked other than Rs 1,
\begin{align}
&=1-p_X\brak{0}\\
&= p_X(1) + p_X(2)
\end{align}
Using Bayes theorem,
\begin{align}
&= p_Y\brak{0} \times \pr{Y=0 | X=1} + p_Y\brak{1} \times \pr{Y=1|X=2}\\
&=\frac{1}{3} \times \frac{6}{25} + \frac{2}{3} \times \frac{5}{50}\\
&=\frac{11}{75}
\end{align}

\newpage

%\tableofcontents

\bigskip

\renewcommand{\thefigure}{\theenumi}
\renewcommand{\thetable}{\theenumi}
%\renewcommand{\theequation}{\theenumi}

%\begin{abstract}
%%\boldmath
%In this letter, an algorithm for evaluating the exact analytical bit error rate  (BER)  for the piecewise linear (PL) combiner for  multiple relays is presented. Previous results were available only for upto three relays. The algorithm is unique in the sense that  the actual mathematical expressions, that are prohibitively large, need not be explicitly obtained. The diversity gain due to multiple relays is shown through plots of the analytical BER, well supported by simulations. 
%
%\end{abstract}
% IEEEtran.cls defaults to using nonbold math in the Abstract.
% This preserves the distinction between vectors and scalars. However,
% if the journal you are submitting to favors bold math in the abstract,
% then you can use LaTeX's standard command \boldmath at the very start
% of the abstract to achieve this. Many IEEE journals frown on math
% in the abstract anyway.

% Note that keywords are not normally used for peerreview papers.
%\begin{IEEEkeywords}
%Cooperative diversity, decode and forward, piecewise linear
%\end{IEEEkeywords}



% For peer review papers, you can put extra information on the cover
% page as needed:
% \ifCLASSOPTIONpeerreview
% \begin{center} \bfseries EDICS Category: 3-BBND \end{center}
% \fi
%
% For peerreview papers, this IEEEtran command inserts a page break and
% creates the second title. It will be ignored for other modes.
%\IEEEpeerreviewmaketitle




\item A team of medical students doing their internship have to assist during surgeries
at a city hospital. The probabilities of surgeries rated as very complex, complex,
routine, simple or very simple are respectively, 0.15, 0.20, 0.31, 0.26, .08. Find
the probabilities that a particular surgery will be rated
\begin{enumerate}
	\item complex or very complex;
	\item neither very complex nor very simple;
	\item routine or complex
	\item routine or simple
\end{enumerate}
\solution
%\begin{table}[H]
	\centering
\begin{tabular}{|c|c|c|}
\hline
Random variable &Value &Definition\\ \hline
\multirow{3}{*}{X} &0 &Slips of Rs 1\\
&1 &Slips of Rs 5\\
&2 &Slips of Rs 13\\ \hline
\multirow{2}{*}{Y} &0 &Box A\\
&1 &Box B\\\hline
\end{tabular}
\caption{}
\label{tab:Distribution}
\end{table}
See \tabref{tab:Distribution}.
\begin{align}
p_{Y}\brak{k}= \begin{cases} 
      \frac{1}{3} & {k=0} \\
      \frac{2}{3 }& {k=1} 
   \end{cases}
   \\
p_{Y|X}\brak{0|0} = \frac{19}{25}\, 
p_{Y|X}\brak{0|1} = \frac{6}{25}\,
p_{Y|X}\brak{1|0} = \frac{45}{50}\,
p_{Y|X}\brak{1|2} = \frac{5}{50}
\end{align}
The desired probability is the probability that a slip drawn at random is marked other than Rs 1,
\begin{align}
&=1-p_X\brak{0}\\
&= p_X(1) + p_X(2)
\end{align}
Using Bayes theorem,
\begin{align}
&= p_Y\brak{0} \times \pr{Y=0 | X=1} + p_Y\brak{1} \times \pr{Y=1|X=2}\\
&=\frac{1}{3} \times \frac{6}{25} + \frac{2}{3} \times \frac{5}{50}\\
&=\frac{11}{75}
\end{align}

\newpage

%\tableofcontents

\bigskip

\renewcommand{\thefigure}{\theenumi}
\renewcommand{\thetable}{\theenumi}
%\renewcommand{\theequation}{\theenumi}

%\begin{abstract}
%%\boldmath
%In this letter, an algorithm for evaluating the exact analytical bit error rate  (BER)  for the piecewise linear (PL) combiner for  multiple relays is presented. Previous results were available only for upto three relays. The algorithm is unique in the sense that  the actual mathematical expressions, that are prohibitively large, need not be explicitly obtained. The diversity gain due to multiple relays is shown through plots of the analytical BER, well supported by simulations. 
%
%\end{abstract}
% IEEEtran.cls defaults to using nonbold math in the Abstract.
% This preserves the distinction between vectors and scalars. However,
% if the journal you are submitting to favors bold math in the abstract,
% then you can use LaTeX's standard command \boldmath at the very start
% of the abstract to achieve this. Many IEEE journals frown on math
% in the abstract anyway.

% Note that keywords are not normally used for peerreview papers.
%\begin{IEEEkeywords}
%Cooperative diversity, decode and forward, piecewise linear
%\end{IEEEkeywords}



% For peer review papers, you can put extra information on the cover
% page as needed:
% \ifCLASSOPTIONpeerreview
% \begin{center} \bfseries EDICS Category: 3-BBND \end{center}
% \fi
%
% For peerreview papers, this IEEEtran command inserts a page break and
% creates the second title. It will be ignored for other modes.
%\IEEEpeerreviewmaketitle




\item A card is selected from a pack of 52 cards.
\begin{enumerate}[label=(\alph*)]
    \item How many points are there in the sample space?
    \item Calculate the probability that the card is an ace of spades.
    \item Calculate the probability that the card is (i) an ace and (ii) black card.
\end{enumerate}
\solution
%Let $X$ be an bernoulli rv defined as in \tabref{tab:exemplar/11/16/3/26}.  Then, 
\begin{equation}
    p =
        \frac{4}{11} 
\end{equation}
\begin{table}[H]
	\centering
	\input{exemplar/11/16/3/26/tables/Table2.tex}
	\caption{}
        \label{tab:exemplar/11/16/3/26}
\end{table}

\item The probability that a non leap year selected at random will contain 53 sundays.
\\
\solution
%\begin{table}[H]
	\centering
\begin{tabular}{|c|c|c|}
\hline
Random variable &Value &Definition\\ \hline
\multirow{3}{*}{X} &0 &Slips of Rs 1\\
&1 &Slips of Rs 5\\
&2 &Slips of Rs 13\\ \hline
\multirow{2}{*}{Y} &0 &Box A\\
&1 &Box B\\\hline
\end{tabular}
\caption{}
\label{tab:Distribution}
\end{table}
See \tabref{tab:Distribution}.
\begin{align}
p_{Y}\brak{k}= \begin{cases} 
      \frac{1}{3} & {k=0} \\
      \frac{2}{3 }& {k=1} 
   \end{cases}
   \\
p_{Y|X}\brak{0|0} = \frac{19}{25}\, 
p_{Y|X}\brak{0|1} = \frac{6}{25}\,
p_{Y|X}\brak{1|0} = \frac{45}{50}\,
p_{Y|X}\brak{1|2} = \frac{5}{50}
\end{align}
The desired probability is the probability that a slip drawn at random is marked other than Rs 1,
\begin{align}
&=1-p_X\brak{0}\\
&= p_X(1) + p_X(2)
\end{align}
Using Bayes theorem,
\begin{align}
&= p_Y\brak{0} \times \pr{Y=0 | X=1} + p_Y\brak{1} \times \pr{Y=1|X=2}\\
&=\frac{1}{3} \times \frac{6}{25} + \frac{2}{3} \times \frac{5}{50}\\
&=\frac{11}{75}
\end{align}

\newpage

%\tableofcontents

\bigskip

\renewcommand{\thefigure}{\theenumi}
\renewcommand{\thetable}{\theenumi}
%\renewcommand{\theequation}{\theenumi}

%\begin{abstract}
%%\boldmath
%In this letter, an algorithm for evaluating the exact analytical bit error rate  (BER)  for the piecewise linear (PL) combiner for  multiple relays is presented. Previous results were available only for upto three relays. The algorithm is unique in the sense that  the actual mathematical expressions, that are prohibitively large, need not be explicitly obtained. The diversity gain due to multiple relays is shown through plots of the analytical BER, well supported by simulations. 
%
%\end{abstract}
% IEEEtran.cls defaults to using nonbold math in the Abstract.
% This preserves the distinction between vectors and scalars. However,
% if the journal you are submitting to favors bold math in the abstract,
% then you can use LaTeX's standard command \boldmath at the very start
% of the abstract to achieve this. Many IEEE journals frown on math
% in the abstract anyway.

% Note that keywords are not normally used for peerreview papers.
%\begin{IEEEkeywords}
%Cooperative diversity, decode and forward, piecewise linear
%\end{IEEEkeywords}



% For peer review papers, you can put extra information on the cover
% page as needed:
% \ifCLASSOPTIONpeerreview
% \begin{center} \bfseries EDICS Category: 3-BBND \end{center}
% \fi
%
% For peerreview papers, this IEEEtran command inserts a page break and
% creates the second title. It will be ignored for other modes.
%\IEEEpeerreviewmaketitle




\item One of the four persons John, Rita, Aslam or Gurpreet will be promoted next
month. Consequently the sample space consists of four elementary outcomes
S = {John promoted, Rita promoted, Aslam promoted, Gurpreet promoted}
You are told that the chances of John’s promotion is same as that of Gurpreet,
Rita’s chances of promotion are twice as likely as Johns. Aslam’s chances are
four times that of John.
\begin{enumerate}
	\item Determine
	\begin{enumerate}
		\item P (John promoted)
		\item P (Rita promoted)
		\item P (Aslam promoted)
		\item P (Gurpreet promoted)
	\end{enumerate}
	\item If A = {John promoted or Gurpreet promoted}, find P (A).
\end{enumerate}
\solution
%\begin{table}[H]
	\centering
\begin{tabular}{|c|c|c|}
\hline
Random variable &Value &Definition\\ \hline
\multirow{3}{*}{X} &0 &Slips of Rs 1\\
&1 &Slips of Rs 5\\
&2 &Slips of Rs 13\\ \hline
\multirow{2}{*}{Y} &0 &Box A\\
&1 &Box B\\\hline
\end{tabular}
\caption{}
\label{tab:Distribution}
\end{table}
See \tabref{tab:Distribution}.
\begin{align}
p_{Y}\brak{k}= \begin{cases} 
      \frac{1}{3} & {k=0} \\
      \frac{2}{3 }& {k=1} 
   \end{cases}
   \\
p_{Y|X}\brak{0|0} = \frac{19}{25}\, 
p_{Y|X}\brak{0|1} = \frac{6}{25}\,
p_{Y|X}\brak{1|0} = \frac{45}{50}\,
p_{Y|X}\brak{1|2} = \frac{5}{50}
\end{align}
The desired probability is the probability that a slip drawn at random is marked other than Rs 1,
\begin{align}
&=1-p_X\brak{0}\\
&= p_X(1) + p_X(2)
\end{align}
Using Bayes theorem,
\begin{align}
&= p_Y\brak{0} \times \pr{Y=0 | X=1} + p_Y\brak{1} \times \pr{Y=1|X=2}\\
&=\frac{1}{3} \times \frac{6}{25} + \frac{2}{3} \times \frac{5}{50}\\
&=\frac{11}{75}
\end{align}

\newpage

%\tableofcontents

\bigskip

\renewcommand{\thefigure}{\theenumi}
\renewcommand{\thetable}{\theenumi}
%\renewcommand{\theequation}{\theenumi}

%\begin{abstract}
%%\boldmath
%In this letter, an algorithm for evaluating the exact analytical bit error rate  (BER)  for the piecewise linear (PL) combiner for  multiple relays is presented. Previous results were available only for upto three relays. The algorithm is unique in the sense that  the actual mathematical expressions, that are prohibitively large, need not be explicitly obtained. The diversity gain due to multiple relays is shown through plots of the analytical BER, well supported by simulations. 
%
%\end{abstract}
% IEEEtran.cls defaults to using nonbold math in the Abstract.
% This preserves the distinction between vectors and scalars. However,
% if the journal you are submitting to favors bold math in the abstract,
% then you can use LaTeX's standard command \boldmath at the very start
% of the abstract to achieve this. Many IEEE journals frown on math
% in the abstract anyway.

% Note that keywords are not normally used for peerreview papers.
%\begin{IEEEkeywords}
%Cooperative diversity, decode and forward, piecewise linear
%\end{IEEEkeywords}



% For peer review papers, you can put extra information on the cover
% page as needed:
% \ifCLASSOPTIONpeerreview
% \begin{center} \bfseries EDICS Category: 3-BBND \end{center}
% \fi
%
% For peerreview papers, this IEEEtran command inserts a page break and
% creates the second title. It will be ignored for other modes.
%\IEEEpeerreviewmaketitle




\item A card is drawn from a deck of 52 cards. Find the probability of getting a king or a heart or a red card.\\
\solution
%\begin{table}[H]
	\centering
\begin{tabular}{|c|c|c|}
\hline
Random variable &Value &Definition\\ \hline
\multirow{3}{*}{X} &0 &Slips of Rs 1\\
&1 &Slips of Rs 5\\
&2 &Slips of Rs 13\\ \hline
\multirow{2}{*}{Y} &0 &Box A\\
&1 &Box B\\\hline
\end{tabular}
\caption{}
\label{tab:Distribution}
\end{table}
See \tabref{tab:Distribution}.
\begin{align}
p_{Y}\brak{k}= \begin{cases} 
      \frac{1}{3} & {k=0} \\
      \frac{2}{3 }& {k=1} 
   \end{cases}
   \\
p_{Y|X}\brak{0|0} = \frac{19}{25}\, 
p_{Y|X}\brak{0|1} = \frac{6}{25}\,
p_{Y|X}\brak{1|0} = \frac{45}{50}\,
p_{Y|X}\brak{1|2} = \frac{5}{50}
\end{align}
The desired probability is the probability that a slip drawn at random is marked other than Rs 1,
\begin{align}
&=1-p_X\brak{0}\\
&= p_X(1) + p_X(2)
\end{align}
Using Bayes theorem,
\begin{align}
&= p_Y\brak{0} \times \pr{Y=0 | X=1} + p_Y\brak{1} \times \pr{Y=1|X=2}\\
&=\frac{1}{3} \times \frac{6}{25} + \frac{2}{3} \times \frac{5}{50}\\
&=\frac{11}{75}
\end{align}

\newpage

%\tableofcontents

\bigskip

\renewcommand{\thefigure}{\theenumi}
\renewcommand{\thetable}{\theenumi}
%\renewcommand{\theequation}{\theenumi}

%\begin{abstract}
%%\boldmath
%In this letter, an algorithm for evaluating the exact analytical bit error rate  (BER)  for the piecewise linear (PL) combiner for  multiple relays is presented. Previous results were available only for upto three relays. The algorithm is unique in the sense that  the actual mathematical expressions, that are prohibitively large, need not be explicitly obtained. The diversity gain due to multiple relays is shown through plots of the analytical BER, well supported by simulations. 
%
%\end{abstract}
% IEEEtran.cls defaults to using nonbold math in the Abstract.
% This preserves the distinction between vectors and scalars. However,
% if the journal you are submitting to favors bold math in the abstract,
% then you can use LaTeX's standard command \boldmath at the very start
% of the abstract to achieve this. Many IEEE journals frown on math
% in the abstract anyway.

% Note that keywords are not normally used for peerreview papers.
%\begin{IEEEkeywords}
%Cooperative diversity, decode and forward, piecewise linear
%\end{IEEEkeywords}



% For peer review papers, you can put extra information on the cover
% page as needed:
% \ifCLASSOPTIONpeerreview
% \begin{center} \bfseries EDICS Category: 3-BBND \end{center}
% \fi
%
% For peerreview papers, this IEEEtran command inserts a page break and
% creates the second title. It will be ignored for other modes.
%\IEEEpeerreviewmaketitle




\item The probability that a student will pass his examination is 0.73, the probability of
the student getting a compartment is 0.13, and the probability that the student will
either pass or get compartment is 0.96. State True or False.\\
\solution
%\begin{table}[H]
	\centering
\begin{tabular}{|c|c|c|}
\hline
Random variable &Value &Definition\\ \hline
\multirow{3}{*}{X} &0 &Slips of Rs 1\\
&1 &Slips of Rs 5\\
&2 &Slips of Rs 13\\ \hline
\multirow{2}{*}{Y} &0 &Box A\\
&1 &Box B\\\hline
\end{tabular}
\caption{}
\label{tab:Distribution}
\end{table}
See \tabref{tab:Distribution}.
\begin{align}
p_{Y}\brak{k}= \begin{cases} 
      \frac{1}{3} & {k=0} \\
      \frac{2}{3 }& {k=1} 
   \end{cases}
   \\
p_{Y|X}\brak{0|0} = \frac{19}{25}\, 
p_{Y|X}\brak{0|1} = \frac{6}{25}\,
p_{Y|X}\brak{1|0} = \frac{45}{50}\,
p_{Y|X}\brak{1|2} = \frac{5}{50}
\end{align}
The desired probability is the probability that a slip drawn at random is marked other than Rs 1,
\begin{align}
&=1-p_X\brak{0}\\
&= p_X(1) + p_X(2)
\end{align}
Using Bayes theorem,
\begin{align}
&= p_Y\brak{0} \times \pr{Y=0 | X=1} + p_Y\brak{1} \times \pr{Y=1|X=2}\\
&=\frac{1}{3} \times \frac{6}{25} + \frac{2}{3} \times \frac{5}{50}\\
&=\frac{11}{75}
\end{align}

\newpage

%\tableofcontents

\bigskip

\renewcommand{\thefigure}{\theenumi}
\renewcommand{\thetable}{\theenumi}
%\renewcommand{\theequation}{\theenumi}

%\begin{abstract}
%%\boldmath
%In this letter, an algorithm for evaluating the exact analytical bit error rate  (BER)  for the piecewise linear (PL) combiner for  multiple relays is presented. Previous results were available only for upto three relays. The algorithm is unique in the sense that  the actual mathematical expressions, that are prohibitively large, need not be explicitly obtained. The diversity gain due to multiple relays is shown through plots of the analytical BER, well supported by simulations. 
%
%\end{abstract}
% IEEEtran.cls defaults to using nonbold math in the Abstract.
% This preserves the distinction between vectors and scalars. However,
% if the journal you are submitting to favors bold math in the abstract,
% then you can use LaTeX's standard command \boldmath at the very start
% of the abstract to achieve this. Many IEEE journals frown on math
% in the abstract anyway.

% Note that keywords are not normally used for peerreview papers.
%\begin{IEEEkeywords}
%Cooperative diversity, decode and forward, piecewise linear
%\end{IEEEkeywords}



% For peer review papers, you can put extra information on the cover
% page as needed:
% \ifCLASSOPTIONpeerreview
% \begin{center} \bfseries EDICS Category: 3-BBND \end{center}
% \fi
%
% For peerreview papers, this IEEEtran command inserts a page break and
% creates the second title. It will be ignored for other modes.
%\IEEEpeerreviewmaketitle




\item A card is selected from a pack of 52 cards\\
\begin{enumerate}[label=(\alph*)]
\item How many points are there in the sample space?
\item Calculate the probability that the cards is an ace of spades.
\item Calculate the probability that the card is (i) an ace (ii)black card.\\
\end{enumerate}
%\input{ncert/11/16/3/4_1/Prob_4.tex}
\item In a non-leap year, the probability of having 53 tuesdays or 53 wednesdays is\\
\solution
%A non-leap year has a total of 365 days, and a week has 7 days.\\
So it can be expressed as 
\begin{align}
365\text{days} &=52\times 7+1 \text{day}
\end{align}
$\implies$ 52 tuesdays or wednesdays\\
Random variable X denotes the days of a week
\begin{align}
p_X\brak{k}&=\frac{1}{7}; \quad \brak{1<k<7}
\end{align}
So the probability of extra day being tuesday or wednesday is
\begin{align}
p_X\brak{3}+p_X\brak{4}&=\frac{1}{7}+\frac{1}{7}=\frac{2}{7}
\end{align}



\item There are 1000 sealed envelopes in a box, 10 of them contain a cash prize of
Rs 100 each, 100 of them contain a cash prize of Rs 50 each and 200 of them
contain a cash prize of Rs 10 each and rest do not contain any cash prize. If they
are well shuffled and an envelope is picked up out, what is the probability that it
contains no cash prize?\\
\solution
%\begin{table}[H]
	\centering
\begin{tabular}{|c|c|c|}
\hline
Random variable &Value &Definition\\ \hline
\multirow{3}{*}{X} &0 &Slips of Rs 1\\
&1 &Slips of Rs 5\\
&2 &Slips of Rs 13\\ \hline
\multirow{2}{*}{Y} &0 &Box A\\
&1 &Box B\\\hline
\end{tabular}
\caption{}
\label{tab:Distribution}
\end{table}
See \tabref{tab:Distribution}.
\begin{align}
p_{Y}\brak{k}= \begin{cases} 
      \frac{1}{3} & {k=0} \\
      \frac{2}{3 }& {k=1} 
   \end{cases}
   \\
p_{Y|X}\brak{0|0} = \frac{19}{25}\, 
p_{Y|X}\brak{0|1} = \frac{6}{25}\,
p_{Y|X}\brak{1|0} = \frac{45}{50}\,
p_{Y|X}\brak{1|2} = \frac{5}{50}
\end{align}
The desired probability is the probability that a slip drawn at random is marked other than Rs 1,
\begin{align}
&=1-p_X\brak{0}\\
&= p_X(1) + p_X(2)
\end{align}
Using Bayes theorem,
\begin{align}
&= p_Y\brak{0} \times \pr{Y=0 | X=1} + p_Y\brak{1} \times \pr{Y=1|X=2}\\
&=\frac{1}{3} \times \frac{6}{25} + \frac{2}{3} \times \frac{5}{50}\\
&=\frac{11}{75}
\end{align}

\newpage

%\tableofcontents

\bigskip

\renewcommand{\thefigure}{\theenumi}
\renewcommand{\thetable}{\theenumi}
%\renewcommand{\theequation}{\theenumi}

%\begin{abstract}
%%\boldmath
%In this letter, an algorithm for evaluating the exact analytical bit error rate  (BER)  for the piecewise linear (PL) combiner for  multiple relays is presented. Previous results were available only for upto three relays. The algorithm is unique in the sense that  the actual mathematical expressions, that are prohibitively large, need not be explicitly obtained. The diversity gain due to multiple relays is shown through plots of the analytical BER, well supported by simulations. 
%
%\end{abstract}
% IEEEtran.cls defaults to using nonbold math in the Abstract.
% This preserves the distinction between vectors and scalars. However,
% if the journal you are submitting to favors bold math in the abstract,
% then you can use LaTeX's standard command \boldmath at the very start
% of the abstract to achieve this. Many IEEE journals frown on math
% in the abstract anyway.

% Note that keywords are not normally used for peerreview papers.
%\begin{IEEEkeywords}
%Cooperative diversity, decode and forward, piecewise linear
%\end{IEEEkeywords}



% For peer review papers, you can put extra information on the cover
% page as needed:
% \ifCLASSOPTIONpeerreview
% \begin{center} \bfseries EDICS Category: 3-BBND \end{center}
% \fi
%
% For peerreview papers, this IEEEtran command inserts a page break and
% creates the second title. It will be ignored for other modes.
%\IEEEpeerreviewmaketitle




\item 
A die is thrown and a card is selected at random from a deck of 52 playing cards. The probability of getting an even number on the die and a spade card.\\
\solution
%\begin{table}[H]
	\centering
\begin{tabular}{|c|c|c|}
\hline
Random variable &Value &Definition\\ \hline
\multirow{3}{*}{X} &0 &Slips of Rs 1\\
&1 &Slips of Rs 5\\
&2 &Slips of Rs 13\\ \hline
\multirow{2}{*}{Y} &0 &Box A\\
&1 &Box B\\\hline
\end{tabular}
\caption{}
\label{tab:Distribution}
\end{table}
See \tabref{tab:Distribution}.
\begin{align}
p_{Y}\brak{k}= \begin{cases} 
      \frac{1}{3} & {k=0} \\
      \frac{2}{3 }& {k=1} 
   \end{cases}
   \\
p_{Y|X}\brak{0|0} = \frac{19}{25}\, 
p_{Y|X}\brak{0|1} = \frac{6}{25}\,
p_{Y|X}\brak{1|0} = \frac{45}{50}\,
p_{Y|X}\brak{1|2} = \frac{5}{50}
\end{align}
The desired probability is the probability that a slip drawn at random is marked other than Rs 1,
\begin{align}
&=1-p_X\brak{0}\\
&= p_X(1) + p_X(2)
\end{align}
Using Bayes theorem,
\begin{align}
&= p_Y\brak{0} \times \pr{Y=0 | X=1} + p_Y\brak{1} \times \pr{Y=1|X=2}\\
&=\frac{1}{3} \times \frac{6}{25} + \frac{2}{3} \times \frac{5}{50}\\
&=\frac{11}{75}
\end{align}

\newpage

%\tableofcontents

\bigskip

\renewcommand{\thefigure}{\theenumi}
\renewcommand{\thetable}{\theenumi}
%\renewcommand{\theequation}{\theenumi}

%\begin{abstract}
%%\boldmath
%In this letter, an algorithm for evaluating the exact analytical bit error rate  (BER)  for the piecewise linear (PL) combiner for  multiple relays is presented. Previous results were available only for upto three relays. The algorithm is unique in the sense that  the actual mathematical expressions, that are prohibitively large, need not be explicitly obtained. The diversity gain due to multiple relays is shown through plots of the analytical BER, well supported by simulations. 
%
%\end{abstract}
% IEEEtran.cls defaults to using nonbold math in the Abstract.
% This preserves the distinction between vectors and scalars. However,
% if the journal you are submitting to favors bold math in the abstract,
% then you can use LaTeX's standard command \boldmath at the very start
% of the abstract to achieve this. Many IEEE journals frown on math
% in the abstract anyway.

% Note that keywords are not normally used for peerreview papers.
%\begin{IEEEkeywords}
%Cooperative diversity, decode and forward, piecewise linear
%\end{IEEEkeywords}



% For peer review papers, you can put extra information on the cover
% page as needed:
% \ifCLASSOPTIONpeerreview
% \begin{center} \bfseries EDICS Category: 3-BBND \end{center}
% \fi
%
% For peerreview papers, this IEEEtran command inserts a page break and
% creates the second title. It will be ignored for other modes.
%\IEEEpeerreviewmaketitle




\item
If 4-digit numbers greater than 5,000 are randomly formed from the digits 0, 1, 3, 5, and 7, what is the probability of forming a number divisible by 5 when:
\begin{enumerate}
    \item The digits are repeated?
    \item The repetition of digits is not allowed?
\end{enumerate}
\solution
%\begin{table}[H]
	\centering
\begin{tabular}{|c|c|c|}
\hline
Random variable &Value &Definition\\ \hline
\multirow{3}{*}{X} &0 &Slips of Rs 1\\
&1 &Slips of Rs 5\\
&2 &Slips of Rs 13\\ \hline
\multirow{2}{*}{Y} &0 &Box A\\
&1 &Box B\\\hline
\end{tabular}
\caption{}
\label{tab:Distribution}
\end{table}
See \tabref{tab:Distribution}.
\begin{align}
p_{Y}\brak{k}= \begin{cases} 
      \frac{1}{3} & {k=0} \\
      \frac{2}{3 }& {k=1} 
   \end{cases}
   \\
p_{Y|X}\brak{0|0} = \frac{19}{25}\, 
p_{Y|X}\brak{0|1} = \frac{6}{25}\,
p_{Y|X}\brak{1|0} = \frac{45}{50}\,
p_{Y|X}\brak{1|2} = \frac{5}{50}
\end{align}
The desired probability is the probability that a slip drawn at random is marked other than Rs 1,
\begin{align}
&=1-p_X\brak{0}\\
&= p_X(1) + p_X(2)
\end{align}
Using Bayes theorem,
\begin{align}
&= p_Y\brak{0} \times \pr{Y=0 | X=1} + p_Y\brak{1} \times \pr{Y=1|X=2}\\
&=\frac{1}{3} \times \frac{6}{25} + \frac{2}{3} \times \frac{5}{50}\\
&=\frac{11}{75}
\end{align}

\newpage

%\tableofcontents

\bigskip

\renewcommand{\thefigure}{\theenumi}
\renewcommand{\thetable}{\theenumi}
%\renewcommand{\theequation}{\theenumi}

%\begin{abstract}
%%\boldmath
%In this letter, an algorithm for evaluating the exact analytical bit error rate  (BER)  for the piecewise linear (PL) combiner for  multiple relays is presented. Previous results were available only for upto three relays. The algorithm is unique in the sense that  the actual mathematical expressions, that are prohibitively large, need not be explicitly obtained. The diversity gain due to multiple relays is shown through plots of the analytical BER, well supported by simulations. 
%
%\end{abstract}
% IEEEtran.cls defaults to using nonbold math in the Abstract.
% This preserves the distinction between vectors and scalars. However,
% if the journal you are submitting to favors bold math in the abstract,
% then you can use LaTeX's standard command \boldmath at the very start
% of the abstract to achieve this. Many IEEE journals frown on math
% in the abstract anyway.

% Note that keywords are not normally used for peerreview papers.
%\begin{IEEEkeywords}
%Cooperative diversity, decode and forward, piecewise linear
%\end{IEEEkeywords}



% For peer review papers, you can put extra information on the cover
% page as needed:
% \ifCLASSOPTIONpeerreview
% \begin{center} \bfseries EDICS Category: 3-BBND \end{center}
% \fi
%
% For peerreview papers, this IEEEtran command inserts a page break and
% creates the second title. It will be ignored for other modes.
%\IEEEpeerreviewmaketitle




\item Consider the probability space $\brak{\Omega, \mathcal{G}, P}$ where $\Omega = [0,2]$ and $\mathcal{G} = \cbrak{\phi, \Omega, [0,1], (1,2]}$. Let $X$ and $Y$ be two functions on $\Omega$ defined as
\begin{align*}
    X(\omega) = 
    \begin{cases}
        1 & \text{if }\omega \in [0, 1]\\
        2 & \text{if }\omega \in (1, 2]
    \end{cases}
\end{align*}
and
\begin{align*}
    Y(\omega) = 
    \begin{cases}
        2 & \text{if }\omega \in [0, 1.5]\\
        3 & \text{if }\omega \in (1.5, 2].
    \end{cases}
\end{align*}
Then which one of the following statements is true?
\begin{enumerate}
    \item [(A)] $X$ is a random variable with respect to $\mathcal{G}$, but $Y$ is not a random variable with respect to $\mathcal{G}$.
    \item [(B)] $Y$ is a random variable with respect to $\mathcal{G}$, but $X$ is not a random variable with respect to $\mathcal{G}$.
    \item [(C)] Neither $X$ nor $Y$ is a random variable with respect to $\mathcal{G}$.
    \item [(D)] Both $X$ and $Y$ are random variables with respect to $\mathcal{G}$.
\end{enumerate} \hfill (GATE ST 2023)\\
\solution
%\begin{table}[H]
	\centering
\begin{tabular}{|c|c|c|}
\hline
Random variable &Value &Definition\\ \hline
\multirow{3}{*}{X} &0 &Slips of Rs 1\\
&1 &Slips of Rs 5\\
&2 &Slips of Rs 13\\ \hline
\multirow{2}{*}{Y} &0 &Box A\\
&1 &Box B\\\hline
\end{tabular}
\caption{}
\label{tab:Distribution}
\end{table}
See \tabref{tab:Distribution}.
\begin{align}
p_{Y}\brak{k}= \begin{cases} 
      \frac{1}{3} & {k=0} \\
      \frac{2}{3 }& {k=1} 
   \end{cases}
   \\
p_{Y|X}\brak{0|0} = \frac{19}{25}\, 
p_{Y|X}\brak{0|1} = \frac{6}{25}\,
p_{Y|X}\brak{1|0} = \frac{45}{50}\,
p_{Y|X}\brak{1|2} = \frac{5}{50}
\end{align}
The desired probability is the probability that a slip drawn at random is marked other than Rs 1,
\begin{align}
&=1-p_X\brak{0}\\
&= p_X(1) + p_X(2)
\end{align}
Using Bayes theorem,
\begin{align}
&= p_Y\brak{0} \times \pr{Y=0 | X=1} + p_Y\brak{1} \times \pr{Y=1|X=2}\\
&=\frac{1}{3} \times \frac{6}{25} + \frac{2}{3} \times \frac{5}{50}\\
&=\frac{11}{75}
\end{align}

\newpage

%\tableofcontents

\bigskip

\renewcommand{\thefigure}{\theenumi}
\renewcommand{\thetable}{\theenumi}
%\renewcommand{\theequation}{\theenumi}

%\begin{abstract}
%%\boldmath
%In this letter, an algorithm for evaluating the exact analytical bit error rate  (BER)  for the piecewise linear (PL) combiner for  multiple relays is presented. Previous results were available only for upto three relays. The algorithm is unique in the sense that  the actual mathematical expressions, that are prohibitively large, need not be explicitly obtained. The diversity gain due to multiple relays is shown through plots of the analytical BER, well supported by simulations. 
%
%\end{abstract}
% IEEEtran.cls defaults to using nonbold math in the Abstract.
% This preserves the distinction between vectors and scalars. However,
% if the journal you are submitting to favors bold math in the abstract,
% then you can use LaTeX's standard command \boldmath at the very start
% of the abstract to achieve this. Many IEEE journals frown on math
% in the abstract anyway.

% Note that keywords are not normally used for peerreview papers.
%\begin{IEEEkeywords}
%Cooperative diversity, decode and forward, piecewise linear
%\end{IEEEkeywords}



% For peer review papers, you can put extra information on the cover
% page as needed:
% \ifCLASSOPTIONpeerreview
% \begin{center} \bfseries EDICS Category: 3-BBND \end{center}
% \fi
%
% For peerreview papers, this IEEEtran command inserts a page break and
% creates the second title. It will be ignored for other modes.
%\IEEEpeerreviewmaketitle




	\item  A die is loaded in such a way that each odd number is twice as likely to occur as
each even number. Find $P(G)$, where $G$ is the event that a number greater than
3 occurs on a single roll of the die.
\\
\solution
		%\begin{table}[H]
	\centering
\begin{tabular}{|c|c|c|}
\hline
Random variable &Value &Definition\\ \hline
\multirow{3}{*}{X} &0 &Slips of Rs 1\\
&1 &Slips of Rs 5\\
&2 &Slips of Rs 13\\ \hline
\multirow{2}{*}{Y} &0 &Box A\\
&1 &Box B\\\hline
\end{tabular}
\caption{}
\label{tab:Distribution}
\end{table}
See \tabref{tab:Distribution}.
\begin{align}
p_{Y}\brak{k}= \begin{cases} 
      \frac{1}{3} & {k=0} \\
      \frac{2}{3 }& {k=1} 
   \end{cases}
   \\
p_{Y|X}\brak{0|0} = \frac{19}{25}\, 
p_{Y|X}\brak{0|1} = \frac{6}{25}\,
p_{Y|X}\brak{1|0} = \frac{45}{50}\,
p_{Y|X}\brak{1|2} = \frac{5}{50}
\end{align}
The desired probability is the probability that a slip drawn at random is marked other than Rs 1,
\begin{align}
&=1-p_X\brak{0}\\
&= p_X(1) + p_X(2)
\end{align}
Using Bayes theorem,
\begin{align}
&= p_Y\brak{0} \times \pr{Y=0 | X=1} + p_Y\brak{1} \times \pr{Y=1|X=2}\\
&=\frac{1}{3} \times \frac{6}{25} + \frac{2}{3} \times \frac{5}{50}\\
&=\frac{11}{75}
\end{align}

\newpage

%\tableofcontents

\bigskip

\renewcommand{\thefigure}{\theenumi}
\renewcommand{\thetable}{\theenumi}
%\renewcommand{\theequation}{\theenumi}

%\begin{abstract}
%%\boldmath
%In this letter, an algorithm for evaluating the exact analytical bit error rate  (BER)  for the piecewise linear (PL) combiner for  multiple relays is presented. Previous results were available only for upto three relays. The algorithm is unique in the sense that  the actual mathematical expressions, that are prohibitively large, need not be explicitly obtained. The diversity gain due to multiple relays is shown through plots of the analytical BER, well supported by simulations. 
%
%\end{abstract}
% IEEEtran.cls defaults to using nonbold math in the Abstract.
% This preserves the distinction between vectors and scalars. However,
% if the journal you are submitting to favors bold math in the abstract,
% then you can use LaTeX's standard command \boldmath at the very start
% of the abstract to achieve this. Many IEEE journals frown on math
% in the abstract anyway.

% Note that keywords are not normally used for peerreview papers.
%\begin{IEEEkeywords}
%Cooperative diversity, decode and forward, piecewise linear
%\end{IEEEkeywords}



% For peer review papers, you can put extra information on the cover
% page as needed:
% \ifCLASSOPTIONpeerreview
% \begin{center} \bfseries EDICS Category: 3-BBND \end{center}
% \fi
%
% For peerreview papers, this IEEEtran command inserts a page break and
% creates the second title. It will be ignored for other modes.
%\IEEEpeerreviewmaketitle




	\item All the jacks, queens and kings are removed from a deck of 52 playing cards. The remaining cards are well shuffled and then one card is drawn at random. Giving ace a value 1 similar value for other cards, find the probability that the card has a value 
		\begin{enumerate}
			\item 7
			\item greater than 7
			\item less than 7
		\end{enumerate}
		%Number of cards left after removing all jacks, queens and kings 
\begin{align}
N	= 52 - 4\times 3
	= 40
\end{align}
%\begin{table}[H]
%\def\arraystretch{1.2}
%\begin{tabular}{|c|c|c|}
%\hline
%	\textbf{Parameter} &\textbf{Value} &\textbf{Description}\\ \hline
%	$X$ &1-10 &Represents the value of the card picked \\ \hline
%\end{tabular}
%\end{table}
Let $1 \le X \le 10$ be the value of the card picked.  Then,
\begin{align}
	p_X(k) &= \Pr(X=k)\ \forall\ 1 \leq k \leq 10\\
	&= \frac{4\times 1}{40}\\
	&= \frac{1}{10}\\
	\therefore p_X(k) &= 
	\begin{cases}
		\frac{1}{10} & 1 \leq k \leq 10\\
		0 & \text{otherwise}
	\end{cases}
\end{align}
and
\begin{align}
	F_{X}(k) &= \sum_{m=0}^{k}p_{X}(m) \quad 1 \leq k \leq 10\\
	&= \frac{k}{10}\\
	\therefore F_{X}(k) &= 
	\begin{cases}
		0 & k \leq 0\\
		\frac{k}{10} & 1\leq k \leq 10\\
		1 & k > 10 
	\end{cases}
\end{align}
\begin{enumerate}
	\item Probability that card has value equal to 7 is
		\begin{align}
			 p_{X}(7)
			= \frac{1}{10}
		\end{align}
	\item Probability that card has value greater than 7 is
		\begin{align}
			1 - F_X(7)
			&= 1 - \frac{7}{10}
			\\
			&= \frac{3}{10}
		\end{align}
	\item Probability that card has value less than 7 is
		\begin{align}
			 F_{X}(6)
			=\frac{6}{10}
		\end{align}
\end{enumerate}

  \item A Lot consists of 48 mobile phones of which 42 are good, 3 have only minor defects and 3 have major defects.Varnika will buy a phone if it is good but the trader will only buy a mobile if it has no major defects. One phone is selected at random from the lot. What is the probability that it is
\begin{enumerate}
	\item acceptable to Varnika?
            \item acceptable to the trader?
\end{enumerate}
\solution
	%\begin{table}[H]
	\centering
\begin{tabular}{|c|c|c|}
\hline
Random variable &Value &Definition\\ \hline
\multirow{3}{*}{X} &0 &Slips of Rs 1\\
&1 &Slips of Rs 5\\
&2 &Slips of Rs 13\\ \hline
\multirow{2}{*}{Y} &0 &Box A\\
&1 &Box B\\\hline
\end{tabular}
\caption{}
\label{tab:Distribution}
\end{table}
See \tabref{tab:Distribution}.
\begin{align}
p_{Y}\brak{k}= \begin{cases} 
      \frac{1}{3} & {k=0} \\
      \frac{2}{3 }& {k=1} 
   \end{cases}
   \\
p_{Y|X}\brak{0|0} = \frac{19}{25}\, 
p_{Y|X}\brak{0|1} = \frac{6}{25}\,
p_{Y|X}\brak{1|0} = \frac{45}{50}\,
p_{Y|X}\brak{1|2} = \frac{5}{50}
\end{align}
The desired probability is the probability that a slip drawn at random is marked other than Rs 1,
\begin{align}
&=1-p_X\brak{0}\\
&= p_X(1) + p_X(2)
\end{align}
Using Bayes theorem,
\begin{align}
&= p_Y\brak{0} \times \pr{Y=0 | X=1} + p_Y\brak{1} \times \pr{Y=1|X=2}\\
&=\frac{1}{3} \times \frac{6}{25} + \frac{2}{3} \times \frac{5}{50}\\
&=\frac{11}{75}
\end{align}

\newpage

%\tableofcontents

\bigskip

\renewcommand{\thefigure}{\theenumi}
\renewcommand{\thetable}{\theenumi}
%\renewcommand{\theequation}{\theenumi}

%\begin{abstract}
%%\boldmath
%In this letter, an algorithm for evaluating the exact analytical bit error rate  (BER)  for the piecewise linear (PL) combiner for  multiple relays is presented. Previous results were available only for upto three relays. The algorithm is unique in the sense that  the actual mathematical expressions, that are prohibitively large, need not be explicitly obtained. The diversity gain due to multiple relays is shown through plots of the analytical BER, well supported by simulations. 
%
%\end{abstract}
% IEEEtran.cls defaults to using nonbold math in the Abstract.
% This preserves the distinction between vectors and scalars. However,
% if the journal you are submitting to favors bold math in the abstract,
% then you can use LaTeX's standard command \boldmath at the very start
% of the abstract to achieve this. Many IEEE journals frown on math
% in the abstract anyway.

% Note that keywords are not normally used for peerreview papers.
%\begin{IEEEkeywords}
%Cooperative diversity, decode and forward, piecewise linear
%\end{IEEEkeywords}



% For peer review papers, you can put extra information on the cover
% page as needed:
% \ifCLASSOPTIONpeerreview
% \begin{center} \bfseries EDICS Category: 3-BBND \end{center}
% \fi
%
% For peerreview papers, this IEEEtran command inserts a page break and
% creates the second title. It will be ignored for other modes.
%\IEEEpeerreviewmaketitle




 \item A student says that if you throw a die, it will show up 1 or not 1. Therefore, the probability of getting 1 and the probability of getting 'not 1' each is equal to $\frac{1}{2}$. Is this correct? Give reasons.\\
 \solution
        %\begin{table}[H]
	\centering
\begin{tabular}{|c|c|c|}
\hline
Random variable &Value &Definition\\ \hline
\multirow{3}{*}{X} &0 &Slips of Rs 1\\
&1 &Slips of Rs 5\\
&2 &Slips of Rs 13\\ \hline
\multirow{2}{*}{Y} &0 &Box A\\
&1 &Box B\\\hline
\end{tabular}
\caption{}
\label{tab:Distribution}
\end{table}
See \tabref{tab:Distribution}.
\begin{align}
p_{Y}\brak{k}= \begin{cases} 
      \frac{1}{3} & {k=0} \\
      \frac{2}{3 }& {k=1} 
   \end{cases}
   \\
p_{Y|X}\brak{0|0} = \frac{19}{25}\, 
p_{Y|X}\brak{0|1} = \frac{6}{25}\,
p_{Y|X}\brak{1|0} = \frac{45}{50}\,
p_{Y|X}\brak{1|2} = \frac{5}{50}
\end{align}
The desired probability is the probability that a slip drawn at random is marked other than Rs 1,
\begin{align}
&=1-p_X\brak{0}\\
&= p_X(1) + p_X(2)
\end{align}
Using Bayes theorem,
\begin{align}
&= p_Y\brak{0} \times \pr{Y=0 | X=1} + p_Y\brak{1} \times \pr{Y=1|X=2}\\
&=\frac{1}{3} \times \frac{6}{25} + \frac{2}{3} \times \frac{5}{50}\\
&=\frac{11}{75}
\end{align}

\newpage

%\tableofcontents

\bigskip

\renewcommand{\thefigure}{\theenumi}
\renewcommand{\thetable}{\theenumi}
%\renewcommand{\theequation}{\theenumi}

%\begin{abstract}
%%\boldmath
%In this letter, an algorithm for evaluating the exact analytical bit error rate  (BER)  for the piecewise linear (PL) combiner for  multiple relays is presented. Previous results were available only for upto three relays. The algorithm is unique in the sense that  the actual mathematical expressions, that are prohibitively large, need not be explicitly obtained. The diversity gain due to multiple relays is shown through plots of the analytical BER, well supported by simulations. 
%
%\end{abstract}
% IEEEtran.cls defaults to using nonbold math in the Abstract.
% This preserves the distinction between vectors and scalars. However,
% if the journal you are submitting to favors bold math in the abstract,
% then you can use LaTeX's standard command \boldmath at the very start
% of the abstract to achieve this. Many IEEE journals frown on math
% in the abstract anyway.

% Note that keywords are not normally used for peerreview papers.
%\begin{IEEEkeywords}
%Cooperative diversity, decode and forward, piecewise linear
%\end{IEEEkeywords}



% For peer review papers, you can put extra information on the cover
% page as needed:
% \ifCLASSOPTIONpeerreview
% \begin{center} \bfseries EDICS Category: 3-BBND \end{center}
% \fi
%
% For peerreview papers, this IEEEtran command inserts a page break and
% creates the second title. It will be ignored for other modes.
%\IEEEpeerreviewmaketitle




   \item Four candidates A, B, C, D have ap-
plied for the assignment to coach a school cricket
team. If A is twice as likely to be selected as B, and
B and C are given about the same chance of being
selected, while C is twice as likely to be selected
as D, what are the probabilities that
\begin{enumerate}
\item C will be selected?
\item A will not be selected?
\end{enumerate}
	%\begin{table}[H]
	\centering
\begin{tabular}{|c|c|c|}
\hline
Random variable &Value &Definition\\ \hline
\multirow{3}{*}{X} &0 &Slips of Rs 1\\
&1 &Slips of Rs 5\\
&2 &Slips of Rs 13\\ \hline
\multirow{2}{*}{Y} &0 &Box A\\
&1 &Box B\\\hline
\end{tabular}
\caption{}
\label{tab:Distribution}
\end{table}
See \tabref{tab:Distribution}.
\begin{align}
p_{Y}\brak{k}= \begin{cases} 
      \frac{1}{3} & {k=0} \\
      \frac{2}{3 }& {k=1} 
   \end{cases}
   \\
p_{Y|X}\brak{0|0} = \frac{19}{25}\, 
p_{Y|X}\brak{0|1} = \frac{6}{25}\,
p_{Y|X}\brak{1|0} = \frac{45}{50}\,
p_{Y|X}\brak{1|2} = \frac{5}{50}
\end{align}
The desired probability is the probability that a slip drawn at random is marked other than Rs 1,
\begin{align}
&=1-p_X\brak{0}\\
&= p_X(1) + p_X(2)
\end{align}
Using Bayes theorem,
\begin{align}
&= p_Y\brak{0} \times \pr{Y=0 | X=1} + p_Y\brak{1} \times \pr{Y=1|X=2}\\
&=\frac{1}{3} \times \frac{6}{25} + \frac{2}{3} \times \frac{5}{50}\\
&=\frac{11}{75}
\end{align}

\newpage

%\tableofcontents

\bigskip

\renewcommand{\thefigure}{\theenumi}
\renewcommand{\thetable}{\theenumi}
%\renewcommand{\theequation}{\theenumi}

%\begin{abstract}
%%\boldmath
%In this letter, an algorithm for evaluating the exact analytical bit error rate  (BER)  for the piecewise linear (PL) combiner for  multiple relays is presented. Previous results were available only for upto three relays. The algorithm is unique in the sense that  the actual mathematical expressions, that are prohibitively large, need not be explicitly obtained. The diversity gain due to multiple relays is shown through plots of the analytical BER, well supported by simulations. 
%
%\end{abstract}
% IEEEtran.cls defaults to using nonbold math in the Abstract.
% This preserves the distinction between vectors and scalars. However,
% if the journal you are submitting to favors bold math in the abstract,
% then you can use LaTeX's standard command \boldmath at the very start
% of the abstract to achieve this. Many IEEE journals frown on math
% in the abstract anyway.

% Note that keywords are not normally used for peerreview papers.
%\begin{IEEEkeywords}
%Cooperative diversity, decode and forward, piecewise linear
%\end{IEEEkeywords}



% For peer review papers, you can put extra information on the cover
% page as needed:
% \ifCLASSOPTIONpeerreview
% \begin{center} \bfseries EDICS Category: 3-BBND \end{center}
% \fi
%
% For peerreview papers, this IEEEtran command inserts a page break and
% creates the second title. It will be ignored for other modes.
%\IEEEpeerreviewmaketitle




 \item A bag contain 24 balls of which $x$ balls are red, $2x$ are white and $3x$ are blue. A ball is selected at random, What is the probability that it is
\begin{enumerate}[label=\alph*)]
\item not red ?
\item white ?
\end{enumerate}
%\begin{table}[H]
	\centering
\begin{tabular}{|c|c|c|}
\hline
Random variable &Value &Definition\\ \hline
\multirow{3}{*}{X} &0 &Slips of Rs 1\\
&1 &Slips of Rs 5\\
&2 &Slips of Rs 13\\ \hline
\multirow{2}{*}{Y} &0 &Box A\\
&1 &Box B\\\hline
\end{tabular}
\caption{}
\label{tab:Distribution}
\end{table}
See \tabref{tab:Distribution}.
\begin{align}
p_{Y}\brak{k}= \begin{cases} 
      \frac{1}{3} & {k=0} \\
      \frac{2}{3 }& {k=1} 
   \end{cases}
   \\
p_{Y|X}\brak{0|0} = \frac{19}{25}\, 
p_{Y|X}\brak{0|1} = \frac{6}{25}\,
p_{Y|X}\brak{1|0} = \frac{45}{50}\,
p_{Y|X}\brak{1|2} = \frac{5}{50}
\end{align}
The desired probability is the probability that a slip drawn at random is marked other than Rs 1,
\begin{align}
&=1-p_X\brak{0}\\
&= p_X(1) + p_X(2)
\end{align}
Using Bayes theorem,
\begin{align}
&= p_Y\brak{0} \times \pr{Y=0 | X=1} + p_Y\brak{1} \times \pr{Y=1|X=2}\\
&=\frac{1}{3} \times \frac{6}{25} + \frac{2}{3} \times \frac{5}{50}\\
&=\frac{11}{75}
\end{align}

\newpage

%\tableofcontents

\bigskip

\renewcommand{\thefigure}{\theenumi}
\renewcommand{\thetable}{\theenumi}
%\renewcommand{\theequation}{\theenumi}

%\begin{abstract}
%%\boldmath
%In this letter, an algorithm for evaluating the exact analytical bit error rate  (BER)  for the piecewise linear (PL) combiner for  multiple relays is presented. Previous results were available only for upto three relays. The algorithm is unique in the sense that  the actual mathematical expressions, that are prohibitively large, need not be explicitly obtained. The diversity gain due to multiple relays is shown through plots of the analytical BER, well supported by simulations. 
%
%\end{abstract}
% IEEEtran.cls defaults to using nonbold math in the Abstract.
% This preserves the distinction between vectors and scalars. However,
% if the journal you are submitting to favors bold math in the abstract,
% then you can use LaTeX's standard command \boldmath at the very start
% of the abstract to achieve this. Many IEEE journals frown on math
% in the abstract anyway.

% Note that keywords are not normally used for peerreview papers.
%\begin{IEEEkeywords}
%Cooperative diversity, decode and forward, piecewise linear
%\end{IEEEkeywords}



% For peer review papers, you can put extra information on the cover
% page as needed:
% \ifCLASSOPTIONpeerreview
% \begin{center} \bfseries EDICS Category: 3-BBND \end{center}
% \fi
%
% For peerreview papers, this IEEEtran command inserts a page break and
% creates the second title. It will be ignored for other modes.
%\IEEEpeerreviewmaketitle




If the letters of the word ASSASSINATION are arranged at random. Find the Probability that
\begin{enumerate}[label=(\alph*)]
\item Four $S's$ come consecutively in the word
\item Two  $I's$ and two $N's$ come together
\item All $A's$ are not coming together
\item No two $A's$ are coming together
\end{enumerate}
%\begin{table}[H]
	\centering
\begin{tabular}{|c|c|c|}
\hline
Random variable &Value &Definition\\ \hline
\multirow{3}{*}{X} &0 &Slips of Rs 1\\
&1 &Slips of Rs 5\\
&2 &Slips of Rs 13\\ \hline
\multirow{2}{*}{Y} &0 &Box A\\
&1 &Box B\\\hline
\end{tabular}
\caption{}
\label{tab:Distribution}
\end{table}
See \tabref{tab:Distribution}.
\begin{align}
p_{Y}\brak{k}= \begin{cases} 
      \frac{1}{3} & {k=0} \\
      \frac{2}{3 }& {k=1} 
   \end{cases}
   \\
p_{Y|X}\brak{0|0} = \frac{19}{25}\, 
p_{Y|X}\brak{0|1} = \frac{6}{25}\,
p_{Y|X}\brak{1|0} = \frac{45}{50}\,
p_{Y|X}\brak{1|2} = \frac{5}{50}
\end{align}
The desired probability is the probability that a slip drawn at random is marked other than Rs 1,
\begin{align}
&=1-p_X\brak{0}\\
&= p_X(1) + p_X(2)
\end{align}
Using Bayes theorem,
\begin{align}
&= p_Y\brak{0} \times \pr{Y=0 | X=1} + p_Y\brak{1} \times \pr{Y=1|X=2}\\
&=\frac{1}{3} \times \frac{6}{25} + \frac{2}{3} \times \frac{5}{50}\\
&=\frac{11}{75}
\end{align}

\newpage

%\tableofcontents

\bigskip

\renewcommand{\thefigure}{\theenumi}
\renewcommand{\thetable}{\theenumi}
%\renewcommand{\theequation}{\theenumi}

%\begin{abstract}
%%\boldmath
%In this letter, an algorithm for evaluating the exact analytical bit error rate  (BER)  for the piecewise linear (PL) combiner for  multiple relays is presented. Previous results were available only for upto three relays. The algorithm is unique in the sense that  the actual mathematical expressions, that are prohibitively large, need not be explicitly obtained. The diversity gain due to multiple relays is shown through plots of the analytical BER, well supported by simulations. 
%
%\end{abstract}
% IEEEtran.cls defaults to using nonbold math in the Abstract.
% This preserves the distinction between vectors and scalars. However,
% if the journal you are submitting to favors bold math in the abstract,
% then you can use LaTeX's standard command \boldmath at the very start
% of the abstract to achieve this. Many IEEE journals frown on math
% in the abstract anyway.

% Note that keywords are not normally used for peerreview papers.
%\begin{IEEEkeywords}
%Cooperative diversity, decode and forward, piecewise linear
%\end{IEEEkeywords}



% For peer review papers, you can put extra information on the cover
% page as needed:
% \ifCLASSOPTIONpeerreview
% \begin{center} \bfseries EDICS Category: 3-BBND \end{center}
% \fi
%
% For peerreview papers, this IEEEtran command inserts a page break and
% creates the second title. It will be ignored for other modes.
%\IEEEpeerreviewmaketitle




	\item One urn contains two black balls (labelled B1 and B2) and one white ball. A
	second urn contains one black ball and two white balls (labelled W1 and W2).
	Suppose the following experiment is performed. One of the two urns is chosen
	at random. Next a ball is randomly chosen from the urn. Then a second ball is
	chosen at random from the same urn without replacing the first ball.
	
	\begin{enumerate}
	\item What is the probability that two black balls are chosen?
	
	\item What is the probability that two balls of opposite colour are chosen?
	\end{enumerate}
	\solution
	%\begin{align}
    \label{eq:12.13.6.18.1}
	\because	\pr{A|B} &> \pr{A},\
\frac{\pr{AB}}{\pr{B}} > \pr{A}
\\
    \label{eq:12.13.6.18.2}
	\implies \pr{AB} &> \pr{A}\pr{B}
	\\
	\text{or, } \frac{\pr{AB}}{\pr{A}} &=\pr{B|A} > \pr{A}
\end{align}

\end{enumerate}

		%
\item 
Out of 100 students, two sections of 40 and 60 are formed. If you and your friend are among the 100 students, what is the probability that
\begin{enumerate}
\item you both enter the same section?
\item you both enter the different sections?
\end{enumerate}
\solution
		%\begin{enumerate}[label=\thesection.\arabic*,ref=\thesection.\theenumi]
	\item One card is drawn from a well-shuffled deck of 52 cards. Find the probability of getting
\begin{enumerate}
\item A king of red colour 
\item A face card 
\item A red face card
\item The jack of hearts
\item A spade
\item The queen of diamonds

\end{enumerate}
\solution
		%\begin{table}[H]
	\centering
\begin{tabular}{|c|c|c|}
\hline
Random variable &Value &Definition\\ \hline
\multirow{3}{*}{X} &0 &Slips of Rs 1\\
&1 &Slips of Rs 5\\
&2 &Slips of Rs 13\\ \hline
\multirow{2}{*}{Y} &0 &Box A\\
&1 &Box B\\\hline
\end{tabular}
\caption{}
\label{tab:Distribution}
\end{table}
See \tabref{tab:Distribution}.
\begin{align}
p_{Y}\brak{k}= \begin{cases} 
      \frac{1}{3} & {k=0} \\
      \frac{2}{3 }& {k=1} 
   \end{cases}
   \\
p_{Y|X}\brak{0|0} = \frac{19}{25}\, 
p_{Y|X}\brak{0|1} = \frac{6}{25}\,
p_{Y|X}\brak{1|0} = \frac{45}{50}\,
p_{Y|X}\brak{1|2} = \frac{5}{50}
\end{align}
The desired probability is the probability that a slip drawn at random is marked other than Rs 1,
\begin{align}
&=1-p_X\brak{0}\\
&= p_X(1) + p_X(2)
\end{align}
Using Bayes theorem,
\begin{align}
&= p_Y\brak{0} \times \pr{Y=0 | X=1} + p_Y\brak{1} \times \pr{Y=1|X=2}\\
&=\frac{1}{3} \times \frac{6}{25} + \frac{2}{3} \times \frac{5}{50}\\
&=\frac{11}{75}
\end{align}

\newpage

%\tableofcontents

\bigskip

\renewcommand{\thefigure}{\theenumi}
\renewcommand{\thetable}{\theenumi}
%\renewcommand{\theequation}{\theenumi}

%\begin{abstract}
%%\boldmath
%In this letter, an algorithm for evaluating the exact analytical bit error rate  (BER)  for the piecewise linear (PL) combiner for  multiple relays is presented. Previous results were available only for upto three relays. The algorithm is unique in the sense that  the actual mathematical expressions, that are prohibitively large, need not be explicitly obtained. The diversity gain due to multiple relays is shown through plots of the analytical BER, well supported by simulations. 
%
%\end{abstract}
% IEEEtran.cls defaults to using nonbold math in the Abstract.
% This preserves the distinction between vectors and scalars. However,
% if the journal you are submitting to favors bold math in the abstract,
% then you can use LaTeX's standard command \boldmath at the very start
% of the abstract to achieve this. Many IEEE journals frown on math
% in the abstract anyway.

% Note that keywords are not normally used for peerreview papers.
%\begin{IEEEkeywords}
%Cooperative diversity, decode and forward, piecewise linear
%\end{IEEEkeywords}



% For peer review papers, you can put extra information on the cover
% page as needed:
% \ifCLASSOPTIONpeerreview
% \begin{center} \bfseries EDICS Category: 3-BBND \end{center}
% \fi
%
% For peerreview papers, this IEEEtran command inserts a page break and
% creates the second title. It will be ignored for other modes.
%\IEEEpeerreviewmaketitle




	\item Five cards—the ten, jack, queen, king and ace of diamonds, are well-shuffled with their face downwards. One card is then picked up at random.
\begin{enumerate}
\item
What is the probability that the card is the queen? 
\item
If the queen is drawn and put aside, what is the probability that the second card picked up is (a) an ace? (b) a queen?\\
\end{enumerate}
\solution
		%\begin{enumerate}[label=\thesection.\arabic*,ref=\thesection.\theenumi]
	\item One card is drawn from a well-shuffled deck of 52 cards. Find the probability of getting
\begin{enumerate}
\item A king of red colour 
\item A face card 
\item A red face card
\item The jack of hearts
\item A spade
\item The queen of diamonds

\end{enumerate}
\solution
		%\input{ncert/10/15/1/14/main.tex}
	\item Five cards—the ten, jack, queen, king and ace of diamonds, are well-shuffled with their face downwards. One card is then picked up at random.
\begin{enumerate}
\item
What is the probability that the card is the queen? 
\item
If the queen is drawn and put aside, what is the probability that the second card picked up is (a) an ace? (b) a queen?\\
\end{enumerate}
\solution
		%\input{ncert/10/15/1/15/defs.tex}
	\item A bag contains $5$ red balls and some blue balls. If the probability of drawing a blue ball is double that if a red ball, determine the number of blue balls in the bag. 
		\\
\solution
		%\input{ncert/10/15/2/3/defs.tex}
	\item A card is selected from a pack of 52 cards.
 \begin{enumerate}[label=(\alph*)] 
                 \item How many points are there in the sample space?
                 \item Calculate the probability that the card is an ace of spades.
                 \item Calculate the probability that the card is (i) an ace and (ii) black card.
 \end{enumerate}
\solution
		%\input{ncert/11/16/3/4/main.tex}
\item Four cards are drawn from a well-shuffled deck of 52 cards. What is the probability of obtaining 3 diamonds and one spade.
\\
\solution
		%\input{ncert/11/16/4/2/defs.tex}
\item In a certain lottery 10,000 tickets are sold and ten equal prizes are awarded. What is the probability of not getting a prize if you buy (a) one ticket (b) two tickets (c) 10 tickets ?	
\\
\solution
		%\input{ncert/11/16/4/4/defs.tex}
		%
\item 
Out of 100 students, two sections of 40 and 60 are formed. If you and your friend are among the 100 students, what is the probability that
\begin{enumerate}
\item you both enter the same section?
\item you both enter the different sections?
\end{enumerate}
\solution
		%\input{ncert/11/16/4/5/defs.tex}
	\item 
The number lock of a suitcase has 4 wheels each labelled with ten digits i.e. from 0 to 9.The lock opens with a sequence of four digits with no repeats.What is the probability of a person getting the right sequence to open the suitcase.
\\
\solution
		%\input{ncert/11/16/4/10/defs.tex}
		%
\item 
Two cards are drawn at random and without replacement from a pack of 52 playing cards. Find the probability that both the cards are black.
\\
\solution
		%\input{ncert/12/13/2/2/defs.tex}
		\item A box of oranges is inspected by examining three randomly selected oranges drawn without replacement. If all the three oranges are good, the box is approved for sale, otherwise, it is rejected. Find the probability that a box containing 15 oranges out of which 12 are good and 3 are bad ones will be approved for sale.
		\label{ncert/12/13/2/3/defs.tex}
		\item Two balls are drawn at random with replacement from a box containing 10 black and 8 red balls. Find the probability that
		\label{ncert/12/13/2/12}
\begin{enumerate}
\item both balls are red.
\item first ball is black and second is red.
\item one of them is black and other is red.
\end{enumerate}

\item In a hostel, 60\% of the students read Hindi newspaper, 40\% read English newspaper and 20\% read both Hindi and English newspapers. A student is selected at random.
		\label{ncert/12/13/2/15}
\begin{enumerate}
\item Find the probability that she reads neither Hindi nor English newspapers.
\item If she reads Hindi newspaper, find the probability that she reads English newspaper.
\item If she reads English newspaper, find the probability that she reads Hindi newspaper.\\
\end{enumerate}
\item The probability of obtaining an even prime number on each die, when a pair of dice is rolled is 
\begin{enumerate}
    \item $0$ 
    
    \item $\frac{1}{3}$ 
    
    \item $\frac{1}{12}$ 
    
    \item $\frac{1}{36}$ 
\end{enumerate}
\solution
		%\input{ncert/12/13/2/17/defs.tex}
	\item A bag contains 4 red and 4 black balls, another bag contains 2 red and 6 black balls. One of the two bags is selected at random and a ball is drawn from the bag which is found to be red. Find the probability that the ball is drawn from the first bag.
\\
\solution
		%\input{ncert/12/13/3/2/main.tex}
  \item
  Cards with numbers 2 to 101 are placed in a box. A card is selected at random.Find the probability that the card has
\begin{enumerate}[label=(\roman*)]
	\item an even number 
	\item a square number
\end{enumerate}
\solution
%\input{exemplar/10/13/3/32/main.tex}
\item
The king, queen and jack of clubs are removed from a deck of 52 playing cards and then well shuffled. Now one card is drawn at random from the remaining cards.  Determine the probability that the card is
\begin{enumerate}[label=(\roman*)]
\item a club
\item 10 of hearts
\end{enumerate}
\solution
%\input{exemplar/10/13/3/29/main.tex}
\item A team of medical students doing their internship have to assist during surgeries
at a city hospital. The probabilities of surgeries rated as very complex, complex,
routine, simple or very simple are respectively, 0.15, 0.20, 0.31, 0.26, .08. Find
the probabilities that a particular surgery will be rated
\begin{enumerate}
	\item complex or very complex;
	\item neither very complex nor very simple;
	\item routine or complex
	\item routine or simple
\end{enumerate}
\solution
%\input{exemplar/11/16/3/8(1)/main.tex}
\item A card is selected from a pack of 52 cards.
\begin{enumerate}[label=(\alph*)]
    \item How many points are there in the sample space?
    \item Calculate the probability that the card is an ace of spades.
    \item Calculate the probability that the card is (i) an ace and (ii) black card.
\end{enumerate}
\solution
%\input{exemplar/11/16/3/4/main2.tex}
\item The probability that a non leap year selected at random will contain 53 sundays.
\\
\solution
%\input{exemplar/10/13/1/19/main.tex}
\item One of the four persons John, Rita, Aslam or Gurpreet will be promoted next
month. Consequently the sample space consists of four elementary outcomes
S = {John promoted, Rita promoted, Aslam promoted, Gurpreet promoted}
You are told that the chances of John’s promotion is same as that of Gurpreet,
Rita’s chances of promotion are twice as likely as Johns. Aslam’s chances are
four times that of John.
\begin{enumerate}
	\item Determine
	\begin{enumerate}
		\item P (John promoted)
		\item P (Rita promoted)
		\item P (Aslam promoted)
		\item P (Gurpreet promoted)
	\end{enumerate}
	\item If A = {John promoted or Gurpreet promoted}, find P (A).
\end{enumerate}
\solution
%\input{exemplar/11/16/3/10/main.tex}
\item A card is drawn from a deck of 52 cards. Find the probability of getting a king or a heart or a red card.\\
\solution
%\input{exemplar/11/16/3/15/main.tex}
\item The probability that a student will pass his examination is 0.73, the probability of
the student getting a compartment is 0.13, and the probability that the student will
either pass or get compartment is 0.96. State True or False.\\
\solution
%\input{exemplar/11/16/3/31/main.tex}
\item A card is selected from a pack of 52 cards\\
\begin{enumerate}[label=(\alph*)]
\item How many points are there in the sample space?
\item Calculate the probability that the cards is an ace of spades.
\item Calculate the probability that the card is (i) an ace (ii)black card.\\
\end{enumerate}
%\input{ncert/11/16/3/4_1/Prob_4.tex}
\item In a non-leap year, the probability of having 53 tuesdays or 53 wednesdays is\\
\solution
%\input{exemplar/11/16/3/18/main.tex}
\item There are 1000 sealed envelopes in a box, 10 of them contain a cash prize of
Rs 100 each, 100 of them contain a cash prize of Rs 50 each and 200 of them
contain a cash prize of Rs 10 each and rest do not contain any cash prize. If they
are well shuffled and an envelope is picked up out, what is the probability that it
contains no cash prize?\\
\solution
%\input{exemplar/10/13/3/34/main.tex}
\item 
A die is thrown and a card is selected at random from a deck of 52 playing cards. The probability of getting an even number on the die and a spade card.\\
\solution
%\input{exemplar/12/13/3/78/main.tex}
\item
If 4-digit numbers greater than 5,000 are randomly formed from the digits 0, 1, 3, 5, and 7, what is the probability of forming a number divisible by 5 when:
\begin{enumerate}
    \item The digits are repeated?
    \item The repetition of digits is not allowed?
\end{enumerate}
\solution
%\input{ncert/11/16/4/9/main.tex}
\item Consider the probability space $\brak{\Omega, \mathcal{G}, P}$ where $\Omega = [0,2]$ and $\mathcal{G} = \cbrak{\phi, \Omega, [0,1], (1,2]}$. Let $X$ and $Y$ be two functions on $\Omega$ defined as
\begin{align*}
    X(\omega) = 
    \begin{cases}
        1 & \text{if }\omega \in [0, 1]\\
        2 & \text{if }\omega \in (1, 2]
    \end{cases}
\end{align*}
and
\begin{align*}
    Y(\omega) = 
    \begin{cases}
        2 & \text{if }\omega \in [0, 1.5]\\
        3 & \text{if }\omega \in (1.5, 2].
    \end{cases}
\end{align*}
Then which one of the following statements is true?
\begin{enumerate}
    \item [(A)] $X$ is a random variable with respect to $\mathcal{G}$, but $Y$ is not a random variable with respect to $\mathcal{G}$.
    \item [(B)] $Y$ is a random variable with respect to $\mathcal{G}$, but $X$ is not a random variable with respect to $\mathcal{G}$.
    \item [(C)] Neither $X$ nor $Y$ is a random variable with respect to $\mathcal{G}$.
    \item [(D)] Both $X$ and $Y$ are random variables with respect to $\mathcal{G}$.
\end{enumerate} \hfill (GATE ST 2023)\\
\solution
%\input{gate/ST/2023/14/main.tex}
	\item  A die is loaded in such a way that each odd number is twice as likely to occur as
each even number. Find $P(G)$, where $G$ is the event that a number greater than
3 occurs on a single roll of the die.
\\
\solution
		%\input{exemplar/11/16/3/5/main.tex}
	\item All the jacks, queens and kings are removed from a deck of 52 playing cards. The remaining cards are well shuffled and then one card is drawn at random. Giving ace a value 1 similar value for other cards, find the probability that the card has a value 
		\begin{enumerate}
			\item 7
			\item greater than 7
			\item less than 7
		\end{enumerate}
		%\input{exemplar/10/13/3/30/main.tex}
  \item A Lot consists of 48 mobile phones of which 42 are good, 3 have only minor defects and 3 have major defects.Varnika will buy a phone if it is good but the trader will only buy a mobile if it has no major defects. One phone is selected at random from the lot. What is the probability that it is
\begin{enumerate}
	\item acceptable to Varnika?
            \item acceptable to the trader?
\end{enumerate}
\solution
	%\input{exemplar/10/13/3/40/main.tex}
 \item A student says that if you throw a die, it will show up 1 or not 1. Therefore, the probability of getting 1 and the probability of getting 'not 1' each is equal to $\frac{1}{2}$. Is this correct? Give reasons.\\
 \solution
        %\input{exemplar/10/13/2/9/main.tex}
   \item Four candidates A, B, C, D have ap-
plied for the assignment to coach a school cricket
team. If A is twice as likely to be selected as B, and
B and C are given about the same chance of being
selected, while C is twice as likely to be selected
as D, what are the probabilities that
\begin{enumerate}
\item C will be selected?
\item A will not be selected?
\end{enumerate}
	%\input{exemplar/11/16/3/9/main.tex}
 \item A bag contain 24 balls of which $x$ balls are red, $2x$ are white and $3x$ are blue. A ball is selected at random, What is the probability that it is
\begin{enumerate}[label=\alph*)]
\item not red ?
\item white ?
\end{enumerate}
%\input{exemplar/10/13/3/41/main.tex}
If the letters of the word ASSASSINATION are arranged at random. Find the Probability that
\begin{enumerate}[label=(\alph*)]
\item Four $S's$ come consecutively in the word
\item Two  $I's$ and two $N's$ come together
\item All $A's$ are not coming together
\item No two $A's$ are coming together
\end{enumerate}
%\input{exemplar/11/16/3/14/main.tex}
	\item One urn contains two black balls (labelled B1 and B2) and one white ball. A
	second urn contains one black ball and two white balls (labelled W1 and W2).
	Suppose the following experiment is performed. One of the two urns is chosen
	at random. Next a ball is randomly chosen from the urn. Then a second ball is
	chosen at random from the same urn without replacing the first ball.
	
	\begin{enumerate}
	\item What is the probability that two black balls are chosen?
	
	\item What is the probability that two balls of opposite colour are chosen?
	\end{enumerate}
	\solution
	%\input{exemplar/11/16/3/12/main1.tex}
\end{enumerate}

	\item A bag contains $5$ red balls and some blue balls. If the probability of drawing a blue ball is double that if a red ball, determine the number of blue balls in the bag. 
		\\
\solution
		%\begin{enumerate}[label=\thesection.\arabic*,ref=\thesection.\theenumi]
	\item One card is drawn from a well-shuffled deck of 52 cards. Find the probability of getting
\begin{enumerate}
\item A king of red colour 
\item A face card 
\item A red face card
\item The jack of hearts
\item A spade
\item The queen of diamonds

\end{enumerate}
\solution
		%\input{ncert/10/15/1/14/main.tex}
	\item Five cards—the ten, jack, queen, king and ace of diamonds, are well-shuffled with their face downwards. One card is then picked up at random.
\begin{enumerate}
\item
What is the probability that the card is the queen? 
\item
If the queen is drawn and put aside, what is the probability that the second card picked up is (a) an ace? (b) a queen?\\
\end{enumerate}
\solution
		%\input{ncert/10/15/1/15/defs.tex}
	\item A bag contains $5$ red balls and some blue balls. If the probability of drawing a blue ball is double that if a red ball, determine the number of blue balls in the bag. 
		\\
\solution
		%\input{ncert/10/15/2/3/defs.tex}
	\item A card is selected from a pack of 52 cards.
 \begin{enumerate}[label=(\alph*)] 
                 \item How many points are there in the sample space?
                 \item Calculate the probability that the card is an ace of spades.
                 \item Calculate the probability that the card is (i) an ace and (ii) black card.
 \end{enumerate}
\solution
		%\input{ncert/11/16/3/4/main.tex}
\item Four cards are drawn from a well-shuffled deck of 52 cards. What is the probability of obtaining 3 diamonds and one spade.
\\
\solution
		%\input{ncert/11/16/4/2/defs.tex}
\item In a certain lottery 10,000 tickets are sold and ten equal prizes are awarded. What is the probability of not getting a prize if you buy (a) one ticket (b) two tickets (c) 10 tickets ?	
\\
\solution
		%\input{ncert/11/16/4/4/defs.tex}
		%
\item 
Out of 100 students, two sections of 40 and 60 are formed. If you and your friend are among the 100 students, what is the probability that
\begin{enumerate}
\item you both enter the same section?
\item you both enter the different sections?
\end{enumerate}
\solution
		%\input{ncert/11/16/4/5/defs.tex}
	\item 
The number lock of a suitcase has 4 wheels each labelled with ten digits i.e. from 0 to 9.The lock opens with a sequence of four digits with no repeats.What is the probability of a person getting the right sequence to open the suitcase.
\\
\solution
		%\input{ncert/11/16/4/10/defs.tex}
		%
\item 
Two cards are drawn at random and without replacement from a pack of 52 playing cards. Find the probability that both the cards are black.
\\
\solution
		%\input{ncert/12/13/2/2/defs.tex}
		\item A box of oranges is inspected by examining three randomly selected oranges drawn without replacement. If all the three oranges are good, the box is approved for sale, otherwise, it is rejected. Find the probability that a box containing 15 oranges out of which 12 are good and 3 are bad ones will be approved for sale.
		\label{ncert/12/13/2/3/defs.tex}
		\item Two balls are drawn at random with replacement from a box containing 10 black and 8 red balls. Find the probability that
		\label{ncert/12/13/2/12}
\begin{enumerate}
\item both balls are red.
\item first ball is black and second is red.
\item one of them is black and other is red.
\end{enumerate}

\item In a hostel, 60\% of the students read Hindi newspaper, 40\% read English newspaper and 20\% read both Hindi and English newspapers. A student is selected at random.
		\label{ncert/12/13/2/15}
\begin{enumerate}
\item Find the probability that she reads neither Hindi nor English newspapers.
\item If she reads Hindi newspaper, find the probability that she reads English newspaper.
\item If she reads English newspaper, find the probability that she reads Hindi newspaper.\\
\end{enumerate}
\item The probability of obtaining an even prime number on each die, when a pair of dice is rolled is 
\begin{enumerate}
    \item $0$ 
    
    \item $\frac{1}{3}$ 
    
    \item $\frac{1}{12}$ 
    
    \item $\frac{1}{36}$ 
\end{enumerate}
\solution
		%\input{ncert/12/13/2/17/defs.tex}
	\item A bag contains 4 red and 4 black balls, another bag contains 2 red and 6 black balls. One of the two bags is selected at random and a ball is drawn from the bag which is found to be red. Find the probability that the ball is drawn from the first bag.
\\
\solution
		%\input{ncert/12/13/3/2/main.tex}
  \item
  Cards with numbers 2 to 101 are placed in a box. A card is selected at random.Find the probability that the card has
\begin{enumerate}[label=(\roman*)]
	\item an even number 
	\item a square number
\end{enumerate}
\solution
%\input{exemplar/10/13/3/32/main.tex}
\item
The king, queen and jack of clubs are removed from a deck of 52 playing cards and then well shuffled. Now one card is drawn at random from the remaining cards.  Determine the probability that the card is
\begin{enumerate}[label=(\roman*)]
\item a club
\item 10 of hearts
\end{enumerate}
\solution
%\input{exemplar/10/13/3/29/main.tex}
\item A team of medical students doing their internship have to assist during surgeries
at a city hospital. The probabilities of surgeries rated as very complex, complex,
routine, simple or very simple are respectively, 0.15, 0.20, 0.31, 0.26, .08. Find
the probabilities that a particular surgery will be rated
\begin{enumerate}
	\item complex or very complex;
	\item neither very complex nor very simple;
	\item routine or complex
	\item routine or simple
\end{enumerate}
\solution
%\input{exemplar/11/16/3/8(1)/main.tex}
\item A card is selected from a pack of 52 cards.
\begin{enumerate}[label=(\alph*)]
    \item How many points are there in the sample space?
    \item Calculate the probability that the card is an ace of spades.
    \item Calculate the probability that the card is (i) an ace and (ii) black card.
\end{enumerate}
\solution
%\input{exemplar/11/16/3/4/main2.tex}
\item The probability that a non leap year selected at random will contain 53 sundays.
\\
\solution
%\input{exemplar/10/13/1/19/main.tex}
\item One of the four persons John, Rita, Aslam or Gurpreet will be promoted next
month. Consequently the sample space consists of four elementary outcomes
S = {John promoted, Rita promoted, Aslam promoted, Gurpreet promoted}
You are told that the chances of John’s promotion is same as that of Gurpreet,
Rita’s chances of promotion are twice as likely as Johns. Aslam’s chances are
four times that of John.
\begin{enumerate}
	\item Determine
	\begin{enumerate}
		\item P (John promoted)
		\item P (Rita promoted)
		\item P (Aslam promoted)
		\item P (Gurpreet promoted)
	\end{enumerate}
	\item If A = {John promoted or Gurpreet promoted}, find P (A).
\end{enumerate}
\solution
%\input{exemplar/11/16/3/10/main.tex}
\item A card is drawn from a deck of 52 cards. Find the probability of getting a king or a heart or a red card.\\
\solution
%\input{exemplar/11/16/3/15/main.tex}
\item The probability that a student will pass his examination is 0.73, the probability of
the student getting a compartment is 0.13, and the probability that the student will
either pass or get compartment is 0.96. State True or False.\\
\solution
%\input{exemplar/11/16/3/31/main.tex}
\item A card is selected from a pack of 52 cards\\
\begin{enumerate}[label=(\alph*)]
\item How many points are there in the sample space?
\item Calculate the probability that the cards is an ace of spades.
\item Calculate the probability that the card is (i) an ace (ii)black card.\\
\end{enumerate}
%\input{ncert/11/16/3/4_1/Prob_4.tex}
\item In a non-leap year, the probability of having 53 tuesdays or 53 wednesdays is\\
\solution
%\input{exemplar/11/16/3/18/main.tex}
\item There are 1000 sealed envelopes in a box, 10 of them contain a cash prize of
Rs 100 each, 100 of them contain a cash prize of Rs 50 each and 200 of them
contain a cash prize of Rs 10 each and rest do not contain any cash prize. If they
are well shuffled and an envelope is picked up out, what is the probability that it
contains no cash prize?\\
\solution
%\input{exemplar/10/13/3/34/main.tex}
\item 
A die is thrown and a card is selected at random from a deck of 52 playing cards. The probability of getting an even number on the die and a spade card.\\
\solution
%\input{exemplar/12/13/3/78/main.tex}
\item
If 4-digit numbers greater than 5,000 are randomly formed from the digits 0, 1, 3, 5, and 7, what is the probability of forming a number divisible by 5 when:
\begin{enumerate}
    \item The digits are repeated?
    \item The repetition of digits is not allowed?
\end{enumerate}
\solution
%\input{ncert/11/16/4/9/main.tex}
\item Consider the probability space $\brak{\Omega, \mathcal{G}, P}$ where $\Omega = [0,2]$ and $\mathcal{G} = \cbrak{\phi, \Omega, [0,1], (1,2]}$. Let $X$ and $Y$ be two functions on $\Omega$ defined as
\begin{align*}
    X(\omega) = 
    \begin{cases}
        1 & \text{if }\omega \in [0, 1]\\
        2 & \text{if }\omega \in (1, 2]
    \end{cases}
\end{align*}
and
\begin{align*}
    Y(\omega) = 
    \begin{cases}
        2 & \text{if }\omega \in [0, 1.5]\\
        3 & \text{if }\omega \in (1.5, 2].
    \end{cases}
\end{align*}
Then which one of the following statements is true?
\begin{enumerate}
    \item [(A)] $X$ is a random variable with respect to $\mathcal{G}$, but $Y$ is not a random variable with respect to $\mathcal{G}$.
    \item [(B)] $Y$ is a random variable with respect to $\mathcal{G}$, but $X$ is not a random variable with respect to $\mathcal{G}$.
    \item [(C)] Neither $X$ nor $Y$ is a random variable with respect to $\mathcal{G}$.
    \item [(D)] Both $X$ and $Y$ are random variables with respect to $\mathcal{G}$.
\end{enumerate} \hfill (GATE ST 2023)\\
\solution
%\input{gate/ST/2023/14/main.tex}
	\item  A die is loaded in such a way that each odd number is twice as likely to occur as
each even number. Find $P(G)$, where $G$ is the event that a number greater than
3 occurs on a single roll of the die.
\\
\solution
		%\input{exemplar/11/16/3/5/main.tex}
	\item All the jacks, queens and kings are removed from a deck of 52 playing cards. The remaining cards are well shuffled and then one card is drawn at random. Giving ace a value 1 similar value for other cards, find the probability that the card has a value 
		\begin{enumerate}
			\item 7
			\item greater than 7
			\item less than 7
		\end{enumerate}
		%\input{exemplar/10/13/3/30/main.tex}
  \item A Lot consists of 48 mobile phones of which 42 are good, 3 have only minor defects and 3 have major defects.Varnika will buy a phone if it is good but the trader will only buy a mobile if it has no major defects. One phone is selected at random from the lot. What is the probability that it is
\begin{enumerate}
	\item acceptable to Varnika?
            \item acceptable to the trader?
\end{enumerate}
\solution
	%\input{exemplar/10/13/3/40/main.tex}
 \item A student says that if you throw a die, it will show up 1 or not 1. Therefore, the probability of getting 1 and the probability of getting 'not 1' each is equal to $\frac{1}{2}$. Is this correct? Give reasons.\\
 \solution
        %\input{exemplar/10/13/2/9/main.tex}
   \item Four candidates A, B, C, D have ap-
plied for the assignment to coach a school cricket
team. If A is twice as likely to be selected as B, and
B and C are given about the same chance of being
selected, while C is twice as likely to be selected
as D, what are the probabilities that
\begin{enumerate}
\item C will be selected?
\item A will not be selected?
\end{enumerate}
	%\input{exemplar/11/16/3/9/main.tex}
 \item A bag contain 24 balls of which $x$ balls are red, $2x$ are white and $3x$ are blue. A ball is selected at random, What is the probability that it is
\begin{enumerate}[label=\alph*)]
\item not red ?
\item white ?
\end{enumerate}
%\input{exemplar/10/13/3/41/main.tex}
If the letters of the word ASSASSINATION are arranged at random. Find the Probability that
\begin{enumerate}[label=(\alph*)]
\item Four $S's$ come consecutively in the word
\item Two  $I's$ and two $N's$ come together
\item All $A's$ are not coming together
\item No two $A's$ are coming together
\end{enumerate}
%\input{exemplar/11/16/3/14/main.tex}
	\item One urn contains two black balls (labelled B1 and B2) and one white ball. A
	second urn contains one black ball and two white balls (labelled W1 and W2).
	Suppose the following experiment is performed. One of the two urns is chosen
	at random. Next a ball is randomly chosen from the urn. Then a second ball is
	chosen at random from the same urn without replacing the first ball.
	
	\begin{enumerate}
	\item What is the probability that two black balls are chosen?
	
	\item What is the probability that two balls of opposite colour are chosen?
	\end{enumerate}
	\solution
	%\input{exemplar/11/16/3/12/main1.tex}
\end{enumerate}

	\item A card is selected from a pack of 52 cards.
 \begin{enumerate}[label=(\alph*)] 
                 \item How many points are there in the sample space?
                 \item Calculate the probability that the card is an ace of spades.
                 \item Calculate the probability that the card is (i) an ace and (ii) black card.
 \end{enumerate}
\solution
		%\begin{table}[H]
	\centering
\begin{tabular}{|c|c|c|}
\hline
Random variable &Value &Definition\\ \hline
\multirow{3}{*}{X} &0 &Slips of Rs 1\\
&1 &Slips of Rs 5\\
&2 &Slips of Rs 13\\ \hline
\multirow{2}{*}{Y} &0 &Box A\\
&1 &Box B\\\hline
\end{tabular}
\caption{}
\label{tab:Distribution}
\end{table}
See \tabref{tab:Distribution}.
\begin{align}
p_{Y}\brak{k}= \begin{cases} 
      \frac{1}{3} & {k=0} \\
      \frac{2}{3 }& {k=1} 
   \end{cases}
   \\
p_{Y|X}\brak{0|0} = \frac{19}{25}\, 
p_{Y|X}\brak{0|1} = \frac{6}{25}\,
p_{Y|X}\brak{1|0} = \frac{45}{50}\,
p_{Y|X}\brak{1|2} = \frac{5}{50}
\end{align}
The desired probability is the probability that a slip drawn at random is marked other than Rs 1,
\begin{align}
&=1-p_X\brak{0}\\
&= p_X(1) + p_X(2)
\end{align}
Using Bayes theorem,
\begin{align}
&= p_Y\brak{0} \times \pr{Y=0 | X=1} + p_Y\brak{1} \times \pr{Y=1|X=2}\\
&=\frac{1}{3} \times \frac{6}{25} + \frac{2}{3} \times \frac{5}{50}\\
&=\frac{11}{75}
\end{align}

\newpage

%\tableofcontents

\bigskip

\renewcommand{\thefigure}{\theenumi}
\renewcommand{\thetable}{\theenumi}
%\renewcommand{\theequation}{\theenumi}

%\begin{abstract}
%%\boldmath
%In this letter, an algorithm for evaluating the exact analytical bit error rate  (BER)  for the piecewise linear (PL) combiner for  multiple relays is presented. Previous results were available only for upto three relays. The algorithm is unique in the sense that  the actual mathematical expressions, that are prohibitively large, need not be explicitly obtained. The diversity gain due to multiple relays is shown through plots of the analytical BER, well supported by simulations. 
%
%\end{abstract}
% IEEEtran.cls defaults to using nonbold math in the Abstract.
% This preserves the distinction between vectors and scalars. However,
% if the journal you are submitting to favors bold math in the abstract,
% then you can use LaTeX's standard command \boldmath at the very start
% of the abstract to achieve this. Many IEEE journals frown on math
% in the abstract anyway.

% Note that keywords are not normally used for peerreview papers.
%\begin{IEEEkeywords}
%Cooperative diversity, decode and forward, piecewise linear
%\end{IEEEkeywords}



% For peer review papers, you can put extra information on the cover
% page as needed:
% \ifCLASSOPTIONpeerreview
% \begin{center} \bfseries EDICS Category: 3-BBND \end{center}
% \fi
%
% For peerreview papers, this IEEEtran command inserts a page break and
% creates the second title. It will be ignored for other modes.
%\IEEEpeerreviewmaketitle




\item Four cards are drawn from a well-shuffled deck of 52 cards. What is the probability of obtaining 3 diamonds and one spade.
\\
\solution
		%\begin{enumerate}[label=\thesection.\arabic*,ref=\thesection.\theenumi]
	\item One card is drawn from a well-shuffled deck of 52 cards. Find the probability of getting
\begin{enumerate}
\item A king of red colour 
\item A face card 
\item A red face card
\item The jack of hearts
\item A spade
\item The queen of diamonds

\end{enumerate}
\solution
		%\input{ncert/10/15/1/14/main.tex}
	\item Five cards—the ten, jack, queen, king and ace of diamonds, are well-shuffled with their face downwards. One card is then picked up at random.
\begin{enumerate}
\item
What is the probability that the card is the queen? 
\item
If the queen is drawn and put aside, what is the probability that the second card picked up is (a) an ace? (b) a queen?\\
\end{enumerate}
\solution
		%\input{ncert/10/15/1/15/defs.tex}
	\item A bag contains $5$ red balls and some blue balls. If the probability of drawing a blue ball is double that if a red ball, determine the number of blue balls in the bag. 
		\\
\solution
		%\input{ncert/10/15/2/3/defs.tex}
	\item A card is selected from a pack of 52 cards.
 \begin{enumerate}[label=(\alph*)] 
                 \item How many points are there in the sample space?
                 \item Calculate the probability that the card is an ace of spades.
                 \item Calculate the probability that the card is (i) an ace and (ii) black card.
 \end{enumerate}
\solution
		%\input{ncert/11/16/3/4/main.tex}
\item Four cards are drawn from a well-shuffled deck of 52 cards. What is the probability of obtaining 3 diamonds and one spade.
\\
\solution
		%\input{ncert/11/16/4/2/defs.tex}
\item In a certain lottery 10,000 tickets are sold and ten equal prizes are awarded. What is the probability of not getting a prize if you buy (a) one ticket (b) two tickets (c) 10 tickets ?	
\\
\solution
		%\input{ncert/11/16/4/4/defs.tex}
		%
\item 
Out of 100 students, two sections of 40 and 60 are formed. If you and your friend are among the 100 students, what is the probability that
\begin{enumerate}
\item you both enter the same section?
\item you both enter the different sections?
\end{enumerate}
\solution
		%\input{ncert/11/16/4/5/defs.tex}
	\item 
The number lock of a suitcase has 4 wheels each labelled with ten digits i.e. from 0 to 9.The lock opens with a sequence of four digits with no repeats.What is the probability of a person getting the right sequence to open the suitcase.
\\
\solution
		%\input{ncert/11/16/4/10/defs.tex}
		%
\item 
Two cards are drawn at random and without replacement from a pack of 52 playing cards. Find the probability that both the cards are black.
\\
\solution
		%\input{ncert/12/13/2/2/defs.tex}
		\item A box of oranges is inspected by examining three randomly selected oranges drawn without replacement. If all the three oranges are good, the box is approved for sale, otherwise, it is rejected. Find the probability that a box containing 15 oranges out of which 12 are good and 3 are bad ones will be approved for sale.
		\label{ncert/12/13/2/3/defs.tex}
		\item Two balls are drawn at random with replacement from a box containing 10 black and 8 red balls. Find the probability that
		\label{ncert/12/13/2/12}
\begin{enumerate}
\item both balls are red.
\item first ball is black and second is red.
\item one of them is black and other is red.
\end{enumerate}

\item In a hostel, 60\% of the students read Hindi newspaper, 40\% read English newspaper and 20\% read both Hindi and English newspapers. A student is selected at random.
		\label{ncert/12/13/2/15}
\begin{enumerate}
\item Find the probability that she reads neither Hindi nor English newspapers.
\item If she reads Hindi newspaper, find the probability that she reads English newspaper.
\item If she reads English newspaper, find the probability that she reads Hindi newspaper.\\
\end{enumerate}
\item The probability of obtaining an even prime number on each die, when a pair of dice is rolled is 
\begin{enumerate}
    \item $0$ 
    
    \item $\frac{1}{3}$ 
    
    \item $\frac{1}{12}$ 
    
    \item $\frac{1}{36}$ 
\end{enumerate}
\solution
		%\input{ncert/12/13/2/17/defs.tex}
	\item A bag contains 4 red and 4 black balls, another bag contains 2 red and 6 black balls. One of the two bags is selected at random and a ball is drawn from the bag which is found to be red. Find the probability that the ball is drawn from the first bag.
\\
\solution
		%\input{ncert/12/13/3/2/main.tex}
  \item
  Cards with numbers 2 to 101 are placed in a box. A card is selected at random.Find the probability that the card has
\begin{enumerate}[label=(\roman*)]
	\item an even number 
	\item a square number
\end{enumerate}
\solution
%\input{exemplar/10/13/3/32/main.tex}
\item
The king, queen and jack of clubs are removed from a deck of 52 playing cards and then well shuffled. Now one card is drawn at random from the remaining cards.  Determine the probability that the card is
\begin{enumerate}[label=(\roman*)]
\item a club
\item 10 of hearts
\end{enumerate}
\solution
%\input{exemplar/10/13/3/29/main.tex}
\item A team of medical students doing their internship have to assist during surgeries
at a city hospital. The probabilities of surgeries rated as very complex, complex,
routine, simple or very simple are respectively, 0.15, 0.20, 0.31, 0.26, .08. Find
the probabilities that a particular surgery will be rated
\begin{enumerate}
	\item complex or very complex;
	\item neither very complex nor very simple;
	\item routine or complex
	\item routine or simple
\end{enumerate}
\solution
%\input{exemplar/11/16/3/8(1)/main.tex}
\item A card is selected from a pack of 52 cards.
\begin{enumerate}[label=(\alph*)]
    \item How many points are there in the sample space?
    \item Calculate the probability that the card is an ace of spades.
    \item Calculate the probability that the card is (i) an ace and (ii) black card.
\end{enumerate}
\solution
%\input{exemplar/11/16/3/4/main2.tex}
\item The probability that a non leap year selected at random will contain 53 sundays.
\\
\solution
%\input{exemplar/10/13/1/19/main.tex}
\item One of the four persons John, Rita, Aslam or Gurpreet will be promoted next
month. Consequently the sample space consists of four elementary outcomes
S = {John promoted, Rita promoted, Aslam promoted, Gurpreet promoted}
You are told that the chances of John’s promotion is same as that of Gurpreet,
Rita’s chances of promotion are twice as likely as Johns. Aslam’s chances are
four times that of John.
\begin{enumerate}
	\item Determine
	\begin{enumerate}
		\item P (John promoted)
		\item P (Rita promoted)
		\item P (Aslam promoted)
		\item P (Gurpreet promoted)
	\end{enumerate}
	\item If A = {John promoted or Gurpreet promoted}, find P (A).
\end{enumerate}
\solution
%\input{exemplar/11/16/3/10/main.tex}
\item A card is drawn from a deck of 52 cards. Find the probability of getting a king or a heart or a red card.\\
\solution
%\input{exemplar/11/16/3/15/main.tex}
\item The probability that a student will pass his examination is 0.73, the probability of
the student getting a compartment is 0.13, and the probability that the student will
either pass or get compartment is 0.96. State True or False.\\
\solution
%\input{exemplar/11/16/3/31/main.tex}
\item A card is selected from a pack of 52 cards\\
\begin{enumerate}[label=(\alph*)]
\item How many points are there in the sample space?
\item Calculate the probability that the cards is an ace of spades.
\item Calculate the probability that the card is (i) an ace (ii)black card.\\
\end{enumerate}
%\input{ncert/11/16/3/4_1/Prob_4.tex}
\item In a non-leap year, the probability of having 53 tuesdays or 53 wednesdays is\\
\solution
%\input{exemplar/11/16/3/18/main.tex}
\item There are 1000 sealed envelopes in a box, 10 of them contain a cash prize of
Rs 100 each, 100 of them contain a cash prize of Rs 50 each and 200 of them
contain a cash prize of Rs 10 each and rest do not contain any cash prize. If they
are well shuffled and an envelope is picked up out, what is the probability that it
contains no cash prize?\\
\solution
%\input{exemplar/10/13/3/34/main.tex}
\item 
A die is thrown and a card is selected at random from a deck of 52 playing cards. The probability of getting an even number on the die and a spade card.\\
\solution
%\input{exemplar/12/13/3/78/main.tex}
\item
If 4-digit numbers greater than 5,000 are randomly formed from the digits 0, 1, 3, 5, and 7, what is the probability of forming a number divisible by 5 when:
\begin{enumerate}
    \item The digits are repeated?
    \item The repetition of digits is not allowed?
\end{enumerate}
\solution
%\input{ncert/11/16/4/9/main.tex}
\item Consider the probability space $\brak{\Omega, \mathcal{G}, P}$ where $\Omega = [0,2]$ and $\mathcal{G} = \cbrak{\phi, \Omega, [0,1], (1,2]}$. Let $X$ and $Y$ be two functions on $\Omega$ defined as
\begin{align*}
    X(\omega) = 
    \begin{cases}
        1 & \text{if }\omega \in [0, 1]\\
        2 & \text{if }\omega \in (1, 2]
    \end{cases}
\end{align*}
and
\begin{align*}
    Y(\omega) = 
    \begin{cases}
        2 & \text{if }\omega \in [0, 1.5]\\
        3 & \text{if }\omega \in (1.5, 2].
    \end{cases}
\end{align*}
Then which one of the following statements is true?
\begin{enumerate}
    \item [(A)] $X$ is a random variable with respect to $\mathcal{G}$, but $Y$ is not a random variable with respect to $\mathcal{G}$.
    \item [(B)] $Y$ is a random variable with respect to $\mathcal{G}$, but $X$ is not a random variable with respect to $\mathcal{G}$.
    \item [(C)] Neither $X$ nor $Y$ is a random variable with respect to $\mathcal{G}$.
    \item [(D)] Both $X$ and $Y$ are random variables with respect to $\mathcal{G}$.
\end{enumerate} \hfill (GATE ST 2023)\\
\solution
%\input{gate/ST/2023/14/main.tex}
	\item  A die is loaded in such a way that each odd number is twice as likely to occur as
each even number. Find $P(G)$, where $G$ is the event that a number greater than
3 occurs on a single roll of the die.
\\
\solution
		%\input{exemplar/11/16/3/5/main.tex}
	\item All the jacks, queens and kings are removed from a deck of 52 playing cards. The remaining cards are well shuffled and then one card is drawn at random. Giving ace a value 1 similar value for other cards, find the probability that the card has a value 
		\begin{enumerate}
			\item 7
			\item greater than 7
			\item less than 7
		\end{enumerate}
		%\input{exemplar/10/13/3/30/main.tex}
  \item A Lot consists of 48 mobile phones of which 42 are good, 3 have only minor defects and 3 have major defects.Varnika will buy a phone if it is good but the trader will only buy a mobile if it has no major defects. One phone is selected at random from the lot. What is the probability that it is
\begin{enumerate}
	\item acceptable to Varnika?
            \item acceptable to the trader?
\end{enumerate}
\solution
	%\input{exemplar/10/13/3/40/main.tex}
 \item A student says that if you throw a die, it will show up 1 or not 1. Therefore, the probability of getting 1 and the probability of getting 'not 1' each is equal to $\frac{1}{2}$. Is this correct? Give reasons.\\
 \solution
        %\input{exemplar/10/13/2/9/main.tex}
   \item Four candidates A, B, C, D have ap-
plied for the assignment to coach a school cricket
team. If A is twice as likely to be selected as B, and
B and C are given about the same chance of being
selected, while C is twice as likely to be selected
as D, what are the probabilities that
\begin{enumerate}
\item C will be selected?
\item A will not be selected?
\end{enumerate}
	%\input{exemplar/11/16/3/9/main.tex}
 \item A bag contain 24 balls of which $x$ balls are red, $2x$ are white and $3x$ are blue. A ball is selected at random, What is the probability that it is
\begin{enumerate}[label=\alph*)]
\item not red ?
\item white ?
\end{enumerate}
%\input{exemplar/10/13/3/41/main.tex}
If the letters of the word ASSASSINATION are arranged at random. Find the Probability that
\begin{enumerate}[label=(\alph*)]
\item Four $S's$ come consecutively in the word
\item Two  $I's$ and two $N's$ come together
\item All $A's$ are not coming together
\item No two $A's$ are coming together
\end{enumerate}
%\input{exemplar/11/16/3/14/main.tex}
	\item One urn contains two black balls (labelled B1 and B2) and one white ball. A
	second urn contains one black ball and two white balls (labelled W1 and W2).
	Suppose the following experiment is performed. One of the two urns is chosen
	at random. Next a ball is randomly chosen from the urn. Then a second ball is
	chosen at random from the same urn without replacing the first ball.
	
	\begin{enumerate}
	\item What is the probability that two black balls are chosen?
	
	\item What is the probability that two balls of opposite colour are chosen?
	\end{enumerate}
	\solution
	%\input{exemplar/11/16/3/12/main1.tex}
\end{enumerate}

\item In a certain lottery 10,000 tickets are sold and ten equal prizes are awarded. What is the probability of not getting a prize if you buy (a) one ticket (b) two tickets (c) 10 tickets ?	
\\
\solution
		%\begin{enumerate}[label=\thesection.\arabic*,ref=\thesection.\theenumi]
	\item One card is drawn from a well-shuffled deck of 52 cards. Find the probability of getting
\begin{enumerate}
\item A king of red colour 
\item A face card 
\item A red face card
\item The jack of hearts
\item A spade
\item The queen of diamonds

\end{enumerate}
\solution
		%\input{ncert/10/15/1/14/main.tex}
	\item Five cards—the ten, jack, queen, king and ace of diamonds, are well-shuffled with their face downwards. One card is then picked up at random.
\begin{enumerate}
\item
What is the probability that the card is the queen? 
\item
If the queen is drawn and put aside, what is the probability that the second card picked up is (a) an ace? (b) a queen?\\
\end{enumerate}
\solution
		%\input{ncert/10/15/1/15/defs.tex}
	\item A bag contains $5$ red balls and some blue balls. If the probability of drawing a blue ball is double that if a red ball, determine the number of blue balls in the bag. 
		\\
\solution
		%\input{ncert/10/15/2/3/defs.tex}
	\item A card is selected from a pack of 52 cards.
 \begin{enumerate}[label=(\alph*)] 
                 \item How many points are there in the sample space?
                 \item Calculate the probability that the card is an ace of spades.
                 \item Calculate the probability that the card is (i) an ace and (ii) black card.
 \end{enumerate}
\solution
		%\input{ncert/11/16/3/4/main.tex}
\item Four cards are drawn from a well-shuffled deck of 52 cards. What is the probability of obtaining 3 diamonds and one spade.
\\
\solution
		%\input{ncert/11/16/4/2/defs.tex}
\item In a certain lottery 10,000 tickets are sold and ten equal prizes are awarded. What is the probability of not getting a prize if you buy (a) one ticket (b) two tickets (c) 10 tickets ?	
\\
\solution
		%\input{ncert/11/16/4/4/defs.tex}
		%
\item 
Out of 100 students, two sections of 40 and 60 are formed. If you and your friend are among the 100 students, what is the probability that
\begin{enumerate}
\item you both enter the same section?
\item you both enter the different sections?
\end{enumerate}
\solution
		%\input{ncert/11/16/4/5/defs.tex}
	\item 
The number lock of a suitcase has 4 wheels each labelled with ten digits i.e. from 0 to 9.The lock opens with a sequence of four digits with no repeats.What is the probability of a person getting the right sequence to open the suitcase.
\\
\solution
		%\input{ncert/11/16/4/10/defs.tex}
		%
\item 
Two cards are drawn at random and without replacement from a pack of 52 playing cards. Find the probability that both the cards are black.
\\
\solution
		%\input{ncert/12/13/2/2/defs.tex}
		\item A box of oranges is inspected by examining three randomly selected oranges drawn without replacement. If all the three oranges are good, the box is approved for sale, otherwise, it is rejected. Find the probability that a box containing 15 oranges out of which 12 are good and 3 are bad ones will be approved for sale.
		\label{ncert/12/13/2/3/defs.tex}
		\item Two balls are drawn at random with replacement from a box containing 10 black and 8 red balls. Find the probability that
		\label{ncert/12/13/2/12}
\begin{enumerate}
\item both balls are red.
\item first ball is black and second is red.
\item one of them is black and other is red.
\end{enumerate}

\item In a hostel, 60\% of the students read Hindi newspaper, 40\% read English newspaper and 20\% read both Hindi and English newspapers. A student is selected at random.
		\label{ncert/12/13/2/15}
\begin{enumerate}
\item Find the probability that she reads neither Hindi nor English newspapers.
\item If she reads Hindi newspaper, find the probability that she reads English newspaper.
\item If she reads English newspaper, find the probability that she reads Hindi newspaper.\\
\end{enumerate}
\item The probability of obtaining an even prime number on each die, when a pair of dice is rolled is 
\begin{enumerate}
    \item $0$ 
    
    \item $\frac{1}{3}$ 
    
    \item $\frac{1}{12}$ 
    
    \item $\frac{1}{36}$ 
\end{enumerate}
\solution
		%\input{ncert/12/13/2/17/defs.tex}
	\item A bag contains 4 red and 4 black balls, another bag contains 2 red and 6 black balls. One of the two bags is selected at random and a ball is drawn from the bag which is found to be red. Find the probability that the ball is drawn from the first bag.
\\
\solution
		%\input{ncert/12/13/3/2/main.tex}
  \item
  Cards with numbers 2 to 101 are placed in a box. A card is selected at random.Find the probability that the card has
\begin{enumerate}[label=(\roman*)]
	\item an even number 
	\item a square number
\end{enumerate}
\solution
%\input{exemplar/10/13/3/32/main.tex}
\item
The king, queen and jack of clubs are removed from a deck of 52 playing cards and then well shuffled. Now one card is drawn at random from the remaining cards.  Determine the probability that the card is
\begin{enumerate}[label=(\roman*)]
\item a club
\item 10 of hearts
\end{enumerate}
\solution
%\input{exemplar/10/13/3/29/main.tex}
\item A team of medical students doing their internship have to assist during surgeries
at a city hospital. The probabilities of surgeries rated as very complex, complex,
routine, simple or very simple are respectively, 0.15, 0.20, 0.31, 0.26, .08. Find
the probabilities that a particular surgery will be rated
\begin{enumerate}
	\item complex or very complex;
	\item neither very complex nor very simple;
	\item routine or complex
	\item routine or simple
\end{enumerate}
\solution
%\input{exemplar/11/16/3/8(1)/main.tex}
\item A card is selected from a pack of 52 cards.
\begin{enumerate}[label=(\alph*)]
    \item How many points are there in the sample space?
    \item Calculate the probability that the card is an ace of spades.
    \item Calculate the probability that the card is (i) an ace and (ii) black card.
\end{enumerate}
\solution
%\input{exemplar/11/16/3/4/main2.tex}
\item The probability that a non leap year selected at random will contain 53 sundays.
\\
\solution
%\input{exemplar/10/13/1/19/main.tex}
\item One of the four persons John, Rita, Aslam or Gurpreet will be promoted next
month. Consequently the sample space consists of four elementary outcomes
S = {John promoted, Rita promoted, Aslam promoted, Gurpreet promoted}
You are told that the chances of John’s promotion is same as that of Gurpreet,
Rita’s chances of promotion are twice as likely as Johns. Aslam’s chances are
four times that of John.
\begin{enumerate}
	\item Determine
	\begin{enumerate}
		\item P (John promoted)
		\item P (Rita promoted)
		\item P (Aslam promoted)
		\item P (Gurpreet promoted)
	\end{enumerate}
	\item If A = {John promoted or Gurpreet promoted}, find P (A).
\end{enumerate}
\solution
%\input{exemplar/11/16/3/10/main.tex}
\item A card is drawn from a deck of 52 cards. Find the probability of getting a king or a heart or a red card.\\
\solution
%\input{exemplar/11/16/3/15/main.tex}
\item The probability that a student will pass his examination is 0.73, the probability of
the student getting a compartment is 0.13, and the probability that the student will
either pass or get compartment is 0.96. State True or False.\\
\solution
%\input{exemplar/11/16/3/31/main.tex}
\item A card is selected from a pack of 52 cards\\
\begin{enumerate}[label=(\alph*)]
\item How many points are there in the sample space?
\item Calculate the probability that the cards is an ace of spades.
\item Calculate the probability that the card is (i) an ace (ii)black card.\\
\end{enumerate}
%\input{ncert/11/16/3/4_1/Prob_4.tex}
\item In a non-leap year, the probability of having 53 tuesdays or 53 wednesdays is\\
\solution
%\input{exemplar/11/16/3/18/main.tex}
\item There are 1000 sealed envelopes in a box, 10 of them contain a cash prize of
Rs 100 each, 100 of them contain a cash prize of Rs 50 each and 200 of them
contain a cash prize of Rs 10 each and rest do not contain any cash prize. If they
are well shuffled and an envelope is picked up out, what is the probability that it
contains no cash prize?\\
\solution
%\input{exemplar/10/13/3/34/main.tex}
\item 
A die is thrown and a card is selected at random from a deck of 52 playing cards. The probability of getting an even number on the die and a spade card.\\
\solution
%\input{exemplar/12/13/3/78/main.tex}
\item
If 4-digit numbers greater than 5,000 are randomly formed from the digits 0, 1, 3, 5, and 7, what is the probability of forming a number divisible by 5 when:
\begin{enumerate}
    \item The digits are repeated?
    \item The repetition of digits is not allowed?
\end{enumerate}
\solution
%\input{ncert/11/16/4/9/main.tex}
\item Consider the probability space $\brak{\Omega, \mathcal{G}, P}$ where $\Omega = [0,2]$ and $\mathcal{G} = \cbrak{\phi, \Omega, [0,1], (1,2]}$. Let $X$ and $Y$ be two functions on $\Omega$ defined as
\begin{align*}
    X(\omega) = 
    \begin{cases}
        1 & \text{if }\omega \in [0, 1]\\
        2 & \text{if }\omega \in (1, 2]
    \end{cases}
\end{align*}
and
\begin{align*}
    Y(\omega) = 
    \begin{cases}
        2 & \text{if }\omega \in [0, 1.5]\\
        3 & \text{if }\omega \in (1.5, 2].
    \end{cases}
\end{align*}
Then which one of the following statements is true?
\begin{enumerate}
    \item [(A)] $X$ is a random variable with respect to $\mathcal{G}$, but $Y$ is not a random variable with respect to $\mathcal{G}$.
    \item [(B)] $Y$ is a random variable with respect to $\mathcal{G}$, but $X$ is not a random variable with respect to $\mathcal{G}$.
    \item [(C)] Neither $X$ nor $Y$ is a random variable with respect to $\mathcal{G}$.
    \item [(D)] Both $X$ and $Y$ are random variables with respect to $\mathcal{G}$.
\end{enumerate} \hfill (GATE ST 2023)\\
\solution
%\input{gate/ST/2023/14/main.tex}
	\item  A die is loaded in such a way that each odd number is twice as likely to occur as
each even number. Find $P(G)$, where $G$ is the event that a number greater than
3 occurs on a single roll of the die.
\\
\solution
		%\input{exemplar/11/16/3/5/main.tex}
	\item All the jacks, queens and kings are removed from a deck of 52 playing cards. The remaining cards are well shuffled and then one card is drawn at random. Giving ace a value 1 similar value for other cards, find the probability that the card has a value 
		\begin{enumerate}
			\item 7
			\item greater than 7
			\item less than 7
		\end{enumerate}
		%\input{exemplar/10/13/3/30/main.tex}
  \item A Lot consists of 48 mobile phones of which 42 are good, 3 have only minor defects and 3 have major defects.Varnika will buy a phone if it is good but the trader will only buy a mobile if it has no major defects. One phone is selected at random from the lot. What is the probability that it is
\begin{enumerate}
	\item acceptable to Varnika?
            \item acceptable to the trader?
\end{enumerate}
\solution
	%\input{exemplar/10/13/3/40/main.tex}
 \item A student says that if you throw a die, it will show up 1 or not 1. Therefore, the probability of getting 1 and the probability of getting 'not 1' each is equal to $\frac{1}{2}$. Is this correct? Give reasons.\\
 \solution
        %\input{exemplar/10/13/2/9/main.tex}
   \item Four candidates A, B, C, D have ap-
plied for the assignment to coach a school cricket
team. If A is twice as likely to be selected as B, and
B and C are given about the same chance of being
selected, while C is twice as likely to be selected
as D, what are the probabilities that
\begin{enumerate}
\item C will be selected?
\item A will not be selected?
\end{enumerate}
	%\input{exemplar/11/16/3/9/main.tex}
 \item A bag contain 24 balls of which $x$ balls are red, $2x$ are white and $3x$ are blue. A ball is selected at random, What is the probability that it is
\begin{enumerate}[label=\alph*)]
\item not red ?
\item white ?
\end{enumerate}
%\input{exemplar/10/13/3/41/main.tex}
If the letters of the word ASSASSINATION are arranged at random. Find the Probability that
\begin{enumerate}[label=(\alph*)]
\item Four $S's$ come consecutively in the word
\item Two  $I's$ and two $N's$ come together
\item All $A's$ are not coming together
\item No two $A's$ are coming together
\end{enumerate}
%\input{exemplar/11/16/3/14/main.tex}
	\item One urn contains two black balls (labelled B1 and B2) and one white ball. A
	second urn contains one black ball and two white balls (labelled W1 and W2).
	Suppose the following experiment is performed. One of the two urns is chosen
	at random. Next a ball is randomly chosen from the urn. Then a second ball is
	chosen at random from the same urn without replacing the first ball.
	
	\begin{enumerate}
	\item What is the probability that two black balls are chosen?
	
	\item What is the probability that two balls of opposite colour are chosen?
	\end{enumerate}
	\solution
	%\input{exemplar/11/16/3/12/main1.tex}
\end{enumerate}

		%
\item 
Out of 100 students, two sections of 40 and 60 are formed. If you and your friend are among the 100 students, what is the probability that
\begin{enumerate}
\item you both enter the same section?
\item you both enter the different sections?
\end{enumerate}
\solution
		%\begin{enumerate}[label=\thesection.\arabic*,ref=\thesection.\theenumi]
	\item One card is drawn from a well-shuffled deck of 52 cards. Find the probability of getting
\begin{enumerate}
\item A king of red colour 
\item A face card 
\item A red face card
\item The jack of hearts
\item A spade
\item The queen of diamonds

\end{enumerate}
\solution
		%\input{ncert/10/15/1/14/main.tex}
	\item Five cards—the ten, jack, queen, king and ace of diamonds, are well-shuffled with their face downwards. One card is then picked up at random.
\begin{enumerate}
\item
What is the probability that the card is the queen? 
\item
If the queen is drawn and put aside, what is the probability that the second card picked up is (a) an ace? (b) a queen?\\
\end{enumerate}
\solution
		%\input{ncert/10/15/1/15/defs.tex}
	\item A bag contains $5$ red balls and some blue balls. If the probability of drawing a blue ball is double that if a red ball, determine the number of blue balls in the bag. 
		\\
\solution
		%\input{ncert/10/15/2/3/defs.tex}
	\item A card is selected from a pack of 52 cards.
 \begin{enumerate}[label=(\alph*)] 
                 \item How many points are there in the sample space?
                 \item Calculate the probability that the card is an ace of spades.
                 \item Calculate the probability that the card is (i) an ace and (ii) black card.
 \end{enumerate}
\solution
		%\input{ncert/11/16/3/4/main.tex}
\item Four cards are drawn from a well-shuffled deck of 52 cards. What is the probability of obtaining 3 diamonds and one spade.
\\
\solution
		%\input{ncert/11/16/4/2/defs.tex}
\item In a certain lottery 10,000 tickets are sold and ten equal prizes are awarded. What is the probability of not getting a prize if you buy (a) one ticket (b) two tickets (c) 10 tickets ?	
\\
\solution
		%\input{ncert/11/16/4/4/defs.tex}
		%
\item 
Out of 100 students, two sections of 40 and 60 are formed. If you and your friend are among the 100 students, what is the probability that
\begin{enumerate}
\item you both enter the same section?
\item you both enter the different sections?
\end{enumerate}
\solution
		%\input{ncert/11/16/4/5/defs.tex}
	\item 
The number lock of a suitcase has 4 wheels each labelled with ten digits i.e. from 0 to 9.The lock opens with a sequence of four digits with no repeats.What is the probability of a person getting the right sequence to open the suitcase.
\\
\solution
		%\input{ncert/11/16/4/10/defs.tex}
		%
\item 
Two cards are drawn at random and without replacement from a pack of 52 playing cards. Find the probability that both the cards are black.
\\
\solution
		%\input{ncert/12/13/2/2/defs.tex}
		\item A box of oranges is inspected by examining three randomly selected oranges drawn without replacement. If all the three oranges are good, the box is approved for sale, otherwise, it is rejected. Find the probability that a box containing 15 oranges out of which 12 are good and 3 are bad ones will be approved for sale.
		\label{ncert/12/13/2/3/defs.tex}
		\item Two balls are drawn at random with replacement from a box containing 10 black and 8 red balls. Find the probability that
		\label{ncert/12/13/2/12}
\begin{enumerate}
\item both balls are red.
\item first ball is black and second is red.
\item one of them is black and other is red.
\end{enumerate}

\item In a hostel, 60\% of the students read Hindi newspaper, 40\% read English newspaper and 20\% read both Hindi and English newspapers. A student is selected at random.
		\label{ncert/12/13/2/15}
\begin{enumerate}
\item Find the probability that she reads neither Hindi nor English newspapers.
\item If she reads Hindi newspaper, find the probability that she reads English newspaper.
\item If she reads English newspaper, find the probability that she reads Hindi newspaper.\\
\end{enumerate}
\item The probability of obtaining an even prime number on each die, when a pair of dice is rolled is 
\begin{enumerate}
    \item $0$ 
    
    \item $\frac{1}{3}$ 
    
    \item $\frac{1}{12}$ 
    
    \item $\frac{1}{36}$ 
\end{enumerate}
\solution
		%\input{ncert/12/13/2/17/defs.tex}
	\item A bag contains 4 red and 4 black balls, another bag contains 2 red and 6 black balls. One of the two bags is selected at random and a ball is drawn from the bag which is found to be red. Find the probability that the ball is drawn from the first bag.
\\
\solution
		%\input{ncert/12/13/3/2/main.tex}
  \item
  Cards with numbers 2 to 101 are placed in a box. A card is selected at random.Find the probability that the card has
\begin{enumerate}[label=(\roman*)]
	\item an even number 
	\item a square number
\end{enumerate}
\solution
%\input{exemplar/10/13/3/32/main.tex}
\item
The king, queen and jack of clubs are removed from a deck of 52 playing cards and then well shuffled. Now one card is drawn at random from the remaining cards.  Determine the probability that the card is
\begin{enumerate}[label=(\roman*)]
\item a club
\item 10 of hearts
\end{enumerate}
\solution
%\input{exemplar/10/13/3/29/main.tex}
\item A team of medical students doing their internship have to assist during surgeries
at a city hospital. The probabilities of surgeries rated as very complex, complex,
routine, simple or very simple are respectively, 0.15, 0.20, 0.31, 0.26, .08. Find
the probabilities that a particular surgery will be rated
\begin{enumerate}
	\item complex or very complex;
	\item neither very complex nor very simple;
	\item routine or complex
	\item routine or simple
\end{enumerate}
\solution
%\input{exemplar/11/16/3/8(1)/main.tex}
\item A card is selected from a pack of 52 cards.
\begin{enumerate}[label=(\alph*)]
    \item How many points are there in the sample space?
    \item Calculate the probability that the card is an ace of spades.
    \item Calculate the probability that the card is (i) an ace and (ii) black card.
\end{enumerate}
\solution
%\input{exemplar/11/16/3/4/main2.tex}
\item The probability that a non leap year selected at random will contain 53 sundays.
\\
\solution
%\input{exemplar/10/13/1/19/main.tex}
\item One of the four persons John, Rita, Aslam or Gurpreet will be promoted next
month. Consequently the sample space consists of four elementary outcomes
S = {John promoted, Rita promoted, Aslam promoted, Gurpreet promoted}
You are told that the chances of John’s promotion is same as that of Gurpreet,
Rita’s chances of promotion are twice as likely as Johns. Aslam’s chances are
four times that of John.
\begin{enumerate}
	\item Determine
	\begin{enumerate}
		\item P (John promoted)
		\item P (Rita promoted)
		\item P (Aslam promoted)
		\item P (Gurpreet promoted)
	\end{enumerate}
	\item If A = {John promoted or Gurpreet promoted}, find P (A).
\end{enumerate}
\solution
%\input{exemplar/11/16/3/10/main.tex}
\item A card is drawn from a deck of 52 cards. Find the probability of getting a king or a heart or a red card.\\
\solution
%\input{exemplar/11/16/3/15/main.tex}
\item The probability that a student will pass his examination is 0.73, the probability of
the student getting a compartment is 0.13, and the probability that the student will
either pass or get compartment is 0.96. State True or False.\\
\solution
%\input{exemplar/11/16/3/31/main.tex}
\item A card is selected from a pack of 52 cards\\
\begin{enumerate}[label=(\alph*)]
\item How many points are there in the sample space?
\item Calculate the probability that the cards is an ace of spades.
\item Calculate the probability that the card is (i) an ace (ii)black card.\\
\end{enumerate}
%\input{ncert/11/16/3/4_1/Prob_4.tex}
\item In a non-leap year, the probability of having 53 tuesdays or 53 wednesdays is\\
\solution
%\input{exemplar/11/16/3/18/main.tex}
\item There are 1000 sealed envelopes in a box, 10 of them contain a cash prize of
Rs 100 each, 100 of them contain a cash prize of Rs 50 each and 200 of them
contain a cash prize of Rs 10 each and rest do not contain any cash prize. If they
are well shuffled and an envelope is picked up out, what is the probability that it
contains no cash prize?\\
\solution
%\input{exemplar/10/13/3/34/main.tex}
\item 
A die is thrown and a card is selected at random from a deck of 52 playing cards. The probability of getting an even number on the die and a spade card.\\
\solution
%\input{exemplar/12/13/3/78/main.tex}
\item
If 4-digit numbers greater than 5,000 are randomly formed from the digits 0, 1, 3, 5, and 7, what is the probability of forming a number divisible by 5 when:
\begin{enumerate}
    \item The digits are repeated?
    \item The repetition of digits is not allowed?
\end{enumerate}
\solution
%\input{ncert/11/16/4/9/main.tex}
\item Consider the probability space $\brak{\Omega, \mathcal{G}, P}$ where $\Omega = [0,2]$ and $\mathcal{G} = \cbrak{\phi, \Omega, [0,1], (1,2]}$. Let $X$ and $Y$ be two functions on $\Omega$ defined as
\begin{align*}
    X(\omega) = 
    \begin{cases}
        1 & \text{if }\omega \in [0, 1]\\
        2 & \text{if }\omega \in (1, 2]
    \end{cases}
\end{align*}
and
\begin{align*}
    Y(\omega) = 
    \begin{cases}
        2 & \text{if }\omega \in [0, 1.5]\\
        3 & \text{if }\omega \in (1.5, 2].
    \end{cases}
\end{align*}
Then which one of the following statements is true?
\begin{enumerate}
    \item [(A)] $X$ is a random variable with respect to $\mathcal{G}$, but $Y$ is not a random variable with respect to $\mathcal{G}$.
    \item [(B)] $Y$ is a random variable with respect to $\mathcal{G}$, but $X$ is not a random variable with respect to $\mathcal{G}$.
    \item [(C)] Neither $X$ nor $Y$ is a random variable with respect to $\mathcal{G}$.
    \item [(D)] Both $X$ and $Y$ are random variables with respect to $\mathcal{G}$.
\end{enumerate} \hfill (GATE ST 2023)\\
\solution
%\input{gate/ST/2023/14/main.tex}
	\item  A die is loaded in such a way that each odd number is twice as likely to occur as
each even number. Find $P(G)$, where $G$ is the event that a number greater than
3 occurs on a single roll of the die.
\\
\solution
		%\input{exemplar/11/16/3/5/main.tex}
	\item All the jacks, queens and kings are removed from a deck of 52 playing cards. The remaining cards are well shuffled and then one card is drawn at random. Giving ace a value 1 similar value for other cards, find the probability that the card has a value 
		\begin{enumerate}
			\item 7
			\item greater than 7
			\item less than 7
		\end{enumerate}
		%\input{exemplar/10/13/3/30/main.tex}
  \item A Lot consists of 48 mobile phones of which 42 are good, 3 have only minor defects and 3 have major defects.Varnika will buy a phone if it is good but the trader will only buy a mobile if it has no major defects. One phone is selected at random from the lot. What is the probability that it is
\begin{enumerate}
	\item acceptable to Varnika?
            \item acceptable to the trader?
\end{enumerate}
\solution
	%\input{exemplar/10/13/3/40/main.tex}
 \item A student says that if you throw a die, it will show up 1 or not 1. Therefore, the probability of getting 1 and the probability of getting 'not 1' each is equal to $\frac{1}{2}$. Is this correct? Give reasons.\\
 \solution
        %\input{exemplar/10/13/2/9/main.tex}
   \item Four candidates A, B, C, D have ap-
plied for the assignment to coach a school cricket
team. If A is twice as likely to be selected as B, and
B and C are given about the same chance of being
selected, while C is twice as likely to be selected
as D, what are the probabilities that
\begin{enumerate}
\item C will be selected?
\item A will not be selected?
\end{enumerate}
	%\input{exemplar/11/16/3/9/main.tex}
 \item A bag contain 24 balls of which $x$ balls are red, $2x$ are white and $3x$ are blue. A ball is selected at random, What is the probability that it is
\begin{enumerate}[label=\alph*)]
\item not red ?
\item white ?
\end{enumerate}
%\input{exemplar/10/13/3/41/main.tex}
If the letters of the word ASSASSINATION are arranged at random. Find the Probability that
\begin{enumerate}[label=(\alph*)]
\item Four $S's$ come consecutively in the word
\item Two  $I's$ and two $N's$ come together
\item All $A's$ are not coming together
\item No two $A's$ are coming together
\end{enumerate}
%\input{exemplar/11/16/3/14/main.tex}
	\item One urn contains two black balls (labelled B1 and B2) and one white ball. A
	second urn contains one black ball and two white balls (labelled W1 and W2).
	Suppose the following experiment is performed. One of the two urns is chosen
	at random. Next a ball is randomly chosen from the urn. Then a second ball is
	chosen at random from the same urn without replacing the first ball.
	
	\begin{enumerate}
	\item What is the probability that two black balls are chosen?
	
	\item What is the probability that two balls of opposite colour are chosen?
	\end{enumerate}
	\solution
	%\input{exemplar/11/16/3/12/main1.tex}
\end{enumerate}

	\item 
The number lock of a suitcase has 4 wheels each labelled with ten digits i.e. from 0 to 9.The lock opens with a sequence of four digits with no repeats.What is the probability of a person getting the right sequence to open the suitcase.
\\
\solution
		%\begin{enumerate}[label=\thesection.\arabic*,ref=\thesection.\theenumi]
	\item One card is drawn from a well-shuffled deck of 52 cards. Find the probability of getting
\begin{enumerate}
\item A king of red colour 
\item A face card 
\item A red face card
\item The jack of hearts
\item A spade
\item The queen of diamonds

\end{enumerate}
\solution
		%\input{ncert/10/15/1/14/main.tex}
	\item Five cards—the ten, jack, queen, king and ace of diamonds, are well-shuffled with their face downwards. One card is then picked up at random.
\begin{enumerate}
\item
What is the probability that the card is the queen? 
\item
If the queen is drawn and put aside, what is the probability that the second card picked up is (a) an ace? (b) a queen?\\
\end{enumerate}
\solution
		%\input{ncert/10/15/1/15/defs.tex}
	\item A bag contains $5$ red balls and some blue balls. If the probability of drawing a blue ball is double that if a red ball, determine the number of blue balls in the bag. 
		\\
\solution
		%\input{ncert/10/15/2/3/defs.tex}
	\item A card is selected from a pack of 52 cards.
 \begin{enumerate}[label=(\alph*)] 
                 \item How many points are there in the sample space?
                 \item Calculate the probability that the card is an ace of spades.
                 \item Calculate the probability that the card is (i) an ace and (ii) black card.
 \end{enumerate}
\solution
		%\input{ncert/11/16/3/4/main.tex}
\item Four cards are drawn from a well-shuffled deck of 52 cards. What is the probability of obtaining 3 diamonds and one spade.
\\
\solution
		%\input{ncert/11/16/4/2/defs.tex}
\item In a certain lottery 10,000 tickets are sold and ten equal prizes are awarded. What is the probability of not getting a prize if you buy (a) one ticket (b) two tickets (c) 10 tickets ?	
\\
\solution
		%\input{ncert/11/16/4/4/defs.tex}
		%
\item 
Out of 100 students, two sections of 40 and 60 are formed. If you and your friend are among the 100 students, what is the probability that
\begin{enumerate}
\item you both enter the same section?
\item you both enter the different sections?
\end{enumerate}
\solution
		%\input{ncert/11/16/4/5/defs.tex}
	\item 
The number lock of a suitcase has 4 wheels each labelled with ten digits i.e. from 0 to 9.The lock opens with a sequence of four digits with no repeats.What is the probability of a person getting the right sequence to open the suitcase.
\\
\solution
		%\input{ncert/11/16/4/10/defs.tex}
		%
\item 
Two cards are drawn at random and without replacement from a pack of 52 playing cards. Find the probability that both the cards are black.
\\
\solution
		%\input{ncert/12/13/2/2/defs.tex}
		\item A box of oranges is inspected by examining three randomly selected oranges drawn without replacement. If all the three oranges are good, the box is approved for sale, otherwise, it is rejected. Find the probability that a box containing 15 oranges out of which 12 are good and 3 are bad ones will be approved for sale.
		\label{ncert/12/13/2/3/defs.tex}
		\item Two balls are drawn at random with replacement from a box containing 10 black and 8 red balls. Find the probability that
		\label{ncert/12/13/2/12}
\begin{enumerate}
\item both balls are red.
\item first ball is black and second is red.
\item one of them is black and other is red.
\end{enumerate}

\item In a hostel, 60\% of the students read Hindi newspaper, 40\% read English newspaper and 20\% read both Hindi and English newspapers. A student is selected at random.
		\label{ncert/12/13/2/15}
\begin{enumerate}
\item Find the probability that she reads neither Hindi nor English newspapers.
\item If she reads Hindi newspaper, find the probability that she reads English newspaper.
\item If she reads English newspaper, find the probability that she reads Hindi newspaper.\\
\end{enumerate}
\item The probability of obtaining an even prime number on each die, when a pair of dice is rolled is 
\begin{enumerate}
    \item $0$ 
    
    \item $\frac{1}{3}$ 
    
    \item $\frac{1}{12}$ 
    
    \item $\frac{1}{36}$ 
\end{enumerate}
\solution
		%\input{ncert/12/13/2/17/defs.tex}
	\item A bag contains 4 red and 4 black balls, another bag contains 2 red and 6 black balls. One of the two bags is selected at random and a ball is drawn from the bag which is found to be red. Find the probability that the ball is drawn from the first bag.
\\
\solution
		%\input{ncert/12/13/3/2/main.tex}
  \item
  Cards with numbers 2 to 101 are placed in a box. A card is selected at random.Find the probability that the card has
\begin{enumerate}[label=(\roman*)]
	\item an even number 
	\item a square number
\end{enumerate}
\solution
%\input{exemplar/10/13/3/32/main.tex}
\item
The king, queen and jack of clubs are removed from a deck of 52 playing cards and then well shuffled. Now one card is drawn at random from the remaining cards.  Determine the probability that the card is
\begin{enumerate}[label=(\roman*)]
\item a club
\item 10 of hearts
\end{enumerate}
\solution
%\input{exemplar/10/13/3/29/main.tex}
\item A team of medical students doing their internship have to assist during surgeries
at a city hospital. The probabilities of surgeries rated as very complex, complex,
routine, simple or very simple are respectively, 0.15, 0.20, 0.31, 0.26, .08. Find
the probabilities that a particular surgery will be rated
\begin{enumerate}
	\item complex or very complex;
	\item neither very complex nor very simple;
	\item routine or complex
	\item routine or simple
\end{enumerate}
\solution
%\input{exemplar/11/16/3/8(1)/main.tex}
\item A card is selected from a pack of 52 cards.
\begin{enumerate}[label=(\alph*)]
    \item How many points are there in the sample space?
    \item Calculate the probability that the card is an ace of spades.
    \item Calculate the probability that the card is (i) an ace and (ii) black card.
\end{enumerate}
\solution
%\input{exemplar/11/16/3/4/main2.tex}
\item The probability that a non leap year selected at random will contain 53 sundays.
\\
\solution
%\input{exemplar/10/13/1/19/main.tex}
\item One of the four persons John, Rita, Aslam or Gurpreet will be promoted next
month. Consequently the sample space consists of four elementary outcomes
S = {John promoted, Rita promoted, Aslam promoted, Gurpreet promoted}
You are told that the chances of John’s promotion is same as that of Gurpreet,
Rita’s chances of promotion are twice as likely as Johns. Aslam’s chances are
four times that of John.
\begin{enumerate}
	\item Determine
	\begin{enumerate}
		\item P (John promoted)
		\item P (Rita promoted)
		\item P (Aslam promoted)
		\item P (Gurpreet promoted)
	\end{enumerate}
	\item If A = {John promoted or Gurpreet promoted}, find P (A).
\end{enumerate}
\solution
%\input{exemplar/11/16/3/10/main.tex}
\item A card is drawn from a deck of 52 cards. Find the probability of getting a king or a heart or a red card.\\
\solution
%\input{exemplar/11/16/3/15/main.tex}
\item The probability that a student will pass his examination is 0.73, the probability of
the student getting a compartment is 0.13, and the probability that the student will
either pass or get compartment is 0.96. State True or False.\\
\solution
%\input{exemplar/11/16/3/31/main.tex}
\item A card is selected from a pack of 52 cards\\
\begin{enumerate}[label=(\alph*)]
\item How many points are there in the sample space?
\item Calculate the probability that the cards is an ace of spades.
\item Calculate the probability that the card is (i) an ace (ii)black card.\\
\end{enumerate}
%\input{ncert/11/16/3/4_1/Prob_4.tex}
\item In a non-leap year, the probability of having 53 tuesdays or 53 wednesdays is\\
\solution
%\input{exemplar/11/16/3/18/main.tex}
\item There are 1000 sealed envelopes in a box, 10 of them contain a cash prize of
Rs 100 each, 100 of them contain a cash prize of Rs 50 each and 200 of them
contain a cash prize of Rs 10 each and rest do not contain any cash prize. If they
are well shuffled and an envelope is picked up out, what is the probability that it
contains no cash prize?\\
\solution
%\input{exemplar/10/13/3/34/main.tex}
\item 
A die is thrown and a card is selected at random from a deck of 52 playing cards. The probability of getting an even number on the die and a spade card.\\
\solution
%\input{exemplar/12/13/3/78/main.tex}
\item
If 4-digit numbers greater than 5,000 are randomly formed from the digits 0, 1, 3, 5, and 7, what is the probability of forming a number divisible by 5 when:
\begin{enumerate}
    \item The digits are repeated?
    \item The repetition of digits is not allowed?
\end{enumerate}
\solution
%\input{ncert/11/16/4/9/main.tex}
\item Consider the probability space $\brak{\Omega, \mathcal{G}, P}$ where $\Omega = [0,2]$ and $\mathcal{G} = \cbrak{\phi, \Omega, [0,1], (1,2]}$. Let $X$ and $Y$ be two functions on $\Omega$ defined as
\begin{align*}
    X(\omega) = 
    \begin{cases}
        1 & \text{if }\omega \in [0, 1]\\
        2 & \text{if }\omega \in (1, 2]
    \end{cases}
\end{align*}
and
\begin{align*}
    Y(\omega) = 
    \begin{cases}
        2 & \text{if }\omega \in [0, 1.5]\\
        3 & \text{if }\omega \in (1.5, 2].
    \end{cases}
\end{align*}
Then which one of the following statements is true?
\begin{enumerate}
    \item [(A)] $X$ is a random variable with respect to $\mathcal{G}$, but $Y$ is not a random variable with respect to $\mathcal{G}$.
    \item [(B)] $Y$ is a random variable with respect to $\mathcal{G}$, but $X$ is not a random variable with respect to $\mathcal{G}$.
    \item [(C)] Neither $X$ nor $Y$ is a random variable with respect to $\mathcal{G}$.
    \item [(D)] Both $X$ and $Y$ are random variables with respect to $\mathcal{G}$.
\end{enumerate} \hfill (GATE ST 2023)\\
\solution
%\input{gate/ST/2023/14/main.tex}
	\item  A die is loaded in such a way that each odd number is twice as likely to occur as
each even number. Find $P(G)$, where $G$ is the event that a number greater than
3 occurs on a single roll of the die.
\\
\solution
		%\input{exemplar/11/16/3/5/main.tex}
	\item All the jacks, queens and kings are removed from a deck of 52 playing cards. The remaining cards are well shuffled and then one card is drawn at random. Giving ace a value 1 similar value for other cards, find the probability that the card has a value 
		\begin{enumerate}
			\item 7
			\item greater than 7
			\item less than 7
		\end{enumerate}
		%\input{exemplar/10/13/3/30/main.tex}
  \item A Lot consists of 48 mobile phones of which 42 are good, 3 have only minor defects and 3 have major defects.Varnika will buy a phone if it is good but the trader will only buy a mobile if it has no major defects. One phone is selected at random from the lot. What is the probability that it is
\begin{enumerate}
	\item acceptable to Varnika?
            \item acceptable to the trader?
\end{enumerate}
\solution
	%\input{exemplar/10/13/3/40/main.tex}
 \item A student says that if you throw a die, it will show up 1 or not 1. Therefore, the probability of getting 1 and the probability of getting 'not 1' each is equal to $\frac{1}{2}$. Is this correct? Give reasons.\\
 \solution
        %\input{exemplar/10/13/2/9/main.tex}
   \item Four candidates A, B, C, D have ap-
plied for the assignment to coach a school cricket
team. If A is twice as likely to be selected as B, and
B and C are given about the same chance of being
selected, while C is twice as likely to be selected
as D, what are the probabilities that
\begin{enumerate}
\item C will be selected?
\item A will not be selected?
\end{enumerate}
	%\input{exemplar/11/16/3/9/main.tex}
 \item A bag contain 24 balls of which $x$ balls are red, $2x$ are white and $3x$ are blue. A ball is selected at random, What is the probability that it is
\begin{enumerate}[label=\alph*)]
\item not red ?
\item white ?
\end{enumerate}
%\input{exemplar/10/13/3/41/main.tex}
If the letters of the word ASSASSINATION are arranged at random. Find the Probability that
\begin{enumerate}[label=(\alph*)]
\item Four $S's$ come consecutively in the word
\item Two  $I's$ and two $N's$ come together
\item All $A's$ are not coming together
\item No two $A's$ are coming together
\end{enumerate}
%\input{exemplar/11/16/3/14/main.tex}
	\item One urn contains two black balls (labelled B1 and B2) and one white ball. A
	second urn contains one black ball and two white balls (labelled W1 and W2).
	Suppose the following experiment is performed. One of the two urns is chosen
	at random. Next a ball is randomly chosen from the urn. Then a second ball is
	chosen at random from the same urn without replacing the first ball.
	
	\begin{enumerate}
	\item What is the probability that two black balls are chosen?
	
	\item What is the probability that two balls of opposite colour are chosen?
	\end{enumerate}
	\solution
	%\input{exemplar/11/16/3/12/main1.tex}
\end{enumerate}

		%
\item 
Two cards are drawn at random and without replacement from a pack of 52 playing cards. Find the probability that both the cards are black.
\\
\solution
		%\begin{enumerate}[label=\thesection.\arabic*,ref=\thesection.\theenumi]
	\item One card is drawn from a well-shuffled deck of 52 cards. Find the probability of getting
\begin{enumerate}
\item A king of red colour 
\item A face card 
\item A red face card
\item The jack of hearts
\item A spade
\item The queen of diamonds

\end{enumerate}
\solution
		%\input{ncert/10/15/1/14/main.tex}
	\item Five cards—the ten, jack, queen, king and ace of diamonds, are well-shuffled with their face downwards. One card is then picked up at random.
\begin{enumerate}
\item
What is the probability that the card is the queen? 
\item
If the queen is drawn and put aside, what is the probability that the second card picked up is (a) an ace? (b) a queen?\\
\end{enumerate}
\solution
		%\input{ncert/10/15/1/15/defs.tex}
	\item A bag contains $5$ red balls and some blue balls. If the probability of drawing a blue ball is double that if a red ball, determine the number of blue balls in the bag. 
		\\
\solution
		%\input{ncert/10/15/2/3/defs.tex}
	\item A card is selected from a pack of 52 cards.
 \begin{enumerate}[label=(\alph*)] 
                 \item How many points are there in the sample space?
                 \item Calculate the probability that the card is an ace of spades.
                 \item Calculate the probability that the card is (i) an ace and (ii) black card.
 \end{enumerate}
\solution
		%\input{ncert/11/16/3/4/main.tex}
\item Four cards are drawn from a well-shuffled deck of 52 cards. What is the probability of obtaining 3 diamonds and one spade.
\\
\solution
		%\input{ncert/11/16/4/2/defs.tex}
\item In a certain lottery 10,000 tickets are sold and ten equal prizes are awarded. What is the probability of not getting a prize if you buy (a) one ticket (b) two tickets (c) 10 tickets ?	
\\
\solution
		%\input{ncert/11/16/4/4/defs.tex}
		%
\item 
Out of 100 students, two sections of 40 and 60 are formed. If you and your friend are among the 100 students, what is the probability that
\begin{enumerate}
\item you both enter the same section?
\item you both enter the different sections?
\end{enumerate}
\solution
		%\input{ncert/11/16/4/5/defs.tex}
	\item 
The number lock of a suitcase has 4 wheels each labelled with ten digits i.e. from 0 to 9.The lock opens with a sequence of four digits with no repeats.What is the probability of a person getting the right sequence to open the suitcase.
\\
\solution
		%\input{ncert/11/16/4/10/defs.tex}
		%
\item 
Two cards are drawn at random and without replacement from a pack of 52 playing cards. Find the probability that both the cards are black.
\\
\solution
		%\input{ncert/12/13/2/2/defs.tex}
		\item A box of oranges is inspected by examining three randomly selected oranges drawn without replacement. If all the three oranges are good, the box is approved for sale, otherwise, it is rejected. Find the probability that a box containing 15 oranges out of which 12 are good and 3 are bad ones will be approved for sale.
		\label{ncert/12/13/2/3/defs.tex}
		\item Two balls are drawn at random with replacement from a box containing 10 black and 8 red balls. Find the probability that
		\label{ncert/12/13/2/12}
\begin{enumerate}
\item both balls are red.
\item first ball is black and second is red.
\item one of them is black and other is red.
\end{enumerate}

\item In a hostel, 60\% of the students read Hindi newspaper, 40\% read English newspaper and 20\% read both Hindi and English newspapers. A student is selected at random.
		\label{ncert/12/13/2/15}
\begin{enumerate}
\item Find the probability that she reads neither Hindi nor English newspapers.
\item If she reads Hindi newspaper, find the probability that she reads English newspaper.
\item If she reads English newspaper, find the probability that she reads Hindi newspaper.\\
\end{enumerate}
\item The probability of obtaining an even prime number on each die, when a pair of dice is rolled is 
\begin{enumerate}
    \item $0$ 
    
    \item $\frac{1}{3}$ 
    
    \item $\frac{1}{12}$ 
    
    \item $\frac{1}{36}$ 
\end{enumerate}
\solution
		%\input{ncert/12/13/2/17/defs.tex}
	\item A bag contains 4 red and 4 black balls, another bag contains 2 red and 6 black balls. One of the two bags is selected at random and a ball is drawn from the bag which is found to be red. Find the probability that the ball is drawn from the first bag.
\\
\solution
		%\input{ncert/12/13/3/2/main.tex}
  \item
  Cards with numbers 2 to 101 are placed in a box. A card is selected at random.Find the probability that the card has
\begin{enumerate}[label=(\roman*)]
	\item an even number 
	\item a square number
\end{enumerate}
\solution
%\input{exemplar/10/13/3/32/main.tex}
\item
The king, queen and jack of clubs are removed from a deck of 52 playing cards and then well shuffled. Now one card is drawn at random from the remaining cards.  Determine the probability that the card is
\begin{enumerate}[label=(\roman*)]
\item a club
\item 10 of hearts
\end{enumerate}
\solution
%\input{exemplar/10/13/3/29/main.tex}
\item A team of medical students doing their internship have to assist during surgeries
at a city hospital. The probabilities of surgeries rated as very complex, complex,
routine, simple or very simple are respectively, 0.15, 0.20, 0.31, 0.26, .08. Find
the probabilities that a particular surgery will be rated
\begin{enumerate}
	\item complex or very complex;
	\item neither very complex nor very simple;
	\item routine or complex
	\item routine or simple
\end{enumerate}
\solution
%\input{exemplar/11/16/3/8(1)/main.tex}
\item A card is selected from a pack of 52 cards.
\begin{enumerate}[label=(\alph*)]
    \item How many points are there in the sample space?
    \item Calculate the probability that the card is an ace of spades.
    \item Calculate the probability that the card is (i) an ace and (ii) black card.
\end{enumerate}
\solution
%\input{exemplar/11/16/3/4/main2.tex}
\item The probability that a non leap year selected at random will contain 53 sundays.
\\
\solution
%\input{exemplar/10/13/1/19/main.tex}
\item One of the four persons John, Rita, Aslam or Gurpreet will be promoted next
month. Consequently the sample space consists of four elementary outcomes
S = {John promoted, Rita promoted, Aslam promoted, Gurpreet promoted}
You are told that the chances of John’s promotion is same as that of Gurpreet,
Rita’s chances of promotion are twice as likely as Johns. Aslam’s chances are
four times that of John.
\begin{enumerate}
	\item Determine
	\begin{enumerate}
		\item P (John promoted)
		\item P (Rita promoted)
		\item P (Aslam promoted)
		\item P (Gurpreet promoted)
	\end{enumerate}
	\item If A = {John promoted or Gurpreet promoted}, find P (A).
\end{enumerate}
\solution
%\input{exemplar/11/16/3/10/main.tex}
\item A card is drawn from a deck of 52 cards. Find the probability of getting a king or a heart or a red card.\\
\solution
%\input{exemplar/11/16/3/15/main.tex}
\item The probability that a student will pass his examination is 0.73, the probability of
the student getting a compartment is 0.13, and the probability that the student will
either pass or get compartment is 0.96. State True or False.\\
\solution
%\input{exemplar/11/16/3/31/main.tex}
\item A card is selected from a pack of 52 cards\\
\begin{enumerate}[label=(\alph*)]
\item How many points are there in the sample space?
\item Calculate the probability that the cards is an ace of spades.
\item Calculate the probability that the card is (i) an ace (ii)black card.\\
\end{enumerate}
%\input{ncert/11/16/3/4_1/Prob_4.tex}
\item In a non-leap year, the probability of having 53 tuesdays or 53 wednesdays is\\
\solution
%\input{exemplar/11/16/3/18/main.tex}
\item There are 1000 sealed envelopes in a box, 10 of them contain a cash prize of
Rs 100 each, 100 of them contain a cash prize of Rs 50 each and 200 of them
contain a cash prize of Rs 10 each and rest do not contain any cash prize. If they
are well shuffled and an envelope is picked up out, what is the probability that it
contains no cash prize?\\
\solution
%\input{exemplar/10/13/3/34/main.tex}
\item 
A die is thrown and a card is selected at random from a deck of 52 playing cards. The probability of getting an even number on the die and a spade card.\\
\solution
%\input{exemplar/12/13/3/78/main.tex}
\item
If 4-digit numbers greater than 5,000 are randomly formed from the digits 0, 1, 3, 5, and 7, what is the probability of forming a number divisible by 5 when:
\begin{enumerate}
    \item The digits are repeated?
    \item The repetition of digits is not allowed?
\end{enumerate}
\solution
%\input{ncert/11/16/4/9/main.tex}
\item Consider the probability space $\brak{\Omega, \mathcal{G}, P}$ where $\Omega = [0,2]$ and $\mathcal{G} = \cbrak{\phi, \Omega, [0,1], (1,2]}$. Let $X$ and $Y$ be two functions on $\Omega$ defined as
\begin{align*}
    X(\omega) = 
    \begin{cases}
        1 & \text{if }\omega \in [0, 1]\\
        2 & \text{if }\omega \in (1, 2]
    \end{cases}
\end{align*}
and
\begin{align*}
    Y(\omega) = 
    \begin{cases}
        2 & \text{if }\omega \in [0, 1.5]\\
        3 & \text{if }\omega \in (1.5, 2].
    \end{cases}
\end{align*}
Then which one of the following statements is true?
\begin{enumerate}
    \item [(A)] $X$ is a random variable with respect to $\mathcal{G}$, but $Y$ is not a random variable with respect to $\mathcal{G}$.
    \item [(B)] $Y$ is a random variable with respect to $\mathcal{G}$, but $X$ is not a random variable with respect to $\mathcal{G}$.
    \item [(C)] Neither $X$ nor $Y$ is a random variable with respect to $\mathcal{G}$.
    \item [(D)] Both $X$ and $Y$ are random variables with respect to $\mathcal{G}$.
\end{enumerate} \hfill (GATE ST 2023)\\
\solution
%\input{gate/ST/2023/14/main.tex}
	\item  A die is loaded in such a way that each odd number is twice as likely to occur as
each even number. Find $P(G)$, where $G$ is the event that a number greater than
3 occurs on a single roll of the die.
\\
\solution
		%\input{exemplar/11/16/3/5/main.tex}
	\item All the jacks, queens and kings are removed from a deck of 52 playing cards. The remaining cards are well shuffled and then one card is drawn at random. Giving ace a value 1 similar value for other cards, find the probability that the card has a value 
		\begin{enumerate}
			\item 7
			\item greater than 7
			\item less than 7
		\end{enumerate}
		%\input{exemplar/10/13/3/30/main.tex}
  \item A Lot consists of 48 mobile phones of which 42 are good, 3 have only minor defects and 3 have major defects.Varnika will buy a phone if it is good but the trader will only buy a mobile if it has no major defects. One phone is selected at random from the lot. What is the probability that it is
\begin{enumerate}
	\item acceptable to Varnika?
            \item acceptable to the trader?
\end{enumerate}
\solution
	%\input{exemplar/10/13/3/40/main.tex}
 \item A student says that if you throw a die, it will show up 1 or not 1. Therefore, the probability of getting 1 and the probability of getting 'not 1' each is equal to $\frac{1}{2}$. Is this correct? Give reasons.\\
 \solution
        %\input{exemplar/10/13/2/9/main.tex}
   \item Four candidates A, B, C, D have ap-
plied for the assignment to coach a school cricket
team. If A is twice as likely to be selected as B, and
B and C are given about the same chance of being
selected, while C is twice as likely to be selected
as D, what are the probabilities that
\begin{enumerate}
\item C will be selected?
\item A will not be selected?
\end{enumerate}
	%\input{exemplar/11/16/3/9/main.tex}
 \item A bag contain 24 balls of which $x$ balls are red, $2x$ are white and $3x$ are blue. A ball is selected at random, What is the probability that it is
\begin{enumerate}[label=\alph*)]
\item not red ?
\item white ?
\end{enumerate}
%\input{exemplar/10/13/3/41/main.tex}
If the letters of the word ASSASSINATION are arranged at random. Find the Probability that
\begin{enumerate}[label=(\alph*)]
\item Four $S's$ come consecutively in the word
\item Two  $I's$ and two $N's$ come together
\item All $A's$ are not coming together
\item No two $A's$ are coming together
\end{enumerate}
%\input{exemplar/11/16/3/14/main.tex}
	\item One urn contains two black balls (labelled B1 and B2) and one white ball. A
	second urn contains one black ball and two white balls (labelled W1 and W2).
	Suppose the following experiment is performed. One of the two urns is chosen
	at random. Next a ball is randomly chosen from the urn. Then a second ball is
	chosen at random from the same urn without replacing the first ball.
	
	\begin{enumerate}
	\item What is the probability that two black balls are chosen?
	
	\item What is the probability that two balls of opposite colour are chosen?
	\end{enumerate}
	\solution
	%\input{exemplar/11/16/3/12/main1.tex}
\end{enumerate}

		\item A box of oranges is inspected by examining three randomly selected oranges drawn without replacement. If all the three oranges are good, the box is approved for sale, otherwise, it is rejected. Find the probability that a box containing 15 oranges out of which 12 are good and 3 are bad ones will be approved for sale.
		\label{ncert/12/13/2/3/defs.tex}
		\item Two balls are drawn at random with replacement from a box containing 10 black and 8 red balls. Find the probability that
		\label{ncert/12/13/2/12}
\begin{enumerate}
\item both balls are red.
\item first ball is black and second is red.
\item one of them is black and other is red.
\end{enumerate}

\item In a hostel, 60\% of the students read Hindi newspaper, 40\% read English newspaper and 20\% read both Hindi and English newspapers. A student is selected at random.
		\label{ncert/12/13/2/15}
\begin{enumerate}
\item Find the probability that she reads neither Hindi nor English newspapers.
\item If she reads Hindi newspaper, find the probability that she reads English newspaper.
\item If she reads English newspaper, find the probability that she reads Hindi newspaper.\\
\end{enumerate}
\item The probability of obtaining an even prime number on each die, when a pair of dice is rolled is 
\begin{enumerate}
    \item $0$ 
    
    \item $\frac{1}{3}$ 
    
    \item $\frac{1}{12}$ 
    
    \item $\frac{1}{36}$ 
\end{enumerate}
\solution
		%\begin{enumerate}[label=\thesection.\arabic*,ref=\thesection.\theenumi]
	\item One card is drawn from a well-shuffled deck of 52 cards. Find the probability of getting
\begin{enumerate}
\item A king of red colour 
\item A face card 
\item A red face card
\item The jack of hearts
\item A spade
\item The queen of diamonds

\end{enumerate}
\solution
		%\input{ncert/10/15/1/14/main.tex}
	\item Five cards—the ten, jack, queen, king and ace of diamonds, are well-shuffled with their face downwards. One card is then picked up at random.
\begin{enumerate}
\item
What is the probability that the card is the queen? 
\item
If the queen is drawn and put aside, what is the probability that the second card picked up is (a) an ace? (b) a queen?\\
\end{enumerate}
\solution
		%\input{ncert/10/15/1/15/defs.tex}
	\item A bag contains $5$ red balls and some blue balls. If the probability of drawing a blue ball is double that if a red ball, determine the number of blue balls in the bag. 
		\\
\solution
		%\input{ncert/10/15/2/3/defs.tex}
	\item A card is selected from a pack of 52 cards.
 \begin{enumerate}[label=(\alph*)] 
                 \item How many points are there in the sample space?
                 \item Calculate the probability that the card is an ace of spades.
                 \item Calculate the probability that the card is (i) an ace and (ii) black card.
 \end{enumerate}
\solution
		%\input{ncert/11/16/3/4/main.tex}
\item Four cards are drawn from a well-shuffled deck of 52 cards. What is the probability of obtaining 3 diamonds and one spade.
\\
\solution
		%\input{ncert/11/16/4/2/defs.tex}
\item In a certain lottery 10,000 tickets are sold and ten equal prizes are awarded. What is the probability of not getting a prize if you buy (a) one ticket (b) two tickets (c) 10 tickets ?	
\\
\solution
		%\input{ncert/11/16/4/4/defs.tex}
		%
\item 
Out of 100 students, two sections of 40 and 60 are formed. If you and your friend are among the 100 students, what is the probability that
\begin{enumerate}
\item you both enter the same section?
\item you both enter the different sections?
\end{enumerate}
\solution
		%\input{ncert/11/16/4/5/defs.tex}
	\item 
The number lock of a suitcase has 4 wheels each labelled with ten digits i.e. from 0 to 9.The lock opens with a sequence of four digits with no repeats.What is the probability of a person getting the right sequence to open the suitcase.
\\
\solution
		%\input{ncert/11/16/4/10/defs.tex}
		%
\item 
Two cards are drawn at random and without replacement from a pack of 52 playing cards. Find the probability that both the cards are black.
\\
\solution
		%\input{ncert/12/13/2/2/defs.tex}
		\item A box of oranges is inspected by examining three randomly selected oranges drawn without replacement. If all the three oranges are good, the box is approved for sale, otherwise, it is rejected. Find the probability that a box containing 15 oranges out of which 12 are good and 3 are bad ones will be approved for sale.
		\label{ncert/12/13/2/3/defs.tex}
		\item Two balls are drawn at random with replacement from a box containing 10 black and 8 red balls. Find the probability that
		\label{ncert/12/13/2/12}
\begin{enumerate}
\item both balls are red.
\item first ball is black and second is red.
\item one of them is black and other is red.
\end{enumerate}

\item In a hostel, 60\% of the students read Hindi newspaper, 40\% read English newspaper and 20\% read both Hindi and English newspapers. A student is selected at random.
		\label{ncert/12/13/2/15}
\begin{enumerate}
\item Find the probability that she reads neither Hindi nor English newspapers.
\item If she reads Hindi newspaper, find the probability that she reads English newspaper.
\item If she reads English newspaper, find the probability that she reads Hindi newspaper.\\
\end{enumerate}
\item The probability of obtaining an even prime number on each die, when a pair of dice is rolled is 
\begin{enumerate}
    \item $0$ 
    
    \item $\frac{1}{3}$ 
    
    \item $\frac{1}{12}$ 
    
    \item $\frac{1}{36}$ 
\end{enumerate}
\solution
		%\input{ncert/12/13/2/17/defs.tex}
	\item A bag contains 4 red and 4 black balls, another bag contains 2 red and 6 black balls. One of the two bags is selected at random and a ball is drawn from the bag which is found to be red. Find the probability that the ball is drawn from the first bag.
\\
\solution
		%\input{ncert/12/13/3/2/main.tex}
  \item
  Cards with numbers 2 to 101 are placed in a box. A card is selected at random.Find the probability that the card has
\begin{enumerate}[label=(\roman*)]
	\item an even number 
	\item a square number
\end{enumerate}
\solution
%\input{exemplar/10/13/3/32/main.tex}
\item
The king, queen and jack of clubs are removed from a deck of 52 playing cards and then well shuffled. Now one card is drawn at random from the remaining cards.  Determine the probability that the card is
\begin{enumerate}[label=(\roman*)]
\item a club
\item 10 of hearts
\end{enumerate}
\solution
%\input{exemplar/10/13/3/29/main.tex}
\item A team of medical students doing their internship have to assist during surgeries
at a city hospital. The probabilities of surgeries rated as very complex, complex,
routine, simple or very simple are respectively, 0.15, 0.20, 0.31, 0.26, .08. Find
the probabilities that a particular surgery will be rated
\begin{enumerate}
	\item complex or very complex;
	\item neither very complex nor very simple;
	\item routine or complex
	\item routine or simple
\end{enumerate}
\solution
%\input{exemplar/11/16/3/8(1)/main.tex}
\item A card is selected from a pack of 52 cards.
\begin{enumerate}[label=(\alph*)]
    \item How many points are there in the sample space?
    \item Calculate the probability that the card is an ace of spades.
    \item Calculate the probability that the card is (i) an ace and (ii) black card.
\end{enumerate}
\solution
%\input{exemplar/11/16/3/4/main2.tex}
\item The probability that a non leap year selected at random will contain 53 sundays.
\\
\solution
%\input{exemplar/10/13/1/19/main.tex}
\item One of the four persons John, Rita, Aslam or Gurpreet will be promoted next
month. Consequently the sample space consists of four elementary outcomes
S = {John promoted, Rita promoted, Aslam promoted, Gurpreet promoted}
You are told that the chances of John’s promotion is same as that of Gurpreet,
Rita’s chances of promotion are twice as likely as Johns. Aslam’s chances are
four times that of John.
\begin{enumerate}
	\item Determine
	\begin{enumerate}
		\item P (John promoted)
		\item P (Rita promoted)
		\item P (Aslam promoted)
		\item P (Gurpreet promoted)
	\end{enumerate}
	\item If A = {John promoted or Gurpreet promoted}, find P (A).
\end{enumerate}
\solution
%\input{exemplar/11/16/3/10/main.tex}
\item A card is drawn from a deck of 52 cards. Find the probability of getting a king or a heart or a red card.\\
\solution
%\input{exemplar/11/16/3/15/main.tex}
\item The probability that a student will pass his examination is 0.73, the probability of
the student getting a compartment is 0.13, and the probability that the student will
either pass or get compartment is 0.96. State True or False.\\
\solution
%\input{exemplar/11/16/3/31/main.tex}
\item A card is selected from a pack of 52 cards\\
\begin{enumerate}[label=(\alph*)]
\item How many points are there in the sample space?
\item Calculate the probability that the cards is an ace of spades.
\item Calculate the probability that the card is (i) an ace (ii)black card.\\
\end{enumerate}
%\input{ncert/11/16/3/4_1/Prob_4.tex}
\item In a non-leap year, the probability of having 53 tuesdays or 53 wednesdays is\\
\solution
%\input{exemplar/11/16/3/18/main.tex}
\item There are 1000 sealed envelopes in a box, 10 of them contain a cash prize of
Rs 100 each, 100 of them contain a cash prize of Rs 50 each and 200 of them
contain a cash prize of Rs 10 each and rest do not contain any cash prize. If they
are well shuffled and an envelope is picked up out, what is the probability that it
contains no cash prize?\\
\solution
%\input{exemplar/10/13/3/34/main.tex}
\item 
A die is thrown and a card is selected at random from a deck of 52 playing cards. The probability of getting an even number on the die and a spade card.\\
\solution
%\input{exemplar/12/13/3/78/main.tex}
\item
If 4-digit numbers greater than 5,000 are randomly formed from the digits 0, 1, 3, 5, and 7, what is the probability of forming a number divisible by 5 when:
\begin{enumerate}
    \item The digits are repeated?
    \item The repetition of digits is not allowed?
\end{enumerate}
\solution
%\input{ncert/11/16/4/9/main.tex}
\item Consider the probability space $\brak{\Omega, \mathcal{G}, P}$ where $\Omega = [0,2]$ and $\mathcal{G} = \cbrak{\phi, \Omega, [0,1], (1,2]}$. Let $X$ and $Y$ be two functions on $\Omega$ defined as
\begin{align*}
    X(\omega) = 
    \begin{cases}
        1 & \text{if }\omega \in [0, 1]\\
        2 & \text{if }\omega \in (1, 2]
    \end{cases}
\end{align*}
and
\begin{align*}
    Y(\omega) = 
    \begin{cases}
        2 & \text{if }\omega \in [0, 1.5]\\
        3 & \text{if }\omega \in (1.5, 2].
    \end{cases}
\end{align*}
Then which one of the following statements is true?
\begin{enumerate}
    \item [(A)] $X$ is a random variable with respect to $\mathcal{G}$, but $Y$ is not a random variable with respect to $\mathcal{G}$.
    \item [(B)] $Y$ is a random variable with respect to $\mathcal{G}$, but $X$ is not a random variable with respect to $\mathcal{G}$.
    \item [(C)] Neither $X$ nor $Y$ is a random variable with respect to $\mathcal{G}$.
    \item [(D)] Both $X$ and $Y$ are random variables with respect to $\mathcal{G}$.
\end{enumerate} \hfill (GATE ST 2023)\\
\solution
%\input{gate/ST/2023/14/main.tex}
	\item  A die is loaded in such a way that each odd number is twice as likely to occur as
each even number. Find $P(G)$, where $G$ is the event that a number greater than
3 occurs on a single roll of the die.
\\
\solution
		%\input{exemplar/11/16/3/5/main.tex}
	\item All the jacks, queens and kings are removed from a deck of 52 playing cards. The remaining cards are well shuffled and then one card is drawn at random. Giving ace a value 1 similar value for other cards, find the probability that the card has a value 
		\begin{enumerate}
			\item 7
			\item greater than 7
			\item less than 7
		\end{enumerate}
		%\input{exemplar/10/13/3/30/main.tex}
  \item A Lot consists of 48 mobile phones of which 42 are good, 3 have only minor defects and 3 have major defects.Varnika will buy a phone if it is good but the trader will only buy a mobile if it has no major defects. One phone is selected at random from the lot. What is the probability that it is
\begin{enumerate}
	\item acceptable to Varnika?
            \item acceptable to the trader?
\end{enumerate}
\solution
	%\input{exemplar/10/13/3/40/main.tex}
 \item A student says that if you throw a die, it will show up 1 or not 1. Therefore, the probability of getting 1 and the probability of getting 'not 1' each is equal to $\frac{1}{2}$. Is this correct? Give reasons.\\
 \solution
        %\input{exemplar/10/13/2/9/main.tex}
   \item Four candidates A, B, C, D have ap-
plied for the assignment to coach a school cricket
team. If A is twice as likely to be selected as B, and
B and C are given about the same chance of being
selected, while C is twice as likely to be selected
as D, what are the probabilities that
\begin{enumerate}
\item C will be selected?
\item A will not be selected?
\end{enumerate}
	%\input{exemplar/11/16/3/9/main.tex}
 \item A bag contain 24 balls of which $x$ balls are red, $2x$ are white and $3x$ are blue. A ball is selected at random, What is the probability that it is
\begin{enumerate}[label=\alph*)]
\item not red ?
\item white ?
\end{enumerate}
%\input{exemplar/10/13/3/41/main.tex}
If the letters of the word ASSASSINATION are arranged at random. Find the Probability that
\begin{enumerate}[label=(\alph*)]
\item Four $S's$ come consecutively in the word
\item Two  $I's$ and two $N's$ come together
\item All $A's$ are not coming together
\item No two $A's$ are coming together
\end{enumerate}
%\input{exemplar/11/16/3/14/main.tex}
	\item One urn contains two black balls (labelled B1 and B2) and one white ball. A
	second urn contains one black ball and two white balls (labelled W1 and W2).
	Suppose the following experiment is performed. One of the two urns is chosen
	at random. Next a ball is randomly chosen from the urn. Then a second ball is
	chosen at random from the same urn without replacing the first ball.
	
	\begin{enumerate}
	\item What is the probability that two black balls are chosen?
	
	\item What is the probability that two balls of opposite colour are chosen?
	\end{enumerate}
	\solution
	%\input{exemplar/11/16/3/12/main1.tex}
\end{enumerate}

	\item A bag contains 4 red and 4 black balls, another bag contains 2 red and 6 black balls. One of the two bags is selected at random and a ball is drawn from the bag which is found to be red. Find the probability that the ball is drawn from the first bag.
\\
\solution
		%\begin{table}[H]
	\centering
\begin{tabular}{|c|c|c|}
\hline
Random variable &Value &Definition\\ \hline
\multirow{3}{*}{X} &0 &Slips of Rs 1\\
&1 &Slips of Rs 5\\
&2 &Slips of Rs 13\\ \hline
\multirow{2}{*}{Y} &0 &Box A\\
&1 &Box B\\\hline
\end{tabular}
\caption{}
\label{tab:Distribution}
\end{table}
See \tabref{tab:Distribution}.
\begin{align}
p_{Y}\brak{k}= \begin{cases} 
      \frac{1}{3} & {k=0} \\
      \frac{2}{3 }& {k=1} 
   \end{cases}
   \\
p_{Y|X}\brak{0|0} = \frac{19}{25}\, 
p_{Y|X}\brak{0|1} = \frac{6}{25}\,
p_{Y|X}\brak{1|0} = \frac{45}{50}\,
p_{Y|X}\brak{1|2} = \frac{5}{50}
\end{align}
The desired probability is the probability that a slip drawn at random is marked other than Rs 1,
\begin{align}
&=1-p_X\brak{0}\\
&= p_X(1) + p_X(2)
\end{align}
Using Bayes theorem,
\begin{align}
&= p_Y\brak{0} \times \pr{Y=0 | X=1} + p_Y\brak{1} \times \pr{Y=1|X=2}\\
&=\frac{1}{3} \times \frac{6}{25} + \frac{2}{3} \times \frac{5}{50}\\
&=\frac{11}{75}
\end{align}

\newpage

%\tableofcontents

\bigskip

\renewcommand{\thefigure}{\theenumi}
\renewcommand{\thetable}{\theenumi}
%\renewcommand{\theequation}{\theenumi}

%\begin{abstract}
%%\boldmath
%In this letter, an algorithm for evaluating the exact analytical bit error rate  (BER)  for the piecewise linear (PL) combiner for  multiple relays is presented. Previous results were available only for upto three relays. The algorithm is unique in the sense that  the actual mathematical expressions, that are prohibitively large, need not be explicitly obtained. The diversity gain due to multiple relays is shown through plots of the analytical BER, well supported by simulations. 
%
%\end{abstract}
% IEEEtran.cls defaults to using nonbold math in the Abstract.
% This preserves the distinction between vectors and scalars. However,
% if the journal you are submitting to favors bold math in the abstract,
% then you can use LaTeX's standard command \boldmath at the very start
% of the abstract to achieve this. Many IEEE journals frown on math
% in the abstract anyway.

% Note that keywords are not normally used for peerreview papers.
%\begin{IEEEkeywords}
%Cooperative diversity, decode and forward, piecewise linear
%\end{IEEEkeywords}



% For peer review papers, you can put extra information on the cover
% page as needed:
% \ifCLASSOPTIONpeerreview
% \begin{center} \bfseries EDICS Category: 3-BBND \end{center}
% \fi
%
% For peerreview papers, this IEEEtran command inserts a page break and
% creates the second title. It will be ignored for other modes.
%\IEEEpeerreviewmaketitle




  \item
  Cards with numbers 2 to 101 are placed in a box. A card is selected at random.Find the probability that the card has
\begin{enumerate}[label=(\roman*)]
	\item an even number 
	\item a square number
\end{enumerate}
\solution
%\begin{table}[H]
	\centering
\begin{tabular}{|c|c|c|}
\hline
Random variable &Value &Definition\\ \hline
\multirow{3}{*}{X} &0 &Slips of Rs 1\\
&1 &Slips of Rs 5\\
&2 &Slips of Rs 13\\ \hline
\multirow{2}{*}{Y} &0 &Box A\\
&1 &Box B\\\hline
\end{tabular}
\caption{}
\label{tab:Distribution}
\end{table}
See \tabref{tab:Distribution}.
\begin{align}
p_{Y}\brak{k}= \begin{cases} 
      \frac{1}{3} & {k=0} \\
      \frac{2}{3 }& {k=1} 
   \end{cases}
   \\
p_{Y|X}\brak{0|0} = \frac{19}{25}\, 
p_{Y|X}\brak{0|1} = \frac{6}{25}\,
p_{Y|X}\brak{1|0} = \frac{45}{50}\,
p_{Y|X}\brak{1|2} = \frac{5}{50}
\end{align}
The desired probability is the probability that a slip drawn at random is marked other than Rs 1,
\begin{align}
&=1-p_X\brak{0}\\
&= p_X(1) + p_X(2)
\end{align}
Using Bayes theorem,
\begin{align}
&= p_Y\brak{0} \times \pr{Y=0 | X=1} + p_Y\brak{1} \times \pr{Y=1|X=2}\\
&=\frac{1}{3} \times \frac{6}{25} + \frac{2}{3} \times \frac{5}{50}\\
&=\frac{11}{75}
\end{align}

\newpage

%\tableofcontents

\bigskip

\renewcommand{\thefigure}{\theenumi}
\renewcommand{\thetable}{\theenumi}
%\renewcommand{\theequation}{\theenumi}

%\begin{abstract}
%%\boldmath
%In this letter, an algorithm for evaluating the exact analytical bit error rate  (BER)  for the piecewise linear (PL) combiner for  multiple relays is presented. Previous results were available only for upto three relays. The algorithm is unique in the sense that  the actual mathematical expressions, that are prohibitively large, need not be explicitly obtained. The diversity gain due to multiple relays is shown through plots of the analytical BER, well supported by simulations. 
%
%\end{abstract}
% IEEEtran.cls defaults to using nonbold math in the Abstract.
% This preserves the distinction between vectors and scalars. However,
% if the journal you are submitting to favors bold math in the abstract,
% then you can use LaTeX's standard command \boldmath at the very start
% of the abstract to achieve this. Many IEEE journals frown on math
% in the abstract anyway.

% Note that keywords are not normally used for peerreview papers.
%\begin{IEEEkeywords}
%Cooperative diversity, decode and forward, piecewise linear
%\end{IEEEkeywords}



% For peer review papers, you can put extra information on the cover
% page as needed:
% \ifCLASSOPTIONpeerreview
% \begin{center} \bfseries EDICS Category: 3-BBND \end{center}
% \fi
%
% For peerreview papers, this IEEEtran command inserts a page break and
% creates the second title. It will be ignored for other modes.
%\IEEEpeerreviewmaketitle




\item
The king, queen and jack of clubs are removed from a deck of 52 playing cards and then well shuffled. Now one card is drawn at random from the remaining cards.  Determine the probability that the card is
\begin{enumerate}[label=(\roman*)]
\item a club
\item 10 of hearts
\end{enumerate}
\solution
%\begin{table}[H]
	\centering
\begin{tabular}{|c|c|c|}
\hline
Random variable &Value &Definition\\ \hline
\multirow{3}{*}{X} &0 &Slips of Rs 1\\
&1 &Slips of Rs 5\\
&2 &Slips of Rs 13\\ \hline
\multirow{2}{*}{Y} &0 &Box A\\
&1 &Box B\\\hline
\end{tabular}
\caption{}
\label{tab:Distribution}
\end{table}
See \tabref{tab:Distribution}.
\begin{align}
p_{Y}\brak{k}= \begin{cases} 
      \frac{1}{3} & {k=0} \\
      \frac{2}{3 }& {k=1} 
   \end{cases}
   \\
p_{Y|X}\brak{0|0} = \frac{19}{25}\, 
p_{Y|X}\brak{0|1} = \frac{6}{25}\,
p_{Y|X}\brak{1|0} = \frac{45}{50}\,
p_{Y|X}\brak{1|2} = \frac{5}{50}
\end{align}
The desired probability is the probability that a slip drawn at random is marked other than Rs 1,
\begin{align}
&=1-p_X\brak{0}\\
&= p_X(1) + p_X(2)
\end{align}
Using Bayes theorem,
\begin{align}
&= p_Y\brak{0} \times \pr{Y=0 | X=1} + p_Y\brak{1} \times \pr{Y=1|X=2}\\
&=\frac{1}{3} \times \frac{6}{25} + \frac{2}{3} \times \frac{5}{50}\\
&=\frac{11}{75}
\end{align}

\newpage

%\tableofcontents

\bigskip

\renewcommand{\thefigure}{\theenumi}
\renewcommand{\thetable}{\theenumi}
%\renewcommand{\theequation}{\theenumi}

%\begin{abstract}
%%\boldmath
%In this letter, an algorithm for evaluating the exact analytical bit error rate  (BER)  for the piecewise linear (PL) combiner for  multiple relays is presented. Previous results were available only for upto three relays. The algorithm is unique in the sense that  the actual mathematical expressions, that are prohibitively large, need not be explicitly obtained. The diversity gain due to multiple relays is shown through plots of the analytical BER, well supported by simulations. 
%
%\end{abstract}
% IEEEtran.cls defaults to using nonbold math in the Abstract.
% This preserves the distinction between vectors and scalars. However,
% if the journal you are submitting to favors bold math in the abstract,
% then you can use LaTeX's standard command \boldmath at the very start
% of the abstract to achieve this. Many IEEE journals frown on math
% in the abstract anyway.

% Note that keywords are not normally used for peerreview papers.
%\begin{IEEEkeywords}
%Cooperative diversity, decode and forward, piecewise linear
%\end{IEEEkeywords}



% For peer review papers, you can put extra information on the cover
% page as needed:
% \ifCLASSOPTIONpeerreview
% \begin{center} \bfseries EDICS Category: 3-BBND \end{center}
% \fi
%
% For peerreview papers, this IEEEtran command inserts a page break and
% creates the second title. It will be ignored for other modes.
%\IEEEpeerreviewmaketitle




\item A team of medical students doing their internship have to assist during surgeries
at a city hospital. The probabilities of surgeries rated as very complex, complex,
routine, simple or very simple are respectively, 0.15, 0.20, 0.31, 0.26, .08. Find
the probabilities that a particular surgery will be rated
\begin{enumerate}
	\item complex or very complex;
	\item neither very complex nor very simple;
	\item routine or complex
	\item routine or simple
\end{enumerate}
\solution
%\begin{table}[H]
	\centering
\begin{tabular}{|c|c|c|}
\hline
Random variable &Value &Definition\\ \hline
\multirow{3}{*}{X} &0 &Slips of Rs 1\\
&1 &Slips of Rs 5\\
&2 &Slips of Rs 13\\ \hline
\multirow{2}{*}{Y} &0 &Box A\\
&1 &Box B\\\hline
\end{tabular}
\caption{}
\label{tab:Distribution}
\end{table}
See \tabref{tab:Distribution}.
\begin{align}
p_{Y}\brak{k}= \begin{cases} 
      \frac{1}{3} & {k=0} \\
      \frac{2}{3 }& {k=1} 
   \end{cases}
   \\
p_{Y|X}\brak{0|0} = \frac{19}{25}\, 
p_{Y|X}\brak{0|1} = \frac{6}{25}\,
p_{Y|X}\brak{1|0} = \frac{45}{50}\,
p_{Y|X}\brak{1|2} = \frac{5}{50}
\end{align}
The desired probability is the probability that a slip drawn at random is marked other than Rs 1,
\begin{align}
&=1-p_X\brak{0}\\
&= p_X(1) + p_X(2)
\end{align}
Using Bayes theorem,
\begin{align}
&= p_Y\brak{0} \times \pr{Y=0 | X=1} + p_Y\brak{1} \times \pr{Y=1|X=2}\\
&=\frac{1}{3} \times \frac{6}{25} + \frac{2}{3} \times \frac{5}{50}\\
&=\frac{11}{75}
\end{align}

\newpage

%\tableofcontents

\bigskip

\renewcommand{\thefigure}{\theenumi}
\renewcommand{\thetable}{\theenumi}
%\renewcommand{\theequation}{\theenumi}

%\begin{abstract}
%%\boldmath
%In this letter, an algorithm for evaluating the exact analytical bit error rate  (BER)  for the piecewise linear (PL) combiner for  multiple relays is presented. Previous results were available only for upto three relays. The algorithm is unique in the sense that  the actual mathematical expressions, that are prohibitively large, need not be explicitly obtained. The diversity gain due to multiple relays is shown through plots of the analytical BER, well supported by simulations. 
%
%\end{abstract}
% IEEEtran.cls defaults to using nonbold math in the Abstract.
% This preserves the distinction between vectors and scalars. However,
% if the journal you are submitting to favors bold math in the abstract,
% then you can use LaTeX's standard command \boldmath at the very start
% of the abstract to achieve this. Many IEEE journals frown on math
% in the abstract anyway.

% Note that keywords are not normally used for peerreview papers.
%\begin{IEEEkeywords}
%Cooperative diversity, decode and forward, piecewise linear
%\end{IEEEkeywords}



% For peer review papers, you can put extra information on the cover
% page as needed:
% \ifCLASSOPTIONpeerreview
% \begin{center} \bfseries EDICS Category: 3-BBND \end{center}
% \fi
%
% For peerreview papers, this IEEEtran command inserts a page break and
% creates the second title. It will be ignored for other modes.
%\IEEEpeerreviewmaketitle




\item A card is selected from a pack of 52 cards.
\begin{enumerate}[label=(\alph*)]
    \item How many points are there in the sample space?
    \item Calculate the probability that the card is an ace of spades.
    \item Calculate the probability that the card is (i) an ace and (ii) black card.
\end{enumerate}
\solution
%Let $X$ be an bernoulli rv defined as in \tabref{tab:exemplar/11/16/3/26}.  Then, 
\begin{equation}
    p =
        \frac{4}{11} 
\end{equation}
\begin{table}[H]
	\centering
	\input{exemplar/11/16/3/26/tables/Table2.tex}
	\caption{}
        \label{tab:exemplar/11/16/3/26}
\end{table}

\item The probability that a non leap year selected at random will contain 53 sundays.
\\
\solution
%\begin{table}[H]
	\centering
\begin{tabular}{|c|c|c|}
\hline
Random variable &Value &Definition\\ \hline
\multirow{3}{*}{X} &0 &Slips of Rs 1\\
&1 &Slips of Rs 5\\
&2 &Slips of Rs 13\\ \hline
\multirow{2}{*}{Y} &0 &Box A\\
&1 &Box B\\\hline
\end{tabular}
\caption{}
\label{tab:Distribution}
\end{table}
See \tabref{tab:Distribution}.
\begin{align}
p_{Y}\brak{k}= \begin{cases} 
      \frac{1}{3} & {k=0} \\
      \frac{2}{3 }& {k=1} 
   \end{cases}
   \\
p_{Y|X}\brak{0|0} = \frac{19}{25}\, 
p_{Y|X}\brak{0|1} = \frac{6}{25}\,
p_{Y|X}\brak{1|0} = \frac{45}{50}\,
p_{Y|X}\brak{1|2} = \frac{5}{50}
\end{align}
The desired probability is the probability that a slip drawn at random is marked other than Rs 1,
\begin{align}
&=1-p_X\brak{0}\\
&= p_X(1) + p_X(2)
\end{align}
Using Bayes theorem,
\begin{align}
&= p_Y\brak{0} \times \pr{Y=0 | X=1} + p_Y\brak{1} \times \pr{Y=1|X=2}\\
&=\frac{1}{3} \times \frac{6}{25} + \frac{2}{3} \times \frac{5}{50}\\
&=\frac{11}{75}
\end{align}

\newpage

%\tableofcontents

\bigskip

\renewcommand{\thefigure}{\theenumi}
\renewcommand{\thetable}{\theenumi}
%\renewcommand{\theequation}{\theenumi}

%\begin{abstract}
%%\boldmath
%In this letter, an algorithm for evaluating the exact analytical bit error rate  (BER)  for the piecewise linear (PL) combiner for  multiple relays is presented. Previous results were available only for upto three relays. The algorithm is unique in the sense that  the actual mathematical expressions, that are prohibitively large, need not be explicitly obtained. The diversity gain due to multiple relays is shown through plots of the analytical BER, well supported by simulations. 
%
%\end{abstract}
% IEEEtran.cls defaults to using nonbold math in the Abstract.
% This preserves the distinction between vectors and scalars. However,
% if the journal you are submitting to favors bold math in the abstract,
% then you can use LaTeX's standard command \boldmath at the very start
% of the abstract to achieve this. Many IEEE journals frown on math
% in the abstract anyway.

% Note that keywords are not normally used for peerreview papers.
%\begin{IEEEkeywords}
%Cooperative diversity, decode and forward, piecewise linear
%\end{IEEEkeywords}



% For peer review papers, you can put extra information on the cover
% page as needed:
% \ifCLASSOPTIONpeerreview
% \begin{center} \bfseries EDICS Category: 3-BBND \end{center}
% \fi
%
% For peerreview papers, this IEEEtran command inserts a page break and
% creates the second title. It will be ignored for other modes.
%\IEEEpeerreviewmaketitle




\item One of the four persons John, Rita, Aslam or Gurpreet will be promoted next
month. Consequently the sample space consists of four elementary outcomes
S = {John promoted, Rita promoted, Aslam promoted, Gurpreet promoted}
You are told that the chances of John’s promotion is same as that of Gurpreet,
Rita’s chances of promotion are twice as likely as Johns. Aslam’s chances are
four times that of John.
\begin{enumerate}
	\item Determine
	\begin{enumerate}
		\item P (John promoted)
		\item P (Rita promoted)
		\item P (Aslam promoted)
		\item P (Gurpreet promoted)
	\end{enumerate}
	\item If A = {John promoted or Gurpreet promoted}, find P (A).
\end{enumerate}
\solution
%\begin{table}[H]
	\centering
\begin{tabular}{|c|c|c|}
\hline
Random variable &Value &Definition\\ \hline
\multirow{3}{*}{X} &0 &Slips of Rs 1\\
&1 &Slips of Rs 5\\
&2 &Slips of Rs 13\\ \hline
\multirow{2}{*}{Y} &0 &Box A\\
&1 &Box B\\\hline
\end{tabular}
\caption{}
\label{tab:Distribution}
\end{table}
See \tabref{tab:Distribution}.
\begin{align}
p_{Y}\brak{k}= \begin{cases} 
      \frac{1}{3} & {k=0} \\
      \frac{2}{3 }& {k=1} 
   \end{cases}
   \\
p_{Y|X}\brak{0|0} = \frac{19}{25}\, 
p_{Y|X}\brak{0|1} = \frac{6}{25}\,
p_{Y|X}\brak{1|0} = \frac{45}{50}\,
p_{Y|X}\brak{1|2} = \frac{5}{50}
\end{align}
The desired probability is the probability that a slip drawn at random is marked other than Rs 1,
\begin{align}
&=1-p_X\brak{0}\\
&= p_X(1) + p_X(2)
\end{align}
Using Bayes theorem,
\begin{align}
&= p_Y\brak{0} \times \pr{Y=0 | X=1} + p_Y\brak{1} \times \pr{Y=1|X=2}\\
&=\frac{1}{3} \times \frac{6}{25} + \frac{2}{3} \times \frac{5}{50}\\
&=\frac{11}{75}
\end{align}

\newpage

%\tableofcontents

\bigskip

\renewcommand{\thefigure}{\theenumi}
\renewcommand{\thetable}{\theenumi}
%\renewcommand{\theequation}{\theenumi}

%\begin{abstract}
%%\boldmath
%In this letter, an algorithm for evaluating the exact analytical bit error rate  (BER)  for the piecewise linear (PL) combiner for  multiple relays is presented. Previous results were available only for upto three relays. The algorithm is unique in the sense that  the actual mathematical expressions, that are prohibitively large, need not be explicitly obtained. The diversity gain due to multiple relays is shown through plots of the analytical BER, well supported by simulations. 
%
%\end{abstract}
% IEEEtran.cls defaults to using nonbold math in the Abstract.
% This preserves the distinction between vectors and scalars. However,
% if the journal you are submitting to favors bold math in the abstract,
% then you can use LaTeX's standard command \boldmath at the very start
% of the abstract to achieve this. Many IEEE journals frown on math
% in the abstract anyway.

% Note that keywords are not normally used for peerreview papers.
%\begin{IEEEkeywords}
%Cooperative diversity, decode and forward, piecewise linear
%\end{IEEEkeywords}



% For peer review papers, you can put extra information on the cover
% page as needed:
% \ifCLASSOPTIONpeerreview
% \begin{center} \bfseries EDICS Category: 3-BBND \end{center}
% \fi
%
% For peerreview papers, this IEEEtran command inserts a page break and
% creates the second title. It will be ignored for other modes.
%\IEEEpeerreviewmaketitle




\item A card is drawn from a deck of 52 cards. Find the probability of getting a king or a heart or a red card.\\
\solution
%\begin{table}[H]
	\centering
\begin{tabular}{|c|c|c|}
\hline
Random variable &Value &Definition\\ \hline
\multirow{3}{*}{X} &0 &Slips of Rs 1\\
&1 &Slips of Rs 5\\
&2 &Slips of Rs 13\\ \hline
\multirow{2}{*}{Y} &0 &Box A\\
&1 &Box B\\\hline
\end{tabular}
\caption{}
\label{tab:Distribution}
\end{table}
See \tabref{tab:Distribution}.
\begin{align}
p_{Y}\brak{k}= \begin{cases} 
      \frac{1}{3} & {k=0} \\
      \frac{2}{3 }& {k=1} 
   \end{cases}
   \\
p_{Y|X}\brak{0|0} = \frac{19}{25}\, 
p_{Y|X}\brak{0|1} = \frac{6}{25}\,
p_{Y|X}\brak{1|0} = \frac{45}{50}\,
p_{Y|X}\brak{1|2} = \frac{5}{50}
\end{align}
The desired probability is the probability that a slip drawn at random is marked other than Rs 1,
\begin{align}
&=1-p_X\brak{0}\\
&= p_X(1) + p_X(2)
\end{align}
Using Bayes theorem,
\begin{align}
&= p_Y\brak{0} \times \pr{Y=0 | X=1} + p_Y\brak{1} \times \pr{Y=1|X=2}\\
&=\frac{1}{3} \times \frac{6}{25} + \frac{2}{3} \times \frac{5}{50}\\
&=\frac{11}{75}
\end{align}

\newpage

%\tableofcontents

\bigskip

\renewcommand{\thefigure}{\theenumi}
\renewcommand{\thetable}{\theenumi}
%\renewcommand{\theequation}{\theenumi}

%\begin{abstract}
%%\boldmath
%In this letter, an algorithm for evaluating the exact analytical bit error rate  (BER)  for the piecewise linear (PL) combiner for  multiple relays is presented. Previous results were available only for upto three relays. The algorithm is unique in the sense that  the actual mathematical expressions, that are prohibitively large, need not be explicitly obtained. The diversity gain due to multiple relays is shown through plots of the analytical BER, well supported by simulations. 
%
%\end{abstract}
% IEEEtran.cls defaults to using nonbold math in the Abstract.
% This preserves the distinction between vectors and scalars. However,
% if the journal you are submitting to favors bold math in the abstract,
% then you can use LaTeX's standard command \boldmath at the very start
% of the abstract to achieve this. Many IEEE journals frown on math
% in the abstract anyway.

% Note that keywords are not normally used for peerreview papers.
%\begin{IEEEkeywords}
%Cooperative diversity, decode and forward, piecewise linear
%\end{IEEEkeywords}



% For peer review papers, you can put extra information on the cover
% page as needed:
% \ifCLASSOPTIONpeerreview
% \begin{center} \bfseries EDICS Category: 3-BBND \end{center}
% \fi
%
% For peerreview papers, this IEEEtran command inserts a page break and
% creates the second title. It will be ignored for other modes.
%\IEEEpeerreviewmaketitle




\item The probability that a student will pass his examination is 0.73, the probability of
the student getting a compartment is 0.13, and the probability that the student will
either pass or get compartment is 0.96. State True or False.\\
\solution
%\begin{table}[H]
	\centering
\begin{tabular}{|c|c|c|}
\hline
Random variable &Value &Definition\\ \hline
\multirow{3}{*}{X} &0 &Slips of Rs 1\\
&1 &Slips of Rs 5\\
&2 &Slips of Rs 13\\ \hline
\multirow{2}{*}{Y} &0 &Box A\\
&1 &Box B\\\hline
\end{tabular}
\caption{}
\label{tab:Distribution}
\end{table}
See \tabref{tab:Distribution}.
\begin{align}
p_{Y}\brak{k}= \begin{cases} 
      \frac{1}{3} & {k=0} \\
      \frac{2}{3 }& {k=1} 
   \end{cases}
   \\
p_{Y|X}\brak{0|0} = \frac{19}{25}\, 
p_{Y|X}\brak{0|1} = \frac{6}{25}\,
p_{Y|X}\brak{1|0} = \frac{45}{50}\,
p_{Y|X}\brak{1|2} = \frac{5}{50}
\end{align}
The desired probability is the probability that a slip drawn at random is marked other than Rs 1,
\begin{align}
&=1-p_X\brak{0}\\
&= p_X(1) + p_X(2)
\end{align}
Using Bayes theorem,
\begin{align}
&= p_Y\brak{0} \times \pr{Y=0 | X=1} + p_Y\brak{1} \times \pr{Y=1|X=2}\\
&=\frac{1}{3} \times \frac{6}{25} + \frac{2}{3} \times \frac{5}{50}\\
&=\frac{11}{75}
\end{align}

\newpage

%\tableofcontents

\bigskip

\renewcommand{\thefigure}{\theenumi}
\renewcommand{\thetable}{\theenumi}
%\renewcommand{\theequation}{\theenumi}

%\begin{abstract}
%%\boldmath
%In this letter, an algorithm for evaluating the exact analytical bit error rate  (BER)  for the piecewise linear (PL) combiner for  multiple relays is presented. Previous results were available only for upto three relays. The algorithm is unique in the sense that  the actual mathematical expressions, that are prohibitively large, need not be explicitly obtained. The diversity gain due to multiple relays is shown through plots of the analytical BER, well supported by simulations. 
%
%\end{abstract}
% IEEEtran.cls defaults to using nonbold math in the Abstract.
% This preserves the distinction between vectors and scalars. However,
% if the journal you are submitting to favors bold math in the abstract,
% then you can use LaTeX's standard command \boldmath at the very start
% of the abstract to achieve this. Many IEEE journals frown on math
% in the abstract anyway.

% Note that keywords are not normally used for peerreview papers.
%\begin{IEEEkeywords}
%Cooperative diversity, decode and forward, piecewise linear
%\end{IEEEkeywords}



% For peer review papers, you can put extra information on the cover
% page as needed:
% \ifCLASSOPTIONpeerreview
% \begin{center} \bfseries EDICS Category: 3-BBND \end{center}
% \fi
%
% For peerreview papers, this IEEEtran command inserts a page break and
% creates the second title. It will be ignored for other modes.
%\IEEEpeerreviewmaketitle




\item A card is selected from a pack of 52 cards\\
\begin{enumerate}[label=(\alph*)]
\item How many points are there in the sample space?
\item Calculate the probability that the cards is an ace of spades.
\item Calculate the probability that the card is (i) an ace (ii)black card.\\
\end{enumerate}
%\input{ncert/11/16/3/4_1/Prob_4.tex}
\item In a non-leap year, the probability of having 53 tuesdays or 53 wednesdays is\\
\solution
%A non-leap year has a total of 365 days, and a week has 7 days.\\
So it can be expressed as 
\begin{align}
365\text{days} &=52\times 7+1 \text{day}
\end{align}
$\implies$ 52 tuesdays or wednesdays\\
Random variable X denotes the days of a week
\begin{align}
p_X\brak{k}&=\frac{1}{7}; \quad \brak{1<k<7}
\end{align}
So the probability of extra day being tuesday or wednesday is
\begin{align}
p_X\brak{3}+p_X\brak{4}&=\frac{1}{7}+\frac{1}{7}=\frac{2}{7}
\end{align}



\item There are 1000 sealed envelopes in a box, 10 of them contain a cash prize of
Rs 100 each, 100 of them contain a cash prize of Rs 50 each and 200 of them
contain a cash prize of Rs 10 each and rest do not contain any cash prize. If they
are well shuffled and an envelope is picked up out, what is the probability that it
contains no cash prize?\\
\solution
%\begin{table}[H]
	\centering
\begin{tabular}{|c|c|c|}
\hline
Random variable &Value &Definition\\ \hline
\multirow{3}{*}{X} &0 &Slips of Rs 1\\
&1 &Slips of Rs 5\\
&2 &Slips of Rs 13\\ \hline
\multirow{2}{*}{Y} &0 &Box A\\
&1 &Box B\\\hline
\end{tabular}
\caption{}
\label{tab:Distribution}
\end{table}
See \tabref{tab:Distribution}.
\begin{align}
p_{Y}\brak{k}= \begin{cases} 
      \frac{1}{3} & {k=0} \\
      \frac{2}{3 }& {k=1} 
   \end{cases}
   \\
p_{Y|X}\brak{0|0} = \frac{19}{25}\, 
p_{Y|X}\brak{0|1} = \frac{6}{25}\,
p_{Y|X}\brak{1|0} = \frac{45}{50}\,
p_{Y|X}\brak{1|2} = \frac{5}{50}
\end{align}
The desired probability is the probability that a slip drawn at random is marked other than Rs 1,
\begin{align}
&=1-p_X\brak{0}\\
&= p_X(1) + p_X(2)
\end{align}
Using Bayes theorem,
\begin{align}
&= p_Y\brak{0} \times \pr{Y=0 | X=1} + p_Y\brak{1} \times \pr{Y=1|X=2}\\
&=\frac{1}{3} \times \frac{6}{25} + \frac{2}{3} \times \frac{5}{50}\\
&=\frac{11}{75}
\end{align}

\newpage

%\tableofcontents

\bigskip

\renewcommand{\thefigure}{\theenumi}
\renewcommand{\thetable}{\theenumi}
%\renewcommand{\theequation}{\theenumi}

%\begin{abstract}
%%\boldmath
%In this letter, an algorithm for evaluating the exact analytical bit error rate  (BER)  for the piecewise linear (PL) combiner for  multiple relays is presented. Previous results were available only for upto three relays. The algorithm is unique in the sense that  the actual mathematical expressions, that are prohibitively large, need not be explicitly obtained. The diversity gain due to multiple relays is shown through plots of the analytical BER, well supported by simulations. 
%
%\end{abstract}
% IEEEtran.cls defaults to using nonbold math in the Abstract.
% This preserves the distinction between vectors and scalars. However,
% if the journal you are submitting to favors bold math in the abstract,
% then you can use LaTeX's standard command \boldmath at the very start
% of the abstract to achieve this. Many IEEE journals frown on math
% in the abstract anyway.

% Note that keywords are not normally used for peerreview papers.
%\begin{IEEEkeywords}
%Cooperative diversity, decode and forward, piecewise linear
%\end{IEEEkeywords}



% For peer review papers, you can put extra information on the cover
% page as needed:
% \ifCLASSOPTIONpeerreview
% \begin{center} \bfseries EDICS Category: 3-BBND \end{center}
% \fi
%
% For peerreview papers, this IEEEtran command inserts a page break and
% creates the second title. It will be ignored for other modes.
%\IEEEpeerreviewmaketitle




\item 
A die is thrown and a card is selected at random from a deck of 52 playing cards. The probability of getting an even number on the die and a spade card.\\
\solution
%\begin{table}[H]
	\centering
\begin{tabular}{|c|c|c|}
\hline
Random variable &Value &Definition\\ \hline
\multirow{3}{*}{X} &0 &Slips of Rs 1\\
&1 &Slips of Rs 5\\
&2 &Slips of Rs 13\\ \hline
\multirow{2}{*}{Y} &0 &Box A\\
&1 &Box B\\\hline
\end{tabular}
\caption{}
\label{tab:Distribution}
\end{table}
See \tabref{tab:Distribution}.
\begin{align}
p_{Y}\brak{k}= \begin{cases} 
      \frac{1}{3} & {k=0} \\
      \frac{2}{3 }& {k=1} 
   \end{cases}
   \\
p_{Y|X}\brak{0|0} = \frac{19}{25}\, 
p_{Y|X}\brak{0|1} = \frac{6}{25}\,
p_{Y|X}\brak{1|0} = \frac{45}{50}\,
p_{Y|X}\brak{1|2} = \frac{5}{50}
\end{align}
The desired probability is the probability that a slip drawn at random is marked other than Rs 1,
\begin{align}
&=1-p_X\brak{0}\\
&= p_X(1) + p_X(2)
\end{align}
Using Bayes theorem,
\begin{align}
&= p_Y\brak{0} \times \pr{Y=0 | X=1} + p_Y\brak{1} \times \pr{Y=1|X=2}\\
&=\frac{1}{3} \times \frac{6}{25} + \frac{2}{3} \times \frac{5}{50}\\
&=\frac{11}{75}
\end{align}

\newpage

%\tableofcontents

\bigskip

\renewcommand{\thefigure}{\theenumi}
\renewcommand{\thetable}{\theenumi}
%\renewcommand{\theequation}{\theenumi}

%\begin{abstract}
%%\boldmath
%In this letter, an algorithm for evaluating the exact analytical bit error rate  (BER)  for the piecewise linear (PL) combiner for  multiple relays is presented. Previous results were available only for upto three relays. The algorithm is unique in the sense that  the actual mathematical expressions, that are prohibitively large, need not be explicitly obtained. The diversity gain due to multiple relays is shown through plots of the analytical BER, well supported by simulations. 
%
%\end{abstract}
% IEEEtran.cls defaults to using nonbold math in the Abstract.
% This preserves the distinction between vectors and scalars. However,
% if the journal you are submitting to favors bold math in the abstract,
% then you can use LaTeX's standard command \boldmath at the very start
% of the abstract to achieve this. Many IEEE journals frown on math
% in the abstract anyway.

% Note that keywords are not normally used for peerreview papers.
%\begin{IEEEkeywords}
%Cooperative diversity, decode and forward, piecewise linear
%\end{IEEEkeywords}



% For peer review papers, you can put extra information on the cover
% page as needed:
% \ifCLASSOPTIONpeerreview
% \begin{center} \bfseries EDICS Category: 3-BBND \end{center}
% \fi
%
% For peerreview papers, this IEEEtran command inserts a page break and
% creates the second title. It will be ignored for other modes.
%\IEEEpeerreviewmaketitle




\item
If 4-digit numbers greater than 5,000 are randomly formed from the digits 0, 1, 3, 5, and 7, what is the probability of forming a number divisible by 5 when:
\begin{enumerate}
    \item The digits are repeated?
    \item The repetition of digits is not allowed?
\end{enumerate}
\solution
%\begin{table}[H]
	\centering
\begin{tabular}{|c|c|c|}
\hline
Random variable &Value &Definition\\ \hline
\multirow{3}{*}{X} &0 &Slips of Rs 1\\
&1 &Slips of Rs 5\\
&2 &Slips of Rs 13\\ \hline
\multirow{2}{*}{Y} &0 &Box A\\
&1 &Box B\\\hline
\end{tabular}
\caption{}
\label{tab:Distribution}
\end{table}
See \tabref{tab:Distribution}.
\begin{align}
p_{Y}\brak{k}= \begin{cases} 
      \frac{1}{3} & {k=0} \\
      \frac{2}{3 }& {k=1} 
   \end{cases}
   \\
p_{Y|X}\brak{0|0} = \frac{19}{25}\, 
p_{Y|X}\brak{0|1} = \frac{6}{25}\,
p_{Y|X}\brak{1|0} = \frac{45}{50}\,
p_{Y|X}\brak{1|2} = \frac{5}{50}
\end{align}
The desired probability is the probability that a slip drawn at random is marked other than Rs 1,
\begin{align}
&=1-p_X\brak{0}\\
&= p_X(1) + p_X(2)
\end{align}
Using Bayes theorem,
\begin{align}
&= p_Y\brak{0} \times \pr{Y=0 | X=1} + p_Y\brak{1} \times \pr{Y=1|X=2}\\
&=\frac{1}{3} \times \frac{6}{25} + \frac{2}{3} \times \frac{5}{50}\\
&=\frac{11}{75}
\end{align}

\newpage

%\tableofcontents

\bigskip

\renewcommand{\thefigure}{\theenumi}
\renewcommand{\thetable}{\theenumi}
%\renewcommand{\theequation}{\theenumi}

%\begin{abstract}
%%\boldmath
%In this letter, an algorithm for evaluating the exact analytical bit error rate  (BER)  for the piecewise linear (PL) combiner for  multiple relays is presented. Previous results were available only for upto three relays. The algorithm is unique in the sense that  the actual mathematical expressions, that are prohibitively large, need not be explicitly obtained. The diversity gain due to multiple relays is shown through plots of the analytical BER, well supported by simulations. 
%
%\end{abstract}
% IEEEtran.cls defaults to using nonbold math in the Abstract.
% This preserves the distinction between vectors and scalars. However,
% if the journal you are submitting to favors bold math in the abstract,
% then you can use LaTeX's standard command \boldmath at the very start
% of the abstract to achieve this. Many IEEE journals frown on math
% in the abstract anyway.

% Note that keywords are not normally used for peerreview papers.
%\begin{IEEEkeywords}
%Cooperative diversity, decode and forward, piecewise linear
%\end{IEEEkeywords}



% For peer review papers, you can put extra information on the cover
% page as needed:
% \ifCLASSOPTIONpeerreview
% \begin{center} \bfseries EDICS Category: 3-BBND \end{center}
% \fi
%
% For peerreview papers, this IEEEtran command inserts a page break and
% creates the second title. It will be ignored for other modes.
%\IEEEpeerreviewmaketitle




\item Consider the probability space $\brak{\Omega, \mathcal{G}, P}$ where $\Omega = [0,2]$ and $\mathcal{G} = \cbrak{\phi, \Omega, [0,1], (1,2]}$. Let $X$ and $Y$ be two functions on $\Omega$ defined as
\begin{align*}
    X(\omega) = 
    \begin{cases}
        1 & \text{if }\omega \in [0, 1]\\
        2 & \text{if }\omega \in (1, 2]
    \end{cases}
\end{align*}
and
\begin{align*}
    Y(\omega) = 
    \begin{cases}
        2 & \text{if }\omega \in [0, 1.5]\\
        3 & \text{if }\omega \in (1.5, 2].
    \end{cases}
\end{align*}
Then which one of the following statements is true?
\begin{enumerate}
    \item [(A)] $X$ is a random variable with respect to $\mathcal{G}$, but $Y$ is not a random variable with respect to $\mathcal{G}$.
    \item [(B)] $Y$ is a random variable with respect to $\mathcal{G}$, but $X$ is not a random variable with respect to $\mathcal{G}$.
    \item [(C)] Neither $X$ nor $Y$ is a random variable with respect to $\mathcal{G}$.
    \item [(D)] Both $X$ and $Y$ are random variables with respect to $\mathcal{G}$.
\end{enumerate} \hfill (GATE ST 2023)\\
\solution
%\begin{table}[H]
	\centering
\begin{tabular}{|c|c|c|}
\hline
Random variable &Value &Definition\\ \hline
\multirow{3}{*}{X} &0 &Slips of Rs 1\\
&1 &Slips of Rs 5\\
&2 &Slips of Rs 13\\ \hline
\multirow{2}{*}{Y} &0 &Box A\\
&1 &Box B\\\hline
\end{tabular}
\caption{}
\label{tab:Distribution}
\end{table}
See \tabref{tab:Distribution}.
\begin{align}
p_{Y}\brak{k}= \begin{cases} 
      \frac{1}{3} & {k=0} \\
      \frac{2}{3 }& {k=1} 
   \end{cases}
   \\
p_{Y|X}\brak{0|0} = \frac{19}{25}\, 
p_{Y|X}\brak{0|1} = \frac{6}{25}\,
p_{Y|X}\brak{1|0} = \frac{45}{50}\,
p_{Y|X}\brak{1|2} = \frac{5}{50}
\end{align}
The desired probability is the probability that a slip drawn at random is marked other than Rs 1,
\begin{align}
&=1-p_X\brak{0}\\
&= p_X(1) + p_X(2)
\end{align}
Using Bayes theorem,
\begin{align}
&= p_Y\brak{0} \times \pr{Y=0 | X=1} + p_Y\brak{1} \times \pr{Y=1|X=2}\\
&=\frac{1}{3} \times \frac{6}{25} + \frac{2}{3} \times \frac{5}{50}\\
&=\frac{11}{75}
\end{align}

\newpage

%\tableofcontents

\bigskip

\renewcommand{\thefigure}{\theenumi}
\renewcommand{\thetable}{\theenumi}
%\renewcommand{\theequation}{\theenumi}

%\begin{abstract}
%%\boldmath
%In this letter, an algorithm for evaluating the exact analytical bit error rate  (BER)  for the piecewise linear (PL) combiner for  multiple relays is presented. Previous results were available only for upto three relays. The algorithm is unique in the sense that  the actual mathematical expressions, that are prohibitively large, need not be explicitly obtained. The diversity gain due to multiple relays is shown through plots of the analytical BER, well supported by simulations. 
%
%\end{abstract}
% IEEEtran.cls defaults to using nonbold math in the Abstract.
% This preserves the distinction between vectors and scalars. However,
% if the journal you are submitting to favors bold math in the abstract,
% then you can use LaTeX's standard command \boldmath at the very start
% of the abstract to achieve this. Many IEEE journals frown on math
% in the abstract anyway.

% Note that keywords are not normally used for peerreview papers.
%\begin{IEEEkeywords}
%Cooperative diversity, decode and forward, piecewise linear
%\end{IEEEkeywords}



% For peer review papers, you can put extra information on the cover
% page as needed:
% \ifCLASSOPTIONpeerreview
% \begin{center} \bfseries EDICS Category: 3-BBND \end{center}
% \fi
%
% For peerreview papers, this IEEEtran command inserts a page break and
% creates the second title. It will be ignored for other modes.
%\IEEEpeerreviewmaketitle




	\item  A die is loaded in such a way that each odd number is twice as likely to occur as
each even number. Find $P(G)$, where $G$ is the event that a number greater than
3 occurs on a single roll of the die.
\\
\solution
		%\begin{table}[H]
	\centering
\begin{tabular}{|c|c|c|}
\hline
Random variable &Value &Definition\\ \hline
\multirow{3}{*}{X} &0 &Slips of Rs 1\\
&1 &Slips of Rs 5\\
&2 &Slips of Rs 13\\ \hline
\multirow{2}{*}{Y} &0 &Box A\\
&1 &Box B\\\hline
\end{tabular}
\caption{}
\label{tab:Distribution}
\end{table}
See \tabref{tab:Distribution}.
\begin{align}
p_{Y}\brak{k}= \begin{cases} 
      \frac{1}{3} & {k=0} \\
      \frac{2}{3 }& {k=1} 
   \end{cases}
   \\
p_{Y|X}\brak{0|0} = \frac{19}{25}\, 
p_{Y|X}\brak{0|1} = \frac{6}{25}\,
p_{Y|X}\brak{1|0} = \frac{45}{50}\,
p_{Y|X}\brak{1|2} = \frac{5}{50}
\end{align}
The desired probability is the probability that a slip drawn at random is marked other than Rs 1,
\begin{align}
&=1-p_X\brak{0}\\
&= p_X(1) + p_X(2)
\end{align}
Using Bayes theorem,
\begin{align}
&= p_Y\brak{0} \times \pr{Y=0 | X=1} + p_Y\brak{1} \times \pr{Y=1|X=2}\\
&=\frac{1}{3} \times \frac{6}{25} + \frac{2}{3} \times \frac{5}{50}\\
&=\frac{11}{75}
\end{align}

\newpage

%\tableofcontents

\bigskip

\renewcommand{\thefigure}{\theenumi}
\renewcommand{\thetable}{\theenumi}
%\renewcommand{\theequation}{\theenumi}

%\begin{abstract}
%%\boldmath
%In this letter, an algorithm for evaluating the exact analytical bit error rate  (BER)  for the piecewise linear (PL) combiner for  multiple relays is presented. Previous results were available only for upto three relays. The algorithm is unique in the sense that  the actual mathematical expressions, that are prohibitively large, need not be explicitly obtained. The diversity gain due to multiple relays is shown through plots of the analytical BER, well supported by simulations. 
%
%\end{abstract}
% IEEEtran.cls defaults to using nonbold math in the Abstract.
% This preserves the distinction between vectors and scalars. However,
% if the journal you are submitting to favors bold math in the abstract,
% then you can use LaTeX's standard command \boldmath at the very start
% of the abstract to achieve this. Many IEEE journals frown on math
% in the abstract anyway.

% Note that keywords are not normally used for peerreview papers.
%\begin{IEEEkeywords}
%Cooperative diversity, decode and forward, piecewise linear
%\end{IEEEkeywords}



% For peer review papers, you can put extra information on the cover
% page as needed:
% \ifCLASSOPTIONpeerreview
% \begin{center} \bfseries EDICS Category: 3-BBND \end{center}
% \fi
%
% For peerreview papers, this IEEEtran command inserts a page break and
% creates the second title. It will be ignored for other modes.
%\IEEEpeerreviewmaketitle




	\item All the jacks, queens and kings are removed from a deck of 52 playing cards. The remaining cards are well shuffled and then one card is drawn at random. Giving ace a value 1 similar value for other cards, find the probability that the card has a value 
		\begin{enumerate}
			\item 7
			\item greater than 7
			\item less than 7
		\end{enumerate}
		%Number of cards left after removing all jacks, queens and kings 
\begin{align}
N	= 52 - 4\times 3
	= 40
\end{align}
%\begin{table}[H]
%\def\arraystretch{1.2}
%\begin{tabular}{|c|c|c|}
%\hline
%	\textbf{Parameter} &\textbf{Value} &\textbf{Description}\\ \hline
%	$X$ &1-10 &Represents the value of the card picked \\ \hline
%\end{tabular}
%\end{table}
Let $1 \le X \le 10$ be the value of the card picked.  Then,
\begin{align}
	p_X(k) &= \Pr(X=k)\ \forall\ 1 \leq k \leq 10\\
	&= \frac{4\times 1}{40}\\
	&= \frac{1}{10}\\
	\therefore p_X(k) &= 
	\begin{cases}
		\frac{1}{10} & 1 \leq k \leq 10\\
		0 & \text{otherwise}
	\end{cases}
\end{align}
and
\begin{align}
	F_{X}(k) &= \sum_{m=0}^{k}p_{X}(m) \quad 1 \leq k \leq 10\\
	&= \frac{k}{10}\\
	\therefore F_{X}(k) &= 
	\begin{cases}
		0 & k \leq 0\\
		\frac{k}{10} & 1\leq k \leq 10\\
		1 & k > 10 
	\end{cases}
\end{align}
\begin{enumerate}
	\item Probability that card has value equal to 7 is
		\begin{align}
			 p_{X}(7)
			= \frac{1}{10}
		\end{align}
	\item Probability that card has value greater than 7 is
		\begin{align}
			1 - F_X(7)
			&= 1 - \frac{7}{10}
			\\
			&= \frac{3}{10}
		\end{align}
	\item Probability that card has value less than 7 is
		\begin{align}
			 F_{X}(6)
			=\frac{6}{10}
		\end{align}
\end{enumerate}

  \item A Lot consists of 48 mobile phones of which 42 are good, 3 have only minor defects and 3 have major defects.Varnika will buy a phone if it is good but the trader will only buy a mobile if it has no major defects. One phone is selected at random from the lot. What is the probability that it is
\begin{enumerate}
	\item acceptable to Varnika?
            \item acceptable to the trader?
\end{enumerate}
\solution
	%\begin{table}[H]
	\centering
\begin{tabular}{|c|c|c|}
\hline
Random variable &Value &Definition\\ \hline
\multirow{3}{*}{X} &0 &Slips of Rs 1\\
&1 &Slips of Rs 5\\
&2 &Slips of Rs 13\\ \hline
\multirow{2}{*}{Y} &0 &Box A\\
&1 &Box B\\\hline
\end{tabular}
\caption{}
\label{tab:Distribution}
\end{table}
See \tabref{tab:Distribution}.
\begin{align}
p_{Y}\brak{k}= \begin{cases} 
      \frac{1}{3} & {k=0} \\
      \frac{2}{3 }& {k=1} 
   \end{cases}
   \\
p_{Y|X}\brak{0|0} = \frac{19}{25}\, 
p_{Y|X}\brak{0|1} = \frac{6}{25}\,
p_{Y|X}\brak{1|0} = \frac{45}{50}\,
p_{Y|X}\brak{1|2} = \frac{5}{50}
\end{align}
The desired probability is the probability that a slip drawn at random is marked other than Rs 1,
\begin{align}
&=1-p_X\brak{0}\\
&= p_X(1) + p_X(2)
\end{align}
Using Bayes theorem,
\begin{align}
&= p_Y\brak{0} \times \pr{Y=0 | X=1} + p_Y\brak{1} \times \pr{Y=1|X=2}\\
&=\frac{1}{3} \times \frac{6}{25} + \frac{2}{3} \times \frac{5}{50}\\
&=\frac{11}{75}
\end{align}

\newpage

%\tableofcontents

\bigskip

\renewcommand{\thefigure}{\theenumi}
\renewcommand{\thetable}{\theenumi}
%\renewcommand{\theequation}{\theenumi}

%\begin{abstract}
%%\boldmath
%In this letter, an algorithm for evaluating the exact analytical bit error rate  (BER)  for the piecewise linear (PL) combiner for  multiple relays is presented. Previous results were available only for upto three relays. The algorithm is unique in the sense that  the actual mathematical expressions, that are prohibitively large, need not be explicitly obtained. The diversity gain due to multiple relays is shown through plots of the analytical BER, well supported by simulations. 
%
%\end{abstract}
% IEEEtran.cls defaults to using nonbold math in the Abstract.
% This preserves the distinction between vectors and scalars. However,
% if the journal you are submitting to favors bold math in the abstract,
% then you can use LaTeX's standard command \boldmath at the very start
% of the abstract to achieve this. Many IEEE journals frown on math
% in the abstract anyway.

% Note that keywords are not normally used for peerreview papers.
%\begin{IEEEkeywords}
%Cooperative diversity, decode and forward, piecewise linear
%\end{IEEEkeywords}



% For peer review papers, you can put extra information on the cover
% page as needed:
% \ifCLASSOPTIONpeerreview
% \begin{center} \bfseries EDICS Category: 3-BBND \end{center}
% \fi
%
% For peerreview papers, this IEEEtran command inserts a page break and
% creates the second title. It will be ignored for other modes.
%\IEEEpeerreviewmaketitle




 \item A student says that if you throw a die, it will show up 1 or not 1. Therefore, the probability of getting 1 and the probability of getting 'not 1' each is equal to $\frac{1}{2}$. Is this correct? Give reasons.\\
 \solution
        %\begin{table}[H]
	\centering
\begin{tabular}{|c|c|c|}
\hline
Random variable &Value &Definition\\ \hline
\multirow{3}{*}{X} &0 &Slips of Rs 1\\
&1 &Slips of Rs 5\\
&2 &Slips of Rs 13\\ \hline
\multirow{2}{*}{Y} &0 &Box A\\
&1 &Box B\\\hline
\end{tabular}
\caption{}
\label{tab:Distribution}
\end{table}
See \tabref{tab:Distribution}.
\begin{align}
p_{Y}\brak{k}= \begin{cases} 
      \frac{1}{3} & {k=0} \\
      \frac{2}{3 }& {k=1} 
   \end{cases}
   \\
p_{Y|X}\brak{0|0} = \frac{19}{25}\, 
p_{Y|X}\brak{0|1} = \frac{6}{25}\,
p_{Y|X}\brak{1|0} = \frac{45}{50}\,
p_{Y|X}\brak{1|2} = \frac{5}{50}
\end{align}
The desired probability is the probability that a slip drawn at random is marked other than Rs 1,
\begin{align}
&=1-p_X\brak{0}\\
&= p_X(1) + p_X(2)
\end{align}
Using Bayes theorem,
\begin{align}
&= p_Y\brak{0} \times \pr{Y=0 | X=1} + p_Y\brak{1} \times \pr{Y=1|X=2}\\
&=\frac{1}{3} \times \frac{6}{25} + \frac{2}{3} \times \frac{5}{50}\\
&=\frac{11}{75}
\end{align}

\newpage

%\tableofcontents

\bigskip

\renewcommand{\thefigure}{\theenumi}
\renewcommand{\thetable}{\theenumi}
%\renewcommand{\theequation}{\theenumi}

%\begin{abstract}
%%\boldmath
%In this letter, an algorithm for evaluating the exact analytical bit error rate  (BER)  for the piecewise linear (PL) combiner for  multiple relays is presented. Previous results were available only for upto three relays. The algorithm is unique in the sense that  the actual mathematical expressions, that are prohibitively large, need not be explicitly obtained. The diversity gain due to multiple relays is shown through plots of the analytical BER, well supported by simulations. 
%
%\end{abstract}
% IEEEtran.cls defaults to using nonbold math in the Abstract.
% This preserves the distinction between vectors and scalars. However,
% if the journal you are submitting to favors bold math in the abstract,
% then you can use LaTeX's standard command \boldmath at the very start
% of the abstract to achieve this. Many IEEE journals frown on math
% in the abstract anyway.

% Note that keywords are not normally used for peerreview papers.
%\begin{IEEEkeywords}
%Cooperative diversity, decode and forward, piecewise linear
%\end{IEEEkeywords}



% For peer review papers, you can put extra information on the cover
% page as needed:
% \ifCLASSOPTIONpeerreview
% \begin{center} \bfseries EDICS Category: 3-BBND \end{center}
% \fi
%
% For peerreview papers, this IEEEtran command inserts a page break and
% creates the second title. It will be ignored for other modes.
%\IEEEpeerreviewmaketitle




   \item Four candidates A, B, C, D have ap-
plied for the assignment to coach a school cricket
team. If A is twice as likely to be selected as B, and
B and C are given about the same chance of being
selected, while C is twice as likely to be selected
as D, what are the probabilities that
\begin{enumerate}
\item C will be selected?
\item A will not be selected?
\end{enumerate}
	%\begin{table}[H]
	\centering
\begin{tabular}{|c|c|c|}
\hline
Random variable &Value &Definition\\ \hline
\multirow{3}{*}{X} &0 &Slips of Rs 1\\
&1 &Slips of Rs 5\\
&2 &Slips of Rs 13\\ \hline
\multirow{2}{*}{Y} &0 &Box A\\
&1 &Box B\\\hline
\end{tabular}
\caption{}
\label{tab:Distribution}
\end{table}
See \tabref{tab:Distribution}.
\begin{align}
p_{Y}\brak{k}= \begin{cases} 
      \frac{1}{3} & {k=0} \\
      \frac{2}{3 }& {k=1} 
   \end{cases}
   \\
p_{Y|X}\brak{0|0} = \frac{19}{25}\, 
p_{Y|X}\brak{0|1} = \frac{6}{25}\,
p_{Y|X}\brak{1|0} = \frac{45}{50}\,
p_{Y|X}\brak{1|2} = \frac{5}{50}
\end{align}
The desired probability is the probability that a slip drawn at random is marked other than Rs 1,
\begin{align}
&=1-p_X\brak{0}\\
&= p_X(1) + p_X(2)
\end{align}
Using Bayes theorem,
\begin{align}
&= p_Y\brak{0} \times \pr{Y=0 | X=1} + p_Y\brak{1} \times \pr{Y=1|X=2}\\
&=\frac{1}{3} \times \frac{6}{25} + \frac{2}{3} \times \frac{5}{50}\\
&=\frac{11}{75}
\end{align}

\newpage

%\tableofcontents

\bigskip

\renewcommand{\thefigure}{\theenumi}
\renewcommand{\thetable}{\theenumi}
%\renewcommand{\theequation}{\theenumi}

%\begin{abstract}
%%\boldmath
%In this letter, an algorithm for evaluating the exact analytical bit error rate  (BER)  for the piecewise linear (PL) combiner for  multiple relays is presented. Previous results were available only for upto three relays. The algorithm is unique in the sense that  the actual mathematical expressions, that are prohibitively large, need not be explicitly obtained. The diversity gain due to multiple relays is shown through plots of the analytical BER, well supported by simulations. 
%
%\end{abstract}
% IEEEtran.cls defaults to using nonbold math in the Abstract.
% This preserves the distinction between vectors and scalars. However,
% if the journal you are submitting to favors bold math in the abstract,
% then you can use LaTeX's standard command \boldmath at the very start
% of the abstract to achieve this. Many IEEE journals frown on math
% in the abstract anyway.

% Note that keywords are not normally used for peerreview papers.
%\begin{IEEEkeywords}
%Cooperative diversity, decode and forward, piecewise linear
%\end{IEEEkeywords}



% For peer review papers, you can put extra information on the cover
% page as needed:
% \ifCLASSOPTIONpeerreview
% \begin{center} \bfseries EDICS Category: 3-BBND \end{center}
% \fi
%
% For peerreview papers, this IEEEtran command inserts a page break and
% creates the second title. It will be ignored for other modes.
%\IEEEpeerreviewmaketitle




 \item A bag contain 24 balls of which $x$ balls are red, $2x$ are white and $3x$ are blue. A ball is selected at random, What is the probability that it is
\begin{enumerate}[label=\alph*)]
\item not red ?
\item white ?
\end{enumerate}
%\begin{table}[H]
	\centering
\begin{tabular}{|c|c|c|}
\hline
Random variable &Value &Definition\\ \hline
\multirow{3}{*}{X} &0 &Slips of Rs 1\\
&1 &Slips of Rs 5\\
&2 &Slips of Rs 13\\ \hline
\multirow{2}{*}{Y} &0 &Box A\\
&1 &Box B\\\hline
\end{tabular}
\caption{}
\label{tab:Distribution}
\end{table}
See \tabref{tab:Distribution}.
\begin{align}
p_{Y}\brak{k}= \begin{cases} 
      \frac{1}{3} & {k=0} \\
      \frac{2}{3 }& {k=1} 
   \end{cases}
   \\
p_{Y|X}\brak{0|0} = \frac{19}{25}\, 
p_{Y|X}\brak{0|1} = \frac{6}{25}\,
p_{Y|X}\brak{1|0} = \frac{45}{50}\,
p_{Y|X}\brak{1|2} = \frac{5}{50}
\end{align}
The desired probability is the probability that a slip drawn at random is marked other than Rs 1,
\begin{align}
&=1-p_X\brak{0}\\
&= p_X(1) + p_X(2)
\end{align}
Using Bayes theorem,
\begin{align}
&= p_Y\brak{0} \times \pr{Y=0 | X=1} + p_Y\brak{1} \times \pr{Y=1|X=2}\\
&=\frac{1}{3} \times \frac{6}{25} + \frac{2}{3} \times \frac{5}{50}\\
&=\frac{11}{75}
\end{align}

\newpage

%\tableofcontents

\bigskip

\renewcommand{\thefigure}{\theenumi}
\renewcommand{\thetable}{\theenumi}
%\renewcommand{\theequation}{\theenumi}

%\begin{abstract}
%%\boldmath
%In this letter, an algorithm for evaluating the exact analytical bit error rate  (BER)  for the piecewise linear (PL) combiner for  multiple relays is presented. Previous results were available only for upto three relays. The algorithm is unique in the sense that  the actual mathematical expressions, that are prohibitively large, need not be explicitly obtained. The diversity gain due to multiple relays is shown through plots of the analytical BER, well supported by simulations. 
%
%\end{abstract}
% IEEEtran.cls defaults to using nonbold math in the Abstract.
% This preserves the distinction between vectors and scalars. However,
% if the journal you are submitting to favors bold math in the abstract,
% then you can use LaTeX's standard command \boldmath at the very start
% of the abstract to achieve this. Many IEEE journals frown on math
% in the abstract anyway.

% Note that keywords are not normally used for peerreview papers.
%\begin{IEEEkeywords}
%Cooperative diversity, decode and forward, piecewise linear
%\end{IEEEkeywords}



% For peer review papers, you can put extra information on the cover
% page as needed:
% \ifCLASSOPTIONpeerreview
% \begin{center} \bfseries EDICS Category: 3-BBND \end{center}
% \fi
%
% For peerreview papers, this IEEEtran command inserts a page break and
% creates the second title. It will be ignored for other modes.
%\IEEEpeerreviewmaketitle




If the letters of the word ASSASSINATION are arranged at random. Find the Probability that
\begin{enumerate}[label=(\alph*)]
\item Four $S's$ come consecutively in the word
\item Two  $I's$ and two $N's$ come together
\item All $A's$ are not coming together
\item No two $A's$ are coming together
\end{enumerate}
%\begin{table}[H]
	\centering
\begin{tabular}{|c|c|c|}
\hline
Random variable &Value &Definition\\ \hline
\multirow{3}{*}{X} &0 &Slips of Rs 1\\
&1 &Slips of Rs 5\\
&2 &Slips of Rs 13\\ \hline
\multirow{2}{*}{Y} &0 &Box A\\
&1 &Box B\\\hline
\end{tabular}
\caption{}
\label{tab:Distribution}
\end{table}
See \tabref{tab:Distribution}.
\begin{align}
p_{Y}\brak{k}= \begin{cases} 
      \frac{1}{3} & {k=0} \\
      \frac{2}{3 }& {k=1} 
   \end{cases}
   \\
p_{Y|X}\brak{0|0} = \frac{19}{25}\, 
p_{Y|X}\brak{0|1} = \frac{6}{25}\,
p_{Y|X}\brak{1|0} = \frac{45}{50}\,
p_{Y|X}\brak{1|2} = \frac{5}{50}
\end{align}
The desired probability is the probability that a slip drawn at random is marked other than Rs 1,
\begin{align}
&=1-p_X\brak{0}\\
&= p_X(1) + p_X(2)
\end{align}
Using Bayes theorem,
\begin{align}
&= p_Y\brak{0} \times \pr{Y=0 | X=1} + p_Y\brak{1} \times \pr{Y=1|X=2}\\
&=\frac{1}{3} \times \frac{6}{25} + \frac{2}{3} \times \frac{5}{50}\\
&=\frac{11}{75}
\end{align}

\newpage

%\tableofcontents

\bigskip

\renewcommand{\thefigure}{\theenumi}
\renewcommand{\thetable}{\theenumi}
%\renewcommand{\theequation}{\theenumi}

%\begin{abstract}
%%\boldmath
%In this letter, an algorithm for evaluating the exact analytical bit error rate  (BER)  for the piecewise linear (PL) combiner for  multiple relays is presented. Previous results were available only for upto three relays. The algorithm is unique in the sense that  the actual mathematical expressions, that are prohibitively large, need not be explicitly obtained. The diversity gain due to multiple relays is shown through plots of the analytical BER, well supported by simulations. 
%
%\end{abstract}
% IEEEtran.cls defaults to using nonbold math in the Abstract.
% This preserves the distinction between vectors and scalars. However,
% if the journal you are submitting to favors bold math in the abstract,
% then you can use LaTeX's standard command \boldmath at the very start
% of the abstract to achieve this. Many IEEE journals frown on math
% in the abstract anyway.

% Note that keywords are not normally used for peerreview papers.
%\begin{IEEEkeywords}
%Cooperative diversity, decode and forward, piecewise linear
%\end{IEEEkeywords}



% For peer review papers, you can put extra information on the cover
% page as needed:
% \ifCLASSOPTIONpeerreview
% \begin{center} \bfseries EDICS Category: 3-BBND \end{center}
% \fi
%
% For peerreview papers, this IEEEtran command inserts a page break and
% creates the second title. It will be ignored for other modes.
%\IEEEpeerreviewmaketitle




	\item One urn contains two black balls (labelled B1 and B2) and one white ball. A
	second urn contains one black ball and two white balls (labelled W1 and W2).
	Suppose the following experiment is performed. One of the two urns is chosen
	at random. Next a ball is randomly chosen from the urn. Then a second ball is
	chosen at random from the same urn without replacing the first ball.
	
	\begin{enumerate}
	\item What is the probability that two black balls are chosen?
	
	\item What is the probability that two balls of opposite colour are chosen?
	\end{enumerate}
	\solution
	%\begin{align}
    \label{eq:12.13.6.18.1}
	\because	\pr{A|B} &> \pr{A},\
\frac{\pr{AB}}{\pr{B}} > \pr{A}
\\
    \label{eq:12.13.6.18.2}
	\implies \pr{AB} &> \pr{A}\pr{B}
	\\
	\text{or, } \frac{\pr{AB}}{\pr{A}} &=\pr{B|A} > \pr{A}
\end{align}

\end{enumerate}

	\item 
The number lock of a suitcase has 4 wheels each labelled with ten digits i.e. from 0 to 9.The lock opens with a sequence of four digits with no repeats.What is the probability of a person getting the right sequence to open the suitcase.
\\
\solution
		%\begin{enumerate}[label=\thesection.\arabic*,ref=\thesection.\theenumi]
	\item One card is drawn from a well-shuffled deck of 52 cards. Find the probability of getting
\begin{enumerate}
\item A king of red colour 
\item A face card 
\item A red face card
\item The jack of hearts
\item A spade
\item The queen of diamonds

\end{enumerate}
\solution
		%\begin{table}[H]
	\centering
\begin{tabular}{|c|c|c|}
\hline
Random variable &Value &Definition\\ \hline
\multirow{3}{*}{X} &0 &Slips of Rs 1\\
&1 &Slips of Rs 5\\
&2 &Slips of Rs 13\\ \hline
\multirow{2}{*}{Y} &0 &Box A\\
&1 &Box B\\\hline
\end{tabular}
\caption{}
\label{tab:Distribution}
\end{table}
See \tabref{tab:Distribution}.
\begin{align}
p_{Y}\brak{k}= \begin{cases} 
      \frac{1}{3} & {k=0} \\
      \frac{2}{3 }& {k=1} 
   \end{cases}
   \\
p_{Y|X}\brak{0|0} = \frac{19}{25}\, 
p_{Y|X}\brak{0|1} = \frac{6}{25}\,
p_{Y|X}\brak{1|0} = \frac{45}{50}\,
p_{Y|X}\brak{1|2} = \frac{5}{50}
\end{align}
The desired probability is the probability that a slip drawn at random is marked other than Rs 1,
\begin{align}
&=1-p_X\brak{0}\\
&= p_X(1) + p_X(2)
\end{align}
Using Bayes theorem,
\begin{align}
&= p_Y\brak{0} \times \pr{Y=0 | X=1} + p_Y\brak{1} \times \pr{Y=1|X=2}\\
&=\frac{1}{3} \times \frac{6}{25} + \frac{2}{3} \times \frac{5}{50}\\
&=\frac{11}{75}
\end{align}

\newpage

%\tableofcontents

\bigskip

\renewcommand{\thefigure}{\theenumi}
\renewcommand{\thetable}{\theenumi}
%\renewcommand{\theequation}{\theenumi}

%\begin{abstract}
%%\boldmath
%In this letter, an algorithm for evaluating the exact analytical bit error rate  (BER)  for the piecewise linear (PL) combiner for  multiple relays is presented. Previous results were available only for upto three relays. The algorithm is unique in the sense that  the actual mathematical expressions, that are prohibitively large, need not be explicitly obtained. The diversity gain due to multiple relays is shown through plots of the analytical BER, well supported by simulations. 
%
%\end{abstract}
% IEEEtran.cls defaults to using nonbold math in the Abstract.
% This preserves the distinction between vectors and scalars. However,
% if the journal you are submitting to favors bold math in the abstract,
% then you can use LaTeX's standard command \boldmath at the very start
% of the abstract to achieve this. Many IEEE journals frown on math
% in the abstract anyway.

% Note that keywords are not normally used for peerreview papers.
%\begin{IEEEkeywords}
%Cooperative diversity, decode and forward, piecewise linear
%\end{IEEEkeywords}



% For peer review papers, you can put extra information on the cover
% page as needed:
% \ifCLASSOPTIONpeerreview
% \begin{center} \bfseries EDICS Category: 3-BBND \end{center}
% \fi
%
% For peerreview papers, this IEEEtran command inserts a page break and
% creates the second title. It will be ignored for other modes.
%\IEEEpeerreviewmaketitle




	\item Five cards—the ten, jack, queen, king and ace of diamonds, are well-shuffled with their face downwards. One card is then picked up at random.
\begin{enumerate}
\item
What is the probability that the card is the queen? 
\item
If the queen is drawn and put aside, what is the probability that the second card picked up is (a) an ace? (b) a queen?\\
\end{enumerate}
\solution
		%\begin{enumerate}[label=\thesection.\arabic*,ref=\thesection.\theenumi]
	\item One card is drawn from a well-shuffled deck of 52 cards. Find the probability of getting
\begin{enumerate}
\item A king of red colour 
\item A face card 
\item A red face card
\item The jack of hearts
\item A spade
\item The queen of diamonds

\end{enumerate}
\solution
		%\input{ncert/10/15/1/14/main.tex}
	\item Five cards—the ten, jack, queen, king and ace of diamonds, are well-shuffled with their face downwards. One card is then picked up at random.
\begin{enumerate}
\item
What is the probability that the card is the queen? 
\item
If the queen is drawn and put aside, what is the probability that the second card picked up is (a) an ace? (b) a queen?\\
\end{enumerate}
\solution
		%\input{ncert/10/15/1/15/defs.tex}
	\item A bag contains $5$ red balls and some blue balls. If the probability of drawing a blue ball is double that if a red ball, determine the number of blue balls in the bag. 
		\\
\solution
		%\input{ncert/10/15/2/3/defs.tex}
	\item A card is selected from a pack of 52 cards.
 \begin{enumerate}[label=(\alph*)] 
                 \item How many points are there in the sample space?
                 \item Calculate the probability that the card is an ace of spades.
                 \item Calculate the probability that the card is (i) an ace and (ii) black card.
 \end{enumerate}
\solution
		%\input{ncert/11/16/3/4/main.tex}
\item Four cards are drawn from a well-shuffled deck of 52 cards. What is the probability of obtaining 3 diamonds and one spade.
\\
\solution
		%\input{ncert/11/16/4/2/defs.tex}
\item In a certain lottery 10,000 tickets are sold and ten equal prizes are awarded. What is the probability of not getting a prize if you buy (a) one ticket (b) two tickets (c) 10 tickets ?	
\\
\solution
		%\input{ncert/11/16/4/4/defs.tex}
		%
\item 
Out of 100 students, two sections of 40 and 60 are formed. If you and your friend are among the 100 students, what is the probability that
\begin{enumerate}
\item you both enter the same section?
\item you both enter the different sections?
\end{enumerate}
\solution
		%\input{ncert/11/16/4/5/defs.tex}
	\item 
The number lock of a suitcase has 4 wheels each labelled with ten digits i.e. from 0 to 9.The lock opens with a sequence of four digits with no repeats.What is the probability of a person getting the right sequence to open the suitcase.
\\
\solution
		%\input{ncert/11/16/4/10/defs.tex}
		%
\item 
Two cards are drawn at random and without replacement from a pack of 52 playing cards. Find the probability that both the cards are black.
\\
\solution
		%\input{ncert/12/13/2/2/defs.tex}
		\item A box of oranges is inspected by examining three randomly selected oranges drawn without replacement. If all the three oranges are good, the box is approved for sale, otherwise, it is rejected. Find the probability that a box containing 15 oranges out of which 12 are good and 3 are bad ones will be approved for sale.
		\label{ncert/12/13/2/3/defs.tex}
		\item Two balls are drawn at random with replacement from a box containing 10 black and 8 red balls. Find the probability that
		\label{ncert/12/13/2/12}
\begin{enumerate}
\item both balls are red.
\item first ball is black and second is red.
\item one of them is black and other is red.
\end{enumerate}

\item In a hostel, 60\% of the students read Hindi newspaper, 40\% read English newspaper and 20\% read both Hindi and English newspapers. A student is selected at random.
		\label{ncert/12/13/2/15}
\begin{enumerate}
\item Find the probability that she reads neither Hindi nor English newspapers.
\item If she reads Hindi newspaper, find the probability that she reads English newspaper.
\item If she reads English newspaper, find the probability that she reads Hindi newspaper.\\
\end{enumerate}
\item The probability of obtaining an even prime number on each die, when a pair of dice is rolled is 
\begin{enumerate}
    \item $0$ 
    
    \item $\frac{1}{3}$ 
    
    \item $\frac{1}{12}$ 
    
    \item $\frac{1}{36}$ 
\end{enumerate}
\solution
		%\input{ncert/12/13/2/17/defs.tex}
	\item A bag contains 4 red and 4 black balls, another bag contains 2 red and 6 black balls. One of the two bags is selected at random and a ball is drawn from the bag which is found to be red. Find the probability that the ball is drawn from the first bag.
\\
\solution
		%\input{ncert/12/13/3/2/main.tex}
  \item
  Cards with numbers 2 to 101 are placed in a box. A card is selected at random.Find the probability that the card has
\begin{enumerate}[label=(\roman*)]
	\item an even number 
	\item a square number
\end{enumerate}
\solution
%\input{exemplar/10/13/3/32/main.tex}
\item
The king, queen and jack of clubs are removed from a deck of 52 playing cards and then well shuffled. Now one card is drawn at random from the remaining cards.  Determine the probability that the card is
\begin{enumerate}[label=(\roman*)]
\item a club
\item 10 of hearts
\end{enumerate}
\solution
%\input{exemplar/10/13/3/29/main.tex}
\item A team of medical students doing their internship have to assist during surgeries
at a city hospital. The probabilities of surgeries rated as very complex, complex,
routine, simple or very simple are respectively, 0.15, 0.20, 0.31, 0.26, .08. Find
the probabilities that a particular surgery will be rated
\begin{enumerate}
	\item complex or very complex;
	\item neither very complex nor very simple;
	\item routine or complex
	\item routine or simple
\end{enumerate}
\solution
%\input{exemplar/11/16/3/8(1)/main.tex}
\item A card is selected from a pack of 52 cards.
\begin{enumerate}[label=(\alph*)]
    \item How many points are there in the sample space?
    \item Calculate the probability that the card is an ace of spades.
    \item Calculate the probability that the card is (i) an ace and (ii) black card.
\end{enumerate}
\solution
%\input{exemplar/11/16/3/4/main2.tex}
\item The probability that a non leap year selected at random will contain 53 sundays.
\\
\solution
%\input{exemplar/10/13/1/19/main.tex}
\item One of the four persons John, Rita, Aslam or Gurpreet will be promoted next
month. Consequently the sample space consists of four elementary outcomes
S = {John promoted, Rita promoted, Aslam promoted, Gurpreet promoted}
You are told that the chances of John’s promotion is same as that of Gurpreet,
Rita’s chances of promotion are twice as likely as Johns. Aslam’s chances are
four times that of John.
\begin{enumerate}
	\item Determine
	\begin{enumerate}
		\item P (John promoted)
		\item P (Rita promoted)
		\item P (Aslam promoted)
		\item P (Gurpreet promoted)
	\end{enumerate}
	\item If A = {John promoted or Gurpreet promoted}, find P (A).
\end{enumerate}
\solution
%\input{exemplar/11/16/3/10/main.tex}
\item A card is drawn from a deck of 52 cards. Find the probability of getting a king or a heart or a red card.\\
\solution
%\input{exemplar/11/16/3/15/main.tex}
\item The probability that a student will pass his examination is 0.73, the probability of
the student getting a compartment is 0.13, and the probability that the student will
either pass or get compartment is 0.96. State True or False.\\
\solution
%\input{exemplar/11/16/3/31/main.tex}
\item A card is selected from a pack of 52 cards\\
\begin{enumerate}[label=(\alph*)]
\item How many points are there in the sample space?
\item Calculate the probability that the cards is an ace of spades.
\item Calculate the probability that the card is (i) an ace (ii)black card.\\
\end{enumerate}
%\input{ncert/11/16/3/4_1/Prob_4.tex}
\item In a non-leap year, the probability of having 53 tuesdays or 53 wednesdays is\\
\solution
%\input{exemplar/11/16/3/18/main.tex}
\item There are 1000 sealed envelopes in a box, 10 of them contain a cash prize of
Rs 100 each, 100 of them contain a cash prize of Rs 50 each and 200 of them
contain a cash prize of Rs 10 each and rest do not contain any cash prize. If they
are well shuffled and an envelope is picked up out, what is the probability that it
contains no cash prize?\\
\solution
%\input{exemplar/10/13/3/34/main.tex}
\item 
A die is thrown and a card is selected at random from a deck of 52 playing cards. The probability of getting an even number on the die and a spade card.\\
\solution
%\input{exemplar/12/13/3/78/main.tex}
\item
If 4-digit numbers greater than 5,000 are randomly formed from the digits 0, 1, 3, 5, and 7, what is the probability of forming a number divisible by 5 when:
\begin{enumerate}
    \item The digits are repeated?
    \item The repetition of digits is not allowed?
\end{enumerate}
\solution
%\input{ncert/11/16/4/9/main.tex}
\item Consider the probability space $\brak{\Omega, \mathcal{G}, P}$ where $\Omega = [0,2]$ and $\mathcal{G} = \cbrak{\phi, \Omega, [0,1], (1,2]}$. Let $X$ and $Y$ be two functions on $\Omega$ defined as
\begin{align*}
    X(\omega) = 
    \begin{cases}
        1 & \text{if }\omega \in [0, 1]\\
        2 & \text{if }\omega \in (1, 2]
    \end{cases}
\end{align*}
and
\begin{align*}
    Y(\omega) = 
    \begin{cases}
        2 & \text{if }\omega \in [0, 1.5]\\
        3 & \text{if }\omega \in (1.5, 2].
    \end{cases}
\end{align*}
Then which one of the following statements is true?
\begin{enumerate}
    \item [(A)] $X$ is a random variable with respect to $\mathcal{G}$, but $Y$ is not a random variable with respect to $\mathcal{G}$.
    \item [(B)] $Y$ is a random variable with respect to $\mathcal{G}$, but $X$ is not a random variable with respect to $\mathcal{G}$.
    \item [(C)] Neither $X$ nor $Y$ is a random variable with respect to $\mathcal{G}$.
    \item [(D)] Both $X$ and $Y$ are random variables with respect to $\mathcal{G}$.
\end{enumerate} \hfill (GATE ST 2023)\\
\solution
%\input{gate/ST/2023/14/main.tex}
	\item  A die is loaded in such a way that each odd number is twice as likely to occur as
each even number. Find $P(G)$, where $G$ is the event that a number greater than
3 occurs on a single roll of the die.
\\
\solution
		%\input{exemplar/11/16/3/5/main.tex}
	\item All the jacks, queens and kings are removed from a deck of 52 playing cards. The remaining cards are well shuffled and then one card is drawn at random. Giving ace a value 1 similar value for other cards, find the probability that the card has a value 
		\begin{enumerate}
			\item 7
			\item greater than 7
			\item less than 7
		\end{enumerate}
		%\input{exemplar/10/13/3/30/main.tex}
  \item A Lot consists of 48 mobile phones of which 42 are good, 3 have only minor defects and 3 have major defects.Varnika will buy a phone if it is good but the trader will only buy a mobile if it has no major defects. One phone is selected at random from the lot. What is the probability that it is
\begin{enumerate}
	\item acceptable to Varnika?
            \item acceptable to the trader?
\end{enumerate}
\solution
	%\input{exemplar/10/13/3/40/main.tex}
 \item A student says that if you throw a die, it will show up 1 or not 1. Therefore, the probability of getting 1 and the probability of getting 'not 1' each is equal to $\frac{1}{2}$. Is this correct? Give reasons.\\
 \solution
        %\input{exemplar/10/13/2/9/main.tex}
   \item Four candidates A, B, C, D have ap-
plied for the assignment to coach a school cricket
team. If A is twice as likely to be selected as B, and
B and C are given about the same chance of being
selected, while C is twice as likely to be selected
as D, what are the probabilities that
\begin{enumerate}
\item C will be selected?
\item A will not be selected?
\end{enumerate}
	%\input{exemplar/11/16/3/9/main.tex}
 \item A bag contain 24 balls of which $x$ balls are red, $2x$ are white and $3x$ are blue. A ball is selected at random, What is the probability that it is
\begin{enumerate}[label=\alph*)]
\item not red ?
\item white ?
\end{enumerate}
%\input{exemplar/10/13/3/41/main.tex}
If the letters of the word ASSASSINATION are arranged at random. Find the Probability that
\begin{enumerate}[label=(\alph*)]
\item Four $S's$ come consecutively in the word
\item Two  $I's$ and two $N's$ come together
\item All $A's$ are not coming together
\item No two $A's$ are coming together
\end{enumerate}
%\input{exemplar/11/16/3/14/main.tex}
	\item One urn contains two black balls (labelled B1 and B2) and one white ball. A
	second urn contains one black ball and two white balls (labelled W1 and W2).
	Suppose the following experiment is performed. One of the two urns is chosen
	at random. Next a ball is randomly chosen from the urn. Then a second ball is
	chosen at random from the same urn without replacing the first ball.
	
	\begin{enumerate}
	\item What is the probability that two black balls are chosen?
	
	\item What is the probability that two balls of opposite colour are chosen?
	\end{enumerate}
	\solution
	%\input{exemplar/11/16/3/12/main1.tex}
\end{enumerate}

	\item A bag contains $5$ red balls and some blue balls. If the probability of drawing a blue ball is double that if a red ball, determine the number of blue balls in the bag. 
		\\
\solution
		%\begin{enumerate}[label=\thesection.\arabic*,ref=\thesection.\theenumi]
	\item One card is drawn from a well-shuffled deck of 52 cards. Find the probability of getting
\begin{enumerate}
\item A king of red colour 
\item A face card 
\item A red face card
\item The jack of hearts
\item A spade
\item The queen of diamonds

\end{enumerate}
\solution
		%\input{ncert/10/15/1/14/main.tex}
	\item Five cards—the ten, jack, queen, king and ace of diamonds, are well-shuffled with their face downwards. One card is then picked up at random.
\begin{enumerate}
\item
What is the probability that the card is the queen? 
\item
If the queen is drawn and put aside, what is the probability that the second card picked up is (a) an ace? (b) a queen?\\
\end{enumerate}
\solution
		%\input{ncert/10/15/1/15/defs.tex}
	\item A bag contains $5$ red balls and some blue balls. If the probability of drawing a blue ball is double that if a red ball, determine the number of blue balls in the bag. 
		\\
\solution
		%\input{ncert/10/15/2/3/defs.tex}
	\item A card is selected from a pack of 52 cards.
 \begin{enumerate}[label=(\alph*)] 
                 \item How many points are there in the sample space?
                 \item Calculate the probability that the card is an ace of spades.
                 \item Calculate the probability that the card is (i) an ace and (ii) black card.
 \end{enumerate}
\solution
		%\input{ncert/11/16/3/4/main.tex}
\item Four cards are drawn from a well-shuffled deck of 52 cards. What is the probability of obtaining 3 diamonds and one spade.
\\
\solution
		%\input{ncert/11/16/4/2/defs.tex}
\item In a certain lottery 10,000 tickets are sold and ten equal prizes are awarded. What is the probability of not getting a prize if you buy (a) one ticket (b) two tickets (c) 10 tickets ?	
\\
\solution
		%\input{ncert/11/16/4/4/defs.tex}
		%
\item 
Out of 100 students, two sections of 40 and 60 are formed. If you and your friend are among the 100 students, what is the probability that
\begin{enumerate}
\item you both enter the same section?
\item you both enter the different sections?
\end{enumerate}
\solution
		%\input{ncert/11/16/4/5/defs.tex}
	\item 
The number lock of a suitcase has 4 wheels each labelled with ten digits i.e. from 0 to 9.The lock opens with a sequence of four digits with no repeats.What is the probability of a person getting the right sequence to open the suitcase.
\\
\solution
		%\input{ncert/11/16/4/10/defs.tex}
		%
\item 
Two cards are drawn at random and without replacement from a pack of 52 playing cards. Find the probability that both the cards are black.
\\
\solution
		%\input{ncert/12/13/2/2/defs.tex}
		\item A box of oranges is inspected by examining three randomly selected oranges drawn without replacement. If all the three oranges are good, the box is approved for sale, otherwise, it is rejected. Find the probability that a box containing 15 oranges out of which 12 are good and 3 are bad ones will be approved for sale.
		\label{ncert/12/13/2/3/defs.tex}
		\item Two balls are drawn at random with replacement from a box containing 10 black and 8 red balls. Find the probability that
		\label{ncert/12/13/2/12}
\begin{enumerate}
\item both balls are red.
\item first ball is black and second is red.
\item one of them is black and other is red.
\end{enumerate}

\item In a hostel, 60\% of the students read Hindi newspaper, 40\% read English newspaper and 20\% read both Hindi and English newspapers. A student is selected at random.
		\label{ncert/12/13/2/15}
\begin{enumerate}
\item Find the probability that she reads neither Hindi nor English newspapers.
\item If she reads Hindi newspaper, find the probability that she reads English newspaper.
\item If she reads English newspaper, find the probability that she reads Hindi newspaper.\\
\end{enumerate}
\item The probability of obtaining an even prime number on each die, when a pair of dice is rolled is 
\begin{enumerate}
    \item $0$ 
    
    \item $\frac{1}{3}$ 
    
    \item $\frac{1}{12}$ 
    
    \item $\frac{1}{36}$ 
\end{enumerate}
\solution
		%\input{ncert/12/13/2/17/defs.tex}
	\item A bag contains 4 red and 4 black balls, another bag contains 2 red and 6 black balls. One of the two bags is selected at random and a ball is drawn from the bag which is found to be red. Find the probability that the ball is drawn from the first bag.
\\
\solution
		%\input{ncert/12/13/3/2/main.tex}
  \item
  Cards with numbers 2 to 101 are placed in a box. A card is selected at random.Find the probability that the card has
\begin{enumerate}[label=(\roman*)]
	\item an even number 
	\item a square number
\end{enumerate}
\solution
%\input{exemplar/10/13/3/32/main.tex}
\item
The king, queen and jack of clubs are removed from a deck of 52 playing cards and then well shuffled. Now one card is drawn at random from the remaining cards.  Determine the probability that the card is
\begin{enumerate}[label=(\roman*)]
\item a club
\item 10 of hearts
\end{enumerate}
\solution
%\input{exemplar/10/13/3/29/main.tex}
\item A team of medical students doing their internship have to assist during surgeries
at a city hospital. The probabilities of surgeries rated as very complex, complex,
routine, simple or very simple are respectively, 0.15, 0.20, 0.31, 0.26, .08. Find
the probabilities that a particular surgery will be rated
\begin{enumerate}
	\item complex or very complex;
	\item neither very complex nor very simple;
	\item routine or complex
	\item routine or simple
\end{enumerate}
\solution
%\input{exemplar/11/16/3/8(1)/main.tex}
\item A card is selected from a pack of 52 cards.
\begin{enumerate}[label=(\alph*)]
    \item How many points are there in the sample space?
    \item Calculate the probability that the card is an ace of spades.
    \item Calculate the probability that the card is (i) an ace and (ii) black card.
\end{enumerate}
\solution
%\input{exemplar/11/16/3/4/main2.tex}
\item The probability that a non leap year selected at random will contain 53 sundays.
\\
\solution
%\input{exemplar/10/13/1/19/main.tex}
\item One of the four persons John, Rita, Aslam or Gurpreet will be promoted next
month. Consequently the sample space consists of four elementary outcomes
S = {John promoted, Rita promoted, Aslam promoted, Gurpreet promoted}
You are told that the chances of John’s promotion is same as that of Gurpreet,
Rita’s chances of promotion are twice as likely as Johns. Aslam’s chances are
four times that of John.
\begin{enumerate}
	\item Determine
	\begin{enumerate}
		\item P (John promoted)
		\item P (Rita promoted)
		\item P (Aslam promoted)
		\item P (Gurpreet promoted)
	\end{enumerate}
	\item If A = {John promoted or Gurpreet promoted}, find P (A).
\end{enumerate}
\solution
%\input{exemplar/11/16/3/10/main.tex}
\item A card is drawn from a deck of 52 cards. Find the probability of getting a king or a heart or a red card.\\
\solution
%\input{exemplar/11/16/3/15/main.tex}
\item The probability that a student will pass his examination is 0.73, the probability of
the student getting a compartment is 0.13, and the probability that the student will
either pass or get compartment is 0.96. State True or False.\\
\solution
%\input{exemplar/11/16/3/31/main.tex}
\item A card is selected from a pack of 52 cards\\
\begin{enumerate}[label=(\alph*)]
\item How many points are there in the sample space?
\item Calculate the probability that the cards is an ace of spades.
\item Calculate the probability that the card is (i) an ace (ii)black card.\\
\end{enumerate}
%\input{ncert/11/16/3/4_1/Prob_4.tex}
\item In a non-leap year, the probability of having 53 tuesdays or 53 wednesdays is\\
\solution
%\input{exemplar/11/16/3/18/main.tex}
\item There are 1000 sealed envelopes in a box, 10 of them contain a cash prize of
Rs 100 each, 100 of them contain a cash prize of Rs 50 each and 200 of them
contain a cash prize of Rs 10 each and rest do not contain any cash prize. If they
are well shuffled and an envelope is picked up out, what is the probability that it
contains no cash prize?\\
\solution
%\input{exemplar/10/13/3/34/main.tex}
\item 
A die is thrown and a card is selected at random from a deck of 52 playing cards. The probability of getting an even number on the die and a spade card.\\
\solution
%\input{exemplar/12/13/3/78/main.tex}
\item
If 4-digit numbers greater than 5,000 are randomly formed from the digits 0, 1, 3, 5, and 7, what is the probability of forming a number divisible by 5 when:
\begin{enumerate}
    \item The digits are repeated?
    \item The repetition of digits is not allowed?
\end{enumerate}
\solution
%\input{ncert/11/16/4/9/main.tex}
\item Consider the probability space $\brak{\Omega, \mathcal{G}, P}$ where $\Omega = [0,2]$ and $\mathcal{G} = \cbrak{\phi, \Omega, [0,1], (1,2]}$. Let $X$ and $Y$ be two functions on $\Omega$ defined as
\begin{align*}
    X(\omega) = 
    \begin{cases}
        1 & \text{if }\omega \in [0, 1]\\
        2 & \text{if }\omega \in (1, 2]
    \end{cases}
\end{align*}
and
\begin{align*}
    Y(\omega) = 
    \begin{cases}
        2 & \text{if }\omega \in [0, 1.5]\\
        3 & \text{if }\omega \in (1.5, 2].
    \end{cases}
\end{align*}
Then which one of the following statements is true?
\begin{enumerate}
    \item [(A)] $X$ is a random variable with respect to $\mathcal{G}$, but $Y$ is not a random variable with respect to $\mathcal{G}$.
    \item [(B)] $Y$ is a random variable with respect to $\mathcal{G}$, but $X$ is not a random variable with respect to $\mathcal{G}$.
    \item [(C)] Neither $X$ nor $Y$ is a random variable with respect to $\mathcal{G}$.
    \item [(D)] Both $X$ and $Y$ are random variables with respect to $\mathcal{G}$.
\end{enumerate} \hfill (GATE ST 2023)\\
\solution
%\input{gate/ST/2023/14/main.tex}
	\item  A die is loaded in such a way that each odd number is twice as likely to occur as
each even number. Find $P(G)$, where $G$ is the event that a number greater than
3 occurs on a single roll of the die.
\\
\solution
		%\input{exemplar/11/16/3/5/main.tex}
	\item All the jacks, queens and kings are removed from a deck of 52 playing cards. The remaining cards are well shuffled and then one card is drawn at random. Giving ace a value 1 similar value for other cards, find the probability that the card has a value 
		\begin{enumerate}
			\item 7
			\item greater than 7
			\item less than 7
		\end{enumerate}
		%\input{exemplar/10/13/3/30/main.tex}
  \item A Lot consists of 48 mobile phones of which 42 are good, 3 have only minor defects and 3 have major defects.Varnika will buy a phone if it is good but the trader will only buy a mobile if it has no major defects. One phone is selected at random from the lot. What is the probability that it is
\begin{enumerate}
	\item acceptable to Varnika?
            \item acceptable to the trader?
\end{enumerate}
\solution
	%\input{exemplar/10/13/3/40/main.tex}
 \item A student says that if you throw a die, it will show up 1 or not 1. Therefore, the probability of getting 1 and the probability of getting 'not 1' each is equal to $\frac{1}{2}$. Is this correct? Give reasons.\\
 \solution
        %\input{exemplar/10/13/2/9/main.tex}
   \item Four candidates A, B, C, D have ap-
plied for the assignment to coach a school cricket
team. If A is twice as likely to be selected as B, and
B and C are given about the same chance of being
selected, while C is twice as likely to be selected
as D, what are the probabilities that
\begin{enumerate}
\item C will be selected?
\item A will not be selected?
\end{enumerate}
	%\input{exemplar/11/16/3/9/main.tex}
 \item A bag contain 24 balls of which $x$ balls are red, $2x$ are white and $3x$ are blue. A ball is selected at random, What is the probability that it is
\begin{enumerate}[label=\alph*)]
\item not red ?
\item white ?
\end{enumerate}
%\input{exemplar/10/13/3/41/main.tex}
If the letters of the word ASSASSINATION are arranged at random. Find the Probability that
\begin{enumerate}[label=(\alph*)]
\item Four $S's$ come consecutively in the word
\item Two  $I's$ and two $N's$ come together
\item All $A's$ are not coming together
\item No two $A's$ are coming together
\end{enumerate}
%\input{exemplar/11/16/3/14/main.tex}
	\item One urn contains two black balls (labelled B1 and B2) and one white ball. A
	second urn contains one black ball and two white balls (labelled W1 and W2).
	Suppose the following experiment is performed. One of the two urns is chosen
	at random. Next a ball is randomly chosen from the urn. Then a second ball is
	chosen at random from the same urn without replacing the first ball.
	
	\begin{enumerate}
	\item What is the probability that two black balls are chosen?
	
	\item What is the probability that two balls of opposite colour are chosen?
	\end{enumerate}
	\solution
	%\input{exemplar/11/16/3/12/main1.tex}
\end{enumerate}

	\item A card is selected from a pack of 52 cards.
 \begin{enumerate}[label=(\alph*)] 
                 \item How many points are there in the sample space?
                 \item Calculate the probability that the card is an ace of spades.
                 \item Calculate the probability that the card is (i) an ace and (ii) black card.
 \end{enumerate}
\solution
		%\begin{table}[H]
	\centering
\begin{tabular}{|c|c|c|}
\hline
Random variable &Value &Definition\\ \hline
\multirow{3}{*}{X} &0 &Slips of Rs 1\\
&1 &Slips of Rs 5\\
&2 &Slips of Rs 13\\ \hline
\multirow{2}{*}{Y} &0 &Box A\\
&1 &Box B\\\hline
\end{tabular}
\caption{}
\label{tab:Distribution}
\end{table}
See \tabref{tab:Distribution}.
\begin{align}
p_{Y}\brak{k}= \begin{cases} 
      \frac{1}{3} & {k=0} \\
      \frac{2}{3 }& {k=1} 
   \end{cases}
   \\
p_{Y|X}\brak{0|0} = \frac{19}{25}\, 
p_{Y|X}\brak{0|1} = \frac{6}{25}\,
p_{Y|X}\brak{1|0} = \frac{45}{50}\,
p_{Y|X}\brak{1|2} = \frac{5}{50}
\end{align}
The desired probability is the probability that a slip drawn at random is marked other than Rs 1,
\begin{align}
&=1-p_X\brak{0}\\
&= p_X(1) + p_X(2)
\end{align}
Using Bayes theorem,
\begin{align}
&= p_Y\brak{0} \times \pr{Y=0 | X=1} + p_Y\brak{1} \times \pr{Y=1|X=2}\\
&=\frac{1}{3} \times \frac{6}{25} + \frac{2}{3} \times \frac{5}{50}\\
&=\frac{11}{75}
\end{align}

\newpage

%\tableofcontents

\bigskip

\renewcommand{\thefigure}{\theenumi}
\renewcommand{\thetable}{\theenumi}
%\renewcommand{\theequation}{\theenumi}

%\begin{abstract}
%%\boldmath
%In this letter, an algorithm for evaluating the exact analytical bit error rate  (BER)  for the piecewise linear (PL) combiner for  multiple relays is presented. Previous results were available only for upto three relays. The algorithm is unique in the sense that  the actual mathematical expressions, that are prohibitively large, need not be explicitly obtained. The diversity gain due to multiple relays is shown through plots of the analytical BER, well supported by simulations. 
%
%\end{abstract}
% IEEEtran.cls defaults to using nonbold math in the Abstract.
% This preserves the distinction between vectors and scalars. However,
% if the journal you are submitting to favors bold math in the abstract,
% then you can use LaTeX's standard command \boldmath at the very start
% of the abstract to achieve this. Many IEEE journals frown on math
% in the abstract anyway.

% Note that keywords are not normally used for peerreview papers.
%\begin{IEEEkeywords}
%Cooperative diversity, decode and forward, piecewise linear
%\end{IEEEkeywords}



% For peer review papers, you can put extra information on the cover
% page as needed:
% \ifCLASSOPTIONpeerreview
% \begin{center} \bfseries EDICS Category: 3-BBND \end{center}
% \fi
%
% For peerreview papers, this IEEEtran command inserts a page break and
% creates the second title. It will be ignored for other modes.
%\IEEEpeerreviewmaketitle




\item Four cards are drawn from a well-shuffled deck of 52 cards. What is the probability of obtaining 3 diamonds and one spade.
\\
\solution
		%\begin{enumerate}[label=\thesection.\arabic*,ref=\thesection.\theenumi]
	\item One card is drawn from a well-shuffled deck of 52 cards. Find the probability of getting
\begin{enumerate}
\item A king of red colour 
\item A face card 
\item A red face card
\item The jack of hearts
\item A spade
\item The queen of diamonds

\end{enumerate}
\solution
		%\input{ncert/10/15/1/14/main.tex}
	\item Five cards—the ten, jack, queen, king and ace of diamonds, are well-shuffled with their face downwards. One card is then picked up at random.
\begin{enumerate}
\item
What is the probability that the card is the queen? 
\item
If the queen is drawn and put aside, what is the probability that the second card picked up is (a) an ace? (b) a queen?\\
\end{enumerate}
\solution
		%\input{ncert/10/15/1/15/defs.tex}
	\item A bag contains $5$ red balls and some blue balls. If the probability of drawing a blue ball is double that if a red ball, determine the number of blue balls in the bag. 
		\\
\solution
		%\input{ncert/10/15/2/3/defs.tex}
	\item A card is selected from a pack of 52 cards.
 \begin{enumerate}[label=(\alph*)] 
                 \item How many points are there in the sample space?
                 \item Calculate the probability that the card is an ace of spades.
                 \item Calculate the probability that the card is (i) an ace and (ii) black card.
 \end{enumerate}
\solution
		%\input{ncert/11/16/3/4/main.tex}
\item Four cards are drawn from a well-shuffled deck of 52 cards. What is the probability of obtaining 3 diamonds and one spade.
\\
\solution
		%\input{ncert/11/16/4/2/defs.tex}
\item In a certain lottery 10,000 tickets are sold and ten equal prizes are awarded. What is the probability of not getting a prize if you buy (a) one ticket (b) two tickets (c) 10 tickets ?	
\\
\solution
		%\input{ncert/11/16/4/4/defs.tex}
		%
\item 
Out of 100 students, two sections of 40 and 60 are formed. If you and your friend are among the 100 students, what is the probability that
\begin{enumerate}
\item you both enter the same section?
\item you both enter the different sections?
\end{enumerate}
\solution
		%\input{ncert/11/16/4/5/defs.tex}
	\item 
The number lock of a suitcase has 4 wheels each labelled with ten digits i.e. from 0 to 9.The lock opens with a sequence of four digits with no repeats.What is the probability of a person getting the right sequence to open the suitcase.
\\
\solution
		%\input{ncert/11/16/4/10/defs.tex}
		%
\item 
Two cards are drawn at random and without replacement from a pack of 52 playing cards. Find the probability that both the cards are black.
\\
\solution
		%\input{ncert/12/13/2/2/defs.tex}
		\item A box of oranges is inspected by examining three randomly selected oranges drawn without replacement. If all the three oranges are good, the box is approved for sale, otherwise, it is rejected. Find the probability that a box containing 15 oranges out of which 12 are good and 3 are bad ones will be approved for sale.
		\label{ncert/12/13/2/3/defs.tex}
		\item Two balls are drawn at random with replacement from a box containing 10 black and 8 red balls. Find the probability that
		\label{ncert/12/13/2/12}
\begin{enumerate}
\item both balls are red.
\item first ball is black and second is red.
\item one of them is black and other is red.
\end{enumerate}

\item In a hostel, 60\% of the students read Hindi newspaper, 40\% read English newspaper and 20\% read both Hindi and English newspapers. A student is selected at random.
		\label{ncert/12/13/2/15}
\begin{enumerate}
\item Find the probability that she reads neither Hindi nor English newspapers.
\item If she reads Hindi newspaper, find the probability that she reads English newspaper.
\item If she reads English newspaper, find the probability that she reads Hindi newspaper.\\
\end{enumerate}
\item The probability of obtaining an even prime number on each die, when a pair of dice is rolled is 
\begin{enumerate}
    \item $0$ 
    
    \item $\frac{1}{3}$ 
    
    \item $\frac{1}{12}$ 
    
    \item $\frac{1}{36}$ 
\end{enumerate}
\solution
		%\input{ncert/12/13/2/17/defs.tex}
	\item A bag contains 4 red and 4 black balls, another bag contains 2 red and 6 black balls. One of the two bags is selected at random and a ball is drawn from the bag which is found to be red. Find the probability that the ball is drawn from the first bag.
\\
\solution
		%\input{ncert/12/13/3/2/main.tex}
  \item
  Cards with numbers 2 to 101 are placed in a box. A card is selected at random.Find the probability that the card has
\begin{enumerate}[label=(\roman*)]
	\item an even number 
	\item a square number
\end{enumerate}
\solution
%\input{exemplar/10/13/3/32/main.tex}
\item
The king, queen and jack of clubs are removed from a deck of 52 playing cards and then well shuffled. Now one card is drawn at random from the remaining cards.  Determine the probability that the card is
\begin{enumerate}[label=(\roman*)]
\item a club
\item 10 of hearts
\end{enumerate}
\solution
%\input{exemplar/10/13/3/29/main.tex}
\item A team of medical students doing their internship have to assist during surgeries
at a city hospital. The probabilities of surgeries rated as very complex, complex,
routine, simple or very simple are respectively, 0.15, 0.20, 0.31, 0.26, .08. Find
the probabilities that a particular surgery will be rated
\begin{enumerate}
	\item complex or very complex;
	\item neither very complex nor very simple;
	\item routine or complex
	\item routine or simple
\end{enumerate}
\solution
%\input{exemplar/11/16/3/8(1)/main.tex}
\item A card is selected from a pack of 52 cards.
\begin{enumerate}[label=(\alph*)]
    \item How many points are there in the sample space?
    \item Calculate the probability that the card is an ace of spades.
    \item Calculate the probability that the card is (i) an ace and (ii) black card.
\end{enumerate}
\solution
%\input{exemplar/11/16/3/4/main2.tex}
\item The probability that a non leap year selected at random will contain 53 sundays.
\\
\solution
%\input{exemplar/10/13/1/19/main.tex}
\item One of the four persons John, Rita, Aslam or Gurpreet will be promoted next
month. Consequently the sample space consists of four elementary outcomes
S = {John promoted, Rita promoted, Aslam promoted, Gurpreet promoted}
You are told that the chances of John’s promotion is same as that of Gurpreet,
Rita’s chances of promotion are twice as likely as Johns. Aslam’s chances are
four times that of John.
\begin{enumerate}
	\item Determine
	\begin{enumerate}
		\item P (John promoted)
		\item P (Rita promoted)
		\item P (Aslam promoted)
		\item P (Gurpreet promoted)
	\end{enumerate}
	\item If A = {John promoted or Gurpreet promoted}, find P (A).
\end{enumerate}
\solution
%\input{exemplar/11/16/3/10/main.tex}
\item A card is drawn from a deck of 52 cards. Find the probability of getting a king or a heart or a red card.\\
\solution
%\input{exemplar/11/16/3/15/main.tex}
\item The probability that a student will pass his examination is 0.73, the probability of
the student getting a compartment is 0.13, and the probability that the student will
either pass or get compartment is 0.96. State True or False.\\
\solution
%\input{exemplar/11/16/3/31/main.tex}
\item A card is selected from a pack of 52 cards\\
\begin{enumerate}[label=(\alph*)]
\item How many points are there in the sample space?
\item Calculate the probability that the cards is an ace of spades.
\item Calculate the probability that the card is (i) an ace (ii)black card.\\
\end{enumerate}
%\input{ncert/11/16/3/4_1/Prob_4.tex}
\item In a non-leap year, the probability of having 53 tuesdays or 53 wednesdays is\\
\solution
%\input{exemplar/11/16/3/18/main.tex}
\item There are 1000 sealed envelopes in a box, 10 of them contain a cash prize of
Rs 100 each, 100 of them contain a cash prize of Rs 50 each and 200 of them
contain a cash prize of Rs 10 each and rest do not contain any cash prize. If they
are well shuffled and an envelope is picked up out, what is the probability that it
contains no cash prize?\\
\solution
%\input{exemplar/10/13/3/34/main.tex}
\item 
A die is thrown and a card is selected at random from a deck of 52 playing cards. The probability of getting an even number on the die and a spade card.\\
\solution
%\input{exemplar/12/13/3/78/main.tex}
\item
If 4-digit numbers greater than 5,000 are randomly formed from the digits 0, 1, 3, 5, and 7, what is the probability of forming a number divisible by 5 when:
\begin{enumerate}
    \item The digits are repeated?
    \item The repetition of digits is not allowed?
\end{enumerate}
\solution
%\input{ncert/11/16/4/9/main.tex}
\item Consider the probability space $\brak{\Omega, \mathcal{G}, P}$ where $\Omega = [0,2]$ and $\mathcal{G} = \cbrak{\phi, \Omega, [0,1], (1,2]}$. Let $X$ and $Y$ be two functions on $\Omega$ defined as
\begin{align*}
    X(\omega) = 
    \begin{cases}
        1 & \text{if }\omega \in [0, 1]\\
        2 & \text{if }\omega \in (1, 2]
    \end{cases}
\end{align*}
and
\begin{align*}
    Y(\omega) = 
    \begin{cases}
        2 & \text{if }\omega \in [0, 1.5]\\
        3 & \text{if }\omega \in (1.5, 2].
    \end{cases}
\end{align*}
Then which one of the following statements is true?
\begin{enumerate}
    \item [(A)] $X$ is a random variable with respect to $\mathcal{G}$, but $Y$ is not a random variable with respect to $\mathcal{G}$.
    \item [(B)] $Y$ is a random variable with respect to $\mathcal{G}$, but $X$ is not a random variable with respect to $\mathcal{G}$.
    \item [(C)] Neither $X$ nor $Y$ is a random variable with respect to $\mathcal{G}$.
    \item [(D)] Both $X$ and $Y$ are random variables with respect to $\mathcal{G}$.
\end{enumerate} \hfill (GATE ST 2023)\\
\solution
%\input{gate/ST/2023/14/main.tex}
	\item  A die is loaded in such a way that each odd number is twice as likely to occur as
each even number. Find $P(G)$, where $G$ is the event that a number greater than
3 occurs on a single roll of the die.
\\
\solution
		%\input{exemplar/11/16/3/5/main.tex}
	\item All the jacks, queens and kings are removed from a deck of 52 playing cards. The remaining cards are well shuffled and then one card is drawn at random. Giving ace a value 1 similar value for other cards, find the probability that the card has a value 
		\begin{enumerate}
			\item 7
			\item greater than 7
			\item less than 7
		\end{enumerate}
		%\input{exemplar/10/13/3/30/main.tex}
  \item A Lot consists of 48 mobile phones of which 42 are good, 3 have only minor defects and 3 have major defects.Varnika will buy a phone if it is good but the trader will only buy a mobile if it has no major defects. One phone is selected at random from the lot. What is the probability that it is
\begin{enumerate}
	\item acceptable to Varnika?
            \item acceptable to the trader?
\end{enumerate}
\solution
	%\input{exemplar/10/13/3/40/main.tex}
 \item A student says that if you throw a die, it will show up 1 or not 1. Therefore, the probability of getting 1 and the probability of getting 'not 1' each is equal to $\frac{1}{2}$. Is this correct? Give reasons.\\
 \solution
        %\input{exemplar/10/13/2/9/main.tex}
   \item Four candidates A, B, C, D have ap-
plied for the assignment to coach a school cricket
team. If A is twice as likely to be selected as B, and
B and C are given about the same chance of being
selected, while C is twice as likely to be selected
as D, what are the probabilities that
\begin{enumerate}
\item C will be selected?
\item A will not be selected?
\end{enumerate}
	%\input{exemplar/11/16/3/9/main.tex}
 \item A bag contain 24 balls of which $x$ balls are red, $2x$ are white and $3x$ are blue. A ball is selected at random, What is the probability that it is
\begin{enumerate}[label=\alph*)]
\item not red ?
\item white ?
\end{enumerate}
%\input{exemplar/10/13/3/41/main.tex}
If the letters of the word ASSASSINATION are arranged at random. Find the Probability that
\begin{enumerate}[label=(\alph*)]
\item Four $S's$ come consecutively in the word
\item Two  $I's$ and two $N's$ come together
\item All $A's$ are not coming together
\item No two $A's$ are coming together
\end{enumerate}
%\input{exemplar/11/16/3/14/main.tex}
	\item One urn contains two black balls (labelled B1 and B2) and one white ball. A
	second urn contains one black ball and two white balls (labelled W1 and W2).
	Suppose the following experiment is performed. One of the two urns is chosen
	at random. Next a ball is randomly chosen from the urn. Then a second ball is
	chosen at random from the same urn without replacing the first ball.
	
	\begin{enumerate}
	\item What is the probability that two black balls are chosen?
	
	\item What is the probability that two balls of opposite colour are chosen?
	\end{enumerate}
	\solution
	%\input{exemplar/11/16/3/12/main1.tex}
\end{enumerate}

\item In a certain lottery 10,000 tickets are sold and ten equal prizes are awarded. What is the probability of not getting a prize if you buy (a) one ticket (b) two tickets (c) 10 tickets ?	
\\
\solution
		%\begin{enumerate}[label=\thesection.\arabic*,ref=\thesection.\theenumi]
	\item One card is drawn from a well-shuffled deck of 52 cards. Find the probability of getting
\begin{enumerate}
\item A king of red colour 
\item A face card 
\item A red face card
\item The jack of hearts
\item A spade
\item The queen of diamonds

\end{enumerate}
\solution
		%\input{ncert/10/15/1/14/main.tex}
	\item Five cards—the ten, jack, queen, king and ace of diamonds, are well-shuffled with their face downwards. One card is then picked up at random.
\begin{enumerate}
\item
What is the probability that the card is the queen? 
\item
If the queen is drawn and put aside, what is the probability that the second card picked up is (a) an ace? (b) a queen?\\
\end{enumerate}
\solution
		%\input{ncert/10/15/1/15/defs.tex}
	\item A bag contains $5$ red balls and some blue balls. If the probability of drawing a blue ball is double that if a red ball, determine the number of blue balls in the bag. 
		\\
\solution
		%\input{ncert/10/15/2/3/defs.tex}
	\item A card is selected from a pack of 52 cards.
 \begin{enumerate}[label=(\alph*)] 
                 \item How many points are there in the sample space?
                 \item Calculate the probability that the card is an ace of spades.
                 \item Calculate the probability that the card is (i) an ace and (ii) black card.
 \end{enumerate}
\solution
		%\input{ncert/11/16/3/4/main.tex}
\item Four cards are drawn from a well-shuffled deck of 52 cards. What is the probability of obtaining 3 diamonds and one spade.
\\
\solution
		%\input{ncert/11/16/4/2/defs.tex}
\item In a certain lottery 10,000 tickets are sold and ten equal prizes are awarded. What is the probability of not getting a prize if you buy (a) one ticket (b) two tickets (c) 10 tickets ?	
\\
\solution
		%\input{ncert/11/16/4/4/defs.tex}
		%
\item 
Out of 100 students, two sections of 40 and 60 are formed. If you and your friend are among the 100 students, what is the probability that
\begin{enumerate}
\item you both enter the same section?
\item you both enter the different sections?
\end{enumerate}
\solution
		%\input{ncert/11/16/4/5/defs.tex}
	\item 
The number lock of a suitcase has 4 wheels each labelled with ten digits i.e. from 0 to 9.The lock opens with a sequence of four digits with no repeats.What is the probability of a person getting the right sequence to open the suitcase.
\\
\solution
		%\input{ncert/11/16/4/10/defs.tex}
		%
\item 
Two cards are drawn at random and without replacement from a pack of 52 playing cards. Find the probability that both the cards are black.
\\
\solution
		%\input{ncert/12/13/2/2/defs.tex}
		\item A box of oranges is inspected by examining three randomly selected oranges drawn without replacement. If all the three oranges are good, the box is approved for sale, otherwise, it is rejected. Find the probability that a box containing 15 oranges out of which 12 are good and 3 are bad ones will be approved for sale.
		\label{ncert/12/13/2/3/defs.tex}
		\item Two balls are drawn at random with replacement from a box containing 10 black and 8 red balls. Find the probability that
		\label{ncert/12/13/2/12}
\begin{enumerate}
\item both balls are red.
\item first ball is black and second is red.
\item one of them is black and other is red.
\end{enumerate}

\item In a hostel, 60\% of the students read Hindi newspaper, 40\% read English newspaper and 20\% read both Hindi and English newspapers. A student is selected at random.
		\label{ncert/12/13/2/15}
\begin{enumerate}
\item Find the probability that she reads neither Hindi nor English newspapers.
\item If she reads Hindi newspaper, find the probability that she reads English newspaper.
\item If she reads English newspaper, find the probability that she reads Hindi newspaper.\\
\end{enumerate}
\item The probability of obtaining an even prime number on each die, when a pair of dice is rolled is 
\begin{enumerate}
    \item $0$ 
    
    \item $\frac{1}{3}$ 
    
    \item $\frac{1}{12}$ 
    
    \item $\frac{1}{36}$ 
\end{enumerate}
\solution
		%\input{ncert/12/13/2/17/defs.tex}
	\item A bag contains 4 red and 4 black balls, another bag contains 2 red and 6 black balls. One of the two bags is selected at random and a ball is drawn from the bag which is found to be red. Find the probability that the ball is drawn from the first bag.
\\
\solution
		%\input{ncert/12/13/3/2/main.tex}
  \item
  Cards with numbers 2 to 101 are placed in a box. A card is selected at random.Find the probability that the card has
\begin{enumerate}[label=(\roman*)]
	\item an even number 
	\item a square number
\end{enumerate}
\solution
%\input{exemplar/10/13/3/32/main.tex}
\item
The king, queen and jack of clubs are removed from a deck of 52 playing cards and then well shuffled. Now one card is drawn at random from the remaining cards.  Determine the probability that the card is
\begin{enumerate}[label=(\roman*)]
\item a club
\item 10 of hearts
\end{enumerate}
\solution
%\input{exemplar/10/13/3/29/main.tex}
\item A team of medical students doing their internship have to assist during surgeries
at a city hospital. The probabilities of surgeries rated as very complex, complex,
routine, simple or very simple are respectively, 0.15, 0.20, 0.31, 0.26, .08. Find
the probabilities that a particular surgery will be rated
\begin{enumerate}
	\item complex or very complex;
	\item neither very complex nor very simple;
	\item routine or complex
	\item routine or simple
\end{enumerate}
\solution
%\input{exemplar/11/16/3/8(1)/main.tex}
\item A card is selected from a pack of 52 cards.
\begin{enumerate}[label=(\alph*)]
    \item How many points are there in the sample space?
    \item Calculate the probability that the card is an ace of spades.
    \item Calculate the probability that the card is (i) an ace and (ii) black card.
\end{enumerate}
\solution
%\input{exemplar/11/16/3/4/main2.tex}
\item The probability that a non leap year selected at random will contain 53 sundays.
\\
\solution
%\input{exemplar/10/13/1/19/main.tex}
\item One of the four persons John, Rita, Aslam or Gurpreet will be promoted next
month. Consequently the sample space consists of four elementary outcomes
S = {John promoted, Rita promoted, Aslam promoted, Gurpreet promoted}
You are told that the chances of John’s promotion is same as that of Gurpreet,
Rita’s chances of promotion are twice as likely as Johns. Aslam’s chances are
four times that of John.
\begin{enumerate}
	\item Determine
	\begin{enumerate}
		\item P (John promoted)
		\item P (Rita promoted)
		\item P (Aslam promoted)
		\item P (Gurpreet promoted)
	\end{enumerate}
	\item If A = {John promoted or Gurpreet promoted}, find P (A).
\end{enumerate}
\solution
%\input{exemplar/11/16/3/10/main.tex}
\item A card is drawn from a deck of 52 cards. Find the probability of getting a king or a heart or a red card.\\
\solution
%\input{exemplar/11/16/3/15/main.tex}
\item The probability that a student will pass his examination is 0.73, the probability of
the student getting a compartment is 0.13, and the probability that the student will
either pass or get compartment is 0.96. State True or False.\\
\solution
%\input{exemplar/11/16/3/31/main.tex}
\item A card is selected from a pack of 52 cards\\
\begin{enumerate}[label=(\alph*)]
\item How many points are there in the sample space?
\item Calculate the probability that the cards is an ace of spades.
\item Calculate the probability that the card is (i) an ace (ii)black card.\\
\end{enumerate}
%\input{ncert/11/16/3/4_1/Prob_4.tex}
\item In a non-leap year, the probability of having 53 tuesdays or 53 wednesdays is\\
\solution
%\input{exemplar/11/16/3/18/main.tex}
\item There are 1000 sealed envelopes in a box, 10 of them contain a cash prize of
Rs 100 each, 100 of them contain a cash prize of Rs 50 each and 200 of them
contain a cash prize of Rs 10 each and rest do not contain any cash prize. If they
are well shuffled and an envelope is picked up out, what is the probability that it
contains no cash prize?\\
\solution
%\input{exemplar/10/13/3/34/main.tex}
\item 
A die is thrown and a card is selected at random from a deck of 52 playing cards. The probability of getting an even number on the die and a spade card.\\
\solution
%\input{exemplar/12/13/3/78/main.tex}
\item
If 4-digit numbers greater than 5,000 are randomly formed from the digits 0, 1, 3, 5, and 7, what is the probability of forming a number divisible by 5 when:
\begin{enumerate}
    \item The digits are repeated?
    \item The repetition of digits is not allowed?
\end{enumerate}
\solution
%\input{ncert/11/16/4/9/main.tex}
\item Consider the probability space $\brak{\Omega, \mathcal{G}, P}$ where $\Omega = [0,2]$ and $\mathcal{G} = \cbrak{\phi, \Omega, [0,1], (1,2]}$. Let $X$ and $Y$ be two functions on $\Omega$ defined as
\begin{align*}
    X(\omega) = 
    \begin{cases}
        1 & \text{if }\omega \in [0, 1]\\
        2 & \text{if }\omega \in (1, 2]
    \end{cases}
\end{align*}
and
\begin{align*}
    Y(\omega) = 
    \begin{cases}
        2 & \text{if }\omega \in [0, 1.5]\\
        3 & \text{if }\omega \in (1.5, 2].
    \end{cases}
\end{align*}
Then which one of the following statements is true?
\begin{enumerate}
    \item [(A)] $X$ is a random variable with respect to $\mathcal{G}$, but $Y$ is not a random variable with respect to $\mathcal{G}$.
    \item [(B)] $Y$ is a random variable with respect to $\mathcal{G}$, but $X$ is not a random variable with respect to $\mathcal{G}$.
    \item [(C)] Neither $X$ nor $Y$ is a random variable with respect to $\mathcal{G}$.
    \item [(D)] Both $X$ and $Y$ are random variables with respect to $\mathcal{G}$.
\end{enumerate} \hfill (GATE ST 2023)\\
\solution
%\input{gate/ST/2023/14/main.tex}
	\item  A die is loaded in such a way that each odd number is twice as likely to occur as
each even number. Find $P(G)$, where $G$ is the event that a number greater than
3 occurs on a single roll of the die.
\\
\solution
		%\input{exemplar/11/16/3/5/main.tex}
	\item All the jacks, queens and kings are removed from a deck of 52 playing cards. The remaining cards are well shuffled and then one card is drawn at random. Giving ace a value 1 similar value for other cards, find the probability that the card has a value 
		\begin{enumerate}
			\item 7
			\item greater than 7
			\item less than 7
		\end{enumerate}
		%\input{exemplar/10/13/3/30/main.tex}
  \item A Lot consists of 48 mobile phones of which 42 are good, 3 have only minor defects and 3 have major defects.Varnika will buy a phone if it is good but the trader will only buy a mobile if it has no major defects. One phone is selected at random from the lot. What is the probability that it is
\begin{enumerate}
	\item acceptable to Varnika?
            \item acceptable to the trader?
\end{enumerate}
\solution
	%\input{exemplar/10/13/3/40/main.tex}
 \item A student says that if you throw a die, it will show up 1 or not 1. Therefore, the probability of getting 1 and the probability of getting 'not 1' each is equal to $\frac{1}{2}$. Is this correct? Give reasons.\\
 \solution
        %\input{exemplar/10/13/2/9/main.tex}
   \item Four candidates A, B, C, D have ap-
plied for the assignment to coach a school cricket
team. If A is twice as likely to be selected as B, and
B and C are given about the same chance of being
selected, while C is twice as likely to be selected
as D, what are the probabilities that
\begin{enumerate}
\item C will be selected?
\item A will not be selected?
\end{enumerate}
	%\input{exemplar/11/16/3/9/main.tex}
 \item A bag contain 24 balls of which $x$ balls are red, $2x$ are white and $3x$ are blue. A ball is selected at random, What is the probability that it is
\begin{enumerate}[label=\alph*)]
\item not red ?
\item white ?
\end{enumerate}
%\input{exemplar/10/13/3/41/main.tex}
If the letters of the word ASSASSINATION are arranged at random. Find the Probability that
\begin{enumerate}[label=(\alph*)]
\item Four $S's$ come consecutively in the word
\item Two  $I's$ and two $N's$ come together
\item All $A's$ are not coming together
\item No two $A's$ are coming together
\end{enumerate}
%\input{exemplar/11/16/3/14/main.tex}
	\item One urn contains two black balls (labelled B1 and B2) and one white ball. A
	second urn contains one black ball and two white balls (labelled W1 and W2).
	Suppose the following experiment is performed. One of the two urns is chosen
	at random. Next a ball is randomly chosen from the urn. Then a second ball is
	chosen at random from the same urn without replacing the first ball.
	
	\begin{enumerate}
	\item What is the probability that two black balls are chosen?
	
	\item What is the probability that two balls of opposite colour are chosen?
	\end{enumerate}
	\solution
	%\input{exemplar/11/16/3/12/main1.tex}
\end{enumerate}

		%
\item 
Out of 100 students, two sections of 40 and 60 are formed. If you and your friend are among the 100 students, what is the probability that
\begin{enumerate}
\item you both enter the same section?
\item you both enter the different sections?
\end{enumerate}
\solution
		%\begin{enumerate}[label=\thesection.\arabic*,ref=\thesection.\theenumi]
	\item One card is drawn from a well-shuffled deck of 52 cards. Find the probability of getting
\begin{enumerate}
\item A king of red colour 
\item A face card 
\item A red face card
\item The jack of hearts
\item A spade
\item The queen of diamonds

\end{enumerate}
\solution
		%\input{ncert/10/15/1/14/main.tex}
	\item Five cards—the ten, jack, queen, king and ace of diamonds, are well-shuffled with their face downwards. One card is then picked up at random.
\begin{enumerate}
\item
What is the probability that the card is the queen? 
\item
If the queen is drawn and put aside, what is the probability that the second card picked up is (a) an ace? (b) a queen?\\
\end{enumerate}
\solution
		%\input{ncert/10/15/1/15/defs.tex}
	\item A bag contains $5$ red balls and some blue balls. If the probability of drawing a blue ball is double that if a red ball, determine the number of blue balls in the bag. 
		\\
\solution
		%\input{ncert/10/15/2/3/defs.tex}
	\item A card is selected from a pack of 52 cards.
 \begin{enumerate}[label=(\alph*)] 
                 \item How many points are there in the sample space?
                 \item Calculate the probability that the card is an ace of spades.
                 \item Calculate the probability that the card is (i) an ace and (ii) black card.
 \end{enumerate}
\solution
		%\input{ncert/11/16/3/4/main.tex}
\item Four cards are drawn from a well-shuffled deck of 52 cards. What is the probability of obtaining 3 diamonds and one spade.
\\
\solution
		%\input{ncert/11/16/4/2/defs.tex}
\item In a certain lottery 10,000 tickets are sold and ten equal prizes are awarded. What is the probability of not getting a prize if you buy (a) one ticket (b) two tickets (c) 10 tickets ?	
\\
\solution
		%\input{ncert/11/16/4/4/defs.tex}
		%
\item 
Out of 100 students, two sections of 40 and 60 are formed. If you and your friend are among the 100 students, what is the probability that
\begin{enumerate}
\item you both enter the same section?
\item you both enter the different sections?
\end{enumerate}
\solution
		%\input{ncert/11/16/4/5/defs.tex}
	\item 
The number lock of a suitcase has 4 wheels each labelled with ten digits i.e. from 0 to 9.The lock opens with a sequence of four digits with no repeats.What is the probability of a person getting the right sequence to open the suitcase.
\\
\solution
		%\input{ncert/11/16/4/10/defs.tex}
		%
\item 
Two cards are drawn at random and without replacement from a pack of 52 playing cards. Find the probability that both the cards are black.
\\
\solution
		%\input{ncert/12/13/2/2/defs.tex}
		\item A box of oranges is inspected by examining three randomly selected oranges drawn without replacement. If all the three oranges are good, the box is approved for sale, otherwise, it is rejected. Find the probability that a box containing 15 oranges out of which 12 are good and 3 are bad ones will be approved for sale.
		\label{ncert/12/13/2/3/defs.tex}
		\item Two balls are drawn at random with replacement from a box containing 10 black and 8 red balls. Find the probability that
		\label{ncert/12/13/2/12}
\begin{enumerate}
\item both balls are red.
\item first ball is black and second is red.
\item one of them is black and other is red.
\end{enumerate}

\item In a hostel, 60\% of the students read Hindi newspaper, 40\% read English newspaper and 20\% read both Hindi and English newspapers. A student is selected at random.
		\label{ncert/12/13/2/15}
\begin{enumerate}
\item Find the probability that she reads neither Hindi nor English newspapers.
\item If she reads Hindi newspaper, find the probability that she reads English newspaper.
\item If she reads English newspaper, find the probability that she reads Hindi newspaper.\\
\end{enumerate}
\item The probability of obtaining an even prime number on each die, when a pair of dice is rolled is 
\begin{enumerate}
    \item $0$ 
    
    \item $\frac{1}{3}$ 
    
    \item $\frac{1}{12}$ 
    
    \item $\frac{1}{36}$ 
\end{enumerate}
\solution
		%\input{ncert/12/13/2/17/defs.tex}
	\item A bag contains 4 red and 4 black balls, another bag contains 2 red and 6 black balls. One of the two bags is selected at random and a ball is drawn from the bag which is found to be red. Find the probability that the ball is drawn from the first bag.
\\
\solution
		%\input{ncert/12/13/3/2/main.tex}
  \item
  Cards with numbers 2 to 101 are placed in a box. A card is selected at random.Find the probability that the card has
\begin{enumerate}[label=(\roman*)]
	\item an even number 
	\item a square number
\end{enumerate}
\solution
%\input{exemplar/10/13/3/32/main.tex}
\item
The king, queen and jack of clubs are removed from a deck of 52 playing cards and then well shuffled. Now one card is drawn at random from the remaining cards.  Determine the probability that the card is
\begin{enumerate}[label=(\roman*)]
\item a club
\item 10 of hearts
\end{enumerate}
\solution
%\input{exemplar/10/13/3/29/main.tex}
\item A team of medical students doing their internship have to assist during surgeries
at a city hospital. The probabilities of surgeries rated as very complex, complex,
routine, simple or very simple are respectively, 0.15, 0.20, 0.31, 0.26, .08. Find
the probabilities that a particular surgery will be rated
\begin{enumerate}
	\item complex or very complex;
	\item neither very complex nor very simple;
	\item routine or complex
	\item routine or simple
\end{enumerate}
\solution
%\input{exemplar/11/16/3/8(1)/main.tex}
\item A card is selected from a pack of 52 cards.
\begin{enumerate}[label=(\alph*)]
    \item How many points are there in the sample space?
    \item Calculate the probability that the card is an ace of spades.
    \item Calculate the probability that the card is (i) an ace and (ii) black card.
\end{enumerate}
\solution
%\input{exemplar/11/16/3/4/main2.tex}
\item The probability that a non leap year selected at random will contain 53 sundays.
\\
\solution
%\input{exemplar/10/13/1/19/main.tex}
\item One of the four persons John, Rita, Aslam or Gurpreet will be promoted next
month. Consequently the sample space consists of four elementary outcomes
S = {John promoted, Rita promoted, Aslam promoted, Gurpreet promoted}
You are told that the chances of John’s promotion is same as that of Gurpreet,
Rita’s chances of promotion are twice as likely as Johns. Aslam’s chances are
four times that of John.
\begin{enumerate}
	\item Determine
	\begin{enumerate}
		\item P (John promoted)
		\item P (Rita promoted)
		\item P (Aslam promoted)
		\item P (Gurpreet promoted)
	\end{enumerate}
	\item If A = {John promoted or Gurpreet promoted}, find P (A).
\end{enumerate}
\solution
%\input{exemplar/11/16/3/10/main.tex}
\item A card is drawn from a deck of 52 cards. Find the probability of getting a king or a heart or a red card.\\
\solution
%\input{exemplar/11/16/3/15/main.tex}
\item The probability that a student will pass his examination is 0.73, the probability of
the student getting a compartment is 0.13, and the probability that the student will
either pass or get compartment is 0.96. State True or False.\\
\solution
%\input{exemplar/11/16/3/31/main.tex}
\item A card is selected from a pack of 52 cards\\
\begin{enumerate}[label=(\alph*)]
\item How many points are there in the sample space?
\item Calculate the probability that the cards is an ace of spades.
\item Calculate the probability that the card is (i) an ace (ii)black card.\\
\end{enumerate}
%\input{ncert/11/16/3/4_1/Prob_4.tex}
\item In a non-leap year, the probability of having 53 tuesdays or 53 wednesdays is\\
\solution
%\input{exemplar/11/16/3/18/main.tex}
\item There are 1000 sealed envelopes in a box, 10 of them contain a cash prize of
Rs 100 each, 100 of them contain a cash prize of Rs 50 each and 200 of them
contain a cash prize of Rs 10 each and rest do not contain any cash prize. If they
are well shuffled and an envelope is picked up out, what is the probability that it
contains no cash prize?\\
\solution
%\input{exemplar/10/13/3/34/main.tex}
\item 
A die is thrown and a card is selected at random from a deck of 52 playing cards. The probability of getting an even number on the die and a spade card.\\
\solution
%\input{exemplar/12/13/3/78/main.tex}
\item
If 4-digit numbers greater than 5,000 are randomly formed from the digits 0, 1, 3, 5, and 7, what is the probability of forming a number divisible by 5 when:
\begin{enumerate}
    \item The digits are repeated?
    \item The repetition of digits is not allowed?
\end{enumerate}
\solution
%\input{ncert/11/16/4/9/main.tex}
\item Consider the probability space $\brak{\Omega, \mathcal{G}, P}$ where $\Omega = [0,2]$ and $\mathcal{G} = \cbrak{\phi, \Omega, [0,1], (1,2]}$. Let $X$ and $Y$ be two functions on $\Omega$ defined as
\begin{align*}
    X(\omega) = 
    \begin{cases}
        1 & \text{if }\omega \in [0, 1]\\
        2 & \text{if }\omega \in (1, 2]
    \end{cases}
\end{align*}
and
\begin{align*}
    Y(\omega) = 
    \begin{cases}
        2 & \text{if }\omega \in [0, 1.5]\\
        3 & \text{if }\omega \in (1.5, 2].
    \end{cases}
\end{align*}
Then which one of the following statements is true?
\begin{enumerate}
    \item [(A)] $X$ is a random variable with respect to $\mathcal{G}$, but $Y$ is not a random variable with respect to $\mathcal{G}$.
    \item [(B)] $Y$ is a random variable with respect to $\mathcal{G}$, but $X$ is not a random variable with respect to $\mathcal{G}$.
    \item [(C)] Neither $X$ nor $Y$ is a random variable with respect to $\mathcal{G}$.
    \item [(D)] Both $X$ and $Y$ are random variables with respect to $\mathcal{G}$.
\end{enumerate} \hfill (GATE ST 2023)\\
\solution
%\input{gate/ST/2023/14/main.tex}
	\item  A die is loaded in such a way that each odd number is twice as likely to occur as
each even number. Find $P(G)$, where $G$ is the event that a number greater than
3 occurs on a single roll of the die.
\\
\solution
		%\input{exemplar/11/16/3/5/main.tex}
	\item All the jacks, queens and kings are removed from a deck of 52 playing cards. The remaining cards are well shuffled and then one card is drawn at random. Giving ace a value 1 similar value for other cards, find the probability that the card has a value 
		\begin{enumerate}
			\item 7
			\item greater than 7
			\item less than 7
		\end{enumerate}
		%\input{exemplar/10/13/3/30/main.tex}
  \item A Lot consists of 48 mobile phones of which 42 are good, 3 have only minor defects and 3 have major defects.Varnika will buy a phone if it is good but the trader will only buy a mobile if it has no major defects. One phone is selected at random from the lot. What is the probability that it is
\begin{enumerate}
	\item acceptable to Varnika?
            \item acceptable to the trader?
\end{enumerate}
\solution
	%\input{exemplar/10/13/3/40/main.tex}
 \item A student says that if you throw a die, it will show up 1 or not 1. Therefore, the probability of getting 1 and the probability of getting 'not 1' each is equal to $\frac{1}{2}$. Is this correct? Give reasons.\\
 \solution
        %\input{exemplar/10/13/2/9/main.tex}
   \item Four candidates A, B, C, D have ap-
plied for the assignment to coach a school cricket
team. If A is twice as likely to be selected as B, and
B and C are given about the same chance of being
selected, while C is twice as likely to be selected
as D, what are the probabilities that
\begin{enumerate}
\item C will be selected?
\item A will not be selected?
\end{enumerate}
	%\input{exemplar/11/16/3/9/main.tex}
 \item A bag contain 24 balls of which $x$ balls are red, $2x$ are white and $3x$ are blue. A ball is selected at random, What is the probability that it is
\begin{enumerate}[label=\alph*)]
\item not red ?
\item white ?
\end{enumerate}
%\input{exemplar/10/13/3/41/main.tex}
If the letters of the word ASSASSINATION are arranged at random. Find the Probability that
\begin{enumerate}[label=(\alph*)]
\item Four $S's$ come consecutively in the word
\item Two  $I's$ and two $N's$ come together
\item All $A's$ are not coming together
\item No two $A's$ are coming together
\end{enumerate}
%\input{exemplar/11/16/3/14/main.tex}
	\item One urn contains two black balls (labelled B1 and B2) and one white ball. A
	second urn contains one black ball and two white balls (labelled W1 and W2).
	Suppose the following experiment is performed. One of the two urns is chosen
	at random. Next a ball is randomly chosen from the urn. Then a second ball is
	chosen at random from the same urn without replacing the first ball.
	
	\begin{enumerate}
	\item What is the probability that two black balls are chosen?
	
	\item What is the probability that two balls of opposite colour are chosen?
	\end{enumerate}
	\solution
	%\input{exemplar/11/16/3/12/main1.tex}
\end{enumerate}

	\item 
The number lock of a suitcase has 4 wheels each labelled with ten digits i.e. from 0 to 9.The lock opens with a sequence of four digits with no repeats.What is the probability of a person getting the right sequence to open the suitcase.
\\
\solution
		%\begin{enumerate}[label=\thesection.\arabic*,ref=\thesection.\theenumi]
	\item One card is drawn from a well-shuffled deck of 52 cards. Find the probability of getting
\begin{enumerate}
\item A king of red colour 
\item A face card 
\item A red face card
\item The jack of hearts
\item A spade
\item The queen of diamonds

\end{enumerate}
\solution
		%\input{ncert/10/15/1/14/main.tex}
	\item Five cards—the ten, jack, queen, king and ace of diamonds, are well-shuffled with their face downwards. One card is then picked up at random.
\begin{enumerate}
\item
What is the probability that the card is the queen? 
\item
If the queen is drawn and put aside, what is the probability that the second card picked up is (a) an ace? (b) a queen?\\
\end{enumerate}
\solution
		%\input{ncert/10/15/1/15/defs.tex}
	\item A bag contains $5$ red balls and some blue balls. If the probability of drawing a blue ball is double that if a red ball, determine the number of blue balls in the bag. 
		\\
\solution
		%\input{ncert/10/15/2/3/defs.tex}
	\item A card is selected from a pack of 52 cards.
 \begin{enumerate}[label=(\alph*)] 
                 \item How many points are there in the sample space?
                 \item Calculate the probability that the card is an ace of spades.
                 \item Calculate the probability that the card is (i) an ace and (ii) black card.
 \end{enumerate}
\solution
		%\input{ncert/11/16/3/4/main.tex}
\item Four cards are drawn from a well-shuffled deck of 52 cards. What is the probability of obtaining 3 diamonds and one spade.
\\
\solution
		%\input{ncert/11/16/4/2/defs.tex}
\item In a certain lottery 10,000 tickets are sold and ten equal prizes are awarded. What is the probability of not getting a prize if you buy (a) one ticket (b) two tickets (c) 10 tickets ?	
\\
\solution
		%\input{ncert/11/16/4/4/defs.tex}
		%
\item 
Out of 100 students, two sections of 40 and 60 are formed. If you and your friend are among the 100 students, what is the probability that
\begin{enumerate}
\item you both enter the same section?
\item you both enter the different sections?
\end{enumerate}
\solution
		%\input{ncert/11/16/4/5/defs.tex}
	\item 
The number lock of a suitcase has 4 wheels each labelled with ten digits i.e. from 0 to 9.The lock opens with a sequence of four digits with no repeats.What is the probability of a person getting the right sequence to open the suitcase.
\\
\solution
		%\input{ncert/11/16/4/10/defs.tex}
		%
\item 
Two cards are drawn at random and without replacement from a pack of 52 playing cards. Find the probability that both the cards are black.
\\
\solution
		%\input{ncert/12/13/2/2/defs.tex}
		\item A box of oranges is inspected by examining three randomly selected oranges drawn without replacement. If all the three oranges are good, the box is approved for sale, otherwise, it is rejected. Find the probability that a box containing 15 oranges out of which 12 are good and 3 are bad ones will be approved for sale.
		\label{ncert/12/13/2/3/defs.tex}
		\item Two balls are drawn at random with replacement from a box containing 10 black and 8 red balls. Find the probability that
		\label{ncert/12/13/2/12}
\begin{enumerate}
\item both balls are red.
\item first ball is black and second is red.
\item one of them is black and other is red.
\end{enumerate}

\item In a hostel, 60\% of the students read Hindi newspaper, 40\% read English newspaper and 20\% read both Hindi and English newspapers. A student is selected at random.
		\label{ncert/12/13/2/15}
\begin{enumerate}
\item Find the probability that she reads neither Hindi nor English newspapers.
\item If she reads Hindi newspaper, find the probability that she reads English newspaper.
\item If she reads English newspaper, find the probability that she reads Hindi newspaper.\\
\end{enumerate}
\item The probability of obtaining an even prime number on each die, when a pair of dice is rolled is 
\begin{enumerate}
    \item $0$ 
    
    \item $\frac{1}{3}$ 
    
    \item $\frac{1}{12}$ 
    
    \item $\frac{1}{36}$ 
\end{enumerate}
\solution
		%\input{ncert/12/13/2/17/defs.tex}
	\item A bag contains 4 red and 4 black balls, another bag contains 2 red and 6 black balls. One of the two bags is selected at random and a ball is drawn from the bag which is found to be red. Find the probability that the ball is drawn from the first bag.
\\
\solution
		%\input{ncert/12/13/3/2/main.tex}
  \item
  Cards with numbers 2 to 101 are placed in a box. A card is selected at random.Find the probability that the card has
\begin{enumerate}[label=(\roman*)]
	\item an even number 
	\item a square number
\end{enumerate}
\solution
%\input{exemplar/10/13/3/32/main.tex}
\item
The king, queen and jack of clubs are removed from a deck of 52 playing cards and then well shuffled. Now one card is drawn at random from the remaining cards.  Determine the probability that the card is
\begin{enumerate}[label=(\roman*)]
\item a club
\item 10 of hearts
\end{enumerate}
\solution
%\input{exemplar/10/13/3/29/main.tex}
\item A team of medical students doing their internship have to assist during surgeries
at a city hospital. The probabilities of surgeries rated as very complex, complex,
routine, simple or very simple are respectively, 0.15, 0.20, 0.31, 0.26, .08. Find
the probabilities that a particular surgery will be rated
\begin{enumerate}
	\item complex or very complex;
	\item neither very complex nor very simple;
	\item routine or complex
	\item routine or simple
\end{enumerate}
\solution
%\input{exemplar/11/16/3/8(1)/main.tex}
\item A card is selected from a pack of 52 cards.
\begin{enumerate}[label=(\alph*)]
    \item How many points are there in the sample space?
    \item Calculate the probability that the card is an ace of spades.
    \item Calculate the probability that the card is (i) an ace and (ii) black card.
\end{enumerate}
\solution
%\input{exemplar/11/16/3/4/main2.tex}
\item The probability that a non leap year selected at random will contain 53 sundays.
\\
\solution
%\input{exemplar/10/13/1/19/main.tex}
\item One of the four persons John, Rita, Aslam or Gurpreet will be promoted next
month. Consequently the sample space consists of four elementary outcomes
S = {John promoted, Rita promoted, Aslam promoted, Gurpreet promoted}
You are told that the chances of John’s promotion is same as that of Gurpreet,
Rita’s chances of promotion are twice as likely as Johns. Aslam’s chances are
four times that of John.
\begin{enumerate}
	\item Determine
	\begin{enumerate}
		\item P (John promoted)
		\item P (Rita promoted)
		\item P (Aslam promoted)
		\item P (Gurpreet promoted)
	\end{enumerate}
	\item If A = {John promoted or Gurpreet promoted}, find P (A).
\end{enumerate}
\solution
%\input{exemplar/11/16/3/10/main.tex}
\item A card is drawn from a deck of 52 cards. Find the probability of getting a king or a heart or a red card.\\
\solution
%\input{exemplar/11/16/3/15/main.tex}
\item The probability that a student will pass his examination is 0.73, the probability of
the student getting a compartment is 0.13, and the probability that the student will
either pass or get compartment is 0.96. State True or False.\\
\solution
%\input{exemplar/11/16/3/31/main.tex}
\item A card is selected from a pack of 52 cards\\
\begin{enumerate}[label=(\alph*)]
\item How many points are there in the sample space?
\item Calculate the probability that the cards is an ace of spades.
\item Calculate the probability that the card is (i) an ace (ii)black card.\\
\end{enumerate}
%\input{ncert/11/16/3/4_1/Prob_4.tex}
\item In a non-leap year, the probability of having 53 tuesdays or 53 wednesdays is\\
\solution
%\input{exemplar/11/16/3/18/main.tex}
\item There are 1000 sealed envelopes in a box, 10 of them contain a cash prize of
Rs 100 each, 100 of them contain a cash prize of Rs 50 each and 200 of them
contain a cash prize of Rs 10 each and rest do not contain any cash prize. If they
are well shuffled and an envelope is picked up out, what is the probability that it
contains no cash prize?\\
\solution
%\input{exemplar/10/13/3/34/main.tex}
\item 
A die is thrown and a card is selected at random from a deck of 52 playing cards. The probability of getting an even number on the die and a spade card.\\
\solution
%\input{exemplar/12/13/3/78/main.tex}
\item
If 4-digit numbers greater than 5,000 are randomly formed from the digits 0, 1, 3, 5, and 7, what is the probability of forming a number divisible by 5 when:
\begin{enumerate}
    \item The digits are repeated?
    \item The repetition of digits is not allowed?
\end{enumerate}
\solution
%\input{ncert/11/16/4/9/main.tex}
\item Consider the probability space $\brak{\Omega, \mathcal{G}, P}$ where $\Omega = [0,2]$ and $\mathcal{G} = \cbrak{\phi, \Omega, [0,1], (1,2]}$. Let $X$ and $Y$ be two functions on $\Omega$ defined as
\begin{align*}
    X(\omega) = 
    \begin{cases}
        1 & \text{if }\omega \in [0, 1]\\
        2 & \text{if }\omega \in (1, 2]
    \end{cases}
\end{align*}
and
\begin{align*}
    Y(\omega) = 
    \begin{cases}
        2 & \text{if }\omega \in [0, 1.5]\\
        3 & \text{if }\omega \in (1.5, 2].
    \end{cases}
\end{align*}
Then which one of the following statements is true?
\begin{enumerate}
    \item [(A)] $X$ is a random variable with respect to $\mathcal{G}$, but $Y$ is not a random variable with respect to $\mathcal{G}$.
    \item [(B)] $Y$ is a random variable with respect to $\mathcal{G}$, but $X$ is not a random variable with respect to $\mathcal{G}$.
    \item [(C)] Neither $X$ nor $Y$ is a random variable with respect to $\mathcal{G}$.
    \item [(D)] Both $X$ and $Y$ are random variables with respect to $\mathcal{G}$.
\end{enumerate} \hfill (GATE ST 2023)\\
\solution
%\input{gate/ST/2023/14/main.tex}
	\item  A die is loaded in such a way that each odd number is twice as likely to occur as
each even number. Find $P(G)$, where $G$ is the event that a number greater than
3 occurs on a single roll of the die.
\\
\solution
		%\input{exemplar/11/16/3/5/main.tex}
	\item All the jacks, queens and kings are removed from a deck of 52 playing cards. The remaining cards are well shuffled and then one card is drawn at random. Giving ace a value 1 similar value for other cards, find the probability that the card has a value 
		\begin{enumerate}
			\item 7
			\item greater than 7
			\item less than 7
		\end{enumerate}
		%\input{exemplar/10/13/3/30/main.tex}
  \item A Lot consists of 48 mobile phones of which 42 are good, 3 have only minor defects and 3 have major defects.Varnika will buy a phone if it is good but the trader will only buy a mobile if it has no major defects. One phone is selected at random from the lot. What is the probability that it is
\begin{enumerate}
	\item acceptable to Varnika?
            \item acceptable to the trader?
\end{enumerate}
\solution
	%\input{exemplar/10/13/3/40/main.tex}
 \item A student says that if you throw a die, it will show up 1 or not 1. Therefore, the probability of getting 1 and the probability of getting 'not 1' each is equal to $\frac{1}{2}$. Is this correct? Give reasons.\\
 \solution
        %\input{exemplar/10/13/2/9/main.tex}
   \item Four candidates A, B, C, D have ap-
plied for the assignment to coach a school cricket
team. If A is twice as likely to be selected as B, and
B and C are given about the same chance of being
selected, while C is twice as likely to be selected
as D, what are the probabilities that
\begin{enumerate}
\item C will be selected?
\item A will not be selected?
\end{enumerate}
	%\input{exemplar/11/16/3/9/main.tex}
 \item A bag contain 24 balls of which $x$ balls are red, $2x$ are white and $3x$ are blue. A ball is selected at random, What is the probability that it is
\begin{enumerate}[label=\alph*)]
\item not red ?
\item white ?
\end{enumerate}
%\input{exemplar/10/13/3/41/main.tex}
If the letters of the word ASSASSINATION are arranged at random. Find the Probability that
\begin{enumerate}[label=(\alph*)]
\item Four $S's$ come consecutively in the word
\item Two  $I's$ and two $N's$ come together
\item All $A's$ are not coming together
\item No two $A's$ are coming together
\end{enumerate}
%\input{exemplar/11/16/3/14/main.tex}
	\item One urn contains two black balls (labelled B1 and B2) and one white ball. A
	second urn contains one black ball and two white balls (labelled W1 and W2).
	Suppose the following experiment is performed. One of the two urns is chosen
	at random. Next a ball is randomly chosen from the urn. Then a second ball is
	chosen at random from the same urn without replacing the first ball.
	
	\begin{enumerate}
	\item What is the probability that two black balls are chosen?
	
	\item What is the probability that two balls of opposite colour are chosen?
	\end{enumerate}
	\solution
	%\input{exemplar/11/16/3/12/main1.tex}
\end{enumerate}

		%
\item 
Two cards are drawn at random and without replacement from a pack of 52 playing cards. Find the probability that both the cards are black.
\\
\solution
		%\begin{enumerate}[label=\thesection.\arabic*,ref=\thesection.\theenumi]
	\item One card is drawn from a well-shuffled deck of 52 cards. Find the probability of getting
\begin{enumerate}
\item A king of red colour 
\item A face card 
\item A red face card
\item The jack of hearts
\item A spade
\item The queen of diamonds

\end{enumerate}
\solution
		%\input{ncert/10/15/1/14/main.tex}
	\item Five cards—the ten, jack, queen, king and ace of diamonds, are well-shuffled with their face downwards. One card is then picked up at random.
\begin{enumerate}
\item
What is the probability that the card is the queen? 
\item
If the queen is drawn and put aside, what is the probability that the second card picked up is (a) an ace? (b) a queen?\\
\end{enumerate}
\solution
		%\input{ncert/10/15/1/15/defs.tex}
	\item A bag contains $5$ red balls and some blue balls. If the probability of drawing a blue ball is double that if a red ball, determine the number of blue balls in the bag. 
		\\
\solution
		%\input{ncert/10/15/2/3/defs.tex}
	\item A card is selected from a pack of 52 cards.
 \begin{enumerate}[label=(\alph*)] 
                 \item How many points are there in the sample space?
                 \item Calculate the probability that the card is an ace of spades.
                 \item Calculate the probability that the card is (i) an ace and (ii) black card.
 \end{enumerate}
\solution
		%\input{ncert/11/16/3/4/main.tex}
\item Four cards are drawn from a well-shuffled deck of 52 cards. What is the probability of obtaining 3 diamonds and one spade.
\\
\solution
		%\input{ncert/11/16/4/2/defs.tex}
\item In a certain lottery 10,000 tickets are sold and ten equal prizes are awarded. What is the probability of not getting a prize if you buy (a) one ticket (b) two tickets (c) 10 tickets ?	
\\
\solution
		%\input{ncert/11/16/4/4/defs.tex}
		%
\item 
Out of 100 students, two sections of 40 and 60 are formed. If you and your friend are among the 100 students, what is the probability that
\begin{enumerate}
\item you both enter the same section?
\item you both enter the different sections?
\end{enumerate}
\solution
		%\input{ncert/11/16/4/5/defs.tex}
	\item 
The number lock of a suitcase has 4 wheels each labelled with ten digits i.e. from 0 to 9.The lock opens with a sequence of four digits with no repeats.What is the probability of a person getting the right sequence to open the suitcase.
\\
\solution
		%\input{ncert/11/16/4/10/defs.tex}
		%
\item 
Two cards are drawn at random and without replacement from a pack of 52 playing cards. Find the probability that both the cards are black.
\\
\solution
		%\input{ncert/12/13/2/2/defs.tex}
		\item A box of oranges is inspected by examining three randomly selected oranges drawn without replacement. If all the three oranges are good, the box is approved for sale, otherwise, it is rejected. Find the probability that a box containing 15 oranges out of which 12 are good and 3 are bad ones will be approved for sale.
		\label{ncert/12/13/2/3/defs.tex}
		\item Two balls are drawn at random with replacement from a box containing 10 black and 8 red balls. Find the probability that
		\label{ncert/12/13/2/12}
\begin{enumerate}
\item both balls are red.
\item first ball is black and second is red.
\item one of them is black and other is red.
\end{enumerate}

\item In a hostel, 60\% of the students read Hindi newspaper, 40\% read English newspaper and 20\% read both Hindi and English newspapers. A student is selected at random.
		\label{ncert/12/13/2/15}
\begin{enumerate}
\item Find the probability that she reads neither Hindi nor English newspapers.
\item If she reads Hindi newspaper, find the probability that she reads English newspaper.
\item If she reads English newspaper, find the probability that she reads Hindi newspaper.\\
\end{enumerate}
\item The probability of obtaining an even prime number on each die, when a pair of dice is rolled is 
\begin{enumerate}
    \item $0$ 
    
    \item $\frac{1}{3}$ 
    
    \item $\frac{1}{12}$ 
    
    \item $\frac{1}{36}$ 
\end{enumerate}
\solution
		%\input{ncert/12/13/2/17/defs.tex}
	\item A bag contains 4 red and 4 black balls, another bag contains 2 red and 6 black balls. One of the two bags is selected at random and a ball is drawn from the bag which is found to be red. Find the probability that the ball is drawn from the first bag.
\\
\solution
		%\input{ncert/12/13/3/2/main.tex}
  \item
  Cards with numbers 2 to 101 are placed in a box. A card is selected at random.Find the probability that the card has
\begin{enumerate}[label=(\roman*)]
	\item an even number 
	\item a square number
\end{enumerate}
\solution
%\input{exemplar/10/13/3/32/main.tex}
\item
The king, queen and jack of clubs are removed from a deck of 52 playing cards and then well shuffled. Now one card is drawn at random from the remaining cards.  Determine the probability that the card is
\begin{enumerate}[label=(\roman*)]
\item a club
\item 10 of hearts
\end{enumerate}
\solution
%\input{exemplar/10/13/3/29/main.tex}
\item A team of medical students doing their internship have to assist during surgeries
at a city hospital. The probabilities of surgeries rated as very complex, complex,
routine, simple or very simple are respectively, 0.15, 0.20, 0.31, 0.26, .08. Find
the probabilities that a particular surgery will be rated
\begin{enumerate}
	\item complex or very complex;
	\item neither very complex nor very simple;
	\item routine or complex
	\item routine or simple
\end{enumerate}
\solution
%\input{exemplar/11/16/3/8(1)/main.tex}
\item A card is selected from a pack of 52 cards.
\begin{enumerate}[label=(\alph*)]
    \item How many points are there in the sample space?
    \item Calculate the probability that the card is an ace of spades.
    \item Calculate the probability that the card is (i) an ace and (ii) black card.
\end{enumerate}
\solution
%\input{exemplar/11/16/3/4/main2.tex}
\item The probability that a non leap year selected at random will contain 53 sundays.
\\
\solution
%\input{exemplar/10/13/1/19/main.tex}
\item One of the four persons John, Rita, Aslam or Gurpreet will be promoted next
month. Consequently the sample space consists of four elementary outcomes
S = {John promoted, Rita promoted, Aslam promoted, Gurpreet promoted}
You are told that the chances of John’s promotion is same as that of Gurpreet,
Rita’s chances of promotion are twice as likely as Johns. Aslam’s chances are
four times that of John.
\begin{enumerate}
	\item Determine
	\begin{enumerate}
		\item P (John promoted)
		\item P (Rita promoted)
		\item P (Aslam promoted)
		\item P (Gurpreet promoted)
	\end{enumerate}
	\item If A = {John promoted or Gurpreet promoted}, find P (A).
\end{enumerate}
\solution
%\input{exemplar/11/16/3/10/main.tex}
\item A card is drawn from a deck of 52 cards. Find the probability of getting a king or a heart or a red card.\\
\solution
%\input{exemplar/11/16/3/15/main.tex}
\item The probability that a student will pass his examination is 0.73, the probability of
the student getting a compartment is 0.13, and the probability that the student will
either pass or get compartment is 0.96. State True or False.\\
\solution
%\input{exemplar/11/16/3/31/main.tex}
\item A card is selected from a pack of 52 cards\\
\begin{enumerate}[label=(\alph*)]
\item How many points are there in the sample space?
\item Calculate the probability that the cards is an ace of spades.
\item Calculate the probability that the card is (i) an ace (ii)black card.\\
\end{enumerate}
%\input{ncert/11/16/3/4_1/Prob_4.tex}
\item In a non-leap year, the probability of having 53 tuesdays or 53 wednesdays is\\
\solution
%\input{exemplar/11/16/3/18/main.tex}
\item There are 1000 sealed envelopes in a box, 10 of them contain a cash prize of
Rs 100 each, 100 of them contain a cash prize of Rs 50 each and 200 of them
contain a cash prize of Rs 10 each and rest do not contain any cash prize. If they
are well shuffled and an envelope is picked up out, what is the probability that it
contains no cash prize?\\
\solution
%\input{exemplar/10/13/3/34/main.tex}
\item 
A die is thrown and a card is selected at random from a deck of 52 playing cards. The probability of getting an even number on the die and a spade card.\\
\solution
%\input{exemplar/12/13/3/78/main.tex}
\item
If 4-digit numbers greater than 5,000 are randomly formed from the digits 0, 1, 3, 5, and 7, what is the probability of forming a number divisible by 5 when:
\begin{enumerate}
    \item The digits are repeated?
    \item The repetition of digits is not allowed?
\end{enumerate}
\solution
%\input{ncert/11/16/4/9/main.tex}
\item Consider the probability space $\brak{\Omega, \mathcal{G}, P}$ where $\Omega = [0,2]$ and $\mathcal{G} = \cbrak{\phi, \Omega, [0,1], (1,2]}$. Let $X$ and $Y$ be two functions on $\Omega$ defined as
\begin{align*}
    X(\omega) = 
    \begin{cases}
        1 & \text{if }\omega \in [0, 1]\\
        2 & \text{if }\omega \in (1, 2]
    \end{cases}
\end{align*}
and
\begin{align*}
    Y(\omega) = 
    \begin{cases}
        2 & \text{if }\omega \in [0, 1.5]\\
        3 & \text{if }\omega \in (1.5, 2].
    \end{cases}
\end{align*}
Then which one of the following statements is true?
\begin{enumerate}
    \item [(A)] $X$ is a random variable with respect to $\mathcal{G}$, but $Y$ is not a random variable with respect to $\mathcal{G}$.
    \item [(B)] $Y$ is a random variable with respect to $\mathcal{G}$, but $X$ is not a random variable with respect to $\mathcal{G}$.
    \item [(C)] Neither $X$ nor $Y$ is a random variable with respect to $\mathcal{G}$.
    \item [(D)] Both $X$ and $Y$ are random variables with respect to $\mathcal{G}$.
\end{enumerate} \hfill (GATE ST 2023)\\
\solution
%\input{gate/ST/2023/14/main.tex}
	\item  A die is loaded in such a way that each odd number is twice as likely to occur as
each even number. Find $P(G)$, where $G$ is the event that a number greater than
3 occurs on a single roll of the die.
\\
\solution
		%\input{exemplar/11/16/3/5/main.tex}
	\item All the jacks, queens and kings are removed from a deck of 52 playing cards. The remaining cards are well shuffled and then one card is drawn at random. Giving ace a value 1 similar value for other cards, find the probability that the card has a value 
		\begin{enumerate}
			\item 7
			\item greater than 7
			\item less than 7
		\end{enumerate}
		%\input{exemplar/10/13/3/30/main.tex}
  \item A Lot consists of 48 mobile phones of which 42 are good, 3 have only minor defects and 3 have major defects.Varnika will buy a phone if it is good but the trader will only buy a mobile if it has no major defects. One phone is selected at random from the lot. What is the probability that it is
\begin{enumerate}
	\item acceptable to Varnika?
            \item acceptable to the trader?
\end{enumerate}
\solution
	%\input{exemplar/10/13/3/40/main.tex}
 \item A student says that if you throw a die, it will show up 1 or not 1. Therefore, the probability of getting 1 and the probability of getting 'not 1' each is equal to $\frac{1}{2}$. Is this correct? Give reasons.\\
 \solution
        %\input{exemplar/10/13/2/9/main.tex}
   \item Four candidates A, B, C, D have ap-
plied for the assignment to coach a school cricket
team. If A is twice as likely to be selected as B, and
B and C are given about the same chance of being
selected, while C is twice as likely to be selected
as D, what are the probabilities that
\begin{enumerate}
\item C will be selected?
\item A will not be selected?
\end{enumerate}
	%\input{exemplar/11/16/3/9/main.tex}
 \item A bag contain 24 balls of which $x$ balls are red, $2x$ are white and $3x$ are blue. A ball is selected at random, What is the probability that it is
\begin{enumerate}[label=\alph*)]
\item not red ?
\item white ?
\end{enumerate}
%\input{exemplar/10/13/3/41/main.tex}
If the letters of the word ASSASSINATION are arranged at random. Find the Probability that
\begin{enumerate}[label=(\alph*)]
\item Four $S's$ come consecutively in the word
\item Two  $I's$ and two $N's$ come together
\item All $A's$ are not coming together
\item No two $A's$ are coming together
\end{enumerate}
%\input{exemplar/11/16/3/14/main.tex}
	\item One urn contains two black balls (labelled B1 and B2) and one white ball. A
	second urn contains one black ball and two white balls (labelled W1 and W2).
	Suppose the following experiment is performed. One of the two urns is chosen
	at random. Next a ball is randomly chosen from the urn. Then a second ball is
	chosen at random from the same urn without replacing the first ball.
	
	\begin{enumerate}
	\item What is the probability that two black balls are chosen?
	
	\item What is the probability that two balls of opposite colour are chosen?
	\end{enumerate}
	\solution
	%\input{exemplar/11/16/3/12/main1.tex}
\end{enumerate}

		\item A box of oranges is inspected by examining three randomly selected oranges drawn without replacement. If all the three oranges are good, the box is approved for sale, otherwise, it is rejected. Find the probability that a box containing 15 oranges out of which 12 are good and 3 are bad ones will be approved for sale.
		\label{ncert/12/13/2/3/defs.tex}
		\item Two balls are drawn at random with replacement from a box containing 10 black and 8 red balls. Find the probability that
		\label{ncert/12/13/2/12}
\begin{enumerate}
\item both balls are red.
\item first ball is black and second is red.
\item one of them is black and other is red.
\end{enumerate}

\item In a hostel, 60\% of the students read Hindi newspaper, 40\% read English newspaper and 20\% read both Hindi and English newspapers. A student is selected at random.
		\label{ncert/12/13/2/15}
\begin{enumerate}
\item Find the probability that she reads neither Hindi nor English newspapers.
\item If she reads Hindi newspaper, find the probability that she reads English newspaper.
\item If she reads English newspaper, find the probability that she reads Hindi newspaper.\\
\end{enumerate}
\item The probability of obtaining an even prime number on each die, when a pair of dice is rolled is 
\begin{enumerate}
    \item $0$ 
    
    \item $\frac{1}{3}$ 
    
    \item $\frac{1}{12}$ 
    
    \item $\frac{1}{36}$ 
\end{enumerate}
\solution
		%\begin{enumerate}[label=\thesection.\arabic*,ref=\thesection.\theenumi]
	\item One card is drawn from a well-shuffled deck of 52 cards. Find the probability of getting
\begin{enumerate}
\item A king of red colour 
\item A face card 
\item A red face card
\item The jack of hearts
\item A spade
\item The queen of diamonds

\end{enumerate}
\solution
		%\input{ncert/10/15/1/14/main.tex}
	\item Five cards—the ten, jack, queen, king and ace of diamonds, are well-shuffled with their face downwards. One card is then picked up at random.
\begin{enumerate}
\item
What is the probability that the card is the queen? 
\item
If the queen is drawn and put aside, what is the probability that the second card picked up is (a) an ace? (b) a queen?\\
\end{enumerate}
\solution
		%\input{ncert/10/15/1/15/defs.tex}
	\item A bag contains $5$ red balls and some blue balls. If the probability of drawing a blue ball is double that if a red ball, determine the number of blue balls in the bag. 
		\\
\solution
		%\input{ncert/10/15/2/3/defs.tex}
	\item A card is selected from a pack of 52 cards.
 \begin{enumerate}[label=(\alph*)] 
                 \item How many points are there in the sample space?
                 \item Calculate the probability that the card is an ace of spades.
                 \item Calculate the probability that the card is (i) an ace and (ii) black card.
 \end{enumerate}
\solution
		%\input{ncert/11/16/3/4/main.tex}
\item Four cards are drawn from a well-shuffled deck of 52 cards. What is the probability of obtaining 3 diamonds and one spade.
\\
\solution
		%\input{ncert/11/16/4/2/defs.tex}
\item In a certain lottery 10,000 tickets are sold and ten equal prizes are awarded. What is the probability of not getting a prize if you buy (a) one ticket (b) two tickets (c) 10 tickets ?	
\\
\solution
		%\input{ncert/11/16/4/4/defs.tex}
		%
\item 
Out of 100 students, two sections of 40 and 60 are formed. If you and your friend are among the 100 students, what is the probability that
\begin{enumerate}
\item you both enter the same section?
\item you both enter the different sections?
\end{enumerate}
\solution
		%\input{ncert/11/16/4/5/defs.tex}
	\item 
The number lock of a suitcase has 4 wheels each labelled with ten digits i.e. from 0 to 9.The lock opens with a sequence of four digits with no repeats.What is the probability of a person getting the right sequence to open the suitcase.
\\
\solution
		%\input{ncert/11/16/4/10/defs.tex}
		%
\item 
Two cards are drawn at random and without replacement from a pack of 52 playing cards. Find the probability that both the cards are black.
\\
\solution
		%\input{ncert/12/13/2/2/defs.tex}
		\item A box of oranges is inspected by examining three randomly selected oranges drawn without replacement. If all the three oranges are good, the box is approved for sale, otherwise, it is rejected. Find the probability that a box containing 15 oranges out of which 12 are good and 3 are bad ones will be approved for sale.
		\label{ncert/12/13/2/3/defs.tex}
		\item Two balls are drawn at random with replacement from a box containing 10 black and 8 red balls. Find the probability that
		\label{ncert/12/13/2/12}
\begin{enumerate}
\item both balls are red.
\item first ball is black and second is red.
\item one of them is black and other is red.
\end{enumerate}

\item In a hostel, 60\% of the students read Hindi newspaper, 40\% read English newspaper and 20\% read both Hindi and English newspapers. A student is selected at random.
		\label{ncert/12/13/2/15}
\begin{enumerate}
\item Find the probability that she reads neither Hindi nor English newspapers.
\item If she reads Hindi newspaper, find the probability that she reads English newspaper.
\item If she reads English newspaper, find the probability that she reads Hindi newspaper.\\
\end{enumerate}
\item The probability of obtaining an even prime number on each die, when a pair of dice is rolled is 
\begin{enumerate}
    \item $0$ 
    
    \item $\frac{1}{3}$ 
    
    \item $\frac{1}{12}$ 
    
    \item $\frac{1}{36}$ 
\end{enumerate}
\solution
		%\input{ncert/12/13/2/17/defs.tex}
	\item A bag contains 4 red and 4 black balls, another bag contains 2 red and 6 black balls. One of the two bags is selected at random and a ball is drawn from the bag which is found to be red. Find the probability that the ball is drawn from the first bag.
\\
\solution
		%\input{ncert/12/13/3/2/main.tex}
  \item
  Cards with numbers 2 to 101 are placed in a box. A card is selected at random.Find the probability that the card has
\begin{enumerate}[label=(\roman*)]
	\item an even number 
	\item a square number
\end{enumerate}
\solution
%\input{exemplar/10/13/3/32/main.tex}
\item
The king, queen and jack of clubs are removed from a deck of 52 playing cards and then well shuffled. Now one card is drawn at random from the remaining cards.  Determine the probability that the card is
\begin{enumerate}[label=(\roman*)]
\item a club
\item 10 of hearts
\end{enumerate}
\solution
%\input{exemplar/10/13/3/29/main.tex}
\item A team of medical students doing their internship have to assist during surgeries
at a city hospital. The probabilities of surgeries rated as very complex, complex,
routine, simple or very simple are respectively, 0.15, 0.20, 0.31, 0.26, .08. Find
the probabilities that a particular surgery will be rated
\begin{enumerate}
	\item complex or very complex;
	\item neither very complex nor very simple;
	\item routine or complex
	\item routine or simple
\end{enumerate}
\solution
%\input{exemplar/11/16/3/8(1)/main.tex}
\item A card is selected from a pack of 52 cards.
\begin{enumerate}[label=(\alph*)]
    \item How many points are there in the sample space?
    \item Calculate the probability that the card is an ace of spades.
    \item Calculate the probability that the card is (i) an ace and (ii) black card.
\end{enumerate}
\solution
%\input{exemplar/11/16/3/4/main2.tex}
\item The probability that a non leap year selected at random will contain 53 sundays.
\\
\solution
%\input{exemplar/10/13/1/19/main.tex}
\item One of the four persons John, Rita, Aslam or Gurpreet will be promoted next
month. Consequently the sample space consists of four elementary outcomes
S = {John promoted, Rita promoted, Aslam promoted, Gurpreet promoted}
You are told that the chances of John’s promotion is same as that of Gurpreet,
Rita’s chances of promotion are twice as likely as Johns. Aslam’s chances are
four times that of John.
\begin{enumerate}
	\item Determine
	\begin{enumerate}
		\item P (John promoted)
		\item P (Rita promoted)
		\item P (Aslam promoted)
		\item P (Gurpreet promoted)
	\end{enumerate}
	\item If A = {John promoted or Gurpreet promoted}, find P (A).
\end{enumerate}
\solution
%\input{exemplar/11/16/3/10/main.tex}
\item A card is drawn from a deck of 52 cards. Find the probability of getting a king or a heart or a red card.\\
\solution
%\input{exemplar/11/16/3/15/main.tex}
\item The probability that a student will pass his examination is 0.73, the probability of
the student getting a compartment is 0.13, and the probability that the student will
either pass or get compartment is 0.96. State True or False.\\
\solution
%\input{exemplar/11/16/3/31/main.tex}
\item A card is selected from a pack of 52 cards\\
\begin{enumerate}[label=(\alph*)]
\item How many points are there in the sample space?
\item Calculate the probability that the cards is an ace of spades.
\item Calculate the probability that the card is (i) an ace (ii)black card.\\
\end{enumerate}
%\input{ncert/11/16/3/4_1/Prob_4.tex}
\item In a non-leap year, the probability of having 53 tuesdays or 53 wednesdays is\\
\solution
%\input{exemplar/11/16/3/18/main.tex}
\item There are 1000 sealed envelopes in a box, 10 of them contain a cash prize of
Rs 100 each, 100 of them contain a cash prize of Rs 50 each and 200 of them
contain a cash prize of Rs 10 each and rest do not contain any cash prize. If they
are well shuffled and an envelope is picked up out, what is the probability that it
contains no cash prize?\\
\solution
%\input{exemplar/10/13/3/34/main.tex}
\item 
A die is thrown and a card is selected at random from a deck of 52 playing cards. The probability of getting an even number on the die and a spade card.\\
\solution
%\input{exemplar/12/13/3/78/main.tex}
\item
If 4-digit numbers greater than 5,000 are randomly formed from the digits 0, 1, 3, 5, and 7, what is the probability of forming a number divisible by 5 when:
\begin{enumerate}
    \item The digits are repeated?
    \item The repetition of digits is not allowed?
\end{enumerate}
\solution
%\input{ncert/11/16/4/9/main.tex}
\item Consider the probability space $\brak{\Omega, \mathcal{G}, P}$ where $\Omega = [0,2]$ and $\mathcal{G} = \cbrak{\phi, \Omega, [0,1], (1,2]}$. Let $X$ and $Y$ be two functions on $\Omega$ defined as
\begin{align*}
    X(\omega) = 
    \begin{cases}
        1 & \text{if }\omega \in [0, 1]\\
        2 & \text{if }\omega \in (1, 2]
    \end{cases}
\end{align*}
and
\begin{align*}
    Y(\omega) = 
    \begin{cases}
        2 & \text{if }\omega \in [0, 1.5]\\
        3 & \text{if }\omega \in (1.5, 2].
    \end{cases}
\end{align*}
Then which one of the following statements is true?
\begin{enumerate}
    \item [(A)] $X$ is a random variable with respect to $\mathcal{G}$, but $Y$ is not a random variable with respect to $\mathcal{G}$.
    \item [(B)] $Y$ is a random variable with respect to $\mathcal{G}$, but $X$ is not a random variable with respect to $\mathcal{G}$.
    \item [(C)] Neither $X$ nor $Y$ is a random variable with respect to $\mathcal{G}$.
    \item [(D)] Both $X$ and $Y$ are random variables with respect to $\mathcal{G}$.
\end{enumerate} \hfill (GATE ST 2023)\\
\solution
%\input{gate/ST/2023/14/main.tex}
	\item  A die is loaded in such a way that each odd number is twice as likely to occur as
each even number. Find $P(G)$, where $G$ is the event that a number greater than
3 occurs on a single roll of the die.
\\
\solution
		%\input{exemplar/11/16/3/5/main.tex}
	\item All the jacks, queens and kings are removed from a deck of 52 playing cards. The remaining cards are well shuffled and then one card is drawn at random. Giving ace a value 1 similar value for other cards, find the probability that the card has a value 
		\begin{enumerate}
			\item 7
			\item greater than 7
			\item less than 7
		\end{enumerate}
		%\input{exemplar/10/13/3/30/main.tex}
  \item A Lot consists of 48 mobile phones of which 42 are good, 3 have only minor defects and 3 have major defects.Varnika will buy a phone if it is good but the trader will only buy a mobile if it has no major defects. One phone is selected at random from the lot. What is the probability that it is
\begin{enumerate}
	\item acceptable to Varnika?
            \item acceptable to the trader?
\end{enumerate}
\solution
	%\input{exemplar/10/13/3/40/main.tex}
 \item A student says that if you throw a die, it will show up 1 or not 1. Therefore, the probability of getting 1 and the probability of getting 'not 1' each is equal to $\frac{1}{2}$. Is this correct? Give reasons.\\
 \solution
        %\input{exemplar/10/13/2/9/main.tex}
   \item Four candidates A, B, C, D have ap-
plied for the assignment to coach a school cricket
team. If A is twice as likely to be selected as B, and
B and C are given about the same chance of being
selected, while C is twice as likely to be selected
as D, what are the probabilities that
\begin{enumerate}
\item C will be selected?
\item A will not be selected?
\end{enumerate}
	%\input{exemplar/11/16/3/9/main.tex}
 \item A bag contain 24 balls of which $x$ balls are red, $2x$ are white and $3x$ are blue. A ball is selected at random, What is the probability that it is
\begin{enumerate}[label=\alph*)]
\item not red ?
\item white ?
\end{enumerate}
%\input{exemplar/10/13/3/41/main.tex}
If the letters of the word ASSASSINATION are arranged at random. Find the Probability that
\begin{enumerate}[label=(\alph*)]
\item Four $S's$ come consecutively in the word
\item Two  $I's$ and two $N's$ come together
\item All $A's$ are not coming together
\item No two $A's$ are coming together
\end{enumerate}
%\input{exemplar/11/16/3/14/main.tex}
	\item One urn contains two black balls (labelled B1 and B2) and one white ball. A
	second urn contains one black ball and two white balls (labelled W1 and W2).
	Suppose the following experiment is performed. One of the two urns is chosen
	at random. Next a ball is randomly chosen from the urn. Then a second ball is
	chosen at random from the same urn without replacing the first ball.
	
	\begin{enumerate}
	\item What is the probability that two black balls are chosen?
	
	\item What is the probability that two balls of opposite colour are chosen?
	\end{enumerate}
	\solution
	%\input{exemplar/11/16/3/12/main1.tex}
\end{enumerate}

	\item A bag contains 4 red and 4 black balls, another bag contains 2 red and 6 black balls. One of the two bags is selected at random and a ball is drawn from the bag which is found to be red. Find the probability that the ball is drawn from the first bag.
\\
\solution
		%\begin{table}[H]
	\centering
\begin{tabular}{|c|c|c|}
\hline
Random variable &Value &Definition\\ \hline
\multirow{3}{*}{X} &0 &Slips of Rs 1\\
&1 &Slips of Rs 5\\
&2 &Slips of Rs 13\\ \hline
\multirow{2}{*}{Y} &0 &Box A\\
&1 &Box B\\\hline
\end{tabular}
\caption{}
\label{tab:Distribution}
\end{table}
See \tabref{tab:Distribution}.
\begin{align}
p_{Y}\brak{k}= \begin{cases} 
      \frac{1}{3} & {k=0} \\
      \frac{2}{3 }& {k=1} 
   \end{cases}
   \\
p_{Y|X}\brak{0|0} = \frac{19}{25}\, 
p_{Y|X}\brak{0|1} = \frac{6}{25}\,
p_{Y|X}\brak{1|0} = \frac{45}{50}\,
p_{Y|X}\brak{1|2} = \frac{5}{50}
\end{align}
The desired probability is the probability that a slip drawn at random is marked other than Rs 1,
\begin{align}
&=1-p_X\brak{0}\\
&= p_X(1) + p_X(2)
\end{align}
Using Bayes theorem,
\begin{align}
&= p_Y\brak{0} \times \pr{Y=0 | X=1} + p_Y\brak{1} \times \pr{Y=1|X=2}\\
&=\frac{1}{3} \times \frac{6}{25} + \frac{2}{3} \times \frac{5}{50}\\
&=\frac{11}{75}
\end{align}

\newpage

%\tableofcontents

\bigskip

\renewcommand{\thefigure}{\theenumi}
\renewcommand{\thetable}{\theenumi}
%\renewcommand{\theequation}{\theenumi}

%\begin{abstract}
%%\boldmath
%In this letter, an algorithm for evaluating the exact analytical bit error rate  (BER)  for the piecewise linear (PL) combiner for  multiple relays is presented. Previous results were available only for upto three relays. The algorithm is unique in the sense that  the actual mathematical expressions, that are prohibitively large, need not be explicitly obtained. The diversity gain due to multiple relays is shown through plots of the analytical BER, well supported by simulations. 
%
%\end{abstract}
% IEEEtran.cls defaults to using nonbold math in the Abstract.
% This preserves the distinction between vectors and scalars. However,
% if the journal you are submitting to favors bold math in the abstract,
% then you can use LaTeX's standard command \boldmath at the very start
% of the abstract to achieve this. Many IEEE journals frown on math
% in the abstract anyway.

% Note that keywords are not normally used for peerreview papers.
%\begin{IEEEkeywords}
%Cooperative diversity, decode and forward, piecewise linear
%\end{IEEEkeywords}



% For peer review papers, you can put extra information on the cover
% page as needed:
% \ifCLASSOPTIONpeerreview
% \begin{center} \bfseries EDICS Category: 3-BBND \end{center}
% \fi
%
% For peerreview papers, this IEEEtran command inserts a page break and
% creates the second title. It will be ignored for other modes.
%\IEEEpeerreviewmaketitle




  \item
  Cards with numbers 2 to 101 are placed in a box. A card is selected at random.Find the probability that the card has
\begin{enumerate}[label=(\roman*)]
	\item an even number 
	\item a square number
\end{enumerate}
\solution
%\begin{table}[H]
	\centering
\begin{tabular}{|c|c|c|}
\hline
Random variable &Value &Definition\\ \hline
\multirow{3}{*}{X} &0 &Slips of Rs 1\\
&1 &Slips of Rs 5\\
&2 &Slips of Rs 13\\ \hline
\multirow{2}{*}{Y} &0 &Box A\\
&1 &Box B\\\hline
\end{tabular}
\caption{}
\label{tab:Distribution}
\end{table}
See \tabref{tab:Distribution}.
\begin{align}
p_{Y}\brak{k}= \begin{cases} 
      \frac{1}{3} & {k=0} \\
      \frac{2}{3 }& {k=1} 
   \end{cases}
   \\
p_{Y|X}\brak{0|0} = \frac{19}{25}\, 
p_{Y|X}\brak{0|1} = \frac{6}{25}\,
p_{Y|X}\brak{1|0} = \frac{45}{50}\,
p_{Y|X}\brak{1|2} = \frac{5}{50}
\end{align}
The desired probability is the probability that a slip drawn at random is marked other than Rs 1,
\begin{align}
&=1-p_X\brak{0}\\
&= p_X(1) + p_X(2)
\end{align}
Using Bayes theorem,
\begin{align}
&= p_Y\brak{0} \times \pr{Y=0 | X=1} + p_Y\brak{1} \times \pr{Y=1|X=2}\\
&=\frac{1}{3} \times \frac{6}{25} + \frac{2}{3} \times \frac{5}{50}\\
&=\frac{11}{75}
\end{align}

\newpage

%\tableofcontents

\bigskip

\renewcommand{\thefigure}{\theenumi}
\renewcommand{\thetable}{\theenumi}
%\renewcommand{\theequation}{\theenumi}

%\begin{abstract}
%%\boldmath
%In this letter, an algorithm for evaluating the exact analytical bit error rate  (BER)  for the piecewise linear (PL) combiner for  multiple relays is presented. Previous results were available only for upto three relays. The algorithm is unique in the sense that  the actual mathematical expressions, that are prohibitively large, need not be explicitly obtained. The diversity gain due to multiple relays is shown through plots of the analytical BER, well supported by simulations. 
%
%\end{abstract}
% IEEEtran.cls defaults to using nonbold math in the Abstract.
% This preserves the distinction between vectors and scalars. However,
% if the journal you are submitting to favors bold math in the abstract,
% then you can use LaTeX's standard command \boldmath at the very start
% of the abstract to achieve this. Many IEEE journals frown on math
% in the abstract anyway.

% Note that keywords are not normally used for peerreview papers.
%\begin{IEEEkeywords}
%Cooperative diversity, decode and forward, piecewise linear
%\end{IEEEkeywords}



% For peer review papers, you can put extra information on the cover
% page as needed:
% \ifCLASSOPTIONpeerreview
% \begin{center} \bfseries EDICS Category: 3-BBND \end{center}
% \fi
%
% For peerreview papers, this IEEEtran command inserts a page break and
% creates the second title. It will be ignored for other modes.
%\IEEEpeerreviewmaketitle




\item
The king, queen and jack of clubs are removed from a deck of 52 playing cards and then well shuffled. Now one card is drawn at random from the remaining cards.  Determine the probability that the card is
\begin{enumerate}[label=(\roman*)]
\item a club
\item 10 of hearts
\end{enumerate}
\solution
%\begin{table}[H]
	\centering
\begin{tabular}{|c|c|c|}
\hline
Random variable &Value &Definition\\ \hline
\multirow{3}{*}{X} &0 &Slips of Rs 1\\
&1 &Slips of Rs 5\\
&2 &Slips of Rs 13\\ \hline
\multirow{2}{*}{Y} &0 &Box A\\
&1 &Box B\\\hline
\end{tabular}
\caption{}
\label{tab:Distribution}
\end{table}
See \tabref{tab:Distribution}.
\begin{align}
p_{Y}\brak{k}= \begin{cases} 
      \frac{1}{3} & {k=0} \\
      \frac{2}{3 }& {k=1} 
   \end{cases}
   \\
p_{Y|X}\brak{0|0} = \frac{19}{25}\, 
p_{Y|X}\brak{0|1} = \frac{6}{25}\,
p_{Y|X}\brak{1|0} = \frac{45}{50}\,
p_{Y|X}\brak{1|2} = \frac{5}{50}
\end{align}
The desired probability is the probability that a slip drawn at random is marked other than Rs 1,
\begin{align}
&=1-p_X\brak{0}\\
&= p_X(1) + p_X(2)
\end{align}
Using Bayes theorem,
\begin{align}
&= p_Y\brak{0} \times \pr{Y=0 | X=1} + p_Y\brak{1} \times \pr{Y=1|X=2}\\
&=\frac{1}{3} \times \frac{6}{25} + \frac{2}{3} \times \frac{5}{50}\\
&=\frac{11}{75}
\end{align}

\newpage

%\tableofcontents

\bigskip

\renewcommand{\thefigure}{\theenumi}
\renewcommand{\thetable}{\theenumi}
%\renewcommand{\theequation}{\theenumi}

%\begin{abstract}
%%\boldmath
%In this letter, an algorithm for evaluating the exact analytical bit error rate  (BER)  for the piecewise linear (PL) combiner for  multiple relays is presented. Previous results were available only for upto three relays. The algorithm is unique in the sense that  the actual mathematical expressions, that are prohibitively large, need not be explicitly obtained. The diversity gain due to multiple relays is shown through plots of the analytical BER, well supported by simulations. 
%
%\end{abstract}
% IEEEtran.cls defaults to using nonbold math in the Abstract.
% This preserves the distinction between vectors and scalars. However,
% if the journal you are submitting to favors bold math in the abstract,
% then you can use LaTeX's standard command \boldmath at the very start
% of the abstract to achieve this. Many IEEE journals frown on math
% in the abstract anyway.

% Note that keywords are not normally used for peerreview papers.
%\begin{IEEEkeywords}
%Cooperative diversity, decode and forward, piecewise linear
%\end{IEEEkeywords}



% For peer review papers, you can put extra information on the cover
% page as needed:
% \ifCLASSOPTIONpeerreview
% \begin{center} \bfseries EDICS Category: 3-BBND \end{center}
% \fi
%
% For peerreview papers, this IEEEtran command inserts a page break and
% creates the second title. It will be ignored for other modes.
%\IEEEpeerreviewmaketitle




\item A team of medical students doing their internship have to assist during surgeries
at a city hospital. The probabilities of surgeries rated as very complex, complex,
routine, simple or very simple are respectively, 0.15, 0.20, 0.31, 0.26, .08. Find
the probabilities that a particular surgery will be rated
\begin{enumerate}
	\item complex or very complex;
	\item neither very complex nor very simple;
	\item routine or complex
	\item routine or simple
\end{enumerate}
\solution
%\begin{table}[H]
	\centering
\begin{tabular}{|c|c|c|}
\hline
Random variable &Value &Definition\\ \hline
\multirow{3}{*}{X} &0 &Slips of Rs 1\\
&1 &Slips of Rs 5\\
&2 &Slips of Rs 13\\ \hline
\multirow{2}{*}{Y} &0 &Box A\\
&1 &Box B\\\hline
\end{tabular}
\caption{}
\label{tab:Distribution}
\end{table}
See \tabref{tab:Distribution}.
\begin{align}
p_{Y}\brak{k}= \begin{cases} 
      \frac{1}{3} & {k=0} \\
      \frac{2}{3 }& {k=1} 
   \end{cases}
   \\
p_{Y|X}\brak{0|0} = \frac{19}{25}\, 
p_{Y|X}\brak{0|1} = \frac{6}{25}\,
p_{Y|X}\brak{1|0} = \frac{45}{50}\,
p_{Y|X}\brak{1|2} = \frac{5}{50}
\end{align}
The desired probability is the probability that a slip drawn at random is marked other than Rs 1,
\begin{align}
&=1-p_X\brak{0}\\
&= p_X(1) + p_X(2)
\end{align}
Using Bayes theorem,
\begin{align}
&= p_Y\brak{0} \times \pr{Y=0 | X=1} + p_Y\brak{1} \times \pr{Y=1|X=2}\\
&=\frac{1}{3} \times \frac{6}{25} + \frac{2}{3} \times \frac{5}{50}\\
&=\frac{11}{75}
\end{align}

\newpage

%\tableofcontents

\bigskip

\renewcommand{\thefigure}{\theenumi}
\renewcommand{\thetable}{\theenumi}
%\renewcommand{\theequation}{\theenumi}

%\begin{abstract}
%%\boldmath
%In this letter, an algorithm for evaluating the exact analytical bit error rate  (BER)  for the piecewise linear (PL) combiner for  multiple relays is presented. Previous results were available only for upto three relays. The algorithm is unique in the sense that  the actual mathematical expressions, that are prohibitively large, need not be explicitly obtained. The diversity gain due to multiple relays is shown through plots of the analytical BER, well supported by simulations. 
%
%\end{abstract}
% IEEEtran.cls defaults to using nonbold math in the Abstract.
% This preserves the distinction between vectors and scalars. However,
% if the journal you are submitting to favors bold math in the abstract,
% then you can use LaTeX's standard command \boldmath at the very start
% of the abstract to achieve this. Many IEEE journals frown on math
% in the abstract anyway.

% Note that keywords are not normally used for peerreview papers.
%\begin{IEEEkeywords}
%Cooperative diversity, decode and forward, piecewise linear
%\end{IEEEkeywords}



% For peer review papers, you can put extra information on the cover
% page as needed:
% \ifCLASSOPTIONpeerreview
% \begin{center} \bfseries EDICS Category: 3-BBND \end{center}
% \fi
%
% For peerreview papers, this IEEEtran command inserts a page break and
% creates the second title. It will be ignored for other modes.
%\IEEEpeerreviewmaketitle




\item A card is selected from a pack of 52 cards.
\begin{enumerate}[label=(\alph*)]
    \item How many points are there in the sample space?
    \item Calculate the probability that the card is an ace of spades.
    \item Calculate the probability that the card is (i) an ace and (ii) black card.
\end{enumerate}
\solution
%Let $X$ be an bernoulli rv defined as in \tabref{tab:exemplar/11/16/3/26}.  Then, 
\begin{equation}
    p =
        \frac{4}{11} 
\end{equation}
\begin{table}[H]
	\centering
	\input{exemplar/11/16/3/26/tables/Table2.tex}
	\caption{}
        \label{tab:exemplar/11/16/3/26}
\end{table}

\item The probability that a non leap year selected at random will contain 53 sundays.
\\
\solution
%\begin{table}[H]
	\centering
\begin{tabular}{|c|c|c|}
\hline
Random variable &Value &Definition\\ \hline
\multirow{3}{*}{X} &0 &Slips of Rs 1\\
&1 &Slips of Rs 5\\
&2 &Slips of Rs 13\\ \hline
\multirow{2}{*}{Y} &0 &Box A\\
&1 &Box B\\\hline
\end{tabular}
\caption{}
\label{tab:Distribution}
\end{table}
See \tabref{tab:Distribution}.
\begin{align}
p_{Y}\brak{k}= \begin{cases} 
      \frac{1}{3} & {k=0} \\
      \frac{2}{3 }& {k=1} 
   \end{cases}
   \\
p_{Y|X}\brak{0|0} = \frac{19}{25}\, 
p_{Y|X}\brak{0|1} = \frac{6}{25}\,
p_{Y|X}\brak{1|0} = \frac{45}{50}\,
p_{Y|X}\brak{1|2} = \frac{5}{50}
\end{align}
The desired probability is the probability that a slip drawn at random is marked other than Rs 1,
\begin{align}
&=1-p_X\brak{0}\\
&= p_X(1) + p_X(2)
\end{align}
Using Bayes theorem,
\begin{align}
&= p_Y\brak{0} \times \pr{Y=0 | X=1} + p_Y\brak{1} \times \pr{Y=1|X=2}\\
&=\frac{1}{3} \times \frac{6}{25} + \frac{2}{3} \times \frac{5}{50}\\
&=\frac{11}{75}
\end{align}

\newpage

%\tableofcontents

\bigskip

\renewcommand{\thefigure}{\theenumi}
\renewcommand{\thetable}{\theenumi}
%\renewcommand{\theequation}{\theenumi}

%\begin{abstract}
%%\boldmath
%In this letter, an algorithm for evaluating the exact analytical bit error rate  (BER)  for the piecewise linear (PL) combiner for  multiple relays is presented. Previous results were available only for upto three relays. The algorithm is unique in the sense that  the actual mathematical expressions, that are prohibitively large, need not be explicitly obtained. The diversity gain due to multiple relays is shown through plots of the analytical BER, well supported by simulations. 
%
%\end{abstract}
% IEEEtran.cls defaults to using nonbold math in the Abstract.
% This preserves the distinction between vectors and scalars. However,
% if the journal you are submitting to favors bold math in the abstract,
% then you can use LaTeX's standard command \boldmath at the very start
% of the abstract to achieve this. Many IEEE journals frown on math
% in the abstract anyway.

% Note that keywords are not normally used for peerreview papers.
%\begin{IEEEkeywords}
%Cooperative diversity, decode and forward, piecewise linear
%\end{IEEEkeywords}



% For peer review papers, you can put extra information on the cover
% page as needed:
% \ifCLASSOPTIONpeerreview
% \begin{center} \bfseries EDICS Category: 3-BBND \end{center}
% \fi
%
% For peerreview papers, this IEEEtran command inserts a page break and
% creates the second title. It will be ignored for other modes.
%\IEEEpeerreviewmaketitle




\item One of the four persons John, Rita, Aslam or Gurpreet will be promoted next
month. Consequently the sample space consists of four elementary outcomes
S = {John promoted, Rita promoted, Aslam promoted, Gurpreet promoted}
You are told that the chances of John’s promotion is same as that of Gurpreet,
Rita’s chances of promotion are twice as likely as Johns. Aslam’s chances are
four times that of John.
\begin{enumerate}
	\item Determine
	\begin{enumerate}
		\item P (John promoted)
		\item P (Rita promoted)
		\item P (Aslam promoted)
		\item P (Gurpreet promoted)
	\end{enumerate}
	\item If A = {John promoted or Gurpreet promoted}, find P (A).
\end{enumerate}
\solution
%\begin{table}[H]
	\centering
\begin{tabular}{|c|c|c|}
\hline
Random variable &Value &Definition\\ \hline
\multirow{3}{*}{X} &0 &Slips of Rs 1\\
&1 &Slips of Rs 5\\
&2 &Slips of Rs 13\\ \hline
\multirow{2}{*}{Y} &0 &Box A\\
&1 &Box B\\\hline
\end{tabular}
\caption{}
\label{tab:Distribution}
\end{table}
See \tabref{tab:Distribution}.
\begin{align}
p_{Y}\brak{k}= \begin{cases} 
      \frac{1}{3} & {k=0} \\
      \frac{2}{3 }& {k=1} 
   \end{cases}
   \\
p_{Y|X}\brak{0|0} = \frac{19}{25}\, 
p_{Y|X}\brak{0|1} = \frac{6}{25}\,
p_{Y|X}\brak{1|0} = \frac{45}{50}\,
p_{Y|X}\brak{1|2} = \frac{5}{50}
\end{align}
The desired probability is the probability that a slip drawn at random is marked other than Rs 1,
\begin{align}
&=1-p_X\brak{0}\\
&= p_X(1) + p_X(2)
\end{align}
Using Bayes theorem,
\begin{align}
&= p_Y\brak{0} \times \pr{Y=0 | X=1} + p_Y\brak{1} \times \pr{Y=1|X=2}\\
&=\frac{1}{3} \times \frac{6}{25} + \frac{2}{3} \times \frac{5}{50}\\
&=\frac{11}{75}
\end{align}

\newpage

%\tableofcontents

\bigskip

\renewcommand{\thefigure}{\theenumi}
\renewcommand{\thetable}{\theenumi}
%\renewcommand{\theequation}{\theenumi}

%\begin{abstract}
%%\boldmath
%In this letter, an algorithm for evaluating the exact analytical bit error rate  (BER)  for the piecewise linear (PL) combiner for  multiple relays is presented. Previous results were available only for upto three relays. The algorithm is unique in the sense that  the actual mathematical expressions, that are prohibitively large, need not be explicitly obtained. The diversity gain due to multiple relays is shown through plots of the analytical BER, well supported by simulations. 
%
%\end{abstract}
% IEEEtran.cls defaults to using nonbold math in the Abstract.
% This preserves the distinction between vectors and scalars. However,
% if the journal you are submitting to favors bold math in the abstract,
% then you can use LaTeX's standard command \boldmath at the very start
% of the abstract to achieve this. Many IEEE journals frown on math
% in the abstract anyway.

% Note that keywords are not normally used for peerreview papers.
%\begin{IEEEkeywords}
%Cooperative diversity, decode and forward, piecewise linear
%\end{IEEEkeywords}



% For peer review papers, you can put extra information on the cover
% page as needed:
% \ifCLASSOPTIONpeerreview
% \begin{center} \bfseries EDICS Category: 3-BBND \end{center}
% \fi
%
% For peerreview papers, this IEEEtran command inserts a page break and
% creates the second title. It will be ignored for other modes.
%\IEEEpeerreviewmaketitle




\item A card is drawn from a deck of 52 cards. Find the probability of getting a king or a heart or a red card.\\
\solution
%\begin{table}[H]
	\centering
\begin{tabular}{|c|c|c|}
\hline
Random variable &Value &Definition\\ \hline
\multirow{3}{*}{X} &0 &Slips of Rs 1\\
&1 &Slips of Rs 5\\
&2 &Slips of Rs 13\\ \hline
\multirow{2}{*}{Y} &0 &Box A\\
&1 &Box B\\\hline
\end{tabular}
\caption{}
\label{tab:Distribution}
\end{table}
See \tabref{tab:Distribution}.
\begin{align}
p_{Y}\brak{k}= \begin{cases} 
      \frac{1}{3} & {k=0} \\
      \frac{2}{3 }& {k=1} 
   \end{cases}
   \\
p_{Y|X}\brak{0|0} = \frac{19}{25}\, 
p_{Y|X}\brak{0|1} = \frac{6}{25}\,
p_{Y|X}\brak{1|0} = \frac{45}{50}\,
p_{Y|X}\brak{1|2} = \frac{5}{50}
\end{align}
The desired probability is the probability that a slip drawn at random is marked other than Rs 1,
\begin{align}
&=1-p_X\brak{0}\\
&= p_X(1) + p_X(2)
\end{align}
Using Bayes theorem,
\begin{align}
&= p_Y\brak{0} \times \pr{Y=0 | X=1} + p_Y\brak{1} \times \pr{Y=1|X=2}\\
&=\frac{1}{3} \times \frac{6}{25} + \frac{2}{3} \times \frac{5}{50}\\
&=\frac{11}{75}
\end{align}

\newpage

%\tableofcontents

\bigskip

\renewcommand{\thefigure}{\theenumi}
\renewcommand{\thetable}{\theenumi}
%\renewcommand{\theequation}{\theenumi}

%\begin{abstract}
%%\boldmath
%In this letter, an algorithm for evaluating the exact analytical bit error rate  (BER)  for the piecewise linear (PL) combiner for  multiple relays is presented. Previous results were available only for upto three relays. The algorithm is unique in the sense that  the actual mathematical expressions, that are prohibitively large, need not be explicitly obtained. The diversity gain due to multiple relays is shown through plots of the analytical BER, well supported by simulations. 
%
%\end{abstract}
% IEEEtran.cls defaults to using nonbold math in the Abstract.
% This preserves the distinction between vectors and scalars. However,
% if the journal you are submitting to favors bold math in the abstract,
% then you can use LaTeX's standard command \boldmath at the very start
% of the abstract to achieve this. Many IEEE journals frown on math
% in the abstract anyway.

% Note that keywords are not normally used for peerreview papers.
%\begin{IEEEkeywords}
%Cooperative diversity, decode and forward, piecewise linear
%\end{IEEEkeywords}



% For peer review papers, you can put extra information on the cover
% page as needed:
% \ifCLASSOPTIONpeerreview
% \begin{center} \bfseries EDICS Category: 3-BBND \end{center}
% \fi
%
% For peerreview papers, this IEEEtran command inserts a page break and
% creates the second title. It will be ignored for other modes.
%\IEEEpeerreviewmaketitle




\item The probability that a student will pass his examination is 0.73, the probability of
the student getting a compartment is 0.13, and the probability that the student will
either pass or get compartment is 0.96. State True or False.\\
\solution
%\begin{table}[H]
	\centering
\begin{tabular}{|c|c|c|}
\hline
Random variable &Value &Definition\\ \hline
\multirow{3}{*}{X} &0 &Slips of Rs 1\\
&1 &Slips of Rs 5\\
&2 &Slips of Rs 13\\ \hline
\multirow{2}{*}{Y} &0 &Box A\\
&1 &Box B\\\hline
\end{tabular}
\caption{}
\label{tab:Distribution}
\end{table}
See \tabref{tab:Distribution}.
\begin{align}
p_{Y}\brak{k}= \begin{cases} 
      \frac{1}{3} & {k=0} \\
      \frac{2}{3 }& {k=1} 
   \end{cases}
   \\
p_{Y|X}\brak{0|0} = \frac{19}{25}\, 
p_{Y|X}\brak{0|1} = \frac{6}{25}\,
p_{Y|X}\brak{1|0} = \frac{45}{50}\,
p_{Y|X}\brak{1|2} = \frac{5}{50}
\end{align}
The desired probability is the probability that a slip drawn at random is marked other than Rs 1,
\begin{align}
&=1-p_X\brak{0}\\
&= p_X(1) + p_X(2)
\end{align}
Using Bayes theorem,
\begin{align}
&= p_Y\brak{0} \times \pr{Y=0 | X=1} + p_Y\brak{1} \times \pr{Y=1|X=2}\\
&=\frac{1}{3} \times \frac{6}{25} + \frac{2}{3} \times \frac{5}{50}\\
&=\frac{11}{75}
\end{align}

\newpage

%\tableofcontents

\bigskip

\renewcommand{\thefigure}{\theenumi}
\renewcommand{\thetable}{\theenumi}
%\renewcommand{\theequation}{\theenumi}

%\begin{abstract}
%%\boldmath
%In this letter, an algorithm for evaluating the exact analytical bit error rate  (BER)  for the piecewise linear (PL) combiner for  multiple relays is presented. Previous results were available only for upto three relays. The algorithm is unique in the sense that  the actual mathematical expressions, that are prohibitively large, need not be explicitly obtained. The diversity gain due to multiple relays is shown through plots of the analytical BER, well supported by simulations. 
%
%\end{abstract}
% IEEEtran.cls defaults to using nonbold math in the Abstract.
% This preserves the distinction between vectors and scalars. However,
% if the journal you are submitting to favors bold math in the abstract,
% then you can use LaTeX's standard command \boldmath at the very start
% of the abstract to achieve this. Many IEEE journals frown on math
% in the abstract anyway.

% Note that keywords are not normally used for peerreview papers.
%\begin{IEEEkeywords}
%Cooperative diversity, decode and forward, piecewise linear
%\end{IEEEkeywords}



% For peer review papers, you can put extra information on the cover
% page as needed:
% \ifCLASSOPTIONpeerreview
% \begin{center} \bfseries EDICS Category: 3-BBND \end{center}
% \fi
%
% For peerreview papers, this IEEEtran command inserts a page break and
% creates the second title. It will be ignored for other modes.
%\IEEEpeerreviewmaketitle




\item A card is selected from a pack of 52 cards\\
\begin{enumerate}[label=(\alph*)]
\item How many points are there in the sample space?
\item Calculate the probability that the cards is an ace of spades.
\item Calculate the probability that the card is (i) an ace (ii)black card.\\
\end{enumerate}
%\input{ncert/11/16/3/4_1/Prob_4.tex}
\item In a non-leap year, the probability of having 53 tuesdays or 53 wednesdays is\\
\solution
%A non-leap year has a total of 365 days, and a week has 7 days.\\
So it can be expressed as 
\begin{align}
365\text{days} &=52\times 7+1 \text{day}
\end{align}
$\implies$ 52 tuesdays or wednesdays\\
Random variable X denotes the days of a week
\begin{align}
p_X\brak{k}&=\frac{1}{7}; \quad \brak{1<k<7}
\end{align}
So the probability of extra day being tuesday or wednesday is
\begin{align}
p_X\brak{3}+p_X\brak{4}&=\frac{1}{7}+\frac{1}{7}=\frac{2}{7}
\end{align}



\item There are 1000 sealed envelopes in a box, 10 of them contain a cash prize of
Rs 100 each, 100 of them contain a cash prize of Rs 50 each and 200 of them
contain a cash prize of Rs 10 each and rest do not contain any cash prize. If they
are well shuffled and an envelope is picked up out, what is the probability that it
contains no cash prize?\\
\solution
%\begin{table}[H]
	\centering
\begin{tabular}{|c|c|c|}
\hline
Random variable &Value &Definition\\ \hline
\multirow{3}{*}{X} &0 &Slips of Rs 1\\
&1 &Slips of Rs 5\\
&2 &Slips of Rs 13\\ \hline
\multirow{2}{*}{Y} &0 &Box A\\
&1 &Box B\\\hline
\end{tabular}
\caption{}
\label{tab:Distribution}
\end{table}
See \tabref{tab:Distribution}.
\begin{align}
p_{Y}\brak{k}= \begin{cases} 
      \frac{1}{3} & {k=0} \\
      \frac{2}{3 }& {k=1} 
   \end{cases}
   \\
p_{Y|X}\brak{0|0} = \frac{19}{25}\, 
p_{Y|X}\brak{0|1} = \frac{6}{25}\,
p_{Y|X}\brak{1|0} = \frac{45}{50}\,
p_{Y|X}\brak{1|2} = \frac{5}{50}
\end{align}
The desired probability is the probability that a slip drawn at random is marked other than Rs 1,
\begin{align}
&=1-p_X\brak{0}\\
&= p_X(1) + p_X(2)
\end{align}
Using Bayes theorem,
\begin{align}
&= p_Y\brak{0} \times \pr{Y=0 | X=1} + p_Y\brak{1} \times \pr{Y=1|X=2}\\
&=\frac{1}{3} \times \frac{6}{25} + \frac{2}{3} \times \frac{5}{50}\\
&=\frac{11}{75}
\end{align}

\newpage

%\tableofcontents

\bigskip

\renewcommand{\thefigure}{\theenumi}
\renewcommand{\thetable}{\theenumi}
%\renewcommand{\theequation}{\theenumi}

%\begin{abstract}
%%\boldmath
%In this letter, an algorithm for evaluating the exact analytical bit error rate  (BER)  for the piecewise linear (PL) combiner for  multiple relays is presented. Previous results were available only for upto three relays. The algorithm is unique in the sense that  the actual mathematical expressions, that are prohibitively large, need not be explicitly obtained. The diversity gain due to multiple relays is shown through plots of the analytical BER, well supported by simulations. 
%
%\end{abstract}
% IEEEtran.cls defaults to using nonbold math in the Abstract.
% This preserves the distinction between vectors and scalars. However,
% if the journal you are submitting to favors bold math in the abstract,
% then you can use LaTeX's standard command \boldmath at the very start
% of the abstract to achieve this. Many IEEE journals frown on math
% in the abstract anyway.

% Note that keywords are not normally used for peerreview papers.
%\begin{IEEEkeywords}
%Cooperative diversity, decode and forward, piecewise linear
%\end{IEEEkeywords}



% For peer review papers, you can put extra information on the cover
% page as needed:
% \ifCLASSOPTIONpeerreview
% \begin{center} \bfseries EDICS Category: 3-BBND \end{center}
% \fi
%
% For peerreview papers, this IEEEtran command inserts a page break and
% creates the second title. It will be ignored for other modes.
%\IEEEpeerreviewmaketitle




\item 
A die is thrown and a card is selected at random from a deck of 52 playing cards. The probability of getting an even number on the die and a spade card.\\
\solution
%\begin{table}[H]
	\centering
\begin{tabular}{|c|c|c|}
\hline
Random variable &Value &Definition\\ \hline
\multirow{3}{*}{X} &0 &Slips of Rs 1\\
&1 &Slips of Rs 5\\
&2 &Slips of Rs 13\\ \hline
\multirow{2}{*}{Y} &0 &Box A\\
&1 &Box B\\\hline
\end{tabular}
\caption{}
\label{tab:Distribution}
\end{table}
See \tabref{tab:Distribution}.
\begin{align}
p_{Y}\brak{k}= \begin{cases} 
      \frac{1}{3} & {k=0} \\
      \frac{2}{3 }& {k=1} 
   \end{cases}
   \\
p_{Y|X}\brak{0|0} = \frac{19}{25}\, 
p_{Y|X}\brak{0|1} = \frac{6}{25}\,
p_{Y|X}\brak{1|0} = \frac{45}{50}\,
p_{Y|X}\brak{1|2} = \frac{5}{50}
\end{align}
The desired probability is the probability that a slip drawn at random is marked other than Rs 1,
\begin{align}
&=1-p_X\brak{0}\\
&= p_X(1) + p_X(2)
\end{align}
Using Bayes theorem,
\begin{align}
&= p_Y\brak{0} \times \pr{Y=0 | X=1} + p_Y\brak{1} \times \pr{Y=1|X=2}\\
&=\frac{1}{3} \times \frac{6}{25} + \frac{2}{3} \times \frac{5}{50}\\
&=\frac{11}{75}
\end{align}

\newpage

%\tableofcontents

\bigskip

\renewcommand{\thefigure}{\theenumi}
\renewcommand{\thetable}{\theenumi}
%\renewcommand{\theequation}{\theenumi}

%\begin{abstract}
%%\boldmath
%In this letter, an algorithm for evaluating the exact analytical bit error rate  (BER)  for the piecewise linear (PL) combiner for  multiple relays is presented. Previous results were available only for upto three relays. The algorithm is unique in the sense that  the actual mathematical expressions, that are prohibitively large, need not be explicitly obtained. The diversity gain due to multiple relays is shown through plots of the analytical BER, well supported by simulations. 
%
%\end{abstract}
% IEEEtran.cls defaults to using nonbold math in the Abstract.
% This preserves the distinction between vectors and scalars. However,
% if the journal you are submitting to favors bold math in the abstract,
% then you can use LaTeX's standard command \boldmath at the very start
% of the abstract to achieve this. Many IEEE journals frown on math
% in the abstract anyway.

% Note that keywords are not normally used for peerreview papers.
%\begin{IEEEkeywords}
%Cooperative diversity, decode and forward, piecewise linear
%\end{IEEEkeywords}



% For peer review papers, you can put extra information on the cover
% page as needed:
% \ifCLASSOPTIONpeerreview
% \begin{center} \bfseries EDICS Category: 3-BBND \end{center}
% \fi
%
% For peerreview papers, this IEEEtran command inserts a page break and
% creates the second title. It will be ignored for other modes.
%\IEEEpeerreviewmaketitle




\item
If 4-digit numbers greater than 5,000 are randomly formed from the digits 0, 1, 3, 5, and 7, what is the probability of forming a number divisible by 5 when:
\begin{enumerate}
    \item The digits are repeated?
    \item The repetition of digits is not allowed?
\end{enumerate}
\solution
%\begin{table}[H]
	\centering
\begin{tabular}{|c|c|c|}
\hline
Random variable &Value &Definition\\ \hline
\multirow{3}{*}{X} &0 &Slips of Rs 1\\
&1 &Slips of Rs 5\\
&2 &Slips of Rs 13\\ \hline
\multirow{2}{*}{Y} &0 &Box A\\
&1 &Box B\\\hline
\end{tabular}
\caption{}
\label{tab:Distribution}
\end{table}
See \tabref{tab:Distribution}.
\begin{align}
p_{Y}\brak{k}= \begin{cases} 
      \frac{1}{3} & {k=0} \\
      \frac{2}{3 }& {k=1} 
   \end{cases}
   \\
p_{Y|X}\brak{0|0} = \frac{19}{25}\, 
p_{Y|X}\brak{0|1} = \frac{6}{25}\,
p_{Y|X}\brak{1|0} = \frac{45}{50}\,
p_{Y|X}\brak{1|2} = \frac{5}{50}
\end{align}
The desired probability is the probability that a slip drawn at random is marked other than Rs 1,
\begin{align}
&=1-p_X\brak{0}\\
&= p_X(1) + p_X(2)
\end{align}
Using Bayes theorem,
\begin{align}
&= p_Y\brak{0} \times \pr{Y=0 | X=1} + p_Y\brak{1} \times \pr{Y=1|X=2}\\
&=\frac{1}{3} \times \frac{6}{25} + \frac{2}{3} \times \frac{5}{50}\\
&=\frac{11}{75}
\end{align}

\newpage

%\tableofcontents

\bigskip

\renewcommand{\thefigure}{\theenumi}
\renewcommand{\thetable}{\theenumi}
%\renewcommand{\theequation}{\theenumi}

%\begin{abstract}
%%\boldmath
%In this letter, an algorithm for evaluating the exact analytical bit error rate  (BER)  for the piecewise linear (PL) combiner for  multiple relays is presented. Previous results were available only for upto three relays. The algorithm is unique in the sense that  the actual mathematical expressions, that are prohibitively large, need not be explicitly obtained. The diversity gain due to multiple relays is shown through plots of the analytical BER, well supported by simulations. 
%
%\end{abstract}
% IEEEtran.cls defaults to using nonbold math in the Abstract.
% This preserves the distinction between vectors and scalars. However,
% if the journal you are submitting to favors bold math in the abstract,
% then you can use LaTeX's standard command \boldmath at the very start
% of the abstract to achieve this. Many IEEE journals frown on math
% in the abstract anyway.

% Note that keywords are not normally used for peerreview papers.
%\begin{IEEEkeywords}
%Cooperative diversity, decode and forward, piecewise linear
%\end{IEEEkeywords}



% For peer review papers, you can put extra information on the cover
% page as needed:
% \ifCLASSOPTIONpeerreview
% \begin{center} \bfseries EDICS Category: 3-BBND \end{center}
% \fi
%
% For peerreview papers, this IEEEtran command inserts a page break and
% creates the second title. It will be ignored for other modes.
%\IEEEpeerreviewmaketitle




\item Consider the probability space $\brak{\Omega, \mathcal{G}, P}$ where $\Omega = [0,2]$ and $\mathcal{G} = \cbrak{\phi, \Omega, [0,1], (1,2]}$. Let $X$ and $Y$ be two functions on $\Omega$ defined as
\begin{align*}
    X(\omega) = 
    \begin{cases}
        1 & \text{if }\omega \in [0, 1]\\
        2 & \text{if }\omega \in (1, 2]
    \end{cases}
\end{align*}
and
\begin{align*}
    Y(\omega) = 
    \begin{cases}
        2 & \text{if }\omega \in [0, 1.5]\\
        3 & \text{if }\omega \in (1.5, 2].
    \end{cases}
\end{align*}
Then which one of the following statements is true?
\begin{enumerate}
    \item [(A)] $X$ is a random variable with respect to $\mathcal{G}$, but $Y$ is not a random variable with respect to $\mathcal{G}$.
    \item [(B)] $Y$ is a random variable with respect to $\mathcal{G}$, but $X$ is not a random variable with respect to $\mathcal{G}$.
    \item [(C)] Neither $X$ nor $Y$ is a random variable with respect to $\mathcal{G}$.
    \item [(D)] Both $X$ and $Y$ are random variables with respect to $\mathcal{G}$.
\end{enumerate} \hfill (GATE ST 2023)\\
\solution
%\begin{table}[H]
	\centering
\begin{tabular}{|c|c|c|}
\hline
Random variable &Value &Definition\\ \hline
\multirow{3}{*}{X} &0 &Slips of Rs 1\\
&1 &Slips of Rs 5\\
&2 &Slips of Rs 13\\ \hline
\multirow{2}{*}{Y} &0 &Box A\\
&1 &Box B\\\hline
\end{tabular}
\caption{}
\label{tab:Distribution}
\end{table}
See \tabref{tab:Distribution}.
\begin{align}
p_{Y}\brak{k}= \begin{cases} 
      \frac{1}{3} & {k=0} \\
      \frac{2}{3 }& {k=1} 
   \end{cases}
   \\
p_{Y|X}\brak{0|0} = \frac{19}{25}\, 
p_{Y|X}\brak{0|1} = \frac{6}{25}\,
p_{Y|X}\brak{1|0} = \frac{45}{50}\,
p_{Y|X}\brak{1|2} = \frac{5}{50}
\end{align}
The desired probability is the probability that a slip drawn at random is marked other than Rs 1,
\begin{align}
&=1-p_X\brak{0}\\
&= p_X(1) + p_X(2)
\end{align}
Using Bayes theorem,
\begin{align}
&= p_Y\brak{0} \times \pr{Y=0 | X=1} + p_Y\brak{1} \times \pr{Y=1|X=2}\\
&=\frac{1}{3} \times \frac{6}{25} + \frac{2}{3} \times \frac{5}{50}\\
&=\frac{11}{75}
\end{align}

\newpage

%\tableofcontents

\bigskip

\renewcommand{\thefigure}{\theenumi}
\renewcommand{\thetable}{\theenumi}
%\renewcommand{\theequation}{\theenumi}

%\begin{abstract}
%%\boldmath
%In this letter, an algorithm for evaluating the exact analytical bit error rate  (BER)  for the piecewise linear (PL) combiner for  multiple relays is presented. Previous results were available only for upto three relays. The algorithm is unique in the sense that  the actual mathematical expressions, that are prohibitively large, need not be explicitly obtained. The diversity gain due to multiple relays is shown through plots of the analytical BER, well supported by simulations. 
%
%\end{abstract}
% IEEEtran.cls defaults to using nonbold math in the Abstract.
% This preserves the distinction between vectors and scalars. However,
% if the journal you are submitting to favors bold math in the abstract,
% then you can use LaTeX's standard command \boldmath at the very start
% of the abstract to achieve this. Many IEEE journals frown on math
% in the abstract anyway.

% Note that keywords are not normally used for peerreview papers.
%\begin{IEEEkeywords}
%Cooperative diversity, decode and forward, piecewise linear
%\end{IEEEkeywords}



% For peer review papers, you can put extra information on the cover
% page as needed:
% \ifCLASSOPTIONpeerreview
% \begin{center} \bfseries EDICS Category: 3-BBND \end{center}
% \fi
%
% For peerreview papers, this IEEEtran command inserts a page break and
% creates the second title. It will be ignored for other modes.
%\IEEEpeerreviewmaketitle




	\item  A die is loaded in such a way that each odd number is twice as likely to occur as
each even number. Find $P(G)$, where $G$ is the event that a number greater than
3 occurs on a single roll of the die.
\\
\solution
		%\begin{table}[H]
	\centering
\begin{tabular}{|c|c|c|}
\hline
Random variable &Value &Definition\\ \hline
\multirow{3}{*}{X} &0 &Slips of Rs 1\\
&1 &Slips of Rs 5\\
&2 &Slips of Rs 13\\ \hline
\multirow{2}{*}{Y} &0 &Box A\\
&1 &Box B\\\hline
\end{tabular}
\caption{}
\label{tab:Distribution}
\end{table}
See \tabref{tab:Distribution}.
\begin{align}
p_{Y}\brak{k}= \begin{cases} 
      \frac{1}{3} & {k=0} \\
      \frac{2}{3 }& {k=1} 
   \end{cases}
   \\
p_{Y|X}\brak{0|0} = \frac{19}{25}\, 
p_{Y|X}\brak{0|1} = \frac{6}{25}\,
p_{Y|X}\brak{1|0} = \frac{45}{50}\,
p_{Y|X}\brak{1|2} = \frac{5}{50}
\end{align}
The desired probability is the probability that a slip drawn at random is marked other than Rs 1,
\begin{align}
&=1-p_X\brak{0}\\
&= p_X(1) + p_X(2)
\end{align}
Using Bayes theorem,
\begin{align}
&= p_Y\brak{0} \times \pr{Y=0 | X=1} + p_Y\brak{1} \times \pr{Y=1|X=2}\\
&=\frac{1}{3} \times \frac{6}{25} + \frac{2}{3} \times \frac{5}{50}\\
&=\frac{11}{75}
\end{align}

\newpage

%\tableofcontents

\bigskip

\renewcommand{\thefigure}{\theenumi}
\renewcommand{\thetable}{\theenumi}
%\renewcommand{\theequation}{\theenumi}

%\begin{abstract}
%%\boldmath
%In this letter, an algorithm for evaluating the exact analytical bit error rate  (BER)  for the piecewise linear (PL) combiner for  multiple relays is presented. Previous results were available only for upto three relays. The algorithm is unique in the sense that  the actual mathematical expressions, that are prohibitively large, need not be explicitly obtained. The diversity gain due to multiple relays is shown through plots of the analytical BER, well supported by simulations. 
%
%\end{abstract}
% IEEEtran.cls defaults to using nonbold math in the Abstract.
% This preserves the distinction between vectors and scalars. However,
% if the journal you are submitting to favors bold math in the abstract,
% then you can use LaTeX's standard command \boldmath at the very start
% of the abstract to achieve this. Many IEEE journals frown on math
% in the abstract anyway.

% Note that keywords are not normally used for peerreview papers.
%\begin{IEEEkeywords}
%Cooperative diversity, decode and forward, piecewise linear
%\end{IEEEkeywords}



% For peer review papers, you can put extra information on the cover
% page as needed:
% \ifCLASSOPTIONpeerreview
% \begin{center} \bfseries EDICS Category: 3-BBND \end{center}
% \fi
%
% For peerreview papers, this IEEEtran command inserts a page break and
% creates the second title. It will be ignored for other modes.
%\IEEEpeerreviewmaketitle




	\item All the jacks, queens and kings are removed from a deck of 52 playing cards. The remaining cards are well shuffled and then one card is drawn at random. Giving ace a value 1 similar value for other cards, find the probability that the card has a value 
		\begin{enumerate}
			\item 7
			\item greater than 7
			\item less than 7
		\end{enumerate}
		%Number of cards left after removing all jacks, queens and kings 
\begin{align}
N	= 52 - 4\times 3
	= 40
\end{align}
%\begin{table}[H]
%\def\arraystretch{1.2}
%\begin{tabular}{|c|c|c|}
%\hline
%	\textbf{Parameter} &\textbf{Value} &\textbf{Description}\\ \hline
%	$X$ &1-10 &Represents the value of the card picked \\ \hline
%\end{tabular}
%\end{table}
Let $1 \le X \le 10$ be the value of the card picked.  Then,
\begin{align}
	p_X(k) &= \Pr(X=k)\ \forall\ 1 \leq k \leq 10\\
	&= \frac{4\times 1}{40}\\
	&= \frac{1}{10}\\
	\therefore p_X(k) &= 
	\begin{cases}
		\frac{1}{10} & 1 \leq k \leq 10\\
		0 & \text{otherwise}
	\end{cases}
\end{align}
and
\begin{align}
	F_{X}(k) &= \sum_{m=0}^{k}p_{X}(m) \quad 1 \leq k \leq 10\\
	&= \frac{k}{10}\\
	\therefore F_{X}(k) &= 
	\begin{cases}
		0 & k \leq 0\\
		\frac{k}{10} & 1\leq k \leq 10\\
		1 & k > 10 
	\end{cases}
\end{align}
\begin{enumerate}
	\item Probability that card has value equal to 7 is
		\begin{align}
			 p_{X}(7)
			= \frac{1}{10}
		\end{align}
	\item Probability that card has value greater than 7 is
		\begin{align}
			1 - F_X(7)
			&= 1 - \frac{7}{10}
			\\
			&= \frac{3}{10}
		\end{align}
	\item Probability that card has value less than 7 is
		\begin{align}
			 F_{X}(6)
			=\frac{6}{10}
		\end{align}
\end{enumerate}

  \item A Lot consists of 48 mobile phones of which 42 are good, 3 have only minor defects and 3 have major defects.Varnika will buy a phone if it is good but the trader will only buy a mobile if it has no major defects. One phone is selected at random from the lot. What is the probability that it is
\begin{enumerate}
	\item acceptable to Varnika?
            \item acceptable to the trader?
\end{enumerate}
\solution
	%\begin{table}[H]
	\centering
\begin{tabular}{|c|c|c|}
\hline
Random variable &Value &Definition\\ \hline
\multirow{3}{*}{X} &0 &Slips of Rs 1\\
&1 &Slips of Rs 5\\
&2 &Slips of Rs 13\\ \hline
\multirow{2}{*}{Y} &0 &Box A\\
&1 &Box B\\\hline
\end{tabular}
\caption{}
\label{tab:Distribution}
\end{table}
See \tabref{tab:Distribution}.
\begin{align}
p_{Y}\brak{k}= \begin{cases} 
      \frac{1}{3} & {k=0} \\
      \frac{2}{3 }& {k=1} 
   \end{cases}
   \\
p_{Y|X}\brak{0|0} = \frac{19}{25}\, 
p_{Y|X}\brak{0|1} = \frac{6}{25}\,
p_{Y|X}\brak{1|0} = \frac{45}{50}\,
p_{Y|X}\brak{1|2} = \frac{5}{50}
\end{align}
The desired probability is the probability that a slip drawn at random is marked other than Rs 1,
\begin{align}
&=1-p_X\brak{0}\\
&= p_X(1) + p_X(2)
\end{align}
Using Bayes theorem,
\begin{align}
&= p_Y\brak{0} \times \pr{Y=0 | X=1} + p_Y\brak{1} \times \pr{Y=1|X=2}\\
&=\frac{1}{3} \times \frac{6}{25} + \frac{2}{3} \times \frac{5}{50}\\
&=\frac{11}{75}
\end{align}

\newpage

%\tableofcontents

\bigskip

\renewcommand{\thefigure}{\theenumi}
\renewcommand{\thetable}{\theenumi}
%\renewcommand{\theequation}{\theenumi}

%\begin{abstract}
%%\boldmath
%In this letter, an algorithm for evaluating the exact analytical bit error rate  (BER)  for the piecewise linear (PL) combiner for  multiple relays is presented. Previous results were available only for upto three relays. The algorithm is unique in the sense that  the actual mathematical expressions, that are prohibitively large, need not be explicitly obtained. The diversity gain due to multiple relays is shown through plots of the analytical BER, well supported by simulations. 
%
%\end{abstract}
% IEEEtran.cls defaults to using nonbold math in the Abstract.
% This preserves the distinction between vectors and scalars. However,
% if the journal you are submitting to favors bold math in the abstract,
% then you can use LaTeX's standard command \boldmath at the very start
% of the abstract to achieve this. Many IEEE journals frown on math
% in the abstract anyway.

% Note that keywords are not normally used for peerreview papers.
%\begin{IEEEkeywords}
%Cooperative diversity, decode and forward, piecewise linear
%\end{IEEEkeywords}



% For peer review papers, you can put extra information on the cover
% page as needed:
% \ifCLASSOPTIONpeerreview
% \begin{center} \bfseries EDICS Category: 3-BBND \end{center}
% \fi
%
% For peerreview papers, this IEEEtran command inserts a page break and
% creates the second title. It will be ignored for other modes.
%\IEEEpeerreviewmaketitle




 \item A student says that if you throw a die, it will show up 1 or not 1. Therefore, the probability of getting 1 and the probability of getting 'not 1' each is equal to $\frac{1}{2}$. Is this correct? Give reasons.\\
 \solution
        %\begin{table}[H]
	\centering
\begin{tabular}{|c|c|c|}
\hline
Random variable &Value &Definition\\ \hline
\multirow{3}{*}{X} &0 &Slips of Rs 1\\
&1 &Slips of Rs 5\\
&2 &Slips of Rs 13\\ \hline
\multirow{2}{*}{Y} &0 &Box A\\
&1 &Box B\\\hline
\end{tabular}
\caption{}
\label{tab:Distribution}
\end{table}
See \tabref{tab:Distribution}.
\begin{align}
p_{Y}\brak{k}= \begin{cases} 
      \frac{1}{3} & {k=0} \\
      \frac{2}{3 }& {k=1} 
   \end{cases}
   \\
p_{Y|X}\brak{0|0} = \frac{19}{25}\, 
p_{Y|X}\brak{0|1} = \frac{6}{25}\,
p_{Y|X}\brak{1|0} = \frac{45}{50}\,
p_{Y|X}\brak{1|2} = \frac{5}{50}
\end{align}
The desired probability is the probability that a slip drawn at random is marked other than Rs 1,
\begin{align}
&=1-p_X\brak{0}\\
&= p_X(1) + p_X(2)
\end{align}
Using Bayes theorem,
\begin{align}
&= p_Y\brak{0} \times \pr{Y=0 | X=1} + p_Y\brak{1} \times \pr{Y=1|X=2}\\
&=\frac{1}{3} \times \frac{6}{25} + \frac{2}{3} \times \frac{5}{50}\\
&=\frac{11}{75}
\end{align}

\newpage

%\tableofcontents

\bigskip

\renewcommand{\thefigure}{\theenumi}
\renewcommand{\thetable}{\theenumi}
%\renewcommand{\theequation}{\theenumi}

%\begin{abstract}
%%\boldmath
%In this letter, an algorithm for evaluating the exact analytical bit error rate  (BER)  for the piecewise linear (PL) combiner for  multiple relays is presented. Previous results were available only for upto three relays. The algorithm is unique in the sense that  the actual mathematical expressions, that are prohibitively large, need not be explicitly obtained. The diversity gain due to multiple relays is shown through plots of the analytical BER, well supported by simulations. 
%
%\end{abstract}
% IEEEtran.cls defaults to using nonbold math in the Abstract.
% This preserves the distinction between vectors and scalars. However,
% if the journal you are submitting to favors bold math in the abstract,
% then you can use LaTeX's standard command \boldmath at the very start
% of the abstract to achieve this. Many IEEE journals frown on math
% in the abstract anyway.

% Note that keywords are not normally used for peerreview papers.
%\begin{IEEEkeywords}
%Cooperative diversity, decode and forward, piecewise linear
%\end{IEEEkeywords}



% For peer review papers, you can put extra information on the cover
% page as needed:
% \ifCLASSOPTIONpeerreview
% \begin{center} \bfseries EDICS Category: 3-BBND \end{center}
% \fi
%
% For peerreview papers, this IEEEtran command inserts a page break and
% creates the second title. It will be ignored for other modes.
%\IEEEpeerreviewmaketitle




   \item Four candidates A, B, C, D have ap-
plied for the assignment to coach a school cricket
team. If A is twice as likely to be selected as B, and
B and C are given about the same chance of being
selected, while C is twice as likely to be selected
as D, what are the probabilities that
\begin{enumerate}
\item C will be selected?
\item A will not be selected?
\end{enumerate}
	%\begin{table}[H]
	\centering
\begin{tabular}{|c|c|c|}
\hline
Random variable &Value &Definition\\ \hline
\multirow{3}{*}{X} &0 &Slips of Rs 1\\
&1 &Slips of Rs 5\\
&2 &Slips of Rs 13\\ \hline
\multirow{2}{*}{Y} &0 &Box A\\
&1 &Box B\\\hline
\end{tabular}
\caption{}
\label{tab:Distribution}
\end{table}
See \tabref{tab:Distribution}.
\begin{align}
p_{Y}\brak{k}= \begin{cases} 
      \frac{1}{3} & {k=0} \\
      \frac{2}{3 }& {k=1} 
   \end{cases}
   \\
p_{Y|X}\brak{0|0} = \frac{19}{25}\, 
p_{Y|X}\brak{0|1} = \frac{6}{25}\,
p_{Y|X}\brak{1|0} = \frac{45}{50}\,
p_{Y|X}\brak{1|2} = \frac{5}{50}
\end{align}
The desired probability is the probability that a slip drawn at random is marked other than Rs 1,
\begin{align}
&=1-p_X\brak{0}\\
&= p_X(1) + p_X(2)
\end{align}
Using Bayes theorem,
\begin{align}
&= p_Y\brak{0} \times \pr{Y=0 | X=1} + p_Y\brak{1} \times \pr{Y=1|X=2}\\
&=\frac{1}{3} \times \frac{6}{25} + \frac{2}{3} \times \frac{5}{50}\\
&=\frac{11}{75}
\end{align}

\newpage

%\tableofcontents

\bigskip

\renewcommand{\thefigure}{\theenumi}
\renewcommand{\thetable}{\theenumi}
%\renewcommand{\theequation}{\theenumi}

%\begin{abstract}
%%\boldmath
%In this letter, an algorithm for evaluating the exact analytical bit error rate  (BER)  for the piecewise linear (PL) combiner for  multiple relays is presented. Previous results were available only for upto three relays. The algorithm is unique in the sense that  the actual mathematical expressions, that are prohibitively large, need not be explicitly obtained. The diversity gain due to multiple relays is shown through plots of the analytical BER, well supported by simulations. 
%
%\end{abstract}
% IEEEtran.cls defaults to using nonbold math in the Abstract.
% This preserves the distinction between vectors and scalars. However,
% if the journal you are submitting to favors bold math in the abstract,
% then you can use LaTeX's standard command \boldmath at the very start
% of the abstract to achieve this. Many IEEE journals frown on math
% in the abstract anyway.

% Note that keywords are not normally used for peerreview papers.
%\begin{IEEEkeywords}
%Cooperative diversity, decode and forward, piecewise linear
%\end{IEEEkeywords}



% For peer review papers, you can put extra information on the cover
% page as needed:
% \ifCLASSOPTIONpeerreview
% \begin{center} \bfseries EDICS Category: 3-BBND \end{center}
% \fi
%
% For peerreview papers, this IEEEtran command inserts a page break and
% creates the second title. It will be ignored for other modes.
%\IEEEpeerreviewmaketitle




 \item A bag contain 24 balls of which $x$ balls are red, $2x$ are white and $3x$ are blue. A ball is selected at random, What is the probability that it is
\begin{enumerate}[label=\alph*)]
\item not red ?
\item white ?
\end{enumerate}
%\begin{table}[H]
	\centering
\begin{tabular}{|c|c|c|}
\hline
Random variable &Value &Definition\\ \hline
\multirow{3}{*}{X} &0 &Slips of Rs 1\\
&1 &Slips of Rs 5\\
&2 &Slips of Rs 13\\ \hline
\multirow{2}{*}{Y} &0 &Box A\\
&1 &Box B\\\hline
\end{tabular}
\caption{}
\label{tab:Distribution}
\end{table}
See \tabref{tab:Distribution}.
\begin{align}
p_{Y}\brak{k}= \begin{cases} 
      \frac{1}{3} & {k=0} \\
      \frac{2}{3 }& {k=1} 
   \end{cases}
   \\
p_{Y|X}\brak{0|0} = \frac{19}{25}\, 
p_{Y|X}\brak{0|1} = \frac{6}{25}\,
p_{Y|X}\brak{1|0} = \frac{45}{50}\,
p_{Y|X}\brak{1|2} = \frac{5}{50}
\end{align}
The desired probability is the probability that a slip drawn at random is marked other than Rs 1,
\begin{align}
&=1-p_X\brak{0}\\
&= p_X(1) + p_X(2)
\end{align}
Using Bayes theorem,
\begin{align}
&= p_Y\brak{0} \times \pr{Y=0 | X=1} + p_Y\brak{1} \times \pr{Y=1|X=2}\\
&=\frac{1}{3} \times \frac{6}{25} + \frac{2}{3} \times \frac{5}{50}\\
&=\frac{11}{75}
\end{align}

\newpage

%\tableofcontents

\bigskip

\renewcommand{\thefigure}{\theenumi}
\renewcommand{\thetable}{\theenumi}
%\renewcommand{\theequation}{\theenumi}

%\begin{abstract}
%%\boldmath
%In this letter, an algorithm for evaluating the exact analytical bit error rate  (BER)  for the piecewise linear (PL) combiner for  multiple relays is presented. Previous results were available only for upto three relays. The algorithm is unique in the sense that  the actual mathematical expressions, that are prohibitively large, need not be explicitly obtained. The diversity gain due to multiple relays is shown through plots of the analytical BER, well supported by simulations. 
%
%\end{abstract}
% IEEEtran.cls defaults to using nonbold math in the Abstract.
% This preserves the distinction between vectors and scalars. However,
% if the journal you are submitting to favors bold math in the abstract,
% then you can use LaTeX's standard command \boldmath at the very start
% of the abstract to achieve this. Many IEEE journals frown on math
% in the abstract anyway.

% Note that keywords are not normally used for peerreview papers.
%\begin{IEEEkeywords}
%Cooperative diversity, decode and forward, piecewise linear
%\end{IEEEkeywords}



% For peer review papers, you can put extra information on the cover
% page as needed:
% \ifCLASSOPTIONpeerreview
% \begin{center} \bfseries EDICS Category: 3-BBND \end{center}
% \fi
%
% For peerreview papers, this IEEEtran command inserts a page break and
% creates the second title. It will be ignored for other modes.
%\IEEEpeerreviewmaketitle




If the letters of the word ASSASSINATION are arranged at random. Find the Probability that
\begin{enumerate}[label=(\alph*)]
\item Four $S's$ come consecutively in the word
\item Two  $I's$ and two $N's$ come together
\item All $A's$ are not coming together
\item No two $A's$ are coming together
\end{enumerate}
%\begin{table}[H]
	\centering
\begin{tabular}{|c|c|c|}
\hline
Random variable &Value &Definition\\ \hline
\multirow{3}{*}{X} &0 &Slips of Rs 1\\
&1 &Slips of Rs 5\\
&2 &Slips of Rs 13\\ \hline
\multirow{2}{*}{Y} &0 &Box A\\
&1 &Box B\\\hline
\end{tabular}
\caption{}
\label{tab:Distribution}
\end{table}
See \tabref{tab:Distribution}.
\begin{align}
p_{Y}\brak{k}= \begin{cases} 
      \frac{1}{3} & {k=0} \\
      \frac{2}{3 }& {k=1} 
   \end{cases}
   \\
p_{Y|X}\brak{0|0} = \frac{19}{25}\, 
p_{Y|X}\brak{0|1} = \frac{6}{25}\,
p_{Y|X}\brak{1|0} = \frac{45}{50}\,
p_{Y|X}\brak{1|2} = \frac{5}{50}
\end{align}
The desired probability is the probability that a slip drawn at random is marked other than Rs 1,
\begin{align}
&=1-p_X\brak{0}\\
&= p_X(1) + p_X(2)
\end{align}
Using Bayes theorem,
\begin{align}
&= p_Y\brak{0} \times \pr{Y=0 | X=1} + p_Y\brak{1} \times \pr{Y=1|X=2}\\
&=\frac{1}{3} \times \frac{6}{25} + \frac{2}{3} \times \frac{5}{50}\\
&=\frac{11}{75}
\end{align}

\newpage

%\tableofcontents

\bigskip

\renewcommand{\thefigure}{\theenumi}
\renewcommand{\thetable}{\theenumi}
%\renewcommand{\theequation}{\theenumi}

%\begin{abstract}
%%\boldmath
%In this letter, an algorithm for evaluating the exact analytical bit error rate  (BER)  for the piecewise linear (PL) combiner for  multiple relays is presented. Previous results were available only for upto three relays. The algorithm is unique in the sense that  the actual mathematical expressions, that are prohibitively large, need not be explicitly obtained. The diversity gain due to multiple relays is shown through plots of the analytical BER, well supported by simulations. 
%
%\end{abstract}
% IEEEtran.cls defaults to using nonbold math in the Abstract.
% This preserves the distinction between vectors and scalars. However,
% if the journal you are submitting to favors bold math in the abstract,
% then you can use LaTeX's standard command \boldmath at the very start
% of the abstract to achieve this. Many IEEE journals frown on math
% in the abstract anyway.

% Note that keywords are not normally used for peerreview papers.
%\begin{IEEEkeywords}
%Cooperative diversity, decode and forward, piecewise linear
%\end{IEEEkeywords}



% For peer review papers, you can put extra information on the cover
% page as needed:
% \ifCLASSOPTIONpeerreview
% \begin{center} \bfseries EDICS Category: 3-BBND \end{center}
% \fi
%
% For peerreview papers, this IEEEtran command inserts a page break and
% creates the second title. It will be ignored for other modes.
%\IEEEpeerreviewmaketitle




	\item One urn contains two black balls (labelled B1 and B2) and one white ball. A
	second urn contains one black ball and two white balls (labelled W1 and W2).
	Suppose the following experiment is performed. One of the two urns is chosen
	at random. Next a ball is randomly chosen from the urn. Then a second ball is
	chosen at random from the same urn without replacing the first ball.
	
	\begin{enumerate}
	\item What is the probability that two black balls are chosen?
	
	\item What is the probability that two balls of opposite colour are chosen?
	\end{enumerate}
	\solution
	%\begin{align}
    \label{eq:12.13.6.18.1}
	\because	\pr{A|B} &> \pr{A},\
\frac{\pr{AB}}{\pr{B}} > \pr{A}
\\
    \label{eq:12.13.6.18.2}
	\implies \pr{AB} &> \pr{A}\pr{B}
	\\
	\text{or, } \frac{\pr{AB}}{\pr{A}} &=\pr{B|A} > \pr{A}
\end{align}

\end{enumerate}

		%
\item 
Two cards are drawn at random and without replacement from a pack of 52 playing cards. Find the probability that both the cards are black.
\\
\solution
		%\begin{enumerate}[label=\thesection.\arabic*,ref=\thesection.\theenumi]
	\item One card is drawn from a well-shuffled deck of 52 cards. Find the probability of getting
\begin{enumerate}
\item A king of red colour 
\item A face card 
\item A red face card
\item The jack of hearts
\item A spade
\item The queen of diamonds

\end{enumerate}
\solution
		%\begin{table}[H]
	\centering
\begin{tabular}{|c|c|c|}
\hline
Random variable &Value &Definition\\ \hline
\multirow{3}{*}{X} &0 &Slips of Rs 1\\
&1 &Slips of Rs 5\\
&2 &Slips of Rs 13\\ \hline
\multirow{2}{*}{Y} &0 &Box A\\
&1 &Box B\\\hline
\end{tabular}
\caption{}
\label{tab:Distribution}
\end{table}
See \tabref{tab:Distribution}.
\begin{align}
p_{Y}\brak{k}= \begin{cases} 
      \frac{1}{3} & {k=0} \\
      \frac{2}{3 }& {k=1} 
   \end{cases}
   \\
p_{Y|X}\brak{0|0} = \frac{19}{25}\, 
p_{Y|X}\brak{0|1} = \frac{6}{25}\,
p_{Y|X}\brak{1|0} = \frac{45}{50}\,
p_{Y|X}\brak{1|2} = \frac{5}{50}
\end{align}
The desired probability is the probability that a slip drawn at random is marked other than Rs 1,
\begin{align}
&=1-p_X\brak{0}\\
&= p_X(1) + p_X(2)
\end{align}
Using Bayes theorem,
\begin{align}
&= p_Y\brak{0} \times \pr{Y=0 | X=1} + p_Y\brak{1} \times \pr{Y=1|X=2}\\
&=\frac{1}{3} \times \frac{6}{25} + \frac{2}{3} \times \frac{5}{50}\\
&=\frac{11}{75}
\end{align}

\newpage

%\tableofcontents

\bigskip

\renewcommand{\thefigure}{\theenumi}
\renewcommand{\thetable}{\theenumi}
%\renewcommand{\theequation}{\theenumi}

%\begin{abstract}
%%\boldmath
%In this letter, an algorithm for evaluating the exact analytical bit error rate  (BER)  for the piecewise linear (PL) combiner for  multiple relays is presented. Previous results were available only for upto three relays. The algorithm is unique in the sense that  the actual mathematical expressions, that are prohibitively large, need not be explicitly obtained. The diversity gain due to multiple relays is shown through plots of the analytical BER, well supported by simulations. 
%
%\end{abstract}
% IEEEtran.cls defaults to using nonbold math in the Abstract.
% This preserves the distinction between vectors and scalars. However,
% if the journal you are submitting to favors bold math in the abstract,
% then you can use LaTeX's standard command \boldmath at the very start
% of the abstract to achieve this. Many IEEE journals frown on math
% in the abstract anyway.

% Note that keywords are not normally used for peerreview papers.
%\begin{IEEEkeywords}
%Cooperative diversity, decode and forward, piecewise linear
%\end{IEEEkeywords}



% For peer review papers, you can put extra information on the cover
% page as needed:
% \ifCLASSOPTIONpeerreview
% \begin{center} \bfseries EDICS Category: 3-BBND \end{center}
% \fi
%
% For peerreview papers, this IEEEtran command inserts a page break and
% creates the second title. It will be ignored for other modes.
%\IEEEpeerreviewmaketitle




	\item Five cards—the ten, jack, queen, king and ace of diamonds, are well-shuffled with their face downwards. One card is then picked up at random.
\begin{enumerate}
\item
What is the probability that the card is the queen? 
\item
If the queen is drawn and put aside, what is the probability that the second card picked up is (a) an ace? (b) a queen?\\
\end{enumerate}
\solution
		%\begin{enumerate}[label=\thesection.\arabic*,ref=\thesection.\theenumi]
	\item One card is drawn from a well-shuffled deck of 52 cards. Find the probability of getting
\begin{enumerate}
\item A king of red colour 
\item A face card 
\item A red face card
\item The jack of hearts
\item A spade
\item The queen of diamonds

\end{enumerate}
\solution
		%\input{ncert/10/15/1/14/main.tex}
	\item Five cards—the ten, jack, queen, king and ace of diamonds, are well-shuffled with their face downwards. One card is then picked up at random.
\begin{enumerate}
\item
What is the probability that the card is the queen? 
\item
If the queen is drawn and put aside, what is the probability that the second card picked up is (a) an ace? (b) a queen?\\
\end{enumerate}
\solution
		%\input{ncert/10/15/1/15/defs.tex}
	\item A bag contains $5$ red balls and some blue balls. If the probability of drawing a blue ball is double that if a red ball, determine the number of blue balls in the bag. 
		\\
\solution
		%\input{ncert/10/15/2/3/defs.tex}
	\item A card is selected from a pack of 52 cards.
 \begin{enumerate}[label=(\alph*)] 
                 \item How many points are there in the sample space?
                 \item Calculate the probability that the card is an ace of spades.
                 \item Calculate the probability that the card is (i) an ace and (ii) black card.
 \end{enumerate}
\solution
		%\input{ncert/11/16/3/4/main.tex}
\item Four cards are drawn from a well-shuffled deck of 52 cards. What is the probability of obtaining 3 diamonds and one spade.
\\
\solution
		%\input{ncert/11/16/4/2/defs.tex}
\item In a certain lottery 10,000 tickets are sold and ten equal prizes are awarded. What is the probability of not getting a prize if you buy (a) one ticket (b) two tickets (c) 10 tickets ?	
\\
\solution
		%\input{ncert/11/16/4/4/defs.tex}
		%
\item 
Out of 100 students, two sections of 40 and 60 are formed. If you and your friend are among the 100 students, what is the probability that
\begin{enumerate}
\item you both enter the same section?
\item you both enter the different sections?
\end{enumerate}
\solution
		%\input{ncert/11/16/4/5/defs.tex}
	\item 
The number lock of a suitcase has 4 wheels each labelled with ten digits i.e. from 0 to 9.The lock opens with a sequence of four digits with no repeats.What is the probability of a person getting the right sequence to open the suitcase.
\\
\solution
		%\input{ncert/11/16/4/10/defs.tex}
		%
\item 
Two cards are drawn at random and without replacement from a pack of 52 playing cards. Find the probability that both the cards are black.
\\
\solution
		%\input{ncert/12/13/2/2/defs.tex}
		\item A box of oranges is inspected by examining three randomly selected oranges drawn without replacement. If all the three oranges are good, the box is approved for sale, otherwise, it is rejected. Find the probability that a box containing 15 oranges out of which 12 are good and 3 are bad ones will be approved for sale.
		\label{ncert/12/13/2/3/defs.tex}
		\item Two balls are drawn at random with replacement from a box containing 10 black and 8 red balls. Find the probability that
		\label{ncert/12/13/2/12}
\begin{enumerate}
\item both balls are red.
\item first ball is black and second is red.
\item one of them is black and other is red.
\end{enumerate}

\item In a hostel, 60\% of the students read Hindi newspaper, 40\% read English newspaper and 20\% read both Hindi and English newspapers. A student is selected at random.
		\label{ncert/12/13/2/15}
\begin{enumerate}
\item Find the probability that she reads neither Hindi nor English newspapers.
\item If she reads Hindi newspaper, find the probability that she reads English newspaper.
\item If she reads English newspaper, find the probability that she reads Hindi newspaper.\\
\end{enumerate}
\item The probability of obtaining an even prime number on each die, when a pair of dice is rolled is 
\begin{enumerate}
    \item $0$ 
    
    \item $\frac{1}{3}$ 
    
    \item $\frac{1}{12}$ 
    
    \item $\frac{1}{36}$ 
\end{enumerate}
\solution
		%\input{ncert/12/13/2/17/defs.tex}
	\item A bag contains 4 red and 4 black balls, another bag contains 2 red and 6 black balls. One of the two bags is selected at random and a ball is drawn from the bag which is found to be red. Find the probability that the ball is drawn from the first bag.
\\
\solution
		%\input{ncert/12/13/3/2/main.tex}
  \item
  Cards with numbers 2 to 101 are placed in a box. A card is selected at random.Find the probability that the card has
\begin{enumerate}[label=(\roman*)]
	\item an even number 
	\item a square number
\end{enumerate}
\solution
%\input{exemplar/10/13/3/32/main.tex}
\item
The king, queen and jack of clubs are removed from a deck of 52 playing cards and then well shuffled. Now one card is drawn at random from the remaining cards.  Determine the probability that the card is
\begin{enumerate}[label=(\roman*)]
\item a club
\item 10 of hearts
\end{enumerate}
\solution
%\input{exemplar/10/13/3/29/main.tex}
\item A team of medical students doing their internship have to assist during surgeries
at a city hospital. The probabilities of surgeries rated as very complex, complex,
routine, simple or very simple are respectively, 0.15, 0.20, 0.31, 0.26, .08. Find
the probabilities that a particular surgery will be rated
\begin{enumerate}
	\item complex or very complex;
	\item neither very complex nor very simple;
	\item routine or complex
	\item routine or simple
\end{enumerate}
\solution
%\input{exemplar/11/16/3/8(1)/main.tex}
\item A card is selected from a pack of 52 cards.
\begin{enumerate}[label=(\alph*)]
    \item How many points are there in the sample space?
    \item Calculate the probability that the card is an ace of spades.
    \item Calculate the probability that the card is (i) an ace and (ii) black card.
\end{enumerate}
\solution
%\input{exemplar/11/16/3/4/main2.tex}
\item The probability that a non leap year selected at random will contain 53 sundays.
\\
\solution
%\input{exemplar/10/13/1/19/main.tex}
\item One of the four persons John, Rita, Aslam or Gurpreet will be promoted next
month. Consequently the sample space consists of four elementary outcomes
S = {John promoted, Rita promoted, Aslam promoted, Gurpreet promoted}
You are told that the chances of John’s promotion is same as that of Gurpreet,
Rita’s chances of promotion are twice as likely as Johns. Aslam’s chances are
four times that of John.
\begin{enumerate}
	\item Determine
	\begin{enumerate}
		\item P (John promoted)
		\item P (Rita promoted)
		\item P (Aslam promoted)
		\item P (Gurpreet promoted)
	\end{enumerate}
	\item If A = {John promoted or Gurpreet promoted}, find P (A).
\end{enumerate}
\solution
%\input{exemplar/11/16/3/10/main.tex}
\item A card is drawn from a deck of 52 cards. Find the probability of getting a king or a heart or a red card.\\
\solution
%\input{exemplar/11/16/3/15/main.tex}
\item The probability that a student will pass his examination is 0.73, the probability of
the student getting a compartment is 0.13, and the probability that the student will
either pass or get compartment is 0.96. State True or False.\\
\solution
%\input{exemplar/11/16/3/31/main.tex}
\item A card is selected from a pack of 52 cards\\
\begin{enumerate}[label=(\alph*)]
\item How many points are there in the sample space?
\item Calculate the probability that the cards is an ace of spades.
\item Calculate the probability that the card is (i) an ace (ii)black card.\\
\end{enumerate}
%\input{ncert/11/16/3/4_1/Prob_4.tex}
\item In a non-leap year, the probability of having 53 tuesdays or 53 wednesdays is\\
\solution
%\input{exemplar/11/16/3/18/main.tex}
\item There are 1000 sealed envelopes in a box, 10 of them contain a cash prize of
Rs 100 each, 100 of them contain a cash prize of Rs 50 each and 200 of them
contain a cash prize of Rs 10 each and rest do not contain any cash prize. If they
are well shuffled and an envelope is picked up out, what is the probability that it
contains no cash prize?\\
\solution
%\input{exemplar/10/13/3/34/main.tex}
\item 
A die is thrown and a card is selected at random from a deck of 52 playing cards. The probability of getting an even number on the die and a spade card.\\
\solution
%\input{exemplar/12/13/3/78/main.tex}
\item
If 4-digit numbers greater than 5,000 are randomly formed from the digits 0, 1, 3, 5, and 7, what is the probability of forming a number divisible by 5 when:
\begin{enumerate}
    \item The digits are repeated?
    \item The repetition of digits is not allowed?
\end{enumerate}
\solution
%\input{ncert/11/16/4/9/main.tex}
\item Consider the probability space $\brak{\Omega, \mathcal{G}, P}$ where $\Omega = [0,2]$ and $\mathcal{G} = \cbrak{\phi, \Omega, [0,1], (1,2]}$. Let $X$ and $Y$ be two functions on $\Omega$ defined as
\begin{align*}
    X(\omega) = 
    \begin{cases}
        1 & \text{if }\omega \in [0, 1]\\
        2 & \text{if }\omega \in (1, 2]
    \end{cases}
\end{align*}
and
\begin{align*}
    Y(\omega) = 
    \begin{cases}
        2 & \text{if }\omega \in [0, 1.5]\\
        3 & \text{if }\omega \in (1.5, 2].
    \end{cases}
\end{align*}
Then which one of the following statements is true?
\begin{enumerate}
    \item [(A)] $X$ is a random variable with respect to $\mathcal{G}$, but $Y$ is not a random variable with respect to $\mathcal{G}$.
    \item [(B)] $Y$ is a random variable with respect to $\mathcal{G}$, but $X$ is not a random variable with respect to $\mathcal{G}$.
    \item [(C)] Neither $X$ nor $Y$ is a random variable with respect to $\mathcal{G}$.
    \item [(D)] Both $X$ and $Y$ are random variables with respect to $\mathcal{G}$.
\end{enumerate} \hfill (GATE ST 2023)\\
\solution
%\input{gate/ST/2023/14/main.tex}
	\item  A die is loaded in such a way that each odd number is twice as likely to occur as
each even number. Find $P(G)$, where $G$ is the event that a number greater than
3 occurs on a single roll of the die.
\\
\solution
		%\input{exemplar/11/16/3/5/main.tex}
	\item All the jacks, queens and kings are removed from a deck of 52 playing cards. The remaining cards are well shuffled and then one card is drawn at random. Giving ace a value 1 similar value for other cards, find the probability that the card has a value 
		\begin{enumerate}
			\item 7
			\item greater than 7
			\item less than 7
		\end{enumerate}
		%\input{exemplar/10/13/3/30/main.tex}
  \item A Lot consists of 48 mobile phones of which 42 are good, 3 have only minor defects and 3 have major defects.Varnika will buy a phone if it is good but the trader will only buy a mobile if it has no major defects. One phone is selected at random from the lot. What is the probability that it is
\begin{enumerate}
	\item acceptable to Varnika?
            \item acceptable to the trader?
\end{enumerate}
\solution
	%\input{exemplar/10/13/3/40/main.tex}
 \item A student says that if you throw a die, it will show up 1 or not 1. Therefore, the probability of getting 1 and the probability of getting 'not 1' each is equal to $\frac{1}{2}$. Is this correct? Give reasons.\\
 \solution
        %\input{exemplar/10/13/2/9/main.tex}
   \item Four candidates A, B, C, D have ap-
plied for the assignment to coach a school cricket
team. If A is twice as likely to be selected as B, and
B and C are given about the same chance of being
selected, while C is twice as likely to be selected
as D, what are the probabilities that
\begin{enumerate}
\item C will be selected?
\item A will not be selected?
\end{enumerate}
	%\input{exemplar/11/16/3/9/main.tex}
 \item A bag contain 24 balls of which $x$ balls are red, $2x$ are white and $3x$ are blue. A ball is selected at random, What is the probability that it is
\begin{enumerate}[label=\alph*)]
\item not red ?
\item white ?
\end{enumerate}
%\input{exemplar/10/13/3/41/main.tex}
If the letters of the word ASSASSINATION are arranged at random. Find the Probability that
\begin{enumerate}[label=(\alph*)]
\item Four $S's$ come consecutively in the word
\item Two  $I's$ and two $N's$ come together
\item All $A's$ are not coming together
\item No two $A's$ are coming together
\end{enumerate}
%\input{exemplar/11/16/3/14/main.tex}
	\item One urn contains two black balls (labelled B1 and B2) and one white ball. A
	second urn contains one black ball and two white balls (labelled W1 and W2).
	Suppose the following experiment is performed. One of the two urns is chosen
	at random. Next a ball is randomly chosen from the urn. Then a second ball is
	chosen at random from the same urn without replacing the first ball.
	
	\begin{enumerate}
	\item What is the probability that two black balls are chosen?
	
	\item What is the probability that two balls of opposite colour are chosen?
	\end{enumerate}
	\solution
	%\input{exemplar/11/16/3/12/main1.tex}
\end{enumerate}

	\item A bag contains $5$ red balls and some blue balls. If the probability of drawing a blue ball is double that if a red ball, determine the number of blue balls in the bag. 
		\\
\solution
		%\begin{enumerate}[label=\thesection.\arabic*,ref=\thesection.\theenumi]
	\item One card is drawn from a well-shuffled deck of 52 cards. Find the probability of getting
\begin{enumerate}
\item A king of red colour 
\item A face card 
\item A red face card
\item The jack of hearts
\item A spade
\item The queen of diamonds

\end{enumerate}
\solution
		%\input{ncert/10/15/1/14/main.tex}
	\item Five cards—the ten, jack, queen, king and ace of diamonds, are well-shuffled with their face downwards. One card is then picked up at random.
\begin{enumerate}
\item
What is the probability that the card is the queen? 
\item
If the queen is drawn and put aside, what is the probability that the second card picked up is (a) an ace? (b) a queen?\\
\end{enumerate}
\solution
		%\input{ncert/10/15/1/15/defs.tex}
	\item A bag contains $5$ red balls and some blue balls. If the probability of drawing a blue ball is double that if a red ball, determine the number of blue balls in the bag. 
		\\
\solution
		%\input{ncert/10/15/2/3/defs.tex}
	\item A card is selected from a pack of 52 cards.
 \begin{enumerate}[label=(\alph*)] 
                 \item How many points are there in the sample space?
                 \item Calculate the probability that the card is an ace of spades.
                 \item Calculate the probability that the card is (i) an ace and (ii) black card.
 \end{enumerate}
\solution
		%\input{ncert/11/16/3/4/main.tex}
\item Four cards are drawn from a well-shuffled deck of 52 cards. What is the probability of obtaining 3 diamonds and one spade.
\\
\solution
		%\input{ncert/11/16/4/2/defs.tex}
\item In a certain lottery 10,000 tickets are sold and ten equal prizes are awarded. What is the probability of not getting a prize if you buy (a) one ticket (b) two tickets (c) 10 tickets ?	
\\
\solution
		%\input{ncert/11/16/4/4/defs.tex}
		%
\item 
Out of 100 students, two sections of 40 and 60 are formed. If you and your friend are among the 100 students, what is the probability that
\begin{enumerate}
\item you both enter the same section?
\item you both enter the different sections?
\end{enumerate}
\solution
		%\input{ncert/11/16/4/5/defs.tex}
	\item 
The number lock of a suitcase has 4 wheels each labelled with ten digits i.e. from 0 to 9.The lock opens with a sequence of four digits with no repeats.What is the probability of a person getting the right sequence to open the suitcase.
\\
\solution
		%\input{ncert/11/16/4/10/defs.tex}
		%
\item 
Two cards are drawn at random and without replacement from a pack of 52 playing cards. Find the probability that both the cards are black.
\\
\solution
		%\input{ncert/12/13/2/2/defs.tex}
		\item A box of oranges is inspected by examining three randomly selected oranges drawn without replacement. If all the three oranges are good, the box is approved for sale, otherwise, it is rejected. Find the probability that a box containing 15 oranges out of which 12 are good and 3 are bad ones will be approved for sale.
		\label{ncert/12/13/2/3/defs.tex}
		\item Two balls are drawn at random with replacement from a box containing 10 black and 8 red balls. Find the probability that
		\label{ncert/12/13/2/12}
\begin{enumerate}
\item both balls are red.
\item first ball is black and second is red.
\item one of them is black and other is red.
\end{enumerate}

\item In a hostel, 60\% of the students read Hindi newspaper, 40\% read English newspaper and 20\% read both Hindi and English newspapers. A student is selected at random.
		\label{ncert/12/13/2/15}
\begin{enumerate}
\item Find the probability that she reads neither Hindi nor English newspapers.
\item If she reads Hindi newspaper, find the probability that she reads English newspaper.
\item If she reads English newspaper, find the probability that she reads Hindi newspaper.\\
\end{enumerate}
\item The probability of obtaining an even prime number on each die, when a pair of dice is rolled is 
\begin{enumerate}
    \item $0$ 
    
    \item $\frac{1}{3}$ 
    
    \item $\frac{1}{12}$ 
    
    \item $\frac{1}{36}$ 
\end{enumerate}
\solution
		%\input{ncert/12/13/2/17/defs.tex}
	\item A bag contains 4 red and 4 black balls, another bag contains 2 red and 6 black balls. One of the two bags is selected at random and a ball is drawn from the bag which is found to be red. Find the probability that the ball is drawn from the first bag.
\\
\solution
		%\input{ncert/12/13/3/2/main.tex}
  \item
  Cards with numbers 2 to 101 are placed in a box. A card is selected at random.Find the probability that the card has
\begin{enumerate}[label=(\roman*)]
	\item an even number 
	\item a square number
\end{enumerate}
\solution
%\input{exemplar/10/13/3/32/main.tex}
\item
The king, queen and jack of clubs are removed from a deck of 52 playing cards and then well shuffled. Now one card is drawn at random from the remaining cards.  Determine the probability that the card is
\begin{enumerate}[label=(\roman*)]
\item a club
\item 10 of hearts
\end{enumerate}
\solution
%\input{exemplar/10/13/3/29/main.tex}
\item A team of medical students doing their internship have to assist during surgeries
at a city hospital. The probabilities of surgeries rated as very complex, complex,
routine, simple or very simple are respectively, 0.15, 0.20, 0.31, 0.26, .08. Find
the probabilities that a particular surgery will be rated
\begin{enumerate}
	\item complex or very complex;
	\item neither very complex nor very simple;
	\item routine or complex
	\item routine or simple
\end{enumerate}
\solution
%\input{exemplar/11/16/3/8(1)/main.tex}
\item A card is selected from a pack of 52 cards.
\begin{enumerate}[label=(\alph*)]
    \item How many points are there in the sample space?
    \item Calculate the probability that the card is an ace of spades.
    \item Calculate the probability that the card is (i) an ace and (ii) black card.
\end{enumerate}
\solution
%\input{exemplar/11/16/3/4/main2.tex}
\item The probability that a non leap year selected at random will contain 53 sundays.
\\
\solution
%\input{exemplar/10/13/1/19/main.tex}
\item One of the four persons John, Rita, Aslam or Gurpreet will be promoted next
month. Consequently the sample space consists of four elementary outcomes
S = {John promoted, Rita promoted, Aslam promoted, Gurpreet promoted}
You are told that the chances of John’s promotion is same as that of Gurpreet,
Rita’s chances of promotion are twice as likely as Johns. Aslam’s chances are
four times that of John.
\begin{enumerate}
	\item Determine
	\begin{enumerate}
		\item P (John promoted)
		\item P (Rita promoted)
		\item P (Aslam promoted)
		\item P (Gurpreet promoted)
	\end{enumerate}
	\item If A = {John promoted or Gurpreet promoted}, find P (A).
\end{enumerate}
\solution
%\input{exemplar/11/16/3/10/main.tex}
\item A card is drawn from a deck of 52 cards. Find the probability of getting a king or a heart or a red card.\\
\solution
%\input{exemplar/11/16/3/15/main.tex}
\item The probability that a student will pass his examination is 0.73, the probability of
the student getting a compartment is 0.13, and the probability that the student will
either pass or get compartment is 0.96. State True or False.\\
\solution
%\input{exemplar/11/16/3/31/main.tex}
\item A card is selected from a pack of 52 cards\\
\begin{enumerate}[label=(\alph*)]
\item How many points are there in the sample space?
\item Calculate the probability that the cards is an ace of spades.
\item Calculate the probability that the card is (i) an ace (ii)black card.\\
\end{enumerate}
%\input{ncert/11/16/3/4_1/Prob_4.tex}
\item In a non-leap year, the probability of having 53 tuesdays or 53 wednesdays is\\
\solution
%\input{exemplar/11/16/3/18/main.tex}
\item There are 1000 sealed envelopes in a box, 10 of them contain a cash prize of
Rs 100 each, 100 of them contain a cash prize of Rs 50 each and 200 of them
contain a cash prize of Rs 10 each and rest do not contain any cash prize. If they
are well shuffled and an envelope is picked up out, what is the probability that it
contains no cash prize?\\
\solution
%\input{exemplar/10/13/3/34/main.tex}
\item 
A die is thrown and a card is selected at random from a deck of 52 playing cards. The probability of getting an even number on the die and a spade card.\\
\solution
%\input{exemplar/12/13/3/78/main.tex}
\item
If 4-digit numbers greater than 5,000 are randomly formed from the digits 0, 1, 3, 5, and 7, what is the probability of forming a number divisible by 5 when:
\begin{enumerate}
    \item The digits are repeated?
    \item The repetition of digits is not allowed?
\end{enumerate}
\solution
%\input{ncert/11/16/4/9/main.tex}
\item Consider the probability space $\brak{\Omega, \mathcal{G}, P}$ where $\Omega = [0,2]$ and $\mathcal{G} = \cbrak{\phi, \Omega, [0,1], (1,2]}$. Let $X$ and $Y$ be two functions on $\Omega$ defined as
\begin{align*}
    X(\omega) = 
    \begin{cases}
        1 & \text{if }\omega \in [0, 1]\\
        2 & \text{if }\omega \in (1, 2]
    \end{cases}
\end{align*}
and
\begin{align*}
    Y(\omega) = 
    \begin{cases}
        2 & \text{if }\omega \in [0, 1.5]\\
        3 & \text{if }\omega \in (1.5, 2].
    \end{cases}
\end{align*}
Then which one of the following statements is true?
\begin{enumerate}
    \item [(A)] $X$ is a random variable with respect to $\mathcal{G}$, but $Y$ is not a random variable with respect to $\mathcal{G}$.
    \item [(B)] $Y$ is a random variable with respect to $\mathcal{G}$, but $X$ is not a random variable with respect to $\mathcal{G}$.
    \item [(C)] Neither $X$ nor $Y$ is a random variable with respect to $\mathcal{G}$.
    \item [(D)] Both $X$ and $Y$ are random variables with respect to $\mathcal{G}$.
\end{enumerate} \hfill (GATE ST 2023)\\
\solution
%\input{gate/ST/2023/14/main.tex}
	\item  A die is loaded in such a way that each odd number is twice as likely to occur as
each even number. Find $P(G)$, where $G$ is the event that a number greater than
3 occurs on a single roll of the die.
\\
\solution
		%\input{exemplar/11/16/3/5/main.tex}
	\item All the jacks, queens and kings are removed from a deck of 52 playing cards. The remaining cards are well shuffled and then one card is drawn at random. Giving ace a value 1 similar value for other cards, find the probability that the card has a value 
		\begin{enumerate}
			\item 7
			\item greater than 7
			\item less than 7
		\end{enumerate}
		%\input{exemplar/10/13/3/30/main.tex}
  \item A Lot consists of 48 mobile phones of which 42 are good, 3 have only minor defects and 3 have major defects.Varnika will buy a phone if it is good but the trader will only buy a mobile if it has no major defects. One phone is selected at random from the lot. What is the probability that it is
\begin{enumerate}
	\item acceptable to Varnika?
            \item acceptable to the trader?
\end{enumerate}
\solution
	%\input{exemplar/10/13/3/40/main.tex}
 \item A student says that if you throw a die, it will show up 1 or not 1. Therefore, the probability of getting 1 and the probability of getting 'not 1' each is equal to $\frac{1}{2}$. Is this correct? Give reasons.\\
 \solution
        %\input{exemplar/10/13/2/9/main.tex}
   \item Four candidates A, B, C, D have ap-
plied for the assignment to coach a school cricket
team. If A is twice as likely to be selected as B, and
B and C are given about the same chance of being
selected, while C is twice as likely to be selected
as D, what are the probabilities that
\begin{enumerate}
\item C will be selected?
\item A will not be selected?
\end{enumerate}
	%\input{exemplar/11/16/3/9/main.tex}
 \item A bag contain 24 balls of which $x$ balls are red, $2x$ are white and $3x$ are blue. A ball is selected at random, What is the probability that it is
\begin{enumerate}[label=\alph*)]
\item not red ?
\item white ?
\end{enumerate}
%\input{exemplar/10/13/3/41/main.tex}
If the letters of the word ASSASSINATION are arranged at random. Find the Probability that
\begin{enumerate}[label=(\alph*)]
\item Four $S's$ come consecutively in the word
\item Two  $I's$ and two $N's$ come together
\item All $A's$ are not coming together
\item No two $A's$ are coming together
\end{enumerate}
%\input{exemplar/11/16/3/14/main.tex}
	\item One urn contains two black balls (labelled B1 and B2) and one white ball. A
	second urn contains one black ball and two white balls (labelled W1 and W2).
	Suppose the following experiment is performed. One of the two urns is chosen
	at random. Next a ball is randomly chosen from the urn. Then a second ball is
	chosen at random from the same urn without replacing the first ball.
	
	\begin{enumerate}
	\item What is the probability that two black balls are chosen?
	
	\item What is the probability that two balls of opposite colour are chosen?
	\end{enumerate}
	\solution
	%\input{exemplar/11/16/3/12/main1.tex}
\end{enumerate}

	\item A card is selected from a pack of 52 cards.
 \begin{enumerate}[label=(\alph*)] 
                 \item How many points are there in the sample space?
                 \item Calculate the probability that the card is an ace of spades.
                 \item Calculate the probability that the card is (i) an ace and (ii) black card.
 \end{enumerate}
\solution
		%\begin{table}[H]
	\centering
\begin{tabular}{|c|c|c|}
\hline
Random variable &Value &Definition\\ \hline
\multirow{3}{*}{X} &0 &Slips of Rs 1\\
&1 &Slips of Rs 5\\
&2 &Slips of Rs 13\\ \hline
\multirow{2}{*}{Y} &0 &Box A\\
&1 &Box B\\\hline
\end{tabular}
\caption{}
\label{tab:Distribution}
\end{table}
See \tabref{tab:Distribution}.
\begin{align}
p_{Y}\brak{k}= \begin{cases} 
      \frac{1}{3} & {k=0} \\
      \frac{2}{3 }& {k=1} 
   \end{cases}
   \\
p_{Y|X}\brak{0|0} = \frac{19}{25}\, 
p_{Y|X}\brak{0|1} = \frac{6}{25}\,
p_{Y|X}\brak{1|0} = \frac{45}{50}\,
p_{Y|X}\brak{1|2} = \frac{5}{50}
\end{align}
The desired probability is the probability that a slip drawn at random is marked other than Rs 1,
\begin{align}
&=1-p_X\brak{0}\\
&= p_X(1) + p_X(2)
\end{align}
Using Bayes theorem,
\begin{align}
&= p_Y\brak{0} \times \pr{Y=0 | X=1} + p_Y\brak{1} \times \pr{Y=1|X=2}\\
&=\frac{1}{3} \times \frac{6}{25} + \frac{2}{3} \times \frac{5}{50}\\
&=\frac{11}{75}
\end{align}

\newpage

%\tableofcontents

\bigskip

\renewcommand{\thefigure}{\theenumi}
\renewcommand{\thetable}{\theenumi}
%\renewcommand{\theequation}{\theenumi}

%\begin{abstract}
%%\boldmath
%In this letter, an algorithm for evaluating the exact analytical bit error rate  (BER)  for the piecewise linear (PL) combiner for  multiple relays is presented. Previous results were available only for upto three relays. The algorithm is unique in the sense that  the actual mathematical expressions, that are prohibitively large, need not be explicitly obtained. The diversity gain due to multiple relays is shown through plots of the analytical BER, well supported by simulations. 
%
%\end{abstract}
% IEEEtran.cls defaults to using nonbold math in the Abstract.
% This preserves the distinction between vectors and scalars. However,
% if the journal you are submitting to favors bold math in the abstract,
% then you can use LaTeX's standard command \boldmath at the very start
% of the abstract to achieve this. Many IEEE journals frown on math
% in the abstract anyway.

% Note that keywords are not normally used for peerreview papers.
%\begin{IEEEkeywords}
%Cooperative diversity, decode and forward, piecewise linear
%\end{IEEEkeywords}



% For peer review papers, you can put extra information on the cover
% page as needed:
% \ifCLASSOPTIONpeerreview
% \begin{center} \bfseries EDICS Category: 3-BBND \end{center}
% \fi
%
% For peerreview papers, this IEEEtran command inserts a page break and
% creates the second title. It will be ignored for other modes.
%\IEEEpeerreviewmaketitle




\item Four cards are drawn from a well-shuffled deck of 52 cards. What is the probability of obtaining 3 diamonds and one spade.
\\
\solution
		%\begin{enumerate}[label=\thesection.\arabic*,ref=\thesection.\theenumi]
	\item One card is drawn from a well-shuffled deck of 52 cards. Find the probability of getting
\begin{enumerate}
\item A king of red colour 
\item A face card 
\item A red face card
\item The jack of hearts
\item A spade
\item The queen of diamonds

\end{enumerate}
\solution
		%\input{ncert/10/15/1/14/main.tex}
	\item Five cards—the ten, jack, queen, king and ace of diamonds, are well-shuffled with their face downwards. One card is then picked up at random.
\begin{enumerate}
\item
What is the probability that the card is the queen? 
\item
If the queen is drawn and put aside, what is the probability that the second card picked up is (a) an ace? (b) a queen?\\
\end{enumerate}
\solution
		%\input{ncert/10/15/1/15/defs.tex}
	\item A bag contains $5$ red balls and some blue balls. If the probability of drawing a blue ball is double that if a red ball, determine the number of blue balls in the bag. 
		\\
\solution
		%\input{ncert/10/15/2/3/defs.tex}
	\item A card is selected from a pack of 52 cards.
 \begin{enumerate}[label=(\alph*)] 
                 \item How many points are there in the sample space?
                 \item Calculate the probability that the card is an ace of spades.
                 \item Calculate the probability that the card is (i) an ace and (ii) black card.
 \end{enumerate}
\solution
		%\input{ncert/11/16/3/4/main.tex}
\item Four cards are drawn from a well-shuffled deck of 52 cards. What is the probability of obtaining 3 diamonds and one spade.
\\
\solution
		%\input{ncert/11/16/4/2/defs.tex}
\item In a certain lottery 10,000 tickets are sold and ten equal prizes are awarded. What is the probability of not getting a prize if you buy (a) one ticket (b) two tickets (c) 10 tickets ?	
\\
\solution
		%\input{ncert/11/16/4/4/defs.tex}
		%
\item 
Out of 100 students, two sections of 40 and 60 are formed. If you and your friend are among the 100 students, what is the probability that
\begin{enumerate}
\item you both enter the same section?
\item you both enter the different sections?
\end{enumerate}
\solution
		%\input{ncert/11/16/4/5/defs.tex}
	\item 
The number lock of a suitcase has 4 wheels each labelled with ten digits i.e. from 0 to 9.The lock opens with a sequence of four digits with no repeats.What is the probability of a person getting the right sequence to open the suitcase.
\\
\solution
		%\input{ncert/11/16/4/10/defs.tex}
		%
\item 
Two cards are drawn at random and without replacement from a pack of 52 playing cards. Find the probability that both the cards are black.
\\
\solution
		%\input{ncert/12/13/2/2/defs.tex}
		\item A box of oranges is inspected by examining three randomly selected oranges drawn without replacement. If all the three oranges are good, the box is approved for sale, otherwise, it is rejected. Find the probability that a box containing 15 oranges out of which 12 are good and 3 are bad ones will be approved for sale.
		\label{ncert/12/13/2/3/defs.tex}
		\item Two balls are drawn at random with replacement from a box containing 10 black and 8 red balls. Find the probability that
		\label{ncert/12/13/2/12}
\begin{enumerate}
\item both balls are red.
\item first ball is black and second is red.
\item one of them is black and other is red.
\end{enumerate}

\item In a hostel, 60\% of the students read Hindi newspaper, 40\% read English newspaper and 20\% read both Hindi and English newspapers. A student is selected at random.
		\label{ncert/12/13/2/15}
\begin{enumerate}
\item Find the probability that she reads neither Hindi nor English newspapers.
\item If she reads Hindi newspaper, find the probability that she reads English newspaper.
\item If she reads English newspaper, find the probability that she reads Hindi newspaper.\\
\end{enumerate}
\item The probability of obtaining an even prime number on each die, when a pair of dice is rolled is 
\begin{enumerate}
    \item $0$ 
    
    \item $\frac{1}{3}$ 
    
    \item $\frac{1}{12}$ 
    
    \item $\frac{1}{36}$ 
\end{enumerate}
\solution
		%\input{ncert/12/13/2/17/defs.tex}
	\item A bag contains 4 red and 4 black balls, another bag contains 2 red and 6 black balls. One of the two bags is selected at random and a ball is drawn from the bag which is found to be red. Find the probability that the ball is drawn from the first bag.
\\
\solution
		%\input{ncert/12/13/3/2/main.tex}
  \item
  Cards with numbers 2 to 101 are placed in a box. A card is selected at random.Find the probability that the card has
\begin{enumerate}[label=(\roman*)]
	\item an even number 
	\item a square number
\end{enumerate}
\solution
%\input{exemplar/10/13/3/32/main.tex}
\item
The king, queen and jack of clubs are removed from a deck of 52 playing cards and then well shuffled. Now one card is drawn at random from the remaining cards.  Determine the probability that the card is
\begin{enumerate}[label=(\roman*)]
\item a club
\item 10 of hearts
\end{enumerate}
\solution
%\input{exemplar/10/13/3/29/main.tex}
\item A team of medical students doing their internship have to assist during surgeries
at a city hospital. The probabilities of surgeries rated as very complex, complex,
routine, simple or very simple are respectively, 0.15, 0.20, 0.31, 0.26, .08. Find
the probabilities that a particular surgery will be rated
\begin{enumerate}
	\item complex or very complex;
	\item neither very complex nor very simple;
	\item routine or complex
	\item routine or simple
\end{enumerate}
\solution
%\input{exemplar/11/16/3/8(1)/main.tex}
\item A card is selected from a pack of 52 cards.
\begin{enumerate}[label=(\alph*)]
    \item How many points are there in the sample space?
    \item Calculate the probability that the card is an ace of spades.
    \item Calculate the probability that the card is (i) an ace and (ii) black card.
\end{enumerate}
\solution
%\input{exemplar/11/16/3/4/main2.tex}
\item The probability that a non leap year selected at random will contain 53 sundays.
\\
\solution
%\input{exemplar/10/13/1/19/main.tex}
\item One of the four persons John, Rita, Aslam or Gurpreet will be promoted next
month. Consequently the sample space consists of four elementary outcomes
S = {John promoted, Rita promoted, Aslam promoted, Gurpreet promoted}
You are told that the chances of John’s promotion is same as that of Gurpreet,
Rita’s chances of promotion are twice as likely as Johns. Aslam’s chances are
four times that of John.
\begin{enumerate}
	\item Determine
	\begin{enumerate}
		\item P (John promoted)
		\item P (Rita promoted)
		\item P (Aslam promoted)
		\item P (Gurpreet promoted)
	\end{enumerate}
	\item If A = {John promoted or Gurpreet promoted}, find P (A).
\end{enumerate}
\solution
%\input{exemplar/11/16/3/10/main.tex}
\item A card is drawn from a deck of 52 cards. Find the probability of getting a king or a heart or a red card.\\
\solution
%\input{exemplar/11/16/3/15/main.tex}
\item The probability that a student will pass his examination is 0.73, the probability of
the student getting a compartment is 0.13, and the probability that the student will
either pass or get compartment is 0.96. State True or False.\\
\solution
%\input{exemplar/11/16/3/31/main.tex}
\item A card is selected from a pack of 52 cards\\
\begin{enumerate}[label=(\alph*)]
\item How many points are there in the sample space?
\item Calculate the probability that the cards is an ace of spades.
\item Calculate the probability that the card is (i) an ace (ii)black card.\\
\end{enumerate}
%\input{ncert/11/16/3/4_1/Prob_4.tex}
\item In a non-leap year, the probability of having 53 tuesdays or 53 wednesdays is\\
\solution
%\input{exemplar/11/16/3/18/main.tex}
\item There are 1000 sealed envelopes in a box, 10 of them contain a cash prize of
Rs 100 each, 100 of them contain a cash prize of Rs 50 each and 200 of them
contain a cash prize of Rs 10 each and rest do not contain any cash prize. If they
are well shuffled and an envelope is picked up out, what is the probability that it
contains no cash prize?\\
\solution
%\input{exemplar/10/13/3/34/main.tex}
\item 
A die is thrown and a card is selected at random from a deck of 52 playing cards. The probability of getting an even number on the die and a spade card.\\
\solution
%\input{exemplar/12/13/3/78/main.tex}
\item
If 4-digit numbers greater than 5,000 are randomly formed from the digits 0, 1, 3, 5, and 7, what is the probability of forming a number divisible by 5 when:
\begin{enumerate}
    \item The digits are repeated?
    \item The repetition of digits is not allowed?
\end{enumerate}
\solution
%\input{ncert/11/16/4/9/main.tex}
\item Consider the probability space $\brak{\Omega, \mathcal{G}, P}$ where $\Omega = [0,2]$ and $\mathcal{G} = \cbrak{\phi, \Omega, [0,1], (1,2]}$. Let $X$ and $Y$ be two functions on $\Omega$ defined as
\begin{align*}
    X(\omega) = 
    \begin{cases}
        1 & \text{if }\omega \in [0, 1]\\
        2 & \text{if }\omega \in (1, 2]
    \end{cases}
\end{align*}
and
\begin{align*}
    Y(\omega) = 
    \begin{cases}
        2 & \text{if }\omega \in [0, 1.5]\\
        3 & \text{if }\omega \in (1.5, 2].
    \end{cases}
\end{align*}
Then which one of the following statements is true?
\begin{enumerate}
    \item [(A)] $X$ is a random variable with respect to $\mathcal{G}$, but $Y$ is not a random variable with respect to $\mathcal{G}$.
    \item [(B)] $Y$ is a random variable with respect to $\mathcal{G}$, but $X$ is not a random variable with respect to $\mathcal{G}$.
    \item [(C)] Neither $X$ nor $Y$ is a random variable with respect to $\mathcal{G}$.
    \item [(D)] Both $X$ and $Y$ are random variables with respect to $\mathcal{G}$.
\end{enumerate} \hfill (GATE ST 2023)\\
\solution
%\input{gate/ST/2023/14/main.tex}
	\item  A die is loaded in such a way that each odd number is twice as likely to occur as
each even number. Find $P(G)$, where $G$ is the event that a number greater than
3 occurs on a single roll of the die.
\\
\solution
		%\input{exemplar/11/16/3/5/main.tex}
	\item All the jacks, queens and kings are removed from a deck of 52 playing cards. The remaining cards are well shuffled and then one card is drawn at random. Giving ace a value 1 similar value for other cards, find the probability that the card has a value 
		\begin{enumerate}
			\item 7
			\item greater than 7
			\item less than 7
		\end{enumerate}
		%\input{exemplar/10/13/3/30/main.tex}
  \item A Lot consists of 48 mobile phones of which 42 are good, 3 have only minor defects and 3 have major defects.Varnika will buy a phone if it is good but the trader will only buy a mobile if it has no major defects. One phone is selected at random from the lot. What is the probability that it is
\begin{enumerate}
	\item acceptable to Varnika?
            \item acceptable to the trader?
\end{enumerate}
\solution
	%\input{exemplar/10/13/3/40/main.tex}
 \item A student says that if you throw a die, it will show up 1 or not 1. Therefore, the probability of getting 1 and the probability of getting 'not 1' each is equal to $\frac{1}{2}$. Is this correct? Give reasons.\\
 \solution
        %\input{exemplar/10/13/2/9/main.tex}
   \item Four candidates A, B, C, D have ap-
plied for the assignment to coach a school cricket
team. If A is twice as likely to be selected as B, and
B and C are given about the same chance of being
selected, while C is twice as likely to be selected
as D, what are the probabilities that
\begin{enumerate}
\item C will be selected?
\item A will not be selected?
\end{enumerate}
	%\input{exemplar/11/16/3/9/main.tex}
 \item A bag contain 24 balls of which $x$ balls are red, $2x$ are white and $3x$ are blue. A ball is selected at random, What is the probability that it is
\begin{enumerate}[label=\alph*)]
\item not red ?
\item white ?
\end{enumerate}
%\input{exemplar/10/13/3/41/main.tex}
If the letters of the word ASSASSINATION are arranged at random. Find the Probability that
\begin{enumerate}[label=(\alph*)]
\item Four $S's$ come consecutively in the word
\item Two  $I's$ and two $N's$ come together
\item All $A's$ are not coming together
\item No two $A's$ are coming together
\end{enumerate}
%\input{exemplar/11/16/3/14/main.tex}
	\item One urn contains two black balls (labelled B1 and B2) and one white ball. A
	second urn contains one black ball and two white balls (labelled W1 and W2).
	Suppose the following experiment is performed. One of the two urns is chosen
	at random. Next a ball is randomly chosen from the urn. Then a second ball is
	chosen at random from the same urn without replacing the first ball.
	
	\begin{enumerate}
	\item What is the probability that two black balls are chosen?
	
	\item What is the probability that two balls of opposite colour are chosen?
	\end{enumerate}
	\solution
	%\input{exemplar/11/16/3/12/main1.tex}
\end{enumerate}

\item In a certain lottery 10,000 tickets are sold and ten equal prizes are awarded. What is the probability of not getting a prize if you buy (a) one ticket (b) two tickets (c) 10 tickets ?	
\\
\solution
		%\begin{enumerate}[label=\thesection.\arabic*,ref=\thesection.\theenumi]
	\item One card is drawn from a well-shuffled deck of 52 cards. Find the probability of getting
\begin{enumerate}
\item A king of red colour 
\item A face card 
\item A red face card
\item The jack of hearts
\item A spade
\item The queen of diamonds

\end{enumerate}
\solution
		%\input{ncert/10/15/1/14/main.tex}
	\item Five cards—the ten, jack, queen, king and ace of diamonds, are well-shuffled with their face downwards. One card is then picked up at random.
\begin{enumerate}
\item
What is the probability that the card is the queen? 
\item
If the queen is drawn and put aside, what is the probability that the second card picked up is (a) an ace? (b) a queen?\\
\end{enumerate}
\solution
		%\input{ncert/10/15/1/15/defs.tex}
	\item A bag contains $5$ red balls and some blue balls. If the probability of drawing a blue ball is double that if a red ball, determine the number of blue balls in the bag. 
		\\
\solution
		%\input{ncert/10/15/2/3/defs.tex}
	\item A card is selected from a pack of 52 cards.
 \begin{enumerate}[label=(\alph*)] 
                 \item How many points are there in the sample space?
                 \item Calculate the probability that the card is an ace of spades.
                 \item Calculate the probability that the card is (i) an ace and (ii) black card.
 \end{enumerate}
\solution
		%\input{ncert/11/16/3/4/main.tex}
\item Four cards are drawn from a well-shuffled deck of 52 cards. What is the probability of obtaining 3 diamonds and one spade.
\\
\solution
		%\input{ncert/11/16/4/2/defs.tex}
\item In a certain lottery 10,000 tickets are sold and ten equal prizes are awarded. What is the probability of not getting a prize if you buy (a) one ticket (b) two tickets (c) 10 tickets ?	
\\
\solution
		%\input{ncert/11/16/4/4/defs.tex}
		%
\item 
Out of 100 students, two sections of 40 and 60 are formed. If you and your friend are among the 100 students, what is the probability that
\begin{enumerate}
\item you both enter the same section?
\item you both enter the different sections?
\end{enumerate}
\solution
		%\input{ncert/11/16/4/5/defs.tex}
	\item 
The number lock of a suitcase has 4 wheels each labelled with ten digits i.e. from 0 to 9.The lock opens with a sequence of four digits with no repeats.What is the probability of a person getting the right sequence to open the suitcase.
\\
\solution
		%\input{ncert/11/16/4/10/defs.tex}
		%
\item 
Two cards are drawn at random and without replacement from a pack of 52 playing cards. Find the probability that both the cards are black.
\\
\solution
		%\input{ncert/12/13/2/2/defs.tex}
		\item A box of oranges is inspected by examining three randomly selected oranges drawn without replacement. If all the three oranges are good, the box is approved for sale, otherwise, it is rejected. Find the probability that a box containing 15 oranges out of which 12 are good and 3 are bad ones will be approved for sale.
		\label{ncert/12/13/2/3/defs.tex}
		\item Two balls are drawn at random with replacement from a box containing 10 black and 8 red balls. Find the probability that
		\label{ncert/12/13/2/12}
\begin{enumerate}
\item both balls are red.
\item first ball is black and second is red.
\item one of them is black and other is red.
\end{enumerate}

\item In a hostel, 60\% of the students read Hindi newspaper, 40\% read English newspaper and 20\% read both Hindi and English newspapers. A student is selected at random.
		\label{ncert/12/13/2/15}
\begin{enumerate}
\item Find the probability that she reads neither Hindi nor English newspapers.
\item If she reads Hindi newspaper, find the probability that she reads English newspaper.
\item If she reads English newspaper, find the probability that she reads Hindi newspaper.\\
\end{enumerate}
\item The probability of obtaining an even prime number on each die, when a pair of dice is rolled is 
\begin{enumerate}
    \item $0$ 
    
    \item $\frac{1}{3}$ 
    
    \item $\frac{1}{12}$ 
    
    \item $\frac{1}{36}$ 
\end{enumerate}
\solution
		%\input{ncert/12/13/2/17/defs.tex}
	\item A bag contains 4 red and 4 black balls, another bag contains 2 red and 6 black balls. One of the two bags is selected at random and a ball is drawn from the bag which is found to be red. Find the probability that the ball is drawn from the first bag.
\\
\solution
		%\input{ncert/12/13/3/2/main.tex}
  \item
  Cards with numbers 2 to 101 are placed in a box. A card is selected at random.Find the probability that the card has
\begin{enumerate}[label=(\roman*)]
	\item an even number 
	\item a square number
\end{enumerate}
\solution
%\input{exemplar/10/13/3/32/main.tex}
\item
The king, queen and jack of clubs are removed from a deck of 52 playing cards and then well shuffled. Now one card is drawn at random from the remaining cards.  Determine the probability that the card is
\begin{enumerate}[label=(\roman*)]
\item a club
\item 10 of hearts
\end{enumerate}
\solution
%\input{exemplar/10/13/3/29/main.tex}
\item A team of medical students doing their internship have to assist during surgeries
at a city hospital. The probabilities of surgeries rated as very complex, complex,
routine, simple or very simple are respectively, 0.15, 0.20, 0.31, 0.26, .08. Find
the probabilities that a particular surgery will be rated
\begin{enumerate}
	\item complex or very complex;
	\item neither very complex nor very simple;
	\item routine or complex
	\item routine or simple
\end{enumerate}
\solution
%\input{exemplar/11/16/3/8(1)/main.tex}
\item A card is selected from a pack of 52 cards.
\begin{enumerate}[label=(\alph*)]
    \item How many points are there in the sample space?
    \item Calculate the probability that the card is an ace of spades.
    \item Calculate the probability that the card is (i) an ace and (ii) black card.
\end{enumerate}
\solution
%\input{exemplar/11/16/3/4/main2.tex}
\item The probability that a non leap year selected at random will contain 53 sundays.
\\
\solution
%\input{exemplar/10/13/1/19/main.tex}
\item One of the four persons John, Rita, Aslam or Gurpreet will be promoted next
month. Consequently the sample space consists of four elementary outcomes
S = {John promoted, Rita promoted, Aslam promoted, Gurpreet promoted}
You are told that the chances of John’s promotion is same as that of Gurpreet,
Rita’s chances of promotion are twice as likely as Johns. Aslam’s chances are
four times that of John.
\begin{enumerate}
	\item Determine
	\begin{enumerate}
		\item P (John promoted)
		\item P (Rita promoted)
		\item P (Aslam promoted)
		\item P (Gurpreet promoted)
	\end{enumerate}
	\item If A = {John promoted or Gurpreet promoted}, find P (A).
\end{enumerate}
\solution
%\input{exemplar/11/16/3/10/main.tex}
\item A card is drawn from a deck of 52 cards. Find the probability of getting a king or a heart or a red card.\\
\solution
%\input{exemplar/11/16/3/15/main.tex}
\item The probability that a student will pass his examination is 0.73, the probability of
the student getting a compartment is 0.13, and the probability that the student will
either pass or get compartment is 0.96. State True or False.\\
\solution
%\input{exemplar/11/16/3/31/main.tex}
\item A card is selected from a pack of 52 cards\\
\begin{enumerate}[label=(\alph*)]
\item How many points are there in the sample space?
\item Calculate the probability that the cards is an ace of spades.
\item Calculate the probability that the card is (i) an ace (ii)black card.\\
\end{enumerate}
%\input{ncert/11/16/3/4_1/Prob_4.tex}
\item In a non-leap year, the probability of having 53 tuesdays or 53 wednesdays is\\
\solution
%\input{exemplar/11/16/3/18/main.tex}
\item There are 1000 sealed envelopes in a box, 10 of them contain a cash prize of
Rs 100 each, 100 of them contain a cash prize of Rs 50 each and 200 of them
contain a cash prize of Rs 10 each and rest do not contain any cash prize. If they
are well shuffled and an envelope is picked up out, what is the probability that it
contains no cash prize?\\
\solution
%\input{exemplar/10/13/3/34/main.tex}
\item 
A die is thrown and a card is selected at random from a deck of 52 playing cards. The probability of getting an even number on the die and a spade card.\\
\solution
%\input{exemplar/12/13/3/78/main.tex}
\item
If 4-digit numbers greater than 5,000 are randomly formed from the digits 0, 1, 3, 5, and 7, what is the probability of forming a number divisible by 5 when:
\begin{enumerate}
    \item The digits are repeated?
    \item The repetition of digits is not allowed?
\end{enumerate}
\solution
%\input{ncert/11/16/4/9/main.tex}
\item Consider the probability space $\brak{\Omega, \mathcal{G}, P}$ where $\Omega = [0,2]$ and $\mathcal{G} = \cbrak{\phi, \Omega, [0,1], (1,2]}$. Let $X$ and $Y$ be two functions on $\Omega$ defined as
\begin{align*}
    X(\omega) = 
    \begin{cases}
        1 & \text{if }\omega \in [0, 1]\\
        2 & \text{if }\omega \in (1, 2]
    \end{cases}
\end{align*}
and
\begin{align*}
    Y(\omega) = 
    \begin{cases}
        2 & \text{if }\omega \in [0, 1.5]\\
        3 & \text{if }\omega \in (1.5, 2].
    \end{cases}
\end{align*}
Then which one of the following statements is true?
\begin{enumerate}
    \item [(A)] $X$ is a random variable with respect to $\mathcal{G}$, but $Y$ is not a random variable with respect to $\mathcal{G}$.
    \item [(B)] $Y$ is a random variable with respect to $\mathcal{G}$, but $X$ is not a random variable with respect to $\mathcal{G}$.
    \item [(C)] Neither $X$ nor $Y$ is a random variable with respect to $\mathcal{G}$.
    \item [(D)] Both $X$ and $Y$ are random variables with respect to $\mathcal{G}$.
\end{enumerate} \hfill (GATE ST 2023)\\
\solution
%\input{gate/ST/2023/14/main.tex}
	\item  A die is loaded in such a way that each odd number is twice as likely to occur as
each even number. Find $P(G)$, where $G$ is the event that a number greater than
3 occurs on a single roll of the die.
\\
\solution
		%\input{exemplar/11/16/3/5/main.tex}
	\item All the jacks, queens and kings are removed from a deck of 52 playing cards. The remaining cards are well shuffled and then one card is drawn at random. Giving ace a value 1 similar value for other cards, find the probability that the card has a value 
		\begin{enumerate}
			\item 7
			\item greater than 7
			\item less than 7
		\end{enumerate}
		%\input{exemplar/10/13/3/30/main.tex}
  \item A Lot consists of 48 mobile phones of which 42 are good, 3 have only minor defects and 3 have major defects.Varnika will buy a phone if it is good but the trader will only buy a mobile if it has no major defects. One phone is selected at random from the lot. What is the probability that it is
\begin{enumerate}
	\item acceptable to Varnika?
            \item acceptable to the trader?
\end{enumerate}
\solution
	%\input{exemplar/10/13/3/40/main.tex}
 \item A student says that if you throw a die, it will show up 1 or not 1. Therefore, the probability of getting 1 and the probability of getting 'not 1' each is equal to $\frac{1}{2}$. Is this correct? Give reasons.\\
 \solution
        %\input{exemplar/10/13/2/9/main.tex}
   \item Four candidates A, B, C, D have ap-
plied for the assignment to coach a school cricket
team. If A is twice as likely to be selected as B, and
B and C are given about the same chance of being
selected, while C is twice as likely to be selected
as D, what are the probabilities that
\begin{enumerate}
\item C will be selected?
\item A will not be selected?
\end{enumerate}
	%\input{exemplar/11/16/3/9/main.tex}
 \item A bag contain 24 balls of which $x$ balls are red, $2x$ are white and $3x$ are blue. A ball is selected at random, What is the probability that it is
\begin{enumerate}[label=\alph*)]
\item not red ?
\item white ?
\end{enumerate}
%\input{exemplar/10/13/3/41/main.tex}
If the letters of the word ASSASSINATION are arranged at random. Find the Probability that
\begin{enumerate}[label=(\alph*)]
\item Four $S's$ come consecutively in the word
\item Two  $I's$ and two $N's$ come together
\item All $A's$ are not coming together
\item No two $A's$ are coming together
\end{enumerate}
%\input{exemplar/11/16/3/14/main.tex}
	\item One urn contains two black balls (labelled B1 and B2) and one white ball. A
	second urn contains one black ball and two white balls (labelled W1 and W2).
	Suppose the following experiment is performed. One of the two urns is chosen
	at random. Next a ball is randomly chosen from the urn. Then a second ball is
	chosen at random from the same urn without replacing the first ball.
	
	\begin{enumerate}
	\item What is the probability that two black balls are chosen?
	
	\item What is the probability that two balls of opposite colour are chosen?
	\end{enumerate}
	\solution
	%\input{exemplar/11/16/3/12/main1.tex}
\end{enumerate}

		%
\item 
Out of 100 students, two sections of 40 and 60 are formed. If you and your friend are among the 100 students, what is the probability that
\begin{enumerate}
\item you both enter the same section?
\item you both enter the different sections?
\end{enumerate}
\solution
		%\begin{enumerate}[label=\thesection.\arabic*,ref=\thesection.\theenumi]
	\item One card is drawn from a well-shuffled deck of 52 cards. Find the probability of getting
\begin{enumerate}
\item A king of red colour 
\item A face card 
\item A red face card
\item The jack of hearts
\item A spade
\item The queen of diamonds

\end{enumerate}
\solution
		%\input{ncert/10/15/1/14/main.tex}
	\item Five cards—the ten, jack, queen, king and ace of diamonds, are well-shuffled with their face downwards. One card is then picked up at random.
\begin{enumerate}
\item
What is the probability that the card is the queen? 
\item
If the queen is drawn and put aside, what is the probability that the second card picked up is (a) an ace? (b) a queen?\\
\end{enumerate}
\solution
		%\input{ncert/10/15/1/15/defs.tex}
	\item A bag contains $5$ red balls and some blue balls. If the probability of drawing a blue ball is double that if a red ball, determine the number of blue balls in the bag. 
		\\
\solution
		%\input{ncert/10/15/2/3/defs.tex}
	\item A card is selected from a pack of 52 cards.
 \begin{enumerate}[label=(\alph*)] 
                 \item How many points are there in the sample space?
                 \item Calculate the probability that the card is an ace of spades.
                 \item Calculate the probability that the card is (i) an ace and (ii) black card.
 \end{enumerate}
\solution
		%\input{ncert/11/16/3/4/main.tex}
\item Four cards are drawn from a well-shuffled deck of 52 cards. What is the probability of obtaining 3 diamonds and one spade.
\\
\solution
		%\input{ncert/11/16/4/2/defs.tex}
\item In a certain lottery 10,000 tickets are sold and ten equal prizes are awarded. What is the probability of not getting a prize if you buy (a) one ticket (b) two tickets (c) 10 tickets ?	
\\
\solution
		%\input{ncert/11/16/4/4/defs.tex}
		%
\item 
Out of 100 students, two sections of 40 and 60 are formed. If you and your friend are among the 100 students, what is the probability that
\begin{enumerate}
\item you both enter the same section?
\item you both enter the different sections?
\end{enumerate}
\solution
		%\input{ncert/11/16/4/5/defs.tex}
	\item 
The number lock of a suitcase has 4 wheels each labelled with ten digits i.e. from 0 to 9.The lock opens with a sequence of four digits with no repeats.What is the probability of a person getting the right sequence to open the suitcase.
\\
\solution
		%\input{ncert/11/16/4/10/defs.tex}
		%
\item 
Two cards are drawn at random and without replacement from a pack of 52 playing cards. Find the probability that both the cards are black.
\\
\solution
		%\input{ncert/12/13/2/2/defs.tex}
		\item A box of oranges is inspected by examining three randomly selected oranges drawn without replacement. If all the three oranges are good, the box is approved for sale, otherwise, it is rejected. Find the probability that a box containing 15 oranges out of which 12 are good and 3 are bad ones will be approved for sale.
		\label{ncert/12/13/2/3/defs.tex}
		\item Two balls are drawn at random with replacement from a box containing 10 black and 8 red balls. Find the probability that
		\label{ncert/12/13/2/12}
\begin{enumerate}
\item both balls are red.
\item first ball is black and second is red.
\item one of them is black and other is red.
\end{enumerate}

\item In a hostel, 60\% of the students read Hindi newspaper, 40\% read English newspaper and 20\% read both Hindi and English newspapers. A student is selected at random.
		\label{ncert/12/13/2/15}
\begin{enumerate}
\item Find the probability that she reads neither Hindi nor English newspapers.
\item If she reads Hindi newspaper, find the probability that she reads English newspaper.
\item If she reads English newspaper, find the probability that she reads Hindi newspaper.\\
\end{enumerate}
\item The probability of obtaining an even prime number on each die, when a pair of dice is rolled is 
\begin{enumerate}
    \item $0$ 
    
    \item $\frac{1}{3}$ 
    
    \item $\frac{1}{12}$ 
    
    \item $\frac{1}{36}$ 
\end{enumerate}
\solution
		%\input{ncert/12/13/2/17/defs.tex}
	\item A bag contains 4 red and 4 black balls, another bag contains 2 red and 6 black balls. One of the two bags is selected at random and a ball is drawn from the bag which is found to be red. Find the probability that the ball is drawn from the first bag.
\\
\solution
		%\input{ncert/12/13/3/2/main.tex}
  \item
  Cards with numbers 2 to 101 are placed in a box. A card is selected at random.Find the probability that the card has
\begin{enumerate}[label=(\roman*)]
	\item an even number 
	\item a square number
\end{enumerate}
\solution
%\input{exemplar/10/13/3/32/main.tex}
\item
The king, queen and jack of clubs are removed from a deck of 52 playing cards and then well shuffled. Now one card is drawn at random from the remaining cards.  Determine the probability that the card is
\begin{enumerate}[label=(\roman*)]
\item a club
\item 10 of hearts
\end{enumerate}
\solution
%\input{exemplar/10/13/3/29/main.tex}
\item A team of medical students doing their internship have to assist during surgeries
at a city hospital. The probabilities of surgeries rated as very complex, complex,
routine, simple or very simple are respectively, 0.15, 0.20, 0.31, 0.26, .08. Find
the probabilities that a particular surgery will be rated
\begin{enumerate}
	\item complex or very complex;
	\item neither very complex nor very simple;
	\item routine or complex
	\item routine or simple
\end{enumerate}
\solution
%\input{exemplar/11/16/3/8(1)/main.tex}
\item A card is selected from a pack of 52 cards.
\begin{enumerate}[label=(\alph*)]
    \item How many points are there in the sample space?
    \item Calculate the probability that the card is an ace of spades.
    \item Calculate the probability that the card is (i) an ace and (ii) black card.
\end{enumerate}
\solution
%\input{exemplar/11/16/3/4/main2.tex}
\item The probability that a non leap year selected at random will contain 53 sundays.
\\
\solution
%\input{exemplar/10/13/1/19/main.tex}
\item One of the four persons John, Rita, Aslam or Gurpreet will be promoted next
month. Consequently the sample space consists of four elementary outcomes
S = {John promoted, Rita promoted, Aslam promoted, Gurpreet promoted}
You are told that the chances of John’s promotion is same as that of Gurpreet,
Rita’s chances of promotion are twice as likely as Johns. Aslam’s chances are
four times that of John.
\begin{enumerate}
	\item Determine
	\begin{enumerate}
		\item P (John promoted)
		\item P (Rita promoted)
		\item P (Aslam promoted)
		\item P (Gurpreet promoted)
	\end{enumerate}
	\item If A = {John promoted or Gurpreet promoted}, find P (A).
\end{enumerate}
\solution
%\input{exemplar/11/16/3/10/main.tex}
\item A card is drawn from a deck of 52 cards. Find the probability of getting a king or a heart or a red card.\\
\solution
%\input{exemplar/11/16/3/15/main.tex}
\item The probability that a student will pass his examination is 0.73, the probability of
the student getting a compartment is 0.13, and the probability that the student will
either pass or get compartment is 0.96. State True or False.\\
\solution
%\input{exemplar/11/16/3/31/main.tex}
\item A card is selected from a pack of 52 cards\\
\begin{enumerate}[label=(\alph*)]
\item How many points are there in the sample space?
\item Calculate the probability that the cards is an ace of spades.
\item Calculate the probability that the card is (i) an ace (ii)black card.\\
\end{enumerate}
%\input{ncert/11/16/3/4_1/Prob_4.tex}
\item In a non-leap year, the probability of having 53 tuesdays or 53 wednesdays is\\
\solution
%\input{exemplar/11/16/3/18/main.tex}
\item There are 1000 sealed envelopes in a box, 10 of them contain a cash prize of
Rs 100 each, 100 of them contain a cash prize of Rs 50 each and 200 of them
contain a cash prize of Rs 10 each and rest do not contain any cash prize. If they
are well shuffled and an envelope is picked up out, what is the probability that it
contains no cash prize?\\
\solution
%\input{exemplar/10/13/3/34/main.tex}
\item 
A die is thrown and a card is selected at random from a deck of 52 playing cards. The probability of getting an even number on the die and a spade card.\\
\solution
%\input{exemplar/12/13/3/78/main.tex}
\item
If 4-digit numbers greater than 5,000 are randomly formed from the digits 0, 1, 3, 5, and 7, what is the probability of forming a number divisible by 5 when:
\begin{enumerate}
    \item The digits are repeated?
    \item The repetition of digits is not allowed?
\end{enumerate}
\solution
%\input{ncert/11/16/4/9/main.tex}
\item Consider the probability space $\brak{\Omega, \mathcal{G}, P}$ where $\Omega = [0,2]$ and $\mathcal{G} = \cbrak{\phi, \Omega, [0,1], (1,2]}$. Let $X$ and $Y$ be two functions on $\Omega$ defined as
\begin{align*}
    X(\omega) = 
    \begin{cases}
        1 & \text{if }\omega \in [0, 1]\\
        2 & \text{if }\omega \in (1, 2]
    \end{cases}
\end{align*}
and
\begin{align*}
    Y(\omega) = 
    \begin{cases}
        2 & \text{if }\omega \in [0, 1.5]\\
        3 & \text{if }\omega \in (1.5, 2].
    \end{cases}
\end{align*}
Then which one of the following statements is true?
\begin{enumerate}
    \item [(A)] $X$ is a random variable with respect to $\mathcal{G}$, but $Y$ is not a random variable with respect to $\mathcal{G}$.
    \item [(B)] $Y$ is a random variable with respect to $\mathcal{G}$, but $X$ is not a random variable with respect to $\mathcal{G}$.
    \item [(C)] Neither $X$ nor $Y$ is a random variable with respect to $\mathcal{G}$.
    \item [(D)] Both $X$ and $Y$ are random variables with respect to $\mathcal{G}$.
\end{enumerate} \hfill (GATE ST 2023)\\
\solution
%\input{gate/ST/2023/14/main.tex}
	\item  A die is loaded in such a way that each odd number is twice as likely to occur as
each even number. Find $P(G)$, where $G$ is the event that a number greater than
3 occurs on a single roll of the die.
\\
\solution
		%\input{exemplar/11/16/3/5/main.tex}
	\item All the jacks, queens and kings are removed from a deck of 52 playing cards. The remaining cards are well shuffled and then one card is drawn at random. Giving ace a value 1 similar value for other cards, find the probability that the card has a value 
		\begin{enumerate}
			\item 7
			\item greater than 7
			\item less than 7
		\end{enumerate}
		%\input{exemplar/10/13/3/30/main.tex}
  \item A Lot consists of 48 mobile phones of which 42 are good, 3 have only minor defects and 3 have major defects.Varnika will buy a phone if it is good but the trader will only buy a mobile if it has no major defects. One phone is selected at random from the lot. What is the probability that it is
\begin{enumerate}
	\item acceptable to Varnika?
            \item acceptable to the trader?
\end{enumerate}
\solution
	%\input{exemplar/10/13/3/40/main.tex}
 \item A student says that if you throw a die, it will show up 1 or not 1. Therefore, the probability of getting 1 and the probability of getting 'not 1' each is equal to $\frac{1}{2}$. Is this correct? Give reasons.\\
 \solution
        %\input{exemplar/10/13/2/9/main.tex}
   \item Four candidates A, B, C, D have ap-
plied for the assignment to coach a school cricket
team. If A is twice as likely to be selected as B, and
B and C are given about the same chance of being
selected, while C is twice as likely to be selected
as D, what are the probabilities that
\begin{enumerate}
\item C will be selected?
\item A will not be selected?
\end{enumerate}
	%\input{exemplar/11/16/3/9/main.tex}
 \item A bag contain 24 balls of which $x$ balls are red, $2x$ are white and $3x$ are blue. A ball is selected at random, What is the probability that it is
\begin{enumerate}[label=\alph*)]
\item not red ?
\item white ?
\end{enumerate}
%\input{exemplar/10/13/3/41/main.tex}
If the letters of the word ASSASSINATION are arranged at random. Find the Probability that
\begin{enumerate}[label=(\alph*)]
\item Four $S's$ come consecutively in the word
\item Two  $I's$ and two $N's$ come together
\item All $A's$ are not coming together
\item No two $A's$ are coming together
\end{enumerate}
%\input{exemplar/11/16/3/14/main.tex}
	\item One urn contains two black balls (labelled B1 and B2) and one white ball. A
	second urn contains one black ball and two white balls (labelled W1 and W2).
	Suppose the following experiment is performed. One of the two urns is chosen
	at random. Next a ball is randomly chosen from the urn. Then a second ball is
	chosen at random from the same urn without replacing the first ball.
	
	\begin{enumerate}
	\item What is the probability that two black balls are chosen?
	
	\item What is the probability that two balls of opposite colour are chosen?
	\end{enumerate}
	\solution
	%\input{exemplar/11/16/3/12/main1.tex}
\end{enumerate}

	\item 
The number lock of a suitcase has 4 wheels each labelled with ten digits i.e. from 0 to 9.The lock opens with a sequence of four digits with no repeats.What is the probability of a person getting the right sequence to open the suitcase.
\\
\solution
		%\begin{enumerate}[label=\thesection.\arabic*,ref=\thesection.\theenumi]
	\item One card is drawn from a well-shuffled deck of 52 cards. Find the probability of getting
\begin{enumerate}
\item A king of red colour 
\item A face card 
\item A red face card
\item The jack of hearts
\item A spade
\item The queen of diamonds

\end{enumerate}
\solution
		%\input{ncert/10/15/1/14/main.tex}
	\item Five cards—the ten, jack, queen, king and ace of diamonds, are well-shuffled with their face downwards. One card is then picked up at random.
\begin{enumerate}
\item
What is the probability that the card is the queen? 
\item
If the queen is drawn and put aside, what is the probability that the second card picked up is (a) an ace? (b) a queen?\\
\end{enumerate}
\solution
		%\input{ncert/10/15/1/15/defs.tex}
	\item A bag contains $5$ red balls and some blue balls. If the probability of drawing a blue ball is double that if a red ball, determine the number of blue balls in the bag. 
		\\
\solution
		%\input{ncert/10/15/2/3/defs.tex}
	\item A card is selected from a pack of 52 cards.
 \begin{enumerate}[label=(\alph*)] 
                 \item How many points are there in the sample space?
                 \item Calculate the probability that the card is an ace of spades.
                 \item Calculate the probability that the card is (i) an ace and (ii) black card.
 \end{enumerate}
\solution
		%\input{ncert/11/16/3/4/main.tex}
\item Four cards are drawn from a well-shuffled deck of 52 cards. What is the probability of obtaining 3 diamonds and one spade.
\\
\solution
		%\input{ncert/11/16/4/2/defs.tex}
\item In a certain lottery 10,000 tickets are sold and ten equal prizes are awarded. What is the probability of not getting a prize if you buy (a) one ticket (b) two tickets (c) 10 tickets ?	
\\
\solution
		%\input{ncert/11/16/4/4/defs.tex}
		%
\item 
Out of 100 students, two sections of 40 and 60 are formed. If you and your friend are among the 100 students, what is the probability that
\begin{enumerate}
\item you both enter the same section?
\item you both enter the different sections?
\end{enumerate}
\solution
		%\input{ncert/11/16/4/5/defs.tex}
	\item 
The number lock of a suitcase has 4 wheels each labelled with ten digits i.e. from 0 to 9.The lock opens with a sequence of four digits with no repeats.What is the probability of a person getting the right sequence to open the suitcase.
\\
\solution
		%\input{ncert/11/16/4/10/defs.tex}
		%
\item 
Two cards are drawn at random and without replacement from a pack of 52 playing cards. Find the probability that both the cards are black.
\\
\solution
		%\input{ncert/12/13/2/2/defs.tex}
		\item A box of oranges is inspected by examining three randomly selected oranges drawn without replacement. If all the three oranges are good, the box is approved for sale, otherwise, it is rejected. Find the probability that a box containing 15 oranges out of which 12 are good and 3 are bad ones will be approved for sale.
		\label{ncert/12/13/2/3/defs.tex}
		\item Two balls are drawn at random with replacement from a box containing 10 black and 8 red balls. Find the probability that
		\label{ncert/12/13/2/12}
\begin{enumerate}
\item both balls are red.
\item first ball is black and second is red.
\item one of them is black and other is red.
\end{enumerate}

\item In a hostel, 60\% of the students read Hindi newspaper, 40\% read English newspaper and 20\% read both Hindi and English newspapers. A student is selected at random.
		\label{ncert/12/13/2/15}
\begin{enumerate}
\item Find the probability that she reads neither Hindi nor English newspapers.
\item If she reads Hindi newspaper, find the probability that she reads English newspaper.
\item If she reads English newspaper, find the probability that she reads Hindi newspaper.\\
\end{enumerate}
\item The probability of obtaining an even prime number on each die, when a pair of dice is rolled is 
\begin{enumerate}
    \item $0$ 
    
    \item $\frac{1}{3}$ 
    
    \item $\frac{1}{12}$ 
    
    \item $\frac{1}{36}$ 
\end{enumerate}
\solution
		%\input{ncert/12/13/2/17/defs.tex}
	\item A bag contains 4 red and 4 black balls, another bag contains 2 red and 6 black balls. One of the two bags is selected at random and a ball is drawn from the bag which is found to be red. Find the probability that the ball is drawn from the first bag.
\\
\solution
		%\input{ncert/12/13/3/2/main.tex}
  \item
  Cards with numbers 2 to 101 are placed in a box. A card is selected at random.Find the probability that the card has
\begin{enumerate}[label=(\roman*)]
	\item an even number 
	\item a square number
\end{enumerate}
\solution
%\input{exemplar/10/13/3/32/main.tex}
\item
The king, queen and jack of clubs are removed from a deck of 52 playing cards and then well shuffled. Now one card is drawn at random from the remaining cards.  Determine the probability that the card is
\begin{enumerate}[label=(\roman*)]
\item a club
\item 10 of hearts
\end{enumerate}
\solution
%\input{exemplar/10/13/3/29/main.tex}
\item A team of medical students doing their internship have to assist during surgeries
at a city hospital. The probabilities of surgeries rated as very complex, complex,
routine, simple or very simple are respectively, 0.15, 0.20, 0.31, 0.26, .08. Find
the probabilities that a particular surgery will be rated
\begin{enumerate}
	\item complex or very complex;
	\item neither very complex nor very simple;
	\item routine or complex
	\item routine or simple
\end{enumerate}
\solution
%\input{exemplar/11/16/3/8(1)/main.tex}
\item A card is selected from a pack of 52 cards.
\begin{enumerate}[label=(\alph*)]
    \item How many points are there in the sample space?
    \item Calculate the probability that the card is an ace of spades.
    \item Calculate the probability that the card is (i) an ace and (ii) black card.
\end{enumerate}
\solution
%\input{exemplar/11/16/3/4/main2.tex}
\item The probability that a non leap year selected at random will contain 53 sundays.
\\
\solution
%\input{exemplar/10/13/1/19/main.tex}
\item One of the four persons John, Rita, Aslam or Gurpreet will be promoted next
month. Consequently the sample space consists of four elementary outcomes
S = {John promoted, Rita promoted, Aslam promoted, Gurpreet promoted}
You are told that the chances of John’s promotion is same as that of Gurpreet,
Rita’s chances of promotion are twice as likely as Johns. Aslam’s chances are
four times that of John.
\begin{enumerate}
	\item Determine
	\begin{enumerate}
		\item P (John promoted)
		\item P (Rita promoted)
		\item P (Aslam promoted)
		\item P (Gurpreet promoted)
	\end{enumerate}
	\item If A = {John promoted or Gurpreet promoted}, find P (A).
\end{enumerate}
\solution
%\input{exemplar/11/16/3/10/main.tex}
\item A card is drawn from a deck of 52 cards. Find the probability of getting a king or a heart or a red card.\\
\solution
%\input{exemplar/11/16/3/15/main.tex}
\item The probability that a student will pass his examination is 0.73, the probability of
the student getting a compartment is 0.13, and the probability that the student will
either pass or get compartment is 0.96. State True or False.\\
\solution
%\input{exemplar/11/16/3/31/main.tex}
\item A card is selected from a pack of 52 cards\\
\begin{enumerate}[label=(\alph*)]
\item How many points are there in the sample space?
\item Calculate the probability that the cards is an ace of spades.
\item Calculate the probability that the card is (i) an ace (ii)black card.\\
\end{enumerate}
%\input{ncert/11/16/3/4_1/Prob_4.tex}
\item In a non-leap year, the probability of having 53 tuesdays or 53 wednesdays is\\
\solution
%\input{exemplar/11/16/3/18/main.tex}
\item There are 1000 sealed envelopes in a box, 10 of them contain a cash prize of
Rs 100 each, 100 of them contain a cash prize of Rs 50 each and 200 of them
contain a cash prize of Rs 10 each and rest do not contain any cash prize. If they
are well shuffled and an envelope is picked up out, what is the probability that it
contains no cash prize?\\
\solution
%\input{exemplar/10/13/3/34/main.tex}
\item 
A die is thrown and a card is selected at random from a deck of 52 playing cards. The probability of getting an even number on the die and a spade card.\\
\solution
%\input{exemplar/12/13/3/78/main.tex}
\item
If 4-digit numbers greater than 5,000 are randomly formed from the digits 0, 1, 3, 5, and 7, what is the probability of forming a number divisible by 5 when:
\begin{enumerate}
    \item The digits are repeated?
    \item The repetition of digits is not allowed?
\end{enumerate}
\solution
%\input{ncert/11/16/4/9/main.tex}
\item Consider the probability space $\brak{\Omega, \mathcal{G}, P}$ where $\Omega = [0,2]$ and $\mathcal{G} = \cbrak{\phi, \Omega, [0,1], (1,2]}$. Let $X$ and $Y$ be two functions on $\Omega$ defined as
\begin{align*}
    X(\omega) = 
    \begin{cases}
        1 & \text{if }\omega \in [0, 1]\\
        2 & \text{if }\omega \in (1, 2]
    \end{cases}
\end{align*}
and
\begin{align*}
    Y(\omega) = 
    \begin{cases}
        2 & \text{if }\omega \in [0, 1.5]\\
        3 & \text{if }\omega \in (1.5, 2].
    \end{cases}
\end{align*}
Then which one of the following statements is true?
\begin{enumerate}
    \item [(A)] $X$ is a random variable with respect to $\mathcal{G}$, but $Y$ is not a random variable with respect to $\mathcal{G}$.
    \item [(B)] $Y$ is a random variable with respect to $\mathcal{G}$, but $X$ is not a random variable with respect to $\mathcal{G}$.
    \item [(C)] Neither $X$ nor $Y$ is a random variable with respect to $\mathcal{G}$.
    \item [(D)] Both $X$ and $Y$ are random variables with respect to $\mathcal{G}$.
\end{enumerate} \hfill (GATE ST 2023)\\
\solution
%\input{gate/ST/2023/14/main.tex}
	\item  A die is loaded in such a way that each odd number is twice as likely to occur as
each even number. Find $P(G)$, where $G$ is the event that a number greater than
3 occurs on a single roll of the die.
\\
\solution
		%\input{exemplar/11/16/3/5/main.tex}
	\item All the jacks, queens and kings are removed from a deck of 52 playing cards. The remaining cards are well shuffled and then one card is drawn at random. Giving ace a value 1 similar value for other cards, find the probability that the card has a value 
		\begin{enumerate}
			\item 7
			\item greater than 7
			\item less than 7
		\end{enumerate}
		%\input{exemplar/10/13/3/30/main.tex}
  \item A Lot consists of 48 mobile phones of which 42 are good, 3 have only minor defects and 3 have major defects.Varnika will buy a phone if it is good but the trader will only buy a mobile if it has no major defects. One phone is selected at random from the lot. What is the probability that it is
\begin{enumerate}
	\item acceptable to Varnika?
            \item acceptable to the trader?
\end{enumerate}
\solution
	%\input{exemplar/10/13/3/40/main.tex}
 \item A student says that if you throw a die, it will show up 1 or not 1. Therefore, the probability of getting 1 and the probability of getting 'not 1' each is equal to $\frac{1}{2}$. Is this correct? Give reasons.\\
 \solution
        %\input{exemplar/10/13/2/9/main.tex}
   \item Four candidates A, B, C, D have ap-
plied for the assignment to coach a school cricket
team. If A is twice as likely to be selected as B, and
B and C are given about the same chance of being
selected, while C is twice as likely to be selected
as D, what are the probabilities that
\begin{enumerate}
\item C will be selected?
\item A will not be selected?
\end{enumerate}
	%\input{exemplar/11/16/3/9/main.tex}
 \item A bag contain 24 balls of which $x$ balls are red, $2x$ are white and $3x$ are blue. A ball is selected at random, What is the probability that it is
\begin{enumerate}[label=\alph*)]
\item not red ?
\item white ?
\end{enumerate}
%\input{exemplar/10/13/3/41/main.tex}
If the letters of the word ASSASSINATION are arranged at random. Find the Probability that
\begin{enumerate}[label=(\alph*)]
\item Four $S's$ come consecutively in the word
\item Two  $I's$ and two $N's$ come together
\item All $A's$ are not coming together
\item No two $A's$ are coming together
\end{enumerate}
%\input{exemplar/11/16/3/14/main.tex}
	\item One urn contains two black balls (labelled B1 and B2) and one white ball. A
	second urn contains one black ball and two white balls (labelled W1 and W2).
	Suppose the following experiment is performed. One of the two urns is chosen
	at random. Next a ball is randomly chosen from the urn. Then a second ball is
	chosen at random from the same urn without replacing the first ball.
	
	\begin{enumerate}
	\item What is the probability that two black balls are chosen?
	
	\item What is the probability that two balls of opposite colour are chosen?
	\end{enumerate}
	\solution
	%\input{exemplar/11/16/3/12/main1.tex}
\end{enumerate}

		%
\item 
Two cards are drawn at random and without replacement from a pack of 52 playing cards. Find the probability that both the cards are black.
\\
\solution
		%\begin{enumerate}[label=\thesection.\arabic*,ref=\thesection.\theenumi]
	\item One card is drawn from a well-shuffled deck of 52 cards. Find the probability of getting
\begin{enumerate}
\item A king of red colour 
\item A face card 
\item A red face card
\item The jack of hearts
\item A spade
\item The queen of diamonds

\end{enumerate}
\solution
		%\input{ncert/10/15/1/14/main.tex}
	\item Five cards—the ten, jack, queen, king and ace of diamonds, are well-shuffled with their face downwards. One card is then picked up at random.
\begin{enumerate}
\item
What is the probability that the card is the queen? 
\item
If the queen is drawn and put aside, what is the probability that the second card picked up is (a) an ace? (b) a queen?\\
\end{enumerate}
\solution
		%\input{ncert/10/15/1/15/defs.tex}
	\item A bag contains $5$ red balls and some blue balls. If the probability of drawing a blue ball is double that if a red ball, determine the number of blue balls in the bag. 
		\\
\solution
		%\input{ncert/10/15/2/3/defs.tex}
	\item A card is selected from a pack of 52 cards.
 \begin{enumerate}[label=(\alph*)] 
                 \item How many points are there in the sample space?
                 \item Calculate the probability that the card is an ace of spades.
                 \item Calculate the probability that the card is (i) an ace and (ii) black card.
 \end{enumerate}
\solution
		%\input{ncert/11/16/3/4/main.tex}
\item Four cards are drawn from a well-shuffled deck of 52 cards. What is the probability of obtaining 3 diamonds and one spade.
\\
\solution
		%\input{ncert/11/16/4/2/defs.tex}
\item In a certain lottery 10,000 tickets are sold and ten equal prizes are awarded. What is the probability of not getting a prize if you buy (a) one ticket (b) two tickets (c) 10 tickets ?	
\\
\solution
		%\input{ncert/11/16/4/4/defs.tex}
		%
\item 
Out of 100 students, two sections of 40 and 60 are formed. If you and your friend are among the 100 students, what is the probability that
\begin{enumerate}
\item you both enter the same section?
\item you both enter the different sections?
\end{enumerate}
\solution
		%\input{ncert/11/16/4/5/defs.tex}
	\item 
The number lock of a suitcase has 4 wheels each labelled with ten digits i.e. from 0 to 9.The lock opens with a sequence of four digits with no repeats.What is the probability of a person getting the right sequence to open the suitcase.
\\
\solution
		%\input{ncert/11/16/4/10/defs.tex}
		%
\item 
Two cards are drawn at random and without replacement from a pack of 52 playing cards. Find the probability that both the cards are black.
\\
\solution
		%\input{ncert/12/13/2/2/defs.tex}
		\item A box of oranges is inspected by examining three randomly selected oranges drawn without replacement. If all the three oranges are good, the box is approved for sale, otherwise, it is rejected. Find the probability that a box containing 15 oranges out of which 12 are good and 3 are bad ones will be approved for sale.
		\label{ncert/12/13/2/3/defs.tex}
		\item Two balls are drawn at random with replacement from a box containing 10 black and 8 red balls. Find the probability that
		\label{ncert/12/13/2/12}
\begin{enumerate}
\item both balls are red.
\item first ball is black and second is red.
\item one of them is black and other is red.
\end{enumerate}

\item In a hostel, 60\% of the students read Hindi newspaper, 40\% read English newspaper and 20\% read both Hindi and English newspapers. A student is selected at random.
		\label{ncert/12/13/2/15}
\begin{enumerate}
\item Find the probability that she reads neither Hindi nor English newspapers.
\item If she reads Hindi newspaper, find the probability that she reads English newspaper.
\item If she reads English newspaper, find the probability that she reads Hindi newspaper.\\
\end{enumerate}
\item The probability of obtaining an even prime number on each die, when a pair of dice is rolled is 
\begin{enumerate}
    \item $0$ 
    
    \item $\frac{1}{3}$ 
    
    \item $\frac{1}{12}$ 
    
    \item $\frac{1}{36}$ 
\end{enumerate}
\solution
		%\input{ncert/12/13/2/17/defs.tex}
	\item A bag contains 4 red and 4 black balls, another bag contains 2 red and 6 black balls. One of the two bags is selected at random and a ball is drawn from the bag which is found to be red. Find the probability that the ball is drawn from the first bag.
\\
\solution
		%\input{ncert/12/13/3/2/main.tex}
  \item
  Cards with numbers 2 to 101 are placed in a box. A card is selected at random.Find the probability that the card has
\begin{enumerate}[label=(\roman*)]
	\item an even number 
	\item a square number
\end{enumerate}
\solution
%\input{exemplar/10/13/3/32/main.tex}
\item
The king, queen and jack of clubs are removed from a deck of 52 playing cards and then well shuffled. Now one card is drawn at random from the remaining cards.  Determine the probability that the card is
\begin{enumerate}[label=(\roman*)]
\item a club
\item 10 of hearts
\end{enumerate}
\solution
%\input{exemplar/10/13/3/29/main.tex}
\item A team of medical students doing their internship have to assist during surgeries
at a city hospital. The probabilities of surgeries rated as very complex, complex,
routine, simple or very simple are respectively, 0.15, 0.20, 0.31, 0.26, .08. Find
the probabilities that a particular surgery will be rated
\begin{enumerate}
	\item complex or very complex;
	\item neither very complex nor very simple;
	\item routine or complex
	\item routine or simple
\end{enumerate}
\solution
%\input{exemplar/11/16/3/8(1)/main.tex}
\item A card is selected from a pack of 52 cards.
\begin{enumerate}[label=(\alph*)]
    \item How many points are there in the sample space?
    \item Calculate the probability that the card is an ace of spades.
    \item Calculate the probability that the card is (i) an ace and (ii) black card.
\end{enumerate}
\solution
%\input{exemplar/11/16/3/4/main2.tex}
\item The probability that a non leap year selected at random will contain 53 sundays.
\\
\solution
%\input{exemplar/10/13/1/19/main.tex}
\item One of the four persons John, Rita, Aslam or Gurpreet will be promoted next
month. Consequently the sample space consists of four elementary outcomes
S = {John promoted, Rita promoted, Aslam promoted, Gurpreet promoted}
You are told that the chances of John’s promotion is same as that of Gurpreet,
Rita’s chances of promotion are twice as likely as Johns. Aslam’s chances are
four times that of John.
\begin{enumerate}
	\item Determine
	\begin{enumerate}
		\item P (John promoted)
		\item P (Rita promoted)
		\item P (Aslam promoted)
		\item P (Gurpreet promoted)
	\end{enumerate}
	\item If A = {John promoted or Gurpreet promoted}, find P (A).
\end{enumerate}
\solution
%\input{exemplar/11/16/3/10/main.tex}
\item A card is drawn from a deck of 52 cards. Find the probability of getting a king or a heart or a red card.\\
\solution
%\input{exemplar/11/16/3/15/main.tex}
\item The probability that a student will pass his examination is 0.73, the probability of
the student getting a compartment is 0.13, and the probability that the student will
either pass or get compartment is 0.96. State True or False.\\
\solution
%\input{exemplar/11/16/3/31/main.tex}
\item A card is selected from a pack of 52 cards\\
\begin{enumerate}[label=(\alph*)]
\item How many points are there in the sample space?
\item Calculate the probability that the cards is an ace of spades.
\item Calculate the probability that the card is (i) an ace (ii)black card.\\
\end{enumerate}
%\input{ncert/11/16/3/4_1/Prob_4.tex}
\item In a non-leap year, the probability of having 53 tuesdays or 53 wednesdays is\\
\solution
%\input{exemplar/11/16/3/18/main.tex}
\item There are 1000 sealed envelopes in a box, 10 of them contain a cash prize of
Rs 100 each, 100 of them contain a cash prize of Rs 50 each and 200 of them
contain a cash prize of Rs 10 each and rest do not contain any cash prize. If they
are well shuffled and an envelope is picked up out, what is the probability that it
contains no cash prize?\\
\solution
%\input{exemplar/10/13/3/34/main.tex}
\item 
A die is thrown and a card is selected at random from a deck of 52 playing cards. The probability of getting an even number on the die and a spade card.\\
\solution
%\input{exemplar/12/13/3/78/main.tex}
\item
If 4-digit numbers greater than 5,000 are randomly formed from the digits 0, 1, 3, 5, and 7, what is the probability of forming a number divisible by 5 when:
\begin{enumerate}
    \item The digits are repeated?
    \item The repetition of digits is not allowed?
\end{enumerate}
\solution
%\input{ncert/11/16/4/9/main.tex}
\item Consider the probability space $\brak{\Omega, \mathcal{G}, P}$ where $\Omega = [0,2]$ and $\mathcal{G} = \cbrak{\phi, \Omega, [0,1], (1,2]}$. Let $X$ and $Y$ be two functions on $\Omega$ defined as
\begin{align*}
    X(\omega) = 
    \begin{cases}
        1 & \text{if }\omega \in [0, 1]\\
        2 & \text{if }\omega \in (1, 2]
    \end{cases}
\end{align*}
and
\begin{align*}
    Y(\omega) = 
    \begin{cases}
        2 & \text{if }\omega \in [0, 1.5]\\
        3 & \text{if }\omega \in (1.5, 2].
    \end{cases}
\end{align*}
Then which one of the following statements is true?
\begin{enumerate}
    \item [(A)] $X$ is a random variable with respect to $\mathcal{G}$, but $Y$ is not a random variable with respect to $\mathcal{G}$.
    \item [(B)] $Y$ is a random variable with respect to $\mathcal{G}$, but $X$ is not a random variable with respect to $\mathcal{G}$.
    \item [(C)] Neither $X$ nor $Y$ is a random variable with respect to $\mathcal{G}$.
    \item [(D)] Both $X$ and $Y$ are random variables with respect to $\mathcal{G}$.
\end{enumerate} \hfill (GATE ST 2023)\\
\solution
%\input{gate/ST/2023/14/main.tex}
	\item  A die is loaded in such a way that each odd number is twice as likely to occur as
each even number. Find $P(G)$, where $G$ is the event that a number greater than
3 occurs on a single roll of the die.
\\
\solution
		%\input{exemplar/11/16/3/5/main.tex}
	\item All the jacks, queens and kings are removed from a deck of 52 playing cards. The remaining cards are well shuffled and then one card is drawn at random. Giving ace a value 1 similar value for other cards, find the probability that the card has a value 
		\begin{enumerate}
			\item 7
			\item greater than 7
			\item less than 7
		\end{enumerate}
		%\input{exemplar/10/13/3/30/main.tex}
  \item A Lot consists of 48 mobile phones of which 42 are good, 3 have only minor defects and 3 have major defects.Varnika will buy a phone if it is good but the trader will only buy a mobile if it has no major defects. One phone is selected at random from the lot. What is the probability that it is
\begin{enumerate}
	\item acceptable to Varnika?
            \item acceptable to the trader?
\end{enumerate}
\solution
	%\input{exemplar/10/13/3/40/main.tex}
 \item A student says that if you throw a die, it will show up 1 or not 1. Therefore, the probability of getting 1 and the probability of getting 'not 1' each is equal to $\frac{1}{2}$. Is this correct? Give reasons.\\
 \solution
        %\input{exemplar/10/13/2/9/main.tex}
   \item Four candidates A, B, C, D have ap-
plied for the assignment to coach a school cricket
team. If A is twice as likely to be selected as B, and
B and C are given about the same chance of being
selected, while C is twice as likely to be selected
as D, what are the probabilities that
\begin{enumerate}
\item C will be selected?
\item A will not be selected?
\end{enumerate}
	%\input{exemplar/11/16/3/9/main.tex}
 \item A bag contain 24 balls of which $x$ balls are red, $2x$ are white and $3x$ are blue. A ball is selected at random, What is the probability that it is
\begin{enumerate}[label=\alph*)]
\item not red ?
\item white ?
\end{enumerate}
%\input{exemplar/10/13/3/41/main.tex}
If the letters of the word ASSASSINATION are arranged at random. Find the Probability that
\begin{enumerate}[label=(\alph*)]
\item Four $S's$ come consecutively in the word
\item Two  $I's$ and two $N's$ come together
\item All $A's$ are not coming together
\item No two $A's$ are coming together
\end{enumerate}
%\input{exemplar/11/16/3/14/main.tex}
	\item One urn contains two black balls (labelled B1 and B2) and one white ball. A
	second urn contains one black ball and two white balls (labelled W1 and W2).
	Suppose the following experiment is performed. One of the two urns is chosen
	at random. Next a ball is randomly chosen from the urn. Then a second ball is
	chosen at random from the same urn without replacing the first ball.
	
	\begin{enumerate}
	\item What is the probability that two black balls are chosen?
	
	\item What is the probability that two balls of opposite colour are chosen?
	\end{enumerate}
	\solution
	%\input{exemplar/11/16/3/12/main1.tex}
\end{enumerate}

		\item A box of oranges is inspected by examining three randomly selected oranges drawn without replacement. If all the three oranges are good, the box is approved for sale, otherwise, it is rejected. Find the probability that a box containing 15 oranges out of which 12 are good and 3 are bad ones will be approved for sale.
		\label{ncert/12/13/2/3/defs.tex}
		\item Two balls are drawn at random with replacement from a box containing 10 black and 8 red balls. Find the probability that
		\label{ncert/12/13/2/12}
\begin{enumerate}
\item both balls are red.
\item first ball is black and second is red.
\item one of them is black and other is red.
\end{enumerate}

\item In a hostel, 60\% of the students read Hindi newspaper, 40\% read English newspaper and 20\% read both Hindi and English newspapers. A student is selected at random.
		\label{ncert/12/13/2/15}
\begin{enumerate}
\item Find the probability that she reads neither Hindi nor English newspapers.
\item If she reads Hindi newspaper, find the probability that she reads English newspaper.
\item If she reads English newspaper, find the probability that she reads Hindi newspaper.\\
\end{enumerate}
\item The probability of obtaining an even prime number on each die, when a pair of dice is rolled is 
\begin{enumerate}
    \item $0$ 
    
    \item $\frac{1}{3}$ 
    
    \item $\frac{1}{12}$ 
    
    \item $\frac{1}{36}$ 
\end{enumerate}
\solution
		%\begin{enumerate}[label=\thesection.\arabic*,ref=\thesection.\theenumi]
	\item One card is drawn from a well-shuffled deck of 52 cards. Find the probability of getting
\begin{enumerate}
\item A king of red colour 
\item A face card 
\item A red face card
\item The jack of hearts
\item A spade
\item The queen of diamonds

\end{enumerate}
\solution
		%\input{ncert/10/15/1/14/main.tex}
	\item Five cards—the ten, jack, queen, king and ace of diamonds, are well-shuffled with their face downwards. One card is then picked up at random.
\begin{enumerate}
\item
What is the probability that the card is the queen? 
\item
If the queen is drawn and put aside, what is the probability that the second card picked up is (a) an ace? (b) a queen?\\
\end{enumerate}
\solution
		%\input{ncert/10/15/1/15/defs.tex}
	\item A bag contains $5$ red balls and some blue balls. If the probability of drawing a blue ball is double that if a red ball, determine the number of blue balls in the bag. 
		\\
\solution
		%\input{ncert/10/15/2/3/defs.tex}
	\item A card is selected from a pack of 52 cards.
 \begin{enumerate}[label=(\alph*)] 
                 \item How many points are there in the sample space?
                 \item Calculate the probability that the card is an ace of spades.
                 \item Calculate the probability that the card is (i) an ace and (ii) black card.
 \end{enumerate}
\solution
		%\input{ncert/11/16/3/4/main.tex}
\item Four cards are drawn from a well-shuffled deck of 52 cards. What is the probability of obtaining 3 diamonds and one spade.
\\
\solution
		%\input{ncert/11/16/4/2/defs.tex}
\item In a certain lottery 10,000 tickets are sold and ten equal prizes are awarded. What is the probability of not getting a prize if you buy (a) one ticket (b) two tickets (c) 10 tickets ?	
\\
\solution
		%\input{ncert/11/16/4/4/defs.tex}
		%
\item 
Out of 100 students, two sections of 40 and 60 are formed. If you and your friend are among the 100 students, what is the probability that
\begin{enumerate}
\item you both enter the same section?
\item you both enter the different sections?
\end{enumerate}
\solution
		%\input{ncert/11/16/4/5/defs.tex}
	\item 
The number lock of a suitcase has 4 wheels each labelled with ten digits i.e. from 0 to 9.The lock opens with a sequence of four digits with no repeats.What is the probability of a person getting the right sequence to open the suitcase.
\\
\solution
		%\input{ncert/11/16/4/10/defs.tex}
		%
\item 
Two cards are drawn at random and without replacement from a pack of 52 playing cards. Find the probability that both the cards are black.
\\
\solution
		%\input{ncert/12/13/2/2/defs.tex}
		\item A box of oranges is inspected by examining three randomly selected oranges drawn without replacement. If all the three oranges are good, the box is approved for sale, otherwise, it is rejected. Find the probability that a box containing 15 oranges out of which 12 are good and 3 are bad ones will be approved for sale.
		\label{ncert/12/13/2/3/defs.tex}
		\item Two balls are drawn at random with replacement from a box containing 10 black and 8 red balls. Find the probability that
		\label{ncert/12/13/2/12}
\begin{enumerate}
\item both balls are red.
\item first ball is black and second is red.
\item one of them is black and other is red.
\end{enumerate}

\item In a hostel, 60\% of the students read Hindi newspaper, 40\% read English newspaper and 20\% read both Hindi and English newspapers. A student is selected at random.
		\label{ncert/12/13/2/15}
\begin{enumerate}
\item Find the probability that she reads neither Hindi nor English newspapers.
\item If she reads Hindi newspaper, find the probability that she reads English newspaper.
\item If she reads English newspaper, find the probability that she reads Hindi newspaper.\\
\end{enumerate}
\item The probability of obtaining an even prime number on each die, when a pair of dice is rolled is 
\begin{enumerate}
    \item $0$ 
    
    \item $\frac{1}{3}$ 
    
    \item $\frac{1}{12}$ 
    
    \item $\frac{1}{36}$ 
\end{enumerate}
\solution
		%\input{ncert/12/13/2/17/defs.tex}
	\item A bag contains 4 red and 4 black balls, another bag contains 2 red and 6 black balls. One of the two bags is selected at random and a ball is drawn from the bag which is found to be red. Find the probability that the ball is drawn from the first bag.
\\
\solution
		%\input{ncert/12/13/3/2/main.tex}
  \item
  Cards with numbers 2 to 101 are placed in a box. A card is selected at random.Find the probability that the card has
\begin{enumerate}[label=(\roman*)]
	\item an even number 
	\item a square number
\end{enumerate}
\solution
%\input{exemplar/10/13/3/32/main.tex}
\item
The king, queen and jack of clubs are removed from a deck of 52 playing cards and then well shuffled. Now one card is drawn at random from the remaining cards.  Determine the probability that the card is
\begin{enumerate}[label=(\roman*)]
\item a club
\item 10 of hearts
\end{enumerate}
\solution
%\input{exemplar/10/13/3/29/main.tex}
\item A team of medical students doing their internship have to assist during surgeries
at a city hospital. The probabilities of surgeries rated as very complex, complex,
routine, simple or very simple are respectively, 0.15, 0.20, 0.31, 0.26, .08. Find
the probabilities that a particular surgery will be rated
\begin{enumerate}
	\item complex or very complex;
	\item neither very complex nor very simple;
	\item routine or complex
	\item routine or simple
\end{enumerate}
\solution
%\input{exemplar/11/16/3/8(1)/main.tex}
\item A card is selected from a pack of 52 cards.
\begin{enumerate}[label=(\alph*)]
    \item How many points are there in the sample space?
    \item Calculate the probability that the card is an ace of spades.
    \item Calculate the probability that the card is (i) an ace and (ii) black card.
\end{enumerate}
\solution
%\input{exemplar/11/16/3/4/main2.tex}
\item The probability that a non leap year selected at random will contain 53 sundays.
\\
\solution
%\input{exemplar/10/13/1/19/main.tex}
\item One of the four persons John, Rita, Aslam or Gurpreet will be promoted next
month. Consequently the sample space consists of four elementary outcomes
S = {John promoted, Rita promoted, Aslam promoted, Gurpreet promoted}
You are told that the chances of John’s promotion is same as that of Gurpreet,
Rita’s chances of promotion are twice as likely as Johns. Aslam’s chances are
four times that of John.
\begin{enumerate}
	\item Determine
	\begin{enumerate}
		\item P (John promoted)
		\item P (Rita promoted)
		\item P (Aslam promoted)
		\item P (Gurpreet promoted)
	\end{enumerate}
	\item If A = {John promoted or Gurpreet promoted}, find P (A).
\end{enumerate}
\solution
%\input{exemplar/11/16/3/10/main.tex}
\item A card is drawn from a deck of 52 cards. Find the probability of getting a king or a heart or a red card.\\
\solution
%\input{exemplar/11/16/3/15/main.tex}
\item The probability that a student will pass his examination is 0.73, the probability of
the student getting a compartment is 0.13, and the probability that the student will
either pass or get compartment is 0.96. State True or False.\\
\solution
%\input{exemplar/11/16/3/31/main.tex}
\item A card is selected from a pack of 52 cards\\
\begin{enumerate}[label=(\alph*)]
\item How many points are there in the sample space?
\item Calculate the probability that the cards is an ace of spades.
\item Calculate the probability that the card is (i) an ace (ii)black card.\\
\end{enumerate}
%\input{ncert/11/16/3/4_1/Prob_4.tex}
\item In a non-leap year, the probability of having 53 tuesdays or 53 wednesdays is\\
\solution
%\input{exemplar/11/16/3/18/main.tex}
\item There are 1000 sealed envelopes in a box, 10 of them contain a cash prize of
Rs 100 each, 100 of them contain a cash prize of Rs 50 each and 200 of them
contain a cash prize of Rs 10 each and rest do not contain any cash prize. If they
are well shuffled and an envelope is picked up out, what is the probability that it
contains no cash prize?\\
\solution
%\input{exemplar/10/13/3/34/main.tex}
\item 
A die is thrown and a card is selected at random from a deck of 52 playing cards. The probability of getting an even number on the die and a spade card.\\
\solution
%\input{exemplar/12/13/3/78/main.tex}
\item
If 4-digit numbers greater than 5,000 are randomly formed from the digits 0, 1, 3, 5, and 7, what is the probability of forming a number divisible by 5 when:
\begin{enumerate}
    \item The digits are repeated?
    \item The repetition of digits is not allowed?
\end{enumerate}
\solution
%\input{ncert/11/16/4/9/main.tex}
\item Consider the probability space $\brak{\Omega, \mathcal{G}, P}$ where $\Omega = [0,2]$ and $\mathcal{G} = \cbrak{\phi, \Omega, [0,1], (1,2]}$. Let $X$ and $Y$ be two functions on $\Omega$ defined as
\begin{align*}
    X(\omega) = 
    \begin{cases}
        1 & \text{if }\omega \in [0, 1]\\
        2 & \text{if }\omega \in (1, 2]
    \end{cases}
\end{align*}
and
\begin{align*}
    Y(\omega) = 
    \begin{cases}
        2 & \text{if }\omega \in [0, 1.5]\\
        3 & \text{if }\omega \in (1.5, 2].
    \end{cases}
\end{align*}
Then which one of the following statements is true?
\begin{enumerate}
    \item [(A)] $X$ is a random variable with respect to $\mathcal{G}$, but $Y$ is not a random variable with respect to $\mathcal{G}$.
    \item [(B)] $Y$ is a random variable with respect to $\mathcal{G}$, but $X$ is not a random variable with respect to $\mathcal{G}$.
    \item [(C)] Neither $X$ nor $Y$ is a random variable with respect to $\mathcal{G}$.
    \item [(D)] Both $X$ and $Y$ are random variables with respect to $\mathcal{G}$.
\end{enumerate} \hfill (GATE ST 2023)\\
\solution
%\input{gate/ST/2023/14/main.tex}
	\item  A die is loaded in such a way that each odd number is twice as likely to occur as
each even number. Find $P(G)$, where $G$ is the event that a number greater than
3 occurs on a single roll of the die.
\\
\solution
		%\input{exemplar/11/16/3/5/main.tex}
	\item All the jacks, queens and kings are removed from a deck of 52 playing cards. The remaining cards are well shuffled and then one card is drawn at random. Giving ace a value 1 similar value for other cards, find the probability that the card has a value 
		\begin{enumerate}
			\item 7
			\item greater than 7
			\item less than 7
		\end{enumerate}
		%\input{exemplar/10/13/3/30/main.tex}
  \item A Lot consists of 48 mobile phones of which 42 are good, 3 have only minor defects and 3 have major defects.Varnika will buy a phone if it is good but the trader will only buy a mobile if it has no major defects. One phone is selected at random from the lot. What is the probability that it is
\begin{enumerate}
	\item acceptable to Varnika?
            \item acceptable to the trader?
\end{enumerate}
\solution
	%\input{exemplar/10/13/3/40/main.tex}
 \item A student says that if you throw a die, it will show up 1 or not 1. Therefore, the probability of getting 1 and the probability of getting 'not 1' each is equal to $\frac{1}{2}$. Is this correct? Give reasons.\\
 \solution
        %\input{exemplar/10/13/2/9/main.tex}
   \item Four candidates A, B, C, D have ap-
plied for the assignment to coach a school cricket
team. If A is twice as likely to be selected as B, and
B and C are given about the same chance of being
selected, while C is twice as likely to be selected
as D, what are the probabilities that
\begin{enumerate}
\item C will be selected?
\item A will not be selected?
\end{enumerate}
	%\input{exemplar/11/16/3/9/main.tex}
 \item A bag contain 24 balls of which $x$ balls are red, $2x$ are white and $3x$ are blue. A ball is selected at random, What is the probability that it is
\begin{enumerate}[label=\alph*)]
\item not red ?
\item white ?
\end{enumerate}
%\input{exemplar/10/13/3/41/main.tex}
If the letters of the word ASSASSINATION are arranged at random. Find the Probability that
\begin{enumerate}[label=(\alph*)]
\item Four $S's$ come consecutively in the word
\item Two  $I's$ and two $N's$ come together
\item All $A's$ are not coming together
\item No two $A's$ are coming together
\end{enumerate}
%\input{exemplar/11/16/3/14/main.tex}
	\item One urn contains two black balls (labelled B1 and B2) and one white ball. A
	second urn contains one black ball and two white balls (labelled W1 and W2).
	Suppose the following experiment is performed. One of the two urns is chosen
	at random. Next a ball is randomly chosen from the urn. Then a second ball is
	chosen at random from the same urn without replacing the first ball.
	
	\begin{enumerate}
	\item What is the probability that two black balls are chosen?
	
	\item What is the probability that two balls of opposite colour are chosen?
	\end{enumerate}
	\solution
	%\input{exemplar/11/16/3/12/main1.tex}
\end{enumerate}

	\item A bag contains 4 red and 4 black balls, another bag contains 2 red and 6 black balls. One of the two bags is selected at random and a ball is drawn from the bag which is found to be red. Find the probability that the ball is drawn from the first bag.
\\
\solution
		%\begin{table}[H]
	\centering
\begin{tabular}{|c|c|c|}
\hline
Random variable &Value &Definition\\ \hline
\multirow{3}{*}{X} &0 &Slips of Rs 1\\
&1 &Slips of Rs 5\\
&2 &Slips of Rs 13\\ \hline
\multirow{2}{*}{Y} &0 &Box A\\
&1 &Box B\\\hline
\end{tabular}
\caption{}
\label{tab:Distribution}
\end{table}
See \tabref{tab:Distribution}.
\begin{align}
p_{Y}\brak{k}= \begin{cases} 
      \frac{1}{3} & {k=0} \\
      \frac{2}{3 }& {k=1} 
   \end{cases}
   \\
p_{Y|X}\brak{0|0} = \frac{19}{25}\, 
p_{Y|X}\brak{0|1} = \frac{6}{25}\,
p_{Y|X}\brak{1|0} = \frac{45}{50}\,
p_{Y|X}\brak{1|2} = \frac{5}{50}
\end{align}
The desired probability is the probability that a slip drawn at random is marked other than Rs 1,
\begin{align}
&=1-p_X\brak{0}\\
&= p_X(1) + p_X(2)
\end{align}
Using Bayes theorem,
\begin{align}
&= p_Y\brak{0} \times \pr{Y=0 | X=1} + p_Y\brak{1} \times \pr{Y=1|X=2}\\
&=\frac{1}{3} \times \frac{6}{25} + \frac{2}{3} \times \frac{5}{50}\\
&=\frac{11}{75}
\end{align}

\newpage

%\tableofcontents

\bigskip

\renewcommand{\thefigure}{\theenumi}
\renewcommand{\thetable}{\theenumi}
%\renewcommand{\theequation}{\theenumi}

%\begin{abstract}
%%\boldmath
%In this letter, an algorithm for evaluating the exact analytical bit error rate  (BER)  for the piecewise linear (PL) combiner for  multiple relays is presented. Previous results were available only for upto three relays. The algorithm is unique in the sense that  the actual mathematical expressions, that are prohibitively large, need not be explicitly obtained. The diversity gain due to multiple relays is shown through plots of the analytical BER, well supported by simulations. 
%
%\end{abstract}
% IEEEtran.cls defaults to using nonbold math in the Abstract.
% This preserves the distinction between vectors and scalars. However,
% if the journal you are submitting to favors bold math in the abstract,
% then you can use LaTeX's standard command \boldmath at the very start
% of the abstract to achieve this. Many IEEE journals frown on math
% in the abstract anyway.

% Note that keywords are not normally used for peerreview papers.
%\begin{IEEEkeywords}
%Cooperative diversity, decode and forward, piecewise linear
%\end{IEEEkeywords}



% For peer review papers, you can put extra information on the cover
% page as needed:
% \ifCLASSOPTIONpeerreview
% \begin{center} \bfseries EDICS Category: 3-BBND \end{center}
% \fi
%
% For peerreview papers, this IEEEtran command inserts a page break and
% creates the second title. It will be ignored for other modes.
%\IEEEpeerreviewmaketitle




  \item
  Cards with numbers 2 to 101 are placed in a box. A card is selected at random.Find the probability that the card has
\begin{enumerate}[label=(\roman*)]
	\item an even number 
	\item a square number
\end{enumerate}
\solution
%\begin{table}[H]
	\centering
\begin{tabular}{|c|c|c|}
\hline
Random variable &Value &Definition\\ \hline
\multirow{3}{*}{X} &0 &Slips of Rs 1\\
&1 &Slips of Rs 5\\
&2 &Slips of Rs 13\\ \hline
\multirow{2}{*}{Y} &0 &Box A\\
&1 &Box B\\\hline
\end{tabular}
\caption{}
\label{tab:Distribution}
\end{table}
See \tabref{tab:Distribution}.
\begin{align}
p_{Y}\brak{k}= \begin{cases} 
      \frac{1}{3} & {k=0} \\
      \frac{2}{3 }& {k=1} 
   \end{cases}
   \\
p_{Y|X}\brak{0|0} = \frac{19}{25}\, 
p_{Y|X}\brak{0|1} = \frac{6}{25}\,
p_{Y|X}\brak{1|0} = \frac{45}{50}\,
p_{Y|X}\brak{1|2} = \frac{5}{50}
\end{align}
The desired probability is the probability that a slip drawn at random is marked other than Rs 1,
\begin{align}
&=1-p_X\brak{0}\\
&= p_X(1) + p_X(2)
\end{align}
Using Bayes theorem,
\begin{align}
&= p_Y\brak{0} \times \pr{Y=0 | X=1} + p_Y\brak{1} \times \pr{Y=1|X=2}\\
&=\frac{1}{3} \times \frac{6}{25} + \frac{2}{3} \times \frac{5}{50}\\
&=\frac{11}{75}
\end{align}

\newpage

%\tableofcontents

\bigskip

\renewcommand{\thefigure}{\theenumi}
\renewcommand{\thetable}{\theenumi}
%\renewcommand{\theequation}{\theenumi}

%\begin{abstract}
%%\boldmath
%In this letter, an algorithm for evaluating the exact analytical bit error rate  (BER)  for the piecewise linear (PL) combiner for  multiple relays is presented. Previous results were available only for upto three relays. The algorithm is unique in the sense that  the actual mathematical expressions, that are prohibitively large, need not be explicitly obtained. The diversity gain due to multiple relays is shown through plots of the analytical BER, well supported by simulations. 
%
%\end{abstract}
% IEEEtran.cls defaults to using nonbold math in the Abstract.
% This preserves the distinction between vectors and scalars. However,
% if the journal you are submitting to favors bold math in the abstract,
% then you can use LaTeX's standard command \boldmath at the very start
% of the abstract to achieve this. Many IEEE journals frown on math
% in the abstract anyway.

% Note that keywords are not normally used for peerreview papers.
%\begin{IEEEkeywords}
%Cooperative diversity, decode and forward, piecewise linear
%\end{IEEEkeywords}



% For peer review papers, you can put extra information on the cover
% page as needed:
% \ifCLASSOPTIONpeerreview
% \begin{center} \bfseries EDICS Category: 3-BBND \end{center}
% \fi
%
% For peerreview papers, this IEEEtran command inserts a page break and
% creates the second title. It will be ignored for other modes.
%\IEEEpeerreviewmaketitle




\item
The king, queen and jack of clubs are removed from a deck of 52 playing cards and then well shuffled. Now one card is drawn at random from the remaining cards.  Determine the probability that the card is
\begin{enumerate}[label=(\roman*)]
\item a club
\item 10 of hearts
\end{enumerate}
\solution
%\begin{table}[H]
	\centering
\begin{tabular}{|c|c|c|}
\hline
Random variable &Value &Definition\\ \hline
\multirow{3}{*}{X} &0 &Slips of Rs 1\\
&1 &Slips of Rs 5\\
&2 &Slips of Rs 13\\ \hline
\multirow{2}{*}{Y} &0 &Box A\\
&1 &Box B\\\hline
\end{tabular}
\caption{}
\label{tab:Distribution}
\end{table}
See \tabref{tab:Distribution}.
\begin{align}
p_{Y}\brak{k}= \begin{cases} 
      \frac{1}{3} & {k=0} \\
      \frac{2}{3 }& {k=1} 
   \end{cases}
   \\
p_{Y|X}\brak{0|0} = \frac{19}{25}\, 
p_{Y|X}\brak{0|1} = \frac{6}{25}\,
p_{Y|X}\brak{1|0} = \frac{45}{50}\,
p_{Y|X}\brak{1|2} = \frac{5}{50}
\end{align}
The desired probability is the probability that a slip drawn at random is marked other than Rs 1,
\begin{align}
&=1-p_X\brak{0}\\
&= p_X(1) + p_X(2)
\end{align}
Using Bayes theorem,
\begin{align}
&= p_Y\brak{0} \times \pr{Y=0 | X=1} + p_Y\brak{1} \times \pr{Y=1|X=2}\\
&=\frac{1}{3} \times \frac{6}{25} + \frac{2}{3} \times \frac{5}{50}\\
&=\frac{11}{75}
\end{align}

\newpage

%\tableofcontents

\bigskip

\renewcommand{\thefigure}{\theenumi}
\renewcommand{\thetable}{\theenumi}
%\renewcommand{\theequation}{\theenumi}

%\begin{abstract}
%%\boldmath
%In this letter, an algorithm for evaluating the exact analytical bit error rate  (BER)  for the piecewise linear (PL) combiner for  multiple relays is presented. Previous results were available only for upto three relays. The algorithm is unique in the sense that  the actual mathematical expressions, that are prohibitively large, need not be explicitly obtained. The diversity gain due to multiple relays is shown through plots of the analytical BER, well supported by simulations. 
%
%\end{abstract}
% IEEEtran.cls defaults to using nonbold math in the Abstract.
% This preserves the distinction between vectors and scalars. However,
% if the journal you are submitting to favors bold math in the abstract,
% then you can use LaTeX's standard command \boldmath at the very start
% of the abstract to achieve this. Many IEEE journals frown on math
% in the abstract anyway.

% Note that keywords are not normally used for peerreview papers.
%\begin{IEEEkeywords}
%Cooperative diversity, decode and forward, piecewise linear
%\end{IEEEkeywords}



% For peer review papers, you can put extra information on the cover
% page as needed:
% \ifCLASSOPTIONpeerreview
% \begin{center} \bfseries EDICS Category: 3-BBND \end{center}
% \fi
%
% For peerreview papers, this IEEEtran command inserts a page break and
% creates the second title. It will be ignored for other modes.
%\IEEEpeerreviewmaketitle




\item A team of medical students doing their internship have to assist during surgeries
at a city hospital. The probabilities of surgeries rated as very complex, complex,
routine, simple or very simple are respectively, 0.15, 0.20, 0.31, 0.26, .08. Find
the probabilities that a particular surgery will be rated
\begin{enumerate}
	\item complex or very complex;
	\item neither very complex nor very simple;
	\item routine or complex
	\item routine or simple
\end{enumerate}
\solution
%\begin{table}[H]
	\centering
\begin{tabular}{|c|c|c|}
\hline
Random variable &Value &Definition\\ \hline
\multirow{3}{*}{X} &0 &Slips of Rs 1\\
&1 &Slips of Rs 5\\
&2 &Slips of Rs 13\\ \hline
\multirow{2}{*}{Y} &0 &Box A\\
&1 &Box B\\\hline
\end{tabular}
\caption{}
\label{tab:Distribution}
\end{table}
See \tabref{tab:Distribution}.
\begin{align}
p_{Y}\brak{k}= \begin{cases} 
      \frac{1}{3} & {k=0} \\
      \frac{2}{3 }& {k=1} 
   \end{cases}
   \\
p_{Y|X}\brak{0|0} = \frac{19}{25}\, 
p_{Y|X}\brak{0|1} = \frac{6}{25}\,
p_{Y|X}\brak{1|0} = \frac{45}{50}\,
p_{Y|X}\brak{1|2} = \frac{5}{50}
\end{align}
The desired probability is the probability that a slip drawn at random is marked other than Rs 1,
\begin{align}
&=1-p_X\brak{0}\\
&= p_X(1) + p_X(2)
\end{align}
Using Bayes theorem,
\begin{align}
&= p_Y\brak{0} \times \pr{Y=0 | X=1} + p_Y\brak{1} \times \pr{Y=1|X=2}\\
&=\frac{1}{3} \times \frac{6}{25} + \frac{2}{3} \times \frac{5}{50}\\
&=\frac{11}{75}
\end{align}

\newpage

%\tableofcontents

\bigskip

\renewcommand{\thefigure}{\theenumi}
\renewcommand{\thetable}{\theenumi}
%\renewcommand{\theequation}{\theenumi}

%\begin{abstract}
%%\boldmath
%In this letter, an algorithm for evaluating the exact analytical bit error rate  (BER)  for the piecewise linear (PL) combiner for  multiple relays is presented. Previous results were available only for upto three relays. The algorithm is unique in the sense that  the actual mathematical expressions, that are prohibitively large, need not be explicitly obtained. The diversity gain due to multiple relays is shown through plots of the analytical BER, well supported by simulations. 
%
%\end{abstract}
% IEEEtran.cls defaults to using nonbold math in the Abstract.
% This preserves the distinction between vectors and scalars. However,
% if the journal you are submitting to favors bold math in the abstract,
% then you can use LaTeX's standard command \boldmath at the very start
% of the abstract to achieve this. Many IEEE journals frown on math
% in the abstract anyway.

% Note that keywords are not normally used for peerreview papers.
%\begin{IEEEkeywords}
%Cooperative diversity, decode and forward, piecewise linear
%\end{IEEEkeywords}



% For peer review papers, you can put extra information on the cover
% page as needed:
% \ifCLASSOPTIONpeerreview
% \begin{center} \bfseries EDICS Category: 3-BBND \end{center}
% \fi
%
% For peerreview papers, this IEEEtran command inserts a page break and
% creates the second title. It will be ignored for other modes.
%\IEEEpeerreviewmaketitle




\item A card is selected from a pack of 52 cards.
\begin{enumerate}[label=(\alph*)]
    \item How many points are there in the sample space?
    \item Calculate the probability that the card is an ace of spades.
    \item Calculate the probability that the card is (i) an ace and (ii) black card.
\end{enumerate}
\solution
%Let $X$ be an bernoulli rv defined as in \tabref{tab:exemplar/11/16/3/26}.  Then, 
\begin{equation}
    p =
        \frac{4}{11} 
\end{equation}
\begin{table}[H]
	\centering
	\input{exemplar/11/16/3/26/tables/Table2.tex}
	\caption{}
        \label{tab:exemplar/11/16/3/26}
\end{table}

\item The probability that a non leap year selected at random will contain 53 sundays.
\\
\solution
%\begin{table}[H]
	\centering
\begin{tabular}{|c|c|c|}
\hline
Random variable &Value &Definition\\ \hline
\multirow{3}{*}{X} &0 &Slips of Rs 1\\
&1 &Slips of Rs 5\\
&2 &Slips of Rs 13\\ \hline
\multirow{2}{*}{Y} &0 &Box A\\
&1 &Box B\\\hline
\end{tabular}
\caption{}
\label{tab:Distribution}
\end{table}
See \tabref{tab:Distribution}.
\begin{align}
p_{Y}\brak{k}= \begin{cases} 
      \frac{1}{3} & {k=0} \\
      \frac{2}{3 }& {k=1} 
   \end{cases}
   \\
p_{Y|X}\brak{0|0} = \frac{19}{25}\, 
p_{Y|X}\brak{0|1} = \frac{6}{25}\,
p_{Y|X}\brak{1|0} = \frac{45}{50}\,
p_{Y|X}\brak{1|2} = \frac{5}{50}
\end{align}
The desired probability is the probability that a slip drawn at random is marked other than Rs 1,
\begin{align}
&=1-p_X\brak{0}\\
&= p_X(1) + p_X(2)
\end{align}
Using Bayes theorem,
\begin{align}
&= p_Y\brak{0} \times \pr{Y=0 | X=1} + p_Y\brak{1} \times \pr{Y=1|X=2}\\
&=\frac{1}{3} \times \frac{6}{25} + \frac{2}{3} \times \frac{5}{50}\\
&=\frac{11}{75}
\end{align}

\newpage

%\tableofcontents

\bigskip

\renewcommand{\thefigure}{\theenumi}
\renewcommand{\thetable}{\theenumi}
%\renewcommand{\theequation}{\theenumi}

%\begin{abstract}
%%\boldmath
%In this letter, an algorithm for evaluating the exact analytical bit error rate  (BER)  for the piecewise linear (PL) combiner for  multiple relays is presented. Previous results were available only for upto three relays. The algorithm is unique in the sense that  the actual mathematical expressions, that are prohibitively large, need not be explicitly obtained. The diversity gain due to multiple relays is shown through plots of the analytical BER, well supported by simulations. 
%
%\end{abstract}
% IEEEtran.cls defaults to using nonbold math in the Abstract.
% This preserves the distinction between vectors and scalars. However,
% if the journal you are submitting to favors bold math in the abstract,
% then you can use LaTeX's standard command \boldmath at the very start
% of the abstract to achieve this. Many IEEE journals frown on math
% in the abstract anyway.

% Note that keywords are not normally used for peerreview papers.
%\begin{IEEEkeywords}
%Cooperative diversity, decode and forward, piecewise linear
%\end{IEEEkeywords}



% For peer review papers, you can put extra information on the cover
% page as needed:
% \ifCLASSOPTIONpeerreview
% \begin{center} \bfseries EDICS Category: 3-BBND \end{center}
% \fi
%
% For peerreview papers, this IEEEtran command inserts a page break and
% creates the second title. It will be ignored for other modes.
%\IEEEpeerreviewmaketitle




\item One of the four persons John, Rita, Aslam or Gurpreet will be promoted next
month. Consequently the sample space consists of four elementary outcomes
S = {John promoted, Rita promoted, Aslam promoted, Gurpreet promoted}
You are told that the chances of John’s promotion is same as that of Gurpreet,
Rita’s chances of promotion are twice as likely as Johns. Aslam’s chances are
four times that of John.
\begin{enumerate}
	\item Determine
	\begin{enumerate}
		\item P (John promoted)
		\item P (Rita promoted)
		\item P (Aslam promoted)
		\item P (Gurpreet promoted)
	\end{enumerate}
	\item If A = {John promoted or Gurpreet promoted}, find P (A).
\end{enumerate}
\solution
%\begin{table}[H]
	\centering
\begin{tabular}{|c|c|c|}
\hline
Random variable &Value &Definition\\ \hline
\multirow{3}{*}{X} &0 &Slips of Rs 1\\
&1 &Slips of Rs 5\\
&2 &Slips of Rs 13\\ \hline
\multirow{2}{*}{Y} &0 &Box A\\
&1 &Box B\\\hline
\end{tabular}
\caption{}
\label{tab:Distribution}
\end{table}
See \tabref{tab:Distribution}.
\begin{align}
p_{Y}\brak{k}= \begin{cases} 
      \frac{1}{3} & {k=0} \\
      \frac{2}{3 }& {k=1} 
   \end{cases}
   \\
p_{Y|X}\brak{0|0} = \frac{19}{25}\, 
p_{Y|X}\brak{0|1} = \frac{6}{25}\,
p_{Y|X}\brak{1|0} = \frac{45}{50}\,
p_{Y|X}\brak{1|2} = \frac{5}{50}
\end{align}
The desired probability is the probability that a slip drawn at random is marked other than Rs 1,
\begin{align}
&=1-p_X\brak{0}\\
&= p_X(1) + p_X(2)
\end{align}
Using Bayes theorem,
\begin{align}
&= p_Y\brak{0} \times \pr{Y=0 | X=1} + p_Y\brak{1} \times \pr{Y=1|X=2}\\
&=\frac{1}{3} \times \frac{6}{25} + \frac{2}{3} \times \frac{5}{50}\\
&=\frac{11}{75}
\end{align}

\newpage

%\tableofcontents

\bigskip

\renewcommand{\thefigure}{\theenumi}
\renewcommand{\thetable}{\theenumi}
%\renewcommand{\theequation}{\theenumi}

%\begin{abstract}
%%\boldmath
%In this letter, an algorithm for evaluating the exact analytical bit error rate  (BER)  for the piecewise linear (PL) combiner for  multiple relays is presented. Previous results were available only for upto three relays. The algorithm is unique in the sense that  the actual mathematical expressions, that are prohibitively large, need not be explicitly obtained. The diversity gain due to multiple relays is shown through plots of the analytical BER, well supported by simulations. 
%
%\end{abstract}
% IEEEtran.cls defaults to using nonbold math in the Abstract.
% This preserves the distinction between vectors and scalars. However,
% if the journal you are submitting to favors bold math in the abstract,
% then you can use LaTeX's standard command \boldmath at the very start
% of the abstract to achieve this. Many IEEE journals frown on math
% in the abstract anyway.

% Note that keywords are not normally used for peerreview papers.
%\begin{IEEEkeywords}
%Cooperative diversity, decode and forward, piecewise linear
%\end{IEEEkeywords}



% For peer review papers, you can put extra information on the cover
% page as needed:
% \ifCLASSOPTIONpeerreview
% \begin{center} \bfseries EDICS Category: 3-BBND \end{center}
% \fi
%
% For peerreview papers, this IEEEtran command inserts a page break and
% creates the second title. It will be ignored for other modes.
%\IEEEpeerreviewmaketitle




\item A card is drawn from a deck of 52 cards. Find the probability of getting a king or a heart or a red card.\\
\solution
%\begin{table}[H]
	\centering
\begin{tabular}{|c|c|c|}
\hline
Random variable &Value &Definition\\ \hline
\multirow{3}{*}{X} &0 &Slips of Rs 1\\
&1 &Slips of Rs 5\\
&2 &Slips of Rs 13\\ \hline
\multirow{2}{*}{Y} &0 &Box A\\
&1 &Box B\\\hline
\end{tabular}
\caption{}
\label{tab:Distribution}
\end{table}
See \tabref{tab:Distribution}.
\begin{align}
p_{Y}\brak{k}= \begin{cases} 
      \frac{1}{3} & {k=0} \\
      \frac{2}{3 }& {k=1} 
   \end{cases}
   \\
p_{Y|X}\brak{0|0} = \frac{19}{25}\, 
p_{Y|X}\brak{0|1} = \frac{6}{25}\,
p_{Y|X}\brak{1|0} = \frac{45}{50}\,
p_{Y|X}\brak{1|2} = \frac{5}{50}
\end{align}
The desired probability is the probability that a slip drawn at random is marked other than Rs 1,
\begin{align}
&=1-p_X\brak{0}\\
&= p_X(1) + p_X(2)
\end{align}
Using Bayes theorem,
\begin{align}
&= p_Y\brak{0} \times \pr{Y=0 | X=1} + p_Y\brak{1} \times \pr{Y=1|X=2}\\
&=\frac{1}{3} \times \frac{6}{25} + \frac{2}{3} \times \frac{5}{50}\\
&=\frac{11}{75}
\end{align}

\newpage

%\tableofcontents

\bigskip

\renewcommand{\thefigure}{\theenumi}
\renewcommand{\thetable}{\theenumi}
%\renewcommand{\theequation}{\theenumi}

%\begin{abstract}
%%\boldmath
%In this letter, an algorithm for evaluating the exact analytical bit error rate  (BER)  for the piecewise linear (PL) combiner for  multiple relays is presented. Previous results were available only for upto three relays. The algorithm is unique in the sense that  the actual mathematical expressions, that are prohibitively large, need not be explicitly obtained. The diversity gain due to multiple relays is shown through plots of the analytical BER, well supported by simulations. 
%
%\end{abstract}
% IEEEtran.cls defaults to using nonbold math in the Abstract.
% This preserves the distinction between vectors and scalars. However,
% if the journal you are submitting to favors bold math in the abstract,
% then you can use LaTeX's standard command \boldmath at the very start
% of the abstract to achieve this. Many IEEE journals frown on math
% in the abstract anyway.

% Note that keywords are not normally used for peerreview papers.
%\begin{IEEEkeywords}
%Cooperative diversity, decode and forward, piecewise linear
%\end{IEEEkeywords}



% For peer review papers, you can put extra information on the cover
% page as needed:
% \ifCLASSOPTIONpeerreview
% \begin{center} \bfseries EDICS Category: 3-BBND \end{center}
% \fi
%
% For peerreview papers, this IEEEtran command inserts a page break and
% creates the second title. It will be ignored for other modes.
%\IEEEpeerreviewmaketitle




\item The probability that a student will pass his examination is 0.73, the probability of
the student getting a compartment is 0.13, and the probability that the student will
either pass or get compartment is 0.96. State True or False.\\
\solution
%\begin{table}[H]
	\centering
\begin{tabular}{|c|c|c|}
\hline
Random variable &Value &Definition\\ \hline
\multirow{3}{*}{X} &0 &Slips of Rs 1\\
&1 &Slips of Rs 5\\
&2 &Slips of Rs 13\\ \hline
\multirow{2}{*}{Y} &0 &Box A\\
&1 &Box B\\\hline
\end{tabular}
\caption{}
\label{tab:Distribution}
\end{table}
See \tabref{tab:Distribution}.
\begin{align}
p_{Y}\brak{k}= \begin{cases} 
      \frac{1}{3} & {k=0} \\
      \frac{2}{3 }& {k=1} 
   \end{cases}
   \\
p_{Y|X}\brak{0|0} = \frac{19}{25}\, 
p_{Y|X}\brak{0|1} = \frac{6}{25}\,
p_{Y|X}\brak{1|0} = \frac{45}{50}\,
p_{Y|X}\brak{1|2} = \frac{5}{50}
\end{align}
The desired probability is the probability that a slip drawn at random is marked other than Rs 1,
\begin{align}
&=1-p_X\brak{0}\\
&= p_X(1) + p_X(2)
\end{align}
Using Bayes theorem,
\begin{align}
&= p_Y\brak{0} \times \pr{Y=0 | X=1} + p_Y\brak{1} \times \pr{Y=1|X=2}\\
&=\frac{1}{3} \times \frac{6}{25} + \frac{2}{3} \times \frac{5}{50}\\
&=\frac{11}{75}
\end{align}

\newpage

%\tableofcontents

\bigskip

\renewcommand{\thefigure}{\theenumi}
\renewcommand{\thetable}{\theenumi}
%\renewcommand{\theequation}{\theenumi}

%\begin{abstract}
%%\boldmath
%In this letter, an algorithm for evaluating the exact analytical bit error rate  (BER)  for the piecewise linear (PL) combiner for  multiple relays is presented. Previous results were available only for upto three relays. The algorithm is unique in the sense that  the actual mathematical expressions, that are prohibitively large, need not be explicitly obtained. The diversity gain due to multiple relays is shown through plots of the analytical BER, well supported by simulations. 
%
%\end{abstract}
% IEEEtran.cls defaults to using nonbold math in the Abstract.
% This preserves the distinction between vectors and scalars. However,
% if the journal you are submitting to favors bold math in the abstract,
% then you can use LaTeX's standard command \boldmath at the very start
% of the abstract to achieve this. Many IEEE journals frown on math
% in the abstract anyway.

% Note that keywords are not normally used for peerreview papers.
%\begin{IEEEkeywords}
%Cooperative diversity, decode and forward, piecewise linear
%\end{IEEEkeywords}



% For peer review papers, you can put extra information on the cover
% page as needed:
% \ifCLASSOPTIONpeerreview
% \begin{center} \bfseries EDICS Category: 3-BBND \end{center}
% \fi
%
% For peerreview papers, this IEEEtran command inserts a page break and
% creates the second title. It will be ignored for other modes.
%\IEEEpeerreviewmaketitle




\item A card is selected from a pack of 52 cards\\
\begin{enumerate}[label=(\alph*)]
\item How many points are there in the sample space?
\item Calculate the probability that the cards is an ace of spades.
\item Calculate the probability that the card is (i) an ace (ii)black card.\\
\end{enumerate}
%\input{ncert/11/16/3/4_1/Prob_4.tex}
\item In a non-leap year, the probability of having 53 tuesdays or 53 wednesdays is\\
\solution
%A non-leap year has a total of 365 days, and a week has 7 days.\\
So it can be expressed as 
\begin{align}
365\text{days} &=52\times 7+1 \text{day}
\end{align}
$\implies$ 52 tuesdays or wednesdays\\
Random variable X denotes the days of a week
\begin{align}
p_X\brak{k}&=\frac{1}{7}; \quad \brak{1<k<7}
\end{align}
So the probability of extra day being tuesday or wednesday is
\begin{align}
p_X\brak{3}+p_X\brak{4}&=\frac{1}{7}+\frac{1}{7}=\frac{2}{7}
\end{align}



\item There are 1000 sealed envelopes in a box, 10 of them contain a cash prize of
Rs 100 each, 100 of them contain a cash prize of Rs 50 each and 200 of them
contain a cash prize of Rs 10 each and rest do not contain any cash prize. If they
are well shuffled and an envelope is picked up out, what is the probability that it
contains no cash prize?\\
\solution
%\begin{table}[H]
	\centering
\begin{tabular}{|c|c|c|}
\hline
Random variable &Value &Definition\\ \hline
\multirow{3}{*}{X} &0 &Slips of Rs 1\\
&1 &Slips of Rs 5\\
&2 &Slips of Rs 13\\ \hline
\multirow{2}{*}{Y} &0 &Box A\\
&1 &Box B\\\hline
\end{tabular}
\caption{}
\label{tab:Distribution}
\end{table}
See \tabref{tab:Distribution}.
\begin{align}
p_{Y}\brak{k}= \begin{cases} 
      \frac{1}{3} & {k=0} \\
      \frac{2}{3 }& {k=1} 
   \end{cases}
   \\
p_{Y|X}\brak{0|0} = \frac{19}{25}\, 
p_{Y|X}\brak{0|1} = \frac{6}{25}\,
p_{Y|X}\brak{1|0} = \frac{45}{50}\,
p_{Y|X}\brak{1|2} = \frac{5}{50}
\end{align}
The desired probability is the probability that a slip drawn at random is marked other than Rs 1,
\begin{align}
&=1-p_X\brak{0}\\
&= p_X(1) + p_X(2)
\end{align}
Using Bayes theorem,
\begin{align}
&= p_Y\brak{0} \times \pr{Y=0 | X=1} + p_Y\brak{1} \times \pr{Y=1|X=2}\\
&=\frac{1}{3} \times \frac{6}{25} + \frac{2}{3} \times \frac{5}{50}\\
&=\frac{11}{75}
\end{align}

\newpage

%\tableofcontents

\bigskip

\renewcommand{\thefigure}{\theenumi}
\renewcommand{\thetable}{\theenumi}
%\renewcommand{\theequation}{\theenumi}

%\begin{abstract}
%%\boldmath
%In this letter, an algorithm for evaluating the exact analytical bit error rate  (BER)  for the piecewise linear (PL) combiner for  multiple relays is presented. Previous results were available only for upto three relays. The algorithm is unique in the sense that  the actual mathematical expressions, that are prohibitively large, need not be explicitly obtained. The diversity gain due to multiple relays is shown through plots of the analytical BER, well supported by simulations. 
%
%\end{abstract}
% IEEEtran.cls defaults to using nonbold math in the Abstract.
% This preserves the distinction between vectors and scalars. However,
% if the journal you are submitting to favors bold math in the abstract,
% then you can use LaTeX's standard command \boldmath at the very start
% of the abstract to achieve this. Many IEEE journals frown on math
% in the abstract anyway.

% Note that keywords are not normally used for peerreview papers.
%\begin{IEEEkeywords}
%Cooperative diversity, decode and forward, piecewise linear
%\end{IEEEkeywords}



% For peer review papers, you can put extra information on the cover
% page as needed:
% \ifCLASSOPTIONpeerreview
% \begin{center} \bfseries EDICS Category: 3-BBND \end{center}
% \fi
%
% For peerreview papers, this IEEEtran command inserts a page break and
% creates the second title. It will be ignored for other modes.
%\IEEEpeerreviewmaketitle




\item 
A die is thrown and a card is selected at random from a deck of 52 playing cards. The probability of getting an even number on the die and a spade card.\\
\solution
%\begin{table}[H]
	\centering
\begin{tabular}{|c|c|c|}
\hline
Random variable &Value &Definition\\ \hline
\multirow{3}{*}{X} &0 &Slips of Rs 1\\
&1 &Slips of Rs 5\\
&2 &Slips of Rs 13\\ \hline
\multirow{2}{*}{Y} &0 &Box A\\
&1 &Box B\\\hline
\end{tabular}
\caption{}
\label{tab:Distribution}
\end{table}
See \tabref{tab:Distribution}.
\begin{align}
p_{Y}\brak{k}= \begin{cases} 
      \frac{1}{3} & {k=0} \\
      \frac{2}{3 }& {k=1} 
   \end{cases}
   \\
p_{Y|X}\brak{0|0} = \frac{19}{25}\, 
p_{Y|X}\brak{0|1} = \frac{6}{25}\,
p_{Y|X}\brak{1|0} = \frac{45}{50}\,
p_{Y|X}\brak{1|2} = \frac{5}{50}
\end{align}
The desired probability is the probability that a slip drawn at random is marked other than Rs 1,
\begin{align}
&=1-p_X\brak{0}\\
&= p_X(1) + p_X(2)
\end{align}
Using Bayes theorem,
\begin{align}
&= p_Y\brak{0} \times \pr{Y=0 | X=1} + p_Y\brak{1} \times \pr{Y=1|X=2}\\
&=\frac{1}{3} \times \frac{6}{25} + \frac{2}{3} \times \frac{5}{50}\\
&=\frac{11}{75}
\end{align}

\newpage

%\tableofcontents

\bigskip

\renewcommand{\thefigure}{\theenumi}
\renewcommand{\thetable}{\theenumi}
%\renewcommand{\theequation}{\theenumi}

%\begin{abstract}
%%\boldmath
%In this letter, an algorithm for evaluating the exact analytical bit error rate  (BER)  for the piecewise linear (PL) combiner for  multiple relays is presented. Previous results were available only for upto three relays. The algorithm is unique in the sense that  the actual mathematical expressions, that are prohibitively large, need not be explicitly obtained. The diversity gain due to multiple relays is shown through plots of the analytical BER, well supported by simulations. 
%
%\end{abstract}
% IEEEtran.cls defaults to using nonbold math in the Abstract.
% This preserves the distinction between vectors and scalars. However,
% if the journal you are submitting to favors bold math in the abstract,
% then you can use LaTeX's standard command \boldmath at the very start
% of the abstract to achieve this. Many IEEE journals frown on math
% in the abstract anyway.

% Note that keywords are not normally used for peerreview papers.
%\begin{IEEEkeywords}
%Cooperative diversity, decode and forward, piecewise linear
%\end{IEEEkeywords}



% For peer review papers, you can put extra information on the cover
% page as needed:
% \ifCLASSOPTIONpeerreview
% \begin{center} \bfseries EDICS Category: 3-BBND \end{center}
% \fi
%
% For peerreview papers, this IEEEtran command inserts a page break and
% creates the second title. It will be ignored for other modes.
%\IEEEpeerreviewmaketitle




\item
If 4-digit numbers greater than 5,000 are randomly formed from the digits 0, 1, 3, 5, and 7, what is the probability of forming a number divisible by 5 when:
\begin{enumerate}
    \item The digits are repeated?
    \item The repetition of digits is not allowed?
\end{enumerate}
\solution
%\begin{table}[H]
	\centering
\begin{tabular}{|c|c|c|}
\hline
Random variable &Value &Definition\\ \hline
\multirow{3}{*}{X} &0 &Slips of Rs 1\\
&1 &Slips of Rs 5\\
&2 &Slips of Rs 13\\ \hline
\multirow{2}{*}{Y} &0 &Box A\\
&1 &Box B\\\hline
\end{tabular}
\caption{}
\label{tab:Distribution}
\end{table}
See \tabref{tab:Distribution}.
\begin{align}
p_{Y}\brak{k}= \begin{cases} 
      \frac{1}{3} & {k=0} \\
      \frac{2}{3 }& {k=1} 
   \end{cases}
   \\
p_{Y|X}\brak{0|0} = \frac{19}{25}\, 
p_{Y|X}\brak{0|1} = \frac{6}{25}\,
p_{Y|X}\brak{1|0} = \frac{45}{50}\,
p_{Y|X}\brak{1|2} = \frac{5}{50}
\end{align}
The desired probability is the probability that a slip drawn at random is marked other than Rs 1,
\begin{align}
&=1-p_X\brak{0}\\
&= p_X(1) + p_X(2)
\end{align}
Using Bayes theorem,
\begin{align}
&= p_Y\brak{0} \times \pr{Y=0 | X=1} + p_Y\brak{1} \times \pr{Y=1|X=2}\\
&=\frac{1}{3} \times \frac{6}{25} + \frac{2}{3} \times \frac{5}{50}\\
&=\frac{11}{75}
\end{align}

\newpage

%\tableofcontents

\bigskip

\renewcommand{\thefigure}{\theenumi}
\renewcommand{\thetable}{\theenumi}
%\renewcommand{\theequation}{\theenumi}

%\begin{abstract}
%%\boldmath
%In this letter, an algorithm for evaluating the exact analytical bit error rate  (BER)  for the piecewise linear (PL) combiner for  multiple relays is presented. Previous results were available only for upto three relays. The algorithm is unique in the sense that  the actual mathematical expressions, that are prohibitively large, need not be explicitly obtained. The diversity gain due to multiple relays is shown through plots of the analytical BER, well supported by simulations. 
%
%\end{abstract}
% IEEEtran.cls defaults to using nonbold math in the Abstract.
% This preserves the distinction between vectors and scalars. However,
% if the journal you are submitting to favors bold math in the abstract,
% then you can use LaTeX's standard command \boldmath at the very start
% of the abstract to achieve this. Many IEEE journals frown on math
% in the abstract anyway.

% Note that keywords are not normally used for peerreview papers.
%\begin{IEEEkeywords}
%Cooperative diversity, decode and forward, piecewise linear
%\end{IEEEkeywords}



% For peer review papers, you can put extra information on the cover
% page as needed:
% \ifCLASSOPTIONpeerreview
% \begin{center} \bfseries EDICS Category: 3-BBND \end{center}
% \fi
%
% For peerreview papers, this IEEEtran command inserts a page break and
% creates the second title. It will be ignored for other modes.
%\IEEEpeerreviewmaketitle




\item Consider the probability space $\brak{\Omega, \mathcal{G}, P}$ where $\Omega = [0,2]$ and $\mathcal{G} = \cbrak{\phi, \Omega, [0,1], (1,2]}$. Let $X$ and $Y$ be two functions on $\Omega$ defined as
\begin{align*}
    X(\omega) = 
    \begin{cases}
        1 & \text{if }\omega \in [0, 1]\\
        2 & \text{if }\omega \in (1, 2]
    \end{cases}
\end{align*}
and
\begin{align*}
    Y(\omega) = 
    \begin{cases}
        2 & \text{if }\omega \in [0, 1.5]\\
        3 & \text{if }\omega \in (1.5, 2].
    \end{cases}
\end{align*}
Then which one of the following statements is true?
\begin{enumerate}
    \item [(A)] $X$ is a random variable with respect to $\mathcal{G}$, but $Y$ is not a random variable with respect to $\mathcal{G}$.
    \item [(B)] $Y$ is a random variable with respect to $\mathcal{G}$, but $X$ is not a random variable with respect to $\mathcal{G}$.
    \item [(C)] Neither $X$ nor $Y$ is a random variable with respect to $\mathcal{G}$.
    \item [(D)] Both $X$ and $Y$ are random variables with respect to $\mathcal{G}$.
\end{enumerate} \hfill (GATE ST 2023)\\
\solution
%\begin{table}[H]
	\centering
\begin{tabular}{|c|c|c|}
\hline
Random variable &Value &Definition\\ \hline
\multirow{3}{*}{X} &0 &Slips of Rs 1\\
&1 &Slips of Rs 5\\
&2 &Slips of Rs 13\\ \hline
\multirow{2}{*}{Y} &0 &Box A\\
&1 &Box B\\\hline
\end{tabular}
\caption{}
\label{tab:Distribution}
\end{table}
See \tabref{tab:Distribution}.
\begin{align}
p_{Y}\brak{k}= \begin{cases} 
      \frac{1}{3} & {k=0} \\
      \frac{2}{3 }& {k=1} 
   \end{cases}
   \\
p_{Y|X}\brak{0|0} = \frac{19}{25}\, 
p_{Y|X}\brak{0|1} = \frac{6}{25}\,
p_{Y|X}\brak{1|0} = \frac{45}{50}\,
p_{Y|X}\brak{1|2} = \frac{5}{50}
\end{align}
The desired probability is the probability that a slip drawn at random is marked other than Rs 1,
\begin{align}
&=1-p_X\brak{0}\\
&= p_X(1) + p_X(2)
\end{align}
Using Bayes theorem,
\begin{align}
&= p_Y\brak{0} \times \pr{Y=0 | X=1} + p_Y\brak{1} \times \pr{Y=1|X=2}\\
&=\frac{1}{3} \times \frac{6}{25} + \frac{2}{3} \times \frac{5}{50}\\
&=\frac{11}{75}
\end{align}

\newpage

%\tableofcontents

\bigskip

\renewcommand{\thefigure}{\theenumi}
\renewcommand{\thetable}{\theenumi}
%\renewcommand{\theequation}{\theenumi}

%\begin{abstract}
%%\boldmath
%In this letter, an algorithm for evaluating the exact analytical bit error rate  (BER)  for the piecewise linear (PL) combiner for  multiple relays is presented. Previous results were available only for upto three relays. The algorithm is unique in the sense that  the actual mathematical expressions, that are prohibitively large, need not be explicitly obtained. The diversity gain due to multiple relays is shown through plots of the analytical BER, well supported by simulations. 
%
%\end{abstract}
% IEEEtran.cls defaults to using nonbold math in the Abstract.
% This preserves the distinction between vectors and scalars. However,
% if the journal you are submitting to favors bold math in the abstract,
% then you can use LaTeX's standard command \boldmath at the very start
% of the abstract to achieve this. Many IEEE journals frown on math
% in the abstract anyway.

% Note that keywords are not normally used for peerreview papers.
%\begin{IEEEkeywords}
%Cooperative diversity, decode and forward, piecewise linear
%\end{IEEEkeywords}



% For peer review papers, you can put extra information on the cover
% page as needed:
% \ifCLASSOPTIONpeerreview
% \begin{center} \bfseries EDICS Category: 3-BBND \end{center}
% \fi
%
% For peerreview papers, this IEEEtran command inserts a page break and
% creates the second title. It will be ignored for other modes.
%\IEEEpeerreviewmaketitle




	\item  A die is loaded in such a way that each odd number is twice as likely to occur as
each even number. Find $P(G)$, where $G$ is the event that a number greater than
3 occurs on a single roll of the die.
\\
\solution
		%\begin{table}[H]
	\centering
\begin{tabular}{|c|c|c|}
\hline
Random variable &Value &Definition\\ \hline
\multirow{3}{*}{X} &0 &Slips of Rs 1\\
&1 &Slips of Rs 5\\
&2 &Slips of Rs 13\\ \hline
\multirow{2}{*}{Y} &0 &Box A\\
&1 &Box B\\\hline
\end{tabular}
\caption{}
\label{tab:Distribution}
\end{table}
See \tabref{tab:Distribution}.
\begin{align}
p_{Y}\brak{k}= \begin{cases} 
      \frac{1}{3} & {k=0} \\
      \frac{2}{3 }& {k=1} 
   \end{cases}
   \\
p_{Y|X}\brak{0|0} = \frac{19}{25}\, 
p_{Y|X}\brak{0|1} = \frac{6}{25}\,
p_{Y|X}\brak{1|0} = \frac{45}{50}\,
p_{Y|X}\brak{1|2} = \frac{5}{50}
\end{align}
The desired probability is the probability that a slip drawn at random is marked other than Rs 1,
\begin{align}
&=1-p_X\brak{0}\\
&= p_X(1) + p_X(2)
\end{align}
Using Bayes theorem,
\begin{align}
&= p_Y\brak{0} \times \pr{Y=0 | X=1} + p_Y\brak{1} \times \pr{Y=1|X=2}\\
&=\frac{1}{3} \times \frac{6}{25} + \frac{2}{3} \times \frac{5}{50}\\
&=\frac{11}{75}
\end{align}

\newpage

%\tableofcontents

\bigskip

\renewcommand{\thefigure}{\theenumi}
\renewcommand{\thetable}{\theenumi}
%\renewcommand{\theequation}{\theenumi}

%\begin{abstract}
%%\boldmath
%In this letter, an algorithm for evaluating the exact analytical bit error rate  (BER)  for the piecewise linear (PL) combiner for  multiple relays is presented. Previous results were available only for upto three relays. The algorithm is unique in the sense that  the actual mathematical expressions, that are prohibitively large, need not be explicitly obtained. The diversity gain due to multiple relays is shown through plots of the analytical BER, well supported by simulations. 
%
%\end{abstract}
% IEEEtran.cls defaults to using nonbold math in the Abstract.
% This preserves the distinction between vectors and scalars. However,
% if the journal you are submitting to favors bold math in the abstract,
% then you can use LaTeX's standard command \boldmath at the very start
% of the abstract to achieve this. Many IEEE journals frown on math
% in the abstract anyway.

% Note that keywords are not normally used for peerreview papers.
%\begin{IEEEkeywords}
%Cooperative diversity, decode and forward, piecewise linear
%\end{IEEEkeywords}



% For peer review papers, you can put extra information on the cover
% page as needed:
% \ifCLASSOPTIONpeerreview
% \begin{center} \bfseries EDICS Category: 3-BBND \end{center}
% \fi
%
% For peerreview papers, this IEEEtran command inserts a page break and
% creates the second title. It will be ignored for other modes.
%\IEEEpeerreviewmaketitle




	\item All the jacks, queens and kings are removed from a deck of 52 playing cards. The remaining cards are well shuffled and then one card is drawn at random. Giving ace a value 1 similar value for other cards, find the probability that the card has a value 
		\begin{enumerate}
			\item 7
			\item greater than 7
			\item less than 7
		\end{enumerate}
		%Number of cards left after removing all jacks, queens and kings 
\begin{align}
N	= 52 - 4\times 3
	= 40
\end{align}
%\begin{table}[H]
%\def\arraystretch{1.2}
%\begin{tabular}{|c|c|c|}
%\hline
%	\textbf{Parameter} &\textbf{Value} &\textbf{Description}\\ \hline
%	$X$ &1-10 &Represents the value of the card picked \\ \hline
%\end{tabular}
%\end{table}
Let $1 \le X \le 10$ be the value of the card picked.  Then,
\begin{align}
	p_X(k) &= \Pr(X=k)\ \forall\ 1 \leq k \leq 10\\
	&= \frac{4\times 1}{40}\\
	&= \frac{1}{10}\\
	\therefore p_X(k) &= 
	\begin{cases}
		\frac{1}{10} & 1 \leq k \leq 10\\
		0 & \text{otherwise}
	\end{cases}
\end{align}
and
\begin{align}
	F_{X}(k) &= \sum_{m=0}^{k}p_{X}(m) \quad 1 \leq k \leq 10\\
	&= \frac{k}{10}\\
	\therefore F_{X}(k) &= 
	\begin{cases}
		0 & k \leq 0\\
		\frac{k}{10} & 1\leq k \leq 10\\
		1 & k > 10 
	\end{cases}
\end{align}
\begin{enumerate}
	\item Probability that card has value equal to 7 is
		\begin{align}
			 p_{X}(7)
			= \frac{1}{10}
		\end{align}
	\item Probability that card has value greater than 7 is
		\begin{align}
			1 - F_X(7)
			&= 1 - \frac{7}{10}
			\\
			&= \frac{3}{10}
		\end{align}
	\item Probability that card has value less than 7 is
		\begin{align}
			 F_{X}(6)
			=\frac{6}{10}
		\end{align}
\end{enumerate}

  \item A Lot consists of 48 mobile phones of which 42 are good, 3 have only minor defects and 3 have major defects.Varnika will buy a phone if it is good but the trader will only buy a mobile if it has no major defects. One phone is selected at random from the lot. What is the probability that it is
\begin{enumerate}
	\item acceptable to Varnika?
            \item acceptable to the trader?
\end{enumerate}
\solution
	%\begin{table}[H]
	\centering
\begin{tabular}{|c|c|c|}
\hline
Random variable &Value &Definition\\ \hline
\multirow{3}{*}{X} &0 &Slips of Rs 1\\
&1 &Slips of Rs 5\\
&2 &Slips of Rs 13\\ \hline
\multirow{2}{*}{Y} &0 &Box A\\
&1 &Box B\\\hline
\end{tabular}
\caption{}
\label{tab:Distribution}
\end{table}
See \tabref{tab:Distribution}.
\begin{align}
p_{Y}\brak{k}= \begin{cases} 
      \frac{1}{3} & {k=0} \\
      \frac{2}{3 }& {k=1} 
   \end{cases}
   \\
p_{Y|X}\brak{0|0} = \frac{19}{25}\, 
p_{Y|X}\brak{0|1} = \frac{6}{25}\,
p_{Y|X}\brak{1|0} = \frac{45}{50}\,
p_{Y|X}\brak{1|2} = \frac{5}{50}
\end{align}
The desired probability is the probability that a slip drawn at random is marked other than Rs 1,
\begin{align}
&=1-p_X\brak{0}\\
&= p_X(1) + p_X(2)
\end{align}
Using Bayes theorem,
\begin{align}
&= p_Y\brak{0} \times \pr{Y=0 | X=1} + p_Y\brak{1} \times \pr{Y=1|X=2}\\
&=\frac{1}{3} \times \frac{6}{25} + \frac{2}{3} \times \frac{5}{50}\\
&=\frac{11}{75}
\end{align}

\newpage

%\tableofcontents

\bigskip

\renewcommand{\thefigure}{\theenumi}
\renewcommand{\thetable}{\theenumi}
%\renewcommand{\theequation}{\theenumi}

%\begin{abstract}
%%\boldmath
%In this letter, an algorithm for evaluating the exact analytical bit error rate  (BER)  for the piecewise linear (PL) combiner for  multiple relays is presented. Previous results were available only for upto three relays. The algorithm is unique in the sense that  the actual mathematical expressions, that are prohibitively large, need not be explicitly obtained. The diversity gain due to multiple relays is shown through plots of the analytical BER, well supported by simulations. 
%
%\end{abstract}
% IEEEtran.cls defaults to using nonbold math in the Abstract.
% This preserves the distinction between vectors and scalars. However,
% if the journal you are submitting to favors bold math in the abstract,
% then you can use LaTeX's standard command \boldmath at the very start
% of the abstract to achieve this. Many IEEE journals frown on math
% in the abstract anyway.

% Note that keywords are not normally used for peerreview papers.
%\begin{IEEEkeywords}
%Cooperative diversity, decode and forward, piecewise linear
%\end{IEEEkeywords}



% For peer review papers, you can put extra information on the cover
% page as needed:
% \ifCLASSOPTIONpeerreview
% \begin{center} \bfseries EDICS Category: 3-BBND \end{center}
% \fi
%
% For peerreview papers, this IEEEtran command inserts a page break and
% creates the second title. It will be ignored for other modes.
%\IEEEpeerreviewmaketitle




 \item A student says that if you throw a die, it will show up 1 or not 1. Therefore, the probability of getting 1 and the probability of getting 'not 1' each is equal to $\frac{1}{2}$. Is this correct? Give reasons.\\
 \solution
        %\begin{table}[H]
	\centering
\begin{tabular}{|c|c|c|}
\hline
Random variable &Value &Definition\\ \hline
\multirow{3}{*}{X} &0 &Slips of Rs 1\\
&1 &Slips of Rs 5\\
&2 &Slips of Rs 13\\ \hline
\multirow{2}{*}{Y} &0 &Box A\\
&1 &Box B\\\hline
\end{tabular}
\caption{}
\label{tab:Distribution}
\end{table}
See \tabref{tab:Distribution}.
\begin{align}
p_{Y}\brak{k}= \begin{cases} 
      \frac{1}{3} & {k=0} \\
      \frac{2}{3 }& {k=1} 
   \end{cases}
   \\
p_{Y|X}\brak{0|0} = \frac{19}{25}\, 
p_{Y|X}\brak{0|1} = \frac{6}{25}\,
p_{Y|X}\brak{1|0} = \frac{45}{50}\,
p_{Y|X}\brak{1|2} = \frac{5}{50}
\end{align}
The desired probability is the probability that a slip drawn at random is marked other than Rs 1,
\begin{align}
&=1-p_X\brak{0}\\
&= p_X(1) + p_X(2)
\end{align}
Using Bayes theorem,
\begin{align}
&= p_Y\brak{0} \times \pr{Y=0 | X=1} + p_Y\brak{1} \times \pr{Y=1|X=2}\\
&=\frac{1}{3} \times \frac{6}{25} + \frac{2}{3} \times \frac{5}{50}\\
&=\frac{11}{75}
\end{align}

\newpage

%\tableofcontents

\bigskip

\renewcommand{\thefigure}{\theenumi}
\renewcommand{\thetable}{\theenumi}
%\renewcommand{\theequation}{\theenumi}

%\begin{abstract}
%%\boldmath
%In this letter, an algorithm for evaluating the exact analytical bit error rate  (BER)  for the piecewise linear (PL) combiner for  multiple relays is presented. Previous results were available only for upto three relays. The algorithm is unique in the sense that  the actual mathematical expressions, that are prohibitively large, need not be explicitly obtained. The diversity gain due to multiple relays is shown through plots of the analytical BER, well supported by simulations. 
%
%\end{abstract}
% IEEEtran.cls defaults to using nonbold math in the Abstract.
% This preserves the distinction between vectors and scalars. However,
% if the journal you are submitting to favors bold math in the abstract,
% then you can use LaTeX's standard command \boldmath at the very start
% of the abstract to achieve this. Many IEEE journals frown on math
% in the abstract anyway.

% Note that keywords are not normally used for peerreview papers.
%\begin{IEEEkeywords}
%Cooperative diversity, decode and forward, piecewise linear
%\end{IEEEkeywords}



% For peer review papers, you can put extra information on the cover
% page as needed:
% \ifCLASSOPTIONpeerreview
% \begin{center} \bfseries EDICS Category: 3-BBND \end{center}
% \fi
%
% For peerreview papers, this IEEEtran command inserts a page break and
% creates the second title. It will be ignored for other modes.
%\IEEEpeerreviewmaketitle




   \item Four candidates A, B, C, D have ap-
plied for the assignment to coach a school cricket
team. If A is twice as likely to be selected as B, and
B and C are given about the same chance of being
selected, while C is twice as likely to be selected
as D, what are the probabilities that
\begin{enumerate}
\item C will be selected?
\item A will not be selected?
\end{enumerate}
	%\begin{table}[H]
	\centering
\begin{tabular}{|c|c|c|}
\hline
Random variable &Value &Definition\\ \hline
\multirow{3}{*}{X} &0 &Slips of Rs 1\\
&1 &Slips of Rs 5\\
&2 &Slips of Rs 13\\ \hline
\multirow{2}{*}{Y} &0 &Box A\\
&1 &Box B\\\hline
\end{tabular}
\caption{}
\label{tab:Distribution}
\end{table}
See \tabref{tab:Distribution}.
\begin{align}
p_{Y}\brak{k}= \begin{cases} 
      \frac{1}{3} & {k=0} \\
      \frac{2}{3 }& {k=1} 
   \end{cases}
   \\
p_{Y|X}\brak{0|0} = \frac{19}{25}\, 
p_{Y|X}\brak{0|1} = \frac{6}{25}\,
p_{Y|X}\brak{1|0} = \frac{45}{50}\,
p_{Y|X}\brak{1|2} = \frac{5}{50}
\end{align}
The desired probability is the probability that a slip drawn at random is marked other than Rs 1,
\begin{align}
&=1-p_X\brak{0}\\
&= p_X(1) + p_X(2)
\end{align}
Using Bayes theorem,
\begin{align}
&= p_Y\brak{0} \times \pr{Y=0 | X=1} + p_Y\brak{1} \times \pr{Y=1|X=2}\\
&=\frac{1}{3} \times \frac{6}{25} + \frac{2}{3} \times \frac{5}{50}\\
&=\frac{11}{75}
\end{align}

\newpage

%\tableofcontents

\bigskip

\renewcommand{\thefigure}{\theenumi}
\renewcommand{\thetable}{\theenumi}
%\renewcommand{\theequation}{\theenumi}

%\begin{abstract}
%%\boldmath
%In this letter, an algorithm for evaluating the exact analytical bit error rate  (BER)  for the piecewise linear (PL) combiner for  multiple relays is presented. Previous results were available only for upto three relays. The algorithm is unique in the sense that  the actual mathematical expressions, that are prohibitively large, need not be explicitly obtained. The diversity gain due to multiple relays is shown through plots of the analytical BER, well supported by simulations. 
%
%\end{abstract}
% IEEEtran.cls defaults to using nonbold math in the Abstract.
% This preserves the distinction between vectors and scalars. However,
% if the journal you are submitting to favors bold math in the abstract,
% then you can use LaTeX's standard command \boldmath at the very start
% of the abstract to achieve this. Many IEEE journals frown on math
% in the abstract anyway.

% Note that keywords are not normally used for peerreview papers.
%\begin{IEEEkeywords}
%Cooperative diversity, decode and forward, piecewise linear
%\end{IEEEkeywords}



% For peer review papers, you can put extra information on the cover
% page as needed:
% \ifCLASSOPTIONpeerreview
% \begin{center} \bfseries EDICS Category: 3-BBND \end{center}
% \fi
%
% For peerreview papers, this IEEEtran command inserts a page break and
% creates the second title. It will be ignored for other modes.
%\IEEEpeerreviewmaketitle




 \item A bag contain 24 balls of which $x$ balls are red, $2x$ are white and $3x$ are blue. A ball is selected at random, What is the probability that it is
\begin{enumerate}[label=\alph*)]
\item not red ?
\item white ?
\end{enumerate}
%\begin{table}[H]
	\centering
\begin{tabular}{|c|c|c|}
\hline
Random variable &Value &Definition\\ \hline
\multirow{3}{*}{X} &0 &Slips of Rs 1\\
&1 &Slips of Rs 5\\
&2 &Slips of Rs 13\\ \hline
\multirow{2}{*}{Y} &0 &Box A\\
&1 &Box B\\\hline
\end{tabular}
\caption{}
\label{tab:Distribution}
\end{table}
See \tabref{tab:Distribution}.
\begin{align}
p_{Y}\brak{k}= \begin{cases} 
      \frac{1}{3} & {k=0} \\
      \frac{2}{3 }& {k=1} 
   \end{cases}
   \\
p_{Y|X}\brak{0|0} = \frac{19}{25}\, 
p_{Y|X}\brak{0|1} = \frac{6}{25}\,
p_{Y|X}\brak{1|0} = \frac{45}{50}\,
p_{Y|X}\brak{1|2} = \frac{5}{50}
\end{align}
The desired probability is the probability that a slip drawn at random is marked other than Rs 1,
\begin{align}
&=1-p_X\brak{0}\\
&= p_X(1) + p_X(2)
\end{align}
Using Bayes theorem,
\begin{align}
&= p_Y\brak{0} \times \pr{Y=0 | X=1} + p_Y\brak{1} \times \pr{Y=1|X=2}\\
&=\frac{1}{3} \times \frac{6}{25} + \frac{2}{3} \times \frac{5}{50}\\
&=\frac{11}{75}
\end{align}

\newpage

%\tableofcontents

\bigskip

\renewcommand{\thefigure}{\theenumi}
\renewcommand{\thetable}{\theenumi}
%\renewcommand{\theequation}{\theenumi}

%\begin{abstract}
%%\boldmath
%In this letter, an algorithm for evaluating the exact analytical bit error rate  (BER)  for the piecewise linear (PL) combiner for  multiple relays is presented. Previous results were available only for upto three relays. The algorithm is unique in the sense that  the actual mathematical expressions, that are prohibitively large, need not be explicitly obtained. The diversity gain due to multiple relays is shown through plots of the analytical BER, well supported by simulations. 
%
%\end{abstract}
% IEEEtran.cls defaults to using nonbold math in the Abstract.
% This preserves the distinction between vectors and scalars. However,
% if the journal you are submitting to favors bold math in the abstract,
% then you can use LaTeX's standard command \boldmath at the very start
% of the abstract to achieve this. Many IEEE journals frown on math
% in the abstract anyway.

% Note that keywords are not normally used for peerreview papers.
%\begin{IEEEkeywords}
%Cooperative diversity, decode and forward, piecewise linear
%\end{IEEEkeywords}



% For peer review papers, you can put extra information on the cover
% page as needed:
% \ifCLASSOPTIONpeerreview
% \begin{center} \bfseries EDICS Category: 3-BBND \end{center}
% \fi
%
% For peerreview papers, this IEEEtran command inserts a page break and
% creates the second title. It will be ignored for other modes.
%\IEEEpeerreviewmaketitle




If the letters of the word ASSASSINATION are arranged at random. Find the Probability that
\begin{enumerate}[label=(\alph*)]
\item Four $S's$ come consecutively in the word
\item Two  $I's$ and two $N's$ come together
\item All $A's$ are not coming together
\item No two $A's$ are coming together
\end{enumerate}
%\begin{table}[H]
	\centering
\begin{tabular}{|c|c|c|}
\hline
Random variable &Value &Definition\\ \hline
\multirow{3}{*}{X} &0 &Slips of Rs 1\\
&1 &Slips of Rs 5\\
&2 &Slips of Rs 13\\ \hline
\multirow{2}{*}{Y} &0 &Box A\\
&1 &Box B\\\hline
\end{tabular}
\caption{}
\label{tab:Distribution}
\end{table}
See \tabref{tab:Distribution}.
\begin{align}
p_{Y}\brak{k}= \begin{cases} 
      \frac{1}{3} & {k=0} \\
      \frac{2}{3 }& {k=1} 
   \end{cases}
   \\
p_{Y|X}\brak{0|0} = \frac{19}{25}\, 
p_{Y|X}\brak{0|1} = \frac{6}{25}\,
p_{Y|X}\brak{1|0} = \frac{45}{50}\,
p_{Y|X}\brak{1|2} = \frac{5}{50}
\end{align}
The desired probability is the probability that a slip drawn at random is marked other than Rs 1,
\begin{align}
&=1-p_X\brak{0}\\
&= p_X(1) + p_X(2)
\end{align}
Using Bayes theorem,
\begin{align}
&= p_Y\brak{0} \times \pr{Y=0 | X=1} + p_Y\brak{1} \times \pr{Y=1|X=2}\\
&=\frac{1}{3} \times \frac{6}{25} + \frac{2}{3} \times \frac{5}{50}\\
&=\frac{11}{75}
\end{align}

\newpage

%\tableofcontents

\bigskip

\renewcommand{\thefigure}{\theenumi}
\renewcommand{\thetable}{\theenumi}
%\renewcommand{\theequation}{\theenumi}

%\begin{abstract}
%%\boldmath
%In this letter, an algorithm for evaluating the exact analytical bit error rate  (BER)  for the piecewise linear (PL) combiner for  multiple relays is presented. Previous results were available only for upto three relays. The algorithm is unique in the sense that  the actual mathematical expressions, that are prohibitively large, need not be explicitly obtained. The diversity gain due to multiple relays is shown through plots of the analytical BER, well supported by simulations. 
%
%\end{abstract}
% IEEEtran.cls defaults to using nonbold math in the Abstract.
% This preserves the distinction between vectors and scalars. However,
% if the journal you are submitting to favors bold math in the abstract,
% then you can use LaTeX's standard command \boldmath at the very start
% of the abstract to achieve this. Many IEEE journals frown on math
% in the abstract anyway.

% Note that keywords are not normally used for peerreview papers.
%\begin{IEEEkeywords}
%Cooperative diversity, decode and forward, piecewise linear
%\end{IEEEkeywords}



% For peer review papers, you can put extra information on the cover
% page as needed:
% \ifCLASSOPTIONpeerreview
% \begin{center} \bfseries EDICS Category: 3-BBND \end{center}
% \fi
%
% For peerreview papers, this IEEEtran command inserts a page break and
% creates the second title. It will be ignored for other modes.
%\IEEEpeerreviewmaketitle




	\item One urn contains two black balls (labelled B1 and B2) and one white ball. A
	second urn contains one black ball and two white balls (labelled W1 and W2).
	Suppose the following experiment is performed. One of the two urns is chosen
	at random. Next a ball is randomly chosen from the urn. Then a second ball is
	chosen at random from the same urn without replacing the first ball.
	
	\begin{enumerate}
	\item What is the probability that two black balls are chosen?
	
	\item What is the probability that two balls of opposite colour are chosen?
	\end{enumerate}
	\solution
	%\begin{align}
    \label{eq:12.13.6.18.1}
	\because	\pr{A|B} &> \pr{A},\
\frac{\pr{AB}}{\pr{B}} > \pr{A}
\\
    \label{eq:12.13.6.18.2}
	\implies \pr{AB} &> \pr{A}\pr{B}
	\\
	\text{or, } \frac{\pr{AB}}{\pr{A}} &=\pr{B|A} > \pr{A}
\end{align}

\end{enumerate}

		\item A box of oranges is inspected by examining three randomly selected oranges drawn without replacement. If all the three oranges are good, the box is approved for sale, otherwise, it is rejected. Find the probability that a box containing 15 oranges out of which 12 are good and 3 are bad ones will be approved for sale.
		\label{ncert/12/13/2/3/defs.tex}
		\item Two balls are drawn at random with replacement from a box containing 10 black and 8 red balls. Find the probability that
		\label{ncert/12/13/2/12}
\begin{enumerate}
\item both balls are red.
\item first ball is black and second is red.
\item one of them is black and other is red.
\end{enumerate}

\item In a hostel, 60\% of the students read Hindi newspaper, 40\% read English newspaper and 20\% read both Hindi and English newspapers. A student is selected at random.
		\label{ncert/12/13/2/15}
\begin{enumerate}
\item Find the probability that she reads neither Hindi nor English newspapers.
\item If she reads Hindi newspaper, find the probability that she reads English newspaper.
\item If she reads English newspaper, find the probability that she reads Hindi newspaper.\\
\end{enumerate}
\item The probability of obtaining an even prime number on each die, when a pair of dice is rolled is 
\begin{enumerate}
    \item $0$ 
    
    \item $\frac{1}{3}$ 
    
    \item $\frac{1}{12}$ 
    
    \item $\frac{1}{36}$ 
\end{enumerate}
\solution
		%\begin{enumerate}[label=\thesection.\arabic*,ref=\thesection.\theenumi]
	\item One card is drawn from a well-shuffled deck of 52 cards. Find the probability of getting
\begin{enumerate}
\item A king of red colour 
\item A face card 
\item A red face card
\item The jack of hearts
\item A spade
\item The queen of diamonds

\end{enumerate}
\solution
		%\begin{table}[H]
	\centering
\begin{tabular}{|c|c|c|}
\hline
Random variable &Value &Definition\\ \hline
\multirow{3}{*}{X} &0 &Slips of Rs 1\\
&1 &Slips of Rs 5\\
&2 &Slips of Rs 13\\ \hline
\multirow{2}{*}{Y} &0 &Box A\\
&1 &Box B\\\hline
\end{tabular}
\caption{}
\label{tab:Distribution}
\end{table}
See \tabref{tab:Distribution}.
\begin{align}
p_{Y}\brak{k}= \begin{cases} 
      \frac{1}{3} & {k=0} \\
      \frac{2}{3 }& {k=1} 
   \end{cases}
   \\
p_{Y|X}\brak{0|0} = \frac{19}{25}\, 
p_{Y|X}\brak{0|1} = \frac{6}{25}\,
p_{Y|X}\brak{1|0} = \frac{45}{50}\,
p_{Y|X}\brak{1|2} = \frac{5}{50}
\end{align}
The desired probability is the probability that a slip drawn at random is marked other than Rs 1,
\begin{align}
&=1-p_X\brak{0}\\
&= p_X(1) + p_X(2)
\end{align}
Using Bayes theorem,
\begin{align}
&= p_Y\brak{0} \times \pr{Y=0 | X=1} + p_Y\brak{1} \times \pr{Y=1|X=2}\\
&=\frac{1}{3} \times \frac{6}{25} + \frac{2}{3} \times \frac{5}{50}\\
&=\frac{11}{75}
\end{align}

\newpage

%\tableofcontents

\bigskip

\renewcommand{\thefigure}{\theenumi}
\renewcommand{\thetable}{\theenumi}
%\renewcommand{\theequation}{\theenumi}

%\begin{abstract}
%%\boldmath
%In this letter, an algorithm for evaluating the exact analytical bit error rate  (BER)  for the piecewise linear (PL) combiner for  multiple relays is presented. Previous results were available only for upto three relays. The algorithm is unique in the sense that  the actual mathematical expressions, that are prohibitively large, need not be explicitly obtained. The diversity gain due to multiple relays is shown through plots of the analytical BER, well supported by simulations. 
%
%\end{abstract}
% IEEEtran.cls defaults to using nonbold math in the Abstract.
% This preserves the distinction between vectors and scalars. However,
% if the journal you are submitting to favors bold math in the abstract,
% then you can use LaTeX's standard command \boldmath at the very start
% of the abstract to achieve this. Many IEEE journals frown on math
% in the abstract anyway.

% Note that keywords are not normally used for peerreview papers.
%\begin{IEEEkeywords}
%Cooperative diversity, decode and forward, piecewise linear
%\end{IEEEkeywords}



% For peer review papers, you can put extra information on the cover
% page as needed:
% \ifCLASSOPTIONpeerreview
% \begin{center} \bfseries EDICS Category: 3-BBND \end{center}
% \fi
%
% For peerreview papers, this IEEEtran command inserts a page break and
% creates the second title. It will be ignored for other modes.
%\IEEEpeerreviewmaketitle




	\item Five cards—the ten, jack, queen, king and ace of diamonds, are well-shuffled with their face downwards. One card is then picked up at random.
\begin{enumerate}
\item
What is the probability that the card is the queen? 
\item
If the queen is drawn and put aside, what is the probability that the second card picked up is (a) an ace? (b) a queen?\\
\end{enumerate}
\solution
		%\begin{enumerate}[label=\thesection.\arabic*,ref=\thesection.\theenumi]
	\item One card is drawn from a well-shuffled deck of 52 cards. Find the probability of getting
\begin{enumerate}
\item A king of red colour 
\item A face card 
\item A red face card
\item The jack of hearts
\item A spade
\item The queen of diamonds

\end{enumerate}
\solution
		%\input{ncert/10/15/1/14/main.tex}
	\item Five cards—the ten, jack, queen, king and ace of diamonds, are well-shuffled with their face downwards. One card is then picked up at random.
\begin{enumerate}
\item
What is the probability that the card is the queen? 
\item
If the queen is drawn and put aside, what is the probability that the second card picked up is (a) an ace? (b) a queen?\\
\end{enumerate}
\solution
		%\input{ncert/10/15/1/15/defs.tex}
	\item A bag contains $5$ red balls and some blue balls. If the probability of drawing a blue ball is double that if a red ball, determine the number of blue balls in the bag. 
		\\
\solution
		%\input{ncert/10/15/2/3/defs.tex}
	\item A card is selected from a pack of 52 cards.
 \begin{enumerate}[label=(\alph*)] 
                 \item How many points are there in the sample space?
                 \item Calculate the probability that the card is an ace of spades.
                 \item Calculate the probability that the card is (i) an ace and (ii) black card.
 \end{enumerate}
\solution
		%\input{ncert/11/16/3/4/main.tex}
\item Four cards are drawn from a well-shuffled deck of 52 cards. What is the probability of obtaining 3 diamonds and one spade.
\\
\solution
		%\input{ncert/11/16/4/2/defs.tex}
\item In a certain lottery 10,000 tickets are sold and ten equal prizes are awarded. What is the probability of not getting a prize if you buy (a) one ticket (b) two tickets (c) 10 tickets ?	
\\
\solution
		%\input{ncert/11/16/4/4/defs.tex}
		%
\item 
Out of 100 students, two sections of 40 and 60 are formed. If you and your friend are among the 100 students, what is the probability that
\begin{enumerate}
\item you both enter the same section?
\item you both enter the different sections?
\end{enumerate}
\solution
		%\input{ncert/11/16/4/5/defs.tex}
	\item 
The number lock of a suitcase has 4 wheels each labelled with ten digits i.e. from 0 to 9.The lock opens with a sequence of four digits with no repeats.What is the probability of a person getting the right sequence to open the suitcase.
\\
\solution
		%\input{ncert/11/16/4/10/defs.tex}
		%
\item 
Two cards are drawn at random and without replacement from a pack of 52 playing cards. Find the probability that both the cards are black.
\\
\solution
		%\input{ncert/12/13/2/2/defs.tex}
		\item A box of oranges is inspected by examining three randomly selected oranges drawn without replacement. If all the three oranges are good, the box is approved for sale, otherwise, it is rejected. Find the probability that a box containing 15 oranges out of which 12 are good and 3 are bad ones will be approved for sale.
		\label{ncert/12/13/2/3/defs.tex}
		\item Two balls are drawn at random with replacement from a box containing 10 black and 8 red balls. Find the probability that
		\label{ncert/12/13/2/12}
\begin{enumerate}
\item both balls are red.
\item first ball is black and second is red.
\item one of them is black and other is red.
\end{enumerate}

\item In a hostel, 60\% of the students read Hindi newspaper, 40\% read English newspaper and 20\% read both Hindi and English newspapers. A student is selected at random.
		\label{ncert/12/13/2/15}
\begin{enumerate}
\item Find the probability that she reads neither Hindi nor English newspapers.
\item If she reads Hindi newspaper, find the probability that she reads English newspaper.
\item If she reads English newspaper, find the probability that she reads Hindi newspaper.\\
\end{enumerate}
\item The probability of obtaining an even prime number on each die, when a pair of dice is rolled is 
\begin{enumerate}
    \item $0$ 
    
    \item $\frac{1}{3}$ 
    
    \item $\frac{1}{12}$ 
    
    \item $\frac{1}{36}$ 
\end{enumerate}
\solution
		%\input{ncert/12/13/2/17/defs.tex}
	\item A bag contains 4 red and 4 black balls, another bag contains 2 red and 6 black balls. One of the two bags is selected at random and a ball is drawn from the bag which is found to be red. Find the probability that the ball is drawn from the first bag.
\\
\solution
		%\input{ncert/12/13/3/2/main.tex}
  \item
  Cards with numbers 2 to 101 are placed in a box. A card is selected at random.Find the probability that the card has
\begin{enumerate}[label=(\roman*)]
	\item an even number 
	\item a square number
\end{enumerate}
\solution
%\input{exemplar/10/13/3/32/main.tex}
\item
The king, queen and jack of clubs are removed from a deck of 52 playing cards and then well shuffled. Now one card is drawn at random from the remaining cards.  Determine the probability that the card is
\begin{enumerate}[label=(\roman*)]
\item a club
\item 10 of hearts
\end{enumerate}
\solution
%\input{exemplar/10/13/3/29/main.tex}
\item A team of medical students doing their internship have to assist during surgeries
at a city hospital. The probabilities of surgeries rated as very complex, complex,
routine, simple or very simple are respectively, 0.15, 0.20, 0.31, 0.26, .08. Find
the probabilities that a particular surgery will be rated
\begin{enumerate}
	\item complex or very complex;
	\item neither very complex nor very simple;
	\item routine or complex
	\item routine or simple
\end{enumerate}
\solution
%\input{exemplar/11/16/3/8(1)/main.tex}
\item A card is selected from a pack of 52 cards.
\begin{enumerate}[label=(\alph*)]
    \item How many points are there in the sample space?
    \item Calculate the probability that the card is an ace of spades.
    \item Calculate the probability that the card is (i) an ace and (ii) black card.
\end{enumerate}
\solution
%\input{exemplar/11/16/3/4/main2.tex}
\item The probability that a non leap year selected at random will contain 53 sundays.
\\
\solution
%\input{exemplar/10/13/1/19/main.tex}
\item One of the four persons John, Rita, Aslam or Gurpreet will be promoted next
month. Consequently the sample space consists of four elementary outcomes
S = {John promoted, Rita promoted, Aslam promoted, Gurpreet promoted}
You are told that the chances of John’s promotion is same as that of Gurpreet,
Rita’s chances of promotion are twice as likely as Johns. Aslam’s chances are
four times that of John.
\begin{enumerate}
	\item Determine
	\begin{enumerate}
		\item P (John promoted)
		\item P (Rita promoted)
		\item P (Aslam promoted)
		\item P (Gurpreet promoted)
	\end{enumerate}
	\item If A = {John promoted or Gurpreet promoted}, find P (A).
\end{enumerate}
\solution
%\input{exemplar/11/16/3/10/main.tex}
\item A card is drawn from a deck of 52 cards. Find the probability of getting a king or a heart or a red card.\\
\solution
%\input{exemplar/11/16/3/15/main.tex}
\item The probability that a student will pass his examination is 0.73, the probability of
the student getting a compartment is 0.13, and the probability that the student will
either pass or get compartment is 0.96. State True or False.\\
\solution
%\input{exemplar/11/16/3/31/main.tex}
\item A card is selected from a pack of 52 cards\\
\begin{enumerate}[label=(\alph*)]
\item How many points are there in the sample space?
\item Calculate the probability that the cards is an ace of spades.
\item Calculate the probability that the card is (i) an ace (ii)black card.\\
\end{enumerate}
%\input{ncert/11/16/3/4_1/Prob_4.tex}
\item In a non-leap year, the probability of having 53 tuesdays or 53 wednesdays is\\
\solution
%\input{exemplar/11/16/3/18/main.tex}
\item There are 1000 sealed envelopes in a box, 10 of them contain a cash prize of
Rs 100 each, 100 of them contain a cash prize of Rs 50 each and 200 of them
contain a cash prize of Rs 10 each and rest do not contain any cash prize. If they
are well shuffled and an envelope is picked up out, what is the probability that it
contains no cash prize?\\
\solution
%\input{exemplar/10/13/3/34/main.tex}
\item 
A die is thrown and a card is selected at random from a deck of 52 playing cards. The probability of getting an even number on the die and a spade card.\\
\solution
%\input{exemplar/12/13/3/78/main.tex}
\item
If 4-digit numbers greater than 5,000 are randomly formed from the digits 0, 1, 3, 5, and 7, what is the probability of forming a number divisible by 5 when:
\begin{enumerate}
    \item The digits are repeated?
    \item The repetition of digits is not allowed?
\end{enumerate}
\solution
%\input{ncert/11/16/4/9/main.tex}
\item Consider the probability space $\brak{\Omega, \mathcal{G}, P}$ where $\Omega = [0,2]$ and $\mathcal{G} = \cbrak{\phi, \Omega, [0,1], (1,2]}$. Let $X$ and $Y$ be two functions on $\Omega$ defined as
\begin{align*}
    X(\omega) = 
    \begin{cases}
        1 & \text{if }\omega \in [0, 1]\\
        2 & \text{if }\omega \in (1, 2]
    \end{cases}
\end{align*}
and
\begin{align*}
    Y(\omega) = 
    \begin{cases}
        2 & \text{if }\omega \in [0, 1.5]\\
        3 & \text{if }\omega \in (1.5, 2].
    \end{cases}
\end{align*}
Then which one of the following statements is true?
\begin{enumerate}
    \item [(A)] $X$ is a random variable with respect to $\mathcal{G}$, but $Y$ is not a random variable with respect to $\mathcal{G}$.
    \item [(B)] $Y$ is a random variable with respect to $\mathcal{G}$, but $X$ is not a random variable with respect to $\mathcal{G}$.
    \item [(C)] Neither $X$ nor $Y$ is a random variable with respect to $\mathcal{G}$.
    \item [(D)] Both $X$ and $Y$ are random variables with respect to $\mathcal{G}$.
\end{enumerate} \hfill (GATE ST 2023)\\
\solution
%\input{gate/ST/2023/14/main.tex}
	\item  A die is loaded in such a way that each odd number is twice as likely to occur as
each even number. Find $P(G)$, where $G$ is the event that a number greater than
3 occurs on a single roll of the die.
\\
\solution
		%\input{exemplar/11/16/3/5/main.tex}
	\item All the jacks, queens and kings are removed from a deck of 52 playing cards. The remaining cards are well shuffled and then one card is drawn at random. Giving ace a value 1 similar value for other cards, find the probability that the card has a value 
		\begin{enumerate}
			\item 7
			\item greater than 7
			\item less than 7
		\end{enumerate}
		%\input{exemplar/10/13/3/30/main.tex}
  \item A Lot consists of 48 mobile phones of which 42 are good, 3 have only minor defects and 3 have major defects.Varnika will buy a phone if it is good but the trader will only buy a mobile if it has no major defects. One phone is selected at random from the lot. What is the probability that it is
\begin{enumerate}
	\item acceptable to Varnika?
            \item acceptable to the trader?
\end{enumerate}
\solution
	%\input{exemplar/10/13/3/40/main.tex}
 \item A student says that if you throw a die, it will show up 1 or not 1. Therefore, the probability of getting 1 and the probability of getting 'not 1' each is equal to $\frac{1}{2}$. Is this correct? Give reasons.\\
 \solution
        %\input{exemplar/10/13/2/9/main.tex}
   \item Four candidates A, B, C, D have ap-
plied for the assignment to coach a school cricket
team. If A is twice as likely to be selected as B, and
B and C are given about the same chance of being
selected, while C is twice as likely to be selected
as D, what are the probabilities that
\begin{enumerate}
\item C will be selected?
\item A will not be selected?
\end{enumerate}
	%\input{exemplar/11/16/3/9/main.tex}
 \item A bag contain 24 balls of which $x$ balls are red, $2x$ are white and $3x$ are blue. A ball is selected at random, What is the probability that it is
\begin{enumerate}[label=\alph*)]
\item not red ?
\item white ?
\end{enumerate}
%\input{exemplar/10/13/3/41/main.tex}
If the letters of the word ASSASSINATION are arranged at random. Find the Probability that
\begin{enumerate}[label=(\alph*)]
\item Four $S's$ come consecutively in the word
\item Two  $I's$ and two $N's$ come together
\item All $A's$ are not coming together
\item No two $A's$ are coming together
\end{enumerate}
%\input{exemplar/11/16/3/14/main.tex}
	\item One urn contains two black balls (labelled B1 and B2) and one white ball. A
	second urn contains one black ball and two white balls (labelled W1 and W2).
	Suppose the following experiment is performed. One of the two urns is chosen
	at random. Next a ball is randomly chosen from the urn. Then a second ball is
	chosen at random from the same urn without replacing the first ball.
	
	\begin{enumerate}
	\item What is the probability that two black balls are chosen?
	
	\item What is the probability that two balls of opposite colour are chosen?
	\end{enumerate}
	\solution
	%\input{exemplar/11/16/3/12/main1.tex}
\end{enumerate}

	\item A bag contains $5$ red balls and some blue balls. If the probability of drawing a blue ball is double that if a red ball, determine the number of blue balls in the bag. 
		\\
\solution
		%\begin{enumerate}[label=\thesection.\arabic*,ref=\thesection.\theenumi]
	\item One card is drawn from a well-shuffled deck of 52 cards. Find the probability of getting
\begin{enumerate}
\item A king of red colour 
\item A face card 
\item A red face card
\item The jack of hearts
\item A spade
\item The queen of diamonds

\end{enumerate}
\solution
		%\input{ncert/10/15/1/14/main.tex}
	\item Five cards—the ten, jack, queen, king and ace of diamonds, are well-shuffled with their face downwards. One card is then picked up at random.
\begin{enumerate}
\item
What is the probability that the card is the queen? 
\item
If the queen is drawn and put aside, what is the probability that the second card picked up is (a) an ace? (b) a queen?\\
\end{enumerate}
\solution
		%\input{ncert/10/15/1/15/defs.tex}
	\item A bag contains $5$ red balls and some blue balls. If the probability of drawing a blue ball is double that if a red ball, determine the number of blue balls in the bag. 
		\\
\solution
		%\input{ncert/10/15/2/3/defs.tex}
	\item A card is selected from a pack of 52 cards.
 \begin{enumerate}[label=(\alph*)] 
                 \item How many points are there in the sample space?
                 \item Calculate the probability that the card is an ace of spades.
                 \item Calculate the probability that the card is (i) an ace and (ii) black card.
 \end{enumerate}
\solution
		%\input{ncert/11/16/3/4/main.tex}
\item Four cards are drawn from a well-shuffled deck of 52 cards. What is the probability of obtaining 3 diamonds and one spade.
\\
\solution
		%\input{ncert/11/16/4/2/defs.tex}
\item In a certain lottery 10,000 tickets are sold and ten equal prizes are awarded. What is the probability of not getting a prize if you buy (a) one ticket (b) two tickets (c) 10 tickets ?	
\\
\solution
		%\input{ncert/11/16/4/4/defs.tex}
		%
\item 
Out of 100 students, two sections of 40 and 60 are formed. If you and your friend are among the 100 students, what is the probability that
\begin{enumerate}
\item you both enter the same section?
\item you both enter the different sections?
\end{enumerate}
\solution
		%\input{ncert/11/16/4/5/defs.tex}
	\item 
The number lock of a suitcase has 4 wheels each labelled with ten digits i.e. from 0 to 9.The lock opens with a sequence of four digits with no repeats.What is the probability of a person getting the right sequence to open the suitcase.
\\
\solution
		%\input{ncert/11/16/4/10/defs.tex}
		%
\item 
Two cards are drawn at random and without replacement from a pack of 52 playing cards. Find the probability that both the cards are black.
\\
\solution
		%\input{ncert/12/13/2/2/defs.tex}
		\item A box of oranges is inspected by examining three randomly selected oranges drawn without replacement. If all the three oranges are good, the box is approved for sale, otherwise, it is rejected. Find the probability that a box containing 15 oranges out of which 12 are good and 3 are bad ones will be approved for sale.
		\label{ncert/12/13/2/3/defs.tex}
		\item Two balls are drawn at random with replacement from a box containing 10 black and 8 red balls. Find the probability that
		\label{ncert/12/13/2/12}
\begin{enumerate}
\item both balls are red.
\item first ball is black and second is red.
\item one of them is black and other is red.
\end{enumerate}

\item In a hostel, 60\% of the students read Hindi newspaper, 40\% read English newspaper and 20\% read both Hindi and English newspapers. A student is selected at random.
		\label{ncert/12/13/2/15}
\begin{enumerate}
\item Find the probability that she reads neither Hindi nor English newspapers.
\item If she reads Hindi newspaper, find the probability that she reads English newspaper.
\item If she reads English newspaper, find the probability that she reads Hindi newspaper.\\
\end{enumerate}
\item The probability of obtaining an even prime number on each die, when a pair of dice is rolled is 
\begin{enumerate}
    \item $0$ 
    
    \item $\frac{1}{3}$ 
    
    \item $\frac{1}{12}$ 
    
    \item $\frac{1}{36}$ 
\end{enumerate}
\solution
		%\input{ncert/12/13/2/17/defs.tex}
	\item A bag contains 4 red and 4 black balls, another bag contains 2 red and 6 black balls. One of the two bags is selected at random and a ball is drawn from the bag which is found to be red. Find the probability that the ball is drawn from the first bag.
\\
\solution
		%\input{ncert/12/13/3/2/main.tex}
  \item
  Cards with numbers 2 to 101 are placed in a box. A card is selected at random.Find the probability that the card has
\begin{enumerate}[label=(\roman*)]
	\item an even number 
	\item a square number
\end{enumerate}
\solution
%\input{exemplar/10/13/3/32/main.tex}
\item
The king, queen and jack of clubs are removed from a deck of 52 playing cards and then well shuffled. Now one card is drawn at random from the remaining cards.  Determine the probability that the card is
\begin{enumerate}[label=(\roman*)]
\item a club
\item 10 of hearts
\end{enumerate}
\solution
%\input{exemplar/10/13/3/29/main.tex}
\item A team of medical students doing their internship have to assist during surgeries
at a city hospital. The probabilities of surgeries rated as very complex, complex,
routine, simple or very simple are respectively, 0.15, 0.20, 0.31, 0.26, .08. Find
the probabilities that a particular surgery will be rated
\begin{enumerate}
	\item complex or very complex;
	\item neither very complex nor very simple;
	\item routine or complex
	\item routine or simple
\end{enumerate}
\solution
%\input{exemplar/11/16/3/8(1)/main.tex}
\item A card is selected from a pack of 52 cards.
\begin{enumerate}[label=(\alph*)]
    \item How many points are there in the sample space?
    \item Calculate the probability that the card is an ace of spades.
    \item Calculate the probability that the card is (i) an ace and (ii) black card.
\end{enumerate}
\solution
%\input{exemplar/11/16/3/4/main2.tex}
\item The probability that a non leap year selected at random will contain 53 sundays.
\\
\solution
%\input{exemplar/10/13/1/19/main.tex}
\item One of the four persons John, Rita, Aslam or Gurpreet will be promoted next
month. Consequently the sample space consists of four elementary outcomes
S = {John promoted, Rita promoted, Aslam promoted, Gurpreet promoted}
You are told that the chances of John’s promotion is same as that of Gurpreet,
Rita’s chances of promotion are twice as likely as Johns. Aslam’s chances are
four times that of John.
\begin{enumerate}
	\item Determine
	\begin{enumerate}
		\item P (John promoted)
		\item P (Rita promoted)
		\item P (Aslam promoted)
		\item P (Gurpreet promoted)
	\end{enumerate}
	\item If A = {John promoted or Gurpreet promoted}, find P (A).
\end{enumerate}
\solution
%\input{exemplar/11/16/3/10/main.tex}
\item A card is drawn from a deck of 52 cards. Find the probability of getting a king or a heart or a red card.\\
\solution
%\input{exemplar/11/16/3/15/main.tex}
\item The probability that a student will pass his examination is 0.73, the probability of
the student getting a compartment is 0.13, and the probability that the student will
either pass or get compartment is 0.96. State True or False.\\
\solution
%\input{exemplar/11/16/3/31/main.tex}
\item A card is selected from a pack of 52 cards\\
\begin{enumerate}[label=(\alph*)]
\item How many points are there in the sample space?
\item Calculate the probability that the cards is an ace of spades.
\item Calculate the probability that the card is (i) an ace (ii)black card.\\
\end{enumerate}
%\input{ncert/11/16/3/4_1/Prob_4.tex}
\item In a non-leap year, the probability of having 53 tuesdays or 53 wednesdays is\\
\solution
%\input{exemplar/11/16/3/18/main.tex}
\item There are 1000 sealed envelopes in a box, 10 of them contain a cash prize of
Rs 100 each, 100 of them contain a cash prize of Rs 50 each and 200 of them
contain a cash prize of Rs 10 each and rest do not contain any cash prize. If they
are well shuffled and an envelope is picked up out, what is the probability that it
contains no cash prize?\\
\solution
%\input{exemplar/10/13/3/34/main.tex}
\item 
A die is thrown and a card is selected at random from a deck of 52 playing cards. The probability of getting an even number on the die and a spade card.\\
\solution
%\input{exemplar/12/13/3/78/main.tex}
\item
If 4-digit numbers greater than 5,000 are randomly formed from the digits 0, 1, 3, 5, and 7, what is the probability of forming a number divisible by 5 when:
\begin{enumerate}
    \item The digits are repeated?
    \item The repetition of digits is not allowed?
\end{enumerate}
\solution
%\input{ncert/11/16/4/9/main.tex}
\item Consider the probability space $\brak{\Omega, \mathcal{G}, P}$ where $\Omega = [0,2]$ and $\mathcal{G} = \cbrak{\phi, \Omega, [0,1], (1,2]}$. Let $X$ and $Y$ be two functions on $\Omega$ defined as
\begin{align*}
    X(\omega) = 
    \begin{cases}
        1 & \text{if }\omega \in [0, 1]\\
        2 & \text{if }\omega \in (1, 2]
    \end{cases}
\end{align*}
and
\begin{align*}
    Y(\omega) = 
    \begin{cases}
        2 & \text{if }\omega \in [0, 1.5]\\
        3 & \text{if }\omega \in (1.5, 2].
    \end{cases}
\end{align*}
Then which one of the following statements is true?
\begin{enumerate}
    \item [(A)] $X$ is a random variable with respect to $\mathcal{G}$, but $Y$ is not a random variable with respect to $\mathcal{G}$.
    \item [(B)] $Y$ is a random variable with respect to $\mathcal{G}$, but $X$ is not a random variable with respect to $\mathcal{G}$.
    \item [(C)] Neither $X$ nor $Y$ is a random variable with respect to $\mathcal{G}$.
    \item [(D)] Both $X$ and $Y$ are random variables with respect to $\mathcal{G}$.
\end{enumerate} \hfill (GATE ST 2023)\\
\solution
%\input{gate/ST/2023/14/main.tex}
	\item  A die is loaded in such a way that each odd number is twice as likely to occur as
each even number. Find $P(G)$, where $G$ is the event that a number greater than
3 occurs on a single roll of the die.
\\
\solution
		%\input{exemplar/11/16/3/5/main.tex}
	\item All the jacks, queens and kings are removed from a deck of 52 playing cards. The remaining cards are well shuffled and then one card is drawn at random. Giving ace a value 1 similar value for other cards, find the probability that the card has a value 
		\begin{enumerate}
			\item 7
			\item greater than 7
			\item less than 7
		\end{enumerate}
		%\input{exemplar/10/13/3/30/main.tex}
  \item A Lot consists of 48 mobile phones of which 42 are good, 3 have only minor defects and 3 have major defects.Varnika will buy a phone if it is good but the trader will only buy a mobile if it has no major defects. One phone is selected at random from the lot. What is the probability that it is
\begin{enumerate}
	\item acceptable to Varnika?
            \item acceptable to the trader?
\end{enumerate}
\solution
	%\input{exemplar/10/13/3/40/main.tex}
 \item A student says that if you throw a die, it will show up 1 or not 1. Therefore, the probability of getting 1 and the probability of getting 'not 1' each is equal to $\frac{1}{2}$. Is this correct? Give reasons.\\
 \solution
        %\input{exemplar/10/13/2/9/main.tex}
   \item Four candidates A, B, C, D have ap-
plied for the assignment to coach a school cricket
team. If A is twice as likely to be selected as B, and
B and C are given about the same chance of being
selected, while C is twice as likely to be selected
as D, what are the probabilities that
\begin{enumerate}
\item C will be selected?
\item A will not be selected?
\end{enumerate}
	%\input{exemplar/11/16/3/9/main.tex}
 \item A bag contain 24 balls of which $x$ balls are red, $2x$ are white and $3x$ are blue. A ball is selected at random, What is the probability that it is
\begin{enumerate}[label=\alph*)]
\item not red ?
\item white ?
\end{enumerate}
%\input{exemplar/10/13/3/41/main.tex}
If the letters of the word ASSASSINATION are arranged at random. Find the Probability that
\begin{enumerate}[label=(\alph*)]
\item Four $S's$ come consecutively in the word
\item Two  $I's$ and two $N's$ come together
\item All $A's$ are not coming together
\item No two $A's$ are coming together
\end{enumerate}
%\input{exemplar/11/16/3/14/main.tex}
	\item One urn contains two black balls (labelled B1 and B2) and one white ball. A
	second urn contains one black ball and two white balls (labelled W1 and W2).
	Suppose the following experiment is performed. One of the two urns is chosen
	at random. Next a ball is randomly chosen from the urn. Then a second ball is
	chosen at random from the same urn without replacing the first ball.
	
	\begin{enumerate}
	\item What is the probability that two black balls are chosen?
	
	\item What is the probability that two balls of opposite colour are chosen?
	\end{enumerate}
	\solution
	%\input{exemplar/11/16/3/12/main1.tex}
\end{enumerate}

	\item A card is selected from a pack of 52 cards.
 \begin{enumerate}[label=(\alph*)] 
                 \item How many points are there in the sample space?
                 \item Calculate the probability that the card is an ace of spades.
                 \item Calculate the probability that the card is (i) an ace and (ii) black card.
 \end{enumerate}
\solution
		%\begin{table}[H]
	\centering
\begin{tabular}{|c|c|c|}
\hline
Random variable &Value &Definition\\ \hline
\multirow{3}{*}{X} &0 &Slips of Rs 1\\
&1 &Slips of Rs 5\\
&2 &Slips of Rs 13\\ \hline
\multirow{2}{*}{Y} &0 &Box A\\
&1 &Box B\\\hline
\end{tabular}
\caption{}
\label{tab:Distribution}
\end{table}
See \tabref{tab:Distribution}.
\begin{align}
p_{Y}\brak{k}= \begin{cases} 
      \frac{1}{3} & {k=0} \\
      \frac{2}{3 }& {k=1} 
   \end{cases}
   \\
p_{Y|X}\brak{0|0} = \frac{19}{25}\, 
p_{Y|X}\brak{0|1} = \frac{6}{25}\,
p_{Y|X}\brak{1|0} = \frac{45}{50}\,
p_{Y|X}\brak{1|2} = \frac{5}{50}
\end{align}
The desired probability is the probability that a slip drawn at random is marked other than Rs 1,
\begin{align}
&=1-p_X\brak{0}\\
&= p_X(1) + p_X(2)
\end{align}
Using Bayes theorem,
\begin{align}
&= p_Y\brak{0} \times \pr{Y=0 | X=1} + p_Y\brak{1} \times \pr{Y=1|X=2}\\
&=\frac{1}{3} \times \frac{6}{25} + \frac{2}{3} \times \frac{5}{50}\\
&=\frac{11}{75}
\end{align}

\newpage

%\tableofcontents

\bigskip

\renewcommand{\thefigure}{\theenumi}
\renewcommand{\thetable}{\theenumi}
%\renewcommand{\theequation}{\theenumi}

%\begin{abstract}
%%\boldmath
%In this letter, an algorithm for evaluating the exact analytical bit error rate  (BER)  for the piecewise linear (PL) combiner for  multiple relays is presented. Previous results were available only for upto three relays. The algorithm is unique in the sense that  the actual mathematical expressions, that are prohibitively large, need not be explicitly obtained. The diversity gain due to multiple relays is shown through plots of the analytical BER, well supported by simulations. 
%
%\end{abstract}
% IEEEtran.cls defaults to using nonbold math in the Abstract.
% This preserves the distinction between vectors and scalars. However,
% if the journal you are submitting to favors bold math in the abstract,
% then you can use LaTeX's standard command \boldmath at the very start
% of the abstract to achieve this. Many IEEE journals frown on math
% in the abstract anyway.

% Note that keywords are not normally used for peerreview papers.
%\begin{IEEEkeywords}
%Cooperative diversity, decode and forward, piecewise linear
%\end{IEEEkeywords}



% For peer review papers, you can put extra information on the cover
% page as needed:
% \ifCLASSOPTIONpeerreview
% \begin{center} \bfseries EDICS Category: 3-BBND \end{center}
% \fi
%
% For peerreview papers, this IEEEtran command inserts a page break and
% creates the second title. It will be ignored for other modes.
%\IEEEpeerreviewmaketitle




\item Four cards are drawn from a well-shuffled deck of 52 cards. What is the probability of obtaining 3 diamonds and one spade.
\\
\solution
		%\begin{enumerate}[label=\thesection.\arabic*,ref=\thesection.\theenumi]
	\item One card is drawn from a well-shuffled deck of 52 cards. Find the probability of getting
\begin{enumerate}
\item A king of red colour 
\item A face card 
\item A red face card
\item The jack of hearts
\item A spade
\item The queen of diamonds

\end{enumerate}
\solution
		%\input{ncert/10/15/1/14/main.tex}
	\item Five cards—the ten, jack, queen, king and ace of diamonds, are well-shuffled with their face downwards. One card is then picked up at random.
\begin{enumerate}
\item
What is the probability that the card is the queen? 
\item
If the queen is drawn and put aside, what is the probability that the second card picked up is (a) an ace? (b) a queen?\\
\end{enumerate}
\solution
		%\input{ncert/10/15/1/15/defs.tex}
	\item A bag contains $5$ red balls and some blue balls. If the probability of drawing a blue ball is double that if a red ball, determine the number of blue balls in the bag. 
		\\
\solution
		%\input{ncert/10/15/2/3/defs.tex}
	\item A card is selected from a pack of 52 cards.
 \begin{enumerate}[label=(\alph*)] 
                 \item How many points are there in the sample space?
                 \item Calculate the probability that the card is an ace of spades.
                 \item Calculate the probability that the card is (i) an ace and (ii) black card.
 \end{enumerate}
\solution
		%\input{ncert/11/16/3/4/main.tex}
\item Four cards are drawn from a well-shuffled deck of 52 cards. What is the probability of obtaining 3 diamonds and one spade.
\\
\solution
		%\input{ncert/11/16/4/2/defs.tex}
\item In a certain lottery 10,000 tickets are sold and ten equal prizes are awarded. What is the probability of not getting a prize if you buy (a) one ticket (b) two tickets (c) 10 tickets ?	
\\
\solution
		%\input{ncert/11/16/4/4/defs.tex}
		%
\item 
Out of 100 students, two sections of 40 and 60 are formed. If you and your friend are among the 100 students, what is the probability that
\begin{enumerate}
\item you both enter the same section?
\item you both enter the different sections?
\end{enumerate}
\solution
		%\input{ncert/11/16/4/5/defs.tex}
	\item 
The number lock of a suitcase has 4 wheels each labelled with ten digits i.e. from 0 to 9.The lock opens with a sequence of four digits with no repeats.What is the probability of a person getting the right sequence to open the suitcase.
\\
\solution
		%\input{ncert/11/16/4/10/defs.tex}
		%
\item 
Two cards are drawn at random and without replacement from a pack of 52 playing cards. Find the probability that both the cards are black.
\\
\solution
		%\input{ncert/12/13/2/2/defs.tex}
		\item A box of oranges is inspected by examining three randomly selected oranges drawn without replacement. If all the three oranges are good, the box is approved for sale, otherwise, it is rejected. Find the probability that a box containing 15 oranges out of which 12 are good and 3 are bad ones will be approved for sale.
		\label{ncert/12/13/2/3/defs.tex}
		\item Two balls are drawn at random with replacement from a box containing 10 black and 8 red balls. Find the probability that
		\label{ncert/12/13/2/12}
\begin{enumerate}
\item both balls are red.
\item first ball is black and second is red.
\item one of them is black and other is red.
\end{enumerate}

\item In a hostel, 60\% of the students read Hindi newspaper, 40\% read English newspaper and 20\% read both Hindi and English newspapers. A student is selected at random.
		\label{ncert/12/13/2/15}
\begin{enumerate}
\item Find the probability that she reads neither Hindi nor English newspapers.
\item If she reads Hindi newspaper, find the probability that she reads English newspaper.
\item If she reads English newspaper, find the probability that she reads Hindi newspaper.\\
\end{enumerate}
\item The probability of obtaining an even prime number on each die, when a pair of dice is rolled is 
\begin{enumerate}
    \item $0$ 
    
    \item $\frac{1}{3}$ 
    
    \item $\frac{1}{12}$ 
    
    \item $\frac{1}{36}$ 
\end{enumerate}
\solution
		%\input{ncert/12/13/2/17/defs.tex}
	\item A bag contains 4 red and 4 black balls, another bag contains 2 red and 6 black balls. One of the two bags is selected at random and a ball is drawn from the bag which is found to be red. Find the probability that the ball is drawn from the first bag.
\\
\solution
		%\input{ncert/12/13/3/2/main.tex}
  \item
  Cards with numbers 2 to 101 are placed in a box. A card is selected at random.Find the probability that the card has
\begin{enumerate}[label=(\roman*)]
	\item an even number 
	\item a square number
\end{enumerate}
\solution
%\input{exemplar/10/13/3/32/main.tex}
\item
The king, queen and jack of clubs are removed from a deck of 52 playing cards and then well shuffled. Now one card is drawn at random from the remaining cards.  Determine the probability that the card is
\begin{enumerate}[label=(\roman*)]
\item a club
\item 10 of hearts
\end{enumerate}
\solution
%\input{exemplar/10/13/3/29/main.tex}
\item A team of medical students doing their internship have to assist during surgeries
at a city hospital. The probabilities of surgeries rated as very complex, complex,
routine, simple or very simple are respectively, 0.15, 0.20, 0.31, 0.26, .08. Find
the probabilities that a particular surgery will be rated
\begin{enumerate}
	\item complex or very complex;
	\item neither very complex nor very simple;
	\item routine or complex
	\item routine or simple
\end{enumerate}
\solution
%\input{exemplar/11/16/3/8(1)/main.tex}
\item A card is selected from a pack of 52 cards.
\begin{enumerate}[label=(\alph*)]
    \item How many points are there in the sample space?
    \item Calculate the probability that the card is an ace of spades.
    \item Calculate the probability that the card is (i) an ace and (ii) black card.
\end{enumerate}
\solution
%\input{exemplar/11/16/3/4/main2.tex}
\item The probability that a non leap year selected at random will contain 53 sundays.
\\
\solution
%\input{exemplar/10/13/1/19/main.tex}
\item One of the four persons John, Rita, Aslam or Gurpreet will be promoted next
month. Consequently the sample space consists of four elementary outcomes
S = {John promoted, Rita promoted, Aslam promoted, Gurpreet promoted}
You are told that the chances of John’s promotion is same as that of Gurpreet,
Rita’s chances of promotion are twice as likely as Johns. Aslam’s chances are
four times that of John.
\begin{enumerate}
	\item Determine
	\begin{enumerate}
		\item P (John promoted)
		\item P (Rita promoted)
		\item P (Aslam promoted)
		\item P (Gurpreet promoted)
	\end{enumerate}
	\item If A = {John promoted or Gurpreet promoted}, find P (A).
\end{enumerate}
\solution
%\input{exemplar/11/16/3/10/main.tex}
\item A card is drawn from a deck of 52 cards. Find the probability of getting a king or a heart or a red card.\\
\solution
%\input{exemplar/11/16/3/15/main.tex}
\item The probability that a student will pass his examination is 0.73, the probability of
the student getting a compartment is 0.13, and the probability that the student will
either pass or get compartment is 0.96. State True or False.\\
\solution
%\input{exemplar/11/16/3/31/main.tex}
\item A card is selected from a pack of 52 cards\\
\begin{enumerate}[label=(\alph*)]
\item How many points are there in the sample space?
\item Calculate the probability that the cards is an ace of spades.
\item Calculate the probability that the card is (i) an ace (ii)black card.\\
\end{enumerate}
%\input{ncert/11/16/3/4_1/Prob_4.tex}
\item In a non-leap year, the probability of having 53 tuesdays or 53 wednesdays is\\
\solution
%\input{exemplar/11/16/3/18/main.tex}
\item There are 1000 sealed envelopes in a box, 10 of them contain a cash prize of
Rs 100 each, 100 of them contain a cash prize of Rs 50 each and 200 of them
contain a cash prize of Rs 10 each and rest do not contain any cash prize. If they
are well shuffled and an envelope is picked up out, what is the probability that it
contains no cash prize?\\
\solution
%\input{exemplar/10/13/3/34/main.tex}
\item 
A die is thrown and a card is selected at random from a deck of 52 playing cards. The probability of getting an even number on the die and a spade card.\\
\solution
%\input{exemplar/12/13/3/78/main.tex}
\item
If 4-digit numbers greater than 5,000 are randomly formed from the digits 0, 1, 3, 5, and 7, what is the probability of forming a number divisible by 5 when:
\begin{enumerate}
    \item The digits are repeated?
    \item The repetition of digits is not allowed?
\end{enumerate}
\solution
%\input{ncert/11/16/4/9/main.tex}
\item Consider the probability space $\brak{\Omega, \mathcal{G}, P}$ where $\Omega = [0,2]$ and $\mathcal{G} = \cbrak{\phi, \Omega, [0,1], (1,2]}$. Let $X$ and $Y$ be two functions on $\Omega$ defined as
\begin{align*}
    X(\omega) = 
    \begin{cases}
        1 & \text{if }\omega \in [0, 1]\\
        2 & \text{if }\omega \in (1, 2]
    \end{cases}
\end{align*}
and
\begin{align*}
    Y(\omega) = 
    \begin{cases}
        2 & \text{if }\omega \in [0, 1.5]\\
        3 & \text{if }\omega \in (1.5, 2].
    \end{cases}
\end{align*}
Then which one of the following statements is true?
\begin{enumerate}
    \item [(A)] $X$ is a random variable with respect to $\mathcal{G}$, but $Y$ is not a random variable with respect to $\mathcal{G}$.
    \item [(B)] $Y$ is a random variable with respect to $\mathcal{G}$, but $X$ is not a random variable with respect to $\mathcal{G}$.
    \item [(C)] Neither $X$ nor $Y$ is a random variable with respect to $\mathcal{G}$.
    \item [(D)] Both $X$ and $Y$ are random variables with respect to $\mathcal{G}$.
\end{enumerate} \hfill (GATE ST 2023)\\
\solution
%\input{gate/ST/2023/14/main.tex}
	\item  A die is loaded in such a way that each odd number is twice as likely to occur as
each even number. Find $P(G)$, where $G$ is the event that a number greater than
3 occurs on a single roll of the die.
\\
\solution
		%\input{exemplar/11/16/3/5/main.tex}
	\item All the jacks, queens and kings are removed from a deck of 52 playing cards. The remaining cards are well shuffled and then one card is drawn at random. Giving ace a value 1 similar value for other cards, find the probability that the card has a value 
		\begin{enumerate}
			\item 7
			\item greater than 7
			\item less than 7
		\end{enumerate}
		%\input{exemplar/10/13/3/30/main.tex}
  \item A Lot consists of 48 mobile phones of which 42 are good, 3 have only minor defects and 3 have major defects.Varnika will buy a phone if it is good but the trader will only buy a mobile if it has no major defects. One phone is selected at random from the lot. What is the probability that it is
\begin{enumerate}
	\item acceptable to Varnika?
            \item acceptable to the trader?
\end{enumerate}
\solution
	%\input{exemplar/10/13/3/40/main.tex}
 \item A student says that if you throw a die, it will show up 1 or not 1. Therefore, the probability of getting 1 and the probability of getting 'not 1' each is equal to $\frac{1}{2}$. Is this correct? Give reasons.\\
 \solution
        %\input{exemplar/10/13/2/9/main.tex}
   \item Four candidates A, B, C, D have ap-
plied for the assignment to coach a school cricket
team. If A is twice as likely to be selected as B, and
B and C are given about the same chance of being
selected, while C is twice as likely to be selected
as D, what are the probabilities that
\begin{enumerate}
\item C will be selected?
\item A will not be selected?
\end{enumerate}
	%\input{exemplar/11/16/3/9/main.tex}
 \item A bag contain 24 balls of which $x$ balls are red, $2x$ are white and $3x$ are blue. A ball is selected at random, What is the probability that it is
\begin{enumerate}[label=\alph*)]
\item not red ?
\item white ?
\end{enumerate}
%\input{exemplar/10/13/3/41/main.tex}
If the letters of the word ASSASSINATION are arranged at random. Find the Probability that
\begin{enumerate}[label=(\alph*)]
\item Four $S's$ come consecutively in the word
\item Two  $I's$ and two $N's$ come together
\item All $A's$ are not coming together
\item No two $A's$ are coming together
\end{enumerate}
%\input{exemplar/11/16/3/14/main.tex}
	\item One urn contains two black balls (labelled B1 and B2) and one white ball. A
	second urn contains one black ball and two white balls (labelled W1 and W2).
	Suppose the following experiment is performed. One of the two urns is chosen
	at random. Next a ball is randomly chosen from the urn. Then a second ball is
	chosen at random from the same urn without replacing the first ball.
	
	\begin{enumerate}
	\item What is the probability that two black balls are chosen?
	
	\item What is the probability that two balls of opposite colour are chosen?
	\end{enumerate}
	\solution
	%\input{exemplar/11/16/3/12/main1.tex}
\end{enumerate}

\item In a certain lottery 10,000 tickets are sold and ten equal prizes are awarded. What is the probability of not getting a prize if you buy (a) one ticket (b) two tickets (c) 10 tickets ?	
\\
\solution
		%\begin{enumerate}[label=\thesection.\arabic*,ref=\thesection.\theenumi]
	\item One card is drawn from a well-shuffled deck of 52 cards. Find the probability of getting
\begin{enumerate}
\item A king of red colour 
\item A face card 
\item A red face card
\item The jack of hearts
\item A spade
\item The queen of diamonds

\end{enumerate}
\solution
		%\input{ncert/10/15/1/14/main.tex}
	\item Five cards—the ten, jack, queen, king and ace of diamonds, are well-shuffled with their face downwards. One card is then picked up at random.
\begin{enumerate}
\item
What is the probability that the card is the queen? 
\item
If the queen is drawn and put aside, what is the probability that the second card picked up is (a) an ace? (b) a queen?\\
\end{enumerate}
\solution
		%\input{ncert/10/15/1/15/defs.tex}
	\item A bag contains $5$ red balls and some blue balls. If the probability of drawing a blue ball is double that if a red ball, determine the number of blue balls in the bag. 
		\\
\solution
		%\input{ncert/10/15/2/3/defs.tex}
	\item A card is selected from a pack of 52 cards.
 \begin{enumerate}[label=(\alph*)] 
                 \item How many points are there in the sample space?
                 \item Calculate the probability that the card is an ace of spades.
                 \item Calculate the probability that the card is (i) an ace and (ii) black card.
 \end{enumerate}
\solution
		%\input{ncert/11/16/3/4/main.tex}
\item Four cards are drawn from a well-shuffled deck of 52 cards. What is the probability of obtaining 3 diamonds and one spade.
\\
\solution
		%\input{ncert/11/16/4/2/defs.tex}
\item In a certain lottery 10,000 tickets are sold and ten equal prizes are awarded. What is the probability of not getting a prize if you buy (a) one ticket (b) two tickets (c) 10 tickets ?	
\\
\solution
		%\input{ncert/11/16/4/4/defs.tex}
		%
\item 
Out of 100 students, two sections of 40 and 60 are formed. If you and your friend are among the 100 students, what is the probability that
\begin{enumerate}
\item you both enter the same section?
\item you both enter the different sections?
\end{enumerate}
\solution
		%\input{ncert/11/16/4/5/defs.tex}
	\item 
The number lock of a suitcase has 4 wheels each labelled with ten digits i.e. from 0 to 9.The lock opens with a sequence of four digits with no repeats.What is the probability of a person getting the right sequence to open the suitcase.
\\
\solution
		%\input{ncert/11/16/4/10/defs.tex}
		%
\item 
Two cards are drawn at random and without replacement from a pack of 52 playing cards. Find the probability that both the cards are black.
\\
\solution
		%\input{ncert/12/13/2/2/defs.tex}
		\item A box of oranges is inspected by examining three randomly selected oranges drawn without replacement. If all the three oranges are good, the box is approved for sale, otherwise, it is rejected. Find the probability that a box containing 15 oranges out of which 12 are good and 3 are bad ones will be approved for sale.
		\label{ncert/12/13/2/3/defs.tex}
		\item Two balls are drawn at random with replacement from a box containing 10 black and 8 red balls. Find the probability that
		\label{ncert/12/13/2/12}
\begin{enumerate}
\item both balls are red.
\item first ball is black and second is red.
\item one of them is black and other is red.
\end{enumerate}

\item In a hostel, 60\% of the students read Hindi newspaper, 40\% read English newspaper and 20\% read both Hindi and English newspapers. A student is selected at random.
		\label{ncert/12/13/2/15}
\begin{enumerate}
\item Find the probability that she reads neither Hindi nor English newspapers.
\item If she reads Hindi newspaper, find the probability that she reads English newspaper.
\item If she reads English newspaper, find the probability that she reads Hindi newspaper.\\
\end{enumerate}
\item The probability of obtaining an even prime number on each die, when a pair of dice is rolled is 
\begin{enumerate}
    \item $0$ 
    
    \item $\frac{1}{3}$ 
    
    \item $\frac{1}{12}$ 
    
    \item $\frac{1}{36}$ 
\end{enumerate}
\solution
		%\input{ncert/12/13/2/17/defs.tex}
	\item A bag contains 4 red and 4 black balls, another bag contains 2 red and 6 black balls. One of the two bags is selected at random and a ball is drawn from the bag which is found to be red. Find the probability that the ball is drawn from the first bag.
\\
\solution
		%\input{ncert/12/13/3/2/main.tex}
  \item
  Cards with numbers 2 to 101 are placed in a box. A card is selected at random.Find the probability that the card has
\begin{enumerate}[label=(\roman*)]
	\item an even number 
	\item a square number
\end{enumerate}
\solution
%\input{exemplar/10/13/3/32/main.tex}
\item
The king, queen and jack of clubs are removed from a deck of 52 playing cards and then well shuffled. Now one card is drawn at random from the remaining cards.  Determine the probability that the card is
\begin{enumerate}[label=(\roman*)]
\item a club
\item 10 of hearts
\end{enumerate}
\solution
%\input{exemplar/10/13/3/29/main.tex}
\item A team of medical students doing their internship have to assist during surgeries
at a city hospital. The probabilities of surgeries rated as very complex, complex,
routine, simple or very simple are respectively, 0.15, 0.20, 0.31, 0.26, .08. Find
the probabilities that a particular surgery will be rated
\begin{enumerate}
	\item complex or very complex;
	\item neither very complex nor very simple;
	\item routine or complex
	\item routine or simple
\end{enumerate}
\solution
%\input{exemplar/11/16/3/8(1)/main.tex}
\item A card is selected from a pack of 52 cards.
\begin{enumerate}[label=(\alph*)]
    \item How many points are there in the sample space?
    \item Calculate the probability that the card is an ace of spades.
    \item Calculate the probability that the card is (i) an ace and (ii) black card.
\end{enumerate}
\solution
%\input{exemplar/11/16/3/4/main2.tex}
\item The probability that a non leap year selected at random will contain 53 sundays.
\\
\solution
%\input{exemplar/10/13/1/19/main.tex}
\item One of the four persons John, Rita, Aslam or Gurpreet will be promoted next
month. Consequently the sample space consists of four elementary outcomes
S = {John promoted, Rita promoted, Aslam promoted, Gurpreet promoted}
You are told that the chances of John’s promotion is same as that of Gurpreet,
Rita’s chances of promotion are twice as likely as Johns. Aslam’s chances are
four times that of John.
\begin{enumerate}
	\item Determine
	\begin{enumerate}
		\item P (John promoted)
		\item P (Rita promoted)
		\item P (Aslam promoted)
		\item P (Gurpreet promoted)
	\end{enumerate}
	\item If A = {John promoted or Gurpreet promoted}, find P (A).
\end{enumerate}
\solution
%\input{exemplar/11/16/3/10/main.tex}
\item A card is drawn from a deck of 52 cards. Find the probability of getting a king or a heart or a red card.\\
\solution
%\input{exemplar/11/16/3/15/main.tex}
\item The probability that a student will pass his examination is 0.73, the probability of
the student getting a compartment is 0.13, and the probability that the student will
either pass or get compartment is 0.96. State True or False.\\
\solution
%\input{exemplar/11/16/3/31/main.tex}
\item A card is selected from a pack of 52 cards\\
\begin{enumerate}[label=(\alph*)]
\item How many points are there in the sample space?
\item Calculate the probability that the cards is an ace of spades.
\item Calculate the probability that the card is (i) an ace (ii)black card.\\
\end{enumerate}
%\input{ncert/11/16/3/4_1/Prob_4.tex}
\item In a non-leap year, the probability of having 53 tuesdays or 53 wednesdays is\\
\solution
%\input{exemplar/11/16/3/18/main.tex}
\item There are 1000 sealed envelopes in a box, 10 of them contain a cash prize of
Rs 100 each, 100 of them contain a cash prize of Rs 50 each and 200 of them
contain a cash prize of Rs 10 each and rest do not contain any cash prize. If they
are well shuffled and an envelope is picked up out, what is the probability that it
contains no cash prize?\\
\solution
%\input{exemplar/10/13/3/34/main.tex}
\item 
A die is thrown and a card is selected at random from a deck of 52 playing cards. The probability of getting an even number on the die and a spade card.\\
\solution
%\input{exemplar/12/13/3/78/main.tex}
\item
If 4-digit numbers greater than 5,000 are randomly formed from the digits 0, 1, 3, 5, and 7, what is the probability of forming a number divisible by 5 when:
\begin{enumerate}
    \item The digits are repeated?
    \item The repetition of digits is not allowed?
\end{enumerate}
\solution
%\input{ncert/11/16/4/9/main.tex}
\item Consider the probability space $\brak{\Omega, \mathcal{G}, P}$ where $\Omega = [0,2]$ and $\mathcal{G} = \cbrak{\phi, \Omega, [0,1], (1,2]}$. Let $X$ and $Y$ be two functions on $\Omega$ defined as
\begin{align*}
    X(\omega) = 
    \begin{cases}
        1 & \text{if }\omega \in [0, 1]\\
        2 & \text{if }\omega \in (1, 2]
    \end{cases}
\end{align*}
and
\begin{align*}
    Y(\omega) = 
    \begin{cases}
        2 & \text{if }\omega \in [0, 1.5]\\
        3 & \text{if }\omega \in (1.5, 2].
    \end{cases}
\end{align*}
Then which one of the following statements is true?
\begin{enumerate}
    \item [(A)] $X$ is a random variable with respect to $\mathcal{G}$, but $Y$ is not a random variable with respect to $\mathcal{G}$.
    \item [(B)] $Y$ is a random variable with respect to $\mathcal{G}$, but $X$ is not a random variable with respect to $\mathcal{G}$.
    \item [(C)] Neither $X$ nor $Y$ is a random variable with respect to $\mathcal{G}$.
    \item [(D)] Both $X$ and $Y$ are random variables with respect to $\mathcal{G}$.
\end{enumerate} \hfill (GATE ST 2023)\\
\solution
%\input{gate/ST/2023/14/main.tex}
	\item  A die is loaded in such a way that each odd number is twice as likely to occur as
each even number. Find $P(G)$, where $G$ is the event that a number greater than
3 occurs on a single roll of the die.
\\
\solution
		%\input{exemplar/11/16/3/5/main.tex}
	\item All the jacks, queens and kings are removed from a deck of 52 playing cards. The remaining cards are well shuffled and then one card is drawn at random. Giving ace a value 1 similar value for other cards, find the probability that the card has a value 
		\begin{enumerate}
			\item 7
			\item greater than 7
			\item less than 7
		\end{enumerate}
		%\input{exemplar/10/13/3/30/main.tex}
  \item A Lot consists of 48 mobile phones of which 42 are good, 3 have only minor defects and 3 have major defects.Varnika will buy a phone if it is good but the trader will only buy a mobile if it has no major defects. One phone is selected at random from the lot. What is the probability that it is
\begin{enumerate}
	\item acceptable to Varnika?
            \item acceptable to the trader?
\end{enumerate}
\solution
	%\input{exemplar/10/13/3/40/main.tex}
 \item A student says that if you throw a die, it will show up 1 or not 1. Therefore, the probability of getting 1 and the probability of getting 'not 1' each is equal to $\frac{1}{2}$. Is this correct? Give reasons.\\
 \solution
        %\input{exemplar/10/13/2/9/main.tex}
   \item Four candidates A, B, C, D have ap-
plied for the assignment to coach a school cricket
team. If A is twice as likely to be selected as B, and
B and C are given about the same chance of being
selected, while C is twice as likely to be selected
as D, what are the probabilities that
\begin{enumerate}
\item C will be selected?
\item A will not be selected?
\end{enumerate}
	%\input{exemplar/11/16/3/9/main.tex}
 \item A bag contain 24 balls of which $x$ balls are red, $2x$ are white and $3x$ are blue. A ball is selected at random, What is the probability that it is
\begin{enumerate}[label=\alph*)]
\item not red ?
\item white ?
\end{enumerate}
%\input{exemplar/10/13/3/41/main.tex}
If the letters of the word ASSASSINATION are arranged at random. Find the Probability that
\begin{enumerate}[label=(\alph*)]
\item Four $S's$ come consecutively in the word
\item Two  $I's$ and two $N's$ come together
\item All $A's$ are not coming together
\item No two $A's$ are coming together
\end{enumerate}
%\input{exemplar/11/16/3/14/main.tex}
	\item One urn contains two black balls (labelled B1 and B2) and one white ball. A
	second urn contains one black ball and two white balls (labelled W1 and W2).
	Suppose the following experiment is performed. One of the two urns is chosen
	at random. Next a ball is randomly chosen from the urn. Then a second ball is
	chosen at random from the same urn without replacing the first ball.
	
	\begin{enumerate}
	\item What is the probability that two black balls are chosen?
	
	\item What is the probability that two balls of opposite colour are chosen?
	\end{enumerate}
	\solution
	%\input{exemplar/11/16/3/12/main1.tex}
\end{enumerate}

		%
\item 
Out of 100 students, two sections of 40 and 60 are formed. If you and your friend are among the 100 students, what is the probability that
\begin{enumerate}
\item you both enter the same section?
\item you both enter the different sections?
\end{enumerate}
\solution
		%\begin{enumerate}[label=\thesection.\arabic*,ref=\thesection.\theenumi]
	\item One card is drawn from a well-shuffled deck of 52 cards. Find the probability of getting
\begin{enumerate}
\item A king of red colour 
\item A face card 
\item A red face card
\item The jack of hearts
\item A spade
\item The queen of diamonds

\end{enumerate}
\solution
		%\input{ncert/10/15/1/14/main.tex}
	\item Five cards—the ten, jack, queen, king and ace of diamonds, are well-shuffled with their face downwards. One card is then picked up at random.
\begin{enumerate}
\item
What is the probability that the card is the queen? 
\item
If the queen is drawn and put aside, what is the probability that the second card picked up is (a) an ace? (b) a queen?\\
\end{enumerate}
\solution
		%\input{ncert/10/15/1/15/defs.tex}
	\item A bag contains $5$ red balls and some blue balls. If the probability of drawing a blue ball is double that if a red ball, determine the number of blue balls in the bag. 
		\\
\solution
		%\input{ncert/10/15/2/3/defs.tex}
	\item A card is selected from a pack of 52 cards.
 \begin{enumerate}[label=(\alph*)] 
                 \item How many points are there in the sample space?
                 \item Calculate the probability that the card is an ace of spades.
                 \item Calculate the probability that the card is (i) an ace and (ii) black card.
 \end{enumerate}
\solution
		%\input{ncert/11/16/3/4/main.tex}
\item Four cards are drawn from a well-shuffled deck of 52 cards. What is the probability of obtaining 3 diamonds and one spade.
\\
\solution
		%\input{ncert/11/16/4/2/defs.tex}
\item In a certain lottery 10,000 tickets are sold and ten equal prizes are awarded. What is the probability of not getting a prize if you buy (a) one ticket (b) two tickets (c) 10 tickets ?	
\\
\solution
		%\input{ncert/11/16/4/4/defs.tex}
		%
\item 
Out of 100 students, two sections of 40 and 60 are formed. If you and your friend are among the 100 students, what is the probability that
\begin{enumerate}
\item you both enter the same section?
\item you both enter the different sections?
\end{enumerate}
\solution
		%\input{ncert/11/16/4/5/defs.tex}
	\item 
The number lock of a suitcase has 4 wheels each labelled with ten digits i.e. from 0 to 9.The lock opens with a sequence of four digits with no repeats.What is the probability of a person getting the right sequence to open the suitcase.
\\
\solution
		%\input{ncert/11/16/4/10/defs.tex}
		%
\item 
Two cards are drawn at random and without replacement from a pack of 52 playing cards. Find the probability that both the cards are black.
\\
\solution
		%\input{ncert/12/13/2/2/defs.tex}
		\item A box of oranges is inspected by examining three randomly selected oranges drawn without replacement. If all the three oranges are good, the box is approved for sale, otherwise, it is rejected. Find the probability that a box containing 15 oranges out of which 12 are good and 3 are bad ones will be approved for sale.
		\label{ncert/12/13/2/3/defs.tex}
		\item Two balls are drawn at random with replacement from a box containing 10 black and 8 red balls. Find the probability that
		\label{ncert/12/13/2/12}
\begin{enumerate}
\item both balls are red.
\item first ball is black and second is red.
\item one of them is black and other is red.
\end{enumerate}

\item In a hostel, 60\% of the students read Hindi newspaper, 40\% read English newspaper and 20\% read both Hindi and English newspapers. A student is selected at random.
		\label{ncert/12/13/2/15}
\begin{enumerate}
\item Find the probability that she reads neither Hindi nor English newspapers.
\item If she reads Hindi newspaper, find the probability that she reads English newspaper.
\item If she reads English newspaper, find the probability that she reads Hindi newspaper.\\
\end{enumerate}
\item The probability of obtaining an even prime number on each die, when a pair of dice is rolled is 
\begin{enumerate}
    \item $0$ 
    
    \item $\frac{1}{3}$ 
    
    \item $\frac{1}{12}$ 
    
    \item $\frac{1}{36}$ 
\end{enumerate}
\solution
		%\input{ncert/12/13/2/17/defs.tex}
	\item A bag contains 4 red and 4 black balls, another bag contains 2 red and 6 black balls. One of the two bags is selected at random and a ball is drawn from the bag which is found to be red. Find the probability that the ball is drawn from the first bag.
\\
\solution
		%\input{ncert/12/13/3/2/main.tex}
  \item
  Cards with numbers 2 to 101 are placed in a box. A card is selected at random.Find the probability that the card has
\begin{enumerate}[label=(\roman*)]
	\item an even number 
	\item a square number
\end{enumerate}
\solution
%\input{exemplar/10/13/3/32/main.tex}
\item
The king, queen and jack of clubs are removed from a deck of 52 playing cards and then well shuffled. Now one card is drawn at random from the remaining cards.  Determine the probability that the card is
\begin{enumerate}[label=(\roman*)]
\item a club
\item 10 of hearts
\end{enumerate}
\solution
%\input{exemplar/10/13/3/29/main.tex}
\item A team of medical students doing their internship have to assist during surgeries
at a city hospital. The probabilities of surgeries rated as very complex, complex,
routine, simple or very simple are respectively, 0.15, 0.20, 0.31, 0.26, .08. Find
the probabilities that a particular surgery will be rated
\begin{enumerate}
	\item complex or very complex;
	\item neither very complex nor very simple;
	\item routine or complex
	\item routine or simple
\end{enumerate}
\solution
%\input{exemplar/11/16/3/8(1)/main.tex}
\item A card is selected from a pack of 52 cards.
\begin{enumerate}[label=(\alph*)]
    \item How many points are there in the sample space?
    \item Calculate the probability that the card is an ace of spades.
    \item Calculate the probability that the card is (i) an ace and (ii) black card.
\end{enumerate}
\solution
%\input{exemplar/11/16/3/4/main2.tex}
\item The probability that a non leap year selected at random will contain 53 sundays.
\\
\solution
%\input{exemplar/10/13/1/19/main.tex}
\item One of the four persons John, Rita, Aslam or Gurpreet will be promoted next
month. Consequently the sample space consists of four elementary outcomes
S = {John promoted, Rita promoted, Aslam promoted, Gurpreet promoted}
You are told that the chances of John’s promotion is same as that of Gurpreet,
Rita’s chances of promotion are twice as likely as Johns. Aslam’s chances are
four times that of John.
\begin{enumerate}
	\item Determine
	\begin{enumerate}
		\item P (John promoted)
		\item P (Rita promoted)
		\item P (Aslam promoted)
		\item P (Gurpreet promoted)
	\end{enumerate}
	\item If A = {John promoted or Gurpreet promoted}, find P (A).
\end{enumerate}
\solution
%\input{exemplar/11/16/3/10/main.tex}
\item A card is drawn from a deck of 52 cards. Find the probability of getting a king or a heart or a red card.\\
\solution
%\input{exemplar/11/16/3/15/main.tex}
\item The probability that a student will pass his examination is 0.73, the probability of
the student getting a compartment is 0.13, and the probability that the student will
either pass or get compartment is 0.96. State True or False.\\
\solution
%\input{exemplar/11/16/3/31/main.tex}
\item A card is selected from a pack of 52 cards\\
\begin{enumerate}[label=(\alph*)]
\item How many points are there in the sample space?
\item Calculate the probability that the cards is an ace of spades.
\item Calculate the probability that the card is (i) an ace (ii)black card.\\
\end{enumerate}
%\input{ncert/11/16/3/4_1/Prob_4.tex}
\item In a non-leap year, the probability of having 53 tuesdays or 53 wednesdays is\\
\solution
%\input{exemplar/11/16/3/18/main.tex}
\item There are 1000 sealed envelopes in a box, 10 of them contain a cash prize of
Rs 100 each, 100 of them contain a cash prize of Rs 50 each and 200 of them
contain a cash prize of Rs 10 each and rest do not contain any cash prize. If they
are well shuffled and an envelope is picked up out, what is the probability that it
contains no cash prize?\\
\solution
%\input{exemplar/10/13/3/34/main.tex}
\item 
A die is thrown and a card is selected at random from a deck of 52 playing cards. The probability of getting an even number on the die and a spade card.\\
\solution
%\input{exemplar/12/13/3/78/main.tex}
\item
If 4-digit numbers greater than 5,000 are randomly formed from the digits 0, 1, 3, 5, and 7, what is the probability of forming a number divisible by 5 when:
\begin{enumerate}
    \item The digits are repeated?
    \item The repetition of digits is not allowed?
\end{enumerate}
\solution
%\input{ncert/11/16/4/9/main.tex}
\item Consider the probability space $\brak{\Omega, \mathcal{G}, P}$ where $\Omega = [0,2]$ and $\mathcal{G} = \cbrak{\phi, \Omega, [0,1], (1,2]}$. Let $X$ and $Y$ be two functions on $\Omega$ defined as
\begin{align*}
    X(\omega) = 
    \begin{cases}
        1 & \text{if }\omega \in [0, 1]\\
        2 & \text{if }\omega \in (1, 2]
    \end{cases}
\end{align*}
and
\begin{align*}
    Y(\omega) = 
    \begin{cases}
        2 & \text{if }\omega \in [0, 1.5]\\
        3 & \text{if }\omega \in (1.5, 2].
    \end{cases}
\end{align*}
Then which one of the following statements is true?
\begin{enumerate}
    \item [(A)] $X$ is a random variable with respect to $\mathcal{G}$, but $Y$ is not a random variable with respect to $\mathcal{G}$.
    \item [(B)] $Y$ is a random variable with respect to $\mathcal{G}$, but $X$ is not a random variable with respect to $\mathcal{G}$.
    \item [(C)] Neither $X$ nor $Y$ is a random variable with respect to $\mathcal{G}$.
    \item [(D)] Both $X$ and $Y$ are random variables with respect to $\mathcal{G}$.
\end{enumerate} \hfill (GATE ST 2023)\\
\solution
%\input{gate/ST/2023/14/main.tex}
	\item  A die is loaded in such a way that each odd number is twice as likely to occur as
each even number. Find $P(G)$, where $G$ is the event that a number greater than
3 occurs on a single roll of the die.
\\
\solution
		%\input{exemplar/11/16/3/5/main.tex}
	\item All the jacks, queens and kings are removed from a deck of 52 playing cards. The remaining cards are well shuffled and then one card is drawn at random. Giving ace a value 1 similar value for other cards, find the probability that the card has a value 
		\begin{enumerate}
			\item 7
			\item greater than 7
			\item less than 7
		\end{enumerate}
		%\input{exemplar/10/13/3/30/main.tex}
  \item A Lot consists of 48 mobile phones of which 42 are good, 3 have only minor defects and 3 have major defects.Varnika will buy a phone if it is good but the trader will only buy a mobile if it has no major defects. One phone is selected at random from the lot. What is the probability that it is
\begin{enumerate}
	\item acceptable to Varnika?
            \item acceptable to the trader?
\end{enumerate}
\solution
	%\input{exemplar/10/13/3/40/main.tex}
 \item A student says that if you throw a die, it will show up 1 or not 1. Therefore, the probability of getting 1 and the probability of getting 'not 1' each is equal to $\frac{1}{2}$. Is this correct? Give reasons.\\
 \solution
        %\input{exemplar/10/13/2/9/main.tex}
   \item Four candidates A, B, C, D have ap-
plied for the assignment to coach a school cricket
team. If A is twice as likely to be selected as B, and
B and C are given about the same chance of being
selected, while C is twice as likely to be selected
as D, what are the probabilities that
\begin{enumerate}
\item C will be selected?
\item A will not be selected?
\end{enumerate}
	%\input{exemplar/11/16/3/9/main.tex}
 \item A bag contain 24 balls of which $x$ balls are red, $2x$ are white and $3x$ are blue. A ball is selected at random, What is the probability that it is
\begin{enumerate}[label=\alph*)]
\item not red ?
\item white ?
\end{enumerate}
%\input{exemplar/10/13/3/41/main.tex}
If the letters of the word ASSASSINATION are arranged at random. Find the Probability that
\begin{enumerate}[label=(\alph*)]
\item Four $S's$ come consecutively in the word
\item Two  $I's$ and two $N's$ come together
\item All $A's$ are not coming together
\item No two $A's$ are coming together
\end{enumerate}
%\input{exemplar/11/16/3/14/main.tex}
	\item One urn contains two black balls (labelled B1 and B2) and one white ball. A
	second urn contains one black ball and two white balls (labelled W1 and W2).
	Suppose the following experiment is performed. One of the two urns is chosen
	at random. Next a ball is randomly chosen from the urn. Then a second ball is
	chosen at random from the same urn without replacing the first ball.
	
	\begin{enumerate}
	\item What is the probability that two black balls are chosen?
	
	\item What is the probability that two balls of opposite colour are chosen?
	\end{enumerate}
	\solution
	%\input{exemplar/11/16/3/12/main1.tex}
\end{enumerate}

	\item 
The number lock of a suitcase has 4 wheels each labelled with ten digits i.e. from 0 to 9.The lock opens with a sequence of four digits with no repeats.What is the probability of a person getting the right sequence to open the suitcase.
\\
\solution
		%\begin{enumerate}[label=\thesection.\arabic*,ref=\thesection.\theenumi]
	\item One card is drawn from a well-shuffled deck of 52 cards. Find the probability of getting
\begin{enumerate}
\item A king of red colour 
\item A face card 
\item A red face card
\item The jack of hearts
\item A spade
\item The queen of diamonds

\end{enumerate}
\solution
		%\input{ncert/10/15/1/14/main.tex}
	\item Five cards—the ten, jack, queen, king and ace of diamonds, are well-shuffled with their face downwards. One card is then picked up at random.
\begin{enumerate}
\item
What is the probability that the card is the queen? 
\item
If the queen is drawn and put aside, what is the probability that the second card picked up is (a) an ace? (b) a queen?\\
\end{enumerate}
\solution
		%\input{ncert/10/15/1/15/defs.tex}
	\item A bag contains $5$ red balls and some blue balls. If the probability of drawing a blue ball is double that if a red ball, determine the number of blue balls in the bag. 
		\\
\solution
		%\input{ncert/10/15/2/3/defs.tex}
	\item A card is selected from a pack of 52 cards.
 \begin{enumerate}[label=(\alph*)] 
                 \item How many points are there in the sample space?
                 \item Calculate the probability that the card is an ace of spades.
                 \item Calculate the probability that the card is (i) an ace and (ii) black card.
 \end{enumerate}
\solution
		%\input{ncert/11/16/3/4/main.tex}
\item Four cards are drawn from a well-shuffled deck of 52 cards. What is the probability of obtaining 3 diamonds and one spade.
\\
\solution
		%\input{ncert/11/16/4/2/defs.tex}
\item In a certain lottery 10,000 tickets are sold and ten equal prizes are awarded. What is the probability of not getting a prize if you buy (a) one ticket (b) two tickets (c) 10 tickets ?	
\\
\solution
		%\input{ncert/11/16/4/4/defs.tex}
		%
\item 
Out of 100 students, two sections of 40 and 60 are formed. If you and your friend are among the 100 students, what is the probability that
\begin{enumerate}
\item you both enter the same section?
\item you both enter the different sections?
\end{enumerate}
\solution
		%\input{ncert/11/16/4/5/defs.tex}
	\item 
The number lock of a suitcase has 4 wheels each labelled with ten digits i.e. from 0 to 9.The lock opens with a sequence of four digits with no repeats.What is the probability of a person getting the right sequence to open the suitcase.
\\
\solution
		%\input{ncert/11/16/4/10/defs.tex}
		%
\item 
Two cards are drawn at random and without replacement from a pack of 52 playing cards. Find the probability that both the cards are black.
\\
\solution
		%\input{ncert/12/13/2/2/defs.tex}
		\item A box of oranges is inspected by examining three randomly selected oranges drawn without replacement. If all the three oranges are good, the box is approved for sale, otherwise, it is rejected. Find the probability that a box containing 15 oranges out of which 12 are good and 3 are bad ones will be approved for sale.
		\label{ncert/12/13/2/3/defs.tex}
		\item Two balls are drawn at random with replacement from a box containing 10 black and 8 red balls. Find the probability that
		\label{ncert/12/13/2/12}
\begin{enumerate}
\item both balls are red.
\item first ball is black and second is red.
\item one of them is black and other is red.
\end{enumerate}

\item In a hostel, 60\% of the students read Hindi newspaper, 40\% read English newspaper and 20\% read both Hindi and English newspapers. A student is selected at random.
		\label{ncert/12/13/2/15}
\begin{enumerate}
\item Find the probability that she reads neither Hindi nor English newspapers.
\item If she reads Hindi newspaper, find the probability that she reads English newspaper.
\item If she reads English newspaper, find the probability that she reads Hindi newspaper.\\
\end{enumerate}
\item The probability of obtaining an even prime number on each die, when a pair of dice is rolled is 
\begin{enumerate}
    \item $0$ 
    
    \item $\frac{1}{3}$ 
    
    \item $\frac{1}{12}$ 
    
    \item $\frac{1}{36}$ 
\end{enumerate}
\solution
		%\input{ncert/12/13/2/17/defs.tex}
	\item A bag contains 4 red and 4 black balls, another bag contains 2 red and 6 black balls. One of the two bags is selected at random and a ball is drawn from the bag which is found to be red. Find the probability that the ball is drawn from the first bag.
\\
\solution
		%\input{ncert/12/13/3/2/main.tex}
  \item
  Cards with numbers 2 to 101 are placed in a box. A card is selected at random.Find the probability that the card has
\begin{enumerate}[label=(\roman*)]
	\item an even number 
	\item a square number
\end{enumerate}
\solution
%\input{exemplar/10/13/3/32/main.tex}
\item
The king, queen and jack of clubs are removed from a deck of 52 playing cards and then well shuffled. Now one card is drawn at random from the remaining cards.  Determine the probability that the card is
\begin{enumerate}[label=(\roman*)]
\item a club
\item 10 of hearts
\end{enumerate}
\solution
%\input{exemplar/10/13/3/29/main.tex}
\item A team of medical students doing their internship have to assist during surgeries
at a city hospital. The probabilities of surgeries rated as very complex, complex,
routine, simple or very simple are respectively, 0.15, 0.20, 0.31, 0.26, .08. Find
the probabilities that a particular surgery will be rated
\begin{enumerate}
	\item complex or very complex;
	\item neither very complex nor very simple;
	\item routine or complex
	\item routine or simple
\end{enumerate}
\solution
%\input{exemplar/11/16/3/8(1)/main.tex}
\item A card is selected from a pack of 52 cards.
\begin{enumerate}[label=(\alph*)]
    \item How many points are there in the sample space?
    \item Calculate the probability that the card is an ace of spades.
    \item Calculate the probability that the card is (i) an ace and (ii) black card.
\end{enumerate}
\solution
%\input{exemplar/11/16/3/4/main2.tex}
\item The probability that a non leap year selected at random will contain 53 sundays.
\\
\solution
%\input{exemplar/10/13/1/19/main.tex}
\item One of the four persons John, Rita, Aslam or Gurpreet will be promoted next
month. Consequently the sample space consists of four elementary outcomes
S = {John promoted, Rita promoted, Aslam promoted, Gurpreet promoted}
You are told that the chances of John’s promotion is same as that of Gurpreet,
Rita’s chances of promotion are twice as likely as Johns. Aslam’s chances are
four times that of John.
\begin{enumerate}
	\item Determine
	\begin{enumerate}
		\item P (John promoted)
		\item P (Rita promoted)
		\item P (Aslam promoted)
		\item P (Gurpreet promoted)
	\end{enumerate}
	\item If A = {John promoted or Gurpreet promoted}, find P (A).
\end{enumerate}
\solution
%\input{exemplar/11/16/3/10/main.tex}
\item A card is drawn from a deck of 52 cards. Find the probability of getting a king or a heart or a red card.\\
\solution
%\input{exemplar/11/16/3/15/main.tex}
\item The probability that a student will pass his examination is 0.73, the probability of
the student getting a compartment is 0.13, and the probability that the student will
either pass or get compartment is 0.96. State True or False.\\
\solution
%\input{exemplar/11/16/3/31/main.tex}
\item A card is selected from a pack of 52 cards\\
\begin{enumerate}[label=(\alph*)]
\item How many points are there in the sample space?
\item Calculate the probability that the cards is an ace of spades.
\item Calculate the probability that the card is (i) an ace (ii)black card.\\
\end{enumerate}
%\input{ncert/11/16/3/4_1/Prob_4.tex}
\item In a non-leap year, the probability of having 53 tuesdays or 53 wednesdays is\\
\solution
%\input{exemplar/11/16/3/18/main.tex}
\item There are 1000 sealed envelopes in a box, 10 of them contain a cash prize of
Rs 100 each, 100 of them contain a cash prize of Rs 50 each and 200 of them
contain a cash prize of Rs 10 each and rest do not contain any cash prize. If they
are well shuffled and an envelope is picked up out, what is the probability that it
contains no cash prize?\\
\solution
%\input{exemplar/10/13/3/34/main.tex}
\item 
A die is thrown and a card is selected at random from a deck of 52 playing cards. The probability of getting an even number on the die and a spade card.\\
\solution
%\input{exemplar/12/13/3/78/main.tex}
\item
If 4-digit numbers greater than 5,000 are randomly formed from the digits 0, 1, 3, 5, and 7, what is the probability of forming a number divisible by 5 when:
\begin{enumerate}
    \item The digits are repeated?
    \item The repetition of digits is not allowed?
\end{enumerate}
\solution
%\input{ncert/11/16/4/9/main.tex}
\item Consider the probability space $\brak{\Omega, \mathcal{G}, P}$ where $\Omega = [0,2]$ and $\mathcal{G} = \cbrak{\phi, \Omega, [0,1], (1,2]}$. Let $X$ and $Y$ be two functions on $\Omega$ defined as
\begin{align*}
    X(\omega) = 
    \begin{cases}
        1 & \text{if }\omega \in [0, 1]\\
        2 & \text{if }\omega \in (1, 2]
    \end{cases}
\end{align*}
and
\begin{align*}
    Y(\omega) = 
    \begin{cases}
        2 & \text{if }\omega \in [0, 1.5]\\
        3 & \text{if }\omega \in (1.5, 2].
    \end{cases}
\end{align*}
Then which one of the following statements is true?
\begin{enumerate}
    \item [(A)] $X$ is a random variable with respect to $\mathcal{G}$, but $Y$ is not a random variable with respect to $\mathcal{G}$.
    \item [(B)] $Y$ is a random variable with respect to $\mathcal{G}$, but $X$ is not a random variable with respect to $\mathcal{G}$.
    \item [(C)] Neither $X$ nor $Y$ is a random variable with respect to $\mathcal{G}$.
    \item [(D)] Both $X$ and $Y$ are random variables with respect to $\mathcal{G}$.
\end{enumerate} \hfill (GATE ST 2023)\\
\solution
%\input{gate/ST/2023/14/main.tex}
	\item  A die is loaded in such a way that each odd number is twice as likely to occur as
each even number. Find $P(G)$, where $G$ is the event that a number greater than
3 occurs on a single roll of the die.
\\
\solution
		%\input{exemplar/11/16/3/5/main.tex}
	\item All the jacks, queens and kings are removed from a deck of 52 playing cards. The remaining cards are well shuffled and then one card is drawn at random. Giving ace a value 1 similar value for other cards, find the probability that the card has a value 
		\begin{enumerate}
			\item 7
			\item greater than 7
			\item less than 7
		\end{enumerate}
		%\input{exemplar/10/13/3/30/main.tex}
  \item A Lot consists of 48 mobile phones of which 42 are good, 3 have only minor defects and 3 have major defects.Varnika will buy a phone if it is good but the trader will only buy a mobile if it has no major defects. One phone is selected at random from the lot. What is the probability that it is
\begin{enumerate}
	\item acceptable to Varnika?
            \item acceptable to the trader?
\end{enumerate}
\solution
	%\input{exemplar/10/13/3/40/main.tex}
 \item A student says that if you throw a die, it will show up 1 or not 1. Therefore, the probability of getting 1 and the probability of getting 'not 1' each is equal to $\frac{1}{2}$. Is this correct? Give reasons.\\
 \solution
        %\input{exemplar/10/13/2/9/main.tex}
   \item Four candidates A, B, C, D have ap-
plied for the assignment to coach a school cricket
team. If A is twice as likely to be selected as B, and
B and C are given about the same chance of being
selected, while C is twice as likely to be selected
as D, what are the probabilities that
\begin{enumerate}
\item C will be selected?
\item A will not be selected?
\end{enumerate}
	%\input{exemplar/11/16/3/9/main.tex}
 \item A bag contain 24 balls of which $x$ balls are red, $2x$ are white and $3x$ are blue. A ball is selected at random, What is the probability that it is
\begin{enumerate}[label=\alph*)]
\item not red ?
\item white ?
\end{enumerate}
%\input{exemplar/10/13/3/41/main.tex}
If the letters of the word ASSASSINATION are arranged at random. Find the Probability that
\begin{enumerate}[label=(\alph*)]
\item Four $S's$ come consecutively in the word
\item Two  $I's$ and two $N's$ come together
\item All $A's$ are not coming together
\item No two $A's$ are coming together
\end{enumerate}
%\input{exemplar/11/16/3/14/main.tex}
	\item One urn contains two black balls (labelled B1 and B2) and one white ball. A
	second urn contains one black ball and two white balls (labelled W1 and W2).
	Suppose the following experiment is performed. One of the two urns is chosen
	at random. Next a ball is randomly chosen from the urn. Then a second ball is
	chosen at random from the same urn without replacing the first ball.
	
	\begin{enumerate}
	\item What is the probability that two black balls are chosen?
	
	\item What is the probability that two balls of opposite colour are chosen?
	\end{enumerate}
	\solution
	%\input{exemplar/11/16/3/12/main1.tex}
\end{enumerate}

		%
\item 
Two cards are drawn at random and without replacement from a pack of 52 playing cards. Find the probability that both the cards are black.
\\
\solution
		%\begin{enumerate}[label=\thesection.\arabic*,ref=\thesection.\theenumi]
	\item One card is drawn from a well-shuffled deck of 52 cards. Find the probability of getting
\begin{enumerate}
\item A king of red colour 
\item A face card 
\item A red face card
\item The jack of hearts
\item A spade
\item The queen of diamonds

\end{enumerate}
\solution
		%\input{ncert/10/15/1/14/main.tex}
	\item Five cards—the ten, jack, queen, king and ace of diamonds, are well-shuffled with their face downwards. One card is then picked up at random.
\begin{enumerate}
\item
What is the probability that the card is the queen? 
\item
If the queen is drawn and put aside, what is the probability that the second card picked up is (a) an ace? (b) a queen?\\
\end{enumerate}
\solution
		%\input{ncert/10/15/1/15/defs.tex}
	\item A bag contains $5$ red balls and some blue balls. If the probability of drawing a blue ball is double that if a red ball, determine the number of blue balls in the bag. 
		\\
\solution
		%\input{ncert/10/15/2/3/defs.tex}
	\item A card is selected from a pack of 52 cards.
 \begin{enumerate}[label=(\alph*)] 
                 \item How many points are there in the sample space?
                 \item Calculate the probability that the card is an ace of spades.
                 \item Calculate the probability that the card is (i) an ace and (ii) black card.
 \end{enumerate}
\solution
		%\input{ncert/11/16/3/4/main.tex}
\item Four cards are drawn from a well-shuffled deck of 52 cards. What is the probability of obtaining 3 diamonds and one spade.
\\
\solution
		%\input{ncert/11/16/4/2/defs.tex}
\item In a certain lottery 10,000 tickets are sold and ten equal prizes are awarded. What is the probability of not getting a prize if you buy (a) one ticket (b) two tickets (c) 10 tickets ?	
\\
\solution
		%\input{ncert/11/16/4/4/defs.tex}
		%
\item 
Out of 100 students, two sections of 40 and 60 are formed. If you and your friend are among the 100 students, what is the probability that
\begin{enumerate}
\item you both enter the same section?
\item you both enter the different sections?
\end{enumerate}
\solution
		%\input{ncert/11/16/4/5/defs.tex}
	\item 
The number lock of a suitcase has 4 wheels each labelled with ten digits i.e. from 0 to 9.The lock opens with a sequence of four digits with no repeats.What is the probability of a person getting the right sequence to open the suitcase.
\\
\solution
		%\input{ncert/11/16/4/10/defs.tex}
		%
\item 
Two cards are drawn at random and without replacement from a pack of 52 playing cards. Find the probability that both the cards are black.
\\
\solution
		%\input{ncert/12/13/2/2/defs.tex}
		\item A box of oranges is inspected by examining three randomly selected oranges drawn without replacement. If all the three oranges are good, the box is approved for sale, otherwise, it is rejected. Find the probability that a box containing 15 oranges out of which 12 are good and 3 are bad ones will be approved for sale.
		\label{ncert/12/13/2/3/defs.tex}
		\item Two balls are drawn at random with replacement from a box containing 10 black and 8 red balls. Find the probability that
		\label{ncert/12/13/2/12}
\begin{enumerate}
\item both balls are red.
\item first ball is black and second is red.
\item one of them is black and other is red.
\end{enumerate}

\item In a hostel, 60\% of the students read Hindi newspaper, 40\% read English newspaper and 20\% read both Hindi and English newspapers. A student is selected at random.
		\label{ncert/12/13/2/15}
\begin{enumerate}
\item Find the probability that she reads neither Hindi nor English newspapers.
\item If she reads Hindi newspaper, find the probability that she reads English newspaper.
\item If she reads English newspaper, find the probability that she reads Hindi newspaper.\\
\end{enumerate}
\item The probability of obtaining an even prime number on each die, when a pair of dice is rolled is 
\begin{enumerate}
    \item $0$ 
    
    \item $\frac{1}{3}$ 
    
    \item $\frac{1}{12}$ 
    
    \item $\frac{1}{36}$ 
\end{enumerate}
\solution
		%\input{ncert/12/13/2/17/defs.tex}
	\item A bag contains 4 red and 4 black balls, another bag contains 2 red and 6 black balls. One of the two bags is selected at random and a ball is drawn from the bag which is found to be red. Find the probability that the ball is drawn from the first bag.
\\
\solution
		%\input{ncert/12/13/3/2/main.tex}
  \item
  Cards with numbers 2 to 101 are placed in a box. A card is selected at random.Find the probability that the card has
\begin{enumerate}[label=(\roman*)]
	\item an even number 
	\item a square number
\end{enumerate}
\solution
%\input{exemplar/10/13/3/32/main.tex}
\item
The king, queen and jack of clubs are removed from a deck of 52 playing cards and then well shuffled. Now one card is drawn at random from the remaining cards.  Determine the probability that the card is
\begin{enumerate}[label=(\roman*)]
\item a club
\item 10 of hearts
\end{enumerate}
\solution
%\input{exemplar/10/13/3/29/main.tex}
\item A team of medical students doing their internship have to assist during surgeries
at a city hospital. The probabilities of surgeries rated as very complex, complex,
routine, simple or very simple are respectively, 0.15, 0.20, 0.31, 0.26, .08. Find
the probabilities that a particular surgery will be rated
\begin{enumerate}
	\item complex or very complex;
	\item neither very complex nor very simple;
	\item routine or complex
	\item routine or simple
\end{enumerate}
\solution
%\input{exemplar/11/16/3/8(1)/main.tex}
\item A card is selected from a pack of 52 cards.
\begin{enumerate}[label=(\alph*)]
    \item How many points are there in the sample space?
    \item Calculate the probability that the card is an ace of spades.
    \item Calculate the probability that the card is (i) an ace and (ii) black card.
\end{enumerate}
\solution
%\input{exemplar/11/16/3/4/main2.tex}
\item The probability that a non leap year selected at random will contain 53 sundays.
\\
\solution
%\input{exemplar/10/13/1/19/main.tex}
\item One of the four persons John, Rita, Aslam or Gurpreet will be promoted next
month. Consequently the sample space consists of four elementary outcomes
S = {John promoted, Rita promoted, Aslam promoted, Gurpreet promoted}
You are told that the chances of John’s promotion is same as that of Gurpreet,
Rita’s chances of promotion are twice as likely as Johns. Aslam’s chances are
four times that of John.
\begin{enumerate}
	\item Determine
	\begin{enumerate}
		\item P (John promoted)
		\item P (Rita promoted)
		\item P (Aslam promoted)
		\item P (Gurpreet promoted)
	\end{enumerate}
	\item If A = {John promoted or Gurpreet promoted}, find P (A).
\end{enumerate}
\solution
%\input{exemplar/11/16/3/10/main.tex}
\item A card is drawn from a deck of 52 cards. Find the probability of getting a king or a heart or a red card.\\
\solution
%\input{exemplar/11/16/3/15/main.tex}
\item The probability that a student will pass his examination is 0.73, the probability of
the student getting a compartment is 0.13, and the probability that the student will
either pass or get compartment is 0.96. State True or False.\\
\solution
%\input{exemplar/11/16/3/31/main.tex}
\item A card is selected from a pack of 52 cards\\
\begin{enumerate}[label=(\alph*)]
\item How many points are there in the sample space?
\item Calculate the probability that the cards is an ace of spades.
\item Calculate the probability that the card is (i) an ace (ii)black card.\\
\end{enumerate}
%\input{ncert/11/16/3/4_1/Prob_4.tex}
\item In a non-leap year, the probability of having 53 tuesdays or 53 wednesdays is\\
\solution
%\input{exemplar/11/16/3/18/main.tex}
\item There are 1000 sealed envelopes in a box, 10 of them contain a cash prize of
Rs 100 each, 100 of them contain a cash prize of Rs 50 each and 200 of them
contain a cash prize of Rs 10 each and rest do not contain any cash prize. If they
are well shuffled and an envelope is picked up out, what is the probability that it
contains no cash prize?\\
\solution
%\input{exemplar/10/13/3/34/main.tex}
\item 
A die is thrown and a card is selected at random from a deck of 52 playing cards. The probability of getting an even number on the die and a spade card.\\
\solution
%\input{exemplar/12/13/3/78/main.tex}
\item
If 4-digit numbers greater than 5,000 are randomly formed from the digits 0, 1, 3, 5, and 7, what is the probability of forming a number divisible by 5 when:
\begin{enumerate}
    \item The digits are repeated?
    \item The repetition of digits is not allowed?
\end{enumerate}
\solution
%\input{ncert/11/16/4/9/main.tex}
\item Consider the probability space $\brak{\Omega, \mathcal{G}, P}$ where $\Omega = [0,2]$ and $\mathcal{G} = \cbrak{\phi, \Omega, [0,1], (1,2]}$. Let $X$ and $Y$ be two functions on $\Omega$ defined as
\begin{align*}
    X(\omega) = 
    \begin{cases}
        1 & \text{if }\omega \in [0, 1]\\
        2 & \text{if }\omega \in (1, 2]
    \end{cases}
\end{align*}
and
\begin{align*}
    Y(\omega) = 
    \begin{cases}
        2 & \text{if }\omega \in [0, 1.5]\\
        3 & \text{if }\omega \in (1.5, 2].
    \end{cases}
\end{align*}
Then which one of the following statements is true?
\begin{enumerate}
    \item [(A)] $X$ is a random variable with respect to $\mathcal{G}$, but $Y$ is not a random variable with respect to $\mathcal{G}$.
    \item [(B)] $Y$ is a random variable with respect to $\mathcal{G}$, but $X$ is not a random variable with respect to $\mathcal{G}$.
    \item [(C)] Neither $X$ nor $Y$ is a random variable with respect to $\mathcal{G}$.
    \item [(D)] Both $X$ and $Y$ are random variables with respect to $\mathcal{G}$.
\end{enumerate} \hfill (GATE ST 2023)\\
\solution
%\input{gate/ST/2023/14/main.tex}
	\item  A die is loaded in such a way that each odd number is twice as likely to occur as
each even number. Find $P(G)$, where $G$ is the event that a number greater than
3 occurs on a single roll of the die.
\\
\solution
		%\input{exemplar/11/16/3/5/main.tex}
	\item All the jacks, queens and kings are removed from a deck of 52 playing cards. The remaining cards are well shuffled and then one card is drawn at random. Giving ace a value 1 similar value for other cards, find the probability that the card has a value 
		\begin{enumerate}
			\item 7
			\item greater than 7
			\item less than 7
		\end{enumerate}
		%\input{exemplar/10/13/3/30/main.tex}
  \item A Lot consists of 48 mobile phones of which 42 are good, 3 have only minor defects and 3 have major defects.Varnika will buy a phone if it is good but the trader will only buy a mobile if it has no major defects. One phone is selected at random from the lot. What is the probability that it is
\begin{enumerate}
	\item acceptable to Varnika?
            \item acceptable to the trader?
\end{enumerate}
\solution
	%\input{exemplar/10/13/3/40/main.tex}
 \item A student says that if you throw a die, it will show up 1 or not 1. Therefore, the probability of getting 1 and the probability of getting 'not 1' each is equal to $\frac{1}{2}$. Is this correct? Give reasons.\\
 \solution
        %\input{exemplar/10/13/2/9/main.tex}
   \item Four candidates A, B, C, D have ap-
plied for the assignment to coach a school cricket
team. If A is twice as likely to be selected as B, and
B and C are given about the same chance of being
selected, while C is twice as likely to be selected
as D, what are the probabilities that
\begin{enumerate}
\item C will be selected?
\item A will not be selected?
\end{enumerate}
	%\input{exemplar/11/16/3/9/main.tex}
 \item A bag contain 24 balls of which $x$ balls are red, $2x$ are white and $3x$ are blue. A ball is selected at random, What is the probability that it is
\begin{enumerate}[label=\alph*)]
\item not red ?
\item white ?
\end{enumerate}
%\input{exemplar/10/13/3/41/main.tex}
If the letters of the word ASSASSINATION are arranged at random. Find the Probability that
\begin{enumerate}[label=(\alph*)]
\item Four $S's$ come consecutively in the word
\item Two  $I's$ and two $N's$ come together
\item All $A's$ are not coming together
\item No two $A's$ are coming together
\end{enumerate}
%\input{exemplar/11/16/3/14/main.tex}
	\item One urn contains two black balls (labelled B1 and B2) and one white ball. A
	second urn contains one black ball and two white balls (labelled W1 and W2).
	Suppose the following experiment is performed. One of the two urns is chosen
	at random. Next a ball is randomly chosen from the urn. Then a second ball is
	chosen at random from the same urn without replacing the first ball.
	
	\begin{enumerate}
	\item What is the probability that two black balls are chosen?
	
	\item What is the probability that two balls of opposite colour are chosen?
	\end{enumerate}
	\solution
	%\input{exemplar/11/16/3/12/main1.tex}
\end{enumerate}

		\item A box of oranges is inspected by examining three randomly selected oranges drawn without replacement. If all the three oranges are good, the box is approved for sale, otherwise, it is rejected. Find the probability that a box containing 15 oranges out of which 12 are good and 3 are bad ones will be approved for sale.
		\label{ncert/12/13/2/3/defs.tex}
		\item Two balls are drawn at random with replacement from a box containing 10 black and 8 red balls. Find the probability that
		\label{ncert/12/13/2/12}
\begin{enumerate}
\item both balls are red.
\item first ball is black and second is red.
\item one of them is black and other is red.
\end{enumerate}

\item In a hostel, 60\% of the students read Hindi newspaper, 40\% read English newspaper and 20\% read both Hindi and English newspapers. A student is selected at random.
		\label{ncert/12/13/2/15}
\begin{enumerate}
\item Find the probability that she reads neither Hindi nor English newspapers.
\item If she reads Hindi newspaper, find the probability that she reads English newspaper.
\item If she reads English newspaper, find the probability that she reads Hindi newspaper.\\
\end{enumerate}
\item The probability of obtaining an even prime number on each die, when a pair of dice is rolled is 
\begin{enumerate}
    \item $0$ 
    
    \item $\frac{1}{3}$ 
    
    \item $\frac{1}{12}$ 
    
    \item $\frac{1}{36}$ 
\end{enumerate}
\solution
		%\begin{enumerate}[label=\thesection.\arabic*,ref=\thesection.\theenumi]
	\item One card is drawn from a well-shuffled deck of 52 cards. Find the probability of getting
\begin{enumerate}
\item A king of red colour 
\item A face card 
\item A red face card
\item The jack of hearts
\item A spade
\item The queen of diamonds

\end{enumerate}
\solution
		%\input{ncert/10/15/1/14/main.tex}
	\item Five cards—the ten, jack, queen, king and ace of diamonds, are well-shuffled with their face downwards. One card is then picked up at random.
\begin{enumerate}
\item
What is the probability that the card is the queen? 
\item
If the queen is drawn and put aside, what is the probability that the second card picked up is (a) an ace? (b) a queen?\\
\end{enumerate}
\solution
		%\input{ncert/10/15/1/15/defs.tex}
	\item A bag contains $5$ red balls and some blue balls. If the probability of drawing a blue ball is double that if a red ball, determine the number of blue balls in the bag. 
		\\
\solution
		%\input{ncert/10/15/2/3/defs.tex}
	\item A card is selected from a pack of 52 cards.
 \begin{enumerate}[label=(\alph*)] 
                 \item How many points are there in the sample space?
                 \item Calculate the probability that the card is an ace of spades.
                 \item Calculate the probability that the card is (i) an ace and (ii) black card.
 \end{enumerate}
\solution
		%\input{ncert/11/16/3/4/main.tex}
\item Four cards are drawn from a well-shuffled deck of 52 cards. What is the probability of obtaining 3 diamonds and one spade.
\\
\solution
		%\input{ncert/11/16/4/2/defs.tex}
\item In a certain lottery 10,000 tickets are sold and ten equal prizes are awarded. What is the probability of not getting a prize if you buy (a) one ticket (b) two tickets (c) 10 tickets ?	
\\
\solution
		%\input{ncert/11/16/4/4/defs.tex}
		%
\item 
Out of 100 students, two sections of 40 and 60 are formed. If you and your friend are among the 100 students, what is the probability that
\begin{enumerate}
\item you both enter the same section?
\item you both enter the different sections?
\end{enumerate}
\solution
		%\input{ncert/11/16/4/5/defs.tex}
	\item 
The number lock of a suitcase has 4 wheels each labelled with ten digits i.e. from 0 to 9.The lock opens with a sequence of four digits with no repeats.What is the probability of a person getting the right sequence to open the suitcase.
\\
\solution
		%\input{ncert/11/16/4/10/defs.tex}
		%
\item 
Two cards are drawn at random and without replacement from a pack of 52 playing cards. Find the probability that both the cards are black.
\\
\solution
		%\input{ncert/12/13/2/2/defs.tex}
		\item A box of oranges is inspected by examining three randomly selected oranges drawn without replacement. If all the three oranges are good, the box is approved for sale, otherwise, it is rejected. Find the probability that a box containing 15 oranges out of which 12 are good and 3 are bad ones will be approved for sale.
		\label{ncert/12/13/2/3/defs.tex}
		\item Two balls are drawn at random with replacement from a box containing 10 black and 8 red balls. Find the probability that
		\label{ncert/12/13/2/12}
\begin{enumerate}
\item both balls are red.
\item first ball is black and second is red.
\item one of them is black and other is red.
\end{enumerate}

\item In a hostel, 60\% of the students read Hindi newspaper, 40\% read English newspaper and 20\% read both Hindi and English newspapers. A student is selected at random.
		\label{ncert/12/13/2/15}
\begin{enumerate}
\item Find the probability that she reads neither Hindi nor English newspapers.
\item If she reads Hindi newspaper, find the probability that she reads English newspaper.
\item If she reads English newspaper, find the probability that she reads Hindi newspaper.\\
\end{enumerate}
\item The probability of obtaining an even prime number on each die, when a pair of dice is rolled is 
\begin{enumerate}
    \item $0$ 
    
    \item $\frac{1}{3}$ 
    
    \item $\frac{1}{12}$ 
    
    \item $\frac{1}{36}$ 
\end{enumerate}
\solution
		%\input{ncert/12/13/2/17/defs.tex}
	\item A bag contains 4 red and 4 black balls, another bag contains 2 red and 6 black balls. One of the two bags is selected at random and a ball is drawn from the bag which is found to be red. Find the probability that the ball is drawn from the first bag.
\\
\solution
		%\input{ncert/12/13/3/2/main.tex}
  \item
  Cards with numbers 2 to 101 are placed in a box. A card is selected at random.Find the probability that the card has
\begin{enumerate}[label=(\roman*)]
	\item an even number 
	\item a square number
\end{enumerate}
\solution
%\input{exemplar/10/13/3/32/main.tex}
\item
The king, queen and jack of clubs are removed from a deck of 52 playing cards and then well shuffled. Now one card is drawn at random from the remaining cards.  Determine the probability that the card is
\begin{enumerate}[label=(\roman*)]
\item a club
\item 10 of hearts
\end{enumerate}
\solution
%\input{exemplar/10/13/3/29/main.tex}
\item A team of medical students doing their internship have to assist during surgeries
at a city hospital. The probabilities of surgeries rated as very complex, complex,
routine, simple or very simple are respectively, 0.15, 0.20, 0.31, 0.26, .08. Find
the probabilities that a particular surgery will be rated
\begin{enumerate}
	\item complex or very complex;
	\item neither very complex nor very simple;
	\item routine or complex
	\item routine or simple
\end{enumerate}
\solution
%\input{exemplar/11/16/3/8(1)/main.tex}
\item A card is selected from a pack of 52 cards.
\begin{enumerate}[label=(\alph*)]
    \item How many points are there in the sample space?
    \item Calculate the probability that the card is an ace of spades.
    \item Calculate the probability that the card is (i) an ace and (ii) black card.
\end{enumerate}
\solution
%\input{exemplar/11/16/3/4/main2.tex}
\item The probability that a non leap year selected at random will contain 53 sundays.
\\
\solution
%\input{exemplar/10/13/1/19/main.tex}
\item One of the four persons John, Rita, Aslam or Gurpreet will be promoted next
month. Consequently the sample space consists of four elementary outcomes
S = {John promoted, Rita promoted, Aslam promoted, Gurpreet promoted}
You are told that the chances of John’s promotion is same as that of Gurpreet,
Rita’s chances of promotion are twice as likely as Johns. Aslam’s chances are
four times that of John.
\begin{enumerate}
	\item Determine
	\begin{enumerate}
		\item P (John promoted)
		\item P (Rita promoted)
		\item P (Aslam promoted)
		\item P (Gurpreet promoted)
	\end{enumerate}
	\item If A = {John promoted or Gurpreet promoted}, find P (A).
\end{enumerate}
\solution
%\input{exemplar/11/16/3/10/main.tex}
\item A card is drawn from a deck of 52 cards. Find the probability of getting a king or a heart or a red card.\\
\solution
%\input{exemplar/11/16/3/15/main.tex}
\item The probability that a student will pass his examination is 0.73, the probability of
the student getting a compartment is 0.13, and the probability that the student will
either pass or get compartment is 0.96. State True or False.\\
\solution
%\input{exemplar/11/16/3/31/main.tex}
\item A card is selected from a pack of 52 cards\\
\begin{enumerate}[label=(\alph*)]
\item How many points are there in the sample space?
\item Calculate the probability that the cards is an ace of spades.
\item Calculate the probability that the card is (i) an ace (ii)black card.\\
\end{enumerate}
%\input{ncert/11/16/3/4_1/Prob_4.tex}
\item In a non-leap year, the probability of having 53 tuesdays or 53 wednesdays is\\
\solution
%\input{exemplar/11/16/3/18/main.tex}
\item There are 1000 sealed envelopes in a box, 10 of them contain a cash prize of
Rs 100 each, 100 of them contain a cash prize of Rs 50 each and 200 of them
contain a cash prize of Rs 10 each and rest do not contain any cash prize. If they
are well shuffled and an envelope is picked up out, what is the probability that it
contains no cash prize?\\
\solution
%\input{exemplar/10/13/3/34/main.tex}
\item 
A die is thrown and a card is selected at random from a deck of 52 playing cards. The probability of getting an even number on the die and a spade card.\\
\solution
%\input{exemplar/12/13/3/78/main.tex}
\item
If 4-digit numbers greater than 5,000 are randomly formed from the digits 0, 1, 3, 5, and 7, what is the probability of forming a number divisible by 5 when:
\begin{enumerate}
    \item The digits are repeated?
    \item The repetition of digits is not allowed?
\end{enumerate}
\solution
%\input{ncert/11/16/4/9/main.tex}
\item Consider the probability space $\brak{\Omega, \mathcal{G}, P}$ where $\Omega = [0,2]$ and $\mathcal{G} = \cbrak{\phi, \Omega, [0,1], (1,2]}$. Let $X$ and $Y$ be two functions on $\Omega$ defined as
\begin{align*}
    X(\omega) = 
    \begin{cases}
        1 & \text{if }\omega \in [0, 1]\\
        2 & \text{if }\omega \in (1, 2]
    \end{cases}
\end{align*}
and
\begin{align*}
    Y(\omega) = 
    \begin{cases}
        2 & \text{if }\omega \in [0, 1.5]\\
        3 & \text{if }\omega \in (1.5, 2].
    \end{cases}
\end{align*}
Then which one of the following statements is true?
\begin{enumerate}
    \item [(A)] $X$ is a random variable with respect to $\mathcal{G}$, but $Y$ is not a random variable with respect to $\mathcal{G}$.
    \item [(B)] $Y$ is a random variable with respect to $\mathcal{G}$, but $X$ is not a random variable with respect to $\mathcal{G}$.
    \item [(C)] Neither $X$ nor $Y$ is a random variable with respect to $\mathcal{G}$.
    \item [(D)] Both $X$ and $Y$ are random variables with respect to $\mathcal{G}$.
\end{enumerate} \hfill (GATE ST 2023)\\
\solution
%\input{gate/ST/2023/14/main.tex}
	\item  A die is loaded in such a way that each odd number is twice as likely to occur as
each even number. Find $P(G)$, where $G$ is the event that a number greater than
3 occurs on a single roll of the die.
\\
\solution
		%\input{exemplar/11/16/3/5/main.tex}
	\item All the jacks, queens and kings are removed from a deck of 52 playing cards. The remaining cards are well shuffled and then one card is drawn at random. Giving ace a value 1 similar value for other cards, find the probability that the card has a value 
		\begin{enumerate}
			\item 7
			\item greater than 7
			\item less than 7
		\end{enumerate}
		%\input{exemplar/10/13/3/30/main.tex}
  \item A Lot consists of 48 mobile phones of which 42 are good, 3 have only minor defects and 3 have major defects.Varnika will buy a phone if it is good but the trader will only buy a mobile if it has no major defects. One phone is selected at random from the lot. What is the probability that it is
\begin{enumerate}
	\item acceptable to Varnika?
            \item acceptable to the trader?
\end{enumerate}
\solution
	%\input{exemplar/10/13/3/40/main.tex}
 \item A student says that if you throw a die, it will show up 1 or not 1. Therefore, the probability of getting 1 and the probability of getting 'not 1' each is equal to $\frac{1}{2}$. Is this correct? Give reasons.\\
 \solution
        %\input{exemplar/10/13/2/9/main.tex}
   \item Four candidates A, B, C, D have ap-
plied for the assignment to coach a school cricket
team. If A is twice as likely to be selected as B, and
B and C are given about the same chance of being
selected, while C is twice as likely to be selected
as D, what are the probabilities that
\begin{enumerate}
\item C will be selected?
\item A will not be selected?
\end{enumerate}
	%\input{exemplar/11/16/3/9/main.tex}
 \item A bag contain 24 balls of which $x$ balls are red, $2x$ are white and $3x$ are blue. A ball is selected at random, What is the probability that it is
\begin{enumerate}[label=\alph*)]
\item not red ?
\item white ?
\end{enumerate}
%\input{exemplar/10/13/3/41/main.tex}
If the letters of the word ASSASSINATION are arranged at random. Find the Probability that
\begin{enumerate}[label=(\alph*)]
\item Four $S's$ come consecutively in the word
\item Two  $I's$ and two $N's$ come together
\item All $A's$ are not coming together
\item No two $A's$ are coming together
\end{enumerate}
%\input{exemplar/11/16/3/14/main.tex}
	\item One urn contains two black balls (labelled B1 and B2) and one white ball. A
	second urn contains one black ball and two white balls (labelled W1 and W2).
	Suppose the following experiment is performed. One of the two urns is chosen
	at random. Next a ball is randomly chosen from the urn. Then a second ball is
	chosen at random from the same urn without replacing the first ball.
	
	\begin{enumerate}
	\item What is the probability that two black balls are chosen?
	
	\item What is the probability that two balls of opposite colour are chosen?
	\end{enumerate}
	\solution
	%\input{exemplar/11/16/3/12/main1.tex}
\end{enumerate}

	\item A bag contains 4 red and 4 black balls, another bag contains 2 red and 6 black balls. One of the two bags is selected at random and a ball is drawn from the bag which is found to be red. Find the probability that the ball is drawn from the first bag.
\\
\solution
		%\begin{table}[H]
	\centering
\begin{tabular}{|c|c|c|}
\hline
Random variable &Value &Definition\\ \hline
\multirow{3}{*}{X} &0 &Slips of Rs 1\\
&1 &Slips of Rs 5\\
&2 &Slips of Rs 13\\ \hline
\multirow{2}{*}{Y} &0 &Box A\\
&1 &Box B\\\hline
\end{tabular}
\caption{}
\label{tab:Distribution}
\end{table}
See \tabref{tab:Distribution}.
\begin{align}
p_{Y}\brak{k}= \begin{cases} 
      \frac{1}{3} & {k=0} \\
      \frac{2}{3 }& {k=1} 
   \end{cases}
   \\
p_{Y|X}\brak{0|0} = \frac{19}{25}\, 
p_{Y|X}\brak{0|1} = \frac{6}{25}\,
p_{Y|X}\brak{1|0} = \frac{45}{50}\,
p_{Y|X}\brak{1|2} = \frac{5}{50}
\end{align}
The desired probability is the probability that a slip drawn at random is marked other than Rs 1,
\begin{align}
&=1-p_X\brak{0}\\
&= p_X(1) + p_X(2)
\end{align}
Using Bayes theorem,
\begin{align}
&= p_Y\brak{0} \times \pr{Y=0 | X=1} + p_Y\brak{1} \times \pr{Y=1|X=2}\\
&=\frac{1}{3} \times \frac{6}{25} + \frac{2}{3} \times \frac{5}{50}\\
&=\frac{11}{75}
\end{align}

\newpage

%\tableofcontents

\bigskip

\renewcommand{\thefigure}{\theenumi}
\renewcommand{\thetable}{\theenumi}
%\renewcommand{\theequation}{\theenumi}

%\begin{abstract}
%%\boldmath
%In this letter, an algorithm for evaluating the exact analytical bit error rate  (BER)  for the piecewise linear (PL) combiner for  multiple relays is presented. Previous results were available only for upto three relays. The algorithm is unique in the sense that  the actual mathematical expressions, that are prohibitively large, need not be explicitly obtained. The diversity gain due to multiple relays is shown through plots of the analytical BER, well supported by simulations. 
%
%\end{abstract}
% IEEEtran.cls defaults to using nonbold math in the Abstract.
% This preserves the distinction between vectors and scalars. However,
% if the journal you are submitting to favors bold math in the abstract,
% then you can use LaTeX's standard command \boldmath at the very start
% of the abstract to achieve this. Many IEEE journals frown on math
% in the abstract anyway.

% Note that keywords are not normally used for peerreview papers.
%\begin{IEEEkeywords}
%Cooperative diversity, decode and forward, piecewise linear
%\end{IEEEkeywords}



% For peer review papers, you can put extra information on the cover
% page as needed:
% \ifCLASSOPTIONpeerreview
% \begin{center} \bfseries EDICS Category: 3-BBND \end{center}
% \fi
%
% For peerreview papers, this IEEEtran command inserts a page break and
% creates the second title. It will be ignored for other modes.
%\IEEEpeerreviewmaketitle




  \item
  Cards with numbers 2 to 101 are placed in a box. A card is selected at random.Find the probability that the card has
\begin{enumerate}[label=(\roman*)]
	\item an even number 
	\item a square number
\end{enumerate}
\solution
%\begin{table}[H]
	\centering
\begin{tabular}{|c|c|c|}
\hline
Random variable &Value &Definition\\ \hline
\multirow{3}{*}{X} &0 &Slips of Rs 1\\
&1 &Slips of Rs 5\\
&2 &Slips of Rs 13\\ \hline
\multirow{2}{*}{Y} &0 &Box A\\
&1 &Box B\\\hline
\end{tabular}
\caption{}
\label{tab:Distribution}
\end{table}
See \tabref{tab:Distribution}.
\begin{align}
p_{Y}\brak{k}= \begin{cases} 
      \frac{1}{3} & {k=0} \\
      \frac{2}{3 }& {k=1} 
   \end{cases}
   \\
p_{Y|X}\brak{0|0} = \frac{19}{25}\, 
p_{Y|X}\brak{0|1} = \frac{6}{25}\,
p_{Y|X}\brak{1|0} = \frac{45}{50}\,
p_{Y|X}\brak{1|2} = \frac{5}{50}
\end{align}
The desired probability is the probability that a slip drawn at random is marked other than Rs 1,
\begin{align}
&=1-p_X\brak{0}\\
&= p_X(1) + p_X(2)
\end{align}
Using Bayes theorem,
\begin{align}
&= p_Y\brak{0} \times \pr{Y=0 | X=1} + p_Y\brak{1} \times \pr{Y=1|X=2}\\
&=\frac{1}{3} \times \frac{6}{25} + \frac{2}{3} \times \frac{5}{50}\\
&=\frac{11}{75}
\end{align}

\newpage

%\tableofcontents

\bigskip

\renewcommand{\thefigure}{\theenumi}
\renewcommand{\thetable}{\theenumi}
%\renewcommand{\theequation}{\theenumi}

%\begin{abstract}
%%\boldmath
%In this letter, an algorithm for evaluating the exact analytical bit error rate  (BER)  for the piecewise linear (PL) combiner for  multiple relays is presented. Previous results were available only for upto three relays. The algorithm is unique in the sense that  the actual mathematical expressions, that are prohibitively large, need not be explicitly obtained. The diversity gain due to multiple relays is shown through plots of the analytical BER, well supported by simulations. 
%
%\end{abstract}
% IEEEtran.cls defaults to using nonbold math in the Abstract.
% This preserves the distinction between vectors and scalars. However,
% if the journal you are submitting to favors bold math in the abstract,
% then you can use LaTeX's standard command \boldmath at the very start
% of the abstract to achieve this. Many IEEE journals frown on math
% in the abstract anyway.

% Note that keywords are not normally used for peerreview papers.
%\begin{IEEEkeywords}
%Cooperative diversity, decode and forward, piecewise linear
%\end{IEEEkeywords}



% For peer review papers, you can put extra information on the cover
% page as needed:
% \ifCLASSOPTIONpeerreview
% \begin{center} \bfseries EDICS Category: 3-BBND \end{center}
% \fi
%
% For peerreview papers, this IEEEtran command inserts a page break and
% creates the second title. It will be ignored for other modes.
%\IEEEpeerreviewmaketitle




\item
The king, queen and jack of clubs are removed from a deck of 52 playing cards and then well shuffled. Now one card is drawn at random from the remaining cards.  Determine the probability that the card is
\begin{enumerate}[label=(\roman*)]
\item a club
\item 10 of hearts
\end{enumerate}
\solution
%\begin{table}[H]
	\centering
\begin{tabular}{|c|c|c|}
\hline
Random variable &Value &Definition\\ \hline
\multirow{3}{*}{X} &0 &Slips of Rs 1\\
&1 &Slips of Rs 5\\
&2 &Slips of Rs 13\\ \hline
\multirow{2}{*}{Y} &0 &Box A\\
&1 &Box B\\\hline
\end{tabular}
\caption{}
\label{tab:Distribution}
\end{table}
See \tabref{tab:Distribution}.
\begin{align}
p_{Y}\brak{k}= \begin{cases} 
      \frac{1}{3} & {k=0} \\
      \frac{2}{3 }& {k=1} 
   \end{cases}
   \\
p_{Y|X}\brak{0|0} = \frac{19}{25}\, 
p_{Y|X}\brak{0|1} = \frac{6}{25}\,
p_{Y|X}\brak{1|0} = \frac{45}{50}\,
p_{Y|X}\brak{1|2} = \frac{5}{50}
\end{align}
The desired probability is the probability that a slip drawn at random is marked other than Rs 1,
\begin{align}
&=1-p_X\brak{0}\\
&= p_X(1) + p_X(2)
\end{align}
Using Bayes theorem,
\begin{align}
&= p_Y\brak{0} \times \pr{Y=0 | X=1} + p_Y\brak{1} \times \pr{Y=1|X=2}\\
&=\frac{1}{3} \times \frac{6}{25} + \frac{2}{3} \times \frac{5}{50}\\
&=\frac{11}{75}
\end{align}

\newpage

%\tableofcontents

\bigskip

\renewcommand{\thefigure}{\theenumi}
\renewcommand{\thetable}{\theenumi}
%\renewcommand{\theequation}{\theenumi}

%\begin{abstract}
%%\boldmath
%In this letter, an algorithm for evaluating the exact analytical bit error rate  (BER)  for the piecewise linear (PL) combiner for  multiple relays is presented. Previous results were available only for upto three relays. The algorithm is unique in the sense that  the actual mathematical expressions, that are prohibitively large, need not be explicitly obtained. The diversity gain due to multiple relays is shown through plots of the analytical BER, well supported by simulations. 
%
%\end{abstract}
% IEEEtran.cls defaults to using nonbold math in the Abstract.
% This preserves the distinction between vectors and scalars. However,
% if the journal you are submitting to favors bold math in the abstract,
% then you can use LaTeX's standard command \boldmath at the very start
% of the abstract to achieve this. Many IEEE journals frown on math
% in the abstract anyway.

% Note that keywords are not normally used for peerreview papers.
%\begin{IEEEkeywords}
%Cooperative diversity, decode and forward, piecewise linear
%\end{IEEEkeywords}



% For peer review papers, you can put extra information on the cover
% page as needed:
% \ifCLASSOPTIONpeerreview
% \begin{center} \bfseries EDICS Category: 3-BBND \end{center}
% \fi
%
% For peerreview papers, this IEEEtran command inserts a page break and
% creates the second title. It will be ignored for other modes.
%\IEEEpeerreviewmaketitle




\item A team of medical students doing their internship have to assist during surgeries
at a city hospital. The probabilities of surgeries rated as very complex, complex,
routine, simple or very simple are respectively, 0.15, 0.20, 0.31, 0.26, .08. Find
the probabilities that a particular surgery will be rated
\begin{enumerate}
	\item complex or very complex;
	\item neither very complex nor very simple;
	\item routine or complex
	\item routine or simple
\end{enumerate}
\solution
%\begin{table}[H]
	\centering
\begin{tabular}{|c|c|c|}
\hline
Random variable &Value &Definition\\ \hline
\multirow{3}{*}{X} &0 &Slips of Rs 1\\
&1 &Slips of Rs 5\\
&2 &Slips of Rs 13\\ \hline
\multirow{2}{*}{Y} &0 &Box A\\
&1 &Box B\\\hline
\end{tabular}
\caption{}
\label{tab:Distribution}
\end{table}
See \tabref{tab:Distribution}.
\begin{align}
p_{Y}\brak{k}= \begin{cases} 
      \frac{1}{3} & {k=0} \\
      \frac{2}{3 }& {k=1} 
   \end{cases}
   \\
p_{Y|X}\brak{0|0} = \frac{19}{25}\, 
p_{Y|X}\brak{0|1} = \frac{6}{25}\,
p_{Y|X}\brak{1|0} = \frac{45}{50}\,
p_{Y|X}\brak{1|2} = \frac{5}{50}
\end{align}
The desired probability is the probability that a slip drawn at random is marked other than Rs 1,
\begin{align}
&=1-p_X\brak{0}\\
&= p_X(1) + p_X(2)
\end{align}
Using Bayes theorem,
\begin{align}
&= p_Y\brak{0} \times \pr{Y=0 | X=1} + p_Y\brak{1} \times \pr{Y=1|X=2}\\
&=\frac{1}{3} \times \frac{6}{25} + \frac{2}{3} \times \frac{5}{50}\\
&=\frac{11}{75}
\end{align}

\newpage

%\tableofcontents

\bigskip

\renewcommand{\thefigure}{\theenumi}
\renewcommand{\thetable}{\theenumi}
%\renewcommand{\theequation}{\theenumi}

%\begin{abstract}
%%\boldmath
%In this letter, an algorithm for evaluating the exact analytical bit error rate  (BER)  for the piecewise linear (PL) combiner for  multiple relays is presented. Previous results were available only for upto three relays. The algorithm is unique in the sense that  the actual mathematical expressions, that are prohibitively large, need not be explicitly obtained. The diversity gain due to multiple relays is shown through plots of the analytical BER, well supported by simulations. 
%
%\end{abstract}
% IEEEtran.cls defaults to using nonbold math in the Abstract.
% This preserves the distinction between vectors and scalars. However,
% if the journal you are submitting to favors bold math in the abstract,
% then you can use LaTeX's standard command \boldmath at the very start
% of the abstract to achieve this. Many IEEE journals frown on math
% in the abstract anyway.

% Note that keywords are not normally used for peerreview papers.
%\begin{IEEEkeywords}
%Cooperative diversity, decode and forward, piecewise linear
%\end{IEEEkeywords}



% For peer review papers, you can put extra information on the cover
% page as needed:
% \ifCLASSOPTIONpeerreview
% \begin{center} \bfseries EDICS Category: 3-BBND \end{center}
% \fi
%
% For peerreview papers, this IEEEtran command inserts a page break and
% creates the second title. It will be ignored for other modes.
%\IEEEpeerreviewmaketitle




\item A card is selected from a pack of 52 cards.
\begin{enumerate}[label=(\alph*)]
    \item How many points are there in the sample space?
    \item Calculate the probability that the card is an ace of spades.
    \item Calculate the probability that the card is (i) an ace and (ii) black card.
\end{enumerate}
\solution
%Let $X$ be an bernoulli rv defined as in \tabref{tab:exemplar/11/16/3/26}.  Then, 
\begin{equation}
    p =
        \frac{4}{11} 
\end{equation}
\begin{table}[H]
	\centering
	\input{exemplar/11/16/3/26/tables/Table2.tex}
	\caption{}
        \label{tab:exemplar/11/16/3/26}
\end{table}

\item The probability that a non leap year selected at random will contain 53 sundays.
\\
\solution
%\begin{table}[H]
	\centering
\begin{tabular}{|c|c|c|}
\hline
Random variable &Value &Definition\\ \hline
\multirow{3}{*}{X} &0 &Slips of Rs 1\\
&1 &Slips of Rs 5\\
&2 &Slips of Rs 13\\ \hline
\multirow{2}{*}{Y} &0 &Box A\\
&1 &Box B\\\hline
\end{tabular}
\caption{}
\label{tab:Distribution}
\end{table}
See \tabref{tab:Distribution}.
\begin{align}
p_{Y}\brak{k}= \begin{cases} 
      \frac{1}{3} & {k=0} \\
      \frac{2}{3 }& {k=1} 
   \end{cases}
   \\
p_{Y|X}\brak{0|0} = \frac{19}{25}\, 
p_{Y|X}\brak{0|1} = \frac{6}{25}\,
p_{Y|X}\brak{1|0} = \frac{45}{50}\,
p_{Y|X}\brak{1|2} = \frac{5}{50}
\end{align}
The desired probability is the probability that a slip drawn at random is marked other than Rs 1,
\begin{align}
&=1-p_X\brak{0}\\
&= p_X(1) + p_X(2)
\end{align}
Using Bayes theorem,
\begin{align}
&= p_Y\brak{0} \times \pr{Y=0 | X=1} + p_Y\brak{1} \times \pr{Y=1|X=2}\\
&=\frac{1}{3} \times \frac{6}{25} + \frac{2}{3} \times \frac{5}{50}\\
&=\frac{11}{75}
\end{align}

\newpage

%\tableofcontents

\bigskip

\renewcommand{\thefigure}{\theenumi}
\renewcommand{\thetable}{\theenumi}
%\renewcommand{\theequation}{\theenumi}

%\begin{abstract}
%%\boldmath
%In this letter, an algorithm for evaluating the exact analytical bit error rate  (BER)  for the piecewise linear (PL) combiner for  multiple relays is presented. Previous results were available only for upto three relays. The algorithm is unique in the sense that  the actual mathematical expressions, that are prohibitively large, need not be explicitly obtained. The diversity gain due to multiple relays is shown through plots of the analytical BER, well supported by simulations. 
%
%\end{abstract}
% IEEEtran.cls defaults to using nonbold math in the Abstract.
% This preserves the distinction between vectors and scalars. However,
% if the journal you are submitting to favors bold math in the abstract,
% then you can use LaTeX's standard command \boldmath at the very start
% of the abstract to achieve this. Many IEEE journals frown on math
% in the abstract anyway.

% Note that keywords are not normally used for peerreview papers.
%\begin{IEEEkeywords}
%Cooperative diversity, decode and forward, piecewise linear
%\end{IEEEkeywords}



% For peer review papers, you can put extra information on the cover
% page as needed:
% \ifCLASSOPTIONpeerreview
% \begin{center} \bfseries EDICS Category: 3-BBND \end{center}
% \fi
%
% For peerreview papers, this IEEEtran command inserts a page break and
% creates the second title. It will be ignored for other modes.
%\IEEEpeerreviewmaketitle




\item One of the four persons John, Rita, Aslam or Gurpreet will be promoted next
month. Consequently the sample space consists of four elementary outcomes
S = {John promoted, Rita promoted, Aslam promoted, Gurpreet promoted}
You are told that the chances of John’s promotion is same as that of Gurpreet,
Rita’s chances of promotion are twice as likely as Johns. Aslam’s chances are
four times that of John.
\begin{enumerate}
	\item Determine
	\begin{enumerate}
		\item P (John promoted)
		\item P (Rita promoted)
		\item P (Aslam promoted)
		\item P (Gurpreet promoted)
	\end{enumerate}
	\item If A = {John promoted or Gurpreet promoted}, find P (A).
\end{enumerate}
\solution
%\begin{table}[H]
	\centering
\begin{tabular}{|c|c|c|}
\hline
Random variable &Value &Definition\\ \hline
\multirow{3}{*}{X} &0 &Slips of Rs 1\\
&1 &Slips of Rs 5\\
&2 &Slips of Rs 13\\ \hline
\multirow{2}{*}{Y} &0 &Box A\\
&1 &Box B\\\hline
\end{tabular}
\caption{}
\label{tab:Distribution}
\end{table}
See \tabref{tab:Distribution}.
\begin{align}
p_{Y}\brak{k}= \begin{cases} 
      \frac{1}{3} & {k=0} \\
      \frac{2}{3 }& {k=1} 
   \end{cases}
   \\
p_{Y|X}\brak{0|0} = \frac{19}{25}\, 
p_{Y|X}\brak{0|1} = \frac{6}{25}\,
p_{Y|X}\brak{1|0} = \frac{45}{50}\,
p_{Y|X}\brak{1|2} = \frac{5}{50}
\end{align}
The desired probability is the probability that a slip drawn at random is marked other than Rs 1,
\begin{align}
&=1-p_X\brak{0}\\
&= p_X(1) + p_X(2)
\end{align}
Using Bayes theorem,
\begin{align}
&= p_Y\brak{0} \times \pr{Y=0 | X=1} + p_Y\brak{1} \times \pr{Y=1|X=2}\\
&=\frac{1}{3} \times \frac{6}{25} + \frac{2}{3} \times \frac{5}{50}\\
&=\frac{11}{75}
\end{align}

\newpage

%\tableofcontents

\bigskip

\renewcommand{\thefigure}{\theenumi}
\renewcommand{\thetable}{\theenumi}
%\renewcommand{\theequation}{\theenumi}

%\begin{abstract}
%%\boldmath
%In this letter, an algorithm for evaluating the exact analytical bit error rate  (BER)  for the piecewise linear (PL) combiner for  multiple relays is presented. Previous results were available only for upto three relays. The algorithm is unique in the sense that  the actual mathematical expressions, that are prohibitively large, need not be explicitly obtained. The diversity gain due to multiple relays is shown through plots of the analytical BER, well supported by simulations. 
%
%\end{abstract}
% IEEEtran.cls defaults to using nonbold math in the Abstract.
% This preserves the distinction between vectors and scalars. However,
% if the journal you are submitting to favors bold math in the abstract,
% then you can use LaTeX's standard command \boldmath at the very start
% of the abstract to achieve this. Many IEEE journals frown on math
% in the abstract anyway.

% Note that keywords are not normally used for peerreview papers.
%\begin{IEEEkeywords}
%Cooperative diversity, decode and forward, piecewise linear
%\end{IEEEkeywords}



% For peer review papers, you can put extra information on the cover
% page as needed:
% \ifCLASSOPTIONpeerreview
% \begin{center} \bfseries EDICS Category: 3-BBND \end{center}
% \fi
%
% For peerreview papers, this IEEEtran command inserts a page break and
% creates the second title. It will be ignored for other modes.
%\IEEEpeerreviewmaketitle




\item A card is drawn from a deck of 52 cards. Find the probability of getting a king or a heart or a red card.\\
\solution
%\begin{table}[H]
	\centering
\begin{tabular}{|c|c|c|}
\hline
Random variable &Value &Definition\\ \hline
\multirow{3}{*}{X} &0 &Slips of Rs 1\\
&1 &Slips of Rs 5\\
&2 &Slips of Rs 13\\ \hline
\multirow{2}{*}{Y} &0 &Box A\\
&1 &Box B\\\hline
\end{tabular}
\caption{}
\label{tab:Distribution}
\end{table}
See \tabref{tab:Distribution}.
\begin{align}
p_{Y}\brak{k}= \begin{cases} 
      \frac{1}{3} & {k=0} \\
      \frac{2}{3 }& {k=1} 
   \end{cases}
   \\
p_{Y|X}\brak{0|0} = \frac{19}{25}\, 
p_{Y|X}\brak{0|1} = \frac{6}{25}\,
p_{Y|X}\brak{1|0} = \frac{45}{50}\,
p_{Y|X}\brak{1|2} = \frac{5}{50}
\end{align}
The desired probability is the probability that a slip drawn at random is marked other than Rs 1,
\begin{align}
&=1-p_X\brak{0}\\
&= p_X(1) + p_X(2)
\end{align}
Using Bayes theorem,
\begin{align}
&= p_Y\brak{0} \times \pr{Y=0 | X=1} + p_Y\brak{1} \times \pr{Y=1|X=2}\\
&=\frac{1}{3} \times \frac{6}{25} + \frac{2}{3} \times \frac{5}{50}\\
&=\frac{11}{75}
\end{align}

\newpage

%\tableofcontents

\bigskip

\renewcommand{\thefigure}{\theenumi}
\renewcommand{\thetable}{\theenumi}
%\renewcommand{\theequation}{\theenumi}

%\begin{abstract}
%%\boldmath
%In this letter, an algorithm for evaluating the exact analytical bit error rate  (BER)  for the piecewise linear (PL) combiner for  multiple relays is presented. Previous results were available only for upto three relays. The algorithm is unique in the sense that  the actual mathematical expressions, that are prohibitively large, need not be explicitly obtained. The diversity gain due to multiple relays is shown through plots of the analytical BER, well supported by simulations. 
%
%\end{abstract}
% IEEEtran.cls defaults to using nonbold math in the Abstract.
% This preserves the distinction between vectors and scalars. However,
% if the journal you are submitting to favors bold math in the abstract,
% then you can use LaTeX's standard command \boldmath at the very start
% of the abstract to achieve this. Many IEEE journals frown on math
% in the abstract anyway.

% Note that keywords are not normally used for peerreview papers.
%\begin{IEEEkeywords}
%Cooperative diversity, decode and forward, piecewise linear
%\end{IEEEkeywords}



% For peer review papers, you can put extra information on the cover
% page as needed:
% \ifCLASSOPTIONpeerreview
% \begin{center} \bfseries EDICS Category: 3-BBND \end{center}
% \fi
%
% For peerreview papers, this IEEEtran command inserts a page break and
% creates the second title. It will be ignored for other modes.
%\IEEEpeerreviewmaketitle




\item The probability that a student will pass his examination is 0.73, the probability of
the student getting a compartment is 0.13, and the probability that the student will
either pass or get compartment is 0.96. State True or False.\\
\solution
%\begin{table}[H]
	\centering
\begin{tabular}{|c|c|c|}
\hline
Random variable &Value &Definition\\ \hline
\multirow{3}{*}{X} &0 &Slips of Rs 1\\
&1 &Slips of Rs 5\\
&2 &Slips of Rs 13\\ \hline
\multirow{2}{*}{Y} &0 &Box A\\
&1 &Box B\\\hline
\end{tabular}
\caption{}
\label{tab:Distribution}
\end{table}
See \tabref{tab:Distribution}.
\begin{align}
p_{Y}\brak{k}= \begin{cases} 
      \frac{1}{3} & {k=0} \\
      \frac{2}{3 }& {k=1} 
   \end{cases}
   \\
p_{Y|X}\brak{0|0} = \frac{19}{25}\, 
p_{Y|X}\brak{0|1} = \frac{6}{25}\,
p_{Y|X}\brak{1|0} = \frac{45}{50}\,
p_{Y|X}\brak{1|2} = \frac{5}{50}
\end{align}
The desired probability is the probability that a slip drawn at random is marked other than Rs 1,
\begin{align}
&=1-p_X\brak{0}\\
&= p_X(1) + p_X(2)
\end{align}
Using Bayes theorem,
\begin{align}
&= p_Y\brak{0} \times \pr{Y=0 | X=1} + p_Y\brak{1} \times \pr{Y=1|X=2}\\
&=\frac{1}{3} \times \frac{6}{25} + \frac{2}{3} \times \frac{5}{50}\\
&=\frac{11}{75}
\end{align}

\newpage

%\tableofcontents

\bigskip

\renewcommand{\thefigure}{\theenumi}
\renewcommand{\thetable}{\theenumi}
%\renewcommand{\theequation}{\theenumi}

%\begin{abstract}
%%\boldmath
%In this letter, an algorithm for evaluating the exact analytical bit error rate  (BER)  for the piecewise linear (PL) combiner for  multiple relays is presented. Previous results were available only for upto three relays. The algorithm is unique in the sense that  the actual mathematical expressions, that are prohibitively large, need not be explicitly obtained. The diversity gain due to multiple relays is shown through plots of the analytical BER, well supported by simulations. 
%
%\end{abstract}
% IEEEtran.cls defaults to using nonbold math in the Abstract.
% This preserves the distinction between vectors and scalars. However,
% if the journal you are submitting to favors bold math in the abstract,
% then you can use LaTeX's standard command \boldmath at the very start
% of the abstract to achieve this. Many IEEE journals frown on math
% in the abstract anyway.

% Note that keywords are not normally used for peerreview papers.
%\begin{IEEEkeywords}
%Cooperative diversity, decode and forward, piecewise linear
%\end{IEEEkeywords}



% For peer review papers, you can put extra information on the cover
% page as needed:
% \ifCLASSOPTIONpeerreview
% \begin{center} \bfseries EDICS Category: 3-BBND \end{center}
% \fi
%
% For peerreview papers, this IEEEtran command inserts a page break and
% creates the second title. It will be ignored for other modes.
%\IEEEpeerreviewmaketitle




\item A card is selected from a pack of 52 cards\\
\begin{enumerate}[label=(\alph*)]
\item How many points are there in the sample space?
\item Calculate the probability that the cards is an ace of spades.
\item Calculate the probability that the card is (i) an ace (ii)black card.\\
\end{enumerate}
%\input{ncert/11/16/3/4_1/Prob_4.tex}
\item In a non-leap year, the probability of having 53 tuesdays or 53 wednesdays is\\
\solution
%A non-leap year has a total of 365 days, and a week has 7 days.\\
So it can be expressed as 
\begin{align}
365\text{days} &=52\times 7+1 \text{day}
\end{align}
$\implies$ 52 tuesdays or wednesdays\\
Random variable X denotes the days of a week
\begin{align}
p_X\brak{k}&=\frac{1}{7}; \quad \brak{1<k<7}
\end{align}
So the probability of extra day being tuesday or wednesday is
\begin{align}
p_X\brak{3}+p_X\brak{4}&=\frac{1}{7}+\frac{1}{7}=\frac{2}{7}
\end{align}



\item There are 1000 sealed envelopes in a box, 10 of them contain a cash prize of
Rs 100 each, 100 of them contain a cash prize of Rs 50 each and 200 of them
contain a cash prize of Rs 10 each and rest do not contain any cash prize. If they
are well shuffled and an envelope is picked up out, what is the probability that it
contains no cash prize?\\
\solution
%\begin{table}[H]
	\centering
\begin{tabular}{|c|c|c|}
\hline
Random variable &Value &Definition\\ \hline
\multirow{3}{*}{X} &0 &Slips of Rs 1\\
&1 &Slips of Rs 5\\
&2 &Slips of Rs 13\\ \hline
\multirow{2}{*}{Y} &0 &Box A\\
&1 &Box B\\\hline
\end{tabular}
\caption{}
\label{tab:Distribution}
\end{table}
See \tabref{tab:Distribution}.
\begin{align}
p_{Y}\brak{k}= \begin{cases} 
      \frac{1}{3} & {k=0} \\
      \frac{2}{3 }& {k=1} 
   \end{cases}
   \\
p_{Y|X}\brak{0|0} = \frac{19}{25}\, 
p_{Y|X}\brak{0|1} = \frac{6}{25}\,
p_{Y|X}\brak{1|0} = \frac{45}{50}\,
p_{Y|X}\brak{1|2} = \frac{5}{50}
\end{align}
The desired probability is the probability that a slip drawn at random is marked other than Rs 1,
\begin{align}
&=1-p_X\brak{0}\\
&= p_X(1) + p_X(2)
\end{align}
Using Bayes theorem,
\begin{align}
&= p_Y\brak{0} \times \pr{Y=0 | X=1} + p_Y\brak{1} \times \pr{Y=1|X=2}\\
&=\frac{1}{3} \times \frac{6}{25} + \frac{2}{3} \times \frac{5}{50}\\
&=\frac{11}{75}
\end{align}

\newpage

%\tableofcontents

\bigskip

\renewcommand{\thefigure}{\theenumi}
\renewcommand{\thetable}{\theenumi}
%\renewcommand{\theequation}{\theenumi}

%\begin{abstract}
%%\boldmath
%In this letter, an algorithm for evaluating the exact analytical bit error rate  (BER)  for the piecewise linear (PL) combiner for  multiple relays is presented. Previous results were available only for upto three relays. The algorithm is unique in the sense that  the actual mathematical expressions, that are prohibitively large, need not be explicitly obtained. The diversity gain due to multiple relays is shown through plots of the analytical BER, well supported by simulations. 
%
%\end{abstract}
% IEEEtran.cls defaults to using nonbold math in the Abstract.
% This preserves the distinction between vectors and scalars. However,
% if the journal you are submitting to favors bold math in the abstract,
% then you can use LaTeX's standard command \boldmath at the very start
% of the abstract to achieve this. Many IEEE journals frown on math
% in the abstract anyway.

% Note that keywords are not normally used for peerreview papers.
%\begin{IEEEkeywords}
%Cooperative diversity, decode and forward, piecewise linear
%\end{IEEEkeywords}



% For peer review papers, you can put extra information on the cover
% page as needed:
% \ifCLASSOPTIONpeerreview
% \begin{center} \bfseries EDICS Category: 3-BBND \end{center}
% \fi
%
% For peerreview papers, this IEEEtran command inserts a page break and
% creates the second title. It will be ignored for other modes.
%\IEEEpeerreviewmaketitle




\item 
A die is thrown and a card is selected at random from a deck of 52 playing cards. The probability of getting an even number on the die and a spade card.\\
\solution
%\begin{table}[H]
	\centering
\begin{tabular}{|c|c|c|}
\hline
Random variable &Value &Definition\\ \hline
\multirow{3}{*}{X} &0 &Slips of Rs 1\\
&1 &Slips of Rs 5\\
&2 &Slips of Rs 13\\ \hline
\multirow{2}{*}{Y} &0 &Box A\\
&1 &Box B\\\hline
\end{tabular}
\caption{}
\label{tab:Distribution}
\end{table}
See \tabref{tab:Distribution}.
\begin{align}
p_{Y}\brak{k}= \begin{cases} 
      \frac{1}{3} & {k=0} \\
      \frac{2}{3 }& {k=1} 
   \end{cases}
   \\
p_{Y|X}\brak{0|0} = \frac{19}{25}\, 
p_{Y|X}\brak{0|1} = \frac{6}{25}\,
p_{Y|X}\brak{1|0} = \frac{45}{50}\,
p_{Y|X}\brak{1|2} = \frac{5}{50}
\end{align}
The desired probability is the probability that a slip drawn at random is marked other than Rs 1,
\begin{align}
&=1-p_X\brak{0}\\
&= p_X(1) + p_X(2)
\end{align}
Using Bayes theorem,
\begin{align}
&= p_Y\brak{0} \times \pr{Y=0 | X=1} + p_Y\brak{1} \times \pr{Y=1|X=2}\\
&=\frac{1}{3} \times \frac{6}{25} + \frac{2}{3} \times \frac{5}{50}\\
&=\frac{11}{75}
\end{align}

\newpage

%\tableofcontents

\bigskip

\renewcommand{\thefigure}{\theenumi}
\renewcommand{\thetable}{\theenumi}
%\renewcommand{\theequation}{\theenumi}

%\begin{abstract}
%%\boldmath
%In this letter, an algorithm for evaluating the exact analytical bit error rate  (BER)  for the piecewise linear (PL) combiner for  multiple relays is presented. Previous results were available only for upto three relays. The algorithm is unique in the sense that  the actual mathematical expressions, that are prohibitively large, need not be explicitly obtained. The diversity gain due to multiple relays is shown through plots of the analytical BER, well supported by simulations. 
%
%\end{abstract}
% IEEEtran.cls defaults to using nonbold math in the Abstract.
% This preserves the distinction between vectors and scalars. However,
% if the journal you are submitting to favors bold math in the abstract,
% then you can use LaTeX's standard command \boldmath at the very start
% of the abstract to achieve this. Many IEEE journals frown on math
% in the abstract anyway.

% Note that keywords are not normally used for peerreview papers.
%\begin{IEEEkeywords}
%Cooperative diversity, decode and forward, piecewise linear
%\end{IEEEkeywords}



% For peer review papers, you can put extra information on the cover
% page as needed:
% \ifCLASSOPTIONpeerreview
% \begin{center} \bfseries EDICS Category: 3-BBND \end{center}
% \fi
%
% For peerreview papers, this IEEEtran command inserts a page break and
% creates the second title. It will be ignored for other modes.
%\IEEEpeerreviewmaketitle




\item
If 4-digit numbers greater than 5,000 are randomly formed from the digits 0, 1, 3, 5, and 7, what is the probability of forming a number divisible by 5 when:
\begin{enumerate}
    \item The digits are repeated?
    \item The repetition of digits is not allowed?
\end{enumerate}
\solution
%\begin{table}[H]
	\centering
\begin{tabular}{|c|c|c|}
\hline
Random variable &Value &Definition\\ \hline
\multirow{3}{*}{X} &0 &Slips of Rs 1\\
&1 &Slips of Rs 5\\
&2 &Slips of Rs 13\\ \hline
\multirow{2}{*}{Y} &0 &Box A\\
&1 &Box B\\\hline
\end{tabular}
\caption{}
\label{tab:Distribution}
\end{table}
See \tabref{tab:Distribution}.
\begin{align}
p_{Y}\brak{k}= \begin{cases} 
      \frac{1}{3} & {k=0} \\
      \frac{2}{3 }& {k=1} 
   \end{cases}
   \\
p_{Y|X}\brak{0|0} = \frac{19}{25}\, 
p_{Y|X}\brak{0|1} = \frac{6}{25}\,
p_{Y|X}\brak{1|0} = \frac{45}{50}\,
p_{Y|X}\brak{1|2} = \frac{5}{50}
\end{align}
The desired probability is the probability that a slip drawn at random is marked other than Rs 1,
\begin{align}
&=1-p_X\brak{0}\\
&= p_X(1) + p_X(2)
\end{align}
Using Bayes theorem,
\begin{align}
&= p_Y\brak{0} \times \pr{Y=0 | X=1} + p_Y\brak{1} \times \pr{Y=1|X=2}\\
&=\frac{1}{3} \times \frac{6}{25} + \frac{2}{3} \times \frac{5}{50}\\
&=\frac{11}{75}
\end{align}

\newpage

%\tableofcontents

\bigskip

\renewcommand{\thefigure}{\theenumi}
\renewcommand{\thetable}{\theenumi}
%\renewcommand{\theequation}{\theenumi}

%\begin{abstract}
%%\boldmath
%In this letter, an algorithm for evaluating the exact analytical bit error rate  (BER)  for the piecewise linear (PL) combiner for  multiple relays is presented. Previous results were available only for upto three relays. The algorithm is unique in the sense that  the actual mathematical expressions, that are prohibitively large, need not be explicitly obtained. The diversity gain due to multiple relays is shown through plots of the analytical BER, well supported by simulations. 
%
%\end{abstract}
% IEEEtran.cls defaults to using nonbold math in the Abstract.
% This preserves the distinction between vectors and scalars. However,
% if the journal you are submitting to favors bold math in the abstract,
% then you can use LaTeX's standard command \boldmath at the very start
% of the abstract to achieve this. Many IEEE journals frown on math
% in the abstract anyway.

% Note that keywords are not normally used for peerreview papers.
%\begin{IEEEkeywords}
%Cooperative diversity, decode and forward, piecewise linear
%\end{IEEEkeywords}



% For peer review papers, you can put extra information on the cover
% page as needed:
% \ifCLASSOPTIONpeerreview
% \begin{center} \bfseries EDICS Category: 3-BBND \end{center}
% \fi
%
% For peerreview papers, this IEEEtran command inserts a page break and
% creates the second title. It will be ignored for other modes.
%\IEEEpeerreviewmaketitle




\item Consider the probability space $\brak{\Omega, \mathcal{G}, P}$ where $\Omega = [0,2]$ and $\mathcal{G} = \cbrak{\phi, \Omega, [0,1], (1,2]}$. Let $X$ and $Y$ be two functions on $\Omega$ defined as
\begin{align*}
    X(\omega) = 
    \begin{cases}
        1 & \text{if }\omega \in [0, 1]\\
        2 & \text{if }\omega \in (1, 2]
    \end{cases}
\end{align*}
and
\begin{align*}
    Y(\omega) = 
    \begin{cases}
        2 & \text{if }\omega \in [0, 1.5]\\
        3 & \text{if }\omega \in (1.5, 2].
    \end{cases}
\end{align*}
Then which one of the following statements is true?
\begin{enumerate}
    \item [(A)] $X$ is a random variable with respect to $\mathcal{G}$, but $Y$ is not a random variable with respect to $\mathcal{G}$.
    \item [(B)] $Y$ is a random variable with respect to $\mathcal{G}$, but $X$ is not a random variable with respect to $\mathcal{G}$.
    \item [(C)] Neither $X$ nor $Y$ is a random variable with respect to $\mathcal{G}$.
    \item [(D)] Both $X$ and $Y$ are random variables with respect to $\mathcal{G}$.
\end{enumerate} \hfill (GATE ST 2023)\\
\solution
%\begin{table}[H]
	\centering
\begin{tabular}{|c|c|c|}
\hline
Random variable &Value &Definition\\ \hline
\multirow{3}{*}{X} &0 &Slips of Rs 1\\
&1 &Slips of Rs 5\\
&2 &Slips of Rs 13\\ \hline
\multirow{2}{*}{Y} &0 &Box A\\
&1 &Box B\\\hline
\end{tabular}
\caption{}
\label{tab:Distribution}
\end{table}
See \tabref{tab:Distribution}.
\begin{align}
p_{Y}\brak{k}= \begin{cases} 
      \frac{1}{3} & {k=0} \\
      \frac{2}{3 }& {k=1} 
   \end{cases}
   \\
p_{Y|X}\brak{0|0} = \frac{19}{25}\, 
p_{Y|X}\brak{0|1} = \frac{6}{25}\,
p_{Y|X}\brak{1|0} = \frac{45}{50}\,
p_{Y|X}\brak{1|2} = \frac{5}{50}
\end{align}
The desired probability is the probability that a slip drawn at random is marked other than Rs 1,
\begin{align}
&=1-p_X\brak{0}\\
&= p_X(1) + p_X(2)
\end{align}
Using Bayes theorem,
\begin{align}
&= p_Y\brak{0} \times \pr{Y=0 | X=1} + p_Y\brak{1} \times \pr{Y=1|X=2}\\
&=\frac{1}{3} \times \frac{6}{25} + \frac{2}{3} \times \frac{5}{50}\\
&=\frac{11}{75}
\end{align}

\newpage

%\tableofcontents

\bigskip

\renewcommand{\thefigure}{\theenumi}
\renewcommand{\thetable}{\theenumi}
%\renewcommand{\theequation}{\theenumi}

%\begin{abstract}
%%\boldmath
%In this letter, an algorithm for evaluating the exact analytical bit error rate  (BER)  for the piecewise linear (PL) combiner for  multiple relays is presented. Previous results were available only for upto three relays. The algorithm is unique in the sense that  the actual mathematical expressions, that are prohibitively large, need not be explicitly obtained. The diversity gain due to multiple relays is shown through plots of the analytical BER, well supported by simulations. 
%
%\end{abstract}
% IEEEtran.cls defaults to using nonbold math in the Abstract.
% This preserves the distinction between vectors and scalars. However,
% if the journal you are submitting to favors bold math in the abstract,
% then you can use LaTeX's standard command \boldmath at the very start
% of the abstract to achieve this. Many IEEE journals frown on math
% in the abstract anyway.

% Note that keywords are not normally used for peerreview papers.
%\begin{IEEEkeywords}
%Cooperative diversity, decode and forward, piecewise linear
%\end{IEEEkeywords}



% For peer review papers, you can put extra information on the cover
% page as needed:
% \ifCLASSOPTIONpeerreview
% \begin{center} \bfseries EDICS Category: 3-BBND \end{center}
% \fi
%
% For peerreview papers, this IEEEtran command inserts a page break and
% creates the second title. It will be ignored for other modes.
%\IEEEpeerreviewmaketitle




	\item  A die is loaded in such a way that each odd number is twice as likely to occur as
each even number. Find $P(G)$, where $G$ is the event that a number greater than
3 occurs on a single roll of the die.
\\
\solution
		%\begin{table}[H]
	\centering
\begin{tabular}{|c|c|c|}
\hline
Random variable &Value &Definition\\ \hline
\multirow{3}{*}{X} &0 &Slips of Rs 1\\
&1 &Slips of Rs 5\\
&2 &Slips of Rs 13\\ \hline
\multirow{2}{*}{Y} &0 &Box A\\
&1 &Box B\\\hline
\end{tabular}
\caption{}
\label{tab:Distribution}
\end{table}
See \tabref{tab:Distribution}.
\begin{align}
p_{Y}\brak{k}= \begin{cases} 
      \frac{1}{3} & {k=0} \\
      \frac{2}{3 }& {k=1} 
   \end{cases}
   \\
p_{Y|X}\brak{0|0} = \frac{19}{25}\, 
p_{Y|X}\brak{0|1} = \frac{6}{25}\,
p_{Y|X}\brak{1|0} = \frac{45}{50}\,
p_{Y|X}\brak{1|2} = \frac{5}{50}
\end{align}
The desired probability is the probability that a slip drawn at random is marked other than Rs 1,
\begin{align}
&=1-p_X\brak{0}\\
&= p_X(1) + p_X(2)
\end{align}
Using Bayes theorem,
\begin{align}
&= p_Y\brak{0} \times \pr{Y=0 | X=1} + p_Y\brak{1} \times \pr{Y=1|X=2}\\
&=\frac{1}{3} \times \frac{6}{25} + \frac{2}{3} \times \frac{5}{50}\\
&=\frac{11}{75}
\end{align}

\newpage

%\tableofcontents

\bigskip

\renewcommand{\thefigure}{\theenumi}
\renewcommand{\thetable}{\theenumi}
%\renewcommand{\theequation}{\theenumi}

%\begin{abstract}
%%\boldmath
%In this letter, an algorithm for evaluating the exact analytical bit error rate  (BER)  for the piecewise linear (PL) combiner for  multiple relays is presented. Previous results were available only for upto three relays. The algorithm is unique in the sense that  the actual mathematical expressions, that are prohibitively large, need not be explicitly obtained. The diversity gain due to multiple relays is shown through plots of the analytical BER, well supported by simulations. 
%
%\end{abstract}
% IEEEtran.cls defaults to using nonbold math in the Abstract.
% This preserves the distinction between vectors and scalars. However,
% if the journal you are submitting to favors bold math in the abstract,
% then you can use LaTeX's standard command \boldmath at the very start
% of the abstract to achieve this. Many IEEE journals frown on math
% in the abstract anyway.

% Note that keywords are not normally used for peerreview papers.
%\begin{IEEEkeywords}
%Cooperative diversity, decode and forward, piecewise linear
%\end{IEEEkeywords}



% For peer review papers, you can put extra information on the cover
% page as needed:
% \ifCLASSOPTIONpeerreview
% \begin{center} \bfseries EDICS Category: 3-BBND \end{center}
% \fi
%
% For peerreview papers, this IEEEtran command inserts a page break and
% creates the second title. It will be ignored for other modes.
%\IEEEpeerreviewmaketitle




	\item All the jacks, queens and kings are removed from a deck of 52 playing cards. The remaining cards are well shuffled and then one card is drawn at random. Giving ace a value 1 similar value for other cards, find the probability that the card has a value 
		\begin{enumerate}
			\item 7
			\item greater than 7
			\item less than 7
		\end{enumerate}
		%Number of cards left after removing all jacks, queens and kings 
\begin{align}
N	= 52 - 4\times 3
	= 40
\end{align}
%\begin{table}[H]
%\def\arraystretch{1.2}
%\begin{tabular}{|c|c|c|}
%\hline
%	\textbf{Parameter} &\textbf{Value} &\textbf{Description}\\ \hline
%	$X$ &1-10 &Represents the value of the card picked \\ \hline
%\end{tabular}
%\end{table}
Let $1 \le X \le 10$ be the value of the card picked.  Then,
\begin{align}
	p_X(k) &= \Pr(X=k)\ \forall\ 1 \leq k \leq 10\\
	&= \frac{4\times 1}{40}\\
	&= \frac{1}{10}\\
	\therefore p_X(k) &= 
	\begin{cases}
		\frac{1}{10} & 1 \leq k \leq 10\\
		0 & \text{otherwise}
	\end{cases}
\end{align}
and
\begin{align}
	F_{X}(k) &= \sum_{m=0}^{k}p_{X}(m) \quad 1 \leq k \leq 10\\
	&= \frac{k}{10}\\
	\therefore F_{X}(k) &= 
	\begin{cases}
		0 & k \leq 0\\
		\frac{k}{10} & 1\leq k \leq 10\\
		1 & k > 10 
	\end{cases}
\end{align}
\begin{enumerate}
	\item Probability that card has value equal to 7 is
		\begin{align}
			 p_{X}(7)
			= \frac{1}{10}
		\end{align}
	\item Probability that card has value greater than 7 is
		\begin{align}
			1 - F_X(7)
			&= 1 - \frac{7}{10}
			\\
			&= \frac{3}{10}
		\end{align}
	\item Probability that card has value less than 7 is
		\begin{align}
			 F_{X}(6)
			=\frac{6}{10}
		\end{align}
\end{enumerate}

  \item A Lot consists of 48 mobile phones of which 42 are good, 3 have only minor defects and 3 have major defects.Varnika will buy a phone if it is good but the trader will only buy a mobile if it has no major defects. One phone is selected at random from the lot. What is the probability that it is
\begin{enumerate}
	\item acceptable to Varnika?
            \item acceptable to the trader?
\end{enumerate}
\solution
	%\begin{table}[H]
	\centering
\begin{tabular}{|c|c|c|}
\hline
Random variable &Value &Definition\\ \hline
\multirow{3}{*}{X} &0 &Slips of Rs 1\\
&1 &Slips of Rs 5\\
&2 &Slips of Rs 13\\ \hline
\multirow{2}{*}{Y} &0 &Box A\\
&1 &Box B\\\hline
\end{tabular}
\caption{}
\label{tab:Distribution}
\end{table}
See \tabref{tab:Distribution}.
\begin{align}
p_{Y}\brak{k}= \begin{cases} 
      \frac{1}{3} & {k=0} \\
      \frac{2}{3 }& {k=1} 
   \end{cases}
   \\
p_{Y|X}\brak{0|0} = \frac{19}{25}\, 
p_{Y|X}\brak{0|1} = \frac{6}{25}\,
p_{Y|X}\brak{1|0} = \frac{45}{50}\,
p_{Y|X}\brak{1|2} = \frac{5}{50}
\end{align}
The desired probability is the probability that a slip drawn at random is marked other than Rs 1,
\begin{align}
&=1-p_X\brak{0}\\
&= p_X(1) + p_X(2)
\end{align}
Using Bayes theorem,
\begin{align}
&= p_Y\brak{0} \times \pr{Y=0 | X=1} + p_Y\brak{1} \times \pr{Y=1|X=2}\\
&=\frac{1}{3} \times \frac{6}{25} + \frac{2}{3} \times \frac{5}{50}\\
&=\frac{11}{75}
\end{align}

\newpage

%\tableofcontents

\bigskip

\renewcommand{\thefigure}{\theenumi}
\renewcommand{\thetable}{\theenumi}
%\renewcommand{\theequation}{\theenumi}

%\begin{abstract}
%%\boldmath
%In this letter, an algorithm for evaluating the exact analytical bit error rate  (BER)  for the piecewise linear (PL) combiner for  multiple relays is presented. Previous results were available only for upto three relays. The algorithm is unique in the sense that  the actual mathematical expressions, that are prohibitively large, need not be explicitly obtained. The diversity gain due to multiple relays is shown through plots of the analytical BER, well supported by simulations. 
%
%\end{abstract}
% IEEEtran.cls defaults to using nonbold math in the Abstract.
% This preserves the distinction between vectors and scalars. However,
% if the journal you are submitting to favors bold math in the abstract,
% then you can use LaTeX's standard command \boldmath at the very start
% of the abstract to achieve this. Many IEEE journals frown on math
% in the abstract anyway.

% Note that keywords are not normally used for peerreview papers.
%\begin{IEEEkeywords}
%Cooperative diversity, decode and forward, piecewise linear
%\end{IEEEkeywords}



% For peer review papers, you can put extra information on the cover
% page as needed:
% \ifCLASSOPTIONpeerreview
% \begin{center} \bfseries EDICS Category: 3-BBND \end{center}
% \fi
%
% For peerreview papers, this IEEEtran command inserts a page break and
% creates the second title. It will be ignored for other modes.
%\IEEEpeerreviewmaketitle




 \item A student says that if you throw a die, it will show up 1 or not 1. Therefore, the probability of getting 1 and the probability of getting 'not 1' each is equal to $\frac{1}{2}$. Is this correct? Give reasons.\\
 \solution
        %\begin{table}[H]
	\centering
\begin{tabular}{|c|c|c|}
\hline
Random variable &Value &Definition\\ \hline
\multirow{3}{*}{X} &0 &Slips of Rs 1\\
&1 &Slips of Rs 5\\
&2 &Slips of Rs 13\\ \hline
\multirow{2}{*}{Y} &0 &Box A\\
&1 &Box B\\\hline
\end{tabular}
\caption{}
\label{tab:Distribution}
\end{table}
See \tabref{tab:Distribution}.
\begin{align}
p_{Y}\brak{k}= \begin{cases} 
      \frac{1}{3} & {k=0} \\
      \frac{2}{3 }& {k=1} 
   \end{cases}
   \\
p_{Y|X}\brak{0|0} = \frac{19}{25}\, 
p_{Y|X}\brak{0|1} = \frac{6}{25}\,
p_{Y|X}\brak{1|0} = \frac{45}{50}\,
p_{Y|X}\brak{1|2} = \frac{5}{50}
\end{align}
The desired probability is the probability that a slip drawn at random is marked other than Rs 1,
\begin{align}
&=1-p_X\brak{0}\\
&= p_X(1) + p_X(2)
\end{align}
Using Bayes theorem,
\begin{align}
&= p_Y\brak{0} \times \pr{Y=0 | X=1} + p_Y\brak{1} \times \pr{Y=1|X=2}\\
&=\frac{1}{3} \times \frac{6}{25} + \frac{2}{3} \times \frac{5}{50}\\
&=\frac{11}{75}
\end{align}

\newpage

%\tableofcontents

\bigskip

\renewcommand{\thefigure}{\theenumi}
\renewcommand{\thetable}{\theenumi}
%\renewcommand{\theequation}{\theenumi}

%\begin{abstract}
%%\boldmath
%In this letter, an algorithm for evaluating the exact analytical bit error rate  (BER)  for the piecewise linear (PL) combiner for  multiple relays is presented. Previous results were available only for upto three relays. The algorithm is unique in the sense that  the actual mathematical expressions, that are prohibitively large, need not be explicitly obtained. The diversity gain due to multiple relays is shown through plots of the analytical BER, well supported by simulations. 
%
%\end{abstract}
% IEEEtran.cls defaults to using nonbold math in the Abstract.
% This preserves the distinction between vectors and scalars. However,
% if the journal you are submitting to favors bold math in the abstract,
% then you can use LaTeX's standard command \boldmath at the very start
% of the abstract to achieve this. Many IEEE journals frown on math
% in the abstract anyway.

% Note that keywords are not normally used for peerreview papers.
%\begin{IEEEkeywords}
%Cooperative diversity, decode and forward, piecewise linear
%\end{IEEEkeywords}



% For peer review papers, you can put extra information on the cover
% page as needed:
% \ifCLASSOPTIONpeerreview
% \begin{center} \bfseries EDICS Category: 3-BBND \end{center}
% \fi
%
% For peerreview papers, this IEEEtran command inserts a page break and
% creates the second title. It will be ignored for other modes.
%\IEEEpeerreviewmaketitle




   \item Four candidates A, B, C, D have ap-
plied for the assignment to coach a school cricket
team. If A is twice as likely to be selected as B, and
B and C are given about the same chance of being
selected, while C is twice as likely to be selected
as D, what are the probabilities that
\begin{enumerate}
\item C will be selected?
\item A will not be selected?
\end{enumerate}
	%\begin{table}[H]
	\centering
\begin{tabular}{|c|c|c|}
\hline
Random variable &Value &Definition\\ \hline
\multirow{3}{*}{X} &0 &Slips of Rs 1\\
&1 &Slips of Rs 5\\
&2 &Slips of Rs 13\\ \hline
\multirow{2}{*}{Y} &0 &Box A\\
&1 &Box B\\\hline
\end{tabular}
\caption{}
\label{tab:Distribution}
\end{table}
See \tabref{tab:Distribution}.
\begin{align}
p_{Y}\brak{k}= \begin{cases} 
      \frac{1}{3} & {k=0} \\
      \frac{2}{3 }& {k=1} 
   \end{cases}
   \\
p_{Y|X}\brak{0|0} = \frac{19}{25}\, 
p_{Y|X}\brak{0|1} = \frac{6}{25}\,
p_{Y|X}\brak{1|0} = \frac{45}{50}\,
p_{Y|X}\brak{1|2} = \frac{5}{50}
\end{align}
The desired probability is the probability that a slip drawn at random is marked other than Rs 1,
\begin{align}
&=1-p_X\brak{0}\\
&= p_X(1) + p_X(2)
\end{align}
Using Bayes theorem,
\begin{align}
&= p_Y\brak{0} \times \pr{Y=0 | X=1} + p_Y\brak{1} \times \pr{Y=1|X=2}\\
&=\frac{1}{3} \times \frac{6}{25} + \frac{2}{3} \times \frac{5}{50}\\
&=\frac{11}{75}
\end{align}

\newpage

%\tableofcontents

\bigskip

\renewcommand{\thefigure}{\theenumi}
\renewcommand{\thetable}{\theenumi}
%\renewcommand{\theequation}{\theenumi}

%\begin{abstract}
%%\boldmath
%In this letter, an algorithm for evaluating the exact analytical bit error rate  (BER)  for the piecewise linear (PL) combiner for  multiple relays is presented. Previous results were available only for upto three relays. The algorithm is unique in the sense that  the actual mathematical expressions, that are prohibitively large, need not be explicitly obtained. The diversity gain due to multiple relays is shown through plots of the analytical BER, well supported by simulations. 
%
%\end{abstract}
% IEEEtran.cls defaults to using nonbold math in the Abstract.
% This preserves the distinction between vectors and scalars. However,
% if the journal you are submitting to favors bold math in the abstract,
% then you can use LaTeX's standard command \boldmath at the very start
% of the abstract to achieve this. Many IEEE journals frown on math
% in the abstract anyway.

% Note that keywords are not normally used for peerreview papers.
%\begin{IEEEkeywords}
%Cooperative diversity, decode and forward, piecewise linear
%\end{IEEEkeywords}



% For peer review papers, you can put extra information on the cover
% page as needed:
% \ifCLASSOPTIONpeerreview
% \begin{center} \bfseries EDICS Category: 3-BBND \end{center}
% \fi
%
% For peerreview papers, this IEEEtran command inserts a page break and
% creates the second title. It will be ignored for other modes.
%\IEEEpeerreviewmaketitle




 \item A bag contain 24 balls of which $x$ balls are red, $2x$ are white and $3x$ are blue. A ball is selected at random, What is the probability that it is
\begin{enumerate}[label=\alph*)]
\item not red ?
\item white ?
\end{enumerate}
%\begin{table}[H]
	\centering
\begin{tabular}{|c|c|c|}
\hline
Random variable &Value &Definition\\ \hline
\multirow{3}{*}{X} &0 &Slips of Rs 1\\
&1 &Slips of Rs 5\\
&2 &Slips of Rs 13\\ \hline
\multirow{2}{*}{Y} &0 &Box A\\
&1 &Box B\\\hline
\end{tabular}
\caption{}
\label{tab:Distribution}
\end{table}
See \tabref{tab:Distribution}.
\begin{align}
p_{Y}\brak{k}= \begin{cases} 
      \frac{1}{3} & {k=0} \\
      \frac{2}{3 }& {k=1} 
   \end{cases}
   \\
p_{Y|X}\brak{0|0} = \frac{19}{25}\, 
p_{Y|X}\brak{0|1} = \frac{6}{25}\,
p_{Y|X}\brak{1|0} = \frac{45}{50}\,
p_{Y|X}\brak{1|2} = \frac{5}{50}
\end{align}
The desired probability is the probability that a slip drawn at random is marked other than Rs 1,
\begin{align}
&=1-p_X\brak{0}\\
&= p_X(1) + p_X(2)
\end{align}
Using Bayes theorem,
\begin{align}
&= p_Y\brak{0} \times \pr{Y=0 | X=1} + p_Y\brak{1} \times \pr{Y=1|X=2}\\
&=\frac{1}{3} \times \frac{6}{25} + \frac{2}{3} \times \frac{5}{50}\\
&=\frac{11}{75}
\end{align}

\newpage

%\tableofcontents

\bigskip

\renewcommand{\thefigure}{\theenumi}
\renewcommand{\thetable}{\theenumi}
%\renewcommand{\theequation}{\theenumi}

%\begin{abstract}
%%\boldmath
%In this letter, an algorithm for evaluating the exact analytical bit error rate  (BER)  for the piecewise linear (PL) combiner for  multiple relays is presented. Previous results were available only for upto three relays. The algorithm is unique in the sense that  the actual mathematical expressions, that are prohibitively large, need not be explicitly obtained. The diversity gain due to multiple relays is shown through plots of the analytical BER, well supported by simulations. 
%
%\end{abstract}
% IEEEtran.cls defaults to using nonbold math in the Abstract.
% This preserves the distinction between vectors and scalars. However,
% if the journal you are submitting to favors bold math in the abstract,
% then you can use LaTeX's standard command \boldmath at the very start
% of the abstract to achieve this. Many IEEE journals frown on math
% in the abstract anyway.

% Note that keywords are not normally used for peerreview papers.
%\begin{IEEEkeywords}
%Cooperative diversity, decode and forward, piecewise linear
%\end{IEEEkeywords}



% For peer review papers, you can put extra information on the cover
% page as needed:
% \ifCLASSOPTIONpeerreview
% \begin{center} \bfseries EDICS Category: 3-BBND \end{center}
% \fi
%
% For peerreview papers, this IEEEtran command inserts a page break and
% creates the second title. It will be ignored for other modes.
%\IEEEpeerreviewmaketitle




If the letters of the word ASSASSINATION are arranged at random. Find the Probability that
\begin{enumerate}[label=(\alph*)]
\item Four $S's$ come consecutively in the word
\item Two  $I's$ and two $N's$ come together
\item All $A's$ are not coming together
\item No two $A's$ are coming together
\end{enumerate}
%\begin{table}[H]
	\centering
\begin{tabular}{|c|c|c|}
\hline
Random variable &Value &Definition\\ \hline
\multirow{3}{*}{X} &0 &Slips of Rs 1\\
&1 &Slips of Rs 5\\
&2 &Slips of Rs 13\\ \hline
\multirow{2}{*}{Y} &0 &Box A\\
&1 &Box B\\\hline
\end{tabular}
\caption{}
\label{tab:Distribution}
\end{table}
See \tabref{tab:Distribution}.
\begin{align}
p_{Y}\brak{k}= \begin{cases} 
      \frac{1}{3} & {k=0} \\
      \frac{2}{3 }& {k=1} 
   \end{cases}
   \\
p_{Y|X}\brak{0|0} = \frac{19}{25}\, 
p_{Y|X}\brak{0|1} = \frac{6}{25}\,
p_{Y|X}\brak{1|0} = \frac{45}{50}\,
p_{Y|X}\brak{1|2} = \frac{5}{50}
\end{align}
The desired probability is the probability that a slip drawn at random is marked other than Rs 1,
\begin{align}
&=1-p_X\brak{0}\\
&= p_X(1) + p_X(2)
\end{align}
Using Bayes theorem,
\begin{align}
&= p_Y\brak{0} \times \pr{Y=0 | X=1} + p_Y\brak{1} \times \pr{Y=1|X=2}\\
&=\frac{1}{3} \times \frac{6}{25} + \frac{2}{3} \times \frac{5}{50}\\
&=\frac{11}{75}
\end{align}

\newpage

%\tableofcontents

\bigskip

\renewcommand{\thefigure}{\theenumi}
\renewcommand{\thetable}{\theenumi}
%\renewcommand{\theequation}{\theenumi}

%\begin{abstract}
%%\boldmath
%In this letter, an algorithm for evaluating the exact analytical bit error rate  (BER)  for the piecewise linear (PL) combiner for  multiple relays is presented. Previous results were available only for upto three relays. The algorithm is unique in the sense that  the actual mathematical expressions, that are prohibitively large, need not be explicitly obtained. The diversity gain due to multiple relays is shown through plots of the analytical BER, well supported by simulations. 
%
%\end{abstract}
% IEEEtran.cls defaults to using nonbold math in the Abstract.
% This preserves the distinction between vectors and scalars. However,
% if the journal you are submitting to favors bold math in the abstract,
% then you can use LaTeX's standard command \boldmath at the very start
% of the abstract to achieve this. Many IEEE journals frown on math
% in the abstract anyway.

% Note that keywords are not normally used for peerreview papers.
%\begin{IEEEkeywords}
%Cooperative diversity, decode and forward, piecewise linear
%\end{IEEEkeywords}



% For peer review papers, you can put extra information on the cover
% page as needed:
% \ifCLASSOPTIONpeerreview
% \begin{center} \bfseries EDICS Category: 3-BBND \end{center}
% \fi
%
% For peerreview papers, this IEEEtran command inserts a page break and
% creates the second title. It will be ignored for other modes.
%\IEEEpeerreviewmaketitle




	\item One urn contains two black balls (labelled B1 and B2) and one white ball. A
	second urn contains one black ball and two white balls (labelled W1 and W2).
	Suppose the following experiment is performed. One of the two urns is chosen
	at random. Next a ball is randomly chosen from the urn. Then a second ball is
	chosen at random from the same urn without replacing the first ball.
	
	\begin{enumerate}
	\item What is the probability that two black balls are chosen?
	
	\item What is the probability that two balls of opposite colour are chosen?
	\end{enumerate}
	\solution
	%\begin{align}
    \label{eq:12.13.6.18.1}
	\because	\pr{A|B} &> \pr{A},\
\frac{\pr{AB}}{\pr{B}} > \pr{A}
\\
    \label{eq:12.13.6.18.2}
	\implies \pr{AB} &> \pr{A}\pr{B}
	\\
	\text{or, } \frac{\pr{AB}}{\pr{A}} &=\pr{B|A} > \pr{A}
\end{align}

\end{enumerate}

	\item A bag contains 4 red and 4 black balls, another bag contains 2 red and 6 black balls. One of the two bags is selected at random and a ball is drawn from the bag which is found to be red. Find the probability that the ball is drawn from the first bag.
\\
\solution
		%\begin{table}[H]
	\centering
\begin{tabular}{|c|c|c|}
\hline
Random variable &Value &Definition\\ \hline
\multirow{3}{*}{X} &0 &Slips of Rs 1\\
&1 &Slips of Rs 5\\
&2 &Slips of Rs 13\\ \hline
\multirow{2}{*}{Y} &0 &Box A\\
&1 &Box B\\\hline
\end{tabular}
\caption{}
\label{tab:Distribution}
\end{table}
See \tabref{tab:Distribution}.
\begin{align}
p_{Y}\brak{k}= \begin{cases} 
      \frac{1}{3} & {k=0} \\
      \frac{2}{3 }& {k=1} 
   \end{cases}
   \\
p_{Y|X}\brak{0|0} = \frac{19}{25}\, 
p_{Y|X}\brak{0|1} = \frac{6}{25}\,
p_{Y|X}\brak{1|0} = \frac{45}{50}\,
p_{Y|X}\brak{1|2} = \frac{5}{50}
\end{align}
The desired probability is the probability that a slip drawn at random is marked other than Rs 1,
\begin{align}
&=1-p_X\brak{0}\\
&= p_X(1) + p_X(2)
\end{align}
Using Bayes theorem,
\begin{align}
&= p_Y\brak{0} \times \pr{Y=0 | X=1} + p_Y\brak{1} \times \pr{Y=1|X=2}\\
&=\frac{1}{3} \times \frac{6}{25} + \frac{2}{3} \times \frac{5}{50}\\
&=\frac{11}{75}
\end{align}

\newpage

%\tableofcontents

\bigskip

\renewcommand{\thefigure}{\theenumi}
\renewcommand{\thetable}{\theenumi}
%\renewcommand{\theequation}{\theenumi}

%\begin{abstract}
%%\boldmath
%In this letter, an algorithm for evaluating the exact analytical bit error rate  (BER)  for the piecewise linear (PL) combiner for  multiple relays is presented. Previous results were available only for upto three relays. The algorithm is unique in the sense that  the actual mathematical expressions, that are prohibitively large, need not be explicitly obtained. The diversity gain due to multiple relays is shown through plots of the analytical BER, well supported by simulations. 
%
%\end{abstract}
% IEEEtran.cls defaults to using nonbold math in the Abstract.
% This preserves the distinction between vectors and scalars. However,
% if the journal you are submitting to favors bold math in the abstract,
% then you can use LaTeX's standard command \boldmath at the very start
% of the abstract to achieve this. Many IEEE journals frown on math
% in the abstract anyway.

% Note that keywords are not normally used for peerreview papers.
%\begin{IEEEkeywords}
%Cooperative diversity, decode and forward, piecewise linear
%\end{IEEEkeywords}



% For peer review papers, you can put extra information on the cover
% page as needed:
% \ifCLASSOPTIONpeerreview
% \begin{center} \bfseries EDICS Category: 3-BBND \end{center}
% \fi
%
% For peerreview papers, this IEEEtran command inserts a page break and
% creates the second title. It will be ignored for other modes.
%\IEEEpeerreviewmaketitle




  \item
  Cards with numbers 2 to 101 are placed in a box. A card is selected at random.Find the probability that the card has
\begin{enumerate}[label=(\roman*)]
	\item an even number 
	\item a square number
\end{enumerate}
\solution
%\begin{table}[H]
	\centering
\begin{tabular}{|c|c|c|}
\hline
Random variable &Value &Definition\\ \hline
\multirow{3}{*}{X} &0 &Slips of Rs 1\\
&1 &Slips of Rs 5\\
&2 &Slips of Rs 13\\ \hline
\multirow{2}{*}{Y} &0 &Box A\\
&1 &Box B\\\hline
\end{tabular}
\caption{}
\label{tab:Distribution}
\end{table}
See \tabref{tab:Distribution}.
\begin{align}
p_{Y}\brak{k}= \begin{cases} 
      \frac{1}{3} & {k=0} \\
      \frac{2}{3 }& {k=1} 
   \end{cases}
   \\
p_{Y|X}\brak{0|0} = \frac{19}{25}\, 
p_{Y|X}\brak{0|1} = \frac{6}{25}\,
p_{Y|X}\brak{1|0} = \frac{45}{50}\,
p_{Y|X}\brak{1|2} = \frac{5}{50}
\end{align}
The desired probability is the probability that a slip drawn at random is marked other than Rs 1,
\begin{align}
&=1-p_X\brak{0}\\
&= p_X(1) + p_X(2)
\end{align}
Using Bayes theorem,
\begin{align}
&= p_Y\brak{0} \times \pr{Y=0 | X=1} + p_Y\brak{1} \times \pr{Y=1|X=2}\\
&=\frac{1}{3} \times \frac{6}{25} + \frac{2}{3} \times \frac{5}{50}\\
&=\frac{11}{75}
\end{align}

\newpage

%\tableofcontents

\bigskip

\renewcommand{\thefigure}{\theenumi}
\renewcommand{\thetable}{\theenumi}
%\renewcommand{\theequation}{\theenumi}

%\begin{abstract}
%%\boldmath
%In this letter, an algorithm for evaluating the exact analytical bit error rate  (BER)  for the piecewise linear (PL) combiner for  multiple relays is presented. Previous results were available only for upto three relays. The algorithm is unique in the sense that  the actual mathematical expressions, that are prohibitively large, need not be explicitly obtained. The diversity gain due to multiple relays is shown through plots of the analytical BER, well supported by simulations. 
%
%\end{abstract}
% IEEEtran.cls defaults to using nonbold math in the Abstract.
% This preserves the distinction between vectors and scalars. However,
% if the journal you are submitting to favors bold math in the abstract,
% then you can use LaTeX's standard command \boldmath at the very start
% of the abstract to achieve this. Many IEEE journals frown on math
% in the abstract anyway.

% Note that keywords are not normally used for peerreview papers.
%\begin{IEEEkeywords}
%Cooperative diversity, decode and forward, piecewise linear
%\end{IEEEkeywords}



% For peer review papers, you can put extra information on the cover
% page as needed:
% \ifCLASSOPTIONpeerreview
% \begin{center} \bfseries EDICS Category: 3-BBND \end{center}
% \fi
%
% For peerreview papers, this IEEEtran command inserts a page break and
% creates the second title. It will be ignored for other modes.
%\IEEEpeerreviewmaketitle




\item
The king, queen and jack of clubs are removed from a deck of 52 playing cards and then well shuffled. Now one card is drawn at random from the remaining cards.  Determine the probability that the card is
\begin{enumerate}[label=(\roman*)]
\item a club
\item 10 of hearts
\end{enumerate}
\solution
%\begin{table}[H]
	\centering
\begin{tabular}{|c|c|c|}
\hline
Random variable &Value &Definition\\ \hline
\multirow{3}{*}{X} &0 &Slips of Rs 1\\
&1 &Slips of Rs 5\\
&2 &Slips of Rs 13\\ \hline
\multirow{2}{*}{Y} &0 &Box A\\
&1 &Box B\\\hline
\end{tabular}
\caption{}
\label{tab:Distribution}
\end{table}
See \tabref{tab:Distribution}.
\begin{align}
p_{Y}\brak{k}= \begin{cases} 
      \frac{1}{3} & {k=0} \\
      \frac{2}{3 }& {k=1} 
   \end{cases}
   \\
p_{Y|X}\brak{0|0} = \frac{19}{25}\, 
p_{Y|X}\brak{0|1} = \frac{6}{25}\,
p_{Y|X}\brak{1|0} = \frac{45}{50}\,
p_{Y|X}\brak{1|2} = \frac{5}{50}
\end{align}
The desired probability is the probability that a slip drawn at random is marked other than Rs 1,
\begin{align}
&=1-p_X\brak{0}\\
&= p_X(1) + p_X(2)
\end{align}
Using Bayes theorem,
\begin{align}
&= p_Y\brak{0} \times \pr{Y=0 | X=1} + p_Y\brak{1} \times \pr{Y=1|X=2}\\
&=\frac{1}{3} \times \frac{6}{25} + \frac{2}{3} \times \frac{5}{50}\\
&=\frac{11}{75}
\end{align}

\newpage

%\tableofcontents

\bigskip

\renewcommand{\thefigure}{\theenumi}
\renewcommand{\thetable}{\theenumi}
%\renewcommand{\theequation}{\theenumi}

%\begin{abstract}
%%\boldmath
%In this letter, an algorithm for evaluating the exact analytical bit error rate  (BER)  for the piecewise linear (PL) combiner for  multiple relays is presented. Previous results were available only for upto three relays. The algorithm is unique in the sense that  the actual mathematical expressions, that are prohibitively large, need not be explicitly obtained. The diversity gain due to multiple relays is shown through plots of the analytical BER, well supported by simulations. 
%
%\end{abstract}
% IEEEtran.cls defaults to using nonbold math in the Abstract.
% This preserves the distinction between vectors and scalars. However,
% if the journal you are submitting to favors bold math in the abstract,
% then you can use LaTeX's standard command \boldmath at the very start
% of the abstract to achieve this. Many IEEE journals frown on math
% in the abstract anyway.

% Note that keywords are not normally used for peerreview papers.
%\begin{IEEEkeywords}
%Cooperative diversity, decode and forward, piecewise linear
%\end{IEEEkeywords}



% For peer review papers, you can put extra information on the cover
% page as needed:
% \ifCLASSOPTIONpeerreview
% \begin{center} \bfseries EDICS Category: 3-BBND \end{center}
% \fi
%
% For peerreview papers, this IEEEtran command inserts a page break and
% creates the second title. It will be ignored for other modes.
%\IEEEpeerreviewmaketitle




\item A team of medical students doing their internship have to assist during surgeries
at a city hospital. The probabilities of surgeries rated as very complex, complex,
routine, simple or very simple are respectively, 0.15, 0.20, 0.31, 0.26, .08. Find
the probabilities that a particular surgery will be rated
\begin{enumerate}
	\item complex or very complex;
	\item neither very complex nor very simple;
	\item routine or complex
	\item routine or simple
\end{enumerate}
\solution
%\begin{table}[H]
	\centering
\begin{tabular}{|c|c|c|}
\hline
Random variable &Value &Definition\\ \hline
\multirow{3}{*}{X} &0 &Slips of Rs 1\\
&1 &Slips of Rs 5\\
&2 &Slips of Rs 13\\ \hline
\multirow{2}{*}{Y} &0 &Box A\\
&1 &Box B\\\hline
\end{tabular}
\caption{}
\label{tab:Distribution}
\end{table}
See \tabref{tab:Distribution}.
\begin{align}
p_{Y}\brak{k}= \begin{cases} 
      \frac{1}{3} & {k=0} \\
      \frac{2}{3 }& {k=1} 
   \end{cases}
   \\
p_{Y|X}\brak{0|0} = \frac{19}{25}\, 
p_{Y|X}\brak{0|1} = \frac{6}{25}\,
p_{Y|X}\brak{1|0} = \frac{45}{50}\,
p_{Y|X}\brak{1|2} = \frac{5}{50}
\end{align}
The desired probability is the probability that a slip drawn at random is marked other than Rs 1,
\begin{align}
&=1-p_X\brak{0}\\
&= p_X(1) + p_X(2)
\end{align}
Using Bayes theorem,
\begin{align}
&= p_Y\brak{0} \times \pr{Y=0 | X=1} + p_Y\brak{1} \times \pr{Y=1|X=2}\\
&=\frac{1}{3} \times \frac{6}{25} + \frac{2}{3} \times \frac{5}{50}\\
&=\frac{11}{75}
\end{align}

\newpage

%\tableofcontents

\bigskip

\renewcommand{\thefigure}{\theenumi}
\renewcommand{\thetable}{\theenumi}
%\renewcommand{\theequation}{\theenumi}

%\begin{abstract}
%%\boldmath
%In this letter, an algorithm for evaluating the exact analytical bit error rate  (BER)  for the piecewise linear (PL) combiner for  multiple relays is presented. Previous results were available only for upto three relays. The algorithm is unique in the sense that  the actual mathematical expressions, that are prohibitively large, need not be explicitly obtained. The diversity gain due to multiple relays is shown through plots of the analytical BER, well supported by simulations. 
%
%\end{abstract}
% IEEEtran.cls defaults to using nonbold math in the Abstract.
% This preserves the distinction between vectors and scalars. However,
% if the journal you are submitting to favors bold math in the abstract,
% then you can use LaTeX's standard command \boldmath at the very start
% of the abstract to achieve this. Many IEEE journals frown on math
% in the abstract anyway.

% Note that keywords are not normally used for peerreview papers.
%\begin{IEEEkeywords}
%Cooperative diversity, decode and forward, piecewise linear
%\end{IEEEkeywords}



% For peer review papers, you can put extra information on the cover
% page as needed:
% \ifCLASSOPTIONpeerreview
% \begin{center} \bfseries EDICS Category: 3-BBND \end{center}
% \fi
%
% For peerreview papers, this IEEEtran command inserts a page break and
% creates the second title. It will be ignored for other modes.
%\IEEEpeerreviewmaketitle




\item A card is selected from a pack of 52 cards.
\begin{enumerate}[label=(\alph*)]
    \item How many points are there in the sample space?
    \item Calculate the probability that the card is an ace of spades.
    \item Calculate the probability that the card is (i) an ace and (ii) black card.
\end{enumerate}
\solution
%Let $X$ be an bernoulli rv defined as in \tabref{tab:exemplar/11/16/3/26}.  Then, 
\begin{equation}
    p =
        \frac{4}{11} 
\end{equation}
\begin{table}[H]
	\centering
	\input{exemplar/11/16/3/26/tables/Table2.tex}
	\caption{}
        \label{tab:exemplar/11/16/3/26}
\end{table}

\item The probability that a non leap year selected at random will contain 53 sundays.
\\
\solution
%\begin{table}[H]
	\centering
\begin{tabular}{|c|c|c|}
\hline
Random variable &Value &Definition\\ \hline
\multirow{3}{*}{X} &0 &Slips of Rs 1\\
&1 &Slips of Rs 5\\
&2 &Slips of Rs 13\\ \hline
\multirow{2}{*}{Y} &0 &Box A\\
&1 &Box B\\\hline
\end{tabular}
\caption{}
\label{tab:Distribution}
\end{table}
See \tabref{tab:Distribution}.
\begin{align}
p_{Y}\brak{k}= \begin{cases} 
      \frac{1}{3} & {k=0} \\
      \frac{2}{3 }& {k=1} 
   \end{cases}
   \\
p_{Y|X}\brak{0|0} = \frac{19}{25}\, 
p_{Y|X}\brak{0|1} = \frac{6}{25}\,
p_{Y|X}\brak{1|0} = \frac{45}{50}\,
p_{Y|X}\brak{1|2} = \frac{5}{50}
\end{align}
The desired probability is the probability that a slip drawn at random is marked other than Rs 1,
\begin{align}
&=1-p_X\brak{0}\\
&= p_X(1) + p_X(2)
\end{align}
Using Bayes theorem,
\begin{align}
&= p_Y\brak{0} \times \pr{Y=0 | X=1} + p_Y\brak{1} \times \pr{Y=1|X=2}\\
&=\frac{1}{3} \times \frac{6}{25} + \frac{2}{3} \times \frac{5}{50}\\
&=\frac{11}{75}
\end{align}

\newpage

%\tableofcontents

\bigskip

\renewcommand{\thefigure}{\theenumi}
\renewcommand{\thetable}{\theenumi}
%\renewcommand{\theequation}{\theenumi}

%\begin{abstract}
%%\boldmath
%In this letter, an algorithm for evaluating the exact analytical bit error rate  (BER)  for the piecewise linear (PL) combiner for  multiple relays is presented. Previous results were available only for upto three relays. The algorithm is unique in the sense that  the actual mathematical expressions, that are prohibitively large, need not be explicitly obtained. The diversity gain due to multiple relays is shown through plots of the analytical BER, well supported by simulations. 
%
%\end{abstract}
% IEEEtran.cls defaults to using nonbold math in the Abstract.
% This preserves the distinction between vectors and scalars. However,
% if the journal you are submitting to favors bold math in the abstract,
% then you can use LaTeX's standard command \boldmath at the very start
% of the abstract to achieve this. Many IEEE journals frown on math
% in the abstract anyway.

% Note that keywords are not normally used for peerreview papers.
%\begin{IEEEkeywords}
%Cooperative diversity, decode and forward, piecewise linear
%\end{IEEEkeywords}



% For peer review papers, you can put extra information on the cover
% page as needed:
% \ifCLASSOPTIONpeerreview
% \begin{center} \bfseries EDICS Category: 3-BBND \end{center}
% \fi
%
% For peerreview papers, this IEEEtran command inserts a page break and
% creates the second title. It will be ignored for other modes.
%\IEEEpeerreviewmaketitle




\item One of the four persons John, Rita, Aslam or Gurpreet will be promoted next
month. Consequently the sample space consists of four elementary outcomes
S = {John promoted, Rita promoted, Aslam promoted, Gurpreet promoted}
You are told that the chances of John’s promotion is same as that of Gurpreet,
Rita’s chances of promotion are twice as likely as Johns. Aslam’s chances are
four times that of John.
\begin{enumerate}
	\item Determine
	\begin{enumerate}
		\item P (John promoted)
		\item P (Rita promoted)
		\item P (Aslam promoted)
		\item P (Gurpreet promoted)
	\end{enumerate}
	\item If A = {John promoted or Gurpreet promoted}, find P (A).
\end{enumerate}
\solution
%\begin{table}[H]
	\centering
\begin{tabular}{|c|c|c|}
\hline
Random variable &Value &Definition\\ \hline
\multirow{3}{*}{X} &0 &Slips of Rs 1\\
&1 &Slips of Rs 5\\
&2 &Slips of Rs 13\\ \hline
\multirow{2}{*}{Y} &0 &Box A\\
&1 &Box B\\\hline
\end{tabular}
\caption{}
\label{tab:Distribution}
\end{table}
See \tabref{tab:Distribution}.
\begin{align}
p_{Y}\brak{k}= \begin{cases} 
      \frac{1}{3} & {k=0} \\
      \frac{2}{3 }& {k=1} 
   \end{cases}
   \\
p_{Y|X}\brak{0|0} = \frac{19}{25}\, 
p_{Y|X}\brak{0|1} = \frac{6}{25}\,
p_{Y|X}\brak{1|0} = \frac{45}{50}\,
p_{Y|X}\brak{1|2} = \frac{5}{50}
\end{align}
The desired probability is the probability that a slip drawn at random is marked other than Rs 1,
\begin{align}
&=1-p_X\brak{0}\\
&= p_X(1) + p_X(2)
\end{align}
Using Bayes theorem,
\begin{align}
&= p_Y\brak{0} \times \pr{Y=0 | X=1} + p_Y\brak{1} \times \pr{Y=1|X=2}\\
&=\frac{1}{3} \times \frac{6}{25} + \frac{2}{3} \times \frac{5}{50}\\
&=\frac{11}{75}
\end{align}

\newpage

%\tableofcontents

\bigskip

\renewcommand{\thefigure}{\theenumi}
\renewcommand{\thetable}{\theenumi}
%\renewcommand{\theequation}{\theenumi}

%\begin{abstract}
%%\boldmath
%In this letter, an algorithm for evaluating the exact analytical bit error rate  (BER)  for the piecewise linear (PL) combiner for  multiple relays is presented. Previous results were available only for upto three relays. The algorithm is unique in the sense that  the actual mathematical expressions, that are prohibitively large, need not be explicitly obtained. The diversity gain due to multiple relays is shown through plots of the analytical BER, well supported by simulations. 
%
%\end{abstract}
% IEEEtran.cls defaults to using nonbold math in the Abstract.
% This preserves the distinction between vectors and scalars. However,
% if the journal you are submitting to favors bold math in the abstract,
% then you can use LaTeX's standard command \boldmath at the very start
% of the abstract to achieve this. Many IEEE journals frown on math
% in the abstract anyway.

% Note that keywords are not normally used for peerreview papers.
%\begin{IEEEkeywords}
%Cooperative diversity, decode and forward, piecewise linear
%\end{IEEEkeywords}



% For peer review papers, you can put extra information on the cover
% page as needed:
% \ifCLASSOPTIONpeerreview
% \begin{center} \bfseries EDICS Category: 3-BBND \end{center}
% \fi
%
% For peerreview papers, this IEEEtran command inserts a page break and
% creates the second title. It will be ignored for other modes.
%\IEEEpeerreviewmaketitle




\item A card is drawn from a deck of 52 cards. Find the probability of getting a king or a heart or a red card.\\
\solution
%\begin{table}[H]
	\centering
\begin{tabular}{|c|c|c|}
\hline
Random variable &Value &Definition\\ \hline
\multirow{3}{*}{X} &0 &Slips of Rs 1\\
&1 &Slips of Rs 5\\
&2 &Slips of Rs 13\\ \hline
\multirow{2}{*}{Y} &0 &Box A\\
&1 &Box B\\\hline
\end{tabular}
\caption{}
\label{tab:Distribution}
\end{table}
See \tabref{tab:Distribution}.
\begin{align}
p_{Y}\brak{k}= \begin{cases} 
      \frac{1}{3} & {k=0} \\
      \frac{2}{3 }& {k=1} 
   \end{cases}
   \\
p_{Y|X}\brak{0|0} = \frac{19}{25}\, 
p_{Y|X}\brak{0|1} = \frac{6}{25}\,
p_{Y|X}\brak{1|0} = \frac{45}{50}\,
p_{Y|X}\brak{1|2} = \frac{5}{50}
\end{align}
The desired probability is the probability that a slip drawn at random is marked other than Rs 1,
\begin{align}
&=1-p_X\brak{0}\\
&= p_X(1) + p_X(2)
\end{align}
Using Bayes theorem,
\begin{align}
&= p_Y\brak{0} \times \pr{Y=0 | X=1} + p_Y\brak{1} \times \pr{Y=1|X=2}\\
&=\frac{1}{3} \times \frac{6}{25} + \frac{2}{3} \times \frac{5}{50}\\
&=\frac{11}{75}
\end{align}

\newpage

%\tableofcontents

\bigskip

\renewcommand{\thefigure}{\theenumi}
\renewcommand{\thetable}{\theenumi}
%\renewcommand{\theequation}{\theenumi}

%\begin{abstract}
%%\boldmath
%In this letter, an algorithm for evaluating the exact analytical bit error rate  (BER)  for the piecewise linear (PL) combiner for  multiple relays is presented. Previous results were available only for upto three relays. The algorithm is unique in the sense that  the actual mathematical expressions, that are prohibitively large, need not be explicitly obtained. The diversity gain due to multiple relays is shown through plots of the analytical BER, well supported by simulations. 
%
%\end{abstract}
% IEEEtran.cls defaults to using nonbold math in the Abstract.
% This preserves the distinction between vectors and scalars. However,
% if the journal you are submitting to favors bold math in the abstract,
% then you can use LaTeX's standard command \boldmath at the very start
% of the abstract to achieve this. Many IEEE journals frown on math
% in the abstract anyway.

% Note that keywords are not normally used for peerreview papers.
%\begin{IEEEkeywords}
%Cooperative diversity, decode and forward, piecewise linear
%\end{IEEEkeywords}



% For peer review papers, you can put extra information on the cover
% page as needed:
% \ifCLASSOPTIONpeerreview
% \begin{center} \bfseries EDICS Category: 3-BBND \end{center}
% \fi
%
% For peerreview papers, this IEEEtran command inserts a page break and
% creates the second title. It will be ignored for other modes.
%\IEEEpeerreviewmaketitle




\item The probability that a student will pass his examination is 0.73, the probability of
the student getting a compartment is 0.13, and the probability that the student will
either pass or get compartment is 0.96. State True or False.\\
\solution
%\begin{table}[H]
	\centering
\begin{tabular}{|c|c|c|}
\hline
Random variable &Value &Definition\\ \hline
\multirow{3}{*}{X} &0 &Slips of Rs 1\\
&1 &Slips of Rs 5\\
&2 &Slips of Rs 13\\ \hline
\multirow{2}{*}{Y} &0 &Box A\\
&1 &Box B\\\hline
\end{tabular}
\caption{}
\label{tab:Distribution}
\end{table}
See \tabref{tab:Distribution}.
\begin{align}
p_{Y}\brak{k}= \begin{cases} 
      \frac{1}{3} & {k=0} \\
      \frac{2}{3 }& {k=1} 
   \end{cases}
   \\
p_{Y|X}\brak{0|0} = \frac{19}{25}\, 
p_{Y|X}\brak{0|1} = \frac{6}{25}\,
p_{Y|X}\brak{1|0} = \frac{45}{50}\,
p_{Y|X}\brak{1|2} = \frac{5}{50}
\end{align}
The desired probability is the probability that a slip drawn at random is marked other than Rs 1,
\begin{align}
&=1-p_X\brak{0}\\
&= p_X(1) + p_X(2)
\end{align}
Using Bayes theorem,
\begin{align}
&= p_Y\brak{0} \times \pr{Y=0 | X=1} + p_Y\brak{1} \times \pr{Y=1|X=2}\\
&=\frac{1}{3} \times \frac{6}{25} + \frac{2}{3} \times \frac{5}{50}\\
&=\frac{11}{75}
\end{align}

\newpage

%\tableofcontents

\bigskip

\renewcommand{\thefigure}{\theenumi}
\renewcommand{\thetable}{\theenumi}
%\renewcommand{\theequation}{\theenumi}

%\begin{abstract}
%%\boldmath
%In this letter, an algorithm for evaluating the exact analytical bit error rate  (BER)  for the piecewise linear (PL) combiner for  multiple relays is presented. Previous results were available only for upto three relays. The algorithm is unique in the sense that  the actual mathematical expressions, that are prohibitively large, need not be explicitly obtained. The diversity gain due to multiple relays is shown through plots of the analytical BER, well supported by simulations. 
%
%\end{abstract}
% IEEEtran.cls defaults to using nonbold math in the Abstract.
% This preserves the distinction between vectors and scalars. However,
% if the journal you are submitting to favors bold math in the abstract,
% then you can use LaTeX's standard command \boldmath at the very start
% of the abstract to achieve this. Many IEEE journals frown on math
% in the abstract anyway.

% Note that keywords are not normally used for peerreview papers.
%\begin{IEEEkeywords}
%Cooperative diversity, decode and forward, piecewise linear
%\end{IEEEkeywords}



% For peer review papers, you can put extra information on the cover
% page as needed:
% \ifCLASSOPTIONpeerreview
% \begin{center} \bfseries EDICS Category: 3-BBND \end{center}
% \fi
%
% For peerreview papers, this IEEEtran command inserts a page break and
% creates the second title. It will be ignored for other modes.
%\IEEEpeerreviewmaketitle




\item A card is selected from a pack of 52 cards\\
\begin{enumerate}[label=(\alph*)]
\item How many points are there in the sample space?
\item Calculate the probability that the cards is an ace of spades.
\item Calculate the probability that the card is (i) an ace (ii)black card.\\
\end{enumerate}
%\input{ncert/11/16/3/4_1/Prob_4.tex}
\item In a non-leap year, the probability of having 53 tuesdays or 53 wednesdays is\\
\solution
%A non-leap year has a total of 365 days, and a week has 7 days.\\
So it can be expressed as 
\begin{align}
365\text{days} &=52\times 7+1 \text{day}
\end{align}
$\implies$ 52 tuesdays or wednesdays\\
Random variable X denotes the days of a week
\begin{align}
p_X\brak{k}&=\frac{1}{7}; \quad \brak{1<k<7}
\end{align}
So the probability of extra day being tuesday or wednesday is
\begin{align}
p_X\brak{3}+p_X\brak{4}&=\frac{1}{7}+\frac{1}{7}=\frac{2}{7}
\end{align}



\item There are 1000 sealed envelopes in a box, 10 of them contain a cash prize of
Rs 100 each, 100 of them contain a cash prize of Rs 50 each and 200 of them
contain a cash prize of Rs 10 each and rest do not contain any cash prize. If they
are well shuffled and an envelope is picked up out, what is the probability that it
contains no cash prize?\\
\solution
%\begin{table}[H]
	\centering
\begin{tabular}{|c|c|c|}
\hline
Random variable &Value &Definition\\ \hline
\multirow{3}{*}{X} &0 &Slips of Rs 1\\
&1 &Slips of Rs 5\\
&2 &Slips of Rs 13\\ \hline
\multirow{2}{*}{Y} &0 &Box A\\
&1 &Box B\\\hline
\end{tabular}
\caption{}
\label{tab:Distribution}
\end{table}
See \tabref{tab:Distribution}.
\begin{align}
p_{Y}\brak{k}= \begin{cases} 
      \frac{1}{3} & {k=0} \\
      \frac{2}{3 }& {k=1} 
   \end{cases}
   \\
p_{Y|X}\brak{0|0} = \frac{19}{25}\, 
p_{Y|X}\brak{0|1} = \frac{6}{25}\,
p_{Y|X}\brak{1|0} = \frac{45}{50}\,
p_{Y|X}\brak{1|2} = \frac{5}{50}
\end{align}
The desired probability is the probability that a slip drawn at random is marked other than Rs 1,
\begin{align}
&=1-p_X\brak{0}\\
&= p_X(1) + p_X(2)
\end{align}
Using Bayes theorem,
\begin{align}
&= p_Y\brak{0} \times \pr{Y=0 | X=1} + p_Y\brak{1} \times \pr{Y=1|X=2}\\
&=\frac{1}{3} \times \frac{6}{25} + \frac{2}{3} \times \frac{5}{50}\\
&=\frac{11}{75}
\end{align}

\newpage

%\tableofcontents

\bigskip

\renewcommand{\thefigure}{\theenumi}
\renewcommand{\thetable}{\theenumi}
%\renewcommand{\theequation}{\theenumi}

%\begin{abstract}
%%\boldmath
%In this letter, an algorithm for evaluating the exact analytical bit error rate  (BER)  for the piecewise linear (PL) combiner for  multiple relays is presented. Previous results were available only for upto three relays. The algorithm is unique in the sense that  the actual mathematical expressions, that are prohibitively large, need not be explicitly obtained. The diversity gain due to multiple relays is shown through plots of the analytical BER, well supported by simulations. 
%
%\end{abstract}
% IEEEtran.cls defaults to using nonbold math in the Abstract.
% This preserves the distinction between vectors and scalars. However,
% if the journal you are submitting to favors bold math in the abstract,
% then you can use LaTeX's standard command \boldmath at the very start
% of the abstract to achieve this. Many IEEE journals frown on math
% in the abstract anyway.

% Note that keywords are not normally used for peerreview papers.
%\begin{IEEEkeywords}
%Cooperative diversity, decode and forward, piecewise linear
%\end{IEEEkeywords}



% For peer review papers, you can put extra information on the cover
% page as needed:
% \ifCLASSOPTIONpeerreview
% \begin{center} \bfseries EDICS Category: 3-BBND \end{center}
% \fi
%
% For peerreview papers, this IEEEtran command inserts a page break and
% creates the second title. It will be ignored for other modes.
%\IEEEpeerreviewmaketitle




\item 
A die is thrown and a card is selected at random from a deck of 52 playing cards. The probability of getting an even number on the die and a spade card.\\
\solution
%\begin{table}[H]
	\centering
\begin{tabular}{|c|c|c|}
\hline
Random variable &Value &Definition\\ \hline
\multirow{3}{*}{X} &0 &Slips of Rs 1\\
&1 &Slips of Rs 5\\
&2 &Slips of Rs 13\\ \hline
\multirow{2}{*}{Y} &0 &Box A\\
&1 &Box B\\\hline
\end{tabular}
\caption{}
\label{tab:Distribution}
\end{table}
See \tabref{tab:Distribution}.
\begin{align}
p_{Y}\brak{k}= \begin{cases} 
      \frac{1}{3} & {k=0} \\
      \frac{2}{3 }& {k=1} 
   \end{cases}
   \\
p_{Y|X}\brak{0|0} = \frac{19}{25}\, 
p_{Y|X}\brak{0|1} = \frac{6}{25}\,
p_{Y|X}\brak{1|0} = \frac{45}{50}\,
p_{Y|X}\brak{1|2} = \frac{5}{50}
\end{align}
The desired probability is the probability that a slip drawn at random is marked other than Rs 1,
\begin{align}
&=1-p_X\brak{0}\\
&= p_X(1) + p_X(2)
\end{align}
Using Bayes theorem,
\begin{align}
&= p_Y\brak{0} \times \pr{Y=0 | X=1} + p_Y\brak{1} \times \pr{Y=1|X=2}\\
&=\frac{1}{3} \times \frac{6}{25} + \frac{2}{3} \times \frac{5}{50}\\
&=\frac{11}{75}
\end{align}

\newpage

%\tableofcontents

\bigskip

\renewcommand{\thefigure}{\theenumi}
\renewcommand{\thetable}{\theenumi}
%\renewcommand{\theequation}{\theenumi}

%\begin{abstract}
%%\boldmath
%In this letter, an algorithm for evaluating the exact analytical bit error rate  (BER)  for the piecewise linear (PL) combiner for  multiple relays is presented. Previous results were available only for upto three relays. The algorithm is unique in the sense that  the actual mathematical expressions, that are prohibitively large, need not be explicitly obtained. The diversity gain due to multiple relays is shown through plots of the analytical BER, well supported by simulations. 
%
%\end{abstract}
% IEEEtran.cls defaults to using nonbold math in the Abstract.
% This preserves the distinction between vectors and scalars. However,
% if the journal you are submitting to favors bold math in the abstract,
% then you can use LaTeX's standard command \boldmath at the very start
% of the abstract to achieve this. Many IEEE journals frown on math
% in the abstract anyway.

% Note that keywords are not normally used for peerreview papers.
%\begin{IEEEkeywords}
%Cooperative diversity, decode and forward, piecewise linear
%\end{IEEEkeywords}



% For peer review papers, you can put extra information on the cover
% page as needed:
% \ifCLASSOPTIONpeerreview
% \begin{center} \bfseries EDICS Category: 3-BBND \end{center}
% \fi
%
% For peerreview papers, this IEEEtran command inserts a page break and
% creates the second title. It will be ignored for other modes.
%\IEEEpeerreviewmaketitle




\item
If 4-digit numbers greater than 5,000 are randomly formed from the digits 0, 1, 3, 5, and 7, what is the probability of forming a number divisible by 5 when:
\begin{enumerate}
    \item The digits are repeated?
    \item The repetition of digits is not allowed?
\end{enumerate}
\solution
%\begin{table}[H]
	\centering
\begin{tabular}{|c|c|c|}
\hline
Random variable &Value &Definition\\ \hline
\multirow{3}{*}{X} &0 &Slips of Rs 1\\
&1 &Slips of Rs 5\\
&2 &Slips of Rs 13\\ \hline
\multirow{2}{*}{Y} &0 &Box A\\
&1 &Box B\\\hline
\end{tabular}
\caption{}
\label{tab:Distribution}
\end{table}
See \tabref{tab:Distribution}.
\begin{align}
p_{Y}\brak{k}= \begin{cases} 
      \frac{1}{3} & {k=0} \\
      \frac{2}{3 }& {k=1} 
   \end{cases}
   \\
p_{Y|X}\brak{0|0} = \frac{19}{25}\, 
p_{Y|X}\brak{0|1} = \frac{6}{25}\,
p_{Y|X}\brak{1|0} = \frac{45}{50}\,
p_{Y|X}\brak{1|2} = \frac{5}{50}
\end{align}
The desired probability is the probability that a slip drawn at random is marked other than Rs 1,
\begin{align}
&=1-p_X\brak{0}\\
&= p_X(1) + p_X(2)
\end{align}
Using Bayes theorem,
\begin{align}
&= p_Y\brak{0} \times \pr{Y=0 | X=1} + p_Y\brak{1} \times \pr{Y=1|X=2}\\
&=\frac{1}{3} \times \frac{6}{25} + \frac{2}{3} \times \frac{5}{50}\\
&=\frac{11}{75}
\end{align}

\newpage

%\tableofcontents

\bigskip

\renewcommand{\thefigure}{\theenumi}
\renewcommand{\thetable}{\theenumi}
%\renewcommand{\theequation}{\theenumi}

%\begin{abstract}
%%\boldmath
%In this letter, an algorithm for evaluating the exact analytical bit error rate  (BER)  for the piecewise linear (PL) combiner for  multiple relays is presented. Previous results were available only for upto three relays. The algorithm is unique in the sense that  the actual mathematical expressions, that are prohibitively large, need not be explicitly obtained. The diversity gain due to multiple relays is shown through plots of the analytical BER, well supported by simulations. 
%
%\end{abstract}
% IEEEtran.cls defaults to using nonbold math in the Abstract.
% This preserves the distinction between vectors and scalars. However,
% if the journal you are submitting to favors bold math in the abstract,
% then you can use LaTeX's standard command \boldmath at the very start
% of the abstract to achieve this. Many IEEE journals frown on math
% in the abstract anyway.

% Note that keywords are not normally used for peerreview papers.
%\begin{IEEEkeywords}
%Cooperative diversity, decode and forward, piecewise linear
%\end{IEEEkeywords}



% For peer review papers, you can put extra information on the cover
% page as needed:
% \ifCLASSOPTIONpeerreview
% \begin{center} \bfseries EDICS Category: 3-BBND \end{center}
% \fi
%
% For peerreview papers, this IEEEtran command inserts a page break and
% creates the second title. It will be ignored for other modes.
%\IEEEpeerreviewmaketitle




\item Consider the probability space $\brak{\Omega, \mathcal{G}, P}$ where $\Omega = [0,2]$ and $\mathcal{G} = \cbrak{\phi, \Omega, [0,1], (1,2]}$. Let $X$ and $Y$ be two functions on $\Omega$ defined as
\begin{align*}
    X(\omega) = 
    \begin{cases}
        1 & \text{if }\omega \in [0, 1]\\
        2 & \text{if }\omega \in (1, 2]
    \end{cases}
\end{align*}
and
\begin{align*}
    Y(\omega) = 
    \begin{cases}
        2 & \text{if }\omega \in [0, 1.5]\\
        3 & \text{if }\omega \in (1.5, 2].
    \end{cases}
\end{align*}
Then which one of the following statements is true?
\begin{enumerate}
    \item [(A)] $X$ is a random variable with respect to $\mathcal{G}$, but $Y$ is not a random variable with respect to $\mathcal{G}$.
    \item [(B)] $Y$ is a random variable with respect to $\mathcal{G}$, but $X$ is not a random variable with respect to $\mathcal{G}$.
    \item [(C)] Neither $X$ nor $Y$ is a random variable with respect to $\mathcal{G}$.
    \item [(D)] Both $X$ and $Y$ are random variables with respect to $\mathcal{G}$.
\end{enumerate} \hfill (GATE ST 2023)\\
\solution
%\begin{table}[H]
	\centering
\begin{tabular}{|c|c|c|}
\hline
Random variable &Value &Definition\\ \hline
\multirow{3}{*}{X} &0 &Slips of Rs 1\\
&1 &Slips of Rs 5\\
&2 &Slips of Rs 13\\ \hline
\multirow{2}{*}{Y} &0 &Box A\\
&1 &Box B\\\hline
\end{tabular}
\caption{}
\label{tab:Distribution}
\end{table}
See \tabref{tab:Distribution}.
\begin{align}
p_{Y}\brak{k}= \begin{cases} 
      \frac{1}{3} & {k=0} \\
      \frac{2}{3 }& {k=1} 
   \end{cases}
   \\
p_{Y|X}\brak{0|0} = \frac{19}{25}\, 
p_{Y|X}\brak{0|1} = \frac{6}{25}\,
p_{Y|X}\brak{1|0} = \frac{45}{50}\,
p_{Y|X}\brak{1|2} = \frac{5}{50}
\end{align}
The desired probability is the probability that a slip drawn at random is marked other than Rs 1,
\begin{align}
&=1-p_X\brak{0}\\
&= p_X(1) + p_X(2)
\end{align}
Using Bayes theorem,
\begin{align}
&= p_Y\brak{0} \times \pr{Y=0 | X=1} + p_Y\brak{1} \times \pr{Y=1|X=2}\\
&=\frac{1}{3} \times \frac{6}{25} + \frac{2}{3} \times \frac{5}{50}\\
&=\frac{11}{75}
\end{align}

\newpage

%\tableofcontents

\bigskip

\renewcommand{\thefigure}{\theenumi}
\renewcommand{\thetable}{\theenumi}
%\renewcommand{\theequation}{\theenumi}

%\begin{abstract}
%%\boldmath
%In this letter, an algorithm for evaluating the exact analytical bit error rate  (BER)  for the piecewise linear (PL) combiner for  multiple relays is presented. Previous results were available only for upto three relays. The algorithm is unique in the sense that  the actual mathematical expressions, that are prohibitively large, need not be explicitly obtained. The diversity gain due to multiple relays is shown through plots of the analytical BER, well supported by simulations. 
%
%\end{abstract}
% IEEEtran.cls defaults to using nonbold math in the Abstract.
% This preserves the distinction between vectors and scalars. However,
% if the journal you are submitting to favors bold math in the abstract,
% then you can use LaTeX's standard command \boldmath at the very start
% of the abstract to achieve this. Many IEEE journals frown on math
% in the abstract anyway.

% Note that keywords are not normally used for peerreview papers.
%\begin{IEEEkeywords}
%Cooperative diversity, decode and forward, piecewise linear
%\end{IEEEkeywords}



% For peer review papers, you can put extra information on the cover
% page as needed:
% \ifCLASSOPTIONpeerreview
% \begin{center} \bfseries EDICS Category: 3-BBND \end{center}
% \fi
%
% For peerreview papers, this IEEEtran command inserts a page break and
% creates the second title. It will be ignored for other modes.
%\IEEEpeerreviewmaketitle




	\item  A die is loaded in such a way that each odd number is twice as likely to occur as
each even number. Find $P(G)$, where $G$ is the event that a number greater than
3 occurs on a single roll of the die.
\\
\solution
		%\begin{table}[H]
	\centering
\begin{tabular}{|c|c|c|}
\hline
Random variable &Value &Definition\\ \hline
\multirow{3}{*}{X} &0 &Slips of Rs 1\\
&1 &Slips of Rs 5\\
&2 &Slips of Rs 13\\ \hline
\multirow{2}{*}{Y} &0 &Box A\\
&1 &Box B\\\hline
\end{tabular}
\caption{}
\label{tab:Distribution}
\end{table}
See \tabref{tab:Distribution}.
\begin{align}
p_{Y}\brak{k}= \begin{cases} 
      \frac{1}{3} & {k=0} \\
      \frac{2}{3 }& {k=1} 
   \end{cases}
   \\
p_{Y|X}\brak{0|0} = \frac{19}{25}\, 
p_{Y|X}\brak{0|1} = \frac{6}{25}\,
p_{Y|X}\brak{1|0} = \frac{45}{50}\,
p_{Y|X}\brak{1|2} = \frac{5}{50}
\end{align}
The desired probability is the probability that a slip drawn at random is marked other than Rs 1,
\begin{align}
&=1-p_X\brak{0}\\
&= p_X(1) + p_X(2)
\end{align}
Using Bayes theorem,
\begin{align}
&= p_Y\brak{0} \times \pr{Y=0 | X=1} + p_Y\brak{1} \times \pr{Y=1|X=2}\\
&=\frac{1}{3} \times \frac{6}{25} + \frac{2}{3} \times \frac{5}{50}\\
&=\frac{11}{75}
\end{align}

\newpage

%\tableofcontents

\bigskip

\renewcommand{\thefigure}{\theenumi}
\renewcommand{\thetable}{\theenumi}
%\renewcommand{\theequation}{\theenumi}

%\begin{abstract}
%%\boldmath
%In this letter, an algorithm for evaluating the exact analytical bit error rate  (BER)  for the piecewise linear (PL) combiner for  multiple relays is presented. Previous results were available only for upto three relays. The algorithm is unique in the sense that  the actual mathematical expressions, that are prohibitively large, need not be explicitly obtained. The diversity gain due to multiple relays is shown through plots of the analytical BER, well supported by simulations. 
%
%\end{abstract}
% IEEEtran.cls defaults to using nonbold math in the Abstract.
% This preserves the distinction between vectors and scalars. However,
% if the journal you are submitting to favors bold math in the abstract,
% then you can use LaTeX's standard command \boldmath at the very start
% of the abstract to achieve this. Many IEEE journals frown on math
% in the abstract anyway.

% Note that keywords are not normally used for peerreview papers.
%\begin{IEEEkeywords}
%Cooperative diversity, decode and forward, piecewise linear
%\end{IEEEkeywords}



% For peer review papers, you can put extra information on the cover
% page as needed:
% \ifCLASSOPTIONpeerreview
% \begin{center} \bfseries EDICS Category: 3-BBND \end{center}
% \fi
%
% For peerreview papers, this IEEEtran command inserts a page break and
% creates the second title. It will be ignored for other modes.
%\IEEEpeerreviewmaketitle




	\item All the jacks, queens and kings are removed from a deck of 52 playing cards. The remaining cards are well shuffled and then one card is drawn at random. Giving ace a value 1 similar value for other cards, find the probability that the card has a value 
		\begin{enumerate}
			\item 7
			\item greater than 7
			\item less than 7
		\end{enumerate}
		%Number of cards left after removing all jacks, queens and kings 
\begin{align}
N	= 52 - 4\times 3
	= 40
\end{align}
%\begin{table}[H]
%\def\arraystretch{1.2}
%\begin{tabular}{|c|c|c|}
%\hline
%	\textbf{Parameter} &\textbf{Value} &\textbf{Description}\\ \hline
%	$X$ &1-10 &Represents the value of the card picked \\ \hline
%\end{tabular}
%\end{table}
Let $1 \le X \le 10$ be the value of the card picked.  Then,
\begin{align}
	p_X(k) &= \Pr(X=k)\ \forall\ 1 \leq k \leq 10\\
	&= \frac{4\times 1}{40}\\
	&= \frac{1}{10}\\
	\therefore p_X(k) &= 
	\begin{cases}
		\frac{1}{10} & 1 \leq k \leq 10\\
		0 & \text{otherwise}
	\end{cases}
\end{align}
and
\begin{align}
	F_{X}(k) &= \sum_{m=0}^{k}p_{X}(m) \quad 1 \leq k \leq 10\\
	&= \frac{k}{10}\\
	\therefore F_{X}(k) &= 
	\begin{cases}
		0 & k \leq 0\\
		\frac{k}{10} & 1\leq k \leq 10\\
		1 & k > 10 
	\end{cases}
\end{align}
\begin{enumerate}
	\item Probability that card has value equal to 7 is
		\begin{align}
			 p_{X}(7)
			= \frac{1}{10}
		\end{align}
	\item Probability that card has value greater than 7 is
		\begin{align}
			1 - F_X(7)
			&= 1 - \frac{7}{10}
			\\
			&= \frac{3}{10}
		\end{align}
	\item Probability that card has value less than 7 is
		\begin{align}
			 F_{X}(6)
			=\frac{6}{10}
		\end{align}
\end{enumerate}

  \item A Lot consists of 48 mobile phones of which 42 are good, 3 have only minor defects and 3 have major defects.Varnika will buy a phone if it is good but the trader will only buy a mobile if it has no major defects. One phone is selected at random from the lot. What is the probability that it is
\begin{enumerate}
	\item acceptable to Varnika?
            \item acceptable to the trader?
\end{enumerate}
\solution
	%\begin{table}[H]
	\centering
\begin{tabular}{|c|c|c|}
\hline
Random variable &Value &Definition\\ \hline
\multirow{3}{*}{X} &0 &Slips of Rs 1\\
&1 &Slips of Rs 5\\
&2 &Slips of Rs 13\\ \hline
\multirow{2}{*}{Y} &0 &Box A\\
&1 &Box B\\\hline
\end{tabular}
\caption{}
\label{tab:Distribution}
\end{table}
See \tabref{tab:Distribution}.
\begin{align}
p_{Y}\brak{k}= \begin{cases} 
      \frac{1}{3} & {k=0} \\
      \frac{2}{3 }& {k=1} 
   \end{cases}
   \\
p_{Y|X}\brak{0|0} = \frac{19}{25}\, 
p_{Y|X}\brak{0|1} = \frac{6}{25}\,
p_{Y|X}\brak{1|0} = \frac{45}{50}\,
p_{Y|X}\brak{1|2} = \frac{5}{50}
\end{align}
The desired probability is the probability that a slip drawn at random is marked other than Rs 1,
\begin{align}
&=1-p_X\brak{0}\\
&= p_X(1) + p_X(2)
\end{align}
Using Bayes theorem,
\begin{align}
&= p_Y\brak{0} \times \pr{Y=0 | X=1} + p_Y\brak{1} \times \pr{Y=1|X=2}\\
&=\frac{1}{3} \times \frac{6}{25} + \frac{2}{3} \times \frac{5}{50}\\
&=\frac{11}{75}
\end{align}

\newpage

%\tableofcontents

\bigskip

\renewcommand{\thefigure}{\theenumi}
\renewcommand{\thetable}{\theenumi}
%\renewcommand{\theequation}{\theenumi}

%\begin{abstract}
%%\boldmath
%In this letter, an algorithm for evaluating the exact analytical bit error rate  (BER)  for the piecewise linear (PL) combiner for  multiple relays is presented. Previous results were available only for upto three relays. The algorithm is unique in the sense that  the actual mathematical expressions, that are prohibitively large, need not be explicitly obtained. The diversity gain due to multiple relays is shown through plots of the analytical BER, well supported by simulations. 
%
%\end{abstract}
% IEEEtran.cls defaults to using nonbold math in the Abstract.
% This preserves the distinction between vectors and scalars. However,
% if the journal you are submitting to favors bold math in the abstract,
% then you can use LaTeX's standard command \boldmath at the very start
% of the abstract to achieve this. Many IEEE journals frown on math
% in the abstract anyway.

% Note that keywords are not normally used for peerreview papers.
%\begin{IEEEkeywords}
%Cooperative diversity, decode and forward, piecewise linear
%\end{IEEEkeywords}



% For peer review papers, you can put extra information on the cover
% page as needed:
% \ifCLASSOPTIONpeerreview
% \begin{center} \bfseries EDICS Category: 3-BBND \end{center}
% \fi
%
% For peerreview papers, this IEEEtran command inserts a page break and
% creates the second title. It will be ignored for other modes.
%\IEEEpeerreviewmaketitle




 \item A student says that if you throw a die, it will show up 1 or not 1. Therefore, the probability of getting 1 and the probability of getting 'not 1' each is equal to $\frac{1}{2}$. Is this correct? Give reasons.\\
 \solution
        %\begin{table}[H]
	\centering
\begin{tabular}{|c|c|c|}
\hline
Random variable &Value &Definition\\ \hline
\multirow{3}{*}{X} &0 &Slips of Rs 1\\
&1 &Slips of Rs 5\\
&2 &Slips of Rs 13\\ \hline
\multirow{2}{*}{Y} &0 &Box A\\
&1 &Box B\\\hline
\end{tabular}
\caption{}
\label{tab:Distribution}
\end{table}
See \tabref{tab:Distribution}.
\begin{align}
p_{Y}\brak{k}= \begin{cases} 
      \frac{1}{3} & {k=0} \\
      \frac{2}{3 }& {k=1} 
   \end{cases}
   \\
p_{Y|X}\brak{0|0} = \frac{19}{25}\, 
p_{Y|X}\brak{0|1} = \frac{6}{25}\,
p_{Y|X}\brak{1|0} = \frac{45}{50}\,
p_{Y|X}\brak{1|2} = \frac{5}{50}
\end{align}
The desired probability is the probability that a slip drawn at random is marked other than Rs 1,
\begin{align}
&=1-p_X\brak{0}\\
&= p_X(1) + p_X(2)
\end{align}
Using Bayes theorem,
\begin{align}
&= p_Y\brak{0} \times \pr{Y=0 | X=1} + p_Y\brak{1} \times \pr{Y=1|X=2}\\
&=\frac{1}{3} \times \frac{6}{25} + \frac{2}{3} \times \frac{5}{50}\\
&=\frac{11}{75}
\end{align}

\newpage

%\tableofcontents

\bigskip

\renewcommand{\thefigure}{\theenumi}
\renewcommand{\thetable}{\theenumi}
%\renewcommand{\theequation}{\theenumi}

%\begin{abstract}
%%\boldmath
%In this letter, an algorithm for evaluating the exact analytical bit error rate  (BER)  for the piecewise linear (PL) combiner for  multiple relays is presented. Previous results were available only for upto three relays. The algorithm is unique in the sense that  the actual mathematical expressions, that are prohibitively large, need not be explicitly obtained. The diversity gain due to multiple relays is shown through plots of the analytical BER, well supported by simulations. 
%
%\end{abstract}
% IEEEtran.cls defaults to using nonbold math in the Abstract.
% This preserves the distinction between vectors and scalars. However,
% if the journal you are submitting to favors bold math in the abstract,
% then you can use LaTeX's standard command \boldmath at the very start
% of the abstract to achieve this. Many IEEE journals frown on math
% in the abstract anyway.

% Note that keywords are not normally used for peerreview papers.
%\begin{IEEEkeywords}
%Cooperative diversity, decode and forward, piecewise linear
%\end{IEEEkeywords}



% For peer review papers, you can put extra information on the cover
% page as needed:
% \ifCLASSOPTIONpeerreview
% \begin{center} \bfseries EDICS Category: 3-BBND \end{center}
% \fi
%
% For peerreview papers, this IEEEtran command inserts a page break and
% creates the second title. It will be ignored for other modes.
%\IEEEpeerreviewmaketitle




   \item Four candidates A, B, C, D have ap-
plied for the assignment to coach a school cricket
team. If A is twice as likely to be selected as B, and
B and C are given about the same chance of being
selected, while C is twice as likely to be selected
as D, what are the probabilities that
\begin{enumerate}
\item C will be selected?
\item A will not be selected?
\end{enumerate}
	%\begin{table}[H]
	\centering
\begin{tabular}{|c|c|c|}
\hline
Random variable &Value &Definition\\ \hline
\multirow{3}{*}{X} &0 &Slips of Rs 1\\
&1 &Slips of Rs 5\\
&2 &Slips of Rs 13\\ \hline
\multirow{2}{*}{Y} &0 &Box A\\
&1 &Box B\\\hline
\end{tabular}
\caption{}
\label{tab:Distribution}
\end{table}
See \tabref{tab:Distribution}.
\begin{align}
p_{Y}\brak{k}= \begin{cases} 
      \frac{1}{3} & {k=0} \\
      \frac{2}{3 }& {k=1} 
   \end{cases}
   \\
p_{Y|X}\brak{0|0} = \frac{19}{25}\, 
p_{Y|X}\brak{0|1} = \frac{6}{25}\,
p_{Y|X}\brak{1|0} = \frac{45}{50}\,
p_{Y|X}\brak{1|2} = \frac{5}{50}
\end{align}
The desired probability is the probability that a slip drawn at random is marked other than Rs 1,
\begin{align}
&=1-p_X\brak{0}\\
&= p_X(1) + p_X(2)
\end{align}
Using Bayes theorem,
\begin{align}
&= p_Y\brak{0} \times \pr{Y=0 | X=1} + p_Y\brak{1} \times \pr{Y=1|X=2}\\
&=\frac{1}{3} \times \frac{6}{25} + \frac{2}{3} \times \frac{5}{50}\\
&=\frac{11}{75}
\end{align}

\newpage

%\tableofcontents

\bigskip

\renewcommand{\thefigure}{\theenumi}
\renewcommand{\thetable}{\theenumi}
%\renewcommand{\theequation}{\theenumi}

%\begin{abstract}
%%\boldmath
%In this letter, an algorithm for evaluating the exact analytical bit error rate  (BER)  for the piecewise linear (PL) combiner for  multiple relays is presented. Previous results were available only for upto three relays. The algorithm is unique in the sense that  the actual mathematical expressions, that are prohibitively large, need not be explicitly obtained. The diversity gain due to multiple relays is shown through plots of the analytical BER, well supported by simulations. 
%
%\end{abstract}
% IEEEtran.cls defaults to using nonbold math in the Abstract.
% This preserves the distinction between vectors and scalars. However,
% if the journal you are submitting to favors bold math in the abstract,
% then you can use LaTeX's standard command \boldmath at the very start
% of the abstract to achieve this. Many IEEE journals frown on math
% in the abstract anyway.

% Note that keywords are not normally used for peerreview papers.
%\begin{IEEEkeywords}
%Cooperative diversity, decode and forward, piecewise linear
%\end{IEEEkeywords}



% For peer review papers, you can put extra information on the cover
% page as needed:
% \ifCLASSOPTIONpeerreview
% \begin{center} \bfseries EDICS Category: 3-BBND \end{center}
% \fi
%
% For peerreview papers, this IEEEtran command inserts a page break and
% creates the second title. It will be ignored for other modes.
%\IEEEpeerreviewmaketitle




 \item A bag contain 24 balls of which $x$ balls are red, $2x$ are white and $3x$ are blue. A ball is selected at random, What is the probability that it is
\begin{enumerate}[label=\alph*)]
\item not red ?
\item white ?
\end{enumerate}
%\begin{table}[H]
	\centering
\begin{tabular}{|c|c|c|}
\hline
Random variable &Value &Definition\\ \hline
\multirow{3}{*}{X} &0 &Slips of Rs 1\\
&1 &Slips of Rs 5\\
&2 &Slips of Rs 13\\ \hline
\multirow{2}{*}{Y} &0 &Box A\\
&1 &Box B\\\hline
\end{tabular}
\caption{}
\label{tab:Distribution}
\end{table}
See \tabref{tab:Distribution}.
\begin{align}
p_{Y}\brak{k}= \begin{cases} 
      \frac{1}{3} & {k=0} \\
      \frac{2}{3 }& {k=1} 
   \end{cases}
   \\
p_{Y|X}\brak{0|0} = \frac{19}{25}\, 
p_{Y|X}\brak{0|1} = \frac{6}{25}\,
p_{Y|X}\brak{1|0} = \frac{45}{50}\,
p_{Y|X}\brak{1|2} = \frac{5}{50}
\end{align}
The desired probability is the probability that a slip drawn at random is marked other than Rs 1,
\begin{align}
&=1-p_X\brak{0}\\
&= p_X(1) + p_X(2)
\end{align}
Using Bayes theorem,
\begin{align}
&= p_Y\brak{0} \times \pr{Y=0 | X=1} + p_Y\brak{1} \times \pr{Y=1|X=2}\\
&=\frac{1}{3} \times \frac{6}{25} + \frac{2}{3} \times \frac{5}{50}\\
&=\frac{11}{75}
\end{align}

\newpage

%\tableofcontents

\bigskip

\renewcommand{\thefigure}{\theenumi}
\renewcommand{\thetable}{\theenumi}
%\renewcommand{\theequation}{\theenumi}

%\begin{abstract}
%%\boldmath
%In this letter, an algorithm for evaluating the exact analytical bit error rate  (BER)  for the piecewise linear (PL) combiner for  multiple relays is presented. Previous results were available only for upto three relays. The algorithm is unique in the sense that  the actual mathematical expressions, that are prohibitively large, need not be explicitly obtained. The diversity gain due to multiple relays is shown through plots of the analytical BER, well supported by simulations. 
%
%\end{abstract}
% IEEEtran.cls defaults to using nonbold math in the Abstract.
% This preserves the distinction between vectors and scalars. However,
% if the journal you are submitting to favors bold math in the abstract,
% then you can use LaTeX's standard command \boldmath at the very start
% of the abstract to achieve this. Many IEEE journals frown on math
% in the abstract anyway.

% Note that keywords are not normally used for peerreview papers.
%\begin{IEEEkeywords}
%Cooperative diversity, decode and forward, piecewise linear
%\end{IEEEkeywords}



% For peer review papers, you can put extra information on the cover
% page as needed:
% \ifCLASSOPTIONpeerreview
% \begin{center} \bfseries EDICS Category: 3-BBND \end{center}
% \fi
%
% For peerreview papers, this IEEEtran command inserts a page break and
% creates the second title. It will be ignored for other modes.
%\IEEEpeerreviewmaketitle




If the letters of the word ASSASSINATION are arranged at random. Find the Probability that
\begin{enumerate}[label=(\alph*)]
\item Four $S's$ come consecutively in the word
\item Two  $I's$ and two $N's$ come together
\item All $A's$ are not coming together
\item No two $A's$ are coming together
\end{enumerate}
%\begin{table}[H]
	\centering
\begin{tabular}{|c|c|c|}
\hline
Random variable &Value &Definition\\ \hline
\multirow{3}{*}{X} &0 &Slips of Rs 1\\
&1 &Slips of Rs 5\\
&2 &Slips of Rs 13\\ \hline
\multirow{2}{*}{Y} &0 &Box A\\
&1 &Box B\\\hline
\end{tabular}
\caption{}
\label{tab:Distribution}
\end{table}
See \tabref{tab:Distribution}.
\begin{align}
p_{Y}\brak{k}= \begin{cases} 
      \frac{1}{3} & {k=0} \\
      \frac{2}{3 }& {k=1} 
   \end{cases}
   \\
p_{Y|X}\brak{0|0} = \frac{19}{25}\, 
p_{Y|X}\brak{0|1} = \frac{6}{25}\,
p_{Y|X}\brak{1|0} = \frac{45}{50}\,
p_{Y|X}\brak{1|2} = \frac{5}{50}
\end{align}
The desired probability is the probability that a slip drawn at random is marked other than Rs 1,
\begin{align}
&=1-p_X\brak{0}\\
&= p_X(1) + p_X(2)
\end{align}
Using Bayes theorem,
\begin{align}
&= p_Y\brak{0} \times \pr{Y=0 | X=1} + p_Y\brak{1} \times \pr{Y=1|X=2}\\
&=\frac{1}{3} \times \frac{6}{25} + \frac{2}{3} \times \frac{5}{50}\\
&=\frac{11}{75}
\end{align}

\newpage

%\tableofcontents

\bigskip

\renewcommand{\thefigure}{\theenumi}
\renewcommand{\thetable}{\theenumi}
%\renewcommand{\theequation}{\theenumi}

%\begin{abstract}
%%\boldmath
%In this letter, an algorithm for evaluating the exact analytical bit error rate  (BER)  for the piecewise linear (PL) combiner for  multiple relays is presented. Previous results were available only for upto three relays. The algorithm is unique in the sense that  the actual mathematical expressions, that are prohibitively large, need not be explicitly obtained. The diversity gain due to multiple relays is shown through plots of the analytical BER, well supported by simulations. 
%
%\end{abstract}
% IEEEtran.cls defaults to using nonbold math in the Abstract.
% This preserves the distinction between vectors and scalars. However,
% if the journal you are submitting to favors bold math in the abstract,
% then you can use LaTeX's standard command \boldmath at the very start
% of the abstract to achieve this. Many IEEE journals frown on math
% in the abstract anyway.

% Note that keywords are not normally used for peerreview papers.
%\begin{IEEEkeywords}
%Cooperative diversity, decode and forward, piecewise linear
%\end{IEEEkeywords}



% For peer review papers, you can put extra information on the cover
% page as needed:
% \ifCLASSOPTIONpeerreview
% \begin{center} \bfseries EDICS Category: 3-BBND \end{center}
% \fi
%
% For peerreview papers, this IEEEtran command inserts a page break and
% creates the second title. It will be ignored for other modes.
%\IEEEpeerreviewmaketitle




	\item One urn contains two black balls (labelled B1 and B2) and one white ball. A
	second urn contains one black ball and two white balls (labelled W1 and W2).
	Suppose the following experiment is performed. One of the two urns is chosen
	at random. Next a ball is randomly chosen from the urn. Then a second ball is
	chosen at random from the same urn without replacing the first ball.
	
	\begin{enumerate}
	\item What is the probability that two black balls are chosen?
	
	\item What is the probability that two balls of opposite colour are chosen?
	\end{enumerate}
	\solution
	%\begin{align}
    \label{eq:12.13.6.18.1}
	\because	\pr{A|B} &> \pr{A},\
\frac{\pr{AB}}{\pr{B}} > \pr{A}
\\
    \label{eq:12.13.6.18.2}
	\implies \pr{AB} &> \pr{A}\pr{B}
	\\
	\text{or, } \frac{\pr{AB}}{\pr{A}} &=\pr{B|A} > \pr{A}
\end{align}

\end{enumerate}

\section{Boolean Logic}
\subsection{Formulae}
\input{app/bool.tex}
\subsection{NCERT}
\begin{enumerate}[label=\thesubsection.\arabic*,ref=\thesubsection.\theenumi]
	\item Which of the following cannot be the probability of an event ? 
	\vspace{-5mm}
\begin{multicols}{4}
\begin{enumerate}
\item $\frac{2}{3}$ 
\item $-1.5$ 
\item $15\%$ 
\item $0.7$ 
\end{enumerate}
\end{multicols}
	\vspace{-5mm}
\solution
We see that 
\begin{align}
\pr{E} = -1.5 
\end{align}
violates
	\eqref{eq:axiom-pos}.
	Hence, 
it cannot be a probability of any event.


\item If $P(E) = 0.05$, what is the probability of ‘not $E$’?\\
\solution
From
	\eqref{eq:axiom-one}
	and
\eqref{eq:axiom_exclusive},
the desired probability is
\begin{align}
\pr{E ^\prime} = 1 - \pr{E}
= 0.95
\end{align}




\item Check whether the following probabilities $\pr{A}$ and $\pr{B}$ are consistently defined
\begin{enumerate}
\item $\pr{A} = 0.5,\, \pr{B} = 0.7,\, \pr{A \cap B} = 0.6$
\item $\pr{A} = 0.5,\, \pr{B} = 0.7,\, \pr{A \cup B} = 0.8$
\end{enumerate}
\solution
\begin{enumerate}
\item 
\begin{align}
\pr{A} < 
\pr{AB} = 0.6
\end{align}
which violates
	\eqref{eq:axiom-prod-ge}.  Inconsistent.
\item Given that 
\begin{align}
\pr{A} = 0.5,\,
\pr{B} = 0.7,\,
\pr{A+B} = 0.8
\end{align}
From \eqref{eq:axiom_sum_AB}, we get, 
\begin{align}
\pr{AB} = 0.5+ 0.7 - 0.8 \\
= 0.4 \label{eq:ncert/11/16/3/12/3}
\end{align}
$\because$ no axioms are violated, 
the given probabilities are consistently defined
\end{enumerate}



\item Given $\pr{A}=\frac{3}{5}$ and $\pr{B}=\frac{1}{5}$. Find $\pr{A+B}$ if $A$ and $B$ are mutually exclusive events.\\
\solution
From \eqref{eq:axiom_exclusive},
\begin{align}
  \pr{A+B} = \pr{A} + \pr{B} 
  = \frac{4}{5}
\end{align}



\item If $E$ and $F$ are events such that $\pr{E}=\frac{1}{4}$, $\pr{F}= \frac{1}{2}$ and $\pr{EF} =\frac{1}{8}$, find
\begin{enumerate}
\item $\pr{E + F}$
\item $\pr{E^\prime F^\prime}$
\end{enumerate}
\solution
\begin{enumerate}
    \item 
	    \begin{align}
     \pr{E+F}&= \pr{E} + \pr{F} - \pr{EF}
    = \frac{5}{8}
	\label{eq:11/16/3/15}
\end{align}
\item From 
	\eqref{eq:demorgan},
\begin{align}
	(E^\prime F^\prime) &= (E+F)^\prime
    \\
\implies     \pr{E^\prime F^\prime} &= \pr{(E+F)^\prime}
\\
	&= 1 - \pr{E+F} 
    = \frac{3}{8}
\end{align}
upon substituting from 
	\eqref{eq:11/16/3/15}.
\end{enumerate}

\item Events $E$ and $F$ are such that P(not $E$ or not $F$) = 0.25, state whether $E$ and $F$ are mutually exclusive.\\
\solution
 \begin{align} 
	 \pr{E^\prime+ F^\prime}&= \pr{\brak{EF}^\prime}
	 \\
	 &=1-\pr{EF}
	 \\
	 \implies \pr{EF}&=0.75 
 \end{align}	
 $\because \pr{EF}\ne 0, E$ and $F$ are not mutually exclusive.
 

\item If A and B are two independent events with $\pr{A}=\frac{3}{5}$ and $\pr{B}=\frac{4}{9}$ then, $\pr{A^\prime B^\prime}$
	\vspace{-5mm}
\begin{multicols}{4}
\begin{enumerate}
%		 \setlength{\itemsep}{1ex} % Adjust the value as needed
\item $\frac{4}{15}$ 
\item $\frac{8}{45}$
\item $\frac{1}{3}$
\item $\frac{2}{9}$
\end{enumerate}
\end{multicols}
	\vspace{-5mm}
		\solution
		\begin{align}
\pr{A^\prime B^\prime} &= \pr{(A+B)^\prime} \\
		  &= 1 - \pr{(A+B)} \\
		  &= 1 - \pr{A} - \pr{B} + \pr{A}\pr{B} \\
		  \label{eq:12.13.3.69/8}&= \frac{2}{9}
\end{align}
from 
\eqref{eq:axiom-ind}
and
\eqref{eq:axiom_sum_AB}.
 
%
	\item $A$ and $B$ are events such that $\pr A=0.42, \pr B=0.48$ and $\pr {A \text{ and } B}=0.16$. \\
Determine 
\begin{enumerate}
\item $\pr{\text{not A}}$ 
\item $\pr{\text{not B}}$  
\item $\pr{\text{A or B}}$ 
\end{enumerate}
\solution
\solution
\begin{enumerate}
\item 
\begin{align}
\pr{A^{\prime}} = 1 - \pr A 
= 0.58 
\end{align}
%
\item 
\begin{align}
\pr{B^{\prime}} = 1 - \pr B 
= 0.52
\end{align}
%
\item 
\begin{align}
\pr{\text{A+B}}  
= 0.42 + 0.48 - 0.16 
= 0.74
\end{align}
\end{enumerate}

\item $A$ and $B$ are two events such that \pr A = 0.54, \pr B = 0.69 and \pr {AB} = 0.35.
%
Find 
	\vspace{-3mm}
\begin{multicols}{4}
	\begin{enumerate}
\item \pr {A+B}
\item \pr{A^\prime  B^\prime }
\item \pr {A B^\prime}
\item \pr {B A^\prime} 
\end{enumerate} 
\end{multicols}
	\vspace{-3mm}
		\solution
\begin{enumerate}
\item  
\begin{align}
	\pr {A+B} &=  \pr A + \pr B - \pr{AB} \\&=  0.88
\end{align}
\item By De Morgan's Law,
\begin{align}
	A^\prime B^\prime &= {(A+B)}^\prime\\
	\implies \pr {A^\prime B^\prime} &= \pr{A+B}^\prime \\&= 1 - \pr {A +B}\\
	&= 0.12
\end{align}
\item 
	From 
\eqref{eq:axiom_sum_A},
\begin{align}
\pr {A } 
	 &= \pr{AB}+\pr{AB^\prime} \\
	 \implies \pr{AB^\prime} 
	&= 0.19
\end{align}
\item Similarly, 
\begin{align}
	\pr {B A^\prime} &= \pr {B} - \pr{AB} 
= 0.34.
\end{align}
\end{enumerate}



\item If $\pr{A} = \frac{3}{5}$ and $\pr{B} = \frac{1}{5}$ find $\pr{A \cap B}$ if A and B are independent events.
	\\
\solution
    From the given data, 
    \begin{align}
        \pr{A} = \frac{30}{60} = \frac{1}{2},\ 
        \pr{B} = \frac{32}{60} = \frac{8}{15},\ 
        \pr{AB}= \frac{24}{60} = \frac{2}{5}.
    \end{align}
    Thus, the desired probabilities are
\begin{enumerate}
	\item $\pr{A+B}=\frac{19}{30}$, from 
\eqref{eq:axiom_sum_AB}.
	\item From
	\eqref{eq:demorgan}
	and the axioms of probability,
        \begin{align}
\pr{A^\prime  B^\prime } =1-\pr{A+B}= \frac{11}{30}.
\end{align}
	\item     
		\begin{align}
	    \pr{A^\prime B}        &= \pr{B} - \pr{AB} = \frac{2}{15} 
    \end{align}
    from
\eqref{eq:axiom-cond-prime}.
\end{enumerate}

%
\item   Let $E$ and $F$ be events with $\pr{E}=\frac{3}{5}, \pr{F}=\frac{3}{10}$ and $\pr{E F}=\frac{1}{5}$. Are $E$ and $F$ independent?
	\\
\solution
From the given information,
\begin{align}
	\pr{E} \pr{F}&=\frac{3}{5} \times \frac{9}{50},
\\
	\pr{E F}&=\frac{1}{5}
\\
	\implies \pr{E F} &\neq P(E) P(F)
\end{align}
$\therefore$ $E$ and $F$ are not independent.

	%
\item Given that the events A and B are such that $P(A)=\frac{1}{2}, P(A + B)=\frac{3}{5}$ and $P(B)=p$. Find $p$ if they are 
\begin{enumerate}
\item mutually exclusive
\item independent
\end{enumerate}
\solution
\begin{enumerate}
	\item 
\begin{align}
\frac{3}{5}&=\frac{1}{2}+p&
\\
\therefore  p &= \frac{1}{10}&
\end{align}
\item 
\begin{align}
\frac{3}{5}&=\frac{1}{2}+ p - \frac{p}{2}&
\\
\therefore p &= \frac{1}{5}&
\end{align}
\end{enumerate}


\item  If A and B are two events such that $\pr{A} = \frac{1}{4}, \pr{B} = \frac{1}{2}$ and $\pr{A B} = \frac{1}{8}$, find $\pr{\text{not A and not B}}$.
	\\
\solution
\begin{align}
	\pr{A+B} 
	&= \frac{1}{4} +\frac{1}{2} - \frac{1}{8} \\
	&= \frac{5}{8}
\end{align}
Hence, 
\begin{align}
	\pr{A^{\prime}B^{\prime}} 
	= 1-\pr{\brak{A+B}}
	= \frac{3}{8}
\end{align}

\item Events A and B are such that 
\begin{align}
    \pr{A}= \frac{1}{2},\pr{B} = \frac{7}{12},  \pr{A^{\prime}+B^{\prime}} = \frac{1}{4}.
\end{align}
State whether A and B are independent. 
	\\
\solution
\begin{align}
\pr{AB}&=1-\pr{A^{\prime}+B^{\prime}}
       =\frac{3}{4},\\
\pr{A}\times\pr{B}&
                  =\frac{7}{24}\\
\implies              \pr{AB}&\neq\pr{A}\pr{B}
\end{align}
$\therefore$ A and B are not independent.

%
%
\item Two events $A$ and $B$ will be independent, if
\begin{enumerate}
\item $A$ and $B$ are mutually exclusive
\item P(not $A$ $\cap$ not $B$) = $\sbrak{1-P(A)}\sbrak{1-P(B)} $
\item $P(A) = P(B)$
\item $P(A) + P(B) = 1$
\end{enumerate}
\solution
\begin{enumerate}
 \item Let 
\label{it:ncert/12/13/2/18/}
 \begin{align}
 \pr{A} = \pr{B} &= \frac{1}{2}
 \implies \pr{A}\times\pr{B} = \frac{1}{4}\\
	 \text{or, }  \pr{AB} &= 0 \neq \pr{A}\times\pr{B}
 \end{align}
  Hence A and B are not independent.
  \item  
\begin{align}
 \pr{A^\prime B^\prime}&=[1-\pr{A}] [1-\pr{B}]
 \\
  \implies 1 - \pr{A + B}&= 1 - \pr{A} - \pr{B} + \pr{A}\pr{B}\\
  \implies \pr{AB} &=  \pr{A}\pr{B}
 \end{align}
Thus, $A$ and $B$ are independent. 
  \item In \ref{it:ncert/12/13/2/18/}, \pr{A} = \pr{B}, but A and B are not independent.
  \item In \ref{it:ncert/12/13/2/18/}, \pr{A} + \pr{B}=1, but A and B are not independent.
  \end{enumerate}


%
\item The probability that at least one of the two events $A$ and $B$ occurs is 0.6. If $A$ and $B$ occur simultaneously with probability 0.3, evaluate $\pr{A^\prime} + \pr{B^\prime}$.\\
\solution
Given:
\begin{align}
    \pr{AB} &= 0.3\\
    \pr{A + B} &= 0.6\\
    &=  \pr{A} + \pr{B} - \pr{AB}\\
    \implies 0.6 &= \pr{A} + \pr{B} - 0.3\\
    \implies 0.9 &= \pr{A} + \pr{B}
\end{align}
But
\begin{align}
    \pr{A^\prime} &= 1 - \pr{A}\\
    \pr{B^\prime} &= 1 - \pr{B}\\
    \therefore \pr{A^\prime} + \pr{B^\prime} &= 2 - \brak{\pr{A} + \pr{B}}\\
    &= 2 - 0.9
    = 1.1
\end{align}

\item Prove that
	\begin{enumerate}
	\item $\pr{A} = \pr{AB} + \pr{AB^\prime}$
	\item $\pr{A+B} = \pr{AB} + \pr{AB^\prime} + \pr{A^\prime B}$
\label{it:exemplar/12/13/3/11}
	\end{enumerate}
\solution
\begin{enumerate}
\item See
\eqref{eq:axiom_sum_A}.
\item From 
\eqref{eq:axiom_sum_A}
and
\eqref{eq:axiom_sum_AB},
\begin{align}
	\pr{A} &= \pr{AB} + \pr{AB^\prime}
\label{eq:exemplar/12/13/3/11/1}
	\\
	\pr{B} &= \pr{AB} + \pr{A^\prime B}
	\\
\pr{A+B} &= \pr{A} + \pr{B} - \pr{AB} 
\label{eq:exemplar/12/13/3/11/2}
\end{align}
yielding
item \ref{it:exemplar/12/13/3/11} after addition. 
\end{enumerate}



\item $A$ and $B$ are events such that $\pr{A} = 0.4$, $\pr{B} = 0.3$ and $\pr{A+B} = 0.5$. Find $\pr{B^\prime A}$.
	\\
\solution
Adding \eqref{eq:exemplar/12/13/3/11/1}
and
\eqref{eq:exemplar/12/13/3/11/2},
\begin{align}
\pr{A+B} &=  \pr{B}+ \pr{AB^\prime} 
\\
\implies
	\pr{AB^\prime}  &= \pr{A+B} -  \pr{B} = 0.2
\end{align}

\end{enumerate}
State True or False.  
\begin{enumerate}[label=\thesubsection.\arabic*,ref=\thesubsection.\theenumi,resume*]
\item 
If $\pr{A}>0$ and $\pr{B}>0$, then $A$ and $B$ can be mutually exclusive and independent. \\
\solution 
Since $\pr{A}>0$ and $\pr{B}>0$, 
\begin{align}
\label{eq:12.13.3.94/1}\pr{A}\pr{B}>0
\end{align}
For $\pr{A}$ and $\pr{B}$ to be mutually exclusive and independent,  
\begin{align}
\label{eq:12.13.3.94/2} \pr{AB}&=0  \\
\label{eq:12.13.3.94/3} \pr{AB}&=\pr{A}\pr{B} \\ 
\implies \label{eq:12.13.3.94/4}\pr{A}\pr{B}&=0
\end{align}
which 
contradicts \eqref{eq:12.13.3.94/1}.
Hence, the above statement is false.

\item If $A$ and $B$ are independent events, then $A'$ and $B'$ are also independent.\\
\solution
Given that 
\begin{align}
\pr{AB} = \pr{A}\pr{B}
\end{align}
If $A^{\prime}$  and $B^{\prime}$ are independent,
\begin{align}
	\pr{A^{\prime} B^{\prime}} &= \pr{A+B}^{\prime}
	=1 - \pr{A+B}
	\\
	&= 1 - \pr{A}-
	 - \pr{B} +  \pr{AB}
	 \\
	&= 1 - \pr{A}-
	 - \pr{B} +  \pr{A}\pr{B}
	 \\
	&= \sbrak{1 - \pr{A}}
	\sbrak{ 1 - \pr{B}}  
	\\
	&=
\pr{A^{\prime}} 
\pr{B^{\prime}} 
\end{align}
Hence, $A^{\prime}$ and $B^{\prime}$ are also independent. 
Therefore, the given statement is true.


\item If $A$ and $B$ are mutually exclusive events, $\pr{A}=0.35$ and $\pr{B}=0.45$ then find
	\vspace{-5mm}
\begin{multicols}{4}
\begin{enumerate}
\item $\pr{A'}$
\item $\pr{B'}$
\item $\pr{A+B}$
\item $\pr{AB}$
\item $\pr{AB'}$
\item $\pr{A'B'}$
\end{enumerate}
\end{multicols}
%
	\vspace{-5mm}
\solution 
See
\tabref{tab:11/16/3/7}.
\begin{table}[ht]
\centering
\caption{}
\label{tab:11/16/3/7}
\begin{tabular}{|l|l|l|}
\hline
	\textbf{Item} & \textbf{Formula} & \textbf{Value} \\
\hline
$\pr{A'}$ & $1-\pr{A}$ & $0.65$ \\
\hline
$\pr{B'}$ & $1-\pr{B}$ & $0.55$ \\
\hline
$\pr{A+B}$& $\pr{A}+\pr{B}-\pr{AB}$ & $0.80$ \\
\hline
$\pr{AB}$ & $\because AB = 0$ & $0$ \\
\hline
$\pr{AB'}$ & $\pr{A}-\pr{AB}$& $0.35$ \\
\hline
$\pr{A'B'}$ & $1-\pr{A+B}$ & $0.20$ \\
\hline
\end{tabular}
\end{table}

%
\item The accompanying Venn diagram shows three events, A, B, and C, and also the probabilities of the various intersections (for instance, $\pr{AB}=0.7$. Determine 
\begin{multicols}{3}
	\begin{enumerate}
		\item \pr{A}
		\item \pr{BC'}
		\item \pr{A+B}
		\item \pr{AB'}
		\item \pr{BC}
		\item		\begin{minipage}[t]{\linewidth}
			Probability that exactly one of the three occurs
    \end{minipage}
	\end{enumerate}
\end{multicols}
	\begin{figure}[H]
		\centering
		\input{exemplar/11/16/3/11/figs/fig.tex}
		\caption {}
		\label{fig:exemplar/11/16/3/11}
	\end{figure}
\begin{table}[H]
\centering
%\scriptsize
%\footnotesize
%\small
	\input{exemplar/11/16/3/11/tables/11.16.3.11.tex }
\caption{}
\label{tab:11/16/3/11}
\end{table}
		\solution
		See 
\tabref{tab:11/16/3/11}.
\figref{fig:exemplar/11/16/3/11} is used to obtain the input probabilities.
\begin{enumerate}
	\item
\begin{align}
BC^\prime	
=
	BC^\prime\brak{A+A^\prime}=	
	BC^\prime A+BC^\prime A^\prime	
\label{eq:11/16/3/11/1}
\end{align}
Also, 
\begin{align}
	AB&=
	AB\brak{C+C^\prime}
	\\
	&=	
	ABC+ABC^\prime = ABC^\prime  \, \because AC=0.
\label{eq:11/16/3/11/2}
\end{align}
From 
\eqref{eq:11/16/3/11/1}
and
\eqref{eq:11/16/3/11/2},
\begin{align}
BC^\prime = AB+A^\prime BC^\prime
\\
\implies
	\pr{BC^\prime} = \pr{AB}+\pr{A^\prime BC^\prime}
\end{align}
\item Also, 
\begin{align}
\pr{B} = \pr{BC}+\pr{BC^\prime} = 0.17+0.15 = 0.32
\end{align}
from 
\tabref{tab:11/16/3/11}.  This is used to evaluate $\pr{A+B}$.
\item $\because AC=0$
\begin{align}
	\pr{AB}=\pr{AB^\prime C^\prime} +\pr{A B^\prime C} &= \pr{AB^\prime C^\prime}
	\\
	\pr{B^\prime C} &= \pr{A^\prime B^\prime C}
\end{align}
\end{enumerate}

\item The probability of happening of an event $A$ is 0.5 and that of $B$ is 0.3. If $A$ and $B$ are mutually exclusive events, then the probability of neither $A$ nor $B$ is \underline{\phantom{Blank}}.
	\\	
\solution
		\begin{align}
	\because \pr{AB} &= 0,
	\\
	\pr{\brak{A+B}^\prime} &= 1 - Pr(A+B) = 1-Pr(A)-Pr(B)\\
	&= 0.2
\end{align}
which is  the desired probability.  

\item If $A$ and $B$ are mutually exclusive events,then
\begin{multicols}{2}
\begin{enumerate}
\item $\pr{A} \leq\pr{B^\prime}$
\item $\pr{A} \geq \pr{B^\prime}$
\item $\pr{A} < \pr{B^\prime}$
\item $\text{none of these}$
\end{enumerate}
\end{multicols}
\solution 
\begin{align}
\because \pr{AB} &= 0
\\
\pr{A+B} \leq 1 
\implies\pr{A}+\pr{B} &\leq 1\\
\implies \pr{A} &\leq\pr{B^\prime}.
\end{align}
where we have used the axiom of probability.

\item The probability of an occurrence of event $A$ is .7 and that of the occurrence of event $B$ is .3 and the probability of occurrence of both is .4. Is this statement true or false?\\
\solution
\begin{align}
\pr{AB} > \pr{B} 
\end{align}        
which violates
	\eqref{eq:axiom-prod-ge}.
Hence, the given statement is false.


\item If \pr{A+B} = \pr{AB} for any two events $A$ and $B$ , then
	\begin{multicols}{4}
\begin{enumerate}
\item $\pr{A}=\pr{B}$
\item $\pr{A} > \pr{B}$
\item $\pr{A} < \pr{B}$
\item none of these
\end{enumerate}
\end{multicols}
\solution
\begin{align}
\pr{A}+\pr{B}-\pr{AB}&=\pr{A+B}\\
\implies \pr{A}+\pr{B}-\pr{AB}&=\pr{AB}\\
\implies\sbrak{\pr{A}-\pr{AB}}+\sbrak{\pr{B}-\pr{AB}}&= 0
\label{eq:11_16_3_24_1}
\end{align}
However, from 
	\eqref{eq:axiom-prod-ge},
\begin{align}
	\begin{split}
	\pr{A}-\pr{AB} &\ge 0\\
	\pr{B}-\pr{AB} &\ge 0
	\end{split}
\label{eq:11_16_3_24_2}
\end{align}
From \eqref{eq:11_16_3_24_1} and \eqref{eq:11_16_3_24_2}, 
\begin{align}
	\pr{A} =\pr{B} =\pr{AB}.
\end{align}

\item If $A$ and $B$ are such that 
$\Pr(A' \cup B') = \frac{2}{3}$ and $\Pr(A \cup B) = \frac{5}{9}$, 
then $\Pr(A') + \Pr(B')$ = \\
\solution
   Using De Morgan's law and axioms of probability,
\begin{align}
\pr{\brak{A+B}^\prime } &=\pr{A^\prime B^\prime }  \\
\pr{A^\prime  + B^\prime } &= \pr{A^\prime } + \pr{B^\prime } - \pr{A^\prime B^\prime }
\end{align}
Adding the above, 
\begin{align}
	\pr{A^\prime } + \pr{B^\prime }  = 1+\pr{A^\prime  + B^\prime }-\pr{A+B}
	=\frac{10}{9}
\end{align}

\item If $A$ and $B$ are independent, then 
	$\pr{\text{exactly one of } A, B \text{ occurs}} =\pr{B}\pr{A^{\prime}}+\pr{A}\pr{B^{\prime}}$.\\
\solution 
\begin{align}
	\because \pr{AB} &= \pr{A} \pr{B}
	\\
	\pr{AB^\prime } = \pr{A} \pr{B^\prime },
	\pr{A^\prime B} &= \pr{A^\prime } \pr{B}
	\\
	\implies \pr{A^\prime B+AB^\prime } &= \pr{A^\prime B}+\pr{A^\prime B}
	\\
	&=\pr{A} \pr{B^\prime }+ \pr{A^\prime } \pr{B}.
\end{align}

\item Given two independent events A and B such that $P(A) = 0.3$, $P(B) = 0.6$. Find
		\label{ncert/12/13/2/10}
		\vspace{-4mm}
\begin{multicols}{2}
\begin{enumerate}
\item $P(A\text{ and } B)$
\item $P(A \text{ and not } B)$
\item $P(A \text{ or } B)$
\item $P(\text{neither } A \text{ nor } B)$
\end{enumerate}
\end{multicols}
		\vspace{-3mm}
\item The probability distribution of a discrete random variable $X$ is given below
in \tabref{tab:exemplar/12/13/3/87}.
	The value of $k$ is equal to
\begin{multicols}{4}
\begin{enumerate}
\item 8
\item 16
\item 32
\item 48
\end{enumerate}
\end{multicols}
\begin{table}[htb]
\centering
\input{exemplar/12/13/3/87/tables/Book1.tex}
\caption{}
\label{tab:exemplar/12/13/3/87}
\end{table}
\solution
	From \eqref{eq:dist-axiom-one},
\begin{align} 
\frac{5}{k} + \frac{7}{k} + \frac{9}{k} + \frac{11}{k} = 1\\
\implies 
k = 32 
\end{align}



\item State True or False for the given statement:
Two independent events are always mutually exclusive.\\
\solution 
The given condition can be expressed as
\begin{align}
 \pr{AB}=\pr{A} \times \pr{B}
 0
	\\
	\implies \pr{A} = 0 \text{or, } \pr{B} = 0,
 \end{align}
 which is not always true.

\item If $A$ and $B^\prime$ are independent events, then $\pr{A^\prime+B}=1-\pr{A}\pr{B^\prime}$.\\
\solution
\begin{align}
\pr{A^\prime+B}&=\pr{(AB^\prime)^\prime}\\
	&=1-\pr{AB^\prime}
\\
&=1-\pr{A}\pr{B^\prime}.
\end{align}


\item Let $E_1$ and $E_2$ be two independent events such that $\pr{E_1} = p_1 $ and $ \pr{E_2} = p_2 $  Describe in words the events whose probabilities are
		\vspace{-3mm}
\begin{multicols}{2}
\begin{enumerate}
\item \ $p_1 p_2$
\quad\item \ $(1 - p_1) p_2 $
\quad\item \ $1 - (1 - p_1)(1 - p_2)$
\quad\item \ $p_1 + p_2 - 2p_1 p_2$
\end{enumerate}
\end{multicols}
		\vspace{-3mm}
\solution
\begin{enumerate}
\item \begin{align}
p_1p_2&=\pr{E_1}\pr{E_2}\\
&=\pr{E_1E_2}
\end{align}
So, $E_1$ and $E_2$ occur simultaneously.
\item \begin{align}
(1 - p_1)(p_2) &= \pr{E_1^\prime}\pr{E_2} \\
&= \pr{{E_1^\prime}E_2}
\end{align}
So $E_1$ does not occur but $E_2$  occurs.
\item \begin{align}
1 - (1 - p_1)(1 - p_2) &= 1 - \pr{E_1^\prime} \pr{E_2^\prime} \\
&= 1 - \pr{E_1^\prime \ E_2^\prime} \\
&= \pr{E_1+E_2} 
\end{align}
So, either $E_1$ or $E_2$ or both $E_1$ and $E_2$  occurs.
\item \begin{align}
p_1 + p_2 - 2p_1p_2 &= \pr{E_1} + \pr{E_2} - 2\pr{E_1} \pr{E_2} \\
&=  \pr{E_1}-\pr{E_1}\pr{E_2}+ \pr{E_2} - \pr{E_1} \pr{E_2}\\
&= \pr{E_1}\brak{1-\pr{E_2}}+\pr{E_2} \brak{1-\pr{E_1}}\\
&=\pr{E_1}\pr{E_2^\prime}+\pr{E_2}\pr{E_1^\prime}\\
&=\pr{E_1E_2^\prime+E_1^\prime E_2}
\end{align}
So, either $E_1$ or $E_2$ occurs but not both.
\end{enumerate}




\end{enumerate}

    From the given data, 
    \begin{align}
        \pr{A} = \frac{30}{60} = \frac{1}{2},\ 
        \pr{B} = \frac{32}{60} = \frac{8}{15},\ 
        \pr{AB}= \frac{24}{60} = \frac{2}{5}.
    \end{align}
    Thus, the desired probabilities are
\begin{enumerate}
	\item $\pr{A+B}=\frac{19}{30}$, from 
\eqref{eq:axiom_sum_AB}.
	\item From
	\eqref{eq:demorgan}
	and the axioms of probability,
        \begin{align}
\pr{A^\prime  B^\prime } =1-\pr{A+B}= \frac{11}{30}.
\end{align}
	\item     
		\begin{align}
	    \pr{A^\prime B}        &= \pr{B} - \pr{AB} = \frac{2}{15} 
    \end{align}
    from
\eqref{eq:axiom-cond-prime}.
\end{enumerate}

\newpage
\section{Bernoulli}
\subsection{Formulae}
\begin{enumerate}[label=\thesubsection.\arabic*,ref=\thesubsection.\theenumi]
	\item The Bernoulli distribution $X \in \cbrak{0,1}$ is defined as
\begin{align}
	X=\text{Ber}(p).
\end{align}
with pmf
\begin{align}
	p_X(k) = 
	\begin{cases}
		1-p & k = 0
		\\
		p & k = 1
		\\
		0 & \text{otherwise}
\end{cases}
\end{align}
\item For a Bernoulli random variable $X$ with success probability $p$,
\begin{align}
M_X(z)= q+pz^{-1}
\end{align}
\item The mean of the Bernoulli distribution is 
\begin{align}
	E\brak{X} = p
\end{align}
\item 	The following code simulates 100 coin tosses
	\lstinputlisting{
codes/bernoulli/coin.py
}
\end{enumerate}

\subsection{NCERT}
\begin{enumerate}[label=\thesubsection.\arabic*,ref=\thesubsection.\theenumi]
	\item A lot consists of 144 ball pens of which 20 are defective and the 
    others are good. Navami will buy a pen if it is good, but will not buy if it 
    is defective. The shopkeeper draws one pen at random and gives it to her. 
    What is the probability that
    \begin{enumerate}
        \item She will buy it?
        \item She will not buy it?
    \end{enumerate}
\solution
		In this case, we have $X \sim Ber\brak{\frac{67}{72}}$. 
Therefore, the desired probabilities are
    \begin{enumerate}
        \item 
	$p_{X}(1) = \frac{67}{72}$
        \item 
	$p_{X}(0) = \frac{5}{72}$
    \end{enumerate}

\item A school has five houses A, B, C, D and E. A class has 23 students, 4 from house
A, 8 from house B, 5 from house C, 2 from house D and rest from house E. A
single student is selected at random to be the class monitor. The probability that the
selected student is not from A, B and C is \\
\solution
See 
  \tabref{tab:exemplar/10/13/1/26/Table1}
\begin{table}[H]
	\centering
  %%%%%%%%%%%%%%%%%%%%%%%%%%%%%%%%%%%%%%%%%%%%%%%%%%%%%%%%%%%%%%%%%%%%%%
%%                                                                  %%
%%  This is a LaTeX2e table fragment exported from Gnumeric.        %%
%%                                                                  %%
%%%%%%%%%%%%%%%%%%%%%%%%%%%%%%%%%%%%%%%%%%%%%%%%%%%%%%%%%%%%%%%%%%%%%%
\begin{tabular}{|c|l|l|}\hline
	\textbf{Variable}&\textbf{Description}&\textbf{Probability}\\\hline
A	&Person with heat attack	&\pr{A}=0.40\\\hline
$E_1$	&Person treated with meditation and yoga	&\pr{E_1}=0.50\\\hline
$E_2$	&Person treated with drug	&\pr{E_2}=0.50\\\hline
\end{tabular}

  \caption{Student distribution in each house}
  \label{tab:exemplar/10/13/1/26/Table1}
\end{table}
Define
\begin{align}
    X=\begin{cases}
	    0 & \text{A, B and C}
	    \\
	    1 & \text{Not from A, B and C}
    \end{cases}
\end{align}
Then, from
\tabref{tab:Table1},
\begin{align}
	X \sim \text{Ber}\brak{\frac{6}{23}}
\end{align}
and the desired probability is
\begin{align}
    p_X\brak{1}=\frac{6}{23}
\end{align}

\item A bag contains slips numbered from 1 to 100. If Phulan chooses a slip at random
from the bag, it will either be an odd number or an even number. Since this situation
has only two possible outcomes, so, the probability of each is $\frac{1}{2}$. Justify.\\
\solution
Let 
\begin{align}
	 X =
	\begin{cases}
		1, & \text{if number is even}\\ 
		0, & \text{if number is odd}
	\end{cases}
\end{align}
Then  
\begin{align}
               p_X\brak{1} &= \frac{50}{100}
               =\frac{1}{2}
	       \\
               p_X\brak{0} &= \frac{50}{100}
               =\frac{1}{2}      
\end{align}
and $X \sim \text{Ber}\brak{\frac{1}{2}}$.



\item A letter of English alphabets is chosen at random. Determine the probability that the letter is a consonant.\\
    \solution
    The desired probability is
\begin{align}
            p =\frac{21}{26}         
\end{align}


\item A carton of 24 bulbs contain 6 defective bulbs. One bulb is drawn at random.
What is the probability that the bulb is not defective? If the bulb selected is defective
and it is not replaced and a second bulb is selected at random from the rest, what
is the probability that the second bulb is defective?\\
\solution
Let 
\begin{align}
	 X_1 =
	\begin{cases}
		1, & \text{if bulb is not defective}\\ 
		0, & \text{if bulb is defective}
	\end{cases}
\end{align}
Then the Bernoulli parameter 
\begin{align}
	p_1 = 1 - \frac{6}{24}
               =\frac{3}{4}          
\end{align}
which is the desired probability.  In the second case, 
\begin{align}
1-p_2 = \frac{6-1}{24-1}
=\frac{5}{23}
\end{align}
which is the desired probability.


\item An integer is chosen between 0 and 100. What is the probability that it is 
	\begin{enumerate}
 		\item divisible by 7
   		\item not divisible by 7
     	\end{enumerate}
\solution
Let $X$ be a random variable such that 
\begin{align}
	X = \begin{cases}
		0 & n \not\equiv 0 \pmod{7}\\
		1 & n \equiv 0 \pmod{7}\end{cases}
\end{align}
Then, 
\begin{enumerate}
	\item $p=\frac{14}{99}$
	\item $1-p= \frac{85}{99}$
\end{enumerate}

\item If the letters of the word \textbf{ALGORITHM} are arranged at random in a row what
is the probability the letters GOR must remain together as a unit?\\
\solution
Let 
\begin{align}
	 X =
	\begin{cases}
		1, & \text{if GOR remain together as a unit}\\ 
		0, & \text{otherwise}
	\end{cases}
\end{align}
Then  
\begin{align}
               p= \frac{7!}{9!}
               =\frac{1}{72}
\end{align}


\item Six new employees, two of whom are married to each other, are to be assigned six desks that
are lined up in a row. If the assignment of employees to desks is made randomly, what is the
probability that the married couple will have nonadjacent desks?\\
\solution
Let $X$ be a Random variable such that 
\begin{table}[h]\centering
	\input{exemplar/11/16/3/2/tables/table.tex}
\end{table}
\begin{align}
	1-p &= \frac{5! \times 2}{6!}
	= \frac{1}{3}\\
	\implies p &= 
	 \frac{2}{3}
\end{align}


\item There are four men and six women on the city council.
If one council member is selected for a committee at random, how likely is it that it is a woman?\\
\solution
\begin{table}[H]
	\centering
    %%%%%%%%%%%%%%%%%%%%%%%%%%%%%%%%%%%%%%%%%%%%%%%%%%%%%%%%%%%%%%%%%%%%%%
%%                                                                  %%
%%  This is a LaTeX2e table fragment exported from Gnumeric.        %%
%%                                                                  %%
%%%%%%%%%%%%%%%%%%%%%%%%%%%%%%%%%%%%%%%%%%%%%%%%%%%%%%%%%%%%%%%%%%%%%%
\begin{tabular}{|c|l|l|}\hline
	\textbf{Variable}&\textbf{Description}&\textbf{Probability}\\\hline
A	&Person with heat attack	&\pr{A}=0.40\\\hline
$E_1$	&Person treated with meditation and yoga	&\pr{E_1}=0.50\\\hline
$E_2$	&Person treated with drug	&\pr{E_2}=0.50\\\hline
\end{tabular}

    \caption{Council distribution}
    \label{tab:Table1}
    \end{table}
\begin{align}
    X=\begin{cases}
        0,& \text{if member is a man}\\
        1,& \text{if member is a woman}
    \end{cases}
\end{align}
From \tabref{tab:Table1}
\begin{align}
    p=\frac{6}{6+4}=\frac{3}{5}
\end{align}

\item A girl calculates that the probability of her winning the first prize in a lottery is 0.08. If 6000 tickets are sold,how many tickets has she bought?
	\begin{multicols}{4}
    \begin{enumerate}
	    \item  40  
	    \item 240  
	    \item 480  
	    \item 750 
\end{enumerate}
\end{multicols}
\solution
%
\begin{table}[H]
	\centering
	\input{exemplar/10/13/1/23/tables/table.tex}
	\caption{Information table}
	\label{table:ncert/10/13/1/23/}	
\end{table}
See
	\tabref{table:ncert/10/13/1/23/}.
The number of tickets bought is
\begin{align}
Np = 0.08 \times 6000
= 480
\end{align}

\item Three numbers are chosen from 1 to 20. Find the probability that they are not consecutive
	\begin{multicols}{4}
\begin{enumerate}
	\item $\frac{186}{190}$
	\item $\frac{187}{190}$
	\item $\frac{188}{190}$
	\item $\frac{18}{\comb{20}{3}}$ 
\end{enumerate}
\end{multicols}
\solution
\begin{table}[H]
\centering
	\input{exemplar/11/16/3/19/table/table.tex}
	\caption{Random variable}
	\label{11.16.3.19}
\end{table}
	See \tabref{11.16.3.19}.
The number of sets of three consecutive numbers from 1 to 20 is 18.  Hence,  
\begin{align}
	p&=\frac{18}{\comb{20}{3}}
	\\
	\implies 
	1-p&=1-\frac{18}{\comb{20}{3}}
	=\frac{187}{190}
\end{align}


 

\item Seven persons are to be seated in a row. What is the probability that two particular persons sit next to each other?\\
\solution
\begin{table}[H]
    \centering
    \input{exemplar/11/16/3/21/tables/randomvar.tex}
    \caption{}
    \label{tab:11/16/3/21}
\end{table}
    See \tabref{tab:11/16/3/21}.
The number of ways to arrange 7 people is $7!$ and the number of ways to arrange 7 people in which the two particular people are adjacent to each other is $6! \times 2$ considering both of them as one unit and considering the arrangements within the unit. Thus,
\begin{align}
    p = \frac{6! \times 2}{7!}
    = \frac{2}{7}
\end{align}

\item A single letter is selected at random from the word `PROBABILITY'. The
probability that it is a vowel is
\rule{1cm}{0.01pt}.
\\
\solution 
Let $X$ be an bernoulli rv defined as in \tabref{tab:exemplar/11/16/3/26}.  Then, 
\begin{equation}
    p =
        \frac{4}{11} 
\end{equation}
\begin{table}[H]
	\centering
	\input{exemplar/11/16/3/26/tables/Table2.tex}
	\caption{}
        \label{tab:exemplar/11/16/3/26}
\end{table}

\item A letter is chosen at random from the word ‘ASSASSINATION’. Find the probability that letter is
\rule{1cm}{0.01pt}.
\begin{enumerate}
  \item  a vowel 
   \item    a consonant 
\end{enumerate}
\solution
The number of vowels is 6 and consonants is 7. Therefore,
\begin{enumerate}[itemsep=1ex]
  \item  
$ p 
    =\frac{6}{13}
    $
 \item    
	 $
1-p  
    =\frac{7}{13}
    $
\end{enumerate}

\item A box contains 12 balls, out of which $x$ are black. If one ball is drawn at random from the box, what is the probability that it will be a black ball? If 6 more black balls are put in the box, the probability of drawing a black ball is now double of what it was before. Find $x$.\\
\solution
From
\tabref{tab:10/15/2/4/1},
\begin{align}
	p_1 &= \frac{x}{12}, 
p_2= \frac{x+6}{18}
\\
	\because p_2&=2p_1,
\frac{x+6}{18}	 =2\brak{\frac{x}{12}}\\
\implies
x &=3
\end{align}
\begin{table}[H]
\centering
%%%%%%%%%%%%%%%%%%%%%%%%%%%%%%%%%%%%%%%%%%%%%%%%%%%%%%%%%%%%%%%%%%%%%%
%%                                                                  %%
%%  This is a LaTeX2e table fragment exported from Gnumeric.        %%
%%                                                                  %%
%%%%%%%%%%%%%%%%%%%%%%%%%%%%%%%%%%%%%%%%%%%%%%%%%%%%%%%%%%%%%%%%%%%%%%
\begin{tabular}{|c|l|l|}\hline
	\textbf{Variable}&\textbf{Description}&\textbf{Probability}\\\hline
A	&Person with heat attack	&\pr{A}=0.40\\\hline
$E_1$	&Person treated with meditation and yoga	&\pr{E_1}=0.50\\\hline
$E_2$	&Person treated with drug	&\pr{E_2}=0.50\\\hline
\end{tabular}

\caption{ }
\label{tab:10/15/2/4/1}
\end{table}

\item Gopi buys a fish from a shop for his aquarium. The shopkeeper takes out one fish at random from a tank containing 5 male fish and 8 female fish . What is the probability that the fish taken out is a male fish?
	\\
\solution
For
\begin{align}
	X &=
	\begin{cases}
		1 & \text{male}\\ 
		0 & \text{female},
	\end{cases}
	\\
	p &= \frac{5}{13}
 \end{align}

\item A bag contains 3 red balls and 5 black balls. A ball is drawn at random from the bag.
What is the probability that the ball drawn is
\begin{enumerate}[label=(\roman*)] \item red ? \item not red? \end{enumerate}
\solution
Let 
\begin{align}
X= \begin{cases} 
      1 & \text{if drawn ball is red} \\
      0 & \text{otherwise.} 
   \end{cases}
\end{align}
\begin{enumerate}[label=(\roman*)]
    \item Probability that the drawn ball is red
    \begin{align}
    \pr{X=1}
    = \frac{3}{8}\\
    \end{align}
    \item Probability that the drawn ball is not red
    \begin{align}
     \pr{X=0}= 1-\frac{3}{8} = \frac{5}{8}
    \end{align}
\end{enumerate}



	\item A lot consists of 144 ball pens of which 20 are defective and the 
    others are good. Nuri will buy a pen if it is good, but will not buy if it 
    is defective. The shopkeeper draws one pen at random and gives it to her. 
    What is the probability that
    \begin{enumerate}
        \item She will buy it?
        \item She will not buy it?
    \end{enumerate}
\solution
%		In this case, we have $X \sim Ber\brak{\frac{67}{72}}$. 
Therefore, the desired probabilities are
    \begin{enumerate}
        \item 
	$p_{X}(1) = \frac{67}{72}$
        \item 
	$p_{X}(0) = \frac{5}{72}$
    \end{enumerate}

\item A school has five houses A, B, C, D and E. A class has 23 students, 4 from house
A, 8 from house B, 5 from house C, 2 from house D and rest from house E. A
single student is selected at random to be the class monitor. The probability that the
selected student is not from A, B and C is \\
\solution
%See 
  \tabref{tab:exemplar/10/13/1/26/Table1}
\begin{table}[H]
	\centering
  %%%%%%%%%%%%%%%%%%%%%%%%%%%%%%%%%%%%%%%%%%%%%%%%%%%%%%%%%%%%%%%%%%%%%%
%%                                                                  %%
%%  This is a LaTeX2e table fragment exported from Gnumeric.        %%
%%                                                                  %%
%%%%%%%%%%%%%%%%%%%%%%%%%%%%%%%%%%%%%%%%%%%%%%%%%%%%%%%%%%%%%%%%%%%%%%
\begin{tabular}{|c|l|l|}\hline
	\textbf{Variable}&\textbf{Description}&\textbf{Probability}\\\hline
A	&Person with heat attack	&\pr{A}=0.40\\\hline
$E_1$	&Person treated with meditation and yoga	&\pr{E_1}=0.50\\\hline
$E_2$	&Person treated with drug	&\pr{E_2}=0.50\\\hline
\end{tabular}

  \caption{Student distribution in each house}
  \label{tab:exemplar/10/13/1/26/Table1}
\end{table}
Define
\begin{align}
    X=\begin{cases}
	    0 & \text{A, B and C}
	    \\
	    1 & \text{Not from A, B and C}
    \end{cases}
\end{align}
Then, from
\tabref{tab:Table1},
\begin{align}
	X \sim \text{Ber}\brak{\frac{6}{23}}
\end{align}
and the desired probability is
\begin{align}
    p_X\brak{1}=\frac{6}{23}
\end{align}

\item A bag contains slips numbered from 1 to 100. If Fatima chooses a slip at random
from the bag, it will either be an odd number or an even number. Since this situation
has only two possible outcomes, so, the probability of each is $\frac{1}{2}$.Justify.\\
\solution
%Let 
\begin{align}
	 X =
	\begin{cases}
		1, & \text{if number is even}\\ 
		0, & \text{if number is odd}
	\end{cases}
\end{align}
Then  
\begin{align}
               p_X\brak{1} &= \frac{50}{100}
               =\frac{1}{2}
	       \\
               p_X\brak{0} &= \frac{50}{100}
               =\frac{1}{2}      
\end{align}
and $X \sim \text{Ber}\brak{\frac{1}{2}}$.



\item A letter of English alphabets is chosen at random. Determine the probability that the letter is a consonant.\\
    \solution
    %The desired probability is
\begin{align}
            p =\frac{21}{26}         
\end{align}


\item A carton of 24 bulbs contain 6 defective bulbs. One bulbs is drawn at random.
What is the probability that the bulb is not defective? If the bulb selected is defective
and it is not replaced and a second bulb is selected at random from the rest, what
is the probability that the second bulb is defective?\\
\solution
%Let 
\begin{align}
	 X_1 =
	\begin{cases}
		1, & \text{if bulb is not defective}\\ 
		0, & \text{if bulb is defective}
	\end{cases}
\end{align}
Then the Bernoulli parameter 
\begin{align}
	p_1 = 1 - \frac{6}{24}
               =\frac{3}{4}          
\end{align}
which is the desired probability.  In the second case, 
\begin{align}
1-p_2 = \frac{6-1}{24-1}
=\frac{5}{23}
\end{align}
which is the desired probability.


\item An integer is chosen between 0 and 100. What is the probability that it is 
	\begin{enumerate}
 		\item divisible by 7
   		\item not divisible by 7
     	\end{enumerate}
\solution
%Let $X$ be a random variable such that 
\begin{align}
	X = \begin{cases}
		0 & n \not\equiv 0 \pmod{7}\\
		1 & n \equiv 0 \pmod{7}\end{cases}
\end{align}
Then, 
\begin{enumerate}
	\item $p=\frac{14}{99}$
	\item $1-p= \frac{85}{99}$
\end{enumerate}

\item In an examination, 20 questions of true-false type are asked. Suppose a student tosses a fair coin to determine his answer to each question. If the coin falls heads, he answer true; if it falls tails, he answer false. Find the probability that he answers at least 12 questions correctly.\\
\\
\solution
%\begin{table}[H]
	\centering
\begin{tabular}{|c|c|c|}
\hline
Random variable &Value &Definition\\ \hline
\multirow{3}{*}{X} &0 &Slips of Rs 1\\
&1 &Slips of Rs 5\\
&2 &Slips of Rs 13\\ \hline
\multirow{2}{*}{Y} &0 &Box A\\
&1 &Box B\\\hline
\end{tabular}
\caption{}
\label{tab:Distribution}
\end{table}
See \tabref{tab:Distribution}.
\begin{align}
p_{Y}\brak{k}= \begin{cases} 
      \frac{1}{3} & {k=0} \\
      \frac{2}{3 }& {k=1} 
   \end{cases}
   \\
p_{Y|X}\brak{0|0} = \frac{19}{25}\, 
p_{Y|X}\brak{0|1} = \frac{6}{25}\,
p_{Y|X}\brak{1|0} = \frac{45}{50}\,
p_{Y|X}\brak{1|2} = \frac{5}{50}
\end{align}
The desired probability is the probability that a slip drawn at random is marked other than Rs 1,
\begin{align}
&=1-p_X\brak{0}\\
&= p_X(1) + p_X(2)
\end{align}
Using Bayes theorem,
\begin{align}
&= p_Y\brak{0} \times \pr{Y=0 | X=1} + p_Y\brak{1} \times \pr{Y=1|X=2}\\
&=\frac{1}{3} \times \frac{6}{25} + \frac{2}{3} \times \frac{5}{50}\\
&=\frac{11}{75}
\end{align}

\newpage

%\tableofcontents

\bigskip

\renewcommand{\thefigure}{\theenumi}
\renewcommand{\thetable}{\theenumi}
%\renewcommand{\theequation}{\theenumi}

%\begin{abstract}
%%\boldmath
%In this letter, an algorithm for evaluating the exact analytical bit error rate  (BER)  for the piecewise linear (PL) combiner for  multiple relays is presented. Previous results were available only for upto three relays. The algorithm is unique in the sense that  the actual mathematical expressions, that are prohibitively large, need not be explicitly obtained. The diversity gain due to multiple relays is shown through plots of the analytical BER, well supported by simulations. 
%
%\end{abstract}
% IEEEtran.cls defaults to using nonbold math in the Abstract.
% This preserves the distinction between vectors and scalars. However,
% if the journal you are submitting to favors bold math in the abstract,
% then you can use LaTeX's standard command \boldmath at the very start
% of the abstract to achieve this. Many IEEE journals frown on math
% in the abstract anyway.

% Note that keywords are not normally used for peerreview papers.
%\begin{IEEEkeywords}
%Cooperative diversity, decode and forward, piecewise linear
%\end{IEEEkeywords}



% For peer review papers, you can put extra information on the cover
% page as needed:
% \ifCLASSOPTIONpeerreview
% \begin{center} \bfseries EDICS Category: 3-BBND \end{center}
% \fi
%
% For peerreview papers, this IEEEtran command inserts a page break and
% creates the second title. It will be ignored for other modes.
%\IEEEpeerreviewmaketitle




\item If the letters of the word \textbf{ALGORITHM} are arranged at random in a row what
is the probability the letters GOR must remain together as a unit?\\
\solution
%Let 
\begin{align}
	 X =
	\begin{cases}
		1, & \text{if GOR remain together as a unit}\\ 
		0, & \text{otherwise}
	\end{cases}
\end{align}
Then  
\begin{align}
               p= \frac{7!}{9!}
               =\frac{1}{72}
\end{align}


\item Six new employees, two of whom are married to each other, are to be assigned six desks that
are lined up in a row. If the assignment of employees to desks is made randomly, what is the
probability that the married couple will have nonadjacent desks?\\
\solution
%Let $X$ be a Random variable such that 
\begin{table}[h]\centering
	\input{exemplar/11/16/3/2/tables/table.tex}
\end{table}
\begin{align}
	1-p &= \frac{5! \times 2}{6!}
	= \frac{1}{3}\\
	\implies p &= 
	 \frac{2}{3}
\end{align}


\item There are four men and six women on the city council.
If one council member is selected for a committee at random,how likely is it that it is a woman?\\
\solution
%\begin{table}[H]
	\centering
    %%%%%%%%%%%%%%%%%%%%%%%%%%%%%%%%%%%%%%%%%%%%%%%%%%%%%%%%%%%%%%%%%%%%%%
%%                                                                  %%
%%  This is a LaTeX2e table fragment exported from Gnumeric.        %%
%%                                                                  %%
%%%%%%%%%%%%%%%%%%%%%%%%%%%%%%%%%%%%%%%%%%%%%%%%%%%%%%%%%%%%%%%%%%%%%%
\begin{tabular}{|c|l|l|}\hline
	\textbf{Variable}&\textbf{Description}&\textbf{Probability}\\\hline
A	&Person with heat attack	&\pr{A}=0.40\\\hline
$E_1$	&Person treated with meditation and yoga	&\pr{E_1}=0.50\\\hline
$E_2$	&Person treated with drug	&\pr{E_2}=0.50\\\hline
\end{tabular}

    \caption{Council distribution}
    \label{tab:Table1}
    \end{table}
\begin{align}
    X=\begin{cases}
        0,& \text{if member is a man}\\
        1,& \text{if member is a woman}
    \end{cases}
\end{align}
From \tabref{tab:Table1}
\begin{align}
    p=\frac{6}{6+4}=\frac{3}{5}
\end{align}

\item A girl calculates that the probability of her winning the first prize in a lottery is 0.08. If 6000 tickets are sold,how many tickets has she bought?\\
(A) 40  (B)240  (C)480  (D)750 
%%
\begin{table}[H]
	\centering
	\input{exemplar/10/13/1/23/tables/table.tex}
	\caption{Information table}
	\label{table:ncert/10/13/1/23/}	
\end{table}
See
	\tabref{table:ncert/10/13/1/23/}.
The number of tickets bought is
\begin{align}
Np = 0.08 \times 6000
= 480
\end{align}

\item Three numbers are chosen from 1 to 20. Find the probability that they are not consecutive
\begin{enumerate}
	\item $\frac{186}{190}$\\
	\item $\frac{187}{190}$\\
	\item $\frac{188}{190}$\\
	\item $\frac{18}{\comb{20}{3}}$ 
\end{enumerate}
\solution
%\begin{table}[H]
\centering
	\input{exemplar/11/16/3/19/table/table.tex}
	\caption{Random variable}
	\label{11.16.3.19}
\end{table}
	See \tabref{11.16.3.19}.
The number of sets of three consecutive numbers from 1 to 20 is 18.  Hence,  
\begin{align}
	p&=\frac{18}{\comb{20}{3}}
	\\
	\implies 
	1-p&=1-\frac{18}{\comb{20}{3}}
	=\frac{187}{190}
\end{align}


 

\item Seven persons are to be seated in a row. What is the probabiity that two particular persons sit next to each other?\\
\solution
%\begin{table}[H]
    \centering
    \input{exemplar/11/16/3/21/tables/randomvar.tex}
    \caption{}
    \label{tab:11/16/3/21}
\end{table}
    See \tabref{tab:11/16/3/21}.
The number of ways to arrange 7 people is $7!$ and the number of ways to arrange 7 people in which the two particular people are adjacent to each other is $6! \times 2$ considering both of them as one unit and considering the arrangements within the unit. Thus,
\begin{align}
    p = \frac{6! \times 2}{7!}
    = \frac{2}{7}
\end{align}

\item A single letter is selected at random from the word ‘PROBABILITY’. The
probability that it is a vowel is
%Let $X$ be an bernoulli rv defined as in \tabref{tab:exemplar/11/16/3/26}.  Then, 
\begin{equation}
    p =
        \frac{4}{11} 
\end{equation}
\begin{table}[H]
	\centering
	\input{exemplar/11/16/3/26/tables/Table2.tex}
	\caption{}
        \label{tab:exemplar/11/16/3/26}
\end{table}

\item The probability of getting a bad egg in a lot of 400 is 0.035. The number of bad eggs in the lot is\\
\solution
%\begin{table}[H]
	\centering
\begin{tabular}{|c|c|c|}
\hline
Random variable &Value &Definition\\ \hline
\multirow{3}{*}{X} &0 &Slips of Rs 1\\
&1 &Slips of Rs 5\\
&2 &Slips of Rs 13\\ \hline
\multirow{2}{*}{Y} &0 &Box A\\
&1 &Box B\\\hline
\end{tabular}
\caption{}
\label{tab:Distribution}
\end{table}
See \tabref{tab:Distribution}.
\begin{align}
p_{Y}\brak{k}= \begin{cases} 
      \frac{1}{3} & {k=0} \\
      \frac{2}{3 }& {k=1} 
   \end{cases}
   \\
p_{Y|X}\brak{0|0} = \frac{19}{25}\, 
p_{Y|X}\brak{0|1} = \frac{6}{25}\,
p_{Y|X}\brak{1|0} = \frac{45}{50}\,
p_{Y|X}\brak{1|2} = \frac{5}{50}
\end{align}
The desired probability is the probability that a slip drawn at random is marked other than Rs 1,
\begin{align}
&=1-p_X\brak{0}\\
&= p_X(1) + p_X(2)
\end{align}
Using Bayes theorem,
\begin{align}
&= p_Y\brak{0} \times \pr{Y=0 | X=1} + p_Y\brak{1} \times \pr{Y=1|X=2}\\
&=\frac{1}{3} \times \frac{6}{25} + \frac{2}{3} \times \frac{5}{50}\\
&=\frac{11}{75}
\end{align}

\newpage

%\tableofcontents

\bigskip

\renewcommand{\thefigure}{\theenumi}
\renewcommand{\thetable}{\theenumi}
%\renewcommand{\theequation}{\theenumi}

%\begin{abstract}
%%\boldmath
%In this letter, an algorithm for evaluating the exact analytical bit error rate  (BER)  for the piecewise linear (PL) combiner for  multiple relays is presented. Previous results were available only for upto three relays. The algorithm is unique in the sense that  the actual mathematical expressions, that are prohibitively large, need not be explicitly obtained. The diversity gain due to multiple relays is shown through plots of the analytical BER, well supported by simulations. 
%
%\end{abstract}
% IEEEtran.cls defaults to using nonbold math in the Abstract.
% This preserves the distinction between vectors and scalars. However,
% if the journal you are submitting to favors bold math in the abstract,
% then you can use LaTeX's standard command \boldmath at the very start
% of the abstract to achieve this. Many IEEE journals frown on math
% in the abstract anyway.

% Note that keywords are not normally used for peerreview papers.
%\begin{IEEEkeywords}
%Cooperative diversity, decode and forward, piecewise linear
%\end{IEEEkeywords}



% For peer review papers, you can put extra information on the cover
% page as needed:
% \ifCLASSOPTIONpeerreview
% \begin{center} \bfseries EDICS Category: 3-BBND \end{center}
% \fi
%
% For peerreview papers, this IEEEtran command inserts a page break and
% creates the second title. It will be ignored for other modes.
%\IEEEpeerreviewmaketitle




\item someone is asked to take a number from 1 to 100.The probability that it is a prime number is\\
\solution
%See \tabref{tab:10.13.1.25}.
Since there are 25 prime numbers in between 1 to 100,
\begin{align}
p=\frac{25}{100}
=\frac{1}{4}
\end{align}
\begin{table}[H]
	\centering
\input{ncert/10/13/1/25/table/table.tex}
\caption{}
\label{tab:10.13.1.25}
\end{table}


    \end{enumerate}

\newpage
\section{Conditional Probability}
\subsection{Formulae}
\begin{enumerate}[label=\thesubsection.\arabic*,ref=\thesubsection.\theenumi]
\item 
\begin{align}
	\pr{A | B} = \frac{\pr{AB}}{\pr{B}}
\label{eq:axiom-cond}
\end{align}
If $A$ and $B$ are independent, from 
\eqref{eq:axiom-cond}
and 
\eqref{eq:axiom-ind},
\begin{align}
	\pr{A | B} = \frac{\pr{A}\pr{B}}{\pr{B}}={\pr{A}}
\label{eq:axiom-cond-ind}
\end{align}
\item 
\begin{align}
	\pr{A' | B} = \frac{\pr{A'B}}{\pr{B}} = \frac{\pr{B}-\pr{AB}}{\pr{B}}
\label{eq:axiom-cond-prime}
\end{align}
\item Total probability
	\begin{align}
		\pr{A} = \sum_{i=1}^{2}\pr{E_i}\pr{A|E_i}
\label{eq:axiom-tot-prob}
\end{align}
\item Bayes' Theorem
\begin{align}
	\pr{E_1|A} &= \frac{\pr{E_1}\pr{A|E_1}}{\sum_{i=1}^{2}\pr{E_i}\pr{A|E_i}}
\label{eq:axiom-bayes}
\end{align}
\item Let $X, Y \in \cbrak{0,1}$ be two random variables.  Then, 
\begin{align}
	\label{eq:12/13/3/43/yixj-def}
	  \pr{Y=1|X=0}\triangleq  p_{Y|X}\brak{1|0}
\end{align}
\item 
\begin{align}
	   p_{Y|X}\brak{1|0}
	   &=
	   \frac{p_{X,Y}\brak{0,1}}{p_{X}\brak{0}}
	   =
	   \frac{p_{X}\brak{0}-p_{X,Y}\brak{0,0}}{p_{X}\brak{0}}
	   \\
	   &=
	   1-\frac{p_{X,Y}\brak{0,0}}{p_{X}\brak{0}} = 1-p_{Y|X}\brak{0|0}
	\label{eq:12/13/3/43/yixj}
\end{align}
\end{enumerate}

\subsection{NCERT}
\begin{enumerate}[label=\thesection.\arabic*,ref=\thesection.\theenumi]
	\item An electronic assembley consists of two subsystems,say A and B.From previous testing procedures, the following probabilities are assumed to be known
\begin{align}
\pr{\text{A fails}}&=0.20
\\ \pr{\text{B alone fails}}&=0.15
\\ \pr{\text{A  and B fails}}&=0.15
\end{align}
 Evaluate the following probabilities
 \begin{enumerate}
 \item $\pr{\text{A fails given B has failed}}$
 \item $\pr{\text{A fails alone}}$
\end{enumerate}
		\solution
		From the  given information,
 \begin{align}
 \pr{A^\prime} = 0.20,\   	
\pr{AB^ \prime}=0.15,\ 	
\pr{A ^\prime B ^\prime}=0.15
 \end{align} 
 %
\begin{enumerate}
\item 
\begin{align}
 \pr{A ^\prime|B ^\prime}=\frac{\pr{ A ^\prime B ^\prime }}{\pr{B^\prime}} \label{eq:ncert/12/13/6/15/dot}
\end{align}
From 
\eqref{eq:axiom_sum_A},
\begin{align}
	\pr{B^ \prime}&=
          0.15 +0.15 
=0.30 
\\
	\pr{A^\prime|B ^\prime}&= 0.15/0.30 
 = 0.50
\end{align}
\item 
Similiarly, 
from 
\eqref{eq:axiom_sum_A},
\begin{align}
\pr{B  A ^\prime}&=\pr{A^\prime}-\pr{A ^\prime B ^\prime} 
 = 0.20-0.15  = 0.05
\end{align}
\end{enumerate}


  \item
  In a hostel, 60\% of the students read Hindi newspaper, 40\% read English
newspaper and 20\% read both Hindi and English newspapers. A student is
selected at random.
\begin{enumerate}
\item Find the probability that she reads neither Hindi nor English newspapers.
\item If she reads Hindi newspaper, find the probability that she reads English
newspaper.
\item If she reads English newspaper, find the probability that she reads Hindi
newspaper.
\end{enumerate}
\solution
%See \tabref{tab:variables-events}.
From the given information,
\begin{align}
\pr{A}=\frac{6}{10},\
\pr{B}=\frac{4}{10},\
\pr{AB} 
=\frac{2}{10}
\end{align}
\iffalse
\begin{table}[htb] % h stands for "here", suggesting to LaTeX to place the table near where it's defined
\centering
\begin{tabular}{|c|c|}
\hline
\textbf{Variable} & \textbf{Event} \\
\hline
A & Hindi \\
\hline
B & English \\
\hline
\end{tabular}
\caption{}
\label{tab:variables-events}
\end{table}
\fi
\begin{enumerate}
\item
\begin{align}
\pr{A^{\prime}B^{\prime}}&=\pr{\brak{A+B}^{\prime}}\\
&=1-\pr{A+B}\\
&=1-\brak{\pr{A}+\pr{B}-\pr{AB}}\\
&=1-\brak{\frac{6}{10}+\frac{4}{10}-\frac{2}{10}}
=\frac{2}{10}
\end{align}
\item
\begin{align}
\pr{B|A}&=\frac{\pr{AB}}{\pr{A}}
=\frac{\frac{2}{10}}{\frac{6}{10}}
=\frac{1}{3}
\end{align}
\item 
\begin{align}
\pr{A|B}&=\frac{\pr{AB}}{\pr{B}}
=\frac{\frac{2}{10}}{\frac{4}{10}}
=\frac{1}{2}
\end{align}
\end{enumerate}

	\item Assume that the chances of a patient having a heart attack is $40\%$. It is also assumed that a meditation and yoga course reduce the risk of heart attack by $30\%$ and prescription of certain drug reduces its chances by $25\%$. At a time a patient can choose any one of the two options with equal probabilities. It is given that after going through one of the two options the patient selected at random suffers a heart attack. Find the probability that the patient followed a course of meditation and yoga?
	\\
		\solution
		The given information is summarised in 
\tabref{tab:ncert/12/13/6/13/1}.
\begin{table}[htb]
	\centering
%%%%%%%%%%%%%%%%%%%%%%%%%%%%%%%%%%%%%%%%%%%%%%%%%%%%%%%%%%%%%%%%%%%%%%
%%                                                                  %%
%%  This is a LaTeX2e table fragment exported from Gnumeric.        %%
%%                                                                  %%
%%%%%%%%%%%%%%%%%%%%%%%%%%%%%%%%%%%%%%%%%%%%%%%%%%%%%%%%%%%%%%%%%%%%%%
\begin{tabular}{|c|l|l|}\hline
	\textbf{Variable}&\textbf{Description}&\textbf{Probability}\\\hline
A	&Person with heat attack	&\pr{A}=0.40\\\hline
$E_1$	&Person treated with meditation and yoga	&\pr{E_1}=0.50\\\hline
$E_2$	&Person treated with drug	&\pr{E_2}=0.50\\\hline
\end{tabular}

\caption{}
\label{tab:ncert/12/13/6/13/1}
\end{table}
%
\begin{align}
	\therefore	\pr{A|E_1} &= \pr{A}\cbrak{1-(0.30)}
	 =  0.28
	\label{eq:ncert/12/13/6/13/4}
	\\
	\pr{A|E_2} &= \pr{A}\cbrak{1-(0.25)}
	=  0.30
	\label{eq:ncert/12/13/6/13/5}
\end{align}
From 
%\eqref{eq:ncert/12/13/6/13/4} and \eqref{eq:ncert/12/13/6/13/5},
\eqref{eq:axiom-bayes},
\begin{align}
	\pr{E_1|A} &= \frac{\pr{E_1}\pr{A|E_1}}{\sum_{i=1}^{2}\pr{E_i}\pr{A|E_i}}
	= \frac{\frac{1}{2}\times 0.28}{\frac{1}{2}\times 0.28 + \frac{1}{2}\times 0.30}
	= \frac{14}{29}
\end{align}
which is the desired probability.

	\item A team of medical students doing their internship have to assist during surgeries
at a city hospital. The probabilities of surgeries rated as very-complex, complex,
routine, simple or very-simple are respectively, 0.15, 0.20, 0.31, 0.26, .08. Find
the probabilities that a particular surgery will be rated
\begin{enumerate}
\item complex or very-complex
\item neither very-complex nor very simple
\item routine or complex
\item routine or simple
\end{enumerate}
		\solution
		%The given information is summarised in 
\tabref{tab:exemplar/11/16/3/8/table2}
%
\begin{enumerate}
\item 
\begin{align} 
	\pr{E_1 + E_2} &= \pr{E_1} + \pr{E_2}   \qquad \because E_1E_2=0\\
	&= 0.15 + 0.20 
	= 0.35
\end{align}
%
\item 
\begin{align} 
	\pr{E_1^{\prime} E_5^{\prime}} &=  \pr{\brak{E_1 + E_5}^{\prime}}  \\
	&= 1- \pr{E_1 + E_5}\\
	&= 1- \sbrak{\pr{E_1} + \pr{E_5} }  \qquad \because E_1E_5=0\\
	&= 1- \sbrak{0.15 + 0.08 }
	= 0.77\\
\end{align}
%
\item 
\begin{align} 
	\pr{E_3 + E_2} &= \pr{E_3} + \pr{E_2}   \qquad \because E_3E_2=0\\
	&= 0.31 + 0.20 
	= 0.51
\end{align}
\item 
\begin{align} 
	\pr{E_3 + E_4} &= \pr{E_3} + \pr{E_4}   \qquad \because E_3E_4=0\\
	&= 0.31 + 0.26 
	= 0.57
\end{align}
\end{enumerate}




	\item One urn contains two black balls (labelled B1 and B2) and one white ball. A
	second urn contains one black ball and two white balls (labelled W1 and W2).
	Suppose the following experiment is performed. One of the two urns is chosen
	at random. Next a ball is randomly chosen from the urn. Then a second ball is
	chosen at random from the same urn without replacing the first ball.
	
	\begin{enumerate}
	\item What is the probability that two black balls are chosen?
	
	\item What is the probability that two balls of opposite colour are chosen?
	\end{enumerate}
	\solution
	%\begin{align}
    \label{eq:12.13.6.18.1}
	\because	\pr{A|B} &> \pr{A},\
\frac{\pr{AB}}{\pr{B}} > \pr{A}
\\
    \label{eq:12.13.6.18.2}
	\implies \pr{AB} &> \pr{A}\pr{B}
	\\
	\text{or, } \frac{\pr{AB}}{\pr{A}} &=\pr{B|A} > \pr{A}
\end{align}

  
\item Suppose an integer from 1 through 1000 is chosen at random, find the probability that the integer is a multiple of 2 or a multiple of 9.
%\begin{table}[htb]
        \centering
        \begin{table}[H]
	\centering
\begin{tabular}{|c|c|c|}
\hline
Random variable &Value &Definition\\ \hline
\multirow{3}{*}{X} &0 &Slips of Rs 1\\
&1 &Slips of Rs 5\\
&2 &Slips of Rs 13\\ \hline
\multirow{2}{*}{Y} &0 &Box A\\
&1 &Box B\\\hline
\end{tabular}
\caption{}
\label{tab:Distribution}
\end{table}
See \tabref{tab:Distribution}.
\begin{align}
p_{Y}\brak{k}= \begin{cases} 
      \frac{1}{3} & {k=0} \\
      \frac{2}{3 }& {k=1} 
   \end{cases}
   \\
p_{Y|X}\brak{0|0} = \frac{19}{25}\, 
p_{Y|X}\brak{0|1} = \frac{6}{25}\,
p_{Y|X}\brak{1|0} = \frac{45}{50}\,
p_{Y|X}\brak{1|2} = \frac{5}{50}
\end{align}
The desired probability is the probability that a slip drawn at random is marked other than Rs 1,
\begin{align}
&=1-p_X\brak{0}\\
&= p_X(1) + p_X(2)
\end{align}
Using Bayes theorem,
\begin{align}
&= p_Y\brak{0} \times \pr{Y=0 | X=1} + p_Y\brak{1} \times \pr{Y=1|X=2}\\
&=\frac{1}{3} \times \frac{6}{25} + \frac{2}{3} \times \frac{5}{50}\\
&=\frac{11}{75}
\end{align}

\newpage

%\tableofcontents

\bigskip

\renewcommand{\thefigure}{\theenumi}
\renewcommand{\thetable}{\theenumi}
%\renewcommand{\theequation}{\theenumi}

%\begin{abstract}
%%\boldmath
%In this letter, an algorithm for evaluating the exact analytical bit error rate  (BER)  for the piecewise linear (PL) combiner for  multiple relays is presented. Previous results were available only for upto three relays. The algorithm is unique in the sense that  the actual mathematical expressions, that are prohibitively large, need not be explicitly obtained. The diversity gain due to multiple relays is shown through plots of the analytical BER, well supported by simulations. 
%
%\end{abstract}
% IEEEtran.cls defaults to using nonbold math in the Abstract.
% This preserves the distinction between vectors and scalars. However,
% if the journal you are submitting to favors bold math in the abstract,
% then you can use LaTeX's standard command \boldmath at the very start
% of the abstract to achieve this. Many IEEE journals frown on math
% in the abstract anyway.

% Note that keywords are not normally used for peerreview papers.
%\begin{IEEEkeywords}
%Cooperative diversity, decode and forward, piecewise linear
%\end{IEEEkeywords}



% For peer review papers, you can put extra information on the cover
% page as needed:
% \ifCLASSOPTIONpeerreview
% \begin{center} \bfseries EDICS Category: 3-BBND \end{center}
% \fi
%
% For peerreview papers, this IEEEtran command inserts a page break and
% creates the second title. It will be ignored for other modes.
%\IEEEpeerreviewmaketitle




        \caption{}
        \label{tab:11.16.3.3}
\end{table}
See 
        \tabref{tab:11.16.3.3}.
From 
\eqref{eq:axiom_sum_AB},
\begin{align}
\pr{A + B} 
= \frac{500}{1000} + \frac{111}{1000} - \frac{55}{1000}
= \frac{556}{1000}
\end{align}



\item State whether the statement is True or False.\\
The probability that a person visiting a zoo will see the giraffe is 0.72, the probability that he will see the bears is 0.84 and the probability that he will see both is 0.52.\\
\solution
\\
%Let 
\begin{align}
\pr{A} = 0.72,\
\pr{B} = 0.84,\
\pr{AB} = 0.52.
\end{align}
Using
\eqref{eq:axiom_sum_AB},
\begin{align}
\pr{A+B} 
= 0.72 + 0.84 - 0.52
= 1.04
\end{align}
which violates
	\eqref{eq:axiom-pos}.
Hence, false. 











\newpage

%\tableofcontents

\bigskip

\renewcommand{\thefigure}{\theenumi}
\renewcommand{\thetable}{\theenumi}
%\renewcommand{\theequation}{\theenumi}

%\begin{abstract}
%%\boldmath
%In this letter, an algorithm for evaluating the exact analytical bit error rate  (BER)  for the piecewise linear (PL) combiner for  multiple relays is presented. Previous results were available only for upto three relays. The algorithm is unique in the sense that  the actual mathematical expressions, that are prohibitively large, need not be explicitly obtained. The diversity gain due to multiple relays is shown through plots of the analytical BER, well supported by simulations. 
%
%\end{abstract}
% IEEEtran.cls defaults to using nonbold math in the Abstract.
% This preserves the distinction between vectors and scalars. However,
% if the journal you are submitting to favors bold math in the abstract,
% then you can use LaTeX's standard command \boldmath at the very start
% of the abstract to achieve this. Many IEEE journals frown on math
% in the abstract anyway.

% Note that keywords are not normally used for peerreview papers.
%\begin{IEEEkeywords}
%Cooperative diversity, decode and forward, piecewise linear
%\end{IEEEkeywords}



% For peer review papers, you can put extra information on the cover
% page as needed:
% \ifCLASSOPTIONpeerreview
% \begin{center} \bfseries EDICS Category: 3-BBND \end{center}
% \fi
%
% For peerreview papers, this IEEEtran command inserts a page break and
% creates the second title. It will be ignored for other modes.
%\IEEEpeerreviewmaketitle



\item State whether the statement is True or False. The probabilities that a typist will make 0, 1, 2, 3, 4, 5 or more mistakes in typing a report are, respectively, 0.12, 0.25, 0.36, 0.14, 0.08, 0.11.\\
\solution
\\
%From the given information, we obtain the distribution
\begin{align}
p_X(k)&=
\begin{cases}
0.12 & k=0\\
0.25 & k=1\\
0.36 & k=2\\
0.14 & k=3\\
0.08 & k=4\\
0.11 & k\geq5\\
\end{cases}
\end{align}
Since
\begin{align}
\sum_{i=0}^5 p_X(k) = 1.06
&>1
\end{align}
	violates \eqref{eq:axiom-pos},
the given statement is false.



\item If A and B are two candidates seeking admission in an engineering College. The probability that A is selected is 0.5 and the probability that both A and B are selected is atmost 0.3. Is it possible that the probability of B getting selected is 0.7?\\
%\begin{align}
\because \pr{AB} &\leq 0.3
\\
	\text{Let }
\pr{AB} &= 0.1.
\end{align}
From \eqref{eq:axiom_sum_AB}
\begin{align}
\pr{A+B} = 0.5+0.7-0.1 = 1.1 > 1,
\end{align}
which violates
	\eqref{eq:axiom-pos}.  Hence, it is not possible.
%\end{document}



\item Let $E_1$ and $E_2$ be two independent events such that $\pr{E_1} = p_1 $ and $ \pr{E_2} = p_2 $ \ Describe in words the events whose probabilities are: 
\begin{enumerate}
\item \ $p_1 p_2$
\quad\item \ $(1 - p_1) p_2 $
\quad\item \ $1 - (1 - p_1)(1 - p_2)$
\quad\item \ $p_1 + p_2 - 2p_1 p_2$
\end{enumerate}
%\begin{enumerate}
\item \begin{align}
p_1p_2&=\pr{E_1}\pr{E_2}\\
&=\pr{E_1E_2}
\end{align}
So, $E_1$ and $E_2$ occur simultaneously.
\item \begin{align}
(1 - p_1)(p_2) &= \pr{E_1^\prime}\pr{E_2} \\
&= \pr{{E_1^\prime}E_2}
\end{align}
So $E_1$ does not occur but $E_2$  occurs.
\item \begin{align}
1 - (1 - p_1)(1 - p_2) &= 1 - \pr{E_1^\prime} \pr{E_2^\prime} \\
&= 1 - \pr{E_1^\prime \ E_2^\prime} \\
&= \pr{E_1+E_2} 
\end{align}
So, either $E_1$ or $E_2$ or both $E_1$ and $E_2$  occurs.
\item \begin{align}
p_1 + p_2 - 2p_1p_2 &= \pr{E_1} + \pr{E_2} - 2\pr{E_1} \pr{E_2} \\
&=  \pr{E_1}-\pr{E_1}\pr{E_2}+ \pr{E_2} - \pr{E_1} \pr{E_2}\\
&= \pr{E_1}\brak{1-\pr{E_2}}+\pr{E_2} \brak{1-\pr{E_1}}\\
&=\pr{E_1}\pr{E_2^\prime}+\pr{E_2}\pr{E_1^\prime}\\
&=\pr{E_1E_2^\prime+E_1^\prime E_2}
\end{align}
So, either $E_1$ or $E_2$ occurs but not both.
\end{enumerate}




\item Suppose that 6\% of the people with blood group O are left handed and 10\% of those with other blood groups are left handed 30\% of the people have blood group O. If a left handed person is selected at random, what is the probability that he/she will have blood group O?
%Let $A$ represent blood group O and $B$ represent left handedness. 
From the given information,
\begin{align}
\pr{A} = 0.3,\
\pr{B|A} = 0.06,\
\pr{B| A^{\prime}} = 0.1.
\end{align}
Using 
\eqref{eq:axiom-bayes},
\begin{align}
	\pr{A|B} &= \frac{\pr{A}\pr{B|A}}{\pr{A}\pr{B|A}+\pr{A^\prime}\pr{B|A^\prime}}
	 = \frac{9}{44}
\end{align}
upon substituting numerical values.

\item %\solution
\begin{enumerate}
	\item If $E_1$ and $E_2$ are mutually exclusive events, then $E_1E_2 = \phi$.\\
	\item If $E_1$ and $E_2$ are mutually exclusive and exhaustive events, then $ E_1 E_2 = \phi $ and $ E_1 + E_2=S $ \\
	\item If $E_1$ and $E_2$ have common outcomes, this means:
\begin{align}
	E_1E_2&\neq0
\end{align}
Let $E_a$ be the outcomes that are present in $E_1$ and not in $E_2$. So,
\begin{align}
	E_a&=E_1-E_2 \label{eq:exemplar/11/16/3/43/1}
\end{align}
Let $E_b$ be the outcomes common between $E_1$ and $E_2$. So,
\begin{align}
	E_b&=E_1E_2 \label{eq:exemplar/11/16/3/43/2}
\end{align}
So, we can say that
\begin{align}
	E_1&=E_a+E_b
\end{align}
Refering to equation \eqref{eq:exemplar/11/16/3/43/1} and \eqref{eq:exemplar/11/16/3/43/2}:
\begin{align}
	E_1&=(E_1-E_2)+(E_1 E_2)
\end{align}
        \item If $E_1$ and $E_2$ are two events such that $E_1 \subset E_2$, then let E be subset of $E_2$ containing elements other than $E_1$. So,
\begin{align}
	E_1 + E &= E_2 \text{  and  } E_1 E = E_2\label{eq:exemplar/11/16/3/43/3}
\end{align}
Refering to equation \eqref{eq:exemplar/11/16/3/43/3}:
\begin{align}
	E_1 E_2 &= E_1 (E_1+E)\\
	&=(E_1 E_1)+(E_1 E)\\
	&=E_1
\end{align}
\end{enumerate}
	Hence,
\begin{multicols}{4}
\begin{enumerate}
	\item $\leftrightarrow(iv)$,\item $\leftrightarrow(iii)$,\item $\leftrightarrow(ii)$,\item $\leftrightarrow(i)$
\end{enumerate}
\end{multicols}




\item Two Coins are tossed once, where\\
\brak{i} E : Tail appears on one coin,\qquad F : one coin shows head\\
\brak{ii}  E : no tail appears,\qquad\qquad\qquad F : no head appears
Determine \pr{E\mid F}.
%\input{ncert/12/13/1/7a/ncertmain.tex}
\item A fair die is rolled. Consider events $E=1,3,5,\, F=2,3$ and $G=2,3,4,5$. Find
\begin{enumerate}
\item $\pr{E \mid F} \text{ and } \pr{F \mid E}$
\item $\pr{E \mid G} \text{ and } \pr{G \mid E}$
\item $\pr{{E \cup F} \mid G} \text{ and } \pr{{E \cap F} \mid G}$
\end{enumerate}
%
\solution
%		\begin{table}[htb]
		\centering
		\input{ncert/12/13/1/11/tables/Table1.tex}
		\caption{}
		\label{tab:ncert/12/13/1/11/table1}
		\end{table}
	See 	
		\tabref{tab:ncert/12/13/1/11/table1}.
		\begin{enumerate}
		\item
		\begin{align}
			\pr{\cond{E}{F}} = \frac{\pr{EF}}{\pr{F}}
		 = \frac{1/6}{1/3}
		 = 1/2
		\end{align}
		\item
		\begin{align}
		\pr{F|E} = \frac{\pr{EF}}{\pr{E}}
         = \frac{1/6}{1/2}
         = 1/3
        \end{align}
		\item
		\begin{align}
			\pr{E|G} = \frac{\pr{EG}}{\pr{G}}
        = \frac{1/3}{2/3}
        = 1/2
        \end{align}
        \item
        \begin{align}
		\pr{G|E} = \frac{\pr{EG}}{\pr{E}}
        = \frac{1/3}{1/2}
        = 2/3
        \end{align}
        \item
        \begin{align}
		\because	\pr{(E+F)G} &= \pr{EG+FG}= \pr{EG}+\pr{FG}-\pr{EFG},
	\nonumber	\\
		 &= \frac{1}{3} + \frac{1}{3} - \frac{1}{6}
		 = \frac{1}{2}
		 \\
		\pr{(E+F)|G} &= \frac{\pr{(E+F)G}}{\pr{G}}
		 = \frac{1/2}{2/3}
		 = \frac{3}{4}
		\end{align}
		\item
		\begin{align}
			\pr{EF|G} &= \frac{\pr{EFG}}{\pr{G}}
         = \frac{1/6}{2/3}
			= \frac{1}{4}
		\end{align}

\end{enumerate}

\end{enumerate}

\begin{enumerate}[label=\thesection.\arabic*,ref=\thesection.\theenumi]
	\item An electronic assembley consists of two subsystems,say A and B.From previous testing procedures, the following probabilities are assumed to be known
\begin{align}
\pr{\text{A fails}}&=0.20
\\ \pr{\text{B alone fails}}&=0.15
\\ \pr{\text{A  and B fails}}&=0.15
\end{align}
 Evaluate the following probabilities
 \begin{enumerate}
 \item $\pr{\text{A fails given B has failed}}$
 \item $\pr{\text{A fails alone}}$
\end{enumerate}
		\solution
		From the  given information,
 \begin{align}
 \pr{A^\prime} = 0.20,\   	
\pr{AB^ \prime}=0.15,\ 	
\pr{A ^\prime B ^\prime}=0.15
 \end{align} 
 %
\begin{enumerate}
\item 
\begin{align}
 \pr{A ^\prime|B ^\prime}=\frac{\pr{ A ^\prime B ^\prime }}{\pr{B^\prime}} \label{eq:ncert/12/13/6/15/dot}
\end{align}
From 
\eqref{eq:axiom_sum_A},
\begin{align}
	\pr{B^ \prime}&=
          0.15 +0.15 
=0.30 
\\
	\pr{A^\prime|B ^\prime}&= 0.15/0.30 
 = 0.50
\end{align}
\item 
Similiarly, 
from 
\eqref{eq:axiom_sum_A},
\begin{align}
\pr{B  A ^\prime}&=\pr{A^\prime}-\pr{A ^\prime B ^\prime} 
 = 0.20-0.15  = 0.05
\end{align}
\end{enumerate}


  \item
  In a hostel, 60\% of the students read Hindi newspaper, 40\% read English
newspaper and 20\% read both Hindi and English newspapers. A student is
selected at random.
\begin{enumerate}
\item Find the probability that she reads neither Hindi nor English newspapers.
\item If she reads Hindi newspaper, find the probability that she reads English
newspaper.
\item If she reads English newspaper, find the probability that she reads Hindi
newspaper.
\end{enumerate}
\solution
%See \tabref{tab:variables-events}.
From the given information,
\begin{align}
\pr{A}=\frac{6}{10},\
\pr{B}=\frac{4}{10},\
\pr{AB} 
=\frac{2}{10}
\end{align}
\iffalse
\begin{table}[htb] % h stands for "here", suggesting to LaTeX to place the table near where it's defined
\centering
\begin{tabular}{|c|c|}
\hline
\textbf{Variable} & \textbf{Event} \\
\hline
A & Hindi \\
\hline
B & English \\
\hline
\end{tabular}
\caption{}
\label{tab:variables-events}
\end{table}
\fi
\begin{enumerate}
\item
\begin{align}
\pr{A^{\prime}B^{\prime}}&=\pr{\brak{A+B}^{\prime}}\\
&=1-\pr{A+B}\\
&=1-\brak{\pr{A}+\pr{B}-\pr{AB}}\\
&=1-\brak{\frac{6}{10}+\frac{4}{10}-\frac{2}{10}}
=\frac{2}{10}
\end{align}
\item
\begin{align}
\pr{B|A}&=\frac{\pr{AB}}{\pr{A}}
=\frac{\frac{2}{10}}{\frac{6}{10}}
=\frac{1}{3}
\end{align}
\item 
\begin{align}
\pr{A|B}&=\frac{\pr{AB}}{\pr{B}}
=\frac{\frac{2}{10}}{\frac{4}{10}}
=\frac{1}{2}
\end{align}
\end{enumerate}

	\item Assume that the chances of a patient having a heart attack is $40\%$. It is also assumed that a meditation and yoga course reduce the risk of heart attack by $30\%$ and prescription of certain drug reduces its chances by $25\%$. At a time a patient can choose any one of the two options with equal probabilities. It is given that after going through one of the two options the patient selected at random suffers a heart attack. Find the probability that the patient followed a course of meditation and yoga?
	\\
		\solution
		The given information is summarised in 
\tabref{tab:ncert/12/13/6/13/1}.
\begin{table}[htb]
	\centering
%%%%%%%%%%%%%%%%%%%%%%%%%%%%%%%%%%%%%%%%%%%%%%%%%%%%%%%%%%%%%%%%%%%%%%
%%                                                                  %%
%%  This is a LaTeX2e table fragment exported from Gnumeric.        %%
%%                                                                  %%
%%%%%%%%%%%%%%%%%%%%%%%%%%%%%%%%%%%%%%%%%%%%%%%%%%%%%%%%%%%%%%%%%%%%%%
\begin{tabular}{|c|l|l|}\hline
	\textbf{Variable}&\textbf{Description}&\textbf{Probability}\\\hline
A	&Person with heat attack	&\pr{A}=0.40\\\hline
$E_1$	&Person treated with meditation and yoga	&\pr{E_1}=0.50\\\hline
$E_2$	&Person treated with drug	&\pr{E_2}=0.50\\\hline
\end{tabular}

\caption{}
\label{tab:ncert/12/13/6/13/1}
\end{table}
%
\begin{align}
	\therefore	\pr{A|E_1} &= \pr{A}\cbrak{1-(0.30)}
	 =  0.28
	\label{eq:ncert/12/13/6/13/4}
	\\
	\pr{A|E_2} &= \pr{A}\cbrak{1-(0.25)}
	=  0.30
	\label{eq:ncert/12/13/6/13/5}
\end{align}
From 
%\eqref{eq:ncert/12/13/6/13/4} and \eqref{eq:ncert/12/13/6/13/5},
\eqref{eq:axiom-bayes},
\begin{align}
	\pr{E_1|A} &= \frac{\pr{E_1}\pr{A|E_1}}{\sum_{i=1}^{2}\pr{E_i}\pr{A|E_i}}
	= \frac{\frac{1}{2}\times 0.28}{\frac{1}{2}\times 0.28 + \frac{1}{2}\times 0.30}
	= \frac{14}{29}
\end{align}
which is the desired probability.

	\item A team of medical students doing their internship have to assist during surgeries
at a city hospital. The probabilities of surgeries rated as very-complex, complex,
routine, simple or very-simple are respectively, 0.15, 0.20, 0.31, 0.26, .08. Find
the probabilities that a particular surgery will be rated
\begin{enumerate}
\item complex or very-complex
\item neither very-complex nor very simple
\item routine or complex
\item routine or simple
\end{enumerate}
		\solution
		%The given information is summarised in 
\tabref{tab:exemplar/11/16/3/8/table2}
%
\begin{enumerate}
\item 
\begin{align} 
	\pr{E_1 + E_2} &= \pr{E_1} + \pr{E_2}   \qquad \because E_1E_2=0\\
	&= 0.15 + 0.20 
	= 0.35
\end{align}
%
\item 
\begin{align} 
	\pr{E_1^{\prime} E_5^{\prime}} &=  \pr{\brak{E_1 + E_5}^{\prime}}  \\
	&= 1- \pr{E_1 + E_5}\\
	&= 1- \sbrak{\pr{E_1} + \pr{E_5} }  \qquad \because E_1E_5=0\\
	&= 1- \sbrak{0.15 + 0.08 }
	= 0.77\\
\end{align}
%
\item 
\begin{align} 
	\pr{E_3 + E_2} &= \pr{E_3} + \pr{E_2}   \qquad \because E_3E_2=0\\
	&= 0.31 + 0.20 
	= 0.51
\end{align}
\item 
\begin{align} 
	\pr{E_3 + E_4} &= \pr{E_3} + \pr{E_4}   \qquad \because E_3E_4=0\\
	&= 0.31 + 0.26 
	= 0.57
\end{align}
\end{enumerate}




	\item One urn contains two black balls (labelled B1 and B2) and one white ball. A
	second urn contains one black ball and two white balls (labelled W1 and W2).
	Suppose the following experiment is performed. One of the two urns is chosen
	at random. Next a ball is randomly chosen from the urn. Then a second ball is
	chosen at random from the same urn without replacing the first ball.
	
	\begin{enumerate}
	\item What is the probability that two black balls are chosen?
	
	\item What is the probability that two balls of opposite colour are chosen?
	\end{enumerate}
	\solution
	%\begin{align}
    \label{eq:12.13.6.18.1}
	\because	\pr{A|B} &> \pr{A},\
\frac{\pr{AB}}{\pr{B}} > \pr{A}
\\
    \label{eq:12.13.6.18.2}
	\implies \pr{AB} &> \pr{A}\pr{B}
	\\
	\text{or, } \frac{\pr{AB}}{\pr{A}} &=\pr{B|A} > \pr{A}
\end{align}

  
\item Suppose an integer from 1 through 1000 is chosen at random, find the probability that the integer is a multiple of 2 or a multiple of 9.
%\begin{table}[htb]
        \centering
        \begin{table}[H]
	\centering
\begin{tabular}{|c|c|c|}
\hline
Random variable &Value &Definition\\ \hline
\multirow{3}{*}{X} &0 &Slips of Rs 1\\
&1 &Slips of Rs 5\\
&2 &Slips of Rs 13\\ \hline
\multirow{2}{*}{Y} &0 &Box A\\
&1 &Box B\\\hline
\end{tabular}
\caption{}
\label{tab:Distribution}
\end{table}
See \tabref{tab:Distribution}.
\begin{align}
p_{Y}\brak{k}= \begin{cases} 
      \frac{1}{3} & {k=0} \\
      \frac{2}{3 }& {k=1} 
   \end{cases}
   \\
p_{Y|X}\brak{0|0} = \frac{19}{25}\, 
p_{Y|X}\brak{0|1} = \frac{6}{25}\,
p_{Y|X}\brak{1|0} = \frac{45}{50}\,
p_{Y|X}\brak{1|2} = \frac{5}{50}
\end{align}
The desired probability is the probability that a slip drawn at random is marked other than Rs 1,
\begin{align}
&=1-p_X\brak{0}\\
&= p_X(1) + p_X(2)
\end{align}
Using Bayes theorem,
\begin{align}
&= p_Y\brak{0} \times \pr{Y=0 | X=1} + p_Y\brak{1} \times \pr{Y=1|X=2}\\
&=\frac{1}{3} \times \frac{6}{25} + \frac{2}{3} \times \frac{5}{50}\\
&=\frac{11}{75}
\end{align}

\newpage

%\tableofcontents

\bigskip

\renewcommand{\thefigure}{\theenumi}
\renewcommand{\thetable}{\theenumi}
%\renewcommand{\theequation}{\theenumi}

%\begin{abstract}
%%\boldmath
%In this letter, an algorithm for evaluating the exact analytical bit error rate  (BER)  for the piecewise linear (PL) combiner for  multiple relays is presented. Previous results were available only for upto three relays. The algorithm is unique in the sense that  the actual mathematical expressions, that are prohibitively large, need not be explicitly obtained. The diversity gain due to multiple relays is shown through plots of the analytical BER, well supported by simulations. 
%
%\end{abstract}
% IEEEtran.cls defaults to using nonbold math in the Abstract.
% This preserves the distinction between vectors and scalars. However,
% if the journal you are submitting to favors bold math in the abstract,
% then you can use LaTeX's standard command \boldmath at the very start
% of the abstract to achieve this. Many IEEE journals frown on math
% in the abstract anyway.

% Note that keywords are not normally used for peerreview papers.
%\begin{IEEEkeywords}
%Cooperative diversity, decode and forward, piecewise linear
%\end{IEEEkeywords}



% For peer review papers, you can put extra information on the cover
% page as needed:
% \ifCLASSOPTIONpeerreview
% \begin{center} \bfseries EDICS Category: 3-BBND \end{center}
% \fi
%
% For peerreview papers, this IEEEtran command inserts a page break and
% creates the second title. It will be ignored for other modes.
%\IEEEpeerreviewmaketitle




        \caption{}
        \label{tab:11.16.3.3}
\end{table}
See 
        \tabref{tab:11.16.3.3}.
From 
\eqref{eq:axiom_sum_AB},
\begin{align}
\pr{A + B} 
= \frac{500}{1000} + \frac{111}{1000} - \frac{55}{1000}
= \frac{556}{1000}
\end{align}



\item State whether the statement is True or False.\\
The probability that a person visiting a zoo will see the giraffe is 0.72, the probability that he will see the bears is 0.84 and the probability that he will see both is 0.52.\\
\solution
\\
%Let 
\begin{align}
\pr{A} = 0.72,\
\pr{B} = 0.84,\
\pr{AB} = 0.52.
\end{align}
Using
\eqref{eq:axiom_sum_AB},
\begin{align}
\pr{A+B} 
= 0.72 + 0.84 - 0.52
= 1.04
\end{align}
which violates
	\eqref{eq:axiom-pos}.
Hence, false. 











\newpage

%\tableofcontents

\bigskip

\renewcommand{\thefigure}{\theenumi}
\renewcommand{\thetable}{\theenumi}
%\renewcommand{\theequation}{\theenumi}

%\begin{abstract}
%%\boldmath
%In this letter, an algorithm for evaluating the exact analytical bit error rate  (BER)  for the piecewise linear (PL) combiner for  multiple relays is presented. Previous results were available only for upto three relays. The algorithm is unique in the sense that  the actual mathematical expressions, that are prohibitively large, need not be explicitly obtained. The diversity gain due to multiple relays is shown through plots of the analytical BER, well supported by simulations. 
%
%\end{abstract}
% IEEEtran.cls defaults to using nonbold math in the Abstract.
% This preserves the distinction between vectors and scalars. However,
% if the journal you are submitting to favors bold math in the abstract,
% then you can use LaTeX's standard command \boldmath at the very start
% of the abstract to achieve this. Many IEEE journals frown on math
% in the abstract anyway.

% Note that keywords are not normally used for peerreview papers.
%\begin{IEEEkeywords}
%Cooperative diversity, decode and forward, piecewise linear
%\end{IEEEkeywords}



% For peer review papers, you can put extra information on the cover
% page as needed:
% \ifCLASSOPTIONpeerreview
% \begin{center} \bfseries EDICS Category: 3-BBND \end{center}
% \fi
%
% For peerreview papers, this IEEEtran command inserts a page break and
% creates the second title. It will be ignored for other modes.
%\IEEEpeerreviewmaketitle



\item State whether the statement is True or False. The probabilities that a typist will make 0, 1, 2, 3, 4, 5 or more mistakes in typing a report are, respectively, 0.12, 0.25, 0.36, 0.14, 0.08, 0.11.\\
\solution
\\
%From the given information, we obtain the distribution
\begin{align}
p_X(k)&=
\begin{cases}
0.12 & k=0\\
0.25 & k=1\\
0.36 & k=2\\
0.14 & k=3\\
0.08 & k=4\\
0.11 & k\geq5\\
\end{cases}
\end{align}
Since
\begin{align}
\sum_{i=0}^5 p_X(k) = 1.06
&>1
\end{align}
	violates \eqref{eq:axiom-pos},
the given statement is false.



\item If A and B are two candidates seeking admission in an engineering College. The probability that A is selected is 0.5 and the probability that both A and B are selected is atmost 0.3. Is it possible that the probability of B getting selected is 0.7?\\
%\begin{align}
\because \pr{AB} &\leq 0.3
\\
	\text{Let }
\pr{AB} &= 0.1.
\end{align}
From \eqref{eq:axiom_sum_AB}
\begin{align}
\pr{A+B} = 0.5+0.7-0.1 = 1.1 > 1,
\end{align}
which violates
	\eqref{eq:axiom-pos}.  Hence, it is not possible.
%\end{document}



\item Let $E_1$ and $E_2$ be two independent events such that $\pr{E_1} = p_1 $ and $ \pr{E_2} = p_2 $ \ Describe in words the events whose probabilities are: 
\begin{enumerate}
\item \ $p_1 p_2$
\quad\item \ $(1 - p_1) p_2 $
\quad\item \ $1 - (1 - p_1)(1 - p_2)$
\quad\item \ $p_1 + p_2 - 2p_1 p_2$
\end{enumerate}
%\begin{enumerate}
\item \begin{align}
p_1p_2&=\pr{E_1}\pr{E_2}\\
&=\pr{E_1E_2}
\end{align}
So, $E_1$ and $E_2$ occur simultaneously.
\item \begin{align}
(1 - p_1)(p_2) &= \pr{E_1^\prime}\pr{E_2} \\
&= \pr{{E_1^\prime}E_2}
\end{align}
So $E_1$ does not occur but $E_2$  occurs.
\item \begin{align}
1 - (1 - p_1)(1 - p_2) &= 1 - \pr{E_1^\prime} \pr{E_2^\prime} \\
&= 1 - \pr{E_1^\prime \ E_2^\prime} \\
&= \pr{E_1+E_2} 
\end{align}
So, either $E_1$ or $E_2$ or both $E_1$ and $E_2$  occurs.
\item \begin{align}
p_1 + p_2 - 2p_1p_2 &= \pr{E_1} + \pr{E_2} - 2\pr{E_1} \pr{E_2} \\
&=  \pr{E_1}-\pr{E_1}\pr{E_2}+ \pr{E_2} - \pr{E_1} \pr{E_2}\\
&= \pr{E_1}\brak{1-\pr{E_2}}+\pr{E_2} \brak{1-\pr{E_1}}\\
&=\pr{E_1}\pr{E_2^\prime}+\pr{E_2}\pr{E_1^\prime}\\
&=\pr{E_1E_2^\prime+E_1^\prime E_2}
\end{align}
So, either $E_1$ or $E_2$ occurs but not both.
\end{enumerate}




\item Suppose that 6\% of the people with blood group O are left handed and 10\% of those with other blood groups are left handed 30\% of the people have blood group O. If a left handed person is selected at random, what is the probability that he/she will have blood group O?
%Let $A$ represent blood group O and $B$ represent left handedness. 
From the given information,
\begin{align}
\pr{A} = 0.3,\
\pr{B|A} = 0.06,\
\pr{B| A^{\prime}} = 0.1.
\end{align}
Using 
\eqref{eq:axiom-bayes},
\begin{align}
	\pr{A|B} &= \frac{\pr{A}\pr{B|A}}{\pr{A}\pr{B|A}+\pr{A^\prime}\pr{B|A^\prime}}
	 = \frac{9}{44}
\end{align}
upon substituting numerical values.

\item %\solution
\begin{enumerate}
	\item If $E_1$ and $E_2$ are mutually exclusive events, then $E_1E_2 = \phi$.\\
	\item If $E_1$ and $E_2$ are mutually exclusive and exhaustive events, then $ E_1 E_2 = \phi $ and $ E_1 + E_2=S $ \\
	\item If $E_1$ and $E_2$ have common outcomes, this means:
\begin{align}
	E_1E_2&\neq0
\end{align}
Let $E_a$ be the outcomes that are present in $E_1$ and not in $E_2$. So,
\begin{align}
	E_a&=E_1-E_2 \label{eq:exemplar/11/16/3/43/1}
\end{align}
Let $E_b$ be the outcomes common between $E_1$ and $E_2$. So,
\begin{align}
	E_b&=E_1E_2 \label{eq:exemplar/11/16/3/43/2}
\end{align}
So, we can say that
\begin{align}
	E_1&=E_a+E_b
\end{align}
Refering to equation \eqref{eq:exemplar/11/16/3/43/1} and \eqref{eq:exemplar/11/16/3/43/2}:
\begin{align}
	E_1&=(E_1-E_2)+(E_1 E_2)
\end{align}
        \item If $E_1$ and $E_2$ are two events such that $E_1 \subset E_2$, then let E be subset of $E_2$ containing elements other than $E_1$. So,
\begin{align}
	E_1 + E &= E_2 \text{  and  } E_1 E = E_2\label{eq:exemplar/11/16/3/43/3}
\end{align}
Refering to equation \eqref{eq:exemplar/11/16/3/43/3}:
\begin{align}
	E_1 E_2 &= E_1 (E_1+E)\\
	&=(E_1 E_1)+(E_1 E)\\
	&=E_1
\end{align}
\end{enumerate}
	Hence,
\begin{multicols}{4}
\begin{enumerate}
	\item $\leftrightarrow(iv)$,\item $\leftrightarrow(iii)$,\item $\leftrightarrow(ii)$,\item $\leftrightarrow(i)$
\end{enumerate}
\end{multicols}




\item Two Coins are tossed once, where\\
\brak{i} E : Tail appears on one coin,\qquad F : one coin shows head\\
\brak{ii}  E : no tail appears,\qquad\qquad\qquad F : no head appears
Determine \pr{E\mid F}.
%\input{ncert/12/13/1/7a/ncertmain.tex}
\item A fair die is rolled. Consider events $E=1,3,5,\, F=2,3$ and $G=2,3,4,5$. Find
\begin{enumerate}
\item $\pr{E \mid F} \text{ and } \pr{F \mid E}$
\item $\pr{E \mid G} \text{ and } \pr{G \mid E}$
\item $\pr{{E \cup F} \mid G} \text{ and } \pr{{E \cap F} \mid G}$
\end{enumerate}
%
\solution
%		\begin{table}[htb]
		\centering
		\input{ncert/12/13/1/11/tables/Table1.tex}
		\caption{}
		\label{tab:ncert/12/13/1/11/table1}
		\end{table}
	See 	
		\tabref{tab:ncert/12/13/1/11/table1}.
		\begin{enumerate}
		\item
		\begin{align}
			\pr{\cond{E}{F}} = \frac{\pr{EF}}{\pr{F}}
		 = \frac{1/6}{1/3}
		 = 1/2
		\end{align}
		\item
		\begin{align}
		\pr{F|E} = \frac{\pr{EF}}{\pr{E}}
         = \frac{1/6}{1/2}
         = 1/3
        \end{align}
		\item
		\begin{align}
			\pr{E|G} = \frac{\pr{EG}}{\pr{G}}
        = \frac{1/3}{2/3}
        = 1/2
        \end{align}
        \item
        \begin{align}
		\pr{G|E} = \frac{\pr{EG}}{\pr{E}}
        = \frac{1/3}{1/2}
        = 2/3
        \end{align}
        \item
        \begin{align}
		\because	\pr{(E+F)G} &= \pr{EG+FG}= \pr{EG}+\pr{FG}-\pr{EFG},
	\nonumber	\\
		 &= \frac{1}{3} + \frac{1}{3} - \frac{1}{6}
		 = \frac{1}{2}
		 \\
		\pr{(E+F)|G} &= \frac{\pr{(E+F)G}}{\pr{G}}
		 = \frac{1/2}{2/3}
		 = \frac{3}{4}
		\end{align}
		\item
		\begin{align}
			\pr{EF|G} &= \frac{\pr{EFG}}{\pr{G}}
         = \frac{1/6}{2/3}
			= \frac{1}{4}
		\end{align}

\end{enumerate}

\end{enumerate}

\begin{enumerate}[label=\thesubsection.\arabic*,ref=\thesubsection.\theenumi]
	\item An electronic assembly consists of two subsystems, say $A$ and $B$. From previous testing procedures, the following probabilities are assumed to be known
\begin{align}
\pr{A\text{ fails}}&=0.20
\\ \pr{B\text{ alone fails}}&=0.15
	\\ \pr{A\text{ and }B \text{ fails}}&=0.15
\end{align}
 Evaluate the following probabilities
 \begin{enumerate}
 \item $\pr{\text{A fails given B has failed}}$
 \item $\pr{\text{A fails alone}}$
\end{enumerate}
		\solution
		From the  given information,
 \begin{align}
 \pr{A^\prime} = 0.20,\   	
\pr{AB^ \prime}=0.15,\ 	
\pr{A ^\prime B ^\prime}=0.15
 \end{align} 
 %
\begin{enumerate}
\item 
\begin{align}
 \pr{A ^\prime|B ^\prime}=\frac{\pr{ A ^\prime B ^\prime }}{\pr{B^\prime}} \label{eq:ncert/12/13/6/15/dot}
\end{align}
From 
\eqref{eq:axiom_sum_A},
\begin{align}
	\pr{B^ \prime}&=
          0.15 +0.15 
=0.30 
\\
	\pr{A^\prime|B ^\prime}&= 0.15/0.30 
 = 0.50
\end{align}
\item 
Similiarly, 
from 
\eqref{eq:axiom_sum_A},
\begin{align}
\pr{B  A ^\prime}&=\pr{A^\prime}-\pr{A ^\prime B ^\prime} 
 = 0.20-0.15  = 0.05
\end{align}
\end{enumerate}


  \item
  In a hostel, 60\% of the students read Hindi newspaper, 40\% read English
newspaper and 20\% read both Hindi and English newspapers. A student is
selected at random.
\begin{enumerate}
\item Find the probability that she reads neither Hindi nor English newspapers.
\item If she reads Hindi newspaper, find the probability that she reads English
newspaper.
\item If she reads English newspaper, find the probability that she reads Hindi
newspaper.
\end{enumerate}
\solution
%See \tabref{tab:variables-events}.
From the given information,
\begin{align}
\pr{A}=\frac{6}{10},\
\pr{B}=\frac{4}{10},\
\pr{AB} 
=\frac{2}{10}
\end{align}
\iffalse
\begin{table}[htb] % h stands for "here", suggesting to LaTeX to place the table near where it's defined
\centering
\begin{tabular}{|c|c|}
\hline
\textbf{Variable} & \textbf{Event} \\
\hline
A & Hindi \\
\hline
B & English \\
\hline
\end{tabular}
\caption{}
\label{tab:variables-events}
\end{table}
\fi
\begin{enumerate}
\item
\begin{align}
\pr{A^{\prime}B^{\prime}}&=\pr{\brak{A+B}^{\prime}}\\
&=1-\pr{A+B}\\
&=1-\brak{\pr{A}+\pr{B}-\pr{AB}}\\
&=1-\brak{\frac{6}{10}+\frac{4}{10}-\frac{2}{10}}
=\frac{2}{10}
\end{align}
\item
\begin{align}
\pr{B|A}&=\frac{\pr{AB}}{\pr{A}}
=\frac{\frac{2}{10}}{\frac{6}{10}}
=\frac{1}{3}
\end{align}
\item 
\begin{align}
\pr{A|B}&=\frac{\pr{AB}}{\pr{B}}
=\frac{\frac{2}{10}}{\frac{4}{10}}
=\frac{1}{2}
\end{align}
\end{enumerate}

	\item Assume that the chances of a patient having a heart attack is $40\%$. It is also assumed that a meditation and yoga course reduce the risk of heart attack by $30\%$ and prescription of certain drug reduces its chances by $25\%$. At a time a patient can choose any one of the two options with equal probabilities. It is given that after going through one of the two options the patient selected at random suffers a heart attack. Find the probability that the patient followed a course of meditation and yoga.
	\\
		\solution
		The given information is summarised in 
\tabref{tab:ncert/12/13/6/13/1}.
\begin{table}[htb]
	\centering
%%%%%%%%%%%%%%%%%%%%%%%%%%%%%%%%%%%%%%%%%%%%%%%%%%%%%%%%%%%%%%%%%%%%%%
%%                                                                  %%
%%  This is a LaTeX2e table fragment exported from Gnumeric.        %%
%%                                                                  %%
%%%%%%%%%%%%%%%%%%%%%%%%%%%%%%%%%%%%%%%%%%%%%%%%%%%%%%%%%%%%%%%%%%%%%%
\begin{tabular}{|c|l|l|}\hline
	\textbf{Variable}&\textbf{Description}&\textbf{Probability}\\\hline
A	&Person with heat attack	&\pr{A}=0.40\\\hline
$E_1$	&Person treated with meditation and yoga	&\pr{E_1}=0.50\\\hline
$E_2$	&Person treated with drug	&\pr{E_2}=0.50\\\hline
\end{tabular}

\caption{}
\label{tab:ncert/12/13/6/13/1}
\end{table}
%
\begin{align}
	\therefore	\pr{A|E_1} &= \pr{A}\cbrak{1-(0.30)}
	 =  0.28
	\label{eq:ncert/12/13/6/13/4}
	\\
	\pr{A|E_2} &= \pr{A}\cbrak{1-(0.25)}
	=  0.30
	\label{eq:ncert/12/13/6/13/5}
\end{align}
From 
%\eqref{eq:ncert/12/13/6/13/4} and \eqref{eq:ncert/12/13/6/13/5},
\eqref{eq:axiom-bayes},
\begin{align}
	\pr{E_1|A} &= \frac{\pr{E_1}\pr{A|E_1}}{\sum_{i=1}^{2}\pr{E_i}\pr{A|E_i}}
	= \frac{\frac{1}{2}\times 0.28}{\frac{1}{2}\times 0.28 + \frac{1}{2}\times 0.30}
	= \frac{14}{29}
\end{align}
which is the desired probability.

\item Suppose that 6\% of the people with blood group O are left handed and 10\% of those with other blood groups are left handed. 30\% of the people have blood group O. If a left handed person is selected at random, what is the probability that he/she will have blood group O?
\\
\solution 
Let $A$ represent blood group O and $B$ represent left handedness. 
From the given information,
\begin{align}
\pr{A} = 0.3,\
\pr{B|A} = 0.06,\
\pr{B| A^{\prime}} = 0.1.
\end{align}
Using 
\eqref{eq:axiom-bayes},
\begin{align}
	\pr{A|B} &= \frac{\pr{A}\pr{B|A}}{\pr{A}\pr{B|A}+\pr{A^\prime}\pr{B|A^\prime}}
	 = \frac{9}{44}
\end{align}
upon substituting numerical values.

\item At a fete, cards bearing numbers 1 to 1000, one number on a card, are put in a box. Each player selects one card at random and that card is not replaced. If the selected card has a perfect square greater than 500, the player wins a prize. What is the probability that 
\begin{enumerate}
\item the first player wins a prize
\item the second player wins a prize, if the first has won?
\end{enumerate}
\solution
\iffalse
Let
\begin{table}[H]
	\label{Defining variables}
	\input{exemplar/10/13/3/42/tables/table.tex}
	\end{table}
	\fi
If $n^2$ is the value of the chosen number that is greater than 500 and also a perfect square, then
\begin{align}
n^2 \in (500,1000] \\
\implies n \in (22.36,31.62]
\end{align}
$n$ can take 9 integer values in the above interval.
If $A, B$ represent the first and second player winning a prize respectively, 
\begin{enumerate}
\item 
\begin{align}
               \pr{A} = \frac{9}{1000}
\end{align}
\item 
Given that the first player has won, the second player has only 8 numbers left to choose.  Hence, 
\begin{align}
\pr{B|A}=\frac{8}{1000}
\end{align}
\end{enumerate}

\item Four cards are successively drawn without replaccement from a deck of 52 playing cards. What is the probability that all the four cards are kings?
\\
\solution
Let $X_i, i = 1,2,3,4$ denote a king in the $i$th draw.
Then,
\begin{multline}
	\pr{X_{1}}=\frac{4}{52},\
\pr{X_{2}|X_{1}} =\frac{3}{51},\
	\pr{X_{3}|X_{2}X_{1}}=\frac{2}{50},\
	\pr{X_{4}|X_{1}X_{2}X_{3}} =\frac{1}{49}
	\\
	\implies \pr{X_{1}X_{2}X_{3}X_{4}}  
                                     = \frac{4}{52} \times \frac{3}{51} \times \frac{2}{50} \times \frac{1}{49} 
                                     = \frac{1}{270725}
\end{multline}
which is the desired probability.

\item Two natural numbers $r, s$ are drawn one at a time, without replacement from
the set $S = {1,2,3, \ldots,n}$. Find $P[r\le p|s\le p]$.
\\
\solution 
There are two conditions, 
\begin{enumerate}
\item  $s$ is chosen first:
\begin{align}
\pr{r\le p\,|\,s\le p} = \frac{\pr{r \le p, \, s \le p}}{\pr{s \le p}}
\end{align}
\begin{enumerate}
\item $p<1$:
This case is never possible as $s,r \ge 1$
\item $1 \le p \le n$:
Then we can say that,
\begin{align}
\pr{r \le p, \, s \le p} &= \frac{p(p-1)}{n(n-1)} \label{eq:12/13/3/34/1},\\
\pr{s \le p} &= \frac{p}{n} \label{eq:12/13/3/34/2}
\end{align}
From \eqref{eq:12/13/3/34/1} and \eqref{eq:12/13/3/34/2}:
\begin{align}
\pr{r\le p|s\le p} &= \frac{\pr{r \le p, \, s \le p}}{\pr{s \le p}}\\
&= \frac{\frac{p(p-1)}{n(n-1)}}{\frac{p}{n}}
=\frac{p-1}{n-1}
\end{align}
\item $p>n$:
\begin{align}
\pr{r \le p, \, s \le p} &= 1 \label{eq:12/13/3/34/3},\\
\pr{s \le p} &= 1 \label{eq:12/13/3/34/4}
\end{align}
From \eqref{eq:12/13/3/34/3} and \eqref{eq:12/13/3/34/4}:
\begin{align}
\pr{r\le p|s\le p} &= \frac{\pr{r \le p, \, s \le p}}{\pr{s \le p}}\\
&= 1 
\end{align}
\end{enumerate}
\item  r is chosen first:
\begin{align}
\pr{r\le p|s\le p} = \frac{\pr{r \le p, \, s \le p}}{\pr{s \le p}}
\end{align}
\begin{enumerate}
\item $p<1$:
This case is never possible as $r,s \ge 1$
\item $1 \le p \le n$:
\begin{align}
\pr{r \le p, \, s \le p} &= \frac{p(p-1)}{n(n-1)} \label{eq:12/13/3/34/5},\\
\pr{s \le p} &= \frac{p-1}{n-1} \label{eq:12/13/3/34/6}
\end{align}
From \eqref{eq:12/13/3/34/5} and \eqref{eq:12/13/3/34/6}:
\begin{align}
\pr{r\le p|s\le p} &= \frac{\pr{r \le p, \, s \le p}}{\pr{s \le p}}\\
&= \frac{\frac{p(p-1)}{n(n-1)}}{\frac{p}{n}}
=\frac{p}{n}
\end{align}
\item $p>n$:
\begin{align}
\pr{r \le p, \, s \le p} &= 1 \label{eq:12/13/3/34/7},\\
\pr{s \le p} &= 1 \label{eq:12/13/3/34/8}
\end{align}
From \eqref{eq:12/13/3/34/7} and \eqref{eq:12/13/3/34/8}:
\begin{align}
\pr{r\le p|s\le p} &= \frac{\pr{r \le p, \, s \le p}}{\pr{s \le p}}\\
&= 1 
\end{align}
\end{enumerate}
\end{enumerate}






















\item Three bags contain a number of red and white balls as follows:
$B_1$: 3 red balls, $B_2$: 2 red balls and 1 white ball, $B_3$: 3 white balls.
The probability that bag $i$ will be chosen and a ball is selected is $i/6, i=1,2,3$.
what is the probability that
\label{prob:12/13/3/41}
\begin{enumerate}
	\item a red ball will be selected?     
	\item  a white ball will be selected?
\end{enumerate}
\solution
\begin{table}[!ht]
\input{exemplar/12/13/3/41/tables/table.tex}
\caption{random variables of objects}
\label{tab:exemplar 12.13.3.41}
\end{table}
\begin{align}
\pr{X=i}=
\begin{cases}
\frac{1}{6},\text{when i=1}\\
\frac{2}{6},\text{when i=2}\\
\frac{3}{6},\text{when i=3}
\end{cases}
\end{align}

we know that that the conditional probability is defined as

                     $\pr{A|B}=\frac{\pr{A,B}}{\pr{B}}$


\begin{enumerate}
\item
The probability that a red ball will be selected is:
\begin{align}
\pr{Y=1}&=\pr{Y=1,X=1}+\pr{Y=1,X=2}+\pr{Y=1,X=3}\\
&=\pr{X=1}\times\pr{Y=1|X=1}+\pr{X=2}\times\pr{Y=1|X=2}+\pr{X=3}\times\pr{Y=1|X=3}\\
&=\frac{1}{6}\times\frac{3}{3}+\frac{2}{6}\times\frac{2}{3}+\frac{3}{6}\times0\\
&=\frac{7}{18}
\end{align}
\item
The probability that a white ball will be selected is:
\begin{align}
\pr{Y=0}&=\pr{Y=0,X=1}+\pr{Y=0,X=2}+\pr{Y=0,X=3}\\
&=\pr{X=1}\times\pr{Y=0|X=1}+\pr{X=2}\times\pr{Y=0|X=2}+\pr{X=3}\times\pr{Y=0|X=3}\\
&=\frac{1}{6}\times0+\frac{2}{6}\times\frac{1}{3}+\frac{3}{6}\times\frac{3}{3}\\
&=\frac{11}{18}
\end{align}
\end{enumerate}




%
\item Refer to \probref{prob:12/13/3/41}. If a white ball is selected, what is the probability that it came from
\begin{enumerate}
\item  $B_2$
\item  $B_3$
\end{enumerate}
\solution
%\iffalse
Referring to the above question, 
\begin{table}[h!]
 \begin{center}
    \begin{tabular}{|l|c|r|}
    \hline
    Parameter & Values & Description\\
    \hline
    $X$ & 0 & red balls\\
    {} & 1 & white balls\\
    \hline
    $Y$ & 1 & Bag 1\\
    {} & 2 & Bag 2\\
    {} & 3 & Bag 3\\
    \hline
    \end{tabular}
    \end{center}
    \caption{Table 1}
  \label{tab:exampler/12/13/3/42} 
\end{table}
\fi
\begin{enumerate}
\item The desired probability is
\begin{align}
	p_{X|Y}\brak{2|0}&= 
	\frac{p_{Y|X}\brak{0|2}{p_X(2)}}{p_Y(0)}
		\\
			 &=\frac{\frac{1}{3}\times \frac{1}{3}}{\frac{11}{18}}
			 =
\frac{2}{11}
\end{align}
\item Similarly, 
\begin{align}
	p_{X|Y}\brak{3|0}&= 
	\frac{p_{Y|X}\brak{0|3}{p_X(3)}}{p_Y(0)}
		\\
			 &=\frac{\frac{1}{2}}{\frac{11}{18}}
			 =
\frac{9}{11}
\end{align}
\end{enumerate}

\item If P(A) = $\frac{4}{5}$ and P(AB) = $\frac{7}{10}$, then $P(B|A)$ is equal to\\
\solution
%From 
\eqref{eq:axiom-cond},
the required probability is
\begin{align}
	\pr{B|A} = \frac{\brak{\frac{7}{10}}}{\brak{\frac{4}{5}}}
           &= \frac{7}{8}
\end{align}


\item A flashlight has 8 batteries out of which 3 are dead. If two batteries are selected without replacement and tested, find the probability that both are dead.\\
%Let $X_i \in \cbrak{0,1}, i \in 1,2$ represent the $i$th battery, 0 denoting the battery being dead.
From the given information,
\begin{align}
	\pr{X_1=0}&=\frac{3}{8}, \pr{X_2=0|X_1=0}=\frac{2}{7}, 
\\
\implies 
\pr{X_1=0, X_2=0}&=\pr{X_1=0}\pr{X_2=0|X_1=0} =\frac{3}{28}
\end{align}
from
\eqref{eq:axiom-cond}.

\item Assume that in a family, each child is equally likely to be a boy or a girl. A family with three children is chosen at random. The probability that the eldest child is a girl given that the family has at least one girl is
\begin{enumerate}
\item $\frac{1}{2}$
\item $\frac{1}{3}$
\item $\frac{2}{3}$
\item $\frac{4}{7}$
\end{enumerate}
%\input{exemplar/12/13/3/77/assignment7.tex}
\item In a college, $30\%$ students fail in physics, $25\%$ fail in mathematics and $10\%$ fail in both. One student is chosen at random. The probability that she fails in physics if she has failed in mathematics is
\begin{enumerate}
    \item $\frac{1}{10}$
    \item $\frac{2}{5}$
    \item $\frac{9}{20}$
    \item $\frac{1}{3}$
\end{enumerate}
%From the given information,
\begin{align}
	\pr{P}&=0.3,\
\pr{M}=0.1,\
\pr{PM}=0.25\\
\implies 
\pr{P|M}&=\frac{\pr{PM}}{\pr{M}}
        =\frac{0.1}{0.25}
        =\frac{2}{5}
        \end{align}


\end{enumerate}

\newpage
    \section{Uniform Distribution}
\subsection{Formulae}
\begin{enumerate}[label=\thesubsection.\arabic*.,ref=\thesubsection.\theenumi]
\item  Let $X \in \cbrak{1,2,3,4,5,6}$ be the random variables representing the outcome for a die.  Assuming the die to be fair, the probability mass function (pmf) is expressed as 
\begin{align}
\label{eq:dice_pmf_xi}
p_{X}(n) = 
\begin{cases}
\frac{1}{6} & 1 \le n \le 6
\\
0 & otherwise
\end{cases}
\end{align}
\item The $Z$-transform of $X$ is given by 
\begin{align}
P_{X}(z) =  \frac{1}{6}\sum_{n = 1}^{6}z^{-n}
=\frac{z^{-1}\brak{1-z^{-6}}}{6\brak{1-z^{-1}}}, \quad \abs{z} > 1
\label{eq:dice_xiz}
\end{align}
upon summing up the geometric progression.  
\item From \eqref{eq:dice_xiz}, the CDF of $X$ is given by
\begin{align}
F_{X}\brak{n} = 	\pr{X \le n} 
 = 
\begin{cases}
0 & n < 1 \\
\frac{n}{6} & 1 \le n \leq 6 \\
1 & \text{otherwise}
\end{cases}
\label{eq:dice_xiF}
\end{align}
and plotted in \figref{fig:ncert/11/16/3/3/1}.
\begin{figure}[ht]
\centering
\includegraphics[width = \columnwidth]{ncert/11/16/3/3/figs/fig.png}
\caption{CDF}
\label{fig:ncert/11/16/3/3/1}
\end{figure}
\end{enumerate}

\subsection{NCERT}
\begin{enumerate}[label=\thesection.\arabic*,ref=\thesection.\theenumi]
\item A die is thrown, find the probability of following events:
\begin{enumerate}
\item A prime number will appear
\item A number greater than or equal to 3 will appear
\item A number less than or equal to one will appear
\item A number more than 6 will appear
\item A number less than 6 will appear
\end{enumerate}
\solution
The CDF of the random variable $X$ representing the roll of a dice, is available in \eqref{eq:dice_xiF}.
\begin{enumerate}
\item The set of possible prime numbers in a die roll contains 2,3,5
\begin{align} 
\pr{X \in \{2,3,5\}} &= p_{X}\brak{2} + p_{X}\brak{3} + p_{X}\brak{5}\\
&= \frac{1}{2}
\end{align}
\item The probability that a number greater than or equal to 3 will appear is given by
\begin{align}
\pr{X \geq 3} &= 1 - \pr{X \leq 2}\\
&= 1 - F_{X}\brak{2}\\
&= \frac{2}{3}
\end{align}
\item The probability that a number less than or equal to 1 will appear is given by
\begin{align}
\pr{X \leq 1} &= F_{X}\brak{1}\\
&= \frac{1}{6}
\end{align}
\item The probability that a number greater than 6 will appear is given by
\begin{align}
\pr{X > 6} &= 1 - \pr{X \leq 6}\\
&= 1 - F_{X}\brak{6}\\
&= 0
\end{align}
\item The probability that a number less than 6 will appear is given by
\begin{align}
\pr{X < 6} &= \pr{X \leq 5}\\
&= F_{X}\brak{5}\\
&= \frac{5}{6}
\end{align}
\end{enumerate}




\item All the jacks, queens and kings are removed from a deck of 52 playing cards. The remaining cards are well shuffled and then one card is drawn at random. Giving ace a value 1 similar value for other cards, find the probability that the card has a value 
\begin{enumerate}
	\item 7
	\item greater than 7
	\item less than 7
\end{enumerate}
\solution
Number of cards left after removing all jacks, queens and kings 
\begin{align}
N	= 52 - 4\times 3
	= 40
\end{align}
%\begin{table}[H]
%\def\arraystretch{1.2}
%\begin{tabular}{|c|c|c|}
%\hline
%	\textbf{Parameter} &\textbf{Value} &\textbf{Description}\\ \hline
%	$X$ &1-10 &Represents the value of the card picked \\ \hline
%\end{tabular}
%\end{table}
Let $1 \le X \le 10$ be the value of the card picked.  Then,
\begin{align}
	p_X(k) &= \Pr(X=k)\ \forall\ 1 \leq k \leq 10\\
	&= \frac{4\times 1}{40}\\
	&= \frac{1}{10}\\
	\therefore p_X(k) &= 
	\begin{cases}
		\frac{1}{10} & 1 \leq k \leq 10\\
		0 & \text{otherwise}
	\end{cases}
\end{align}
and
\begin{align}
	F_{X}(k) &= \sum_{m=0}^{k}p_{X}(m) \quad 1 \leq k \leq 10\\
	&= \frac{k}{10}\\
	\therefore F_{X}(k) &= 
	\begin{cases}
		0 & k \leq 0\\
		\frac{k}{10} & 1\leq k \leq 10\\
		1 & k > 10 
	\end{cases}
\end{align}
\begin{enumerate}
	\item Probability that card has value equal to 7 is
		\begin{align}
			 p_{X}(7)
			= \frac{1}{10}
		\end{align}
	\item Probability that card has value greater than 7 is
		\begin{align}
			1 - F_X(7)
			&= 1 - \frac{7}{10}
			\\
			&= \frac{3}{10}
		\end{align}
	\item Probability that card has value less than 7 is
		\begin{align}
			 F_{X}(6)
			=\frac{6}{10}
		\end{align}
\end{enumerate}

\item Two dice are numbered 1,2,3,4,5,6 and 1,1,2,2,3,3 respectively. They are thrown and the sum of then numbers on them is noted. Find the probability of getting each sum from 2 to 9 seperately
	\\
\solution
%From \eqref{eq:dice_pmf_xi},
The $Z$-transform of the first die $X_1$ is given by 
\eqref{eq:dice_xiz}.
The pmf of the second die is 
\begin{align}
\label{eq:dice_pmf_x2}
p_{X_2}(n) = 
\begin{cases}
\frac{1}{3} & 1 \le n \le 3 
\\
0 & \text{otherwise}
\end{cases}
\end{align}
yielding 
\begin{align}
M_{X_2}(z) =  \frac{1}{3}\sum_{n = 1}^{3}z^{-n}
=\frac{z^{-1}{(1-z^{-3})}}{3{(1-z^{-1})}}, {|z|} > 1
\end{align}
upon substituting in 
\eqref{eq:dice_xz}.
From 
\eqref{eq:dice_xzprod_def},
The $Z$-transform of X is given as
\begin{align}
	M_X(z) &= \frac{z^{-1}{(1-z^{-6})}}{6{(1-z^{-1})}} \times \frac{z^{-1}{(1-z^{-3})}}{3{(1-z^{-1})}}
\\
	       &= \frac{1}{18}\left[\frac{z^{-2}\brak{(1-z^{-3}-z^{-6}-z^{-9})}}{{(1-z^{-1})}^2}\right]
\end{align}
Using 
\eqref{eq:dice_xdef_props},
after some algebra, it can be shown that,
\begin{multline}
\frac{1}{18}[{{n-1}u(n-1)-{n-4}u(n-4)}
{-(n-7)u(n-7)-(n-10)u(n-10)}]\\
\system{Z}\\
\frac{1}{18} \left[{\frac{z^{-2}{1-z^{-3}-z^{-6}-z^{-9}}}{({1-z^{-1}})^2}}\right]
\end{multline}
Hence,
\iffalse
\begin{multline}
p_{X}(n) = \frac{1}{18}[{{n-1}u(n-1)-{n-4}u(n-4)-}\\
              {(n-7)u(n-7)-(n-10)u(n-10)}]
\end{multline}
\fi
\begin{align}
  p_X(n) &= 
  \begin{cases}
  0 & n \le 1
  \\
  \frac{n-1}{18} &  2 \le n \le  4
  \\
  \frac{1}{6} & 5 \le n \le 7
  \\
  \frac{10-n}{18} & 8 \le n \le 9
  \\
  0 & n \ge 10
  \end{cases}
  \end{align}
  \iffalse
hence, the probabilities are,
\begin{align}
  p_X(n) &= 
\begin{cases}
   \frac{1}{18} & n = 2 \\
   \frac{1}{9} & n = 3 \\
   \frac{1}{6} & n = 4 \\
   \frac{1}{6} & n = 5 \\
   \frac{1}{6} & n = 6 \\
   \frac{1}{6} & n = 7 \\
   \frac{1}{9} & n = 8 \\
   \frac{1}{18} & n = 9 
\end{cases}
\end{align}
\fi
See \figref{fig:mgf-sum-2}.
The experiment of rolling the dice was simulated using Python for 10000 samples.  
%
\begin{figure}[H]
\centering
\includegraphics[width=\columnwidth]{exemplar/10/13/3/23/figs/pmf.png}
\caption{Plot of $p_X(n)$.Simulations are close to the analysis. }
\label{fig:mgf-sum-2}
\end{figure}

\item A game of chance consists of spinning an arrow
which comes to rest pointing at one of the numbers
1, 2, 3, 4, 5, 6, 7, 8 (see \figref{fig:ncert/10/15/1/12/my_label}), and these are equally
likely outcomes. What is the probability that it will
point at:
\begin{enumerate}
\item 
8?
\item 
an odd number?
\item 
a number greater than 2?
\item 
a number less than 9?
\end{enumerate}
\begin{figure} [p h]
	\centering
	\resizebox{\columnwidth}{!}{\input{ncert/10/15/1/12/figs/tikz}}
    \caption{Spinner}
    \label{fig:ncert/10/15/1/12/my_label}
\end{figure}
\solution
Let X be a random variable defined as the value given by the pointer.  Then,
%The distribution is unform since all the outcomes are equally likely.
\begin{align}
	\pr{X=i}&=\frac{1}{8}\quad 1 \le i \le 8\\
	F_{X} \brak{i} &= \pr{X \leq i}\\
    &=
    \begin{cases}
        0, & i \le 0\\
        \frac{i}{8} & 1 \le i \le 8\\
        1, & i \ge 9
    \end{cases}
\end{align}
which are plotted in
\figref{fig:ncert/10/15/1/12/pmf}
and
\figref{fig:ncert/10/15/1/12/cdf}
respectively.
\begin{figure}[H]
	\centering
    \includegraphics[width=0.7\columnwidth]{ncert/10/15/1/12/figs/pmf.png}
\caption{Plot of Probability Mass Function}
\label{fig:ncert/10/15/1/12/pmf}
\end{figure}

\begin{figure}[H]
	\centering
    \includegraphics[width=0.7\columnwidth]{ncert/10/15/1/12/figs/cdf.png}
\caption{Plot of Cumulative Distribution Function}
\label{fig:ncert/10/15/1/12/cdf}
\end{figure}
%
\begin{enumerate}
\item 
\begin{align}
    \pr{X=8} = \frac{1}{8}
    = 0.125
\end{align}
\item 
For $i$ being odd, 
\begin{align}
    \pr{X=\cbrak{1,3,5,7}} = \frac{4}{8}
    = 0.5
\end{align}
\item 
\begin{align}
    \pr{X > 2} &= 1 - \pr{X \le 2}\\
    &= 1 - \brak{F_{X}\brak{2}-F_{X}\brak{0}}\\
    &= \frac{6}{8}
\end{align}
%
\item 
\begin{align}
    \pr{1 \le X < 9} &= F_{X}\brak{8}-F_{X}\brak{0}
= 1
\end{align}
\end{enumerate}

    \end{enumerate}


\newpage
    \section{Sum of Random Variables}
\subsection{Formulae}
\begin{enumerate}[label=\thesubsection.\arabic*,ref=\thesubsection.\theenumi]
	\item Consider the rv 
\begin{align}
	X = X_1 + X_2,
\label{eq:dice_xdef}
\end{align}
where $X_1$ and $X_2$ are independent uniform rvs with pmf given in 
\eqref{eq:dice_pmf_xi}.
\item {\em Convolution: }
From \eqref{eq:dice_xdef},
\begin{align}
p_X(n) &= \pr{X_1 + X_2 = n} = \pr{X_1  = n -X_2}
\\
&= \sum_{k}^{}\pr{X_1  = n -k | X_2 = k}p_{X_2}(k)
\label{eq:dice_x_sum}
\end{align}
after unconditioning.  $\because X_1$ and $X_2$ are independent,
\begin{align}
\pr{X_1  = n -k | X_2 = k} 
= \pr{X_1  = n -k} = p_{X_1}(n-k)
\label{eq:dice_x1_indep}
\end{align}
From \eqref{eq:dice_x_sum} and \eqref{eq:dice_x1_indep},
\begin{align}
p_X(n) = \sum_{k}^{}p_{X_1}(n-k)p_{X_2}(k) = p_{X_1}(n)*p_{X_2}(n)
\label{eq:dice_x_conv}
\end{align}
where $*$ denotes the convolution operation. 
%\cite{proakis_dsp}.  
\item (Triangular PMF:) 
Substituting from \eqref{eq:dice_pmf_xi}
in \eqref{eq:dice_x_conv},
\begin{align}
p_X(n) = \frac{1}{6}\sum_{k=1}^{6}p_{X_1}(n-k)= \frac{1}{6}\sum_{k=n-6}^{n-1}p_{X_1}(k)
\label{eq:dice_x_conv_x1}
\end{align}
\begin{align}
\because p_{X_1}(k) &= 0, \quad k \le 1, k \ge 6.
\end{align}
From \eqref{eq:dice_x_conv_x1},
%
\begin{align}
p_X(n) &= 
\begin{cases}
0 & n < 1
\\
\frac{1}{6}\sum_{k=1}^{n-1}p_{X_1}(k) &  1 \le n-1 \le  6
\\
\frac{1}{6}\sum_{k=n-6}^{6}p_{X_1}(k) & 1 < n-6 \le 6
\\
0 & n > 12
\end{cases}
\label{eq:dice_x_conv_cond}
\end{align}
	Substituting from \eqref{eq:dice_pmf_xi} in \eqref{eq:dice_x_conv_cond},
\begin{align}
p_X(n) &= 
\begin{cases}
0 & n < 1
\\
\frac{n-1}{36} &  2 \le n \le  7
\\
\frac{13-n}{36} & 7 < n \le 12
\\
0 & n > 12
\end{cases}
\label{eq:dice_x_conv_final}
\end{align}
%satisfying \eqref{eq:dice_wrong}.
%\section{Triangular Distribution}
%\begin{enumerate}[label=\thesection.\arabic*.,ref=\thesection.\theenumi]
%
\item 
The experiment of rolling the dice was simulated using Python for 10000 samples.  These were generated using Python libraries for uniform distribution. The frequencies for each outcome were then used to compute the resulting pmf, which  is plotted in Figure \ref{fig:dice}.  The theoretical pmf obtained in \eqref{eq:dice_x_conv_final} is plotted for comparison.  
%
\begin{figure}[H]
\centering
\includegraphics[width=\columnwidth]{./figs/sum/pmf.png}
\caption{Plot of $p_X(n)$.  Simulations are close to the analysis. }
\label{fig:dice}
\end{figure}
\item The python code is available below

	\lstinputlisting{codes/sum/dice.py}
	\iffalse
\begin{lstlisting}
/codes/sum/dice.py
\end{lstlisting}
\fi
\item 
The $Z$-transform of $X$ is defined as
\begin{align}
M_X(z) = E\sbrak{z^{-X}} = \sum_{k=-\infty}^{\infty}p_X(k)z^{-k}
\label{eq:dice_xz}
\end{align}
\item If $X_1$ and $X_2$ are independent, the MGF of 
\begin{align}
	X = X_1 + X_2
\end{align}
is given by 
	\begin{align}
%p_X(n) &= p_{X_1}(n)*p_{X_2}(n),
%\\
M_X(z) &= M_{X_1}(z)M_{X_2}(z)
\label{eq:dice_xzprod_def}
\end{align}
The above property follows from Fourier analysis and is fundamental to signal processing. 
\item For
\eqref{eq:dice_pmf_xi},
the $Z$-transform of $X_1$ is given by 
\begin{align}
M_{X_1}(z) =  \frac{1}{6}\sum_{n = 1}^{6}z^{-n}
=\frac{z^{-1}\brak{1-z^{-6}}}{6\brak{1-z^{-1}}}, \quad \abs{z} > 1
\label{eq:dice_xiz}
\end{align}
upon summing up the geometric progression.  
\item From \eqref{eq:dice_xiz} and \eqref{eq:dice_xzprod_def},
\begin{align}
M_X(z) &= \cbrak{\frac{z^{-1}\brak{1-z^{-6}}}{6\brak{1-z^{-1}}}}^2
\\
&= \frac{1}{36}\frac{z^{-2}\brak{1-2z^{-6}+z^{-12}}}{\brak{1-z^{-1}}^2}
\label{eq:dice_xzprod}
\end{align}
Using the fact that 
%\cite{proakis_dsp}
\begin{align}
\begin{split}
p_X(n-k) &\system{Z}P_X(z)z^{-k},
\\
nu(n)&\system{Z} \frac{z^{-1}}{\brak{1-z^{-1}}^2}
\end{split}
\label{eq:dice_xdef_props}
\end{align}
after some algebra, it can be shown that
%{\tiny
\begin{multline}
\frac{1}{36}\sbrak{\brak{n-1}u(n-1) - 2 \brak{n-7}u(n-7)
 +\brak{n-13}u(n-13)}
\\
\system{Z}
\frac{1}{36}\frac{z^{-2}\brak{1-2z^{-6}+z^{-12}}}{\brak{1-z^{-1}}^2}
\label{eq:dice_xz_closed}
\end{multline}
%}

where 
\begin{align}
u(n) =
\begin{cases}
1 & n \ge 0
\\
0 & n < 0
\end{cases}
\end{align}
From \eqref{eq:dice_xz}, \eqref{eq:dice_xzprod} and \eqref{eq:dice_xz_closed}
\begin{align}
p_{X}(n) = \frac{1}{36}\lsbrak{\brak{n-1}u(n-1) 
}
\rsbrak{- 2 \brak{n-7}u(n-7)+\brak{n-13}u(n-13)}
\end{align}
which is the same as \eqref{eq:dice_x_conv_final}.  Note that  \eqref{eq:dice_x_conv_final} can be obtained from \eqref{eq:dice_xz_closed} using contour integration as well.
% \cite{proakis_dsp}.  


%\item 
%We have shown how a simple problem of throwing a dice can be used for learning not just probability but concepts in  signal processing as well.  Inversion of the $Z$-transform for finding the pmf using contour integration, though not discussed here, opens a window to complex analysis too.  Note that the solutions that are provided  can be easily understood using high school math like arithmetic and geometric sums.  Thus, school students can use a bit of college level math to obtain simpler solutions to their problems.  College students can be exposed to advanced mathematics through simple problems from high school texts.  In addition, both can learn how to verify theoretical results through computer simulations.

%\bibliographystyle{tMES}
%\bibliography{school}
%
%\end{document}
\end{enumerate}



\subsection{NCERT}
\begin{enumerate}[label=\thesection.\arabic*,ref=\thesection.\theenumi]
	\item Two dice, one blue and one grey, are thrown at the same time.   The event defined by the sum of the two numbers appearing on the top of the dice can have 11 possible outcomes 2, 3, 4, 5, 6, 6, 8, 9, 10, 11 and 12.  A student argues that each of these outcomes has a probability $\frac{1}{11}$.  Do you agree with this argument?  Justify your answer.
\item Two dice are numbered 1,2,3,4,5,6 and 1,1,2,2,3,3 respectively. They are thrown and the sum of then numbers on them is noted. Find the probability of getting each sum from 2 to 9 seperately
	\\
\solution
%%From \eqref{eq:dice_pmf_xi},
The $Z$-transform of the first die $X_1$ is given by 
\eqref{eq:dice_xiz}.
The pmf of the second die is 
\begin{align}
\label{eq:dice_pmf_x2}
p_{X_2}(n) = 
\begin{cases}
\frac{1}{3} & 1 \le n \le 3 
\\
0 & \text{otherwise}
\end{cases}
\end{align}
yielding 
\begin{align}
M_{X_2}(z) =  \frac{1}{3}\sum_{n = 1}^{3}z^{-n}
=\frac{z^{-1}{(1-z^{-3})}}{3{(1-z^{-1})}}, {|z|} > 1
\end{align}
upon substituting in 
\eqref{eq:dice_xz}.
From 
\eqref{eq:dice_xzprod_def},
The $Z$-transform of X is given as
\begin{align}
	M_X(z) &= \frac{z^{-1}{(1-z^{-6})}}{6{(1-z^{-1})}} \times \frac{z^{-1}{(1-z^{-3})}}{3{(1-z^{-1})}}
\\
	       &= \frac{1}{18}\left[\frac{z^{-2}\brak{(1-z^{-3}-z^{-6}-z^{-9})}}{{(1-z^{-1})}^2}\right]
\end{align}
Using 
\eqref{eq:dice_xdef_props},
after some algebra, it can be shown that,
\begin{multline}
\frac{1}{18}[{{n-1}u(n-1)-{n-4}u(n-4)}
{-(n-7)u(n-7)-(n-10)u(n-10)}]\\
\system{Z}\\
\frac{1}{18} \left[{\frac{z^{-2}{1-z^{-3}-z^{-6}-z^{-9}}}{({1-z^{-1}})^2}}\right]
\end{multline}
Hence,
\iffalse
\begin{multline}
p_{X}(n) = \frac{1}{18}[{{n-1}u(n-1)-{n-4}u(n-4)-}\\
              {(n-7)u(n-7)-(n-10)u(n-10)}]
\end{multline}
\fi
\begin{align}
  p_X(n) &= 
  \begin{cases}
  0 & n \le 1
  \\
  \frac{n-1}{18} &  2 \le n \le  4
  \\
  \frac{1}{6} & 5 \le n \le 7
  \\
  \frac{10-n}{18} & 8 \le n \le 9
  \\
  0 & n \ge 10
  \end{cases}
  \end{align}
  \iffalse
hence, the probabilities are,
\begin{align}
  p_X(n) &= 
\begin{cases}
   \frac{1}{18} & n = 2 \\
   \frac{1}{9} & n = 3 \\
   \frac{1}{6} & n = 4 \\
   \frac{1}{6} & n = 5 \\
   \frac{1}{6} & n = 6 \\
   \frac{1}{6} & n = 7 \\
   \frac{1}{9} & n = 8 \\
   \frac{1}{18} & n = 9 
\end{cases}
\end{align}
\fi
See \figref{fig:mgf-sum-2}.
The experiment of rolling the dice was simulated using Python for 10000 samples.  
%
\begin{figure}[H]
\centering
\includegraphics[width=\columnwidth]{exemplar/10/13/3/23/figs/pmf.png}
\caption{Plot of $p_X(n)$.Simulations are close to the analysis. }
\label{fig:mgf-sum-2}
\end{figure}

\end{enumerate}

\newpage
	\section{Binomial}
\subsection{Formulae}
\begin{enumerate}[label=\thesubsection.\arabic*.,ref=\thesubsection.\theenumi]
\item The Binomial distribution is defined as 
\begin{align}
X &= X_1+X_2+...+X_n,
\end{align}
Where $X_i$ are i.i.d bernoulli.
\item For a Binomial random variable $X$ with parameters $n,p$,
\begin{align}
\label{eq:mgf-binom}
M_X(z)= \brak{q+pz^{-1}}^n
\end{align}
\item The mean for the Binomial r.v. is
\begin{align}
\label{eq:mean-binom}
E\sbrak{X}= np
\end{align}
\solution
From 
\eqref{eq:mgf-mean}
and 
\eqref{eq:mgf-binom},
\begin{align}
E\sbrak{X} 
&= \frac{d(q + pz)^n}{dz}\vert_{z=1}
\\
&=np(q + pz)^{n-1}|_{z=1}
\\
&= np(q + p)^{n-1}
\end{align}
yielding
\eqref{eq:mean-binom}
\begin{align}
\because p+q=1
\end{align}
\item In a hurdle race, a player has to cross 10 hurdles. The probability that he will
clear each hurdle is $\frac{5}{6}$. What is the probability that he will knock down fewer than 2 hurdles?
\solution See the following code
		\lstinputlisting{codes/binomial/binomial.py}
\end{enumerate}
%

\subsection{NCERT}
%    \section{Uniform}
\begin{enumerate}[label=\thesection.\arabic*,ref=\thesection.\theenumi]
\item A die is thrown, find the probability of following events:
\begin{enumerate}
\item A prime number will appear
\item A number greater than or equal to 3 will appear
\item A number less than or equal to one will appear
\item A number more than 6 will appear
\item A number less than 6 will appear
\end{enumerate}
\solution
The CDF of the random variable $X$ representing the roll of a dice, is available in \eqref{eq:dice_xiF}.
\begin{enumerate}
\item The set of possible prime numbers in a die roll contains 2,3,5
\begin{align} 
\pr{X \in \{2,3,5\}} &= p_{X}\brak{2} + p_{X}\brak{3} + p_{X}\brak{5}\\
&= \frac{1}{2}
\end{align}
\item The probability that a number greater than or equal to 3 will appear is given by
\begin{align}
\pr{X \geq 3} &= 1 - \pr{X \leq 2}\\
&= 1 - F_{X}\brak{2}\\
&= \frac{2}{3}
\end{align}
\item The probability that a number less than or equal to 1 will appear is given by
\begin{align}
\pr{X \leq 1} &= F_{X}\brak{1}\\
&= \frac{1}{6}
\end{align}
\item The probability that a number greater than 6 will appear is given by
\begin{align}
\pr{X > 6} &= 1 - \pr{X \leq 6}\\
&= 1 - F_{X}\brak{6}\\
&= 0
\end{align}
\item The probability that a number less than 6 will appear is given by
\begin{align}
\pr{X < 6} &= \pr{X \leq 5}\\
&= F_{X}\brak{5}\\
&= \frac{5}{6}
\end{align}
\end{enumerate}




\item All the jacks, queens and kings are removed from a deck of 52 playing cards. The remaining cards are well shuffled and then one card is drawn at random. Giving ace a value 1 similar value for other cards, find the probability that the card has a value 
\begin{enumerate}
	\item 7
	\item greater than 7
	\item less than 7
\end{enumerate}
Number of cards left after removing all jacks, queens and kings 
\begin{align}
N	= 52 - 4\times 3
	= 40
\end{align}
%\begin{table}[H]
%\def\arraystretch{1.2}
%\begin{tabular}{|c|c|c|}
%\hline
%	\textbf{Parameter} &\textbf{Value} &\textbf{Description}\\ \hline
%	$X$ &1-10 &Represents the value of the card picked \\ \hline
%\end{tabular}
%\end{table}
Let $1 \le X \le 10$ be the value of the card picked.  Then,
\begin{align}
	p_X(k) &= \Pr(X=k)\ \forall\ 1 \leq k \leq 10\\
	&= \frac{4\times 1}{40}\\
	&= \frac{1}{10}\\
	\therefore p_X(k) &= 
	\begin{cases}
		\frac{1}{10} & 1 \leq k \leq 10\\
		0 & \text{otherwise}
	\end{cases}
\end{align}
and
\begin{align}
	F_{X}(k) &= \sum_{m=0}^{k}p_{X}(m) \quad 1 \leq k \leq 10\\
	&= \frac{k}{10}\\
	\therefore F_{X}(k) &= 
	\begin{cases}
		0 & k \leq 0\\
		\frac{k}{10} & 1\leq k \leq 10\\
		1 & k > 10 
	\end{cases}
\end{align}
\begin{enumerate}
	\item Probability that card has value equal to 7 is
		\begin{align}
			 p_{X}(7)
			= \frac{1}{10}
		\end{align}
	\item Probability that card has value greater than 7 is
		\begin{align}
			1 - F_X(7)
			&= 1 - \frac{7}{10}
			\\
			&= \frac{3}{10}
		\end{align}
	\item Probability that card has value less than 7 is
		\begin{align}
			 F_{X}(6)
			=\frac{6}{10}
		\end{align}
\end{enumerate}

\item Two dice are numbered 1,2,3,4,5,6 and 1,1,2,2,3,3 respectively. They are thrown and the sum of then numbers on them is noted. Find the probability of getting each sum from 2 to 9 seperately
\solution
%From \eqref{eq:dice_pmf_xi},
The $Z$-transform of the first die $X_1$ is given by 
\eqref{eq:dice_xiz}.
The pmf of the second die is 
\begin{align}
\label{eq:dice_pmf_x2}
p_{X_2}(n) = 
\begin{cases}
\frac{1}{3} & 1 \le n \le 3 
\\
0 & \text{otherwise}
\end{cases}
\end{align}
yielding 
\begin{align}
M_{X_2}(z) =  \frac{1}{3}\sum_{n = 1}^{3}z^{-n}
=\frac{z^{-1}{(1-z^{-3})}}{3{(1-z^{-1})}}, {|z|} > 1
\end{align}
upon substituting in 
\eqref{eq:dice_xz}.
From 
\eqref{eq:dice_xzprod_def},
The $Z$-transform of X is given as
\begin{align}
	M_X(z) &= \frac{z^{-1}{(1-z^{-6})}}{6{(1-z^{-1})}} \times \frac{z^{-1}{(1-z^{-3})}}{3{(1-z^{-1})}}
\\
	       &= \frac{1}{18}\left[\frac{z^{-2}\brak{(1-z^{-3}-z^{-6}-z^{-9})}}{{(1-z^{-1})}^2}\right]
\end{align}
Using 
\eqref{eq:dice_xdef_props},
after some algebra, it can be shown that,
\begin{multline}
\frac{1}{18}[{{n-1}u(n-1)-{n-4}u(n-4)}
{-(n-7)u(n-7)-(n-10)u(n-10)}]\\
\system{Z}\\
\frac{1}{18} \left[{\frac{z^{-2}{1-z^{-3}-z^{-6}-z^{-9}}}{({1-z^{-1}})^2}}\right]
\end{multline}
Hence,
\iffalse
\begin{multline}
p_{X}(n) = \frac{1}{18}[{{n-1}u(n-1)-{n-4}u(n-4)-}\\
              {(n-7)u(n-7)-(n-10)u(n-10)}]
\end{multline}
\fi
\begin{align}
  p_X(n) &= 
  \begin{cases}
  0 & n \le 1
  \\
  \frac{n-1}{18} &  2 \le n \le  4
  \\
  \frac{1}{6} & 5 \le n \le 7
  \\
  \frac{10-n}{18} & 8 \le n \le 9
  \\
  0 & n \ge 10
  \end{cases}
  \end{align}
  \iffalse
hence, the probabilities are,
\begin{align}
  p_X(n) &= 
\begin{cases}
   \frac{1}{18} & n = 2 \\
   \frac{1}{9} & n = 3 \\
   \frac{1}{6} & n = 4 \\
   \frac{1}{6} & n = 5 \\
   \frac{1}{6} & n = 6 \\
   \frac{1}{6} & n = 7 \\
   \frac{1}{9} & n = 8 \\
   \frac{1}{18} & n = 9 
\end{cases}
\end{align}
\fi
See \figref{fig:mgf-sum-2}.
The experiment of rolling the dice was simulated using Python for 10000 samples.  
%
\begin{figure}[H]
\centering
\includegraphics[width=\columnwidth]{exemplar/10/13/3/23/figs/pmf.png}
\caption{Plot of $p_X(n)$.Simulations are close to the analysis. }
\label{fig:mgf-sum-2}
\end{figure}

    \end{enumerate}

	\section{Binomial}
\begin{enumerate}[label=\thesection.\arabic*,ref=\thesection.\theenumi]
\item A die is thrown twice. What is the probability that
\begin{enumerate}[font=\large\bfseries]
	\item 5 will not come up either time? 
	\item 5 will come up at least once?
\end{enumerate}
\solution
\iffalse
\ref{tab:ncert/10/15/1/24/table1}
\begin{table}[h]\centering
	\input{ncert/10/15/1/24/tables/table.tex}
	\caption{}
	\label{tab:ncert/10/15/1/24/table1}
\end{table}	
From 
	\tabref{tab:ncert/10/15/1/24/table1},
the PMF of $X$ is 
\fi
Since $X \sim \text{Binom}\brak{\frac{1}{2}}$, the desired probabilites are
\begin{enumerate}
	\item 
	\begin{align} 
		\pr{X = 0} = \comb{2}{0}  \left(\frac{1}{6}\right)^{0}  \left(\frac{5}{6}\right)^{2}
		=\frac{25}{36}
	\end{align}
	\item 
	\begin{align} 
		\pr{X \geq 1} &= 1 - \pr{X \leq 0}
		=  1 - F_{X}(0)\\
		&= 1 - \frac{25}{36}
		= \frac{11}{36}
	\end{align}
\end{enumerate}
using 
\eqref{eq:cdf-binom}.

  \item Three coins are tossed once. Find the probability of getting 
    \begin{enumerate}
        \item 3 heads
        \item 2 heads
        \item atleast 2 heads 
        \item atmost 2 heads
        \item no head
        \item 3 tails 
        \item exactly two tails
        \item no tail
        \item atmost two tails
    \end{enumerate}
\solution
    The rv representing the given event is
    \begin{align}
	    Y \sim \textrm{Binom}\brak{3,\frac{1}{2}}
        \label{eq:ncert/11/16/3/8/pmf-Y}
    \end{align}
    \begin{enumerate}
        \item 
            \begin{align}
	    p_{Y}\brak{3} 
                         &= \frac{1}{8}
                         \label{eq:ncert/11/16/3/8/ans-i}
            \end{align}

        \item 
            \begin{align}
             p_{Y}\brak{2}  
                         = \frac{3}{8}
            \end{align}
        \item 
            \begin{align}
                \pr{Y\ge2} &= 1 - \pr{Y<2} \\
                           &= F_Y\brak{3} - F_Y\brak{1} \\
                           &= \frac{1}{2}
            \end{align}

        \item 
            \begin{align}
                \pr{Y\le2} &= \sum_{k=0}^2\myvec{n\\k}p^k\brak{1-p}^{n-k} \\
                           &= \frac{7}{8}
            \end{align}

        \item 
\begin{align}
p_X\brak{0}&=\comb{3}{0} \brak{0.5}^3\brak{0.5}^{0}\\
&=\frac{1}{8}
\end{align}

        \item 
            \begin{align}
	    p_{Y}\brak{1} &= \myvec{n\\1}p^1\brak{1-p}^{n-1} \\
                           &= \frac{3}{8}
            \end{align}

\item $p_{Y}\brak{3} = \frac{1}{8}$ from \eqref{eq:ncert/11/16/3/8/ans-i}.
        \item 
		\begin{align}
                \pr{Y\ge1} &= 1 - \pr{Y<1} \\
                           &= 1 - F_Y\brak{0} \\
                           &= \frac{7}{8}
            \end{align}
    \end{enumerate}

	\item 
	A game consists of tossing a one rupee coin 3 times and noting its outcome each time. Hanif wins if all the tosses give the same result i.e., three heads or three tails, and loses otherwise. Calculate the probability that Hanif will lose the game
\\
\solution
%The rv $X\sim \binom{3,\frac{1}{2}}$ is 
Since $X\sim \bnm{3}{\frac{1}{2}}$,  the desired probability is
\begin{align}
\pr{X = 1} + \pr{X = 2} &= \comb{3}{1} \brak{\frac{1}{2}}^{3} + \comb{3}{2} \brak{\frac{1}{2}}^{3}\\
			&= \brak{\frac{3}{4}}
\end{align}




%\solution
From \eqref{eq:axiom-cond}
\begin{align}
\pr{E|F}=&\frac{\pr{E  F}}{\pr{F}}=\frac{0.2}{0.3}=\frac{2}{3}&  
\\
\pr{F|E}=&\frac{\pr{E  F}}{\pr{E}}=\frac{0.2}{0.6}=\frac{1}{3}&  
\end{align}


\item A coin is tossed three times, where. Determine \pr{E \mid F} where
\begin{enumerate}
\item $E$ : head on third toss, $F$ : heads on first two tosses
\item $E$ : at least two heads, $F$ : at most two heads
\item $E$ : at most two tails, $F$ : at least one tail
\end{enumerate}
\solution 
\begin{enumerate}[label=\thesection.\arabic*,ref=\thesection.\theenumi]
	\item An electronic assembley consists of two subsystems,say A and B.From previous testing procedures, the following probabilities are assumed to be known
\begin{align}
\pr{\text{A fails}}&=0.20
\\ \pr{\text{B alone fails}}&=0.15
\\ \pr{\text{A  and B fails}}&=0.15
\end{align}
 Evaluate the following probabilities
 \begin{enumerate}
 \item $\pr{\text{A fails given B has failed}}$
 \item $\pr{\text{A fails alone}}$
\end{enumerate}
		\solution
		From the  given information,
 \begin{align}
 \pr{A^\prime} = 0.20,\   	
\pr{AB^ \prime}=0.15,\ 	
\pr{A ^\prime B ^\prime}=0.15
 \end{align} 
 %
\begin{enumerate}
\item 
\begin{align}
 \pr{A ^\prime|B ^\prime}=\frac{\pr{ A ^\prime B ^\prime }}{\pr{B^\prime}} \label{eq:ncert/12/13/6/15/dot}
\end{align}
From 
\eqref{eq:axiom_sum_A},
\begin{align}
	\pr{B^ \prime}&=
          0.15 +0.15 
=0.30 
\\
	\pr{A^\prime|B ^\prime}&= 0.15/0.30 
 = 0.50
\end{align}
\item 
Similiarly, 
from 
\eqref{eq:axiom_sum_A},
\begin{align}
\pr{B  A ^\prime}&=\pr{A^\prime}-\pr{A ^\prime B ^\prime} 
 = 0.20-0.15  = 0.05
\end{align}
\end{enumerate}


  \item
  In a hostel, 60\% of the students read Hindi newspaper, 40\% read English
newspaper and 20\% read both Hindi and English newspapers. A student is
selected at random.
\begin{enumerate}
\item Find the probability that she reads neither Hindi nor English newspapers.
\item If she reads Hindi newspaper, find the probability that she reads English
newspaper.
\item If she reads English newspaper, find the probability that she reads Hindi
newspaper.
\end{enumerate}
\solution
%See \tabref{tab:variables-events}.
From the given information,
\begin{align}
\pr{A}=\frac{6}{10},\
\pr{B}=\frac{4}{10},\
\pr{AB} 
=\frac{2}{10}
\end{align}
\iffalse
\begin{table}[htb] % h stands for "here", suggesting to LaTeX to place the table near where it's defined
\centering
\begin{tabular}{|c|c|}
\hline
\textbf{Variable} & \textbf{Event} \\
\hline
A & Hindi \\
\hline
B & English \\
\hline
\end{tabular}
\caption{}
\label{tab:variables-events}
\end{table}
\fi
\begin{enumerate}
\item
\begin{align}
\pr{A^{\prime}B^{\prime}}&=\pr{\brak{A+B}^{\prime}}\\
&=1-\pr{A+B}\\
&=1-\brak{\pr{A}+\pr{B}-\pr{AB}}\\
&=1-\brak{\frac{6}{10}+\frac{4}{10}-\frac{2}{10}}
=\frac{2}{10}
\end{align}
\item
\begin{align}
\pr{B|A}&=\frac{\pr{AB}}{\pr{A}}
=\frac{\frac{2}{10}}{\frac{6}{10}}
=\frac{1}{3}
\end{align}
\item 
\begin{align}
\pr{A|B}&=\frac{\pr{AB}}{\pr{B}}
=\frac{\frac{2}{10}}{\frac{4}{10}}
=\frac{1}{2}
\end{align}
\end{enumerate}

	\item Assume that the chances of a patient having a heart attack is $40\%$. It is also assumed that a meditation and yoga course reduce the risk of heart attack by $30\%$ and prescription of certain drug reduces its chances by $25\%$. At a time a patient can choose any one of the two options with equal probabilities. It is given that after going through one of the two options the patient selected at random suffers a heart attack. Find the probability that the patient followed a course of meditation and yoga?
	\\
		\solution
		The given information is summarised in 
\tabref{tab:ncert/12/13/6/13/1}.
\begin{table}[htb]
	\centering
\input{ncert/12/13/6/13/tables/table1.tex}
\caption{}
\label{tab:ncert/12/13/6/13/1}
\end{table}
%
\begin{align}
	\therefore	\pr{A|E_1} &= \pr{A}\cbrak{1-(0.30)}
	 =  0.28
	\label{eq:ncert/12/13/6/13/4}
	\\
	\pr{A|E_2} &= \pr{A}\cbrak{1-(0.25)}
	=  0.30
	\label{eq:ncert/12/13/6/13/5}
\end{align}
From 
%\eqref{eq:ncert/12/13/6/13/4} and \eqref{eq:ncert/12/13/6/13/5},
\eqref{eq:axiom-bayes},
\begin{align}
	\pr{E_1|A} &= \frac{\pr{E_1}\pr{A|E_1}}{\sum_{i=1}^{2}\pr{E_i}\pr{A|E_i}}
	= \frac{\frac{1}{2}\times 0.28}{\frac{1}{2}\times 0.28 + \frac{1}{2}\times 0.30}
	= \frac{14}{29}
\end{align}
which is the desired probability.

	\item A team of medical students doing their internship have to assist during surgeries
at a city hospital. The probabilities of surgeries rated as very-complex, complex,
routine, simple or very-simple are respectively, 0.15, 0.20, 0.31, 0.26, .08. Find
the probabilities that a particular surgery will be rated
\begin{enumerate}
\item complex or very-complex
\item neither very-complex nor very simple
\item routine or complex
\item routine or simple
\end{enumerate}
		\solution
		%The given information is summarised in 
\tabref{tab:exemplar/11/16/3/8/table2}
%
\begin{enumerate}
\item 
\begin{align} 
	\pr{E_1 + E_2} &= \pr{E_1} + \pr{E_2}   \qquad \because E_1E_2=0\\
	&= 0.15 + 0.20 
	= 0.35
\end{align}
%
\item 
\begin{align} 
	\pr{E_1^{\prime} E_5^{\prime}} &=  \pr{\brak{E_1 + E_5}^{\prime}}  \\
	&= 1- \pr{E_1 + E_5}\\
	&= 1- \sbrak{\pr{E_1} + \pr{E_5} }  \qquad \because E_1E_5=0\\
	&= 1- \sbrak{0.15 + 0.08 }
	= 0.77\\
\end{align}
%
\item 
\begin{align} 
	\pr{E_3 + E_2} &= \pr{E_3} + \pr{E_2}   \qquad \because E_3E_2=0\\
	&= 0.31 + 0.20 
	= 0.51
\end{align}
\item 
\begin{align} 
	\pr{E_3 + E_4} &= \pr{E_3} + \pr{E_4}   \qquad \because E_3E_4=0\\
	&= 0.31 + 0.26 
	= 0.57
\end{align}
\end{enumerate}




	\item One urn contains two black balls (labelled B1 and B2) and one white ball. A
	second urn contains one black ball and two white balls (labelled W1 and W2).
	Suppose the following experiment is performed. One of the two urns is chosen
	at random. Next a ball is randomly chosen from the urn. Then a second ball is
	chosen at random from the same urn without replacing the first ball.
	
	\begin{enumerate}
	\item What is the probability that two black balls are chosen?
	
	\item What is the probability that two balls of opposite colour are chosen?
	\end{enumerate}
	\solution
	%\begin{align}
    \label{eq:12.13.6.18.1}
	\because	\pr{A|B} &> \pr{A},\
\frac{\pr{AB}}{\pr{B}} > \pr{A}
\\
    \label{eq:12.13.6.18.2}
	\implies \pr{AB} &> \pr{A}\pr{B}
	\\
	\text{or, } \frac{\pr{AB}}{\pr{A}} &=\pr{B|A} > \pr{A}
\end{align}

  
\item Suppose an integer from 1 through 1000 is chosen at random, find the probability that the integer is a multiple of 2 or a multiple of 9.
%\begin{table}[htb]
        \centering
        \input{exemplar/11/16/3/3/tables/main.tex}
        \caption{}
        \label{tab:11.16.3.3}
\end{table}
See 
        \tabref{tab:11.16.3.3}.
From 
\eqref{eq:axiom_sum_AB},
\begin{align}
\pr{A + B} 
= \frac{500}{1000} + \frac{111}{1000} - \frac{55}{1000}
= \frac{556}{1000}
\end{align}



\item State whether the statement is True or False.\\
The probability that a person visiting a zoo will see the giraffe is 0.72, the probability that he will see the bears is 0.84 and the probability that he will see both is 0.52.\\
\solution
\\
%Let 
\begin{align}
\pr{A} = 0.72,\
\pr{B} = 0.84,\
\pr{AB} = 0.52.
\end{align}
Using
\eqref{eq:axiom_sum_AB},
\begin{align}
\pr{A+B} 
= 0.72 + 0.84 - 0.52
= 1.04
\end{align}
which violates
	\eqref{eq:axiom-pos}.
Hence, false. 











\newpage

%\tableofcontents

\bigskip

\renewcommand{\thefigure}{\theenumi}
\renewcommand{\thetable}{\theenumi}
%\renewcommand{\theequation}{\theenumi}

%\begin{abstract}
%%\boldmath
%In this letter, an algorithm for evaluating the exact analytical bit error rate  (BER)  for the piecewise linear (PL) combiner for  multiple relays is presented. Previous results were available only for upto three relays. The algorithm is unique in the sense that  the actual mathematical expressions, that are prohibitively large, need not be explicitly obtained. The diversity gain due to multiple relays is shown through plots of the analytical BER, well supported by simulations. 
%
%\end{abstract}
% IEEEtran.cls defaults to using nonbold math in the Abstract.
% This preserves the distinction between vectors and scalars. However,
% if the journal you are submitting to favors bold math in the abstract,
% then you can use LaTeX's standard command \boldmath at the very start
% of the abstract to achieve this. Many IEEE journals frown on math
% in the abstract anyway.

% Note that keywords are not normally used for peerreview papers.
%\begin{IEEEkeywords}
%Cooperative diversity, decode and forward, piecewise linear
%\end{IEEEkeywords}



% For peer review papers, you can put extra information on the cover
% page as needed:
% \ifCLASSOPTIONpeerreview
% \begin{center} \bfseries EDICS Category: 3-BBND \end{center}
% \fi
%
% For peerreview papers, this IEEEtran command inserts a page break and
% creates the second title. It will be ignored for other modes.
%\IEEEpeerreviewmaketitle



\item State whether the statement is True or False. The probabilities that a typist will make 0, 1, 2, 3, 4, 5 or more mistakes in typing a report are, respectively, 0.12, 0.25, 0.36, 0.14, 0.08, 0.11.\\
\solution
\\
%From the given information, we obtain the distribution
\begin{align}
p_X(k)&=
\begin{cases}
0.12 & k=0\\
0.25 & k=1\\
0.36 & k=2\\
0.14 & k=3\\
0.08 & k=4\\
0.11 & k\geq5\\
\end{cases}
\end{align}
Since
\begin{align}
\sum_{i=0}^5 p_X(k) = 1.06
&>1
\end{align}
	violates \eqref{eq:axiom-pos},
the given statement is false.



\item If A and B are two candidates seeking admission in an engineering College. The probability that A is selected is 0.5 and the probability that both A and B are selected is atmost 0.3. Is it possible that the probability of B getting selected is 0.7?\\
%\begin{align}
\because \pr{AB} &\leq 0.3
\\
	\text{Let }
\pr{AB} &= 0.1.
\end{align}
From \eqref{eq:axiom_sum_AB}
\begin{align}
\pr{A+B} = 0.5+0.7-0.1 = 1.1 > 1,
\end{align}
which violates
	\eqref{eq:axiom-pos}.  Hence, it is not possible.
%\end{document}



\item Let $E_1$ and $E_2$ be two independent events such that $\pr{E_1} = p_1 $ and $ \pr{E_2} = p_2 $ \ Describe in words the events whose probabilities are: 
\begin{enumerate}
\item \ $p_1 p_2$
\quad\item \ $(1 - p_1) p_2 $
\quad\item \ $1 - (1 - p_1)(1 - p_2)$
\quad\item \ $p_1 + p_2 - 2p_1 p_2$
\end{enumerate}
%\begin{enumerate}
\item \begin{align}
p_1p_2&=\pr{E_1}\pr{E_2}\\
&=\pr{E_1E_2}
\end{align}
So, $E_1$ and $E_2$ occur simultaneously.
\item \begin{align}
(1 - p_1)(p_2) &= \pr{E_1^\prime}\pr{E_2} \\
&= \pr{{E_1^\prime}E_2}
\end{align}
So $E_1$ does not occur but $E_2$  occurs.
\item \begin{align}
1 - (1 - p_1)(1 - p_2) &= 1 - \pr{E_1^\prime} \pr{E_2^\prime} \\
&= 1 - \pr{E_1^\prime \ E_2^\prime} \\
&= \pr{E_1+E_2} 
\end{align}
So, either $E_1$ or $E_2$ or both $E_1$ and $E_2$  occurs.
\item \begin{align}
p_1 + p_2 - 2p_1p_2 &= \pr{E_1} + \pr{E_2} - 2\pr{E_1} \pr{E_2} \\
&=  \pr{E_1}-\pr{E_1}\pr{E_2}+ \pr{E_2} - \pr{E_1} \pr{E_2}\\
&= \pr{E_1}\brak{1-\pr{E_2}}+\pr{E_2} \brak{1-\pr{E_1}}\\
&=\pr{E_1}\pr{E_2^\prime}+\pr{E_2}\pr{E_1^\prime}\\
&=\pr{E_1E_2^\prime+E_1^\prime E_2}
\end{align}
So, either $E_1$ or $E_2$ occurs but not both.
\end{enumerate}




\item Suppose that 6\% of the people with blood group O are left handed and 10\% of those with other blood groups are left handed 30\% of the people have blood group O. If a left handed person is selected at random, what is the probability that he/she will have blood group O?
%Let $A$ represent blood group O and $B$ represent left handedness. 
From the given information,
\begin{align}
\pr{A} = 0.3,\
\pr{B|A} = 0.06,\
\pr{B| A^{\prime}} = 0.1.
\end{align}
Using 
\eqref{eq:axiom-bayes},
\begin{align}
	\pr{A|B} &= \frac{\pr{A}\pr{B|A}}{\pr{A}\pr{B|A}+\pr{A^\prime}\pr{B|A^\prime}}
	 = \frac{9}{44}
\end{align}
upon substituting numerical values.

\item %\solution
\begin{enumerate}
	\item If $E_1$ and $E_2$ are mutually exclusive events, then $E_1E_2 = \phi$.\\
	\item If $E_1$ and $E_2$ are mutually exclusive and exhaustive events, then $ E_1 E_2 = \phi $ and $ E_1 + E_2=S $ \\
	\item If $E_1$ and $E_2$ have common outcomes, this means:
\begin{align}
	E_1E_2&\neq0
\end{align}
Let $E_a$ be the outcomes that are present in $E_1$ and not in $E_2$. So,
\begin{align}
	E_a&=E_1-E_2 \label{eq:exemplar/11/16/3/43/1}
\end{align}
Let $E_b$ be the outcomes common between $E_1$ and $E_2$. So,
\begin{align}
	E_b&=E_1E_2 \label{eq:exemplar/11/16/3/43/2}
\end{align}
So, we can say that
\begin{align}
	E_1&=E_a+E_b
\end{align}
Refering to equation \eqref{eq:exemplar/11/16/3/43/1} and \eqref{eq:exemplar/11/16/3/43/2}:
\begin{align}
	E_1&=(E_1-E_2)+(E_1 E_2)
\end{align}
        \item If $E_1$ and $E_2$ are two events such that $E_1 \subset E_2$, then let E be subset of $E_2$ containing elements other than $E_1$. So,
\begin{align}
	E_1 + E &= E_2 \text{  and  } E_1 E = E_2\label{eq:exemplar/11/16/3/43/3}
\end{align}
Refering to equation \eqref{eq:exemplar/11/16/3/43/3}:
\begin{align}
	E_1 E_2 &= E_1 (E_1+E)\\
	&=(E_1 E_1)+(E_1 E)\\
	&=E_1
\end{align}
\end{enumerate}
	Hence,
\begin{multicols}{4}
\begin{enumerate}
	\item $\leftrightarrow(iv)$,\item $\leftrightarrow(iii)$,\item $\leftrightarrow(ii)$,\item $\leftrightarrow(i)$
\end{enumerate}
\end{multicols}




\item Two Coins are tossed once, where\\
\brak{i} E : Tail appears on one coin,\qquad F : one coin shows head\\
\brak{ii}  E : no tail appears,\qquad\qquad\qquad F : no head appears
Determine \pr{E\mid F}.
%\input{ncert/12/13/1/7a/ncertmain.tex}
\item A fair die is rolled. Consider events $E=1,3,5,\, F=2,3$ and $G=2,3,4,5$. Find
\begin{enumerate}
\item $\pr{E \mid F} \text{ and } \pr{F \mid E}$
\item $\pr{E \mid G} \text{ and } \pr{G \mid E}$
\item $\pr{{E \cup F} \mid G} \text{ and } \pr{{E \cap F} \mid G}$
\end{enumerate}
%
\solution
%		\begin{table}[htb]
		\centering
		\input{ncert/12/13/1/11/tables/Table1.tex}
		\caption{}
		\label{tab:ncert/12/13/1/11/table1}
		\end{table}
	See 	
		\tabref{tab:ncert/12/13/1/11/table1}.
		\begin{enumerate}
		\item
		\begin{align}
			\pr{\cond{E}{F}} = \frac{\pr{EF}}{\pr{F}}
		 = \frac{1/6}{1/3}
		 = 1/2
		\end{align}
		\item
		\begin{align}
		\pr{F|E} = \frac{\pr{EF}}{\pr{E}}
         = \frac{1/6}{1/2}
         = 1/3
        \end{align}
		\item
		\begin{align}
			\pr{E|G} = \frac{\pr{EG}}{\pr{G}}
        = \frac{1/3}{2/3}
        = 1/2
        \end{align}
        \item
        \begin{align}
		\pr{G|E} = \frac{\pr{EG}}{\pr{E}}
        = \frac{1/3}{1/2}
        = 2/3
        \end{align}
        \item
        \begin{align}
		\because	\pr{(E+F)G} &= \pr{EG+FG}= \pr{EG}+\pr{FG}-\pr{EFG},
	\nonumber	\\
		 &= \frac{1}{3} + \frac{1}{3} - \frac{1}{6}
		 = \frac{1}{2}
		 \\
		\pr{(E+F)|G} &= \frac{\pr{(E+F)G}}{\pr{G}}
		 = \frac{1/2}{2/3}
		 = \frac{3}{4}
		\end{align}
		\item
		\begin{align}
			\pr{EF|G} &= \frac{\pr{EFG}}{\pr{G}}
         = \frac{1/6}{2/3}
			= \frac{1}{4}
		\end{align}

\end{enumerate}

\end{enumerate}

\item A die is tossed thrice. Find the probability of getting an odd number at least once.
%		\label{ncert/12/13/2/12}
		\\
	\solution
The equivalent distribution is $X \sim \bnm{3}{\frac{1}{2}}$.  The desired probability is
\begin{align}
	p_X(1)+p_X(3) = 
\end{align}

\item Find the probability distribution of
\begin{enumerate}
	\item number of heads in two tosses of a coin.
	\item number of tails in the simultaneous tosses of three coins.
	\item number of heads in four tosses of a coin.
\end{enumerate}
\solution
The desired probabilities are 
% 
\begin{enumerate} 
\item $X \sim \bnm{2}{\frac{1}{2}}$
 \begin{align}
	 p_{X}(2) = \comb{2}{2}\frac{1}{2}^{2} = \frac{1}{4}
\end{align}
\item $X \sim \bnm{3}{\frac{1}{2}}$
 \begin{align}
	 p_{X}(3) = \comb{2}{0}\frac{1}{2}^{3} = \frac{1}{8}
\end{align}
\item $X \sim \bnm{4}{\frac{1}{2}}$
 \begin{align}
	 p_{X}(4) = \comb{4}{4}\frac{1}{2}^{4} = \frac{1}{16}
\end{align}
\end{enumerate}

\item Find the probability distribution of the number of successes in two tosses of a die,
where a success is defined as\\
\begin{enumerate}
\item number greater than 4
\item six appears on at least one die
\end{enumerate}
\solution
\begin{enumerate}
\item
From \eqref{eq:dice_xiF},  
\begin{align}
\pr{X > 4}&= 1 - F_X\brak{3}
          =\frac{1}{3}
\end{align}
The distribution is then given by 
$Y\sim \bnm{2}{\frac{1}{3}}$ 
%
\item 
	In this case,
from \eqref{eq:dice_pmf_xi},
$p_X(6)=\frac{1}{6}$
yielding the distribution
$Y\sim \bnm{2}{\frac{1}{6}}$ 
\end{enumerate}

\item There are 5\% defective items in a large bulk of items. What is the probability that a sample of 10 items will include not more than one defective item?
	\\
\solution
Since $X \sim \bnm{10}{0.5}$, the desired probability is $F_X(1)$.

\item Five cards are drawn successively with replacement from a well-shuffled deck
of 52 cards. What is the probability that
\begin{enumerate}
    \item all the five cards are spades?
    \item only 3 cards are spades?
    \item none is a spade?
\end{enumerate}
\solution
The probability of getting spade on any draw is
\begin{align}
p = \frac{13}{52} = \frac{1}{4}
\end{align}
The given distribution is $X \sim \bnm{5}{\frac{1}{4}}$.
The desired probabilities are
\begin{enumerate}
	\item  $p_X(5)$
	\item  $p_X(3)$
	\item  $p_X(0)$
\end{enumerate}



\item The probability that a bulb produced by a factory will fuse after 150 days of use
is $0.05$. Find the probability that out of 5 such bulbs
\begin{enumerate}
\item  none
\item not more than one
\item more than one
\item at least one
\end{enumerate}
will fuse after 150 days of use.
\\
\solution
The given distribution can be expressed as $X \sim \bnm{5}{0.05}$.  The desired probabilities are
\begin{enumerate}
	\item     $p_{X}\brak{0}$
	\item     $F_X(1)$
	\item     $\pr{ X > 1}=1 - F_X(2)$
	\item     $\pr{X \ge 1}=1-F_X(0)$
\end{enumerate}

\item A bag consists of 10 balls each marked with one of the digits 0 to 9. If 4 balls are drawn successively with replacement from the bag, what is the probability that none is marked with the digit 0?
\\
\solution
The probability that a ball is marked 0 is the Bernoulli parameter
\begin{align}
	p = 
	 \frac{1}{10}
\end{align}
In the given problem, the relevant Binomial distribution is $X \sim \bnm{4}{\frac{1}{10}}$.
The desired probability is then given by 
\begin{align}
1-p_X(4) 
\end{align}

\item How many times must a man toss a fair coin so that the probability of having at least one head is more than 90\%?
\\
If $n$ be the number of coin tosses, the distribution is $X \sim \bnm{n}{\frac{1}{2}}$, where $X$ represents the number of heads.  The given probability is 
\begin{align}
\pr{X\geq1} > 0.9 \\
\implies 1 - p_{X}(0) > 0.9\\
\text{or, }\brak{2}^n > 10
\\
\therefore n>\log_2(10)\\
\implies n > 3.32 \implies n = 4,
\end{align}
Since $n$ is a positive integer. 

\item In an examination, 20 questions of true-false type are asked. Suppose a student tosses a fair coin to determine his answer to each question. If the coin falls heads, he answer true; if it falls tails, he answer false. Find the probability that he answers at least 12 questions correctly.\\
\\
\solution
Here, $X \sim \bnm{20}{\frac{1}{2}}$. Therefore,
the desired probability is given by 
\begin{align}
	\pr{X>=12} = 1 - F_{X}(11)
	= 0.2517
\end{align}



\item Find the probability of getting 5 twice in 7 throws of a dice.\\
\\
\solution
The Binomial distribution here is 
$ X \sim \bnm{7}{\frac{1}{6}}.  \therefore$ the desired probability is
$p_{X}\brak{2}$.


\item On a multiple choice examination with three possible answers for each of the five questions, what is the probability that a candidate would get four or more correct answers just by guessing$?$
\\
\solution
The relevant distribution here is $X \sim \bnm{5}{\frac{1}{3}}$
The desired probability is
\begin{align}
\pr{X\ge4}&
=1-F_X(3)
=\frac{11}{243}.
\end{align}

\item Find the probability of throwing at most 2 sixes in 6 throws of a single die.\\
\solution
The given distribution is
$
X \sim \bnm{6}{\frac{1}{6}}
    $
and the desired probability is 
\begin{align}
  F_X\brak{2}   = \frac{21875}{23328} 
\end{align}

\item Suppose that 90 \% of people are right-handed. What is the probability that atmost 6 of a random sample of 10 people are right-handed. 
\\
\solution
\begin{table}[H]
	\centering
\begin{tabular}{|c|c|c|}
\hline
Random variable &Value &Definition\\ \hline
\multirow{3}{*}{X} &0 &Slips of Rs 1\\
&1 &Slips of Rs 5\\
&2 &Slips of Rs 13\\ \hline
\multirow{2}{*}{Y} &0 &Box A\\
&1 &Box B\\\hline
\end{tabular}
\caption{}
\label{tab:Distribution}
\end{table}
See \tabref{tab:Distribution}.
\begin{align}
p_{Y}\brak{k}= \begin{cases} 
      \frac{1}{3} & {k=0} \\
      \frac{2}{3 }& {k=1} 
   \end{cases}
   \\
p_{Y|X}\brak{0|0} = \frac{19}{25}\, 
p_{Y|X}\brak{0|1} = \frac{6}{25}\,
p_{Y|X}\brak{1|0} = \frac{45}{50}\,
p_{Y|X}\brak{1|2} = \frac{5}{50}
\end{align}
The desired probability is the probability that a slip drawn at random is marked other than Rs 1,
\begin{align}
&=1-p_X\brak{0}\\
&= p_X(1) + p_X(2)
\end{align}
Using Bayes theorem,
\begin{align}
&= p_Y\brak{0} \times \pr{Y=0 | X=1} + p_Y\brak{1} \times \pr{Y=1|X=2}\\
&=\frac{1}{3} \times \frac{6}{25} + \frac{2}{3} \times \frac{5}{50}\\
&=\frac{11}{75}
\end{align}

\newpage

%\tableofcontents

\bigskip

\renewcommand{\thefigure}{\theenumi}
\renewcommand{\thetable}{\theenumi}
%\renewcommand{\theequation}{\theenumi}

%\begin{abstract}
%%\boldmath
%In this letter, an algorithm for evaluating the exact analytical bit error rate  (BER)  for the piecewise linear (PL) combiner for  multiple relays is presented. Previous results were available only for upto three relays. The algorithm is unique in the sense that  the actual mathematical expressions, that are prohibitively large, need not be explicitly obtained. The diversity gain due to multiple relays is shown through plots of the analytical BER, well supported by simulations. 
%
%\end{abstract}
% IEEEtran.cls defaults to using nonbold math in the Abstract.
% This preserves the distinction between vectors and scalars. However,
% if the journal you are submitting to favors bold math in the abstract,
% then you can use LaTeX's standard command \boldmath at the very start
% of the abstract to achieve this. Many IEEE journals frown on math
% in the abstract anyway.

% Note that keywords are not normally used for peerreview papers.
%\begin{IEEEkeywords}
%Cooperative diversity, decode and forward, piecewise linear
%\end{IEEEkeywords}



% For peer review papers, you can put extra information on the cover
% page as needed:
% \ifCLASSOPTIONpeerreview
% \begin{center} \bfseries EDICS Category: 3-BBND \end{center}
% \fi
%
% For peerreview papers, this IEEEtran command inserts a page break and
% creates the second title. It will be ignored for other modes.
%\IEEEpeerreviewmaketitle




  \item An urn contains 25 balls of which 10 balls bear a mark `X' and the 
    remaining 15 bear a mark `Y'. A ball is drawn at random from the urn, its 
    mark is noted down and it is replaced. If 6 balls are drawn in this way, 
    find the probability that 
    \begin{enumerate}
        \item all will bear `X' mark. 
        \item not more than 2 will bear `Y' mark. 
        \item at least one ball will bear `Y' mark. 
        \item the number of balls with `X' mark and `Y' mark will be equal.
    \end{enumerate}
\solution
The given distribution is $X \sim \bnm{6}{\frac{2}{5}}$.  The desired probabilities are
%
\begin{enumerate}
	\begin{multicols}{2}
	\item 
	$p_{X}\brak{6}$ 
%	
	\item 
$
		\pr{X\geq4} = 1-F_X\brak{3}
		$
%%	
	\item $
		\pr{X < 6} = F_X\brak{5}
		$
%	
	\item 
		$
	p_{X}\brak{3} 
		$
	\end{multicols}
\end{enumerate}



\item An urn contains 5 red and 2 black balls. Two balls are randomly drawn. Let X
represent the number of black balls. What are the possible values of X? Is X a
random variable ? 
\item Find the probability distribution of
\begin{enumerate}
\item number of heads in two tosses of a coin.
\item number of tails in the simultaneous tosses of three coins.
\item number of heads in four tosses of a coin.
\end{enumerate}

\item Find the probability distribution of the number of successes in two tosses of a die,
where a success is defined as
\begin{enumerate}
\item number greater than 4
\item six appears on at least one die
\end{enumerate}
\item From a lot of 30 bulbs which include 6 defectives, a sample of 4 bulbs is drawn
at random with replacement. Find the probability distribution of the number of
defective bulbs.
\item A coin is biased so that the head is 3 times as likely to occur as tail. If the coin is
tossed twice, find the probability distribution of number of tails.
\item A coin is tossed twice, what is the probability that atleast one tail occurs?
\\
\solution
Here, $X \sim \bnm{2}{\frac{1}{2}}$, where $X$ denotes the number of heads.  The desired probability is
$	\pr{X\geq 1}=1-p_X\brak{0}
	= \frac{3}{4}
	$

\item From a lot of 30 bulbs which include 6 defectives, a sample of 4 bulbs is drawn
at random with replacement. Find the probability distribution of the number of
defective bulbs.
 \item Suppose $X$ is a binomial distribution $B\left(6,\frac{1}{2}\right)$. Show that $X=3$ is the most likely outcome.
(Hint : $P(X=3)$ is the maximum among all $P(x_i),x_i=0,1,2,3,4,5,6$)
From the given information, 
\begin{align}
    p_X(k) = \comb{n}{k} p^k (1-p)^{n-k}, \quad 
	n = 6, p =\frac{1}{2}.
\end{align}
yielding
\begin{align}
    p_X(k) &= \comb{n}{k} \brak{\frac{1}{2}}^k \brak{\frac{1}{2}}^{n-k}\\
    &= \comb{n}{k} \brak{\frac{1}{2}}^n
\end{align}
upon substituting for $p$.
For $p_X\brak{k}$ to be maximum, 
\begin{align}
	\comb{n}{k} &\geq \comb{n}{k-1} \label{eq: 1.1.3}\quad \text{and} \\
	\comb{n}{k} &\geq \comb{n}{k+1} \label{eq: 1.1.4}
	\\
\because    \comb{n}{k}&= \frac{n!}{(n-k)!k!},
    \label{eq: 1.1.2}
\end{align}
from \eqref{eq: 1.1.2} and \eqref{eq: 1.1.3},  
\begin{align}
	\frac{n!}{(n-k)!k!} &\geq \frac{n!}{(n-k+1)!(k-1)!}\\
	\implies \frac{n!}{(n-k)!k!} &\geq \frac{n!}{(n-k)!k!}\frac{k}{n-k+1}\\
	\implies 1 &\geq \frac{k}{n-k+1}\\
	\therefore k &\leq \frac{n+1}{2} \label{eq: 1.1.5}
\end{align}
From \eqref{eq: 1.1.2} and \eqref{eq: 1.1.4},
\begin{align}
	\frac{n!}{(n-k)!k!} &\geq \frac{n!}{(n-k-1)!(k+1)!}\\
	\implies \frac{n!}{(n-k)!k!} &\geq \frac{n!}{(n-k)!k!}\frac{n-k}{k+1}\\
	\implies 1 &\geq \frac{n-k}{k+1}\\
	\therefore k &\geq \frac{n-1}{2}  \label{eq: 1.1.6}
\end{align} 
Thus, from \eqref{eq: 1.1.5} and \eqref{eq: 1.1.6},
\begin{align}
	\frac{n-1}{2} &\leq k \leq \frac{n+1}{2} \label{eq: Final}
    \\
    \implies
    k &=
    \begin{cases}
        \frac{n}{2}, &  n \text{ even} \\
        \frac{n+1}{2} \text{ or } \frac{n-1}{2}, &  n \text{ odd} 
    \end{cases}
\end{align}
Since
\begin{align}
   	n=6,
   	 k=\frac{n}{2}
   	=3
\end{align}
See \figref{fig:Triangle}.
\begin{figure}[H]
\centering
\includegraphics[width=\columnwidth]{ncert/12/13/5/8/figs/figure1.png}
\caption{}
\label{fig:Triangle}
\end{figure}

\item A fair coin is tossed four times, and a person win Re $1$ for each head and lose Rs $1.5$ for each tail that turns up.\\
From the sample space calculate how many different amounts of money you can have after four tosses and the probability of having each of these amounts.
\\
\solution
\input{ncert/11/16/3/7a/Q11.16.3.7.tex}
\item It is known that 10 $\%$ of certain articles manufactured are defective. What is the probability that in a random sample space of 12 such articles,9 are defective? \\
\solution
\begin{table}[H]
	\centering
\begin{tabular}{|c|c|c|}
\hline
Random variable &Value &Definition\\ \hline
\multirow{3}{*}{X} &0 &Slips of Rs 1\\
&1 &Slips of Rs 5\\
&2 &Slips of Rs 13\\ \hline
\multirow{2}{*}{Y} &0 &Box A\\
&1 &Box B\\\hline
\end{tabular}
\caption{}
\label{tab:Distribution}
\end{table}
See \tabref{tab:Distribution}.
\begin{align}
p_{Y}\brak{k}= \begin{cases} 
      \frac{1}{3} & {k=0} \\
      \frac{2}{3 }& {k=1} 
   \end{cases}
   \\
p_{Y|X}\brak{0|0} = \frac{19}{25}\, 
p_{Y|X}\brak{0|1} = \frac{6}{25}\,
p_{Y|X}\brak{1|0} = \frac{45}{50}\,
p_{Y|X}\brak{1|2} = \frac{5}{50}
\end{align}
The desired probability is the probability that a slip drawn at random is marked other than Rs 1,
\begin{align}
&=1-p_X\brak{0}\\
&= p_X(1) + p_X(2)
\end{align}
Using Bayes theorem,
\begin{align}
&= p_Y\brak{0} \times \pr{Y=0 | X=1} + p_Y\brak{1} \times \pr{Y=1|X=2}\\
&=\frac{1}{3} \times \frac{6}{25} + \frac{2}{3} \times \frac{5}{50}\\
&=\frac{11}{75}
\end{align}

\newpage

%\tableofcontents

\bigskip

\renewcommand{\thefigure}{\theenumi}
\renewcommand{\thetable}{\theenumi}
%\renewcommand{\theequation}{\theenumi}

%\begin{abstract}
%%\boldmath
%In this letter, an algorithm for evaluating the exact analytical bit error rate  (BER)  for the piecewise linear (PL) combiner for  multiple relays is presented. Previous results were available only for upto three relays. The algorithm is unique in the sense that  the actual mathematical expressions, that are prohibitively large, need not be explicitly obtained. The diversity gain due to multiple relays is shown through plots of the analytical BER, well supported by simulations. 
%
%\end{abstract}
% IEEEtran.cls defaults to using nonbold math in the Abstract.
% This preserves the distinction between vectors and scalars. However,
% if the journal you are submitting to favors bold math in the abstract,
% then you can use LaTeX's standard command \boldmath at the very start
% of the abstract to achieve this. Many IEEE journals frown on math
% in the abstract anyway.

% Note that keywords are not normally used for peerreview papers.
%\begin{IEEEkeywords}
%Cooperative diversity, decode and forward, piecewise linear
%\end{IEEEkeywords}



% For peer review papers, you can put extra information on the cover
% page as needed:
% \ifCLASSOPTIONpeerreview
% \begin{center} \bfseries EDICS Category: 3-BBND \end{center}
% \fi
%
% For peerreview papers, this IEEEtran command inserts a page break and
% creates the second title. It will be ignored for other modes.
%\IEEEpeerreviewmaketitle




\item A coin is tossed two times. Find the probability of getting at most one head.
\\
\solution
In this case, the distribution is $X \sim \bnm{2}{\frac{1}{2}}$, where $X$ represents a head.
The desired probability is 
$
F_X(1) 
=\frac{3}{4}
$

\item An experiment succeeds twice as often as it fails. Find the probability that in the next
six trials, there will be atleast 4 successes.
\solution
The success distribution is given by
$X \sim \bnm{6}{\frac{2}{3}}$.
Thus, the desired probability is
$\pr{X \ge 4}= 
1-F_X(3)
$

\item A die is thrown 5 times. Find the probability that an odd number will come up exactly three times.
		\solution
	    The odd number proability is $p = \frac{1}{2}.  \therefore
X \sim \bnm{5}{\frac{1}{2}}$.  Thus, the desired probability is
                $ p_Y\brak{3}$.


\item A coin is tossed 3 times. List the possible outcomes. Find the probability of getting
(i) all heads (ii) at least 2 heads \\
\solution
In this case, 
$X \sim \bnm{3}{\frac{1}{2}}$.
\begin{enumerate}
\item To get all heads, the probability is 
$p_{X}\brak{3}$.
\item To get atleast 2 heads, the desired probability is
$\pr{Z \geq 2} 
= F_Z\brak{1}
$
\end{enumerate}


\item Ten coins are tossed. What is the probability of getting atleast 8 heads?
\\
Here, $X \sim \bnm{10}{\frac{1}{2}}$.
The desired probability is
\begin{align}
	\pr{X \ge 8} = 
F_X(10)-F_X(7) 
=\frac{7}{128}
\end{align}

\item A factory produces bulbs. The probability that any one bulb is defective is $\frac{1}{50}$ and they are packed in boxes of 10. From a single box, find the probability that
\begin{enumerate}
\item none of the bulb is defective 
\item exactly two bulbs are defective 
\item more than 8 bulbs are working properly
\end{enumerate}
\solution
The defective bulb distribution is $X \sim \bnm{10}{\frac{1}{50}}$. The desired probabilities are
\begin{enumerate}
\item 
      $ p_X\brak{0}$ 
\item 
$ p_X\brak{2}$ 
\item 
    $\Pr({X\le1}) = F_X(1)$
            \end{enumerate}



\item A lot of 100 watches is known to have 10 defective watches. If 8 watches are
selected (one by one with replacement) at random, what is the probability that
there will be at least one defective watch?
\\	The defective distribution is $X \sim \bnm{8}{\frac{10}{100}}$.
The desired probability is
	$ F_Y(8)-F_Y(0)$.

    \section{Triangular}
\begin{enumerate}[label=\thesection.\arabic*,ref=\thesection.\theenumi]
	\item Two dice, one blue and one grey, are thrown at the same time.   The event defined by the sum of the two numbers appearing on the top of the dice can have 11 possible outcomes 2, 3, 4, 5, 6, 6, 8, 9, 10, 11 and 12.  A student argues that each of these outcomes has a probability $\frac{1}{11}$.  Do you agree with this argument?  Justify your answer.
		%
    \end{enumerate}
    \section{Miscellaneous}
\begin{enumerate}[label=\thesection.\arabic*,ref=\thesection.\theenumi]
\item  The random variable $X$ has a probability distribution \pr{X} of the following form, where $k$ is some number 
\begin{align}
  \pr{X} =
    \begin{cases}
      k,  & x=0\\
      2k, & x=1\\
      3k, & x=2\\
      0 , & \text{otherwise}
    \end{cases}       
\end{align}
		\begin{enumerate}
			\item
 Determine the value of $k$ 

\item  Find \pr{X < 2},\pr{X \leq 2},\pr{X \geq 2}  
		\end{enumerate}
\solution
The desired probabilities are 
% 
\begin{enumerate} 
\item $X \sim \bnm{2}{\frac{1}{2}}$
 \begin{align}
	 p_{X}(2) = \comb{2}{2}\frac{1}{2}^{2} = \frac{1}{4}
\end{align}
\item $X \sim \bnm{3}{\frac{1}{2}}$
 \begin{align}
	 p_{X}(3) = \comb{2}{0}\frac{1}{2}^{3} = \frac{1}{8}
\end{align}
\item $X \sim \bnm{4}{\frac{1}{2}}$
 \begin{align}
	 p_{X}(4) = \comb{4}{4}\frac{1}{2}^{4} = \frac{1}{16}
\end{align}
\end{enumerate}

\item State which of the following are not the probability distributions of a random 
variable. Give reasons for your answer
\renewcommand{\labelenumii}{\roman{enumii}}
\begin{enumerate}

\item \begin{table}[ht!]\centering
\input{ncert/13/4/tables/Book.tex}
\end{table}

\item \begin{table}[ht!]\centering
\input{ncert/13/4/tables/Book2.tex}
\end{table}

\item  \begin{table}[ht!]\centering
\input{ncert/13/4/tables/Book3.tex}	
\end{table}

\item  \begin{table}[ht!]\centering
\input{ncert/13/4/tables/Book5.tex}	
\end{table} 


\end{enumerate}
\item A random variable X has the following probability distribution\\

Determine

\begin{enumerate}
\begin{table}[ht!]\centering
\input{ncert/13/4/tables/Book10.tex}
\end{table}
\item k
\item P$(X < 3)$
\item P$(X > 6)$
\item P$(0 < X < 3)$

\end{enumerate}

\item The random variable X has a probability distribution P(X) of the following form,
where k is some number :
\[P(x)=\begin{cases}
k, & \mbox{if}\; x= 0\\
2k, & \mbox{if}\; x= 1\\
3k, & \mbox{if}\; x= 2\\
0, & otherwise
\end{cases}\]
\begin{enumerate}
\item Determine the value of k.
\item Find P $(X < 2)$, P $(X \leq 2)$, P$(X \geq 2)$
\end{enumerate}
\item
A game consists of spinning an arrow which comes to rest pointing at one of the regions (1, 2 or 3) (Fig. 13.1). Are the outcomes 1, 2 and 3 equally likely to occur? Give reasons.\\
\begin{figure}[!ht]
	\begin{center}
		
		\resizebox{\columnwidth}{!}{\input{exemplar/10/13/2/6/figs/circle.tex}}
	\end{center}
	\caption{Fig.13.1}
	\label{fig:circle.tex}	
\end{figure}\\
\solution
\begin{table}[H]
	\centering
\begin{tabular}{|c|c|c|}
\hline
Random variable &Value &Definition\\ \hline
\multirow{3}{*}{X} &0 &Slips of Rs 1\\
&1 &Slips of Rs 5\\
&2 &Slips of Rs 13\\ \hline
\multirow{2}{*}{Y} &0 &Box A\\
&1 &Box B\\\hline
\end{tabular}
\caption{}
\label{tab:Distribution}
\end{table}
See \tabref{tab:Distribution}.
\begin{align}
p_{Y}\brak{k}= \begin{cases} 
      \frac{1}{3} & {k=0} \\
      \frac{2}{3 }& {k=1} 
   \end{cases}
   \\
p_{Y|X}\brak{0|0} = \frac{19}{25}\, 
p_{Y|X}\brak{0|1} = \frac{6}{25}\,
p_{Y|X}\brak{1|0} = \frac{45}{50}\,
p_{Y|X}\brak{1|2} = \frac{5}{50}
\end{align}
The desired probability is the probability that a slip drawn at random is marked other than Rs 1,
\begin{align}
&=1-p_X\brak{0}\\
&= p_X(1) + p_X(2)
\end{align}
Using Bayes theorem,
\begin{align}
&= p_Y\brak{0} \times \pr{Y=0 | X=1} + p_Y\brak{1} \times \pr{Y=1|X=2}\\
&=\frac{1}{3} \times \frac{6}{25} + \frac{2}{3} \times \frac{5}{50}\\
&=\frac{11}{75}
\end{align}

\newpage

%\tableofcontents

\bigskip

\renewcommand{\thefigure}{\theenumi}
\renewcommand{\thetable}{\theenumi}
%\renewcommand{\theequation}{\theenumi}

%\begin{abstract}
%%\boldmath
%In this letter, an algorithm for evaluating the exact analytical bit error rate  (BER)  for the piecewise linear (PL) combiner for  multiple relays is presented. Previous results were available only for upto three relays. The algorithm is unique in the sense that  the actual mathematical expressions, that are prohibitively large, need not be explicitly obtained. The diversity gain due to multiple relays is shown through plots of the analytical BER, well supported by simulations. 
%
%\end{abstract}
% IEEEtran.cls defaults to using nonbold math in the Abstract.
% This preserves the distinction between vectors and scalars. However,
% if the journal you are submitting to favors bold math in the abstract,
% then you can use LaTeX's standard command \boldmath at the very start
% of the abstract to achieve this. Many IEEE journals frown on math
% in the abstract anyway.

% Note that keywords are not normally used for peerreview papers.
%\begin{IEEEkeywords}
%Cooperative diversity, decode and forward, piecewise linear
%\end{IEEEkeywords}



% For peer review papers, you can put extra information on the cover
% page as needed:
% \ifCLASSOPTIONpeerreview
% \begin{center} \bfseries EDICS Category: 3-BBND \end{center}
% \fi
%
% For peerreview papers, this IEEEtran command inserts a page break and
% creates the second title. It will be ignored for other modes.
%\IEEEpeerreviewmaketitle




\item Apoorv throws two dice once and computes the product of the numbers appearing on the dice. Peehu throws one die and squares the number that appears on it. Who has the better chance of getting the number 36? Why?\\
\solution
\begin{table}[!ht]
\input{exemplar/12/13/3/41/tables/table.tex}
\caption{random variables of objects}
\label{tab:exemplar 12.13.3.41}
\end{table}
\begin{align}
\pr{X=i}=
\begin{cases}
\frac{1}{6},\text{when i=1}\\
\frac{2}{6},\text{when i=2}\\
\frac{3}{6},\text{when i=3}
\end{cases}
\end{align}

we know that that the conditional probability is defined as

                     $\pr{A|B}=\frac{\pr{A,B}}{\pr{B}}$


\begin{enumerate}
\item
The probability that a red ball will be selected is:
\begin{align}
\pr{Y=1}&=\pr{Y=1,X=1}+\pr{Y=1,X=2}+\pr{Y=1,X=3}\\
&=\pr{X=1}\times\pr{Y=1|X=1}+\pr{X=2}\times\pr{Y=1|X=2}+\pr{X=3}\times\pr{Y=1|X=3}\\
&=\frac{1}{6}\times\frac{3}{3}+\frac{2}{6}\times\frac{2}{3}+\frac{3}{6}\times0\\
&=\frac{7}{18}
\end{align}
\item
The probability that a white ball will be selected is:
\begin{align}
\pr{Y=0}&=\pr{Y=0,X=1}+\pr{Y=0,X=2}+\pr{Y=0,X=3}\\
&=\pr{X=1}\times\pr{Y=0|X=1}+\pr{X=2}\times\pr{Y=0|X=2}+\pr{X=3}\times\pr{Y=0|X=3}\\
&=\frac{1}{6}\times0+\frac{2}{6}\times\frac{1}{3}+\frac{3}{6}\times\frac{3}{3}\\
&=\frac{11}{18}
\end{align}
\end{enumerate}




\item 6 boys and 6 girls sit in a row at random. The probability that all the girls sit
together is
\begin{enumerate}
	\item $\frac{1}{432}$
	\item $\frac{12}{431}$
	\item $\frac{1}{132}$
	\item none of the above 
\end{enumerate}
			\begin{table}[H]
	\centering
\begin{tabular}{|c|c|c|}
\hline
Random variable &Value &Definition\\ \hline
\multirow{3}{*}{X} &0 &Slips of Rs 1\\
&1 &Slips of Rs 5\\
&2 &Slips of Rs 13\\ \hline
\multirow{2}{*}{Y} &0 &Box A\\
&1 &Box B\\\hline
\end{tabular}
\caption{}
\label{tab:Distribution}
\end{table}
See \tabref{tab:Distribution}.
\begin{align}
p_{Y}\brak{k}= \begin{cases} 
      \frac{1}{3} & {k=0} \\
      \frac{2}{3 }& {k=1} 
   \end{cases}
   \\
p_{Y|X}\brak{0|0} = \frac{19}{25}\, 
p_{Y|X}\brak{0|1} = \frac{6}{25}\,
p_{Y|X}\brak{1|0} = \frac{45}{50}\,
p_{Y|X}\brak{1|2} = \frac{5}{50}
\end{align}
The desired probability is the probability that a slip drawn at random is marked other than Rs 1,
\begin{align}
&=1-p_X\brak{0}\\
&= p_X(1) + p_X(2)
\end{align}
Using Bayes theorem,
\begin{align}
&= p_Y\brak{0} \times \pr{Y=0 | X=1} + p_Y\brak{1} \times \pr{Y=1|X=2}\\
&=\frac{1}{3} \times \frac{6}{25} + \frac{2}{3} \times \frac{5}{50}\\
&=\frac{11}{75}
\end{align}

\newpage

%\tableofcontents

\bigskip

\renewcommand{\thefigure}{\theenumi}
\renewcommand{\thetable}{\theenumi}
%\renewcommand{\theequation}{\theenumi}

%\begin{abstract}
%%\boldmath
%In this letter, an algorithm for evaluating the exact analytical bit error rate  (BER)  for the piecewise linear (PL) combiner for  multiple relays is presented. Previous results were available only for upto three relays. The algorithm is unique in the sense that  the actual mathematical expressions, that are prohibitively large, need not be explicitly obtained. The diversity gain due to multiple relays is shown through plots of the analytical BER, well supported by simulations. 
%
%\end{abstract}
% IEEEtran.cls defaults to using nonbold math in the Abstract.
% This preserves the distinction between vectors and scalars. However,
% if the journal you are submitting to favors bold math in the abstract,
% then you can use LaTeX's standard command \boldmath at the very start
% of the abstract to achieve this. Many IEEE journals frown on math
% in the abstract anyway.

% Note that keywords are not normally used for peerreview papers.
%\begin{IEEEkeywords}
%Cooperative diversity, decode and forward, piecewise linear
%\end{IEEEkeywords}



% For peer review papers, you can put extra information on the cover
% page as needed:
% \ifCLASSOPTIONpeerreview
% \begin{center} \bfseries EDICS Category: 3-BBND \end{center}
% \fi
%
% For peerreview papers, this IEEEtran command inserts a page break and
% creates the second title. It will be ignored for other modes.
%\IEEEpeerreviewmaketitle




\item A card is selected from a deck of 52 cards. The probability of its being a red face card is
\begin{table}[H]
	\centering
\begin{tabular}{|c|c|c|}
\hline
Random variable &Value &Definition\\ \hline
\multirow{3}{*}{X} &0 &Slips of Rs 1\\
&1 &Slips of Rs 5\\
&2 &Slips of Rs 13\\ \hline
\multirow{2}{*}{Y} &0 &Box A\\
&1 &Box B\\\hline
\end{tabular}
\caption{}
\label{tab:Distribution}
\end{table}
See \tabref{tab:Distribution}.
\begin{align}
p_{Y}\brak{k}= \begin{cases} 
      \frac{1}{3} & {k=0} \\
      \frac{2}{3 }& {k=1} 
   \end{cases}
   \\
p_{Y|X}\brak{0|0} = \frac{19}{25}\, 
p_{Y|X}\brak{0|1} = \frac{6}{25}\,
p_{Y|X}\brak{1|0} = \frac{45}{50}\,
p_{Y|X}\brak{1|2} = \frac{5}{50}
\end{align}
The desired probability is the probability that a slip drawn at random is marked other than Rs 1,
\begin{align}
&=1-p_X\brak{0}\\
&= p_X(1) + p_X(2)
\end{align}
Using Bayes theorem,
\begin{align}
&= p_Y\brak{0} \times \pr{Y=0 | X=1} + p_Y\brak{1} \times \pr{Y=1|X=2}\\
&=\frac{1}{3} \times \frac{6}{25} + \frac{2}{3} \times \frac{5}{50}\\
&=\frac{11}{75}
\end{align}

\newpage

%\tableofcontents

\bigskip

\renewcommand{\thefigure}{\theenumi}
\renewcommand{\thetable}{\theenumi}
%\renewcommand{\theequation}{\theenumi}

%\begin{abstract}
%%\boldmath
%In this letter, an algorithm for evaluating the exact analytical bit error rate  (BER)  for the piecewise linear (PL) combiner for  multiple relays is presented. Previous results were available only for upto three relays. The algorithm is unique in the sense that  the actual mathematical expressions, that are prohibitively large, need not be explicitly obtained. The diversity gain due to multiple relays is shown through plots of the analytical BER, well supported by simulations. 
%
%\end{abstract}
% IEEEtran.cls defaults to using nonbold math in the Abstract.
% This preserves the distinction between vectors and scalars. However,
% if the journal you are submitting to favors bold math in the abstract,
% then you can use LaTeX's standard command \boldmath at the very start
% of the abstract to achieve this. Many IEEE journals frown on math
% in the abstract anyway.

% Note that keywords are not normally used for peerreview papers.
%\begin{IEEEkeywords}
%Cooperative diversity, decode and forward, piecewise linear
%\end{IEEEkeywords}



% For peer review papers, you can put extra information on the cover
% page as needed:
% \ifCLASSOPTIONpeerreview
% \begin{center} \bfseries EDICS Category: 3-BBND \end{center}
% \fi
%
% For peerreview papers, this IEEEtran command inserts a page break and
% creates the second title. It will be ignored for other modes.
%\IEEEpeerreviewmaketitle




\item A die is loaded in such a way that each odd number is twice as likely to occur as each even number. Find P(G), where G is the event that a number greater than 3 occurs on a single roll of the die.\\
\begin{table}[H]
	\centering
\begin{tabular}{|c|c|c|}
\hline
Random variable &Value &Definition\\ \hline
\multirow{3}{*}{X} &0 &Slips of Rs 1\\
&1 &Slips of Rs 5\\
&2 &Slips of Rs 13\\ \hline
\multirow{2}{*}{Y} &0 &Box A\\
&1 &Box B\\\hline
\end{tabular}
\caption{}
\label{tab:Distribution}
\end{table}
See \tabref{tab:Distribution}.
\begin{align}
p_{Y}\brak{k}= \begin{cases} 
      \frac{1}{3} & {k=0} \\
      \frac{2}{3 }& {k=1} 
   \end{cases}
   \\
p_{Y|X}\brak{0|0} = \frac{19}{25}\, 
p_{Y|X}\brak{0|1} = \frac{6}{25}\,
p_{Y|X}\brak{1|0} = \frac{45}{50}\,
p_{Y|X}\brak{1|2} = \frac{5}{50}
\end{align}
The desired probability is the probability that a slip drawn at random is marked other than Rs 1,
\begin{align}
&=1-p_X\brak{0}\\
&= p_X(1) + p_X(2)
\end{align}
Using Bayes theorem,
\begin{align}
&= p_Y\brak{0} \times \pr{Y=0 | X=1} + p_Y\brak{1} \times \pr{Y=1|X=2}\\
&=\frac{1}{3} \times \frac{6}{25} + \frac{2}{3} \times \frac{5}{50}\\
&=\frac{11}{75}
\end{align}

\newpage

%\tableofcontents

\bigskip

\renewcommand{\thefigure}{\theenumi}
\renewcommand{\thetable}{\theenumi}
%\renewcommand{\theequation}{\theenumi}

%\begin{abstract}
%%\boldmath
%In this letter, an algorithm for evaluating the exact analytical bit error rate  (BER)  for the piecewise linear (PL) combiner for  multiple relays is presented. Previous results were available only for upto three relays. The algorithm is unique in the sense that  the actual mathematical expressions, that are prohibitively large, need not be explicitly obtained. The diversity gain due to multiple relays is shown through plots of the analytical BER, well supported by simulations. 
%
%\end{abstract}
% IEEEtran.cls defaults to using nonbold math in the Abstract.
% This preserves the distinction between vectors and scalars. However,
% if the journal you are submitting to favors bold math in the abstract,
% then you can use LaTeX's standard command \boldmath at the very start
% of the abstract to achieve this. Many IEEE journals frown on math
% in the abstract anyway.

% Note that keywords are not normally used for peerreview papers.
%\begin{IEEEkeywords}
%Cooperative diversity, decode and forward, piecewise linear
%\end{IEEEkeywords}



% For peer review papers, you can put extra information on the cover
% page as needed:
% \ifCLASSOPTIONpeerreview
% \begin{center} \bfseries EDICS Category: 3-BBND \end{center}
% \fi
%
% For peerreview papers, this IEEEtran command inserts a page break and
% creates the second title. It will be ignored for other modes.
%\IEEEpeerreviewmaketitle





\item Determine the probability $p$,for each of following events.
\begin{enumerate}
\item An odd number appears in a single roll of dice.
\item Atleast one head appears in two tosses of fair coin.
\item A king,9 of hearts or 3 of spades appears in drawing a single card from a well shuffled deck of 52 cards.
\item The sum of 6 appears in single toss of a pair of fair dice.
\end{enumerate}
\begin{table}[!ht]
\input{exemplar/12/13/3/41/tables/table.tex}
\caption{random variables of objects}
\label{tab:exemplar 12.13.3.41}
\end{table}
\begin{align}
\pr{X=i}=
\begin{cases}
\frac{1}{6},\text{when i=1}\\
\frac{2}{6},\text{when i=2}\\
\frac{3}{6},\text{when i=3}
\end{cases}
\end{align}

we know that that the conditional probability is defined as

                     $\pr{A|B}=\frac{\pr{A,B}}{\pr{B}}$


\begin{enumerate}
\item
The probability that a red ball will be selected is:
\begin{align}
\pr{Y=1}&=\pr{Y=1,X=1}+\pr{Y=1,X=2}+\pr{Y=1,X=3}\\
&=\pr{X=1}\times\pr{Y=1|X=1}+\pr{X=2}\times\pr{Y=1|X=2}+\pr{X=3}\times\pr{Y=1|X=3}\\
&=\frac{1}{6}\times\frac{3}{3}+\frac{2}{6}\times\frac{2}{3}+\frac{3}{6}\times0\\
&=\frac{7}{18}
\end{align}
\item
The probability that a white ball will be selected is:
\begin{align}
\pr{Y=0}&=\pr{Y=0,X=1}+\pr{Y=0,X=2}+\pr{Y=0,X=3}\\
&=\pr{X=1}\times\pr{Y=0|X=1}+\pr{X=2}\times\pr{Y=0|X=2}+\pr{X=3}\times\pr{Y=0|X=3}\\
&=\frac{1}{6}\times0+\frac{2}{6}\times\frac{1}{3}+\frac{3}{6}\times\frac{3}{3}\\
&=\frac{11}{18}
\end{align}
\end{enumerate}




\item Determine the probability p, for each of the following events.\\
(a) An odd number appears in a single toss of a fair die.\\
(b) At least one head appears in two tosses of a fair coin.\\
(c) A king, 9 of hearts, or 3 of spades appears in drawing a single card from a
well shuffled ordinary deck of 52 cards.\\
(d) The sum of 6 appears in a single toss of a pair of fair dice.
\begin{table}[H]
	\centering
\begin{tabular}{|c|c|c|}
\hline
Random variable &Value &Definition\\ \hline
\multirow{3}{*}{X} &0 &Slips of Rs 1\\
&1 &Slips of Rs 5\\
&2 &Slips of Rs 13\\ \hline
\multirow{2}{*}{Y} &0 &Box A\\
&1 &Box B\\\hline
\end{tabular}
\caption{}
\label{tab:Distribution}
\end{table}
See \tabref{tab:Distribution}.
\begin{align}
p_{Y}\brak{k}= \begin{cases} 
      \frac{1}{3} & {k=0} \\
      \frac{2}{3 }& {k=1} 
   \end{cases}
   \\
p_{Y|X}\brak{0|0} = \frac{19}{25}\, 
p_{Y|X}\brak{0|1} = \frac{6}{25}\,
p_{Y|X}\brak{1|0} = \frac{45}{50}\,
p_{Y|X}\brak{1|2} = \frac{5}{50}
\end{align}
The desired probability is the probability that a slip drawn at random is marked other than Rs 1,
\begin{align}
&=1-p_X\brak{0}\\
&= p_X(1) + p_X(2)
\end{align}
Using Bayes theorem,
\begin{align}
&= p_Y\brak{0} \times \pr{Y=0 | X=1} + p_Y\brak{1} \times \pr{Y=1|X=2}\\
&=\frac{1}{3} \times \frac{6}{25} + \frac{2}{3} \times \frac{5}{50}\\
&=\frac{11}{75}
\end{align}

\newpage

%\tableofcontents

\bigskip

\renewcommand{\thefigure}{\theenumi}
\renewcommand{\thetable}{\theenumi}
%\renewcommand{\theequation}{\theenumi}

%\begin{abstract}
%%\boldmath
%In this letter, an algorithm for evaluating the exact analytical bit error rate  (BER)  for the piecewise linear (PL) combiner for  multiple relays is presented. Previous results were available only for upto three relays. The algorithm is unique in the sense that  the actual mathematical expressions, that are prohibitively large, need not be explicitly obtained. The diversity gain due to multiple relays is shown through plots of the analytical BER, well supported by simulations. 
%
%\end{abstract}
% IEEEtran.cls defaults to using nonbold math in the Abstract.
% This preserves the distinction between vectors and scalars. However,
% if the journal you are submitting to favors bold math in the abstract,
% then you can use LaTeX's standard command \boldmath at the very start
% of the abstract to achieve this. Many IEEE journals frown on math
% in the abstract anyway.

% Note that keywords are not normally used for peerreview papers.
%\begin{IEEEkeywords}
%Cooperative diversity, decode and forward, piecewise linear
%\end{IEEEkeywords}



% For peer review papers, you can put extra information on the cover
% page as needed:
% \ifCLASSOPTIONpeerreview
% \begin{center} \bfseries EDICS Category: 3-BBND \end{center}
% \fi
%
% For peerreview papers, this IEEEtran command inserts a page break and
% creates the second title. It will be ignored for other modes.
%\IEEEpeerreviewmaketitle




\end{enumerate}
    \section{Gaussian}
\begin{enumerate}[label=\thesection.\arabic*,ref=\thesection.\theenumi]
\item There are 5\% defective items in a large bulk of items. What is the probability that a sample of 10 items will include not more than one defective item?
	\\
\solution
%The given distribution is
$
X \sim \bnm{10}{\frac{9}{10}}
    $
and the desired probability is 
$
  F_X\brak{6} 
  $.

\item Five cards are drawn successively with replacement from a well-shuffled deck
of 52 cards. What is the probability that
\begin{enumerate}
    \item all the five cards are spades?
    \item only 3 cards are spades?
    \item none is a spade?
\end{enumerate}
\solution
%The probability of getting spade on any draw is
\begin{align}
p = \frac{13}{52} = \frac{1}{4}
\end{align}
The given distribution is $X \sim \bnm{5}{\frac{1}{4}}$.
The desired probabilities are
\begin{enumerate}
	\item  $p_X(5)$
	\item  $p_X(3)$
	\item  $p_X(0)$
\end{enumerate}



\item In an examination, 20 questions of true-false type are asked. Suppose a student tosses a fair coin to determine his answer to each question. If the coin falls heads, he answer true; if it falls tails, he answer false. Find the probability that he answers at least 12 questions correctly.\\
\\
\solution
%Here, $X \sim \bnm{20}{\frac{1}{2}}$. Therefore,
the desired probability is given by 
\begin{align}
	\pr{X>=12} = 1 - F_{X}(11)
	= 0.2517
\end{align}



\item It is known that 10 $\%$ of certain articles manufactured are defective. What is the probability that in a random sample space of 12 such articles,9 are defective? \\
\solution
The given distribution is
$
X \sim \bnm{10}{\frac{9}{10}}
    $
and the desired probability is 
$
  F_X\brak{6} 
  $.

\item The probability that a student is not a swimmer is $\frac{1}{5}$ .Then the probability that out of five students, four are swimmers 
\begin{enumerate}
    \item $\comb{5}{4}\brak{\frac{4}{5}}^{4}\frac{1}{5}$ \label{item:12/13/5/15/1} \\
    \item $\brak{\frac{4}{5}}^{4}\frac{1}{5}$ \label{item:12/13/5/15/2}\\
    \item \label{3}$\comb{5}{1}\frac{1}{5}\brak{\frac{4}{5}}^{4}$\\
    \item None of these
\end{enumerate}
\solution
The given distribution is
$
X \sim \bnm{10}{\frac{9}{10}}
    $
and the desired probability is 
$
  F_X\brak{6} 
  $.


\item Suppose that 90 \% of people are right-handed. What is the probability that atmost 6 of a random sample of 10 people are right-handed. 
	\\
\solution
From the given information, $X \sim \bnm{10}{\frac{9}{10}}$.  
	From \eqref{clt:bin-mu-sigma},
\begin{align}
\mu &= 10 \times \frac{9}{10} = 9,
	\\
	\sigma^2 &= 10 \times \frac{9}{10}\times \frac{1}{10} = \frac{9}{10}  
\end{align}
The desired probability is
\begin{align}
	\pr{X \le 6} \approx F_Y\brak{6} &= \qfunc{\frac{9-6}{3}\sqrt{10}}
	\\
    &=\qfunc{\sqrt{10}}
\end{align}
from 
\eqref{clt:gauss-cdf-mu-sigma}.

\item A die is thrown again and again until three sixes are obtained. Find the probability of obtaining third six on sixth throw of a die.
	\\
\solution 
\begin{table}[H]
	\centering
\begin{tabular}{|c|c|c|}
\hline
Random variable &Value &Definition\\ \hline
\multirow{3}{*}{X} &0 &Slips of Rs 1\\
&1 &Slips of Rs 5\\
&2 &Slips of Rs 13\\ \hline
\multirow{2}{*}{Y} &0 &Box A\\
&1 &Box B\\\hline
\end{tabular}
\caption{}
\label{tab:Distribution}
\end{table}
See \tabref{tab:Distribution}.
\begin{align}
p_{Y}\brak{k}= \begin{cases} 
      \frac{1}{3} & {k=0} \\
      \frac{2}{3 }& {k=1} 
   \end{cases}
   \\
p_{Y|X}\brak{0|0} = \frac{19}{25}\, 
p_{Y|X}\brak{0|1} = \frac{6}{25}\,
p_{Y|X}\brak{1|0} = \frac{45}{50}\,
p_{Y|X}\brak{1|2} = \frac{5}{50}
\end{align}
The desired probability is the probability that a slip drawn at random is marked other than Rs 1,
\begin{align}
&=1-p_X\brak{0}\\
&= p_X(1) + p_X(2)
\end{align}
Using Bayes theorem,
\begin{align}
&= p_Y\brak{0} \times \pr{Y=0 | X=1} + p_Y\brak{1} \times \pr{Y=1|X=2}\\
&=\frac{1}{3} \times \frac{6}{25} + \frac{2}{3} \times \frac{5}{50}\\
&=\frac{11}{75}
\end{align}

\newpage

%\tableofcontents

\bigskip

\renewcommand{\thefigure}{\theenumi}
\renewcommand{\thetable}{\theenumi}
%\renewcommand{\theequation}{\theenumi}

%\begin{abstract}
%%\boldmath
%In this letter, an algorithm for evaluating the exact analytical bit error rate  (BER)  for the piecewise linear (PL) combiner for  multiple relays is presented. Previous results were available only for upto three relays. The algorithm is unique in the sense that  the actual mathematical expressions, that are prohibitively large, need not be explicitly obtained. The diversity gain due to multiple relays is shown through plots of the analytical BER, well supported by simulations. 
%
%\end{abstract}
% IEEEtran.cls defaults to using nonbold math in the Abstract.
% This preserves the distinction between vectors and scalars. However,
% if the journal you are submitting to favors bold math in the abstract,
% then you can use LaTeX's standard command \boldmath at the very start
% of the abstract to achieve this. Many IEEE journals frown on math
% in the abstract anyway.

% Note that keywords are not normally used for peerreview papers.
%\begin{IEEEkeywords}
%Cooperative diversity, decode and forward, piecewise linear
%\end{IEEEkeywords}



% For peer review papers, you can put extra information on the cover
% page as needed:
% \ifCLASSOPTIONpeerreview
% \begin{center} \bfseries EDICS Category: 3-BBND \end{center}
% \fi
%
% For peerreview papers, this IEEEtran command inserts a page break and
% creates the second title. It will be ignored for other modes.
%\IEEEpeerreviewmaketitle




\end{enumerate}

    \section{Gaussian}
\subsection{NCERT}
%From the given information, $X \sim \bnm{10}{\frac{9}{10}}$.  
	From \eqref{clt:bin-mu-sigma},
\begin{align}
\mu &= 10 \times \frac{9}{10} = 9,
	\\
	\sigma^2 &= 10 \times \frac{9}{10}\times \frac{1}{10} = \frac{9}{10}  
\end{align}
The desired probability is
\begin{align}
	\pr{X \le 6} \approx F_Y\brak{6} &= \qfunc{\frac{9-6}{3}\sqrt{10}}
	\\
    &=\qfunc{\sqrt{10}}
\end{align}
from 
\eqref{clt:gauss-cdf-mu-sigma}.

\newpage
\section{Random Variables}
\subsection{NCERT}
\begin{enumerate}[label=\thesubsection.\arabic*,ref=\thesubsection.\theenumi]
\item State whether the statement is True or False. The probabilities that a typist will make 0, 1, 2, 3, 4, 5 or more mistakes in typing a report are, respectively, 0.12, 0.25, 0.36, 0.14, 0.08, 0.11.\\
\solution
From the given information, we obtain the distribution
\begin{align}
p_X(k)&=
\begin{cases}
0.12 & k=0\\
0.25 & k=1\\
0.36 & k=2\\
0.14 & k=3\\
0.08 & k=4\\
0.11 & k\geq5\\
\end{cases}
\end{align}
Since
\begin{align}
\sum_{i=0}^5 p_X(k) = 1.06
&>1
\end{align}
	violates \eqref{eq:axiom-pos},
the given statement is false.



\item State which of the following are not the probability distributions of a random
variable. Give reasons for your answer.
\begin{multicols}{2}  %
 \begin{enumerate}
		 \setlength{\itemsep}{1ex}
  \item
\input{ncert/12/13/4/1/tables/table1_1.tex}

 \item
	\resizebox{0.95\columnwidth}{!}{%
 \input{ncert/12/13/4/1/tables/table1_2.tex}
		 }
\item
  \input{ncert/12/13/4/1/tables/table1_3.tex}
\item
	\resizebox{0.95\columnwidth}{!}{%
  \input{ncert/12/13/4/1/tables/table1_4.tex}
		 }
\end{enumerate}
\end{multicols}  %
\solution
%
\begin{enumerate}
	\item The given distribution satisfies
	\eqref{eq:axiom-pos}
	and
	\eqref{eq:dist-axiom-one},
so it is a valid probability distribution.
   \item 
   \begin{align}
	   p_{X}(3) =-0.1 < 0
   \end{align}
   which violates 
	\eqref{eq:axiom-pos}.
Hence, not a probability distribution.  
 \item 
   \begin{align}
	   \sum_{k=-1}^{1}p_{X}(k) = 0.9 < 1
   \end{align}
   which violates
	\eqref{eq:dist-axiom-one}.
So, not a probability distribution. 
   \item 
   \begin{align}
       \sum_{k=0}^{4}p_{X}(k)  = 1.05 > 1
  \end{align}
   which violates
	\eqref{eq:dist-axiom-one}.
So, not a probability distribution. 
\end{enumerate}

\item A die has two faces each with number ‘1’, three faces each with number ‘2’ and one face with number ‘3’. If die is rolled once, determine
\begin{enumerate}
\item $\pr{2}$
\item $\pr{1 \text{ or } 3}$
\item $\pr{\text{not } 3}$
\end{enumerate}
		\solution
    From the given data, 
    \begin{align}
        \pr{A} = \frac{30}{60} = \frac{1}{2},\ 
        \pr{B} = \frac{32}{60} = \frac{8}{15},\ 
        \pr{AB}= \frac{24}{60} = \frac{2}{5}.
    \end{align}
    Thus, the desired probabilities are
\begin{enumerate}
	\item $\pr{A+B}=\frac{19}{30}$, from 
\eqref{eq:axiom_sum_AB}.
	\item From
	\eqref{eq:demorgan}
	and the axioms of probability,
        \begin{align}
\pr{A^\prime  B^\prime } =1-\pr{A+B}= \frac{11}{30}.
\end{align}
	\item     
		\begin{align}
	    \pr{A^\prime B}        &= \pr{B} - \pr{AB} = \frac{2}{15} 
    \end{align}
    from
\eqref{eq:axiom-cond-prime}.
\end{enumerate}

\item A die marked 1, 2, 3 in red and 4, 5, 6 in green is tossed. Let A be the event,‘the number is even,’ and B be the event, ‘the number is red’. Are A and B independent?
	\\
\solution
\begin{table}[H]
	\centering
\begin{tabular}{|c|c|c|}
\hline
Random variable &Value &Definition\\ \hline
\multirow{3}{*}{X} &0 &Slips of Rs 1\\
&1 &Slips of Rs 5\\
&2 &Slips of Rs 13\\ \hline
\multirow{2}{*}{Y} &0 &Box A\\
&1 &Box B\\\hline
\end{tabular}
\caption{}
\label{tab:Distribution}
\end{table}
See \tabref{tab:Distribution}.
\begin{align}
p_{Y}\brak{k}= \begin{cases} 
      \frac{1}{3} & {k=0} \\
      \frac{2}{3 }& {k=1} 
   \end{cases}
   \\
p_{Y|X}\brak{0|0} = \frac{19}{25}\, 
p_{Y|X}\brak{0|1} = \frac{6}{25}\,
p_{Y|X}\brak{1|0} = \frac{45}{50}\,
p_{Y|X}\brak{1|2} = \frac{5}{50}
\end{align}
The desired probability is the probability that a slip drawn at random is marked other than Rs 1,
\begin{align}
&=1-p_X\brak{0}\\
&= p_X(1) + p_X(2)
\end{align}
Using Bayes theorem,
\begin{align}
&= p_Y\brak{0} \times \pr{Y=0 | X=1} + p_Y\brak{1} \times \pr{Y=1|X=2}\\
&=\frac{1}{3} \times \frac{6}{25} + \frac{2}{3} \times \frac{5}{50}\\
&=\frac{11}{75}
\end{align}

\newpage

%\tableofcontents

\bigskip

\renewcommand{\thefigure}{\theenumi}
\renewcommand{\thetable}{\theenumi}
%\renewcommand{\theequation}{\theenumi}

%\begin{abstract}
%%\boldmath
%In this letter, an algorithm for evaluating the exact analytical bit error rate  (BER)  for the piecewise linear (PL) combiner for  multiple relays is presented. Previous results were available only for upto three relays. The algorithm is unique in the sense that  the actual mathematical expressions, that are prohibitively large, need not be explicitly obtained. The diversity gain due to multiple relays is shown through plots of the analytical BER, well supported by simulations. 
%
%\end{abstract}
% IEEEtran.cls defaults to using nonbold math in the Abstract.
% This preserves the distinction between vectors and scalars. However,
% if the journal you are submitting to favors bold math in the abstract,
% then you can use LaTeX's standard command \boldmath at the very start
% of the abstract to achieve this. Many IEEE journals frown on math
% in the abstract anyway.

% Note that keywords are not normally used for peerreview papers.
%\begin{IEEEkeywords}
%Cooperative diversity, decode and forward, piecewise linear
%\end{IEEEkeywords}



% For peer review papers, you can put extra information on the cover
% page as needed:
% \ifCLASSOPTIONpeerreview
% \begin{center} \bfseries EDICS Category: 3-BBND \end{center}
% \fi
%
% For peerreview papers, this IEEEtran command inserts a page break and
% creates the second title. It will be ignored for other modes.
%\IEEEpeerreviewmaketitle




\item  A fair coin and an unbiased die are tossed. Let $A$ be the event 'head appears on the coin' and $B$ be the event '3 on the die'. Check whether $A$ and $B$ are independent events or not.
		\label{ncert/12/13/2/4}
\end{enumerate}

%\subsection{Examples}
\begin{enumerate}[label=\thesection.\arabic*,ref=\thesection.\theenumi]
	\item One card is drawn from a well-shuffled deck of 52 cards. Find the probability of getting
\begin{enumerate}
\item A king of red colour 
\item A face card 
\item A red face card
\item The jack of hearts
\item A spade
\item The queen of diamonds

\end{enumerate}
\solution
		%\begin{table}[H]
	\centering
\begin{tabular}{|c|c|c|}
\hline
Random variable &Value &Definition\\ \hline
\multirow{3}{*}{X} &0 &Slips of Rs 1\\
&1 &Slips of Rs 5\\
&2 &Slips of Rs 13\\ \hline
\multirow{2}{*}{Y} &0 &Box A\\
&1 &Box B\\\hline
\end{tabular}
\caption{}
\label{tab:Distribution}
\end{table}
See \tabref{tab:Distribution}.
\begin{align}
p_{Y}\brak{k}= \begin{cases} 
      \frac{1}{3} & {k=0} \\
      \frac{2}{3 }& {k=1} 
   \end{cases}
   \\
p_{Y|X}\brak{0|0} = \frac{19}{25}\, 
p_{Y|X}\brak{0|1} = \frac{6}{25}\,
p_{Y|X}\brak{1|0} = \frac{45}{50}\,
p_{Y|X}\brak{1|2} = \frac{5}{50}
\end{align}
The desired probability is the probability that a slip drawn at random is marked other than Rs 1,
\begin{align}
&=1-p_X\brak{0}\\
&= p_X(1) + p_X(2)
\end{align}
Using Bayes theorem,
\begin{align}
&= p_Y\brak{0} \times \pr{Y=0 | X=1} + p_Y\brak{1} \times \pr{Y=1|X=2}\\
&=\frac{1}{3} \times \frac{6}{25} + \frac{2}{3} \times \frac{5}{50}\\
&=\frac{11}{75}
\end{align}

\newpage

%\tableofcontents

\bigskip

\renewcommand{\thefigure}{\theenumi}
\renewcommand{\thetable}{\theenumi}
%\renewcommand{\theequation}{\theenumi}

%\begin{abstract}
%%\boldmath
%In this letter, an algorithm for evaluating the exact analytical bit error rate  (BER)  for the piecewise linear (PL) combiner for  multiple relays is presented. Previous results were available only for upto three relays. The algorithm is unique in the sense that  the actual mathematical expressions, that are prohibitively large, need not be explicitly obtained. The diversity gain due to multiple relays is shown through plots of the analytical BER, well supported by simulations. 
%
%\end{abstract}
% IEEEtran.cls defaults to using nonbold math in the Abstract.
% This preserves the distinction between vectors and scalars. However,
% if the journal you are submitting to favors bold math in the abstract,
% then you can use LaTeX's standard command \boldmath at the very start
% of the abstract to achieve this. Many IEEE journals frown on math
% in the abstract anyway.

% Note that keywords are not normally used for peerreview papers.
%\begin{IEEEkeywords}
%Cooperative diversity, decode and forward, piecewise linear
%\end{IEEEkeywords}



% For peer review papers, you can put extra information on the cover
% page as needed:
% \ifCLASSOPTIONpeerreview
% \begin{center} \bfseries EDICS Category: 3-BBND \end{center}
% \fi
%
% For peerreview papers, this IEEEtran command inserts a page break and
% creates the second title. It will be ignored for other modes.
%\IEEEpeerreviewmaketitle




	\item Five cards—the ten, jack, queen, king and ace of diamonds, are well-shuffled with their face downwards. One card is then picked up at random.
\begin{enumerate}
\item
What is the probability that the card is the queen? 
\item
If the queen is drawn and put aside, what is the probability that the second card picked up is (a) an ace? (b) a queen?\\
\end{enumerate}
\solution
		%\begin{enumerate}[label=\thesection.\arabic*,ref=\thesection.\theenumi]
	\item One card is drawn from a well-shuffled deck of 52 cards. Find the probability of getting
\begin{enumerate}
\item A king of red colour 
\item A face card 
\item A red face card
\item The jack of hearts
\item A spade
\item The queen of diamonds

\end{enumerate}
\solution
		%\begin{table}[H]
	\centering
\begin{tabular}{|c|c|c|}
\hline
Random variable &Value &Definition\\ \hline
\multirow{3}{*}{X} &0 &Slips of Rs 1\\
&1 &Slips of Rs 5\\
&2 &Slips of Rs 13\\ \hline
\multirow{2}{*}{Y} &0 &Box A\\
&1 &Box B\\\hline
\end{tabular}
\caption{}
\label{tab:Distribution}
\end{table}
See \tabref{tab:Distribution}.
\begin{align}
p_{Y}\brak{k}= \begin{cases} 
      \frac{1}{3} & {k=0} \\
      \frac{2}{3 }& {k=1} 
   \end{cases}
   \\
p_{Y|X}\brak{0|0} = \frac{19}{25}\, 
p_{Y|X}\brak{0|1} = \frac{6}{25}\,
p_{Y|X}\brak{1|0} = \frac{45}{50}\,
p_{Y|X}\brak{1|2} = \frac{5}{50}
\end{align}
The desired probability is the probability that a slip drawn at random is marked other than Rs 1,
\begin{align}
&=1-p_X\brak{0}\\
&= p_X(1) + p_X(2)
\end{align}
Using Bayes theorem,
\begin{align}
&= p_Y\brak{0} \times \pr{Y=0 | X=1} + p_Y\brak{1} \times \pr{Y=1|X=2}\\
&=\frac{1}{3} \times \frac{6}{25} + \frac{2}{3} \times \frac{5}{50}\\
&=\frac{11}{75}
\end{align}

\newpage

%\tableofcontents

\bigskip

\renewcommand{\thefigure}{\theenumi}
\renewcommand{\thetable}{\theenumi}
%\renewcommand{\theequation}{\theenumi}

%\begin{abstract}
%%\boldmath
%In this letter, an algorithm for evaluating the exact analytical bit error rate  (BER)  for the piecewise linear (PL) combiner for  multiple relays is presented. Previous results were available only for upto three relays. The algorithm is unique in the sense that  the actual mathematical expressions, that are prohibitively large, need not be explicitly obtained. The diversity gain due to multiple relays is shown through plots of the analytical BER, well supported by simulations. 
%
%\end{abstract}
% IEEEtran.cls defaults to using nonbold math in the Abstract.
% This preserves the distinction between vectors and scalars. However,
% if the journal you are submitting to favors bold math in the abstract,
% then you can use LaTeX's standard command \boldmath at the very start
% of the abstract to achieve this. Many IEEE journals frown on math
% in the abstract anyway.

% Note that keywords are not normally used for peerreview papers.
%\begin{IEEEkeywords}
%Cooperative diversity, decode and forward, piecewise linear
%\end{IEEEkeywords}



% For peer review papers, you can put extra information on the cover
% page as needed:
% \ifCLASSOPTIONpeerreview
% \begin{center} \bfseries EDICS Category: 3-BBND \end{center}
% \fi
%
% For peerreview papers, this IEEEtran command inserts a page break and
% creates the second title. It will be ignored for other modes.
%\IEEEpeerreviewmaketitle




	\item Five cards—the ten, jack, queen, king and ace of diamonds, are well-shuffled with their face downwards. One card is then picked up at random.
\begin{enumerate}
\item
What is the probability that the card is the queen? 
\item
If the queen is drawn and put aside, what is the probability that the second card picked up is (a) an ace? (b) a queen?\\
\end{enumerate}
\solution
		%\begin{enumerate}[label=\thesection.\arabic*,ref=\thesection.\theenumi]
	\item One card is drawn from a well-shuffled deck of 52 cards. Find the probability of getting
\begin{enumerate}
\item A king of red colour 
\item A face card 
\item A red face card
\item The jack of hearts
\item A spade
\item The queen of diamonds

\end{enumerate}
\solution
		%\input{ncert/10/15/1/14/main.tex}
	\item Five cards—the ten, jack, queen, king and ace of diamonds, are well-shuffled with their face downwards. One card is then picked up at random.
\begin{enumerate}
\item
What is the probability that the card is the queen? 
\item
If the queen is drawn and put aside, what is the probability that the second card picked up is (a) an ace? (b) a queen?\\
\end{enumerate}
\solution
		%\input{ncert/10/15/1/15/defs.tex}
	\item A bag contains $5$ red balls and some blue balls. If the probability of drawing a blue ball is double that if a red ball, determine the number of blue balls in the bag. 
		\\
\solution
		%\input{ncert/10/15/2/3/defs.tex}
	\item A card is selected from a pack of 52 cards.
 \begin{enumerate}[label=(\alph*)] 
                 \item How many points are there in the sample space?
                 \item Calculate the probability that the card is an ace of spades.
                 \item Calculate the probability that the card is (i) an ace and (ii) black card.
 \end{enumerate}
\solution
		%\input{ncert/11/16/3/4/main.tex}
\item Four cards are drawn from a well-shuffled deck of 52 cards. What is the probability of obtaining 3 diamonds and one spade.
\\
\solution
		%\input{ncert/11/16/4/2/defs.tex}
\item In a certain lottery 10,000 tickets are sold and ten equal prizes are awarded. What is the probability of not getting a prize if you buy (a) one ticket (b) two tickets (c) 10 tickets ?	
\\
\solution
		%\input{ncert/11/16/4/4/defs.tex}
		%
\item 
Out of 100 students, two sections of 40 and 60 are formed. If you and your friend are among the 100 students, what is the probability that
\begin{enumerate}
\item you both enter the same section?
\item you both enter the different sections?
\end{enumerate}
\solution
		%\input{ncert/11/16/4/5/defs.tex}
	\item 
The number lock of a suitcase has 4 wheels each labelled with ten digits i.e. from 0 to 9.The lock opens with a sequence of four digits with no repeats.What is the probability of a person getting the right sequence to open the suitcase.
\\
\solution
		%\input{ncert/11/16/4/10/defs.tex}
		%
\item 
Two cards are drawn at random and without replacement from a pack of 52 playing cards. Find the probability that both the cards are black.
\\
\solution
		%\input{ncert/12/13/2/2/defs.tex}
		\item A box of oranges is inspected by examining three randomly selected oranges drawn without replacement. If all the three oranges are good, the box is approved for sale, otherwise, it is rejected. Find the probability that a box containing 15 oranges out of which 12 are good and 3 are bad ones will be approved for sale.
		\label{ncert/12/13/2/3/defs.tex}
		\item Two balls are drawn at random with replacement from a box containing 10 black and 8 red balls. Find the probability that
		\label{ncert/12/13/2/12}
\begin{enumerate}
\item both balls are red.
\item first ball is black and second is red.
\item one of them is black and other is red.
\end{enumerate}

\item In a hostel, 60\% of the students read Hindi newspaper, 40\% read English newspaper and 20\% read both Hindi and English newspapers. A student is selected at random.
		\label{ncert/12/13/2/15}
\begin{enumerate}
\item Find the probability that she reads neither Hindi nor English newspapers.
\item If she reads Hindi newspaper, find the probability that she reads English newspaper.
\item If she reads English newspaper, find the probability that she reads Hindi newspaper.\\
\end{enumerate}
\item The probability of obtaining an even prime number on each die, when a pair of dice is rolled is 
\begin{enumerate}
    \item $0$ 
    
    \item $\frac{1}{3}$ 
    
    \item $\frac{1}{12}$ 
    
    \item $\frac{1}{36}$ 
\end{enumerate}
\solution
		%\input{ncert/12/13/2/17/defs.tex}
	\item A bag contains 4 red and 4 black balls, another bag contains 2 red and 6 black balls. One of the two bags is selected at random and a ball is drawn from the bag which is found to be red. Find the probability that the ball is drawn from the first bag.
\\
\solution
		%\input{ncert/12/13/3/2/main.tex}
  \item
  Cards with numbers 2 to 101 are placed in a box. A card is selected at random.Find the probability that the card has
\begin{enumerate}[label=(\roman*)]
	\item an even number 
	\item a square number
\end{enumerate}
\solution
%\input{exemplar/10/13/3/32/main.tex}
\item
The king, queen and jack of clubs are removed from a deck of 52 playing cards and then well shuffled. Now one card is drawn at random from the remaining cards.  Determine the probability that the card is
\begin{enumerate}[label=(\roman*)]
\item a club
\item 10 of hearts
\end{enumerate}
\solution
%\input{exemplar/10/13/3/29/main.tex}
\item A team of medical students doing their internship have to assist during surgeries
at a city hospital. The probabilities of surgeries rated as very complex, complex,
routine, simple or very simple are respectively, 0.15, 0.20, 0.31, 0.26, .08. Find
the probabilities that a particular surgery will be rated
\begin{enumerate}
	\item complex or very complex;
	\item neither very complex nor very simple;
	\item routine or complex
	\item routine or simple
\end{enumerate}
\solution
%\input{exemplar/11/16/3/8(1)/main.tex}
\item A card is selected from a pack of 52 cards.
\begin{enumerate}[label=(\alph*)]
    \item How many points are there in the sample space?
    \item Calculate the probability that the card is an ace of spades.
    \item Calculate the probability that the card is (i) an ace and (ii) black card.
\end{enumerate}
\solution
%\input{exemplar/11/16/3/4/main2.tex}
\item The probability that a non leap year selected at random will contain 53 sundays.
\\
\solution
%\input{exemplar/10/13/1/19/main.tex}
\item One of the four persons John, Rita, Aslam or Gurpreet will be promoted next
month. Consequently the sample space consists of four elementary outcomes
S = {John promoted, Rita promoted, Aslam promoted, Gurpreet promoted}
You are told that the chances of John’s promotion is same as that of Gurpreet,
Rita’s chances of promotion are twice as likely as Johns. Aslam’s chances are
four times that of John.
\begin{enumerate}
	\item Determine
	\begin{enumerate}
		\item P (John promoted)
		\item P (Rita promoted)
		\item P (Aslam promoted)
		\item P (Gurpreet promoted)
	\end{enumerate}
	\item If A = {John promoted or Gurpreet promoted}, find P (A).
\end{enumerate}
\solution
%\input{exemplar/11/16/3/10/main.tex}
\item A card is drawn from a deck of 52 cards. Find the probability of getting a king or a heart or a red card.\\
\solution
%\input{exemplar/11/16/3/15/main.tex}
\item The probability that a student will pass his examination is 0.73, the probability of
the student getting a compartment is 0.13, and the probability that the student will
either pass or get compartment is 0.96. State True or False.\\
\solution
%\input{exemplar/11/16/3/31/main.tex}
\item A card is selected from a pack of 52 cards\\
\begin{enumerate}[label=(\alph*)]
\item How many points are there in the sample space?
\item Calculate the probability that the cards is an ace of spades.
\item Calculate the probability that the card is (i) an ace (ii)black card.\\
\end{enumerate}
%\input{ncert/11/16/3/4_1/Prob_4.tex}
\item In a non-leap year, the probability of having 53 tuesdays or 53 wednesdays is\\
\solution
%\input{exemplar/11/16/3/18/main.tex}
\item There are 1000 sealed envelopes in a box, 10 of them contain a cash prize of
Rs 100 each, 100 of them contain a cash prize of Rs 50 each and 200 of them
contain a cash prize of Rs 10 each and rest do not contain any cash prize. If they
are well shuffled and an envelope is picked up out, what is the probability that it
contains no cash prize?\\
\solution
%\input{exemplar/10/13/3/34/main.tex}
\item 
A die is thrown and a card is selected at random from a deck of 52 playing cards. The probability of getting an even number on the die and a spade card.\\
\solution
%\input{exemplar/12/13/3/78/main.tex}
\item
If 4-digit numbers greater than 5,000 are randomly formed from the digits 0, 1, 3, 5, and 7, what is the probability of forming a number divisible by 5 when:
\begin{enumerate}
    \item The digits are repeated?
    \item The repetition of digits is not allowed?
\end{enumerate}
\solution
%\input{ncert/11/16/4/9/main.tex}
\item Consider the probability space $\brak{\Omega, \mathcal{G}, P}$ where $\Omega = [0,2]$ and $\mathcal{G} = \cbrak{\phi, \Omega, [0,1], (1,2]}$. Let $X$ and $Y$ be two functions on $\Omega$ defined as
\begin{align*}
    X(\omega) = 
    \begin{cases}
        1 & \text{if }\omega \in [0, 1]\\
        2 & \text{if }\omega \in (1, 2]
    \end{cases}
\end{align*}
and
\begin{align*}
    Y(\omega) = 
    \begin{cases}
        2 & \text{if }\omega \in [0, 1.5]\\
        3 & \text{if }\omega \in (1.5, 2].
    \end{cases}
\end{align*}
Then which one of the following statements is true?
\begin{enumerate}
    \item [(A)] $X$ is a random variable with respect to $\mathcal{G}$, but $Y$ is not a random variable with respect to $\mathcal{G}$.
    \item [(B)] $Y$ is a random variable with respect to $\mathcal{G}$, but $X$ is not a random variable with respect to $\mathcal{G}$.
    \item [(C)] Neither $X$ nor $Y$ is a random variable with respect to $\mathcal{G}$.
    \item [(D)] Both $X$ and $Y$ are random variables with respect to $\mathcal{G}$.
\end{enumerate} \hfill (GATE ST 2023)\\
\solution
%\input{gate/ST/2023/14/main.tex}
	\item  A die is loaded in such a way that each odd number is twice as likely to occur as
each even number. Find $P(G)$, where $G$ is the event that a number greater than
3 occurs on a single roll of the die.
\\
\solution
		%\input{exemplar/11/16/3/5/main.tex}
	\item All the jacks, queens and kings are removed from a deck of 52 playing cards. The remaining cards are well shuffled and then one card is drawn at random. Giving ace a value 1 similar value for other cards, find the probability that the card has a value 
		\begin{enumerate}
			\item 7
			\item greater than 7
			\item less than 7
		\end{enumerate}
		%\input{exemplar/10/13/3/30/main.tex}
  \item A Lot consists of 48 mobile phones of which 42 are good, 3 have only minor defects and 3 have major defects.Varnika will buy a phone if it is good but the trader will only buy a mobile if it has no major defects. One phone is selected at random from the lot. What is the probability that it is
\begin{enumerate}
	\item acceptable to Varnika?
            \item acceptable to the trader?
\end{enumerate}
\solution
	%\input{exemplar/10/13/3/40/main.tex}
 \item A student says that if you throw a die, it will show up 1 or not 1. Therefore, the probability of getting 1 and the probability of getting 'not 1' each is equal to $\frac{1}{2}$. Is this correct? Give reasons.\\
 \solution
        %\input{exemplar/10/13/2/9/main.tex}
   \item Four candidates A, B, C, D have ap-
plied for the assignment to coach a school cricket
team. If A is twice as likely to be selected as B, and
B and C are given about the same chance of being
selected, while C is twice as likely to be selected
as D, what are the probabilities that
\begin{enumerate}
\item C will be selected?
\item A will not be selected?
\end{enumerate}
	%\input{exemplar/11/16/3/9/main.tex}
 \item A bag contain 24 balls of which $x$ balls are red, $2x$ are white and $3x$ are blue. A ball is selected at random, What is the probability that it is
\begin{enumerate}[label=\alph*)]
\item not red ?
\item white ?
\end{enumerate}
%\input{exemplar/10/13/3/41/main.tex}
If the letters of the word ASSASSINATION are arranged at random. Find the Probability that
\begin{enumerate}[label=(\alph*)]
\item Four $S's$ come consecutively in the word
\item Two  $I's$ and two $N's$ come together
\item All $A's$ are not coming together
\item No two $A's$ are coming together
\end{enumerate}
%\input{exemplar/11/16/3/14/main.tex}
	\item One urn contains two black balls (labelled B1 and B2) and one white ball. A
	second urn contains one black ball and two white balls (labelled W1 and W2).
	Suppose the following experiment is performed. One of the two urns is chosen
	at random. Next a ball is randomly chosen from the urn. Then a second ball is
	chosen at random from the same urn without replacing the first ball.
	
	\begin{enumerate}
	\item What is the probability that two black balls are chosen?
	
	\item What is the probability that two balls of opposite colour are chosen?
	\end{enumerate}
	\solution
	%\input{exemplar/11/16/3/12/main1.tex}
\end{enumerate}

	\item A bag contains $5$ red balls and some blue balls. If the probability of drawing a blue ball is double that if a red ball, determine the number of blue balls in the bag. 
		\\
\solution
		%\begin{enumerate}[label=\thesection.\arabic*,ref=\thesection.\theenumi]
	\item One card is drawn from a well-shuffled deck of 52 cards. Find the probability of getting
\begin{enumerate}
\item A king of red colour 
\item A face card 
\item A red face card
\item The jack of hearts
\item A spade
\item The queen of diamonds

\end{enumerate}
\solution
		%\input{ncert/10/15/1/14/main.tex}
	\item Five cards—the ten, jack, queen, king and ace of diamonds, are well-shuffled with their face downwards. One card is then picked up at random.
\begin{enumerate}
\item
What is the probability that the card is the queen? 
\item
If the queen is drawn and put aside, what is the probability that the second card picked up is (a) an ace? (b) a queen?\\
\end{enumerate}
\solution
		%\input{ncert/10/15/1/15/defs.tex}
	\item A bag contains $5$ red balls and some blue balls. If the probability of drawing a blue ball is double that if a red ball, determine the number of blue balls in the bag. 
		\\
\solution
		%\input{ncert/10/15/2/3/defs.tex}
	\item A card is selected from a pack of 52 cards.
 \begin{enumerate}[label=(\alph*)] 
                 \item How many points are there in the sample space?
                 \item Calculate the probability that the card is an ace of spades.
                 \item Calculate the probability that the card is (i) an ace and (ii) black card.
 \end{enumerate}
\solution
		%\input{ncert/11/16/3/4/main.tex}
\item Four cards are drawn from a well-shuffled deck of 52 cards. What is the probability of obtaining 3 diamonds and one spade.
\\
\solution
		%\input{ncert/11/16/4/2/defs.tex}
\item In a certain lottery 10,000 tickets are sold and ten equal prizes are awarded. What is the probability of not getting a prize if you buy (a) one ticket (b) two tickets (c) 10 tickets ?	
\\
\solution
		%\input{ncert/11/16/4/4/defs.tex}
		%
\item 
Out of 100 students, two sections of 40 and 60 are formed. If you and your friend are among the 100 students, what is the probability that
\begin{enumerate}
\item you both enter the same section?
\item you both enter the different sections?
\end{enumerate}
\solution
		%\input{ncert/11/16/4/5/defs.tex}
	\item 
The number lock of a suitcase has 4 wheels each labelled with ten digits i.e. from 0 to 9.The lock opens with a sequence of four digits with no repeats.What is the probability of a person getting the right sequence to open the suitcase.
\\
\solution
		%\input{ncert/11/16/4/10/defs.tex}
		%
\item 
Two cards are drawn at random and without replacement from a pack of 52 playing cards. Find the probability that both the cards are black.
\\
\solution
		%\input{ncert/12/13/2/2/defs.tex}
		\item A box of oranges is inspected by examining three randomly selected oranges drawn without replacement. If all the three oranges are good, the box is approved for sale, otherwise, it is rejected. Find the probability that a box containing 15 oranges out of which 12 are good and 3 are bad ones will be approved for sale.
		\label{ncert/12/13/2/3/defs.tex}
		\item Two balls are drawn at random with replacement from a box containing 10 black and 8 red balls. Find the probability that
		\label{ncert/12/13/2/12}
\begin{enumerate}
\item both balls are red.
\item first ball is black and second is red.
\item one of them is black and other is red.
\end{enumerate}

\item In a hostel, 60\% of the students read Hindi newspaper, 40\% read English newspaper and 20\% read both Hindi and English newspapers. A student is selected at random.
		\label{ncert/12/13/2/15}
\begin{enumerate}
\item Find the probability that she reads neither Hindi nor English newspapers.
\item If she reads Hindi newspaper, find the probability that she reads English newspaper.
\item If she reads English newspaper, find the probability that she reads Hindi newspaper.\\
\end{enumerate}
\item The probability of obtaining an even prime number on each die, when a pair of dice is rolled is 
\begin{enumerate}
    \item $0$ 
    
    \item $\frac{1}{3}$ 
    
    \item $\frac{1}{12}$ 
    
    \item $\frac{1}{36}$ 
\end{enumerate}
\solution
		%\input{ncert/12/13/2/17/defs.tex}
	\item A bag contains 4 red and 4 black balls, another bag contains 2 red and 6 black balls. One of the two bags is selected at random and a ball is drawn from the bag which is found to be red. Find the probability that the ball is drawn from the first bag.
\\
\solution
		%\input{ncert/12/13/3/2/main.tex}
  \item
  Cards with numbers 2 to 101 are placed in a box. A card is selected at random.Find the probability that the card has
\begin{enumerate}[label=(\roman*)]
	\item an even number 
	\item a square number
\end{enumerate}
\solution
%\input{exemplar/10/13/3/32/main.tex}
\item
The king, queen and jack of clubs are removed from a deck of 52 playing cards and then well shuffled. Now one card is drawn at random from the remaining cards.  Determine the probability that the card is
\begin{enumerate}[label=(\roman*)]
\item a club
\item 10 of hearts
\end{enumerate}
\solution
%\input{exemplar/10/13/3/29/main.tex}
\item A team of medical students doing their internship have to assist during surgeries
at a city hospital. The probabilities of surgeries rated as very complex, complex,
routine, simple or very simple are respectively, 0.15, 0.20, 0.31, 0.26, .08. Find
the probabilities that a particular surgery will be rated
\begin{enumerate}
	\item complex or very complex;
	\item neither very complex nor very simple;
	\item routine or complex
	\item routine or simple
\end{enumerate}
\solution
%\input{exemplar/11/16/3/8(1)/main.tex}
\item A card is selected from a pack of 52 cards.
\begin{enumerate}[label=(\alph*)]
    \item How many points are there in the sample space?
    \item Calculate the probability that the card is an ace of spades.
    \item Calculate the probability that the card is (i) an ace and (ii) black card.
\end{enumerate}
\solution
%\input{exemplar/11/16/3/4/main2.tex}
\item The probability that a non leap year selected at random will contain 53 sundays.
\\
\solution
%\input{exemplar/10/13/1/19/main.tex}
\item One of the four persons John, Rita, Aslam or Gurpreet will be promoted next
month. Consequently the sample space consists of four elementary outcomes
S = {John promoted, Rita promoted, Aslam promoted, Gurpreet promoted}
You are told that the chances of John’s promotion is same as that of Gurpreet,
Rita’s chances of promotion are twice as likely as Johns. Aslam’s chances are
four times that of John.
\begin{enumerate}
	\item Determine
	\begin{enumerate}
		\item P (John promoted)
		\item P (Rita promoted)
		\item P (Aslam promoted)
		\item P (Gurpreet promoted)
	\end{enumerate}
	\item If A = {John promoted or Gurpreet promoted}, find P (A).
\end{enumerate}
\solution
%\input{exemplar/11/16/3/10/main.tex}
\item A card is drawn from a deck of 52 cards. Find the probability of getting a king or a heart or a red card.\\
\solution
%\input{exemplar/11/16/3/15/main.tex}
\item The probability that a student will pass his examination is 0.73, the probability of
the student getting a compartment is 0.13, and the probability that the student will
either pass or get compartment is 0.96. State True or False.\\
\solution
%\input{exemplar/11/16/3/31/main.tex}
\item A card is selected from a pack of 52 cards\\
\begin{enumerate}[label=(\alph*)]
\item How many points are there in the sample space?
\item Calculate the probability that the cards is an ace of spades.
\item Calculate the probability that the card is (i) an ace (ii)black card.\\
\end{enumerate}
%\input{ncert/11/16/3/4_1/Prob_4.tex}
\item In a non-leap year, the probability of having 53 tuesdays or 53 wednesdays is\\
\solution
%\input{exemplar/11/16/3/18/main.tex}
\item There are 1000 sealed envelopes in a box, 10 of them contain a cash prize of
Rs 100 each, 100 of them contain a cash prize of Rs 50 each and 200 of them
contain a cash prize of Rs 10 each and rest do not contain any cash prize. If they
are well shuffled and an envelope is picked up out, what is the probability that it
contains no cash prize?\\
\solution
%\input{exemplar/10/13/3/34/main.tex}
\item 
A die is thrown and a card is selected at random from a deck of 52 playing cards. The probability of getting an even number on the die and a spade card.\\
\solution
%\input{exemplar/12/13/3/78/main.tex}
\item
If 4-digit numbers greater than 5,000 are randomly formed from the digits 0, 1, 3, 5, and 7, what is the probability of forming a number divisible by 5 when:
\begin{enumerate}
    \item The digits are repeated?
    \item The repetition of digits is not allowed?
\end{enumerate}
\solution
%\input{ncert/11/16/4/9/main.tex}
\item Consider the probability space $\brak{\Omega, \mathcal{G}, P}$ where $\Omega = [0,2]$ and $\mathcal{G} = \cbrak{\phi, \Omega, [0,1], (1,2]}$. Let $X$ and $Y$ be two functions on $\Omega$ defined as
\begin{align*}
    X(\omega) = 
    \begin{cases}
        1 & \text{if }\omega \in [0, 1]\\
        2 & \text{if }\omega \in (1, 2]
    \end{cases}
\end{align*}
and
\begin{align*}
    Y(\omega) = 
    \begin{cases}
        2 & \text{if }\omega \in [0, 1.5]\\
        3 & \text{if }\omega \in (1.5, 2].
    \end{cases}
\end{align*}
Then which one of the following statements is true?
\begin{enumerate}
    \item [(A)] $X$ is a random variable with respect to $\mathcal{G}$, but $Y$ is not a random variable with respect to $\mathcal{G}$.
    \item [(B)] $Y$ is a random variable with respect to $\mathcal{G}$, but $X$ is not a random variable with respect to $\mathcal{G}$.
    \item [(C)] Neither $X$ nor $Y$ is a random variable with respect to $\mathcal{G}$.
    \item [(D)] Both $X$ and $Y$ are random variables with respect to $\mathcal{G}$.
\end{enumerate} \hfill (GATE ST 2023)\\
\solution
%\input{gate/ST/2023/14/main.tex}
	\item  A die is loaded in such a way that each odd number is twice as likely to occur as
each even number. Find $P(G)$, where $G$ is the event that a number greater than
3 occurs on a single roll of the die.
\\
\solution
		%\input{exemplar/11/16/3/5/main.tex}
	\item All the jacks, queens and kings are removed from a deck of 52 playing cards. The remaining cards are well shuffled and then one card is drawn at random. Giving ace a value 1 similar value for other cards, find the probability that the card has a value 
		\begin{enumerate}
			\item 7
			\item greater than 7
			\item less than 7
		\end{enumerate}
		%\input{exemplar/10/13/3/30/main.tex}
  \item A Lot consists of 48 mobile phones of which 42 are good, 3 have only minor defects and 3 have major defects.Varnika will buy a phone if it is good but the trader will only buy a mobile if it has no major defects. One phone is selected at random from the lot. What is the probability that it is
\begin{enumerate}
	\item acceptable to Varnika?
            \item acceptable to the trader?
\end{enumerate}
\solution
	%\input{exemplar/10/13/3/40/main.tex}
 \item A student says that if you throw a die, it will show up 1 or not 1. Therefore, the probability of getting 1 and the probability of getting 'not 1' each is equal to $\frac{1}{2}$. Is this correct? Give reasons.\\
 \solution
        %\input{exemplar/10/13/2/9/main.tex}
   \item Four candidates A, B, C, D have ap-
plied for the assignment to coach a school cricket
team. If A is twice as likely to be selected as B, and
B and C are given about the same chance of being
selected, while C is twice as likely to be selected
as D, what are the probabilities that
\begin{enumerate}
\item C will be selected?
\item A will not be selected?
\end{enumerate}
	%\input{exemplar/11/16/3/9/main.tex}
 \item A bag contain 24 balls of which $x$ balls are red, $2x$ are white and $3x$ are blue. A ball is selected at random, What is the probability that it is
\begin{enumerate}[label=\alph*)]
\item not red ?
\item white ?
\end{enumerate}
%\input{exemplar/10/13/3/41/main.tex}
If the letters of the word ASSASSINATION are arranged at random. Find the Probability that
\begin{enumerate}[label=(\alph*)]
\item Four $S's$ come consecutively in the word
\item Two  $I's$ and two $N's$ come together
\item All $A's$ are not coming together
\item No two $A's$ are coming together
\end{enumerate}
%\input{exemplar/11/16/3/14/main.tex}
	\item One urn contains two black balls (labelled B1 and B2) and one white ball. A
	second urn contains one black ball and two white balls (labelled W1 and W2).
	Suppose the following experiment is performed. One of the two urns is chosen
	at random. Next a ball is randomly chosen from the urn. Then a second ball is
	chosen at random from the same urn without replacing the first ball.
	
	\begin{enumerate}
	\item What is the probability that two black balls are chosen?
	
	\item What is the probability that two balls of opposite colour are chosen?
	\end{enumerate}
	\solution
	%\input{exemplar/11/16/3/12/main1.tex}
\end{enumerate}

	\item A card is selected from a pack of 52 cards.
 \begin{enumerate}[label=(\alph*)] 
                 \item How many points are there in the sample space?
                 \item Calculate the probability that the card is an ace of spades.
                 \item Calculate the probability that the card is (i) an ace and (ii) black card.
 \end{enumerate}
\solution
		%\begin{table}[H]
	\centering
\begin{tabular}{|c|c|c|}
\hline
Random variable &Value &Definition\\ \hline
\multirow{3}{*}{X} &0 &Slips of Rs 1\\
&1 &Slips of Rs 5\\
&2 &Slips of Rs 13\\ \hline
\multirow{2}{*}{Y} &0 &Box A\\
&1 &Box B\\\hline
\end{tabular}
\caption{}
\label{tab:Distribution}
\end{table}
See \tabref{tab:Distribution}.
\begin{align}
p_{Y}\brak{k}= \begin{cases} 
      \frac{1}{3} & {k=0} \\
      \frac{2}{3 }& {k=1} 
   \end{cases}
   \\
p_{Y|X}\brak{0|0} = \frac{19}{25}\, 
p_{Y|X}\brak{0|1} = \frac{6}{25}\,
p_{Y|X}\brak{1|0} = \frac{45}{50}\,
p_{Y|X}\brak{1|2} = \frac{5}{50}
\end{align}
The desired probability is the probability that a slip drawn at random is marked other than Rs 1,
\begin{align}
&=1-p_X\brak{0}\\
&= p_X(1) + p_X(2)
\end{align}
Using Bayes theorem,
\begin{align}
&= p_Y\brak{0} \times \pr{Y=0 | X=1} + p_Y\brak{1} \times \pr{Y=1|X=2}\\
&=\frac{1}{3} \times \frac{6}{25} + \frac{2}{3} \times \frac{5}{50}\\
&=\frac{11}{75}
\end{align}

\newpage

%\tableofcontents

\bigskip

\renewcommand{\thefigure}{\theenumi}
\renewcommand{\thetable}{\theenumi}
%\renewcommand{\theequation}{\theenumi}

%\begin{abstract}
%%\boldmath
%In this letter, an algorithm for evaluating the exact analytical bit error rate  (BER)  for the piecewise linear (PL) combiner for  multiple relays is presented. Previous results were available only for upto three relays. The algorithm is unique in the sense that  the actual mathematical expressions, that are prohibitively large, need not be explicitly obtained. The diversity gain due to multiple relays is shown through plots of the analytical BER, well supported by simulations. 
%
%\end{abstract}
% IEEEtran.cls defaults to using nonbold math in the Abstract.
% This preserves the distinction between vectors and scalars. However,
% if the journal you are submitting to favors bold math in the abstract,
% then you can use LaTeX's standard command \boldmath at the very start
% of the abstract to achieve this. Many IEEE journals frown on math
% in the abstract anyway.

% Note that keywords are not normally used for peerreview papers.
%\begin{IEEEkeywords}
%Cooperative diversity, decode and forward, piecewise linear
%\end{IEEEkeywords}



% For peer review papers, you can put extra information on the cover
% page as needed:
% \ifCLASSOPTIONpeerreview
% \begin{center} \bfseries EDICS Category: 3-BBND \end{center}
% \fi
%
% For peerreview papers, this IEEEtran command inserts a page break and
% creates the second title. It will be ignored for other modes.
%\IEEEpeerreviewmaketitle




\item Four cards are drawn from a well-shuffled deck of 52 cards. What is the probability of obtaining 3 diamonds and one spade.
\\
\solution
		%\begin{enumerate}[label=\thesection.\arabic*,ref=\thesection.\theenumi]
	\item One card is drawn from a well-shuffled deck of 52 cards. Find the probability of getting
\begin{enumerate}
\item A king of red colour 
\item A face card 
\item A red face card
\item The jack of hearts
\item A spade
\item The queen of diamonds

\end{enumerate}
\solution
		%\input{ncert/10/15/1/14/main.tex}
	\item Five cards—the ten, jack, queen, king and ace of diamonds, are well-shuffled with their face downwards. One card is then picked up at random.
\begin{enumerate}
\item
What is the probability that the card is the queen? 
\item
If the queen is drawn and put aside, what is the probability that the second card picked up is (a) an ace? (b) a queen?\\
\end{enumerate}
\solution
		%\input{ncert/10/15/1/15/defs.tex}
	\item A bag contains $5$ red balls and some blue balls. If the probability of drawing a blue ball is double that if a red ball, determine the number of blue balls in the bag. 
		\\
\solution
		%\input{ncert/10/15/2/3/defs.tex}
	\item A card is selected from a pack of 52 cards.
 \begin{enumerate}[label=(\alph*)] 
                 \item How many points are there in the sample space?
                 \item Calculate the probability that the card is an ace of spades.
                 \item Calculate the probability that the card is (i) an ace and (ii) black card.
 \end{enumerate}
\solution
		%\input{ncert/11/16/3/4/main.tex}
\item Four cards are drawn from a well-shuffled deck of 52 cards. What is the probability of obtaining 3 diamonds and one spade.
\\
\solution
		%\input{ncert/11/16/4/2/defs.tex}
\item In a certain lottery 10,000 tickets are sold and ten equal prizes are awarded. What is the probability of not getting a prize if you buy (a) one ticket (b) two tickets (c) 10 tickets ?	
\\
\solution
		%\input{ncert/11/16/4/4/defs.tex}
		%
\item 
Out of 100 students, two sections of 40 and 60 are formed. If you and your friend are among the 100 students, what is the probability that
\begin{enumerate}
\item you both enter the same section?
\item you both enter the different sections?
\end{enumerate}
\solution
		%\input{ncert/11/16/4/5/defs.tex}
	\item 
The number lock of a suitcase has 4 wheels each labelled with ten digits i.e. from 0 to 9.The lock opens with a sequence of four digits with no repeats.What is the probability of a person getting the right sequence to open the suitcase.
\\
\solution
		%\input{ncert/11/16/4/10/defs.tex}
		%
\item 
Two cards are drawn at random and without replacement from a pack of 52 playing cards. Find the probability that both the cards are black.
\\
\solution
		%\input{ncert/12/13/2/2/defs.tex}
		\item A box of oranges is inspected by examining three randomly selected oranges drawn without replacement. If all the three oranges are good, the box is approved for sale, otherwise, it is rejected. Find the probability that a box containing 15 oranges out of which 12 are good and 3 are bad ones will be approved for sale.
		\label{ncert/12/13/2/3/defs.tex}
		\item Two balls are drawn at random with replacement from a box containing 10 black and 8 red balls. Find the probability that
		\label{ncert/12/13/2/12}
\begin{enumerate}
\item both balls are red.
\item first ball is black and second is red.
\item one of them is black and other is red.
\end{enumerate}

\item In a hostel, 60\% of the students read Hindi newspaper, 40\% read English newspaper and 20\% read both Hindi and English newspapers. A student is selected at random.
		\label{ncert/12/13/2/15}
\begin{enumerate}
\item Find the probability that she reads neither Hindi nor English newspapers.
\item If she reads Hindi newspaper, find the probability that she reads English newspaper.
\item If she reads English newspaper, find the probability that she reads Hindi newspaper.\\
\end{enumerate}
\item The probability of obtaining an even prime number on each die, when a pair of dice is rolled is 
\begin{enumerate}
    \item $0$ 
    
    \item $\frac{1}{3}$ 
    
    \item $\frac{1}{12}$ 
    
    \item $\frac{1}{36}$ 
\end{enumerate}
\solution
		%\input{ncert/12/13/2/17/defs.tex}
	\item A bag contains 4 red and 4 black balls, another bag contains 2 red and 6 black balls. One of the two bags is selected at random and a ball is drawn from the bag which is found to be red. Find the probability that the ball is drawn from the first bag.
\\
\solution
		%\input{ncert/12/13/3/2/main.tex}
  \item
  Cards with numbers 2 to 101 are placed in a box. A card is selected at random.Find the probability that the card has
\begin{enumerate}[label=(\roman*)]
	\item an even number 
	\item a square number
\end{enumerate}
\solution
%\input{exemplar/10/13/3/32/main.tex}
\item
The king, queen and jack of clubs are removed from a deck of 52 playing cards and then well shuffled. Now one card is drawn at random from the remaining cards.  Determine the probability that the card is
\begin{enumerate}[label=(\roman*)]
\item a club
\item 10 of hearts
\end{enumerate}
\solution
%\input{exemplar/10/13/3/29/main.tex}
\item A team of medical students doing their internship have to assist during surgeries
at a city hospital. The probabilities of surgeries rated as very complex, complex,
routine, simple or very simple are respectively, 0.15, 0.20, 0.31, 0.26, .08. Find
the probabilities that a particular surgery will be rated
\begin{enumerate}
	\item complex or very complex;
	\item neither very complex nor very simple;
	\item routine or complex
	\item routine or simple
\end{enumerate}
\solution
%\input{exemplar/11/16/3/8(1)/main.tex}
\item A card is selected from a pack of 52 cards.
\begin{enumerate}[label=(\alph*)]
    \item How many points are there in the sample space?
    \item Calculate the probability that the card is an ace of spades.
    \item Calculate the probability that the card is (i) an ace and (ii) black card.
\end{enumerate}
\solution
%\input{exemplar/11/16/3/4/main2.tex}
\item The probability that a non leap year selected at random will contain 53 sundays.
\\
\solution
%\input{exemplar/10/13/1/19/main.tex}
\item One of the four persons John, Rita, Aslam or Gurpreet will be promoted next
month. Consequently the sample space consists of four elementary outcomes
S = {John promoted, Rita promoted, Aslam promoted, Gurpreet promoted}
You are told that the chances of John’s promotion is same as that of Gurpreet,
Rita’s chances of promotion are twice as likely as Johns. Aslam’s chances are
four times that of John.
\begin{enumerate}
	\item Determine
	\begin{enumerate}
		\item P (John promoted)
		\item P (Rita promoted)
		\item P (Aslam promoted)
		\item P (Gurpreet promoted)
	\end{enumerate}
	\item If A = {John promoted or Gurpreet promoted}, find P (A).
\end{enumerate}
\solution
%\input{exemplar/11/16/3/10/main.tex}
\item A card is drawn from a deck of 52 cards. Find the probability of getting a king or a heart or a red card.\\
\solution
%\input{exemplar/11/16/3/15/main.tex}
\item The probability that a student will pass his examination is 0.73, the probability of
the student getting a compartment is 0.13, and the probability that the student will
either pass or get compartment is 0.96. State True or False.\\
\solution
%\input{exemplar/11/16/3/31/main.tex}
\item A card is selected from a pack of 52 cards\\
\begin{enumerate}[label=(\alph*)]
\item How many points are there in the sample space?
\item Calculate the probability that the cards is an ace of spades.
\item Calculate the probability that the card is (i) an ace (ii)black card.\\
\end{enumerate}
%\input{ncert/11/16/3/4_1/Prob_4.tex}
\item In a non-leap year, the probability of having 53 tuesdays or 53 wednesdays is\\
\solution
%\input{exemplar/11/16/3/18/main.tex}
\item There are 1000 sealed envelopes in a box, 10 of them contain a cash prize of
Rs 100 each, 100 of them contain a cash prize of Rs 50 each and 200 of them
contain a cash prize of Rs 10 each and rest do not contain any cash prize. If they
are well shuffled and an envelope is picked up out, what is the probability that it
contains no cash prize?\\
\solution
%\input{exemplar/10/13/3/34/main.tex}
\item 
A die is thrown and a card is selected at random from a deck of 52 playing cards. The probability of getting an even number on the die and a spade card.\\
\solution
%\input{exemplar/12/13/3/78/main.tex}
\item
If 4-digit numbers greater than 5,000 are randomly formed from the digits 0, 1, 3, 5, and 7, what is the probability of forming a number divisible by 5 when:
\begin{enumerate}
    \item The digits are repeated?
    \item The repetition of digits is not allowed?
\end{enumerate}
\solution
%\input{ncert/11/16/4/9/main.tex}
\item Consider the probability space $\brak{\Omega, \mathcal{G}, P}$ where $\Omega = [0,2]$ and $\mathcal{G} = \cbrak{\phi, \Omega, [0,1], (1,2]}$. Let $X$ and $Y$ be two functions on $\Omega$ defined as
\begin{align*}
    X(\omega) = 
    \begin{cases}
        1 & \text{if }\omega \in [0, 1]\\
        2 & \text{if }\omega \in (1, 2]
    \end{cases}
\end{align*}
and
\begin{align*}
    Y(\omega) = 
    \begin{cases}
        2 & \text{if }\omega \in [0, 1.5]\\
        3 & \text{if }\omega \in (1.5, 2].
    \end{cases}
\end{align*}
Then which one of the following statements is true?
\begin{enumerate}
    \item [(A)] $X$ is a random variable with respect to $\mathcal{G}$, but $Y$ is not a random variable with respect to $\mathcal{G}$.
    \item [(B)] $Y$ is a random variable with respect to $\mathcal{G}$, but $X$ is not a random variable with respect to $\mathcal{G}$.
    \item [(C)] Neither $X$ nor $Y$ is a random variable with respect to $\mathcal{G}$.
    \item [(D)] Both $X$ and $Y$ are random variables with respect to $\mathcal{G}$.
\end{enumerate} \hfill (GATE ST 2023)\\
\solution
%\input{gate/ST/2023/14/main.tex}
	\item  A die is loaded in such a way that each odd number is twice as likely to occur as
each even number. Find $P(G)$, where $G$ is the event that a number greater than
3 occurs on a single roll of the die.
\\
\solution
		%\input{exemplar/11/16/3/5/main.tex}
	\item All the jacks, queens and kings are removed from a deck of 52 playing cards. The remaining cards are well shuffled and then one card is drawn at random. Giving ace a value 1 similar value for other cards, find the probability that the card has a value 
		\begin{enumerate}
			\item 7
			\item greater than 7
			\item less than 7
		\end{enumerate}
		%\input{exemplar/10/13/3/30/main.tex}
  \item A Lot consists of 48 mobile phones of which 42 are good, 3 have only minor defects and 3 have major defects.Varnika will buy a phone if it is good but the trader will only buy a mobile if it has no major defects. One phone is selected at random from the lot. What is the probability that it is
\begin{enumerate}
	\item acceptable to Varnika?
            \item acceptable to the trader?
\end{enumerate}
\solution
	%\input{exemplar/10/13/3/40/main.tex}
 \item A student says that if you throw a die, it will show up 1 or not 1. Therefore, the probability of getting 1 and the probability of getting 'not 1' each is equal to $\frac{1}{2}$. Is this correct? Give reasons.\\
 \solution
        %\input{exemplar/10/13/2/9/main.tex}
   \item Four candidates A, B, C, D have ap-
plied for the assignment to coach a school cricket
team. If A is twice as likely to be selected as B, and
B and C are given about the same chance of being
selected, while C is twice as likely to be selected
as D, what are the probabilities that
\begin{enumerate}
\item C will be selected?
\item A will not be selected?
\end{enumerate}
	%\input{exemplar/11/16/3/9/main.tex}
 \item A bag contain 24 balls of which $x$ balls are red, $2x$ are white and $3x$ are blue. A ball is selected at random, What is the probability that it is
\begin{enumerate}[label=\alph*)]
\item not red ?
\item white ?
\end{enumerate}
%\input{exemplar/10/13/3/41/main.tex}
If the letters of the word ASSASSINATION are arranged at random. Find the Probability that
\begin{enumerate}[label=(\alph*)]
\item Four $S's$ come consecutively in the word
\item Two  $I's$ and two $N's$ come together
\item All $A's$ are not coming together
\item No two $A's$ are coming together
\end{enumerate}
%\input{exemplar/11/16/3/14/main.tex}
	\item One urn contains two black balls (labelled B1 and B2) and one white ball. A
	second urn contains one black ball and two white balls (labelled W1 and W2).
	Suppose the following experiment is performed. One of the two urns is chosen
	at random. Next a ball is randomly chosen from the urn. Then a second ball is
	chosen at random from the same urn without replacing the first ball.
	
	\begin{enumerate}
	\item What is the probability that two black balls are chosen?
	
	\item What is the probability that two balls of opposite colour are chosen?
	\end{enumerate}
	\solution
	%\input{exemplar/11/16/3/12/main1.tex}
\end{enumerate}

\item In a certain lottery 10,000 tickets are sold and ten equal prizes are awarded. What is the probability of not getting a prize if you buy (a) one ticket (b) two tickets (c) 10 tickets ?	
\\
\solution
		%\begin{enumerate}[label=\thesection.\arabic*,ref=\thesection.\theenumi]
	\item One card is drawn from a well-shuffled deck of 52 cards. Find the probability of getting
\begin{enumerate}
\item A king of red colour 
\item A face card 
\item A red face card
\item The jack of hearts
\item A spade
\item The queen of diamonds

\end{enumerate}
\solution
		%\input{ncert/10/15/1/14/main.tex}
	\item Five cards—the ten, jack, queen, king and ace of diamonds, are well-shuffled with their face downwards. One card is then picked up at random.
\begin{enumerate}
\item
What is the probability that the card is the queen? 
\item
If the queen is drawn and put aside, what is the probability that the second card picked up is (a) an ace? (b) a queen?\\
\end{enumerate}
\solution
		%\input{ncert/10/15/1/15/defs.tex}
	\item A bag contains $5$ red balls and some blue balls. If the probability of drawing a blue ball is double that if a red ball, determine the number of blue balls in the bag. 
		\\
\solution
		%\input{ncert/10/15/2/3/defs.tex}
	\item A card is selected from a pack of 52 cards.
 \begin{enumerate}[label=(\alph*)] 
                 \item How many points are there in the sample space?
                 \item Calculate the probability that the card is an ace of spades.
                 \item Calculate the probability that the card is (i) an ace and (ii) black card.
 \end{enumerate}
\solution
		%\input{ncert/11/16/3/4/main.tex}
\item Four cards are drawn from a well-shuffled deck of 52 cards. What is the probability of obtaining 3 diamonds and one spade.
\\
\solution
		%\input{ncert/11/16/4/2/defs.tex}
\item In a certain lottery 10,000 tickets are sold and ten equal prizes are awarded. What is the probability of not getting a prize if you buy (a) one ticket (b) two tickets (c) 10 tickets ?	
\\
\solution
		%\input{ncert/11/16/4/4/defs.tex}
		%
\item 
Out of 100 students, two sections of 40 and 60 are formed. If you and your friend are among the 100 students, what is the probability that
\begin{enumerate}
\item you both enter the same section?
\item you both enter the different sections?
\end{enumerate}
\solution
		%\input{ncert/11/16/4/5/defs.tex}
	\item 
The number lock of a suitcase has 4 wheels each labelled with ten digits i.e. from 0 to 9.The lock opens with a sequence of four digits with no repeats.What is the probability of a person getting the right sequence to open the suitcase.
\\
\solution
		%\input{ncert/11/16/4/10/defs.tex}
		%
\item 
Two cards are drawn at random and without replacement from a pack of 52 playing cards. Find the probability that both the cards are black.
\\
\solution
		%\input{ncert/12/13/2/2/defs.tex}
		\item A box of oranges is inspected by examining three randomly selected oranges drawn without replacement. If all the three oranges are good, the box is approved for sale, otherwise, it is rejected. Find the probability that a box containing 15 oranges out of which 12 are good and 3 are bad ones will be approved for sale.
		\label{ncert/12/13/2/3/defs.tex}
		\item Two balls are drawn at random with replacement from a box containing 10 black and 8 red balls. Find the probability that
		\label{ncert/12/13/2/12}
\begin{enumerate}
\item both balls are red.
\item first ball is black and second is red.
\item one of them is black and other is red.
\end{enumerate}

\item In a hostel, 60\% of the students read Hindi newspaper, 40\% read English newspaper and 20\% read both Hindi and English newspapers. A student is selected at random.
		\label{ncert/12/13/2/15}
\begin{enumerate}
\item Find the probability that she reads neither Hindi nor English newspapers.
\item If she reads Hindi newspaper, find the probability that she reads English newspaper.
\item If she reads English newspaper, find the probability that she reads Hindi newspaper.\\
\end{enumerate}
\item The probability of obtaining an even prime number on each die, when a pair of dice is rolled is 
\begin{enumerate}
    \item $0$ 
    
    \item $\frac{1}{3}$ 
    
    \item $\frac{1}{12}$ 
    
    \item $\frac{1}{36}$ 
\end{enumerate}
\solution
		%\input{ncert/12/13/2/17/defs.tex}
	\item A bag contains 4 red and 4 black balls, another bag contains 2 red and 6 black balls. One of the two bags is selected at random and a ball is drawn from the bag which is found to be red. Find the probability that the ball is drawn from the first bag.
\\
\solution
		%\input{ncert/12/13/3/2/main.tex}
  \item
  Cards with numbers 2 to 101 are placed in a box. A card is selected at random.Find the probability that the card has
\begin{enumerate}[label=(\roman*)]
	\item an even number 
	\item a square number
\end{enumerate}
\solution
%\input{exemplar/10/13/3/32/main.tex}
\item
The king, queen and jack of clubs are removed from a deck of 52 playing cards and then well shuffled. Now one card is drawn at random from the remaining cards.  Determine the probability that the card is
\begin{enumerate}[label=(\roman*)]
\item a club
\item 10 of hearts
\end{enumerate}
\solution
%\input{exemplar/10/13/3/29/main.tex}
\item A team of medical students doing their internship have to assist during surgeries
at a city hospital. The probabilities of surgeries rated as very complex, complex,
routine, simple or very simple are respectively, 0.15, 0.20, 0.31, 0.26, .08. Find
the probabilities that a particular surgery will be rated
\begin{enumerate}
	\item complex or very complex;
	\item neither very complex nor very simple;
	\item routine or complex
	\item routine or simple
\end{enumerate}
\solution
%\input{exemplar/11/16/3/8(1)/main.tex}
\item A card is selected from a pack of 52 cards.
\begin{enumerate}[label=(\alph*)]
    \item How many points are there in the sample space?
    \item Calculate the probability that the card is an ace of spades.
    \item Calculate the probability that the card is (i) an ace and (ii) black card.
\end{enumerate}
\solution
%\input{exemplar/11/16/3/4/main2.tex}
\item The probability that a non leap year selected at random will contain 53 sundays.
\\
\solution
%\input{exemplar/10/13/1/19/main.tex}
\item One of the four persons John, Rita, Aslam or Gurpreet will be promoted next
month. Consequently the sample space consists of four elementary outcomes
S = {John promoted, Rita promoted, Aslam promoted, Gurpreet promoted}
You are told that the chances of John’s promotion is same as that of Gurpreet,
Rita’s chances of promotion are twice as likely as Johns. Aslam’s chances are
four times that of John.
\begin{enumerate}
	\item Determine
	\begin{enumerate}
		\item P (John promoted)
		\item P (Rita promoted)
		\item P (Aslam promoted)
		\item P (Gurpreet promoted)
	\end{enumerate}
	\item If A = {John promoted or Gurpreet promoted}, find P (A).
\end{enumerate}
\solution
%\input{exemplar/11/16/3/10/main.tex}
\item A card is drawn from a deck of 52 cards. Find the probability of getting a king or a heart or a red card.\\
\solution
%\input{exemplar/11/16/3/15/main.tex}
\item The probability that a student will pass his examination is 0.73, the probability of
the student getting a compartment is 0.13, and the probability that the student will
either pass or get compartment is 0.96. State True or False.\\
\solution
%\input{exemplar/11/16/3/31/main.tex}
\item A card is selected from a pack of 52 cards\\
\begin{enumerate}[label=(\alph*)]
\item How many points are there in the sample space?
\item Calculate the probability that the cards is an ace of spades.
\item Calculate the probability that the card is (i) an ace (ii)black card.\\
\end{enumerate}
%\input{ncert/11/16/3/4_1/Prob_4.tex}
\item In a non-leap year, the probability of having 53 tuesdays or 53 wednesdays is\\
\solution
%\input{exemplar/11/16/3/18/main.tex}
\item There are 1000 sealed envelopes in a box, 10 of them contain a cash prize of
Rs 100 each, 100 of them contain a cash prize of Rs 50 each and 200 of them
contain a cash prize of Rs 10 each and rest do not contain any cash prize. If they
are well shuffled and an envelope is picked up out, what is the probability that it
contains no cash prize?\\
\solution
%\input{exemplar/10/13/3/34/main.tex}
\item 
A die is thrown and a card is selected at random from a deck of 52 playing cards. The probability of getting an even number on the die and a spade card.\\
\solution
%\input{exemplar/12/13/3/78/main.tex}
\item
If 4-digit numbers greater than 5,000 are randomly formed from the digits 0, 1, 3, 5, and 7, what is the probability of forming a number divisible by 5 when:
\begin{enumerate}
    \item The digits are repeated?
    \item The repetition of digits is not allowed?
\end{enumerate}
\solution
%\input{ncert/11/16/4/9/main.tex}
\item Consider the probability space $\brak{\Omega, \mathcal{G}, P}$ where $\Omega = [0,2]$ and $\mathcal{G} = \cbrak{\phi, \Omega, [0,1], (1,2]}$. Let $X$ and $Y$ be two functions on $\Omega$ defined as
\begin{align*}
    X(\omega) = 
    \begin{cases}
        1 & \text{if }\omega \in [0, 1]\\
        2 & \text{if }\omega \in (1, 2]
    \end{cases}
\end{align*}
and
\begin{align*}
    Y(\omega) = 
    \begin{cases}
        2 & \text{if }\omega \in [0, 1.5]\\
        3 & \text{if }\omega \in (1.5, 2].
    \end{cases}
\end{align*}
Then which one of the following statements is true?
\begin{enumerate}
    \item [(A)] $X$ is a random variable with respect to $\mathcal{G}$, but $Y$ is not a random variable with respect to $\mathcal{G}$.
    \item [(B)] $Y$ is a random variable with respect to $\mathcal{G}$, but $X$ is not a random variable with respect to $\mathcal{G}$.
    \item [(C)] Neither $X$ nor $Y$ is a random variable with respect to $\mathcal{G}$.
    \item [(D)] Both $X$ and $Y$ are random variables with respect to $\mathcal{G}$.
\end{enumerate} \hfill (GATE ST 2023)\\
\solution
%\input{gate/ST/2023/14/main.tex}
	\item  A die is loaded in such a way that each odd number is twice as likely to occur as
each even number. Find $P(G)$, where $G$ is the event that a number greater than
3 occurs on a single roll of the die.
\\
\solution
		%\input{exemplar/11/16/3/5/main.tex}
	\item All the jacks, queens and kings are removed from a deck of 52 playing cards. The remaining cards are well shuffled and then one card is drawn at random. Giving ace a value 1 similar value for other cards, find the probability that the card has a value 
		\begin{enumerate}
			\item 7
			\item greater than 7
			\item less than 7
		\end{enumerate}
		%\input{exemplar/10/13/3/30/main.tex}
  \item A Lot consists of 48 mobile phones of which 42 are good, 3 have only minor defects and 3 have major defects.Varnika will buy a phone if it is good but the trader will only buy a mobile if it has no major defects. One phone is selected at random from the lot. What is the probability that it is
\begin{enumerate}
	\item acceptable to Varnika?
            \item acceptable to the trader?
\end{enumerate}
\solution
	%\input{exemplar/10/13/3/40/main.tex}
 \item A student says that if you throw a die, it will show up 1 or not 1. Therefore, the probability of getting 1 and the probability of getting 'not 1' each is equal to $\frac{1}{2}$. Is this correct? Give reasons.\\
 \solution
        %\input{exemplar/10/13/2/9/main.tex}
   \item Four candidates A, B, C, D have ap-
plied for the assignment to coach a school cricket
team. If A is twice as likely to be selected as B, and
B and C are given about the same chance of being
selected, while C is twice as likely to be selected
as D, what are the probabilities that
\begin{enumerate}
\item C will be selected?
\item A will not be selected?
\end{enumerate}
	%\input{exemplar/11/16/3/9/main.tex}
 \item A bag contain 24 balls of which $x$ balls are red, $2x$ are white and $3x$ are blue. A ball is selected at random, What is the probability that it is
\begin{enumerate}[label=\alph*)]
\item not red ?
\item white ?
\end{enumerate}
%\input{exemplar/10/13/3/41/main.tex}
If the letters of the word ASSASSINATION are arranged at random. Find the Probability that
\begin{enumerate}[label=(\alph*)]
\item Four $S's$ come consecutively in the word
\item Two  $I's$ and two $N's$ come together
\item All $A's$ are not coming together
\item No two $A's$ are coming together
\end{enumerate}
%\input{exemplar/11/16/3/14/main.tex}
	\item One urn contains two black balls (labelled B1 and B2) and one white ball. A
	second urn contains one black ball and two white balls (labelled W1 and W2).
	Suppose the following experiment is performed. One of the two urns is chosen
	at random. Next a ball is randomly chosen from the urn. Then a second ball is
	chosen at random from the same urn without replacing the first ball.
	
	\begin{enumerate}
	\item What is the probability that two black balls are chosen?
	
	\item What is the probability that two balls of opposite colour are chosen?
	\end{enumerate}
	\solution
	%\input{exemplar/11/16/3/12/main1.tex}
\end{enumerate}

		%
\item 
Out of 100 students, two sections of 40 and 60 are formed. If you and your friend are among the 100 students, what is the probability that
\begin{enumerate}
\item you both enter the same section?
\item you both enter the different sections?
\end{enumerate}
\solution
		%\begin{enumerate}[label=\thesection.\arabic*,ref=\thesection.\theenumi]
	\item One card is drawn from a well-shuffled deck of 52 cards. Find the probability of getting
\begin{enumerate}
\item A king of red colour 
\item A face card 
\item A red face card
\item The jack of hearts
\item A spade
\item The queen of diamonds

\end{enumerate}
\solution
		%\input{ncert/10/15/1/14/main.tex}
	\item Five cards—the ten, jack, queen, king and ace of diamonds, are well-shuffled with their face downwards. One card is then picked up at random.
\begin{enumerate}
\item
What is the probability that the card is the queen? 
\item
If the queen is drawn and put aside, what is the probability that the second card picked up is (a) an ace? (b) a queen?\\
\end{enumerate}
\solution
		%\input{ncert/10/15/1/15/defs.tex}
	\item A bag contains $5$ red balls and some blue balls. If the probability of drawing a blue ball is double that if a red ball, determine the number of blue balls in the bag. 
		\\
\solution
		%\input{ncert/10/15/2/3/defs.tex}
	\item A card is selected from a pack of 52 cards.
 \begin{enumerate}[label=(\alph*)] 
                 \item How many points are there in the sample space?
                 \item Calculate the probability that the card is an ace of spades.
                 \item Calculate the probability that the card is (i) an ace and (ii) black card.
 \end{enumerate}
\solution
		%\input{ncert/11/16/3/4/main.tex}
\item Four cards are drawn from a well-shuffled deck of 52 cards. What is the probability of obtaining 3 diamonds and one spade.
\\
\solution
		%\input{ncert/11/16/4/2/defs.tex}
\item In a certain lottery 10,000 tickets are sold and ten equal prizes are awarded. What is the probability of not getting a prize if you buy (a) one ticket (b) two tickets (c) 10 tickets ?	
\\
\solution
		%\input{ncert/11/16/4/4/defs.tex}
		%
\item 
Out of 100 students, two sections of 40 and 60 are formed. If you and your friend are among the 100 students, what is the probability that
\begin{enumerate}
\item you both enter the same section?
\item you both enter the different sections?
\end{enumerate}
\solution
		%\input{ncert/11/16/4/5/defs.tex}
	\item 
The number lock of a suitcase has 4 wheels each labelled with ten digits i.e. from 0 to 9.The lock opens with a sequence of four digits with no repeats.What is the probability of a person getting the right sequence to open the suitcase.
\\
\solution
		%\input{ncert/11/16/4/10/defs.tex}
		%
\item 
Two cards are drawn at random and without replacement from a pack of 52 playing cards. Find the probability that both the cards are black.
\\
\solution
		%\input{ncert/12/13/2/2/defs.tex}
		\item A box of oranges is inspected by examining three randomly selected oranges drawn without replacement. If all the three oranges are good, the box is approved for sale, otherwise, it is rejected. Find the probability that a box containing 15 oranges out of which 12 are good and 3 are bad ones will be approved for sale.
		\label{ncert/12/13/2/3/defs.tex}
		\item Two balls are drawn at random with replacement from a box containing 10 black and 8 red balls. Find the probability that
		\label{ncert/12/13/2/12}
\begin{enumerate}
\item both balls are red.
\item first ball is black and second is red.
\item one of them is black and other is red.
\end{enumerate}

\item In a hostel, 60\% of the students read Hindi newspaper, 40\% read English newspaper and 20\% read both Hindi and English newspapers. A student is selected at random.
		\label{ncert/12/13/2/15}
\begin{enumerate}
\item Find the probability that she reads neither Hindi nor English newspapers.
\item If she reads Hindi newspaper, find the probability that she reads English newspaper.
\item If she reads English newspaper, find the probability that she reads Hindi newspaper.\\
\end{enumerate}
\item The probability of obtaining an even prime number on each die, when a pair of dice is rolled is 
\begin{enumerate}
    \item $0$ 
    
    \item $\frac{1}{3}$ 
    
    \item $\frac{1}{12}$ 
    
    \item $\frac{1}{36}$ 
\end{enumerate}
\solution
		%\input{ncert/12/13/2/17/defs.tex}
	\item A bag contains 4 red and 4 black balls, another bag contains 2 red and 6 black balls. One of the two bags is selected at random and a ball is drawn from the bag which is found to be red. Find the probability that the ball is drawn from the first bag.
\\
\solution
		%\input{ncert/12/13/3/2/main.tex}
  \item
  Cards with numbers 2 to 101 are placed in a box. A card is selected at random.Find the probability that the card has
\begin{enumerate}[label=(\roman*)]
	\item an even number 
	\item a square number
\end{enumerate}
\solution
%\input{exemplar/10/13/3/32/main.tex}
\item
The king, queen and jack of clubs are removed from a deck of 52 playing cards and then well shuffled. Now one card is drawn at random from the remaining cards.  Determine the probability that the card is
\begin{enumerate}[label=(\roman*)]
\item a club
\item 10 of hearts
\end{enumerate}
\solution
%\input{exemplar/10/13/3/29/main.tex}
\item A team of medical students doing their internship have to assist during surgeries
at a city hospital. The probabilities of surgeries rated as very complex, complex,
routine, simple or very simple are respectively, 0.15, 0.20, 0.31, 0.26, .08. Find
the probabilities that a particular surgery will be rated
\begin{enumerate}
	\item complex or very complex;
	\item neither very complex nor very simple;
	\item routine or complex
	\item routine or simple
\end{enumerate}
\solution
%\input{exemplar/11/16/3/8(1)/main.tex}
\item A card is selected from a pack of 52 cards.
\begin{enumerate}[label=(\alph*)]
    \item How many points are there in the sample space?
    \item Calculate the probability that the card is an ace of spades.
    \item Calculate the probability that the card is (i) an ace and (ii) black card.
\end{enumerate}
\solution
%\input{exemplar/11/16/3/4/main2.tex}
\item The probability that a non leap year selected at random will contain 53 sundays.
\\
\solution
%\input{exemplar/10/13/1/19/main.tex}
\item One of the four persons John, Rita, Aslam or Gurpreet will be promoted next
month. Consequently the sample space consists of four elementary outcomes
S = {John promoted, Rita promoted, Aslam promoted, Gurpreet promoted}
You are told that the chances of John’s promotion is same as that of Gurpreet,
Rita’s chances of promotion are twice as likely as Johns. Aslam’s chances are
four times that of John.
\begin{enumerate}
	\item Determine
	\begin{enumerate}
		\item P (John promoted)
		\item P (Rita promoted)
		\item P (Aslam promoted)
		\item P (Gurpreet promoted)
	\end{enumerate}
	\item If A = {John promoted or Gurpreet promoted}, find P (A).
\end{enumerate}
\solution
%\input{exemplar/11/16/3/10/main.tex}
\item A card is drawn from a deck of 52 cards. Find the probability of getting a king or a heart or a red card.\\
\solution
%\input{exemplar/11/16/3/15/main.tex}
\item The probability that a student will pass his examination is 0.73, the probability of
the student getting a compartment is 0.13, and the probability that the student will
either pass or get compartment is 0.96. State True or False.\\
\solution
%\input{exemplar/11/16/3/31/main.tex}
\item A card is selected from a pack of 52 cards\\
\begin{enumerate}[label=(\alph*)]
\item How many points are there in the sample space?
\item Calculate the probability that the cards is an ace of spades.
\item Calculate the probability that the card is (i) an ace (ii)black card.\\
\end{enumerate}
%\input{ncert/11/16/3/4_1/Prob_4.tex}
\item In a non-leap year, the probability of having 53 tuesdays or 53 wednesdays is\\
\solution
%\input{exemplar/11/16/3/18/main.tex}
\item There are 1000 sealed envelopes in a box, 10 of them contain a cash prize of
Rs 100 each, 100 of them contain a cash prize of Rs 50 each and 200 of them
contain a cash prize of Rs 10 each and rest do not contain any cash prize. If they
are well shuffled and an envelope is picked up out, what is the probability that it
contains no cash prize?\\
\solution
%\input{exemplar/10/13/3/34/main.tex}
\item 
A die is thrown and a card is selected at random from a deck of 52 playing cards. The probability of getting an even number on the die and a spade card.\\
\solution
%\input{exemplar/12/13/3/78/main.tex}
\item
If 4-digit numbers greater than 5,000 are randomly formed from the digits 0, 1, 3, 5, and 7, what is the probability of forming a number divisible by 5 when:
\begin{enumerate}
    \item The digits are repeated?
    \item The repetition of digits is not allowed?
\end{enumerate}
\solution
%\input{ncert/11/16/4/9/main.tex}
\item Consider the probability space $\brak{\Omega, \mathcal{G}, P}$ where $\Omega = [0,2]$ and $\mathcal{G} = \cbrak{\phi, \Omega, [0,1], (1,2]}$. Let $X$ and $Y$ be two functions on $\Omega$ defined as
\begin{align*}
    X(\omega) = 
    \begin{cases}
        1 & \text{if }\omega \in [0, 1]\\
        2 & \text{if }\omega \in (1, 2]
    \end{cases}
\end{align*}
and
\begin{align*}
    Y(\omega) = 
    \begin{cases}
        2 & \text{if }\omega \in [0, 1.5]\\
        3 & \text{if }\omega \in (1.5, 2].
    \end{cases}
\end{align*}
Then which one of the following statements is true?
\begin{enumerate}
    \item [(A)] $X$ is a random variable with respect to $\mathcal{G}$, but $Y$ is not a random variable with respect to $\mathcal{G}$.
    \item [(B)] $Y$ is a random variable with respect to $\mathcal{G}$, but $X$ is not a random variable with respect to $\mathcal{G}$.
    \item [(C)] Neither $X$ nor $Y$ is a random variable with respect to $\mathcal{G}$.
    \item [(D)] Both $X$ and $Y$ are random variables with respect to $\mathcal{G}$.
\end{enumerate} \hfill (GATE ST 2023)\\
\solution
%\input{gate/ST/2023/14/main.tex}
	\item  A die is loaded in such a way that each odd number is twice as likely to occur as
each even number. Find $P(G)$, where $G$ is the event that a number greater than
3 occurs on a single roll of the die.
\\
\solution
		%\input{exemplar/11/16/3/5/main.tex}
	\item All the jacks, queens and kings are removed from a deck of 52 playing cards. The remaining cards are well shuffled and then one card is drawn at random. Giving ace a value 1 similar value for other cards, find the probability that the card has a value 
		\begin{enumerate}
			\item 7
			\item greater than 7
			\item less than 7
		\end{enumerate}
		%\input{exemplar/10/13/3/30/main.tex}
  \item A Lot consists of 48 mobile phones of which 42 are good, 3 have only minor defects and 3 have major defects.Varnika will buy a phone if it is good but the trader will only buy a mobile if it has no major defects. One phone is selected at random from the lot. What is the probability that it is
\begin{enumerate}
	\item acceptable to Varnika?
            \item acceptable to the trader?
\end{enumerate}
\solution
	%\input{exemplar/10/13/3/40/main.tex}
 \item A student says that if you throw a die, it will show up 1 or not 1. Therefore, the probability of getting 1 and the probability of getting 'not 1' each is equal to $\frac{1}{2}$. Is this correct? Give reasons.\\
 \solution
        %\input{exemplar/10/13/2/9/main.tex}
   \item Four candidates A, B, C, D have ap-
plied for the assignment to coach a school cricket
team. If A is twice as likely to be selected as B, and
B and C are given about the same chance of being
selected, while C is twice as likely to be selected
as D, what are the probabilities that
\begin{enumerate}
\item C will be selected?
\item A will not be selected?
\end{enumerate}
	%\input{exemplar/11/16/3/9/main.tex}
 \item A bag contain 24 balls of which $x$ balls are red, $2x$ are white and $3x$ are blue. A ball is selected at random, What is the probability that it is
\begin{enumerate}[label=\alph*)]
\item not red ?
\item white ?
\end{enumerate}
%\input{exemplar/10/13/3/41/main.tex}
If the letters of the word ASSASSINATION are arranged at random. Find the Probability that
\begin{enumerate}[label=(\alph*)]
\item Four $S's$ come consecutively in the word
\item Two  $I's$ and two $N's$ come together
\item All $A's$ are not coming together
\item No two $A's$ are coming together
\end{enumerate}
%\input{exemplar/11/16/3/14/main.tex}
	\item One urn contains two black balls (labelled B1 and B2) and one white ball. A
	second urn contains one black ball and two white balls (labelled W1 and W2).
	Suppose the following experiment is performed. One of the two urns is chosen
	at random. Next a ball is randomly chosen from the urn. Then a second ball is
	chosen at random from the same urn without replacing the first ball.
	
	\begin{enumerate}
	\item What is the probability that two black balls are chosen?
	
	\item What is the probability that two balls of opposite colour are chosen?
	\end{enumerate}
	\solution
	%\input{exemplar/11/16/3/12/main1.tex}
\end{enumerate}

	\item 
The number lock of a suitcase has 4 wheels each labelled with ten digits i.e. from 0 to 9.The lock opens with a sequence of four digits with no repeats.What is the probability of a person getting the right sequence to open the suitcase.
\\
\solution
		%\begin{enumerate}[label=\thesection.\arabic*,ref=\thesection.\theenumi]
	\item One card is drawn from a well-shuffled deck of 52 cards. Find the probability of getting
\begin{enumerate}
\item A king of red colour 
\item A face card 
\item A red face card
\item The jack of hearts
\item A spade
\item The queen of diamonds

\end{enumerate}
\solution
		%\input{ncert/10/15/1/14/main.tex}
	\item Five cards—the ten, jack, queen, king and ace of diamonds, are well-shuffled with their face downwards. One card is then picked up at random.
\begin{enumerate}
\item
What is the probability that the card is the queen? 
\item
If the queen is drawn and put aside, what is the probability that the second card picked up is (a) an ace? (b) a queen?\\
\end{enumerate}
\solution
		%\input{ncert/10/15/1/15/defs.tex}
	\item A bag contains $5$ red balls and some blue balls. If the probability of drawing a blue ball is double that if a red ball, determine the number of blue balls in the bag. 
		\\
\solution
		%\input{ncert/10/15/2/3/defs.tex}
	\item A card is selected from a pack of 52 cards.
 \begin{enumerate}[label=(\alph*)] 
                 \item How many points are there in the sample space?
                 \item Calculate the probability that the card is an ace of spades.
                 \item Calculate the probability that the card is (i) an ace and (ii) black card.
 \end{enumerate}
\solution
		%\input{ncert/11/16/3/4/main.tex}
\item Four cards are drawn from a well-shuffled deck of 52 cards. What is the probability of obtaining 3 diamonds and one spade.
\\
\solution
		%\input{ncert/11/16/4/2/defs.tex}
\item In a certain lottery 10,000 tickets are sold and ten equal prizes are awarded. What is the probability of not getting a prize if you buy (a) one ticket (b) two tickets (c) 10 tickets ?	
\\
\solution
		%\input{ncert/11/16/4/4/defs.tex}
		%
\item 
Out of 100 students, two sections of 40 and 60 are formed. If you and your friend are among the 100 students, what is the probability that
\begin{enumerate}
\item you both enter the same section?
\item you both enter the different sections?
\end{enumerate}
\solution
		%\input{ncert/11/16/4/5/defs.tex}
	\item 
The number lock of a suitcase has 4 wheels each labelled with ten digits i.e. from 0 to 9.The lock opens with a sequence of four digits with no repeats.What is the probability of a person getting the right sequence to open the suitcase.
\\
\solution
		%\input{ncert/11/16/4/10/defs.tex}
		%
\item 
Two cards are drawn at random and without replacement from a pack of 52 playing cards. Find the probability that both the cards are black.
\\
\solution
		%\input{ncert/12/13/2/2/defs.tex}
		\item A box of oranges is inspected by examining three randomly selected oranges drawn without replacement. If all the three oranges are good, the box is approved for sale, otherwise, it is rejected. Find the probability that a box containing 15 oranges out of which 12 are good and 3 are bad ones will be approved for sale.
		\label{ncert/12/13/2/3/defs.tex}
		\item Two balls are drawn at random with replacement from a box containing 10 black and 8 red balls. Find the probability that
		\label{ncert/12/13/2/12}
\begin{enumerate}
\item both balls are red.
\item first ball is black and second is red.
\item one of them is black and other is red.
\end{enumerate}

\item In a hostel, 60\% of the students read Hindi newspaper, 40\% read English newspaper and 20\% read both Hindi and English newspapers. A student is selected at random.
		\label{ncert/12/13/2/15}
\begin{enumerate}
\item Find the probability that she reads neither Hindi nor English newspapers.
\item If she reads Hindi newspaper, find the probability that she reads English newspaper.
\item If she reads English newspaper, find the probability that she reads Hindi newspaper.\\
\end{enumerate}
\item The probability of obtaining an even prime number on each die, when a pair of dice is rolled is 
\begin{enumerate}
    \item $0$ 
    
    \item $\frac{1}{3}$ 
    
    \item $\frac{1}{12}$ 
    
    \item $\frac{1}{36}$ 
\end{enumerate}
\solution
		%\input{ncert/12/13/2/17/defs.tex}
	\item A bag contains 4 red and 4 black balls, another bag contains 2 red and 6 black balls. One of the two bags is selected at random and a ball is drawn from the bag which is found to be red. Find the probability that the ball is drawn from the first bag.
\\
\solution
		%\input{ncert/12/13/3/2/main.tex}
  \item
  Cards with numbers 2 to 101 are placed in a box. A card is selected at random.Find the probability that the card has
\begin{enumerate}[label=(\roman*)]
	\item an even number 
	\item a square number
\end{enumerate}
\solution
%\input{exemplar/10/13/3/32/main.tex}
\item
The king, queen and jack of clubs are removed from a deck of 52 playing cards and then well shuffled. Now one card is drawn at random from the remaining cards.  Determine the probability that the card is
\begin{enumerate}[label=(\roman*)]
\item a club
\item 10 of hearts
\end{enumerate}
\solution
%\input{exemplar/10/13/3/29/main.tex}
\item A team of medical students doing their internship have to assist during surgeries
at a city hospital. The probabilities of surgeries rated as very complex, complex,
routine, simple or very simple are respectively, 0.15, 0.20, 0.31, 0.26, .08. Find
the probabilities that a particular surgery will be rated
\begin{enumerate}
	\item complex or very complex;
	\item neither very complex nor very simple;
	\item routine or complex
	\item routine or simple
\end{enumerate}
\solution
%\input{exemplar/11/16/3/8(1)/main.tex}
\item A card is selected from a pack of 52 cards.
\begin{enumerate}[label=(\alph*)]
    \item How many points are there in the sample space?
    \item Calculate the probability that the card is an ace of spades.
    \item Calculate the probability that the card is (i) an ace and (ii) black card.
\end{enumerate}
\solution
%\input{exemplar/11/16/3/4/main2.tex}
\item The probability that a non leap year selected at random will contain 53 sundays.
\\
\solution
%\input{exemplar/10/13/1/19/main.tex}
\item One of the four persons John, Rita, Aslam or Gurpreet will be promoted next
month. Consequently the sample space consists of four elementary outcomes
S = {John promoted, Rita promoted, Aslam promoted, Gurpreet promoted}
You are told that the chances of John’s promotion is same as that of Gurpreet,
Rita’s chances of promotion are twice as likely as Johns. Aslam’s chances are
four times that of John.
\begin{enumerate}
	\item Determine
	\begin{enumerate}
		\item P (John promoted)
		\item P (Rita promoted)
		\item P (Aslam promoted)
		\item P (Gurpreet promoted)
	\end{enumerate}
	\item If A = {John promoted or Gurpreet promoted}, find P (A).
\end{enumerate}
\solution
%\input{exemplar/11/16/3/10/main.tex}
\item A card is drawn from a deck of 52 cards. Find the probability of getting a king or a heart or a red card.\\
\solution
%\input{exemplar/11/16/3/15/main.tex}
\item The probability that a student will pass his examination is 0.73, the probability of
the student getting a compartment is 0.13, and the probability that the student will
either pass or get compartment is 0.96. State True or False.\\
\solution
%\input{exemplar/11/16/3/31/main.tex}
\item A card is selected from a pack of 52 cards\\
\begin{enumerate}[label=(\alph*)]
\item How many points are there in the sample space?
\item Calculate the probability that the cards is an ace of spades.
\item Calculate the probability that the card is (i) an ace (ii)black card.\\
\end{enumerate}
%\input{ncert/11/16/3/4_1/Prob_4.tex}
\item In a non-leap year, the probability of having 53 tuesdays or 53 wednesdays is\\
\solution
%\input{exemplar/11/16/3/18/main.tex}
\item There are 1000 sealed envelopes in a box, 10 of them contain a cash prize of
Rs 100 each, 100 of them contain a cash prize of Rs 50 each and 200 of them
contain a cash prize of Rs 10 each and rest do not contain any cash prize. If they
are well shuffled and an envelope is picked up out, what is the probability that it
contains no cash prize?\\
\solution
%\input{exemplar/10/13/3/34/main.tex}
\item 
A die is thrown and a card is selected at random from a deck of 52 playing cards. The probability of getting an even number on the die and a spade card.\\
\solution
%\input{exemplar/12/13/3/78/main.tex}
\item
If 4-digit numbers greater than 5,000 are randomly formed from the digits 0, 1, 3, 5, and 7, what is the probability of forming a number divisible by 5 when:
\begin{enumerate}
    \item The digits are repeated?
    \item The repetition of digits is not allowed?
\end{enumerate}
\solution
%\input{ncert/11/16/4/9/main.tex}
\item Consider the probability space $\brak{\Omega, \mathcal{G}, P}$ where $\Omega = [0,2]$ and $\mathcal{G} = \cbrak{\phi, \Omega, [0,1], (1,2]}$. Let $X$ and $Y$ be two functions on $\Omega$ defined as
\begin{align*}
    X(\omega) = 
    \begin{cases}
        1 & \text{if }\omega \in [0, 1]\\
        2 & \text{if }\omega \in (1, 2]
    \end{cases}
\end{align*}
and
\begin{align*}
    Y(\omega) = 
    \begin{cases}
        2 & \text{if }\omega \in [0, 1.5]\\
        3 & \text{if }\omega \in (1.5, 2].
    \end{cases}
\end{align*}
Then which one of the following statements is true?
\begin{enumerate}
    \item [(A)] $X$ is a random variable with respect to $\mathcal{G}$, but $Y$ is not a random variable with respect to $\mathcal{G}$.
    \item [(B)] $Y$ is a random variable with respect to $\mathcal{G}$, but $X$ is not a random variable with respect to $\mathcal{G}$.
    \item [(C)] Neither $X$ nor $Y$ is a random variable with respect to $\mathcal{G}$.
    \item [(D)] Both $X$ and $Y$ are random variables with respect to $\mathcal{G}$.
\end{enumerate} \hfill (GATE ST 2023)\\
\solution
%\input{gate/ST/2023/14/main.tex}
	\item  A die is loaded in such a way that each odd number is twice as likely to occur as
each even number. Find $P(G)$, where $G$ is the event that a number greater than
3 occurs on a single roll of the die.
\\
\solution
		%\input{exemplar/11/16/3/5/main.tex}
	\item All the jacks, queens and kings are removed from a deck of 52 playing cards. The remaining cards are well shuffled and then one card is drawn at random. Giving ace a value 1 similar value for other cards, find the probability that the card has a value 
		\begin{enumerate}
			\item 7
			\item greater than 7
			\item less than 7
		\end{enumerate}
		%\input{exemplar/10/13/3/30/main.tex}
  \item A Lot consists of 48 mobile phones of which 42 are good, 3 have only minor defects and 3 have major defects.Varnika will buy a phone if it is good but the trader will only buy a mobile if it has no major defects. One phone is selected at random from the lot. What is the probability that it is
\begin{enumerate}
	\item acceptable to Varnika?
            \item acceptable to the trader?
\end{enumerate}
\solution
	%\input{exemplar/10/13/3/40/main.tex}
 \item A student says that if you throw a die, it will show up 1 or not 1. Therefore, the probability of getting 1 and the probability of getting 'not 1' each is equal to $\frac{1}{2}$. Is this correct? Give reasons.\\
 \solution
        %\input{exemplar/10/13/2/9/main.tex}
   \item Four candidates A, B, C, D have ap-
plied for the assignment to coach a school cricket
team. If A is twice as likely to be selected as B, and
B and C are given about the same chance of being
selected, while C is twice as likely to be selected
as D, what are the probabilities that
\begin{enumerate}
\item C will be selected?
\item A will not be selected?
\end{enumerate}
	%\input{exemplar/11/16/3/9/main.tex}
 \item A bag contain 24 balls of which $x$ balls are red, $2x$ are white and $3x$ are blue. A ball is selected at random, What is the probability that it is
\begin{enumerate}[label=\alph*)]
\item not red ?
\item white ?
\end{enumerate}
%\input{exemplar/10/13/3/41/main.tex}
If the letters of the word ASSASSINATION are arranged at random. Find the Probability that
\begin{enumerate}[label=(\alph*)]
\item Four $S's$ come consecutively in the word
\item Two  $I's$ and two $N's$ come together
\item All $A's$ are not coming together
\item No two $A's$ are coming together
\end{enumerate}
%\input{exemplar/11/16/3/14/main.tex}
	\item One urn contains two black balls (labelled B1 and B2) and one white ball. A
	second urn contains one black ball and two white balls (labelled W1 and W2).
	Suppose the following experiment is performed. One of the two urns is chosen
	at random. Next a ball is randomly chosen from the urn. Then a second ball is
	chosen at random from the same urn without replacing the first ball.
	
	\begin{enumerate}
	\item What is the probability that two black balls are chosen?
	
	\item What is the probability that two balls of opposite colour are chosen?
	\end{enumerate}
	\solution
	%\input{exemplar/11/16/3/12/main1.tex}
\end{enumerate}

		%
\item 
Two cards are drawn at random and without replacement from a pack of 52 playing cards. Find the probability that both the cards are black.
\\
\solution
		%\begin{enumerate}[label=\thesection.\arabic*,ref=\thesection.\theenumi]
	\item One card is drawn from a well-shuffled deck of 52 cards. Find the probability of getting
\begin{enumerate}
\item A king of red colour 
\item A face card 
\item A red face card
\item The jack of hearts
\item A spade
\item The queen of diamonds

\end{enumerate}
\solution
		%\input{ncert/10/15/1/14/main.tex}
	\item Five cards—the ten, jack, queen, king and ace of diamonds, are well-shuffled with their face downwards. One card is then picked up at random.
\begin{enumerate}
\item
What is the probability that the card is the queen? 
\item
If the queen is drawn and put aside, what is the probability that the second card picked up is (a) an ace? (b) a queen?\\
\end{enumerate}
\solution
		%\input{ncert/10/15/1/15/defs.tex}
	\item A bag contains $5$ red balls and some blue balls. If the probability of drawing a blue ball is double that if a red ball, determine the number of blue balls in the bag. 
		\\
\solution
		%\input{ncert/10/15/2/3/defs.tex}
	\item A card is selected from a pack of 52 cards.
 \begin{enumerate}[label=(\alph*)] 
                 \item How many points are there in the sample space?
                 \item Calculate the probability that the card is an ace of spades.
                 \item Calculate the probability that the card is (i) an ace and (ii) black card.
 \end{enumerate}
\solution
		%\input{ncert/11/16/3/4/main.tex}
\item Four cards are drawn from a well-shuffled deck of 52 cards. What is the probability of obtaining 3 diamonds and one spade.
\\
\solution
		%\input{ncert/11/16/4/2/defs.tex}
\item In a certain lottery 10,000 tickets are sold and ten equal prizes are awarded. What is the probability of not getting a prize if you buy (a) one ticket (b) two tickets (c) 10 tickets ?	
\\
\solution
		%\input{ncert/11/16/4/4/defs.tex}
		%
\item 
Out of 100 students, two sections of 40 and 60 are formed. If you and your friend are among the 100 students, what is the probability that
\begin{enumerate}
\item you both enter the same section?
\item you both enter the different sections?
\end{enumerate}
\solution
		%\input{ncert/11/16/4/5/defs.tex}
	\item 
The number lock of a suitcase has 4 wheels each labelled with ten digits i.e. from 0 to 9.The lock opens with a sequence of four digits with no repeats.What is the probability of a person getting the right sequence to open the suitcase.
\\
\solution
		%\input{ncert/11/16/4/10/defs.tex}
		%
\item 
Two cards are drawn at random and without replacement from a pack of 52 playing cards. Find the probability that both the cards are black.
\\
\solution
		%\input{ncert/12/13/2/2/defs.tex}
		\item A box of oranges is inspected by examining three randomly selected oranges drawn without replacement. If all the three oranges are good, the box is approved for sale, otherwise, it is rejected. Find the probability that a box containing 15 oranges out of which 12 are good and 3 are bad ones will be approved for sale.
		\label{ncert/12/13/2/3/defs.tex}
		\item Two balls are drawn at random with replacement from a box containing 10 black and 8 red balls. Find the probability that
		\label{ncert/12/13/2/12}
\begin{enumerate}
\item both balls are red.
\item first ball is black and second is red.
\item one of them is black and other is red.
\end{enumerate}

\item In a hostel, 60\% of the students read Hindi newspaper, 40\% read English newspaper and 20\% read both Hindi and English newspapers. A student is selected at random.
		\label{ncert/12/13/2/15}
\begin{enumerate}
\item Find the probability that she reads neither Hindi nor English newspapers.
\item If she reads Hindi newspaper, find the probability that she reads English newspaper.
\item If she reads English newspaper, find the probability that she reads Hindi newspaper.\\
\end{enumerate}
\item The probability of obtaining an even prime number on each die, when a pair of dice is rolled is 
\begin{enumerate}
    \item $0$ 
    
    \item $\frac{1}{3}$ 
    
    \item $\frac{1}{12}$ 
    
    \item $\frac{1}{36}$ 
\end{enumerate}
\solution
		%\input{ncert/12/13/2/17/defs.tex}
	\item A bag contains 4 red and 4 black balls, another bag contains 2 red and 6 black balls. One of the two bags is selected at random and a ball is drawn from the bag which is found to be red. Find the probability that the ball is drawn from the first bag.
\\
\solution
		%\input{ncert/12/13/3/2/main.tex}
  \item
  Cards with numbers 2 to 101 are placed in a box. A card is selected at random.Find the probability that the card has
\begin{enumerate}[label=(\roman*)]
	\item an even number 
	\item a square number
\end{enumerate}
\solution
%\input{exemplar/10/13/3/32/main.tex}
\item
The king, queen and jack of clubs are removed from a deck of 52 playing cards and then well shuffled. Now one card is drawn at random from the remaining cards.  Determine the probability that the card is
\begin{enumerate}[label=(\roman*)]
\item a club
\item 10 of hearts
\end{enumerate}
\solution
%\input{exemplar/10/13/3/29/main.tex}
\item A team of medical students doing their internship have to assist during surgeries
at a city hospital. The probabilities of surgeries rated as very complex, complex,
routine, simple or very simple are respectively, 0.15, 0.20, 0.31, 0.26, .08. Find
the probabilities that a particular surgery will be rated
\begin{enumerate}
	\item complex or very complex;
	\item neither very complex nor very simple;
	\item routine or complex
	\item routine or simple
\end{enumerate}
\solution
%\input{exemplar/11/16/3/8(1)/main.tex}
\item A card is selected from a pack of 52 cards.
\begin{enumerate}[label=(\alph*)]
    \item How many points are there in the sample space?
    \item Calculate the probability that the card is an ace of spades.
    \item Calculate the probability that the card is (i) an ace and (ii) black card.
\end{enumerate}
\solution
%\input{exemplar/11/16/3/4/main2.tex}
\item The probability that a non leap year selected at random will contain 53 sundays.
\\
\solution
%\input{exemplar/10/13/1/19/main.tex}
\item One of the four persons John, Rita, Aslam or Gurpreet will be promoted next
month. Consequently the sample space consists of four elementary outcomes
S = {John promoted, Rita promoted, Aslam promoted, Gurpreet promoted}
You are told that the chances of John’s promotion is same as that of Gurpreet,
Rita’s chances of promotion are twice as likely as Johns. Aslam’s chances are
four times that of John.
\begin{enumerate}
	\item Determine
	\begin{enumerate}
		\item P (John promoted)
		\item P (Rita promoted)
		\item P (Aslam promoted)
		\item P (Gurpreet promoted)
	\end{enumerate}
	\item If A = {John promoted or Gurpreet promoted}, find P (A).
\end{enumerate}
\solution
%\input{exemplar/11/16/3/10/main.tex}
\item A card is drawn from a deck of 52 cards. Find the probability of getting a king or a heart or a red card.\\
\solution
%\input{exemplar/11/16/3/15/main.tex}
\item The probability that a student will pass his examination is 0.73, the probability of
the student getting a compartment is 0.13, and the probability that the student will
either pass or get compartment is 0.96. State True or False.\\
\solution
%\input{exemplar/11/16/3/31/main.tex}
\item A card is selected from a pack of 52 cards\\
\begin{enumerate}[label=(\alph*)]
\item How many points are there in the sample space?
\item Calculate the probability that the cards is an ace of spades.
\item Calculate the probability that the card is (i) an ace (ii)black card.\\
\end{enumerate}
%\input{ncert/11/16/3/4_1/Prob_4.tex}
\item In a non-leap year, the probability of having 53 tuesdays or 53 wednesdays is\\
\solution
%\input{exemplar/11/16/3/18/main.tex}
\item There are 1000 sealed envelopes in a box, 10 of them contain a cash prize of
Rs 100 each, 100 of them contain a cash prize of Rs 50 each and 200 of them
contain a cash prize of Rs 10 each and rest do not contain any cash prize. If they
are well shuffled and an envelope is picked up out, what is the probability that it
contains no cash prize?\\
\solution
%\input{exemplar/10/13/3/34/main.tex}
\item 
A die is thrown and a card is selected at random from a deck of 52 playing cards. The probability of getting an even number on the die and a spade card.\\
\solution
%\input{exemplar/12/13/3/78/main.tex}
\item
If 4-digit numbers greater than 5,000 are randomly formed from the digits 0, 1, 3, 5, and 7, what is the probability of forming a number divisible by 5 when:
\begin{enumerate}
    \item The digits are repeated?
    \item The repetition of digits is not allowed?
\end{enumerate}
\solution
%\input{ncert/11/16/4/9/main.tex}
\item Consider the probability space $\brak{\Omega, \mathcal{G}, P}$ where $\Omega = [0,2]$ and $\mathcal{G} = \cbrak{\phi, \Omega, [0,1], (1,2]}$. Let $X$ and $Y$ be two functions on $\Omega$ defined as
\begin{align*}
    X(\omega) = 
    \begin{cases}
        1 & \text{if }\omega \in [0, 1]\\
        2 & \text{if }\omega \in (1, 2]
    \end{cases}
\end{align*}
and
\begin{align*}
    Y(\omega) = 
    \begin{cases}
        2 & \text{if }\omega \in [0, 1.5]\\
        3 & \text{if }\omega \in (1.5, 2].
    \end{cases}
\end{align*}
Then which one of the following statements is true?
\begin{enumerate}
    \item [(A)] $X$ is a random variable with respect to $\mathcal{G}$, but $Y$ is not a random variable with respect to $\mathcal{G}$.
    \item [(B)] $Y$ is a random variable with respect to $\mathcal{G}$, but $X$ is not a random variable with respect to $\mathcal{G}$.
    \item [(C)] Neither $X$ nor $Y$ is a random variable with respect to $\mathcal{G}$.
    \item [(D)] Both $X$ and $Y$ are random variables with respect to $\mathcal{G}$.
\end{enumerate} \hfill (GATE ST 2023)\\
\solution
%\input{gate/ST/2023/14/main.tex}
	\item  A die is loaded in such a way that each odd number is twice as likely to occur as
each even number. Find $P(G)$, where $G$ is the event that a number greater than
3 occurs on a single roll of the die.
\\
\solution
		%\input{exemplar/11/16/3/5/main.tex}
	\item All the jacks, queens and kings are removed from a deck of 52 playing cards. The remaining cards are well shuffled and then one card is drawn at random. Giving ace a value 1 similar value for other cards, find the probability that the card has a value 
		\begin{enumerate}
			\item 7
			\item greater than 7
			\item less than 7
		\end{enumerate}
		%\input{exemplar/10/13/3/30/main.tex}
  \item A Lot consists of 48 mobile phones of which 42 are good, 3 have only minor defects and 3 have major defects.Varnika will buy a phone if it is good but the trader will only buy a mobile if it has no major defects. One phone is selected at random from the lot. What is the probability that it is
\begin{enumerate}
	\item acceptable to Varnika?
            \item acceptable to the trader?
\end{enumerate}
\solution
	%\input{exemplar/10/13/3/40/main.tex}
 \item A student says that if you throw a die, it will show up 1 or not 1. Therefore, the probability of getting 1 and the probability of getting 'not 1' each is equal to $\frac{1}{2}$. Is this correct? Give reasons.\\
 \solution
        %\input{exemplar/10/13/2/9/main.tex}
   \item Four candidates A, B, C, D have ap-
plied for the assignment to coach a school cricket
team. If A is twice as likely to be selected as B, and
B and C are given about the same chance of being
selected, while C is twice as likely to be selected
as D, what are the probabilities that
\begin{enumerate}
\item C will be selected?
\item A will not be selected?
\end{enumerate}
	%\input{exemplar/11/16/3/9/main.tex}
 \item A bag contain 24 balls of which $x$ balls are red, $2x$ are white and $3x$ are blue. A ball is selected at random, What is the probability that it is
\begin{enumerate}[label=\alph*)]
\item not red ?
\item white ?
\end{enumerate}
%\input{exemplar/10/13/3/41/main.tex}
If the letters of the word ASSASSINATION are arranged at random. Find the Probability that
\begin{enumerate}[label=(\alph*)]
\item Four $S's$ come consecutively in the word
\item Two  $I's$ and two $N's$ come together
\item All $A's$ are not coming together
\item No two $A's$ are coming together
\end{enumerate}
%\input{exemplar/11/16/3/14/main.tex}
	\item One urn contains two black balls (labelled B1 and B2) and one white ball. A
	second urn contains one black ball and two white balls (labelled W1 and W2).
	Suppose the following experiment is performed. One of the two urns is chosen
	at random. Next a ball is randomly chosen from the urn. Then a second ball is
	chosen at random from the same urn without replacing the first ball.
	
	\begin{enumerate}
	\item What is the probability that two black balls are chosen?
	
	\item What is the probability that two balls of opposite colour are chosen?
	\end{enumerate}
	\solution
	%\input{exemplar/11/16/3/12/main1.tex}
\end{enumerate}

		\item A box of oranges is inspected by examining three randomly selected oranges drawn without replacement. If all the three oranges are good, the box is approved for sale, otherwise, it is rejected. Find the probability that a box containing 15 oranges out of which 12 are good and 3 are bad ones will be approved for sale.
		\label{ncert/12/13/2/3/defs.tex}
		\item Two balls are drawn at random with replacement from a box containing 10 black and 8 red balls. Find the probability that
		\label{ncert/12/13/2/12}
\begin{enumerate}
\item both balls are red.
\item first ball is black and second is red.
\item one of them is black and other is red.
\end{enumerate}

\item In a hostel, 60\% of the students read Hindi newspaper, 40\% read English newspaper and 20\% read both Hindi and English newspapers. A student is selected at random.
		\label{ncert/12/13/2/15}
\begin{enumerate}
\item Find the probability that she reads neither Hindi nor English newspapers.
\item If she reads Hindi newspaper, find the probability that she reads English newspaper.
\item If she reads English newspaper, find the probability that she reads Hindi newspaper.\\
\end{enumerate}
\item The probability of obtaining an even prime number on each die, when a pair of dice is rolled is 
\begin{enumerate}
    \item $0$ 
    
    \item $\frac{1}{3}$ 
    
    \item $\frac{1}{12}$ 
    
    \item $\frac{1}{36}$ 
\end{enumerate}
\solution
		%\begin{enumerate}[label=\thesection.\arabic*,ref=\thesection.\theenumi]
	\item One card is drawn from a well-shuffled deck of 52 cards. Find the probability of getting
\begin{enumerate}
\item A king of red colour 
\item A face card 
\item A red face card
\item The jack of hearts
\item A spade
\item The queen of diamonds

\end{enumerate}
\solution
		%\input{ncert/10/15/1/14/main.tex}
	\item Five cards—the ten, jack, queen, king and ace of diamonds, are well-shuffled with their face downwards. One card is then picked up at random.
\begin{enumerate}
\item
What is the probability that the card is the queen? 
\item
If the queen is drawn and put aside, what is the probability that the second card picked up is (a) an ace? (b) a queen?\\
\end{enumerate}
\solution
		%\input{ncert/10/15/1/15/defs.tex}
	\item A bag contains $5$ red balls and some blue balls. If the probability of drawing a blue ball is double that if a red ball, determine the number of blue balls in the bag. 
		\\
\solution
		%\input{ncert/10/15/2/3/defs.tex}
	\item A card is selected from a pack of 52 cards.
 \begin{enumerate}[label=(\alph*)] 
                 \item How many points are there in the sample space?
                 \item Calculate the probability that the card is an ace of spades.
                 \item Calculate the probability that the card is (i) an ace and (ii) black card.
 \end{enumerate}
\solution
		%\input{ncert/11/16/3/4/main.tex}
\item Four cards are drawn from a well-shuffled deck of 52 cards. What is the probability of obtaining 3 diamonds and one spade.
\\
\solution
		%\input{ncert/11/16/4/2/defs.tex}
\item In a certain lottery 10,000 tickets are sold and ten equal prizes are awarded. What is the probability of not getting a prize if you buy (a) one ticket (b) two tickets (c) 10 tickets ?	
\\
\solution
		%\input{ncert/11/16/4/4/defs.tex}
		%
\item 
Out of 100 students, two sections of 40 and 60 are formed. If you and your friend are among the 100 students, what is the probability that
\begin{enumerate}
\item you both enter the same section?
\item you both enter the different sections?
\end{enumerate}
\solution
		%\input{ncert/11/16/4/5/defs.tex}
	\item 
The number lock of a suitcase has 4 wheels each labelled with ten digits i.e. from 0 to 9.The lock opens with a sequence of four digits with no repeats.What is the probability of a person getting the right sequence to open the suitcase.
\\
\solution
		%\input{ncert/11/16/4/10/defs.tex}
		%
\item 
Two cards are drawn at random and without replacement from a pack of 52 playing cards. Find the probability that both the cards are black.
\\
\solution
		%\input{ncert/12/13/2/2/defs.tex}
		\item A box of oranges is inspected by examining three randomly selected oranges drawn without replacement. If all the three oranges are good, the box is approved for sale, otherwise, it is rejected. Find the probability that a box containing 15 oranges out of which 12 are good and 3 are bad ones will be approved for sale.
		\label{ncert/12/13/2/3/defs.tex}
		\item Two balls are drawn at random with replacement from a box containing 10 black and 8 red balls. Find the probability that
		\label{ncert/12/13/2/12}
\begin{enumerate}
\item both balls are red.
\item first ball is black and second is red.
\item one of them is black and other is red.
\end{enumerate}

\item In a hostel, 60\% of the students read Hindi newspaper, 40\% read English newspaper and 20\% read both Hindi and English newspapers. A student is selected at random.
		\label{ncert/12/13/2/15}
\begin{enumerate}
\item Find the probability that she reads neither Hindi nor English newspapers.
\item If she reads Hindi newspaper, find the probability that she reads English newspaper.
\item If she reads English newspaper, find the probability that she reads Hindi newspaper.\\
\end{enumerate}
\item The probability of obtaining an even prime number on each die, when a pair of dice is rolled is 
\begin{enumerate}
    \item $0$ 
    
    \item $\frac{1}{3}$ 
    
    \item $\frac{1}{12}$ 
    
    \item $\frac{1}{36}$ 
\end{enumerate}
\solution
		%\input{ncert/12/13/2/17/defs.tex}
	\item A bag contains 4 red and 4 black balls, another bag contains 2 red and 6 black balls. One of the two bags is selected at random and a ball is drawn from the bag which is found to be red. Find the probability that the ball is drawn from the first bag.
\\
\solution
		%\input{ncert/12/13/3/2/main.tex}
  \item
  Cards with numbers 2 to 101 are placed in a box. A card is selected at random.Find the probability that the card has
\begin{enumerate}[label=(\roman*)]
	\item an even number 
	\item a square number
\end{enumerate}
\solution
%\input{exemplar/10/13/3/32/main.tex}
\item
The king, queen and jack of clubs are removed from a deck of 52 playing cards and then well shuffled. Now one card is drawn at random from the remaining cards.  Determine the probability that the card is
\begin{enumerate}[label=(\roman*)]
\item a club
\item 10 of hearts
\end{enumerate}
\solution
%\input{exemplar/10/13/3/29/main.tex}
\item A team of medical students doing their internship have to assist during surgeries
at a city hospital. The probabilities of surgeries rated as very complex, complex,
routine, simple or very simple are respectively, 0.15, 0.20, 0.31, 0.26, .08. Find
the probabilities that a particular surgery will be rated
\begin{enumerate}
	\item complex or very complex;
	\item neither very complex nor very simple;
	\item routine or complex
	\item routine or simple
\end{enumerate}
\solution
%\input{exemplar/11/16/3/8(1)/main.tex}
\item A card is selected from a pack of 52 cards.
\begin{enumerate}[label=(\alph*)]
    \item How many points are there in the sample space?
    \item Calculate the probability that the card is an ace of spades.
    \item Calculate the probability that the card is (i) an ace and (ii) black card.
\end{enumerate}
\solution
%\input{exemplar/11/16/3/4/main2.tex}
\item The probability that a non leap year selected at random will contain 53 sundays.
\\
\solution
%\input{exemplar/10/13/1/19/main.tex}
\item One of the four persons John, Rita, Aslam or Gurpreet will be promoted next
month. Consequently the sample space consists of four elementary outcomes
S = {John promoted, Rita promoted, Aslam promoted, Gurpreet promoted}
You are told that the chances of John’s promotion is same as that of Gurpreet,
Rita’s chances of promotion are twice as likely as Johns. Aslam’s chances are
four times that of John.
\begin{enumerate}
	\item Determine
	\begin{enumerate}
		\item P (John promoted)
		\item P (Rita promoted)
		\item P (Aslam promoted)
		\item P (Gurpreet promoted)
	\end{enumerate}
	\item If A = {John promoted or Gurpreet promoted}, find P (A).
\end{enumerate}
\solution
%\input{exemplar/11/16/3/10/main.tex}
\item A card is drawn from a deck of 52 cards. Find the probability of getting a king or a heart or a red card.\\
\solution
%\input{exemplar/11/16/3/15/main.tex}
\item The probability that a student will pass his examination is 0.73, the probability of
the student getting a compartment is 0.13, and the probability that the student will
either pass or get compartment is 0.96. State True or False.\\
\solution
%\input{exemplar/11/16/3/31/main.tex}
\item A card is selected from a pack of 52 cards\\
\begin{enumerate}[label=(\alph*)]
\item How many points are there in the sample space?
\item Calculate the probability that the cards is an ace of spades.
\item Calculate the probability that the card is (i) an ace (ii)black card.\\
\end{enumerate}
%\input{ncert/11/16/3/4_1/Prob_4.tex}
\item In a non-leap year, the probability of having 53 tuesdays or 53 wednesdays is\\
\solution
%\input{exemplar/11/16/3/18/main.tex}
\item There are 1000 sealed envelopes in a box, 10 of them contain a cash prize of
Rs 100 each, 100 of them contain a cash prize of Rs 50 each and 200 of them
contain a cash prize of Rs 10 each and rest do not contain any cash prize. If they
are well shuffled and an envelope is picked up out, what is the probability that it
contains no cash prize?\\
\solution
%\input{exemplar/10/13/3/34/main.tex}
\item 
A die is thrown and a card is selected at random from a deck of 52 playing cards. The probability of getting an even number on the die and a spade card.\\
\solution
%\input{exemplar/12/13/3/78/main.tex}
\item
If 4-digit numbers greater than 5,000 are randomly formed from the digits 0, 1, 3, 5, and 7, what is the probability of forming a number divisible by 5 when:
\begin{enumerate}
    \item The digits are repeated?
    \item The repetition of digits is not allowed?
\end{enumerate}
\solution
%\input{ncert/11/16/4/9/main.tex}
\item Consider the probability space $\brak{\Omega, \mathcal{G}, P}$ where $\Omega = [0,2]$ and $\mathcal{G} = \cbrak{\phi, \Omega, [0,1], (1,2]}$. Let $X$ and $Y$ be two functions on $\Omega$ defined as
\begin{align*}
    X(\omega) = 
    \begin{cases}
        1 & \text{if }\omega \in [0, 1]\\
        2 & \text{if }\omega \in (1, 2]
    \end{cases}
\end{align*}
and
\begin{align*}
    Y(\omega) = 
    \begin{cases}
        2 & \text{if }\omega \in [0, 1.5]\\
        3 & \text{if }\omega \in (1.5, 2].
    \end{cases}
\end{align*}
Then which one of the following statements is true?
\begin{enumerate}
    \item [(A)] $X$ is a random variable with respect to $\mathcal{G}$, but $Y$ is not a random variable with respect to $\mathcal{G}$.
    \item [(B)] $Y$ is a random variable with respect to $\mathcal{G}$, but $X$ is not a random variable with respect to $\mathcal{G}$.
    \item [(C)] Neither $X$ nor $Y$ is a random variable with respect to $\mathcal{G}$.
    \item [(D)] Both $X$ and $Y$ are random variables with respect to $\mathcal{G}$.
\end{enumerate} \hfill (GATE ST 2023)\\
\solution
%\input{gate/ST/2023/14/main.tex}
	\item  A die is loaded in such a way that each odd number is twice as likely to occur as
each even number. Find $P(G)$, where $G$ is the event that a number greater than
3 occurs on a single roll of the die.
\\
\solution
		%\input{exemplar/11/16/3/5/main.tex}
	\item All the jacks, queens and kings are removed from a deck of 52 playing cards. The remaining cards are well shuffled and then one card is drawn at random. Giving ace a value 1 similar value for other cards, find the probability that the card has a value 
		\begin{enumerate}
			\item 7
			\item greater than 7
			\item less than 7
		\end{enumerate}
		%\input{exemplar/10/13/3/30/main.tex}
  \item A Lot consists of 48 mobile phones of which 42 are good, 3 have only minor defects and 3 have major defects.Varnika will buy a phone if it is good but the trader will only buy a mobile if it has no major defects. One phone is selected at random from the lot. What is the probability that it is
\begin{enumerate}
	\item acceptable to Varnika?
            \item acceptable to the trader?
\end{enumerate}
\solution
	%\input{exemplar/10/13/3/40/main.tex}
 \item A student says that if you throw a die, it will show up 1 or not 1. Therefore, the probability of getting 1 and the probability of getting 'not 1' each is equal to $\frac{1}{2}$. Is this correct? Give reasons.\\
 \solution
        %\input{exemplar/10/13/2/9/main.tex}
   \item Four candidates A, B, C, D have ap-
plied for the assignment to coach a school cricket
team. If A is twice as likely to be selected as B, and
B and C are given about the same chance of being
selected, while C is twice as likely to be selected
as D, what are the probabilities that
\begin{enumerate}
\item C will be selected?
\item A will not be selected?
\end{enumerate}
	%\input{exemplar/11/16/3/9/main.tex}
 \item A bag contain 24 balls of which $x$ balls are red, $2x$ are white and $3x$ are blue. A ball is selected at random, What is the probability that it is
\begin{enumerate}[label=\alph*)]
\item not red ?
\item white ?
\end{enumerate}
%\input{exemplar/10/13/3/41/main.tex}
If the letters of the word ASSASSINATION are arranged at random. Find the Probability that
\begin{enumerate}[label=(\alph*)]
\item Four $S's$ come consecutively in the word
\item Two  $I's$ and two $N's$ come together
\item All $A's$ are not coming together
\item No two $A's$ are coming together
\end{enumerate}
%\input{exemplar/11/16/3/14/main.tex}
	\item One urn contains two black balls (labelled B1 and B2) and one white ball. A
	second urn contains one black ball and two white balls (labelled W1 and W2).
	Suppose the following experiment is performed. One of the two urns is chosen
	at random. Next a ball is randomly chosen from the urn. Then a second ball is
	chosen at random from the same urn without replacing the first ball.
	
	\begin{enumerate}
	\item What is the probability that two black balls are chosen?
	
	\item What is the probability that two balls of opposite colour are chosen?
	\end{enumerate}
	\solution
	%\input{exemplar/11/16/3/12/main1.tex}
\end{enumerate}

	\item A bag contains 4 red and 4 black balls, another bag contains 2 red and 6 black balls. One of the two bags is selected at random and a ball is drawn from the bag which is found to be red. Find the probability that the ball is drawn from the first bag.
\\
\solution
		%\begin{table}[H]
	\centering
\begin{tabular}{|c|c|c|}
\hline
Random variable &Value &Definition\\ \hline
\multirow{3}{*}{X} &0 &Slips of Rs 1\\
&1 &Slips of Rs 5\\
&2 &Slips of Rs 13\\ \hline
\multirow{2}{*}{Y} &0 &Box A\\
&1 &Box B\\\hline
\end{tabular}
\caption{}
\label{tab:Distribution}
\end{table}
See \tabref{tab:Distribution}.
\begin{align}
p_{Y}\brak{k}= \begin{cases} 
      \frac{1}{3} & {k=0} \\
      \frac{2}{3 }& {k=1} 
   \end{cases}
   \\
p_{Y|X}\brak{0|0} = \frac{19}{25}\, 
p_{Y|X}\brak{0|1} = \frac{6}{25}\,
p_{Y|X}\brak{1|0} = \frac{45}{50}\,
p_{Y|X}\brak{1|2} = \frac{5}{50}
\end{align}
The desired probability is the probability that a slip drawn at random is marked other than Rs 1,
\begin{align}
&=1-p_X\brak{0}\\
&= p_X(1) + p_X(2)
\end{align}
Using Bayes theorem,
\begin{align}
&= p_Y\brak{0} \times \pr{Y=0 | X=1} + p_Y\brak{1} \times \pr{Y=1|X=2}\\
&=\frac{1}{3} \times \frac{6}{25} + \frac{2}{3} \times \frac{5}{50}\\
&=\frac{11}{75}
\end{align}

\newpage

%\tableofcontents

\bigskip

\renewcommand{\thefigure}{\theenumi}
\renewcommand{\thetable}{\theenumi}
%\renewcommand{\theequation}{\theenumi}

%\begin{abstract}
%%\boldmath
%In this letter, an algorithm for evaluating the exact analytical bit error rate  (BER)  for the piecewise linear (PL) combiner for  multiple relays is presented. Previous results were available only for upto three relays. The algorithm is unique in the sense that  the actual mathematical expressions, that are prohibitively large, need not be explicitly obtained. The diversity gain due to multiple relays is shown through plots of the analytical BER, well supported by simulations. 
%
%\end{abstract}
% IEEEtran.cls defaults to using nonbold math in the Abstract.
% This preserves the distinction between vectors and scalars. However,
% if the journal you are submitting to favors bold math in the abstract,
% then you can use LaTeX's standard command \boldmath at the very start
% of the abstract to achieve this. Many IEEE journals frown on math
% in the abstract anyway.

% Note that keywords are not normally used for peerreview papers.
%\begin{IEEEkeywords}
%Cooperative diversity, decode and forward, piecewise linear
%\end{IEEEkeywords}



% For peer review papers, you can put extra information on the cover
% page as needed:
% \ifCLASSOPTIONpeerreview
% \begin{center} \bfseries EDICS Category: 3-BBND \end{center}
% \fi
%
% For peerreview papers, this IEEEtran command inserts a page break and
% creates the second title. It will be ignored for other modes.
%\IEEEpeerreviewmaketitle




  \item
  Cards with numbers 2 to 101 are placed in a box. A card is selected at random.Find the probability that the card has
\begin{enumerate}[label=(\roman*)]
	\item an even number 
	\item a square number
\end{enumerate}
\solution
%\begin{table}[H]
	\centering
\begin{tabular}{|c|c|c|}
\hline
Random variable &Value &Definition\\ \hline
\multirow{3}{*}{X} &0 &Slips of Rs 1\\
&1 &Slips of Rs 5\\
&2 &Slips of Rs 13\\ \hline
\multirow{2}{*}{Y} &0 &Box A\\
&1 &Box B\\\hline
\end{tabular}
\caption{}
\label{tab:Distribution}
\end{table}
See \tabref{tab:Distribution}.
\begin{align}
p_{Y}\brak{k}= \begin{cases} 
      \frac{1}{3} & {k=0} \\
      \frac{2}{3 }& {k=1} 
   \end{cases}
   \\
p_{Y|X}\brak{0|0} = \frac{19}{25}\, 
p_{Y|X}\brak{0|1} = \frac{6}{25}\,
p_{Y|X}\brak{1|0} = \frac{45}{50}\,
p_{Y|X}\brak{1|2} = \frac{5}{50}
\end{align}
The desired probability is the probability that a slip drawn at random is marked other than Rs 1,
\begin{align}
&=1-p_X\brak{0}\\
&= p_X(1) + p_X(2)
\end{align}
Using Bayes theorem,
\begin{align}
&= p_Y\brak{0} \times \pr{Y=0 | X=1} + p_Y\brak{1} \times \pr{Y=1|X=2}\\
&=\frac{1}{3} \times \frac{6}{25} + \frac{2}{3} \times \frac{5}{50}\\
&=\frac{11}{75}
\end{align}

\newpage

%\tableofcontents

\bigskip

\renewcommand{\thefigure}{\theenumi}
\renewcommand{\thetable}{\theenumi}
%\renewcommand{\theequation}{\theenumi}

%\begin{abstract}
%%\boldmath
%In this letter, an algorithm for evaluating the exact analytical bit error rate  (BER)  for the piecewise linear (PL) combiner for  multiple relays is presented. Previous results were available only for upto three relays. The algorithm is unique in the sense that  the actual mathematical expressions, that are prohibitively large, need not be explicitly obtained. The diversity gain due to multiple relays is shown through plots of the analytical BER, well supported by simulations. 
%
%\end{abstract}
% IEEEtran.cls defaults to using nonbold math in the Abstract.
% This preserves the distinction between vectors and scalars. However,
% if the journal you are submitting to favors bold math in the abstract,
% then you can use LaTeX's standard command \boldmath at the very start
% of the abstract to achieve this. Many IEEE journals frown on math
% in the abstract anyway.

% Note that keywords are not normally used for peerreview papers.
%\begin{IEEEkeywords}
%Cooperative diversity, decode and forward, piecewise linear
%\end{IEEEkeywords}



% For peer review papers, you can put extra information on the cover
% page as needed:
% \ifCLASSOPTIONpeerreview
% \begin{center} \bfseries EDICS Category: 3-BBND \end{center}
% \fi
%
% For peerreview papers, this IEEEtran command inserts a page break and
% creates the second title. It will be ignored for other modes.
%\IEEEpeerreviewmaketitle




\item
The king, queen and jack of clubs are removed from a deck of 52 playing cards and then well shuffled. Now one card is drawn at random from the remaining cards.  Determine the probability that the card is
\begin{enumerate}[label=(\roman*)]
\item a club
\item 10 of hearts
\end{enumerate}
\solution
%\begin{table}[H]
	\centering
\begin{tabular}{|c|c|c|}
\hline
Random variable &Value &Definition\\ \hline
\multirow{3}{*}{X} &0 &Slips of Rs 1\\
&1 &Slips of Rs 5\\
&2 &Slips of Rs 13\\ \hline
\multirow{2}{*}{Y} &0 &Box A\\
&1 &Box B\\\hline
\end{tabular}
\caption{}
\label{tab:Distribution}
\end{table}
See \tabref{tab:Distribution}.
\begin{align}
p_{Y}\brak{k}= \begin{cases} 
      \frac{1}{3} & {k=0} \\
      \frac{2}{3 }& {k=1} 
   \end{cases}
   \\
p_{Y|X}\brak{0|0} = \frac{19}{25}\, 
p_{Y|X}\brak{0|1} = \frac{6}{25}\,
p_{Y|X}\brak{1|0} = \frac{45}{50}\,
p_{Y|X}\brak{1|2} = \frac{5}{50}
\end{align}
The desired probability is the probability that a slip drawn at random is marked other than Rs 1,
\begin{align}
&=1-p_X\brak{0}\\
&= p_X(1) + p_X(2)
\end{align}
Using Bayes theorem,
\begin{align}
&= p_Y\brak{0} \times \pr{Y=0 | X=1} + p_Y\brak{1} \times \pr{Y=1|X=2}\\
&=\frac{1}{3} \times \frac{6}{25} + \frac{2}{3} \times \frac{5}{50}\\
&=\frac{11}{75}
\end{align}

\newpage

%\tableofcontents

\bigskip

\renewcommand{\thefigure}{\theenumi}
\renewcommand{\thetable}{\theenumi}
%\renewcommand{\theequation}{\theenumi}

%\begin{abstract}
%%\boldmath
%In this letter, an algorithm for evaluating the exact analytical bit error rate  (BER)  for the piecewise linear (PL) combiner for  multiple relays is presented. Previous results were available only for upto three relays. The algorithm is unique in the sense that  the actual mathematical expressions, that are prohibitively large, need not be explicitly obtained. The diversity gain due to multiple relays is shown through plots of the analytical BER, well supported by simulations. 
%
%\end{abstract}
% IEEEtran.cls defaults to using nonbold math in the Abstract.
% This preserves the distinction between vectors and scalars. However,
% if the journal you are submitting to favors bold math in the abstract,
% then you can use LaTeX's standard command \boldmath at the very start
% of the abstract to achieve this. Many IEEE journals frown on math
% in the abstract anyway.

% Note that keywords are not normally used for peerreview papers.
%\begin{IEEEkeywords}
%Cooperative diversity, decode and forward, piecewise linear
%\end{IEEEkeywords}



% For peer review papers, you can put extra information on the cover
% page as needed:
% \ifCLASSOPTIONpeerreview
% \begin{center} \bfseries EDICS Category: 3-BBND \end{center}
% \fi
%
% For peerreview papers, this IEEEtran command inserts a page break and
% creates the second title. It will be ignored for other modes.
%\IEEEpeerreviewmaketitle




\item A team of medical students doing their internship have to assist during surgeries
at a city hospital. The probabilities of surgeries rated as very complex, complex,
routine, simple or very simple are respectively, 0.15, 0.20, 0.31, 0.26, .08. Find
the probabilities that a particular surgery will be rated
\begin{enumerate}
	\item complex or very complex;
	\item neither very complex nor very simple;
	\item routine or complex
	\item routine or simple
\end{enumerate}
\solution
%\begin{table}[H]
	\centering
\begin{tabular}{|c|c|c|}
\hline
Random variable &Value &Definition\\ \hline
\multirow{3}{*}{X} &0 &Slips of Rs 1\\
&1 &Slips of Rs 5\\
&2 &Slips of Rs 13\\ \hline
\multirow{2}{*}{Y} &0 &Box A\\
&1 &Box B\\\hline
\end{tabular}
\caption{}
\label{tab:Distribution}
\end{table}
See \tabref{tab:Distribution}.
\begin{align}
p_{Y}\brak{k}= \begin{cases} 
      \frac{1}{3} & {k=0} \\
      \frac{2}{3 }& {k=1} 
   \end{cases}
   \\
p_{Y|X}\brak{0|0} = \frac{19}{25}\, 
p_{Y|X}\brak{0|1} = \frac{6}{25}\,
p_{Y|X}\brak{1|0} = \frac{45}{50}\,
p_{Y|X}\brak{1|2} = \frac{5}{50}
\end{align}
The desired probability is the probability that a slip drawn at random is marked other than Rs 1,
\begin{align}
&=1-p_X\brak{0}\\
&= p_X(1) + p_X(2)
\end{align}
Using Bayes theorem,
\begin{align}
&= p_Y\brak{0} \times \pr{Y=0 | X=1} + p_Y\brak{1} \times \pr{Y=1|X=2}\\
&=\frac{1}{3} \times \frac{6}{25} + \frac{2}{3} \times \frac{5}{50}\\
&=\frac{11}{75}
\end{align}

\newpage

%\tableofcontents

\bigskip

\renewcommand{\thefigure}{\theenumi}
\renewcommand{\thetable}{\theenumi}
%\renewcommand{\theequation}{\theenumi}

%\begin{abstract}
%%\boldmath
%In this letter, an algorithm for evaluating the exact analytical bit error rate  (BER)  for the piecewise linear (PL) combiner for  multiple relays is presented. Previous results were available only for upto three relays. The algorithm is unique in the sense that  the actual mathematical expressions, that are prohibitively large, need not be explicitly obtained. The diversity gain due to multiple relays is shown through plots of the analytical BER, well supported by simulations. 
%
%\end{abstract}
% IEEEtran.cls defaults to using nonbold math in the Abstract.
% This preserves the distinction between vectors and scalars. However,
% if the journal you are submitting to favors bold math in the abstract,
% then you can use LaTeX's standard command \boldmath at the very start
% of the abstract to achieve this. Many IEEE journals frown on math
% in the abstract anyway.

% Note that keywords are not normally used for peerreview papers.
%\begin{IEEEkeywords}
%Cooperative diversity, decode and forward, piecewise linear
%\end{IEEEkeywords}



% For peer review papers, you can put extra information on the cover
% page as needed:
% \ifCLASSOPTIONpeerreview
% \begin{center} \bfseries EDICS Category: 3-BBND \end{center}
% \fi
%
% For peerreview papers, this IEEEtran command inserts a page break and
% creates the second title. It will be ignored for other modes.
%\IEEEpeerreviewmaketitle




\item A card is selected from a pack of 52 cards.
\begin{enumerate}[label=(\alph*)]
    \item How many points are there in the sample space?
    \item Calculate the probability that the card is an ace of spades.
    \item Calculate the probability that the card is (i) an ace and (ii) black card.
\end{enumerate}
\solution
%Let $X$ be an bernoulli rv defined as in \tabref{tab:exemplar/11/16/3/26}.  Then, 
\begin{equation}
    p =
        \frac{4}{11} 
\end{equation}
\begin{table}[H]
	\centering
	\input{exemplar/11/16/3/26/tables/Table2.tex}
	\caption{}
        \label{tab:exemplar/11/16/3/26}
\end{table}

\item The probability that a non leap year selected at random will contain 53 sundays.
\\
\solution
%\begin{table}[H]
	\centering
\begin{tabular}{|c|c|c|}
\hline
Random variable &Value &Definition\\ \hline
\multirow{3}{*}{X} &0 &Slips of Rs 1\\
&1 &Slips of Rs 5\\
&2 &Slips of Rs 13\\ \hline
\multirow{2}{*}{Y} &0 &Box A\\
&1 &Box B\\\hline
\end{tabular}
\caption{}
\label{tab:Distribution}
\end{table}
See \tabref{tab:Distribution}.
\begin{align}
p_{Y}\brak{k}= \begin{cases} 
      \frac{1}{3} & {k=0} \\
      \frac{2}{3 }& {k=1} 
   \end{cases}
   \\
p_{Y|X}\brak{0|0} = \frac{19}{25}\, 
p_{Y|X}\brak{0|1} = \frac{6}{25}\,
p_{Y|X}\brak{1|0} = \frac{45}{50}\,
p_{Y|X}\brak{1|2} = \frac{5}{50}
\end{align}
The desired probability is the probability that a slip drawn at random is marked other than Rs 1,
\begin{align}
&=1-p_X\brak{0}\\
&= p_X(1) + p_X(2)
\end{align}
Using Bayes theorem,
\begin{align}
&= p_Y\brak{0} \times \pr{Y=0 | X=1} + p_Y\brak{1} \times \pr{Y=1|X=2}\\
&=\frac{1}{3} \times \frac{6}{25} + \frac{2}{3} \times \frac{5}{50}\\
&=\frac{11}{75}
\end{align}

\newpage

%\tableofcontents

\bigskip

\renewcommand{\thefigure}{\theenumi}
\renewcommand{\thetable}{\theenumi}
%\renewcommand{\theequation}{\theenumi}

%\begin{abstract}
%%\boldmath
%In this letter, an algorithm for evaluating the exact analytical bit error rate  (BER)  for the piecewise linear (PL) combiner for  multiple relays is presented. Previous results were available only for upto three relays. The algorithm is unique in the sense that  the actual mathematical expressions, that are prohibitively large, need not be explicitly obtained. The diversity gain due to multiple relays is shown through plots of the analytical BER, well supported by simulations. 
%
%\end{abstract}
% IEEEtran.cls defaults to using nonbold math in the Abstract.
% This preserves the distinction between vectors and scalars. However,
% if the journal you are submitting to favors bold math in the abstract,
% then you can use LaTeX's standard command \boldmath at the very start
% of the abstract to achieve this. Many IEEE journals frown on math
% in the abstract anyway.

% Note that keywords are not normally used for peerreview papers.
%\begin{IEEEkeywords}
%Cooperative diversity, decode and forward, piecewise linear
%\end{IEEEkeywords}



% For peer review papers, you can put extra information on the cover
% page as needed:
% \ifCLASSOPTIONpeerreview
% \begin{center} \bfseries EDICS Category: 3-BBND \end{center}
% \fi
%
% For peerreview papers, this IEEEtran command inserts a page break and
% creates the second title. It will be ignored for other modes.
%\IEEEpeerreviewmaketitle




\item One of the four persons John, Rita, Aslam or Gurpreet will be promoted next
month. Consequently the sample space consists of four elementary outcomes
S = {John promoted, Rita promoted, Aslam promoted, Gurpreet promoted}
You are told that the chances of John’s promotion is same as that of Gurpreet,
Rita’s chances of promotion are twice as likely as Johns. Aslam’s chances are
four times that of John.
\begin{enumerate}
	\item Determine
	\begin{enumerate}
		\item P (John promoted)
		\item P (Rita promoted)
		\item P (Aslam promoted)
		\item P (Gurpreet promoted)
	\end{enumerate}
	\item If A = {John promoted or Gurpreet promoted}, find P (A).
\end{enumerate}
\solution
%\begin{table}[H]
	\centering
\begin{tabular}{|c|c|c|}
\hline
Random variable &Value &Definition\\ \hline
\multirow{3}{*}{X} &0 &Slips of Rs 1\\
&1 &Slips of Rs 5\\
&2 &Slips of Rs 13\\ \hline
\multirow{2}{*}{Y} &0 &Box A\\
&1 &Box B\\\hline
\end{tabular}
\caption{}
\label{tab:Distribution}
\end{table}
See \tabref{tab:Distribution}.
\begin{align}
p_{Y}\brak{k}= \begin{cases} 
      \frac{1}{3} & {k=0} \\
      \frac{2}{3 }& {k=1} 
   \end{cases}
   \\
p_{Y|X}\brak{0|0} = \frac{19}{25}\, 
p_{Y|X}\brak{0|1} = \frac{6}{25}\,
p_{Y|X}\brak{1|0} = \frac{45}{50}\,
p_{Y|X}\brak{1|2} = \frac{5}{50}
\end{align}
The desired probability is the probability that a slip drawn at random is marked other than Rs 1,
\begin{align}
&=1-p_X\brak{0}\\
&= p_X(1) + p_X(2)
\end{align}
Using Bayes theorem,
\begin{align}
&= p_Y\brak{0} \times \pr{Y=0 | X=1} + p_Y\brak{1} \times \pr{Y=1|X=2}\\
&=\frac{1}{3} \times \frac{6}{25} + \frac{2}{3} \times \frac{5}{50}\\
&=\frac{11}{75}
\end{align}

\newpage

%\tableofcontents

\bigskip

\renewcommand{\thefigure}{\theenumi}
\renewcommand{\thetable}{\theenumi}
%\renewcommand{\theequation}{\theenumi}

%\begin{abstract}
%%\boldmath
%In this letter, an algorithm for evaluating the exact analytical bit error rate  (BER)  for the piecewise linear (PL) combiner for  multiple relays is presented. Previous results were available only for upto three relays. The algorithm is unique in the sense that  the actual mathematical expressions, that are prohibitively large, need not be explicitly obtained. The diversity gain due to multiple relays is shown through plots of the analytical BER, well supported by simulations. 
%
%\end{abstract}
% IEEEtran.cls defaults to using nonbold math in the Abstract.
% This preserves the distinction between vectors and scalars. However,
% if the journal you are submitting to favors bold math in the abstract,
% then you can use LaTeX's standard command \boldmath at the very start
% of the abstract to achieve this. Many IEEE journals frown on math
% in the abstract anyway.

% Note that keywords are not normally used for peerreview papers.
%\begin{IEEEkeywords}
%Cooperative diversity, decode and forward, piecewise linear
%\end{IEEEkeywords}



% For peer review papers, you can put extra information on the cover
% page as needed:
% \ifCLASSOPTIONpeerreview
% \begin{center} \bfseries EDICS Category: 3-BBND \end{center}
% \fi
%
% For peerreview papers, this IEEEtran command inserts a page break and
% creates the second title. It will be ignored for other modes.
%\IEEEpeerreviewmaketitle




\item A card is drawn from a deck of 52 cards. Find the probability of getting a king or a heart or a red card.\\
\solution
%\begin{table}[H]
	\centering
\begin{tabular}{|c|c|c|}
\hline
Random variable &Value &Definition\\ \hline
\multirow{3}{*}{X} &0 &Slips of Rs 1\\
&1 &Slips of Rs 5\\
&2 &Slips of Rs 13\\ \hline
\multirow{2}{*}{Y} &0 &Box A\\
&1 &Box B\\\hline
\end{tabular}
\caption{}
\label{tab:Distribution}
\end{table}
See \tabref{tab:Distribution}.
\begin{align}
p_{Y}\brak{k}= \begin{cases} 
      \frac{1}{3} & {k=0} \\
      \frac{2}{3 }& {k=1} 
   \end{cases}
   \\
p_{Y|X}\brak{0|0} = \frac{19}{25}\, 
p_{Y|X}\brak{0|1} = \frac{6}{25}\,
p_{Y|X}\brak{1|0} = \frac{45}{50}\,
p_{Y|X}\brak{1|2} = \frac{5}{50}
\end{align}
The desired probability is the probability that a slip drawn at random is marked other than Rs 1,
\begin{align}
&=1-p_X\brak{0}\\
&= p_X(1) + p_X(2)
\end{align}
Using Bayes theorem,
\begin{align}
&= p_Y\brak{0} \times \pr{Y=0 | X=1} + p_Y\brak{1} \times \pr{Y=1|X=2}\\
&=\frac{1}{3} \times \frac{6}{25} + \frac{2}{3} \times \frac{5}{50}\\
&=\frac{11}{75}
\end{align}

\newpage

%\tableofcontents

\bigskip

\renewcommand{\thefigure}{\theenumi}
\renewcommand{\thetable}{\theenumi}
%\renewcommand{\theequation}{\theenumi}

%\begin{abstract}
%%\boldmath
%In this letter, an algorithm for evaluating the exact analytical bit error rate  (BER)  for the piecewise linear (PL) combiner for  multiple relays is presented. Previous results were available only for upto three relays. The algorithm is unique in the sense that  the actual mathematical expressions, that are prohibitively large, need not be explicitly obtained. The diversity gain due to multiple relays is shown through plots of the analytical BER, well supported by simulations. 
%
%\end{abstract}
% IEEEtran.cls defaults to using nonbold math in the Abstract.
% This preserves the distinction between vectors and scalars. However,
% if the journal you are submitting to favors bold math in the abstract,
% then you can use LaTeX's standard command \boldmath at the very start
% of the abstract to achieve this. Many IEEE journals frown on math
% in the abstract anyway.

% Note that keywords are not normally used for peerreview papers.
%\begin{IEEEkeywords}
%Cooperative diversity, decode and forward, piecewise linear
%\end{IEEEkeywords}



% For peer review papers, you can put extra information on the cover
% page as needed:
% \ifCLASSOPTIONpeerreview
% \begin{center} \bfseries EDICS Category: 3-BBND \end{center}
% \fi
%
% For peerreview papers, this IEEEtran command inserts a page break and
% creates the second title. It will be ignored for other modes.
%\IEEEpeerreviewmaketitle




\item The probability that a student will pass his examination is 0.73, the probability of
the student getting a compartment is 0.13, and the probability that the student will
either pass or get compartment is 0.96. State True or False.\\
\solution
%\begin{table}[H]
	\centering
\begin{tabular}{|c|c|c|}
\hline
Random variable &Value &Definition\\ \hline
\multirow{3}{*}{X} &0 &Slips of Rs 1\\
&1 &Slips of Rs 5\\
&2 &Slips of Rs 13\\ \hline
\multirow{2}{*}{Y} &0 &Box A\\
&1 &Box B\\\hline
\end{tabular}
\caption{}
\label{tab:Distribution}
\end{table}
See \tabref{tab:Distribution}.
\begin{align}
p_{Y}\brak{k}= \begin{cases} 
      \frac{1}{3} & {k=0} \\
      \frac{2}{3 }& {k=1} 
   \end{cases}
   \\
p_{Y|X}\brak{0|0} = \frac{19}{25}\, 
p_{Y|X}\brak{0|1} = \frac{6}{25}\,
p_{Y|X}\brak{1|0} = \frac{45}{50}\,
p_{Y|X}\brak{1|2} = \frac{5}{50}
\end{align}
The desired probability is the probability that a slip drawn at random is marked other than Rs 1,
\begin{align}
&=1-p_X\brak{0}\\
&= p_X(1) + p_X(2)
\end{align}
Using Bayes theorem,
\begin{align}
&= p_Y\brak{0} \times \pr{Y=0 | X=1} + p_Y\brak{1} \times \pr{Y=1|X=2}\\
&=\frac{1}{3} \times \frac{6}{25} + \frac{2}{3} \times \frac{5}{50}\\
&=\frac{11}{75}
\end{align}

\newpage

%\tableofcontents

\bigskip

\renewcommand{\thefigure}{\theenumi}
\renewcommand{\thetable}{\theenumi}
%\renewcommand{\theequation}{\theenumi}

%\begin{abstract}
%%\boldmath
%In this letter, an algorithm for evaluating the exact analytical bit error rate  (BER)  for the piecewise linear (PL) combiner for  multiple relays is presented. Previous results were available only for upto three relays. The algorithm is unique in the sense that  the actual mathematical expressions, that are prohibitively large, need not be explicitly obtained. The diversity gain due to multiple relays is shown through plots of the analytical BER, well supported by simulations. 
%
%\end{abstract}
% IEEEtran.cls defaults to using nonbold math in the Abstract.
% This preserves the distinction between vectors and scalars. However,
% if the journal you are submitting to favors bold math in the abstract,
% then you can use LaTeX's standard command \boldmath at the very start
% of the abstract to achieve this. Many IEEE journals frown on math
% in the abstract anyway.

% Note that keywords are not normally used for peerreview papers.
%\begin{IEEEkeywords}
%Cooperative diversity, decode and forward, piecewise linear
%\end{IEEEkeywords}



% For peer review papers, you can put extra information on the cover
% page as needed:
% \ifCLASSOPTIONpeerreview
% \begin{center} \bfseries EDICS Category: 3-BBND \end{center}
% \fi
%
% For peerreview papers, this IEEEtran command inserts a page break and
% creates the second title. It will be ignored for other modes.
%\IEEEpeerreviewmaketitle




\item A card is selected from a pack of 52 cards\\
\begin{enumerate}[label=(\alph*)]
\item How many points are there in the sample space?
\item Calculate the probability that the cards is an ace of spades.
\item Calculate the probability that the card is (i) an ace (ii)black card.\\
\end{enumerate}
%\input{ncert/11/16/3/4_1/Prob_4.tex}
\item In a non-leap year, the probability of having 53 tuesdays or 53 wednesdays is\\
\solution
%A non-leap year has a total of 365 days, and a week has 7 days.\\
So it can be expressed as 
\begin{align}
365\text{days} &=52\times 7+1 \text{day}
\end{align}
$\implies$ 52 tuesdays or wednesdays\\
Random variable X denotes the days of a week
\begin{align}
p_X\brak{k}&=\frac{1}{7}; \quad \brak{1<k<7}
\end{align}
So the probability of extra day being tuesday or wednesday is
\begin{align}
p_X\brak{3}+p_X\brak{4}&=\frac{1}{7}+\frac{1}{7}=\frac{2}{7}
\end{align}



\item There are 1000 sealed envelopes in a box, 10 of them contain a cash prize of
Rs 100 each, 100 of them contain a cash prize of Rs 50 each and 200 of them
contain a cash prize of Rs 10 each and rest do not contain any cash prize. If they
are well shuffled and an envelope is picked up out, what is the probability that it
contains no cash prize?\\
\solution
%\begin{table}[H]
	\centering
\begin{tabular}{|c|c|c|}
\hline
Random variable &Value &Definition\\ \hline
\multirow{3}{*}{X} &0 &Slips of Rs 1\\
&1 &Slips of Rs 5\\
&2 &Slips of Rs 13\\ \hline
\multirow{2}{*}{Y} &0 &Box A\\
&1 &Box B\\\hline
\end{tabular}
\caption{}
\label{tab:Distribution}
\end{table}
See \tabref{tab:Distribution}.
\begin{align}
p_{Y}\brak{k}= \begin{cases} 
      \frac{1}{3} & {k=0} \\
      \frac{2}{3 }& {k=1} 
   \end{cases}
   \\
p_{Y|X}\brak{0|0} = \frac{19}{25}\, 
p_{Y|X}\brak{0|1} = \frac{6}{25}\,
p_{Y|X}\brak{1|0} = \frac{45}{50}\,
p_{Y|X}\brak{1|2} = \frac{5}{50}
\end{align}
The desired probability is the probability that a slip drawn at random is marked other than Rs 1,
\begin{align}
&=1-p_X\brak{0}\\
&= p_X(1) + p_X(2)
\end{align}
Using Bayes theorem,
\begin{align}
&= p_Y\brak{0} \times \pr{Y=0 | X=1} + p_Y\brak{1} \times \pr{Y=1|X=2}\\
&=\frac{1}{3} \times \frac{6}{25} + \frac{2}{3} \times \frac{5}{50}\\
&=\frac{11}{75}
\end{align}

\newpage

%\tableofcontents

\bigskip

\renewcommand{\thefigure}{\theenumi}
\renewcommand{\thetable}{\theenumi}
%\renewcommand{\theequation}{\theenumi}

%\begin{abstract}
%%\boldmath
%In this letter, an algorithm for evaluating the exact analytical bit error rate  (BER)  for the piecewise linear (PL) combiner for  multiple relays is presented. Previous results were available only for upto three relays. The algorithm is unique in the sense that  the actual mathematical expressions, that are prohibitively large, need not be explicitly obtained. The diversity gain due to multiple relays is shown through plots of the analytical BER, well supported by simulations. 
%
%\end{abstract}
% IEEEtran.cls defaults to using nonbold math in the Abstract.
% This preserves the distinction between vectors and scalars. However,
% if the journal you are submitting to favors bold math in the abstract,
% then you can use LaTeX's standard command \boldmath at the very start
% of the abstract to achieve this. Many IEEE journals frown on math
% in the abstract anyway.

% Note that keywords are not normally used for peerreview papers.
%\begin{IEEEkeywords}
%Cooperative diversity, decode and forward, piecewise linear
%\end{IEEEkeywords}



% For peer review papers, you can put extra information on the cover
% page as needed:
% \ifCLASSOPTIONpeerreview
% \begin{center} \bfseries EDICS Category: 3-BBND \end{center}
% \fi
%
% For peerreview papers, this IEEEtran command inserts a page break and
% creates the second title. It will be ignored for other modes.
%\IEEEpeerreviewmaketitle




\item 
A die is thrown and a card is selected at random from a deck of 52 playing cards. The probability of getting an even number on the die and a spade card.\\
\solution
%\begin{table}[H]
	\centering
\begin{tabular}{|c|c|c|}
\hline
Random variable &Value &Definition\\ \hline
\multirow{3}{*}{X} &0 &Slips of Rs 1\\
&1 &Slips of Rs 5\\
&2 &Slips of Rs 13\\ \hline
\multirow{2}{*}{Y} &0 &Box A\\
&1 &Box B\\\hline
\end{tabular}
\caption{}
\label{tab:Distribution}
\end{table}
See \tabref{tab:Distribution}.
\begin{align}
p_{Y}\brak{k}= \begin{cases} 
      \frac{1}{3} & {k=0} \\
      \frac{2}{3 }& {k=1} 
   \end{cases}
   \\
p_{Y|X}\brak{0|0} = \frac{19}{25}\, 
p_{Y|X}\brak{0|1} = \frac{6}{25}\,
p_{Y|X}\brak{1|0} = \frac{45}{50}\,
p_{Y|X}\brak{1|2} = \frac{5}{50}
\end{align}
The desired probability is the probability that a slip drawn at random is marked other than Rs 1,
\begin{align}
&=1-p_X\brak{0}\\
&= p_X(1) + p_X(2)
\end{align}
Using Bayes theorem,
\begin{align}
&= p_Y\brak{0} \times \pr{Y=0 | X=1} + p_Y\brak{1} \times \pr{Y=1|X=2}\\
&=\frac{1}{3} \times \frac{6}{25} + \frac{2}{3} \times \frac{5}{50}\\
&=\frac{11}{75}
\end{align}

\newpage

%\tableofcontents

\bigskip

\renewcommand{\thefigure}{\theenumi}
\renewcommand{\thetable}{\theenumi}
%\renewcommand{\theequation}{\theenumi}

%\begin{abstract}
%%\boldmath
%In this letter, an algorithm for evaluating the exact analytical bit error rate  (BER)  for the piecewise linear (PL) combiner for  multiple relays is presented. Previous results were available only for upto three relays. The algorithm is unique in the sense that  the actual mathematical expressions, that are prohibitively large, need not be explicitly obtained. The diversity gain due to multiple relays is shown through plots of the analytical BER, well supported by simulations. 
%
%\end{abstract}
% IEEEtran.cls defaults to using nonbold math in the Abstract.
% This preserves the distinction between vectors and scalars. However,
% if the journal you are submitting to favors bold math in the abstract,
% then you can use LaTeX's standard command \boldmath at the very start
% of the abstract to achieve this. Many IEEE journals frown on math
% in the abstract anyway.

% Note that keywords are not normally used for peerreview papers.
%\begin{IEEEkeywords}
%Cooperative diversity, decode and forward, piecewise linear
%\end{IEEEkeywords}



% For peer review papers, you can put extra information on the cover
% page as needed:
% \ifCLASSOPTIONpeerreview
% \begin{center} \bfseries EDICS Category: 3-BBND \end{center}
% \fi
%
% For peerreview papers, this IEEEtran command inserts a page break and
% creates the second title. It will be ignored for other modes.
%\IEEEpeerreviewmaketitle




\item
If 4-digit numbers greater than 5,000 are randomly formed from the digits 0, 1, 3, 5, and 7, what is the probability of forming a number divisible by 5 when:
\begin{enumerate}
    \item The digits are repeated?
    \item The repetition of digits is not allowed?
\end{enumerate}
\solution
%\begin{table}[H]
	\centering
\begin{tabular}{|c|c|c|}
\hline
Random variable &Value &Definition\\ \hline
\multirow{3}{*}{X} &0 &Slips of Rs 1\\
&1 &Slips of Rs 5\\
&2 &Slips of Rs 13\\ \hline
\multirow{2}{*}{Y} &0 &Box A\\
&1 &Box B\\\hline
\end{tabular}
\caption{}
\label{tab:Distribution}
\end{table}
See \tabref{tab:Distribution}.
\begin{align}
p_{Y}\brak{k}= \begin{cases} 
      \frac{1}{3} & {k=0} \\
      \frac{2}{3 }& {k=1} 
   \end{cases}
   \\
p_{Y|X}\brak{0|0} = \frac{19}{25}\, 
p_{Y|X}\brak{0|1} = \frac{6}{25}\,
p_{Y|X}\brak{1|0} = \frac{45}{50}\,
p_{Y|X}\brak{1|2} = \frac{5}{50}
\end{align}
The desired probability is the probability that a slip drawn at random is marked other than Rs 1,
\begin{align}
&=1-p_X\brak{0}\\
&= p_X(1) + p_X(2)
\end{align}
Using Bayes theorem,
\begin{align}
&= p_Y\brak{0} \times \pr{Y=0 | X=1} + p_Y\brak{1} \times \pr{Y=1|X=2}\\
&=\frac{1}{3} \times \frac{6}{25} + \frac{2}{3} \times \frac{5}{50}\\
&=\frac{11}{75}
\end{align}

\newpage

%\tableofcontents

\bigskip

\renewcommand{\thefigure}{\theenumi}
\renewcommand{\thetable}{\theenumi}
%\renewcommand{\theequation}{\theenumi}

%\begin{abstract}
%%\boldmath
%In this letter, an algorithm for evaluating the exact analytical bit error rate  (BER)  for the piecewise linear (PL) combiner for  multiple relays is presented. Previous results were available only for upto three relays. The algorithm is unique in the sense that  the actual mathematical expressions, that are prohibitively large, need not be explicitly obtained. The diversity gain due to multiple relays is shown through plots of the analytical BER, well supported by simulations. 
%
%\end{abstract}
% IEEEtran.cls defaults to using nonbold math in the Abstract.
% This preserves the distinction between vectors and scalars. However,
% if the journal you are submitting to favors bold math in the abstract,
% then you can use LaTeX's standard command \boldmath at the very start
% of the abstract to achieve this. Many IEEE journals frown on math
% in the abstract anyway.

% Note that keywords are not normally used for peerreview papers.
%\begin{IEEEkeywords}
%Cooperative diversity, decode and forward, piecewise linear
%\end{IEEEkeywords}



% For peer review papers, you can put extra information on the cover
% page as needed:
% \ifCLASSOPTIONpeerreview
% \begin{center} \bfseries EDICS Category: 3-BBND \end{center}
% \fi
%
% For peerreview papers, this IEEEtran command inserts a page break and
% creates the second title. It will be ignored for other modes.
%\IEEEpeerreviewmaketitle




\item Consider the probability space $\brak{\Omega, \mathcal{G}, P}$ where $\Omega = [0,2]$ and $\mathcal{G} = \cbrak{\phi, \Omega, [0,1], (1,2]}$. Let $X$ and $Y$ be two functions on $\Omega$ defined as
\begin{align*}
    X(\omega) = 
    \begin{cases}
        1 & \text{if }\omega \in [0, 1]\\
        2 & \text{if }\omega \in (1, 2]
    \end{cases}
\end{align*}
and
\begin{align*}
    Y(\omega) = 
    \begin{cases}
        2 & \text{if }\omega \in [0, 1.5]\\
        3 & \text{if }\omega \in (1.5, 2].
    \end{cases}
\end{align*}
Then which one of the following statements is true?
\begin{enumerate}
    \item [(A)] $X$ is a random variable with respect to $\mathcal{G}$, but $Y$ is not a random variable with respect to $\mathcal{G}$.
    \item [(B)] $Y$ is a random variable with respect to $\mathcal{G}$, but $X$ is not a random variable with respect to $\mathcal{G}$.
    \item [(C)] Neither $X$ nor $Y$ is a random variable with respect to $\mathcal{G}$.
    \item [(D)] Both $X$ and $Y$ are random variables with respect to $\mathcal{G}$.
\end{enumerate} \hfill (GATE ST 2023)\\
\solution
%\begin{table}[H]
	\centering
\begin{tabular}{|c|c|c|}
\hline
Random variable &Value &Definition\\ \hline
\multirow{3}{*}{X} &0 &Slips of Rs 1\\
&1 &Slips of Rs 5\\
&2 &Slips of Rs 13\\ \hline
\multirow{2}{*}{Y} &0 &Box A\\
&1 &Box B\\\hline
\end{tabular}
\caption{}
\label{tab:Distribution}
\end{table}
See \tabref{tab:Distribution}.
\begin{align}
p_{Y}\brak{k}= \begin{cases} 
      \frac{1}{3} & {k=0} \\
      \frac{2}{3 }& {k=1} 
   \end{cases}
   \\
p_{Y|X}\brak{0|0} = \frac{19}{25}\, 
p_{Y|X}\brak{0|1} = \frac{6}{25}\,
p_{Y|X}\brak{1|0} = \frac{45}{50}\,
p_{Y|X}\brak{1|2} = \frac{5}{50}
\end{align}
The desired probability is the probability that a slip drawn at random is marked other than Rs 1,
\begin{align}
&=1-p_X\brak{0}\\
&= p_X(1) + p_X(2)
\end{align}
Using Bayes theorem,
\begin{align}
&= p_Y\brak{0} \times \pr{Y=0 | X=1} + p_Y\brak{1} \times \pr{Y=1|X=2}\\
&=\frac{1}{3} \times \frac{6}{25} + \frac{2}{3} \times \frac{5}{50}\\
&=\frac{11}{75}
\end{align}

\newpage

%\tableofcontents

\bigskip

\renewcommand{\thefigure}{\theenumi}
\renewcommand{\thetable}{\theenumi}
%\renewcommand{\theequation}{\theenumi}

%\begin{abstract}
%%\boldmath
%In this letter, an algorithm for evaluating the exact analytical bit error rate  (BER)  for the piecewise linear (PL) combiner for  multiple relays is presented. Previous results were available only for upto three relays. The algorithm is unique in the sense that  the actual mathematical expressions, that are prohibitively large, need not be explicitly obtained. The diversity gain due to multiple relays is shown through plots of the analytical BER, well supported by simulations. 
%
%\end{abstract}
% IEEEtran.cls defaults to using nonbold math in the Abstract.
% This preserves the distinction between vectors and scalars. However,
% if the journal you are submitting to favors bold math in the abstract,
% then you can use LaTeX's standard command \boldmath at the very start
% of the abstract to achieve this. Many IEEE journals frown on math
% in the abstract anyway.

% Note that keywords are not normally used for peerreview papers.
%\begin{IEEEkeywords}
%Cooperative diversity, decode and forward, piecewise linear
%\end{IEEEkeywords}



% For peer review papers, you can put extra information on the cover
% page as needed:
% \ifCLASSOPTIONpeerreview
% \begin{center} \bfseries EDICS Category: 3-BBND \end{center}
% \fi
%
% For peerreview papers, this IEEEtran command inserts a page break and
% creates the second title. It will be ignored for other modes.
%\IEEEpeerreviewmaketitle




	\item  A die is loaded in such a way that each odd number is twice as likely to occur as
each even number. Find $P(G)$, where $G$ is the event that a number greater than
3 occurs on a single roll of the die.
\\
\solution
		%\begin{table}[H]
	\centering
\begin{tabular}{|c|c|c|}
\hline
Random variable &Value &Definition\\ \hline
\multirow{3}{*}{X} &0 &Slips of Rs 1\\
&1 &Slips of Rs 5\\
&2 &Slips of Rs 13\\ \hline
\multirow{2}{*}{Y} &0 &Box A\\
&1 &Box B\\\hline
\end{tabular}
\caption{}
\label{tab:Distribution}
\end{table}
See \tabref{tab:Distribution}.
\begin{align}
p_{Y}\brak{k}= \begin{cases} 
      \frac{1}{3} & {k=0} \\
      \frac{2}{3 }& {k=1} 
   \end{cases}
   \\
p_{Y|X}\brak{0|0} = \frac{19}{25}\, 
p_{Y|X}\brak{0|1} = \frac{6}{25}\,
p_{Y|X}\brak{1|0} = \frac{45}{50}\,
p_{Y|X}\brak{1|2} = \frac{5}{50}
\end{align}
The desired probability is the probability that a slip drawn at random is marked other than Rs 1,
\begin{align}
&=1-p_X\brak{0}\\
&= p_X(1) + p_X(2)
\end{align}
Using Bayes theorem,
\begin{align}
&= p_Y\brak{0} \times \pr{Y=0 | X=1} + p_Y\brak{1} \times \pr{Y=1|X=2}\\
&=\frac{1}{3} \times \frac{6}{25} + \frac{2}{3} \times \frac{5}{50}\\
&=\frac{11}{75}
\end{align}

\newpage

%\tableofcontents

\bigskip

\renewcommand{\thefigure}{\theenumi}
\renewcommand{\thetable}{\theenumi}
%\renewcommand{\theequation}{\theenumi}

%\begin{abstract}
%%\boldmath
%In this letter, an algorithm for evaluating the exact analytical bit error rate  (BER)  for the piecewise linear (PL) combiner for  multiple relays is presented. Previous results were available only for upto three relays. The algorithm is unique in the sense that  the actual mathematical expressions, that are prohibitively large, need not be explicitly obtained. The diversity gain due to multiple relays is shown through plots of the analytical BER, well supported by simulations. 
%
%\end{abstract}
% IEEEtran.cls defaults to using nonbold math in the Abstract.
% This preserves the distinction between vectors and scalars. However,
% if the journal you are submitting to favors bold math in the abstract,
% then you can use LaTeX's standard command \boldmath at the very start
% of the abstract to achieve this. Many IEEE journals frown on math
% in the abstract anyway.

% Note that keywords are not normally used for peerreview papers.
%\begin{IEEEkeywords}
%Cooperative diversity, decode and forward, piecewise linear
%\end{IEEEkeywords}



% For peer review papers, you can put extra information on the cover
% page as needed:
% \ifCLASSOPTIONpeerreview
% \begin{center} \bfseries EDICS Category: 3-BBND \end{center}
% \fi
%
% For peerreview papers, this IEEEtran command inserts a page break and
% creates the second title. It will be ignored for other modes.
%\IEEEpeerreviewmaketitle




	\item All the jacks, queens and kings are removed from a deck of 52 playing cards. The remaining cards are well shuffled and then one card is drawn at random. Giving ace a value 1 similar value for other cards, find the probability that the card has a value 
		\begin{enumerate}
			\item 7
			\item greater than 7
			\item less than 7
		\end{enumerate}
		%Number of cards left after removing all jacks, queens and kings 
\begin{align}
N	= 52 - 4\times 3
	= 40
\end{align}
%\begin{table}[H]
%\def\arraystretch{1.2}
%\begin{tabular}{|c|c|c|}
%\hline
%	\textbf{Parameter} &\textbf{Value} &\textbf{Description}\\ \hline
%	$X$ &1-10 &Represents the value of the card picked \\ \hline
%\end{tabular}
%\end{table}
Let $1 \le X \le 10$ be the value of the card picked.  Then,
\begin{align}
	p_X(k) &= \Pr(X=k)\ \forall\ 1 \leq k \leq 10\\
	&= \frac{4\times 1}{40}\\
	&= \frac{1}{10}\\
	\therefore p_X(k) &= 
	\begin{cases}
		\frac{1}{10} & 1 \leq k \leq 10\\
		0 & \text{otherwise}
	\end{cases}
\end{align}
and
\begin{align}
	F_{X}(k) &= \sum_{m=0}^{k}p_{X}(m) \quad 1 \leq k \leq 10\\
	&= \frac{k}{10}\\
	\therefore F_{X}(k) &= 
	\begin{cases}
		0 & k \leq 0\\
		\frac{k}{10} & 1\leq k \leq 10\\
		1 & k > 10 
	\end{cases}
\end{align}
\begin{enumerate}
	\item Probability that card has value equal to 7 is
		\begin{align}
			 p_{X}(7)
			= \frac{1}{10}
		\end{align}
	\item Probability that card has value greater than 7 is
		\begin{align}
			1 - F_X(7)
			&= 1 - \frac{7}{10}
			\\
			&= \frac{3}{10}
		\end{align}
	\item Probability that card has value less than 7 is
		\begin{align}
			 F_{X}(6)
			=\frac{6}{10}
		\end{align}
\end{enumerate}

  \item A Lot consists of 48 mobile phones of which 42 are good, 3 have only minor defects and 3 have major defects.Varnika will buy a phone if it is good but the trader will only buy a mobile if it has no major defects. One phone is selected at random from the lot. What is the probability that it is
\begin{enumerate}
	\item acceptable to Varnika?
            \item acceptable to the trader?
\end{enumerate}
\solution
	%\begin{table}[H]
	\centering
\begin{tabular}{|c|c|c|}
\hline
Random variable &Value &Definition\\ \hline
\multirow{3}{*}{X} &0 &Slips of Rs 1\\
&1 &Slips of Rs 5\\
&2 &Slips of Rs 13\\ \hline
\multirow{2}{*}{Y} &0 &Box A\\
&1 &Box B\\\hline
\end{tabular}
\caption{}
\label{tab:Distribution}
\end{table}
See \tabref{tab:Distribution}.
\begin{align}
p_{Y}\brak{k}= \begin{cases} 
      \frac{1}{3} & {k=0} \\
      \frac{2}{3 }& {k=1} 
   \end{cases}
   \\
p_{Y|X}\brak{0|0} = \frac{19}{25}\, 
p_{Y|X}\brak{0|1} = \frac{6}{25}\,
p_{Y|X}\brak{1|0} = \frac{45}{50}\,
p_{Y|X}\brak{1|2} = \frac{5}{50}
\end{align}
The desired probability is the probability that a slip drawn at random is marked other than Rs 1,
\begin{align}
&=1-p_X\brak{0}\\
&= p_X(1) + p_X(2)
\end{align}
Using Bayes theorem,
\begin{align}
&= p_Y\brak{0} \times \pr{Y=0 | X=1} + p_Y\brak{1} \times \pr{Y=1|X=2}\\
&=\frac{1}{3} \times \frac{6}{25} + \frac{2}{3} \times \frac{5}{50}\\
&=\frac{11}{75}
\end{align}

\newpage

%\tableofcontents

\bigskip

\renewcommand{\thefigure}{\theenumi}
\renewcommand{\thetable}{\theenumi}
%\renewcommand{\theequation}{\theenumi}

%\begin{abstract}
%%\boldmath
%In this letter, an algorithm for evaluating the exact analytical bit error rate  (BER)  for the piecewise linear (PL) combiner for  multiple relays is presented. Previous results were available only for upto three relays. The algorithm is unique in the sense that  the actual mathematical expressions, that are prohibitively large, need not be explicitly obtained. The diversity gain due to multiple relays is shown through plots of the analytical BER, well supported by simulations. 
%
%\end{abstract}
% IEEEtran.cls defaults to using nonbold math in the Abstract.
% This preserves the distinction between vectors and scalars. However,
% if the journal you are submitting to favors bold math in the abstract,
% then you can use LaTeX's standard command \boldmath at the very start
% of the abstract to achieve this. Many IEEE journals frown on math
% in the abstract anyway.

% Note that keywords are not normally used for peerreview papers.
%\begin{IEEEkeywords}
%Cooperative diversity, decode and forward, piecewise linear
%\end{IEEEkeywords}



% For peer review papers, you can put extra information on the cover
% page as needed:
% \ifCLASSOPTIONpeerreview
% \begin{center} \bfseries EDICS Category: 3-BBND \end{center}
% \fi
%
% For peerreview papers, this IEEEtran command inserts a page break and
% creates the second title. It will be ignored for other modes.
%\IEEEpeerreviewmaketitle




 \item A student says that if you throw a die, it will show up 1 or not 1. Therefore, the probability of getting 1 and the probability of getting 'not 1' each is equal to $\frac{1}{2}$. Is this correct? Give reasons.\\
 \solution
        %\begin{table}[H]
	\centering
\begin{tabular}{|c|c|c|}
\hline
Random variable &Value &Definition\\ \hline
\multirow{3}{*}{X} &0 &Slips of Rs 1\\
&1 &Slips of Rs 5\\
&2 &Slips of Rs 13\\ \hline
\multirow{2}{*}{Y} &0 &Box A\\
&1 &Box B\\\hline
\end{tabular}
\caption{}
\label{tab:Distribution}
\end{table}
See \tabref{tab:Distribution}.
\begin{align}
p_{Y}\brak{k}= \begin{cases} 
      \frac{1}{3} & {k=0} \\
      \frac{2}{3 }& {k=1} 
   \end{cases}
   \\
p_{Y|X}\brak{0|0} = \frac{19}{25}\, 
p_{Y|X}\brak{0|1} = \frac{6}{25}\,
p_{Y|X}\brak{1|0} = \frac{45}{50}\,
p_{Y|X}\brak{1|2} = \frac{5}{50}
\end{align}
The desired probability is the probability that a slip drawn at random is marked other than Rs 1,
\begin{align}
&=1-p_X\brak{0}\\
&= p_X(1) + p_X(2)
\end{align}
Using Bayes theorem,
\begin{align}
&= p_Y\brak{0} \times \pr{Y=0 | X=1} + p_Y\brak{1} \times \pr{Y=1|X=2}\\
&=\frac{1}{3} \times \frac{6}{25} + \frac{2}{3} \times \frac{5}{50}\\
&=\frac{11}{75}
\end{align}

\newpage

%\tableofcontents

\bigskip

\renewcommand{\thefigure}{\theenumi}
\renewcommand{\thetable}{\theenumi}
%\renewcommand{\theequation}{\theenumi}

%\begin{abstract}
%%\boldmath
%In this letter, an algorithm for evaluating the exact analytical bit error rate  (BER)  for the piecewise linear (PL) combiner for  multiple relays is presented. Previous results were available only for upto three relays. The algorithm is unique in the sense that  the actual mathematical expressions, that are prohibitively large, need not be explicitly obtained. The diversity gain due to multiple relays is shown through plots of the analytical BER, well supported by simulations. 
%
%\end{abstract}
% IEEEtran.cls defaults to using nonbold math in the Abstract.
% This preserves the distinction between vectors and scalars. However,
% if the journal you are submitting to favors bold math in the abstract,
% then you can use LaTeX's standard command \boldmath at the very start
% of the abstract to achieve this. Many IEEE journals frown on math
% in the abstract anyway.

% Note that keywords are not normally used for peerreview papers.
%\begin{IEEEkeywords}
%Cooperative diversity, decode and forward, piecewise linear
%\end{IEEEkeywords}



% For peer review papers, you can put extra information on the cover
% page as needed:
% \ifCLASSOPTIONpeerreview
% \begin{center} \bfseries EDICS Category: 3-BBND \end{center}
% \fi
%
% For peerreview papers, this IEEEtran command inserts a page break and
% creates the second title. It will be ignored for other modes.
%\IEEEpeerreviewmaketitle




   \item Four candidates A, B, C, D have ap-
plied for the assignment to coach a school cricket
team. If A is twice as likely to be selected as B, and
B and C are given about the same chance of being
selected, while C is twice as likely to be selected
as D, what are the probabilities that
\begin{enumerate}
\item C will be selected?
\item A will not be selected?
\end{enumerate}
	%\begin{table}[H]
	\centering
\begin{tabular}{|c|c|c|}
\hline
Random variable &Value &Definition\\ \hline
\multirow{3}{*}{X} &0 &Slips of Rs 1\\
&1 &Slips of Rs 5\\
&2 &Slips of Rs 13\\ \hline
\multirow{2}{*}{Y} &0 &Box A\\
&1 &Box B\\\hline
\end{tabular}
\caption{}
\label{tab:Distribution}
\end{table}
See \tabref{tab:Distribution}.
\begin{align}
p_{Y}\brak{k}= \begin{cases} 
      \frac{1}{3} & {k=0} \\
      \frac{2}{3 }& {k=1} 
   \end{cases}
   \\
p_{Y|X}\brak{0|0} = \frac{19}{25}\, 
p_{Y|X}\brak{0|1} = \frac{6}{25}\,
p_{Y|X}\brak{1|0} = \frac{45}{50}\,
p_{Y|X}\brak{1|2} = \frac{5}{50}
\end{align}
The desired probability is the probability that a slip drawn at random is marked other than Rs 1,
\begin{align}
&=1-p_X\brak{0}\\
&= p_X(1) + p_X(2)
\end{align}
Using Bayes theorem,
\begin{align}
&= p_Y\brak{0} \times \pr{Y=0 | X=1} + p_Y\brak{1} \times \pr{Y=1|X=2}\\
&=\frac{1}{3} \times \frac{6}{25} + \frac{2}{3} \times \frac{5}{50}\\
&=\frac{11}{75}
\end{align}

\newpage

%\tableofcontents

\bigskip

\renewcommand{\thefigure}{\theenumi}
\renewcommand{\thetable}{\theenumi}
%\renewcommand{\theequation}{\theenumi}

%\begin{abstract}
%%\boldmath
%In this letter, an algorithm for evaluating the exact analytical bit error rate  (BER)  for the piecewise linear (PL) combiner for  multiple relays is presented. Previous results were available only for upto three relays. The algorithm is unique in the sense that  the actual mathematical expressions, that are prohibitively large, need not be explicitly obtained. The diversity gain due to multiple relays is shown through plots of the analytical BER, well supported by simulations. 
%
%\end{abstract}
% IEEEtran.cls defaults to using nonbold math in the Abstract.
% This preserves the distinction between vectors and scalars. However,
% if the journal you are submitting to favors bold math in the abstract,
% then you can use LaTeX's standard command \boldmath at the very start
% of the abstract to achieve this. Many IEEE journals frown on math
% in the abstract anyway.

% Note that keywords are not normally used for peerreview papers.
%\begin{IEEEkeywords}
%Cooperative diversity, decode and forward, piecewise linear
%\end{IEEEkeywords}



% For peer review papers, you can put extra information on the cover
% page as needed:
% \ifCLASSOPTIONpeerreview
% \begin{center} \bfseries EDICS Category: 3-BBND \end{center}
% \fi
%
% For peerreview papers, this IEEEtran command inserts a page break and
% creates the second title. It will be ignored for other modes.
%\IEEEpeerreviewmaketitle




 \item A bag contain 24 balls of which $x$ balls are red, $2x$ are white and $3x$ are blue. A ball is selected at random, What is the probability that it is
\begin{enumerate}[label=\alph*)]
\item not red ?
\item white ?
\end{enumerate}
%\begin{table}[H]
	\centering
\begin{tabular}{|c|c|c|}
\hline
Random variable &Value &Definition\\ \hline
\multirow{3}{*}{X} &0 &Slips of Rs 1\\
&1 &Slips of Rs 5\\
&2 &Slips of Rs 13\\ \hline
\multirow{2}{*}{Y} &0 &Box A\\
&1 &Box B\\\hline
\end{tabular}
\caption{}
\label{tab:Distribution}
\end{table}
See \tabref{tab:Distribution}.
\begin{align}
p_{Y}\brak{k}= \begin{cases} 
      \frac{1}{3} & {k=0} \\
      \frac{2}{3 }& {k=1} 
   \end{cases}
   \\
p_{Y|X}\brak{0|0} = \frac{19}{25}\, 
p_{Y|X}\brak{0|1} = \frac{6}{25}\,
p_{Y|X}\brak{1|0} = \frac{45}{50}\,
p_{Y|X}\brak{1|2} = \frac{5}{50}
\end{align}
The desired probability is the probability that a slip drawn at random is marked other than Rs 1,
\begin{align}
&=1-p_X\brak{0}\\
&= p_X(1) + p_X(2)
\end{align}
Using Bayes theorem,
\begin{align}
&= p_Y\brak{0} \times \pr{Y=0 | X=1} + p_Y\brak{1} \times \pr{Y=1|X=2}\\
&=\frac{1}{3} \times \frac{6}{25} + \frac{2}{3} \times \frac{5}{50}\\
&=\frac{11}{75}
\end{align}

\newpage

%\tableofcontents

\bigskip

\renewcommand{\thefigure}{\theenumi}
\renewcommand{\thetable}{\theenumi}
%\renewcommand{\theequation}{\theenumi}

%\begin{abstract}
%%\boldmath
%In this letter, an algorithm for evaluating the exact analytical bit error rate  (BER)  for the piecewise linear (PL) combiner for  multiple relays is presented. Previous results were available only for upto three relays. The algorithm is unique in the sense that  the actual mathematical expressions, that are prohibitively large, need not be explicitly obtained. The diversity gain due to multiple relays is shown through plots of the analytical BER, well supported by simulations. 
%
%\end{abstract}
% IEEEtran.cls defaults to using nonbold math in the Abstract.
% This preserves the distinction between vectors and scalars. However,
% if the journal you are submitting to favors bold math in the abstract,
% then you can use LaTeX's standard command \boldmath at the very start
% of the abstract to achieve this. Many IEEE journals frown on math
% in the abstract anyway.

% Note that keywords are not normally used for peerreview papers.
%\begin{IEEEkeywords}
%Cooperative diversity, decode and forward, piecewise linear
%\end{IEEEkeywords}



% For peer review papers, you can put extra information on the cover
% page as needed:
% \ifCLASSOPTIONpeerreview
% \begin{center} \bfseries EDICS Category: 3-BBND \end{center}
% \fi
%
% For peerreview papers, this IEEEtran command inserts a page break and
% creates the second title. It will be ignored for other modes.
%\IEEEpeerreviewmaketitle




If the letters of the word ASSASSINATION are arranged at random. Find the Probability that
\begin{enumerate}[label=(\alph*)]
\item Four $S's$ come consecutively in the word
\item Two  $I's$ and two $N's$ come together
\item All $A's$ are not coming together
\item No two $A's$ are coming together
\end{enumerate}
%\begin{table}[H]
	\centering
\begin{tabular}{|c|c|c|}
\hline
Random variable &Value &Definition\\ \hline
\multirow{3}{*}{X} &0 &Slips of Rs 1\\
&1 &Slips of Rs 5\\
&2 &Slips of Rs 13\\ \hline
\multirow{2}{*}{Y} &0 &Box A\\
&1 &Box B\\\hline
\end{tabular}
\caption{}
\label{tab:Distribution}
\end{table}
See \tabref{tab:Distribution}.
\begin{align}
p_{Y}\brak{k}= \begin{cases} 
      \frac{1}{3} & {k=0} \\
      \frac{2}{3 }& {k=1} 
   \end{cases}
   \\
p_{Y|X}\brak{0|0} = \frac{19}{25}\, 
p_{Y|X}\brak{0|1} = \frac{6}{25}\,
p_{Y|X}\brak{1|0} = \frac{45}{50}\,
p_{Y|X}\brak{1|2} = \frac{5}{50}
\end{align}
The desired probability is the probability that a slip drawn at random is marked other than Rs 1,
\begin{align}
&=1-p_X\brak{0}\\
&= p_X(1) + p_X(2)
\end{align}
Using Bayes theorem,
\begin{align}
&= p_Y\brak{0} \times \pr{Y=0 | X=1} + p_Y\brak{1} \times \pr{Y=1|X=2}\\
&=\frac{1}{3} \times \frac{6}{25} + \frac{2}{3} \times \frac{5}{50}\\
&=\frac{11}{75}
\end{align}

\newpage

%\tableofcontents

\bigskip

\renewcommand{\thefigure}{\theenumi}
\renewcommand{\thetable}{\theenumi}
%\renewcommand{\theequation}{\theenumi}

%\begin{abstract}
%%\boldmath
%In this letter, an algorithm for evaluating the exact analytical bit error rate  (BER)  for the piecewise linear (PL) combiner for  multiple relays is presented. Previous results were available only for upto three relays. The algorithm is unique in the sense that  the actual mathematical expressions, that are prohibitively large, need not be explicitly obtained. The diversity gain due to multiple relays is shown through plots of the analytical BER, well supported by simulations. 
%
%\end{abstract}
% IEEEtran.cls defaults to using nonbold math in the Abstract.
% This preserves the distinction between vectors and scalars. However,
% if the journal you are submitting to favors bold math in the abstract,
% then you can use LaTeX's standard command \boldmath at the very start
% of the abstract to achieve this. Many IEEE journals frown on math
% in the abstract anyway.

% Note that keywords are not normally used for peerreview papers.
%\begin{IEEEkeywords}
%Cooperative diversity, decode and forward, piecewise linear
%\end{IEEEkeywords}



% For peer review papers, you can put extra information on the cover
% page as needed:
% \ifCLASSOPTIONpeerreview
% \begin{center} \bfseries EDICS Category: 3-BBND \end{center}
% \fi
%
% For peerreview papers, this IEEEtran command inserts a page break and
% creates the second title. It will be ignored for other modes.
%\IEEEpeerreviewmaketitle




	\item One urn contains two black balls (labelled B1 and B2) and one white ball. A
	second urn contains one black ball and two white balls (labelled W1 and W2).
	Suppose the following experiment is performed. One of the two urns is chosen
	at random. Next a ball is randomly chosen from the urn. Then a second ball is
	chosen at random from the same urn without replacing the first ball.
	
	\begin{enumerate}
	\item What is the probability that two black balls are chosen?
	
	\item What is the probability that two balls of opposite colour are chosen?
	\end{enumerate}
	\solution
	%\begin{align}
    \label{eq:12.13.6.18.1}
	\because	\pr{A|B} &> \pr{A},\
\frac{\pr{AB}}{\pr{B}} > \pr{A}
\\
    \label{eq:12.13.6.18.2}
	\implies \pr{AB} &> \pr{A}\pr{B}
	\\
	\text{or, } \frac{\pr{AB}}{\pr{A}} &=\pr{B|A} > \pr{A}
\end{align}

\end{enumerate}

	\item A bag contains $5$ red balls and some blue balls. If the probability of drawing a blue ball is double that if a red ball, determine the number of blue balls in the bag. 
		\\
\solution
		%\begin{enumerate}[label=\thesection.\arabic*,ref=\thesection.\theenumi]
	\item One card is drawn from a well-shuffled deck of 52 cards. Find the probability of getting
\begin{enumerate}
\item A king of red colour 
\item A face card 
\item A red face card
\item The jack of hearts
\item A spade
\item The queen of diamonds

\end{enumerate}
\solution
		%\begin{table}[H]
	\centering
\begin{tabular}{|c|c|c|}
\hline
Random variable &Value &Definition\\ \hline
\multirow{3}{*}{X} &0 &Slips of Rs 1\\
&1 &Slips of Rs 5\\
&2 &Slips of Rs 13\\ \hline
\multirow{2}{*}{Y} &0 &Box A\\
&1 &Box B\\\hline
\end{tabular}
\caption{}
\label{tab:Distribution}
\end{table}
See \tabref{tab:Distribution}.
\begin{align}
p_{Y}\brak{k}= \begin{cases} 
      \frac{1}{3} & {k=0} \\
      \frac{2}{3 }& {k=1} 
   \end{cases}
   \\
p_{Y|X}\brak{0|0} = \frac{19}{25}\, 
p_{Y|X}\brak{0|1} = \frac{6}{25}\,
p_{Y|X}\brak{1|0} = \frac{45}{50}\,
p_{Y|X}\brak{1|2} = \frac{5}{50}
\end{align}
The desired probability is the probability that a slip drawn at random is marked other than Rs 1,
\begin{align}
&=1-p_X\brak{0}\\
&= p_X(1) + p_X(2)
\end{align}
Using Bayes theorem,
\begin{align}
&= p_Y\brak{0} \times \pr{Y=0 | X=1} + p_Y\brak{1} \times \pr{Y=1|X=2}\\
&=\frac{1}{3} \times \frac{6}{25} + \frac{2}{3} \times \frac{5}{50}\\
&=\frac{11}{75}
\end{align}

\newpage

%\tableofcontents

\bigskip

\renewcommand{\thefigure}{\theenumi}
\renewcommand{\thetable}{\theenumi}
%\renewcommand{\theequation}{\theenumi}

%\begin{abstract}
%%\boldmath
%In this letter, an algorithm for evaluating the exact analytical bit error rate  (BER)  for the piecewise linear (PL) combiner for  multiple relays is presented. Previous results were available only for upto three relays. The algorithm is unique in the sense that  the actual mathematical expressions, that are prohibitively large, need not be explicitly obtained. The diversity gain due to multiple relays is shown through plots of the analytical BER, well supported by simulations. 
%
%\end{abstract}
% IEEEtran.cls defaults to using nonbold math in the Abstract.
% This preserves the distinction between vectors and scalars. However,
% if the journal you are submitting to favors bold math in the abstract,
% then you can use LaTeX's standard command \boldmath at the very start
% of the abstract to achieve this. Many IEEE journals frown on math
% in the abstract anyway.

% Note that keywords are not normally used for peerreview papers.
%\begin{IEEEkeywords}
%Cooperative diversity, decode and forward, piecewise linear
%\end{IEEEkeywords}



% For peer review papers, you can put extra information on the cover
% page as needed:
% \ifCLASSOPTIONpeerreview
% \begin{center} \bfseries EDICS Category: 3-BBND \end{center}
% \fi
%
% For peerreview papers, this IEEEtran command inserts a page break and
% creates the second title. It will be ignored for other modes.
%\IEEEpeerreviewmaketitle




	\item Five cards—the ten, jack, queen, king and ace of diamonds, are well-shuffled with their face downwards. One card is then picked up at random.
\begin{enumerate}
\item
What is the probability that the card is the queen? 
\item
If the queen is drawn and put aside, what is the probability that the second card picked up is (a) an ace? (b) a queen?\\
\end{enumerate}
\solution
		%\begin{enumerate}[label=\thesection.\arabic*,ref=\thesection.\theenumi]
	\item One card is drawn from a well-shuffled deck of 52 cards. Find the probability of getting
\begin{enumerate}
\item A king of red colour 
\item A face card 
\item A red face card
\item The jack of hearts
\item A spade
\item The queen of diamonds

\end{enumerate}
\solution
		%\input{ncert/10/15/1/14/main.tex}
	\item Five cards—the ten, jack, queen, king and ace of diamonds, are well-shuffled with their face downwards. One card is then picked up at random.
\begin{enumerate}
\item
What is the probability that the card is the queen? 
\item
If the queen is drawn and put aside, what is the probability that the second card picked up is (a) an ace? (b) a queen?\\
\end{enumerate}
\solution
		%\input{ncert/10/15/1/15/defs.tex}
	\item A bag contains $5$ red balls and some blue balls. If the probability of drawing a blue ball is double that if a red ball, determine the number of blue balls in the bag. 
		\\
\solution
		%\input{ncert/10/15/2/3/defs.tex}
	\item A card is selected from a pack of 52 cards.
 \begin{enumerate}[label=(\alph*)] 
                 \item How many points are there in the sample space?
                 \item Calculate the probability that the card is an ace of spades.
                 \item Calculate the probability that the card is (i) an ace and (ii) black card.
 \end{enumerate}
\solution
		%\input{ncert/11/16/3/4/main.tex}
\item Four cards are drawn from a well-shuffled deck of 52 cards. What is the probability of obtaining 3 diamonds and one spade.
\\
\solution
		%\input{ncert/11/16/4/2/defs.tex}
\item In a certain lottery 10,000 tickets are sold and ten equal prizes are awarded. What is the probability of not getting a prize if you buy (a) one ticket (b) two tickets (c) 10 tickets ?	
\\
\solution
		%\input{ncert/11/16/4/4/defs.tex}
		%
\item 
Out of 100 students, two sections of 40 and 60 are formed. If you and your friend are among the 100 students, what is the probability that
\begin{enumerate}
\item you both enter the same section?
\item you both enter the different sections?
\end{enumerate}
\solution
		%\input{ncert/11/16/4/5/defs.tex}
	\item 
The number lock of a suitcase has 4 wheels each labelled with ten digits i.e. from 0 to 9.The lock opens with a sequence of four digits with no repeats.What is the probability of a person getting the right sequence to open the suitcase.
\\
\solution
		%\input{ncert/11/16/4/10/defs.tex}
		%
\item 
Two cards are drawn at random and without replacement from a pack of 52 playing cards. Find the probability that both the cards are black.
\\
\solution
		%\input{ncert/12/13/2/2/defs.tex}
		\item A box of oranges is inspected by examining three randomly selected oranges drawn without replacement. If all the three oranges are good, the box is approved for sale, otherwise, it is rejected. Find the probability that a box containing 15 oranges out of which 12 are good and 3 are bad ones will be approved for sale.
		\label{ncert/12/13/2/3/defs.tex}
		\item Two balls are drawn at random with replacement from a box containing 10 black and 8 red balls. Find the probability that
		\label{ncert/12/13/2/12}
\begin{enumerate}
\item both balls are red.
\item first ball is black and second is red.
\item one of them is black and other is red.
\end{enumerate}

\item In a hostel, 60\% of the students read Hindi newspaper, 40\% read English newspaper and 20\% read both Hindi and English newspapers. A student is selected at random.
		\label{ncert/12/13/2/15}
\begin{enumerate}
\item Find the probability that she reads neither Hindi nor English newspapers.
\item If she reads Hindi newspaper, find the probability that she reads English newspaper.
\item If she reads English newspaper, find the probability that she reads Hindi newspaper.\\
\end{enumerate}
\item The probability of obtaining an even prime number on each die, when a pair of dice is rolled is 
\begin{enumerate}
    \item $0$ 
    
    \item $\frac{1}{3}$ 
    
    \item $\frac{1}{12}$ 
    
    \item $\frac{1}{36}$ 
\end{enumerate}
\solution
		%\input{ncert/12/13/2/17/defs.tex}
	\item A bag contains 4 red and 4 black balls, another bag contains 2 red and 6 black balls. One of the two bags is selected at random and a ball is drawn from the bag which is found to be red. Find the probability that the ball is drawn from the first bag.
\\
\solution
		%\input{ncert/12/13/3/2/main.tex}
  \item
  Cards with numbers 2 to 101 are placed in a box. A card is selected at random.Find the probability that the card has
\begin{enumerate}[label=(\roman*)]
	\item an even number 
	\item a square number
\end{enumerate}
\solution
%\input{exemplar/10/13/3/32/main.tex}
\item
The king, queen and jack of clubs are removed from a deck of 52 playing cards and then well shuffled. Now one card is drawn at random from the remaining cards.  Determine the probability that the card is
\begin{enumerate}[label=(\roman*)]
\item a club
\item 10 of hearts
\end{enumerate}
\solution
%\input{exemplar/10/13/3/29/main.tex}
\item A team of medical students doing their internship have to assist during surgeries
at a city hospital. The probabilities of surgeries rated as very complex, complex,
routine, simple or very simple are respectively, 0.15, 0.20, 0.31, 0.26, .08. Find
the probabilities that a particular surgery will be rated
\begin{enumerate}
	\item complex or very complex;
	\item neither very complex nor very simple;
	\item routine or complex
	\item routine or simple
\end{enumerate}
\solution
%\input{exemplar/11/16/3/8(1)/main.tex}
\item A card is selected from a pack of 52 cards.
\begin{enumerate}[label=(\alph*)]
    \item How many points are there in the sample space?
    \item Calculate the probability that the card is an ace of spades.
    \item Calculate the probability that the card is (i) an ace and (ii) black card.
\end{enumerate}
\solution
%\input{exemplar/11/16/3/4/main2.tex}
\item The probability that a non leap year selected at random will contain 53 sundays.
\\
\solution
%\input{exemplar/10/13/1/19/main.tex}
\item One of the four persons John, Rita, Aslam or Gurpreet will be promoted next
month. Consequently the sample space consists of four elementary outcomes
S = {John promoted, Rita promoted, Aslam promoted, Gurpreet promoted}
You are told that the chances of John’s promotion is same as that of Gurpreet,
Rita’s chances of promotion are twice as likely as Johns. Aslam’s chances are
four times that of John.
\begin{enumerate}
	\item Determine
	\begin{enumerate}
		\item P (John promoted)
		\item P (Rita promoted)
		\item P (Aslam promoted)
		\item P (Gurpreet promoted)
	\end{enumerate}
	\item If A = {John promoted or Gurpreet promoted}, find P (A).
\end{enumerate}
\solution
%\input{exemplar/11/16/3/10/main.tex}
\item A card is drawn from a deck of 52 cards. Find the probability of getting a king or a heart or a red card.\\
\solution
%\input{exemplar/11/16/3/15/main.tex}
\item The probability that a student will pass his examination is 0.73, the probability of
the student getting a compartment is 0.13, and the probability that the student will
either pass or get compartment is 0.96. State True or False.\\
\solution
%\input{exemplar/11/16/3/31/main.tex}
\item A card is selected from a pack of 52 cards\\
\begin{enumerate}[label=(\alph*)]
\item How many points are there in the sample space?
\item Calculate the probability that the cards is an ace of spades.
\item Calculate the probability that the card is (i) an ace (ii)black card.\\
\end{enumerate}
%\input{ncert/11/16/3/4_1/Prob_4.tex}
\item In a non-leap year, the probability of having 53 tuesdays or 53 wednesdays is\\
\solution
%\input{exemplar/11/16/3/18/main.tex}
\item There are 1000 sealed envelopes in a box, 10 of them contain a cash prize of
Rs 100 each, 100 of them contain a cash prize of Rs 50 each and 200 of them
contain a cash prize of Rs 10 each and rest do not contain any cash prize. If they
are well shuffled and an envelope is picked up out, what is the probability that it
contains no cash prize?\\
\solution
%\input{exemplar/10/13/3/34/main.tex}
\item 
A die is thrown and a card is selected at random from a deck of 52 playing cards. The probability of getting an even number on the die and a spade card.\\
\solution
%\input{exemplar/12/13/3/78/main.tex}
\item
If 4-digit numbers greater than 5,000 are randomly formed from the digits 0, 1, 3, 5, and 7, what is the probability of forming a number divisible by 5 when:
\begin{enumerate}
    \item The digits are repeated?
    \item The repetition of digits is not allowed?
\end{enumerate}
\solution
%\input{ncert/11/16/4/9/main.tex}
\item Consider the probability space $\brak{\Omega, \mathcal{G}, P}$ where $\Omega = [0,2]$ and $\mathcal{G} = \cbrak{\phi, \Omega, [0,1], (1,2]}$. Let $X$ and $Y$ be two functions on $\Omega$ defined as
\begin{align*}
    X(\omega) = 
    \begin{cases}
        1 & \text{if }\omega \in [0, 1]\\
        2 & \text{if }\omega \in (1, 2]
    \end{cases}
\end{align*}
and
\begin{align*}
    Y(\omega) = 
    \begin{cases}
        2 & \text{if }\omega \in [0, 1.5]\\
        3 & \text{if }\omega \in (1.5, 2].
    \end{cases}
\end{align*}
Then which one of the following statements is true?
\begin{enumerate}
    \item [(A)] $X$ is a random variable with respect to $\mathcal{G}$, but $Y$ is not a random variable with respect to $\mathcal{G}$.
    \item [(B)] $Y$ is a random variable with respect to $\mathcal{G}$, but $X$ is not a random variable with respect to $\mathcal{G}$.
    \item [(C)] Neither $X$ nor $Y$ is a random variable with respect to $\mathcal{G}$.
    \item [(D)] Both $X$ and $Y$ are random variables with respect to $\mathcal{G}$.
\end{enumerate} \hfill (GATE ST 2023)\\
\solution
%\input{gate/ST/2023/14/main.tex}
	\item  A die is loaded in such a way that each odd number is twice as likely to occur as
each even number. Find $P(G)$, where $G$ is the event that a number greater than
3 occurs on a single roll of the die.
\\
\solution
		%\input{exemplar/11/16/3/5/main.tex}
	\item All the jacks, queens and kings are removed from a deck of 52 playing cards. The remaining cards are well shuffled and then one card is drawn at random. Giving ace a value 1 similar value for other cards, find the probability that the card has a value 
		\begin{enumerate}
			\item 7
			\item greater than 7
			\item less than 7
		\end{enumerate}
		%\input{exemplar/10/13/3/30/main.tex}
  \item A Lot consists of 48 mobile phones of which 42 are good, 3 have only minor defects and 3 have major defects.Varnika will buy a phone if it is good but the trader will only buy a mobile if it has no major defects. One phone is selected at random from the lot. What is the probability that it is
\begin{enumerate}
	\item acceptable to Varnika?
            \item acceptable to the trader?
\end{enumerate}
\solution
	%\input{exemplar/10/13/3/40/main.tex}
 \item A student says that if you throw a die, it will show up 1 or not 1. Therefore, the probability of getting 1 and the probability of getting 'not 1' each is equal to $\frac{1}{2}$. Is this correct? Give reasons.\\
 \solution
        %\input{exemplar/10/13/2/9/main.tex}
   \item Four candidates A, B, C, D have ap-
plied for the assignment to coach a school cricket
team. If A is twice as likely to be selected as B, and
B and C are given about the same chance of being
selected, while C is twice as likely to be selected
as D, what are the probabilities that
\begin{enumerate}
\item C will be selected?
\item A will not be selected?
\end{enumerate}
	%\input{exemplar/11/16/3/9/main.tex}
 \item A bag contain 24 balls of which $x$ balls are red, $2x$ are white and $3x$ are blue. A ball is selected at random, What is the probability that it is
\begin{enumerate}[label=\alph*)]
\item not red ?
\item white ?
\end{enumerate}
%\input{exemplar/10/13/3/41/main.tex}
If the letters of the word ASSASSINATION are arranged at random. Find the Probability that
\begin{enumerate}[label=(\alph*)]
\item Four $S's$ come consecutively in the word
\item Two  $I's$ and two $N's$ come together
\item All $A's$ are not coming together
\item No two $A's$ are coming together
\end{enumerate}
%\input{exemplar/11/16/3/14/main.tex}
	\item One urn contains two black balls (labelled B1 and B2) and one white ball. A
	second urn contains one black ball and two white balls (labelled W1 and W2).
	Suppose the following experiment is performed. One of the two urns is chosen
	at random. Next a ball is randomly chosen from the urn. Then a second ball is
	chosen at random from the same urn without replacing the first ball.
	
	\begin{enumerate}
	\item What is the probability that two black balls are chosen?
	
	\item What is the probability that two balls of opposite colour are chosen?
	\end{enumerate}
	\solution
	%\input{exemplar/11/16/3/12/main1.tex}
\end{enumerate}

	\item A bag contains $5$ red balls and some blue balls. If the probability of drawing a blue ball is double that if a red ball, determine the number of blue balls in the bag. 
		\\
\solution
		%\begin{enumerate}[label=\thesection.\arabic*,ref=\thesection.\theenumi]
	\item One card is drawn from a well-shuffled deck of 52 cards. Find the probability of getting
\begin{enumerate}
\item A king of red colour 
\item A face card 
\item A red face card
\item The jack of hearts
\item A spade
\item The queen of diamonds

\end{enumerate}
\solution
		%\input{ncert/10/15/1/14/main.tex}
	\item Five cards—the ten, jack, queen, king and ace of diamonds, are well-shuffled with their face downwards. One card is then picked up at random.
\begin{enumerate}
\item
What is the probability that the card is the queen? 
\item
If the queen is drawn and put aside, what is the probability that the second card picked up is (a) an ace? (b) a queen?\\
\end{enumerate}
\solution
		%\input{ncert/10/15/1/15/defs.tex}
	\item A bag contains $5$ red balls and some blue balls. If the probability of drawing a blue ball is double that if a red ball, determine the number of blue balls in the bag. 
		\\
\solution
		%\input{ncert/10/15/2/3/defs.tex}
	\item A card is selected from a pack of 52 cards.
 \begin{enumerate}[label=(\alph*)] 
                 \item How many points are there in the sample space?
                 \item Calculate the probability that the card is an ace of spades.
                 \item Calculate the probability that the card is (i) an ace and (ii) black card.
 \end{enumerate}
\solution
		%\input{ncert/11/16/3/4/main.tex}
\item Four cards are drawn from a well-shuffled deck of 52 cards. What is the probability of obtaining 3 diamonds and one spade.
\\
\solution
		%\input{ncert/11/16/4/2/defs.tex}
\item In a certain lottery 10,000 tickets are sold and ten equal prizes are awarded. What is the probability of not getting a prize if you buy (a) one ticket (b) two tickets (c) 10 tickets ?	
\\
\solution
		%\input{ncert/11/16/4/4/defs.tex}
		%
\item 
Out of 100 students, two sections of 40 and 60 are formed. If you and your friend are among the 100 students, what is the probability that
\begin{enumerate}
\item you both enter the same section?
\item you both enter the different sections?
\end{enumerate}
\solution
		%\input{ncert/11/16/4/5/defs.tex}
	\item 
The number lock of a suitcase has 4 wheels each labelled with ten digits i.e. from 0 to 9.The lock opens with a sequence of four digits with no repeats.What is the probability of a person getting the right sequence to open the suitcase.
\\
\solution
		%\input{ncert/11/16/4/10/defs.tex}
		%
\item 
Two cards are drawn at random and without replacement from a pack of 52 playing cards. Find the probability that both the cards are black.
\\
\solution
		%\input{ncert/12/13/2/2/defs.tex}
		\item A box of oranges is inspected by examining three randomly selected oranges drawn without replacement. If all the three oranges are good, the box is approved for sale, otherwise, it is rejected. Find the probability that a box containing 15 oranges out of which 12 are good and 3 are bad ones will be approved for sale.
		\label{ncert/12/13/2/3/defs.tex}
		\item Two balls are drawn at random with replacement from a box containing 10 black and 8 red balls. Find the probability that
		\label{ncert/12/13/2/12}
\begin{enumerate}
\item both balls are red.
\item first ball is black and second is red.
\item one of them is black and other is red.
\end{enumerate}

\item In a hostel, 60\% of the students read Hindi newspaper, 40\% read English newspaper and 20\% read both Hindi and English newspapers. A student is selected at random.
		\label{ncert/12/13/2/15}
\begin{enumerate}
\item Find the probability that she reads neither Hindi nor English newspapers.
\item If she reads Hindi newspaper, find the probability that she reads English newspaper.
\item If she reads English newspaper, find the probability that she reads Hindi newspaper.\\
\end{enumerate}
\item The probability of obtaining an even prime number on each die, when a pair of dice is rolled is 
\begin{enumerate}
    \item $0$ 
    
    \item $\frac{1}{3}$ 
    
    \item $\frac{1}{12}$ 
    
    \item $\frac{1}{36}$ 
\end{enumerate}
\solution
		%\input{ncert/12/13/2/17/defs.tex}
	\item A bag contains 4 red and 4 black balls, another bag contains 2 red and 6 black balls. One of the two bags is selected at random and a ball is drawn from the bag which is found to be red. Find the probability that the ball is drawn from the first bag.
\\
\solution
		%\input{ncert/12/13/3/2/main.tex}
  \item
  Cards with numbers 2 to 101 are placed in a box. A card is selected at random.Find the probability that the card has
\begin{enumerate}[label=(\roman*)]
	\item an even number 
	\item a square number
\end{enumerate}
\solution
%\input{exemplar/10/13/3/32/main.tex}
\item
The king, queen and jack of clubs are removed from a deck of 52 playing cards and then well shuffled. Now one card is drawn at random from the remaining cards.  Determine the probability that the card is
\begin{enumerate}[label=(\roman*)]
\item a club
\item 10 of hearts
\end{enumerate}
\solution
%\input{exemplar/10/13/3/29/main.tex}
\item A team of medical students doing their internship have to assist during surgeries
at a city hospital. The probabilities of surgeries rated as very complex, complex,
routine, simple or very simple are respectively, 0.15, 0.20, 0.31, 0.26, .08. Find
the probabilities that a particular surgery will be rated
\begin{enumerate}
	\item complex or very complex;
	\item neither very complex nor very simple;
	\item routine or complex
	\item routine or simple
\end{enumerate}
\solution
%\input{exemplar/11/16/3/8(1)/main.tex}
\item A card is selected from a pack of 52 cards.
\begin{enumerate}[label=(\alph*)]
    \item How many points are there in the sample space?
    \item Calculate the probability that the card is an ace of spades.
    \item Calculate the probability that the card is (i) an ace and (ii) black card.
\end{enumerate}
\solution
%\input{exemplar/11/16/3/4/main2.tex}
\item The probability that a non leap year selected at random will contain 53 sundays.
\\
\solution
%\input{exemplar/10/13/1/19/main.tex}
\item One of the four persons John, Rita, Aslam or Gurpreet will be promoted next
month. Consequently the sample space consists of four elementary outcomes
S = {John promoted, Rita promoted, Aslam promoted, Gurpreet promoted}
You are told that the chances of John’s promotion is same as that of Gurpreet,
Rita’s chances of promotion are twice as likely as Johns. Aslam’s chances are
four times that of John.
\begin{enumerate}
	\item Determine
	\begin{enumerate}
		\item P (John promoted)
		\item P (Rita promoted)
		\item P (Aslam promoted)
		\item P (Gurpreet promoted)
	\end{enumerate}
	\item If A = {John promoted or Gurpreet promoted}, find P (A).
\end{enumerate}
\solution
%\input{exemplar/11/16/3/10/main.tex}
\item A card is drawn from a deck of 52 cards. Find the probability of getting a king or a heart or a red card.\\
\solution
%\input{exemplar/11/16/3/15/main.tex}
\item The probability that a student will pass his examination is 0.73, the probability of
the student getting a compartment is 0.13, and the probability that the student will
either pass or get compartment is 0.96. State True or False.\\
\solution
%\input{exemplar/11/16/3/31/main.tex}
\item A card is selected from a pack of 52 cards\\
\begin{enumerate}[label=(\alph*)]
\item How many points are there in the sample space?
\item Calculate the probability that the cards is an ace of spades.
\item Calculate the probability that the card is (i) an ace (ii)black card.\\
\end{enumerate}
%\input{ncert/11/16/3/4_1/Prob_4.tex}
\item In a non-leap year, the probability of having 53 tuesdays or 53 wednesdays is\\
\solution
%\input{exemplar/11/16/3/18/main.tex}
\item There are 1000 sealed envelopes in a box, 10 of them contain a cash prize of
Rs 100 each, 100 of them contain a cash prize of Rs 50 each and 200 of them
contain a cash prize of Rs 10 each and rest do not contain any cash prize. If they
are well shuffled and an envelope is picked up out, what is the probability that it
contains no cash prize?\\
\solution
%\input{exemplar/10/13/3/34/main.tex}
\item 
A die is thrown and a card is selected at random from a deck of 52 playing cards. The probability of getting an even number on the die and a spade card.\\
\solution
%\input{exemplar/12/13/3/78/main.tex}
\item
If 4-digit numbers greater than 5,000 are randomly formed from the digits 0, 1, 3, 5, and 7, what is the probability of forming a number divisible by 5 when:
\begin{enumerate}
    \item The digits are repeated?
    \item The repetition of digits is not allowed?
\end{enumerate}
\solution
%\input{ncert/11/16/4/9/main.tex}
\item Consider the probability space $\brak{\Omega, \mathcal{G}, P}$ where $\Omega = [0,2]$ and $\mathcal{G} = \cbrak{\phi, \Omega, [0,1], (1,2]}$. Let $X$ and $Y$ be two functions on $\Omega$ defined as
\begin{align*}
    X(\omega) = 
    \begin{cases}
        1 & \text{if }\omega \in [0, 1]\\
        2 & \text{if }\omega \in (1, 2]
    \end{cases}
\end{align*}
and
\begin{align*}
    Y(\omega) = 
    \begin{cases}
        2 & \text{if }\omega \in [0, 1.5]\\
        3 & \text{if }\omega \in (1.5, 2].
    \end{cases}
\end{align*}
Then which one of the following statements is true?
\begin{enumerate}
    \item [(A)] $X$ is a random variable with respect to $\mathcal{G}$, but $Y$ is not a random variable with respect to $\mathcal{G}$.
    \item [(B)] $Y$ is a random variable with respect to $\mathcal{G}$, but $X$ is not a random variable with respect to $\mathcal{G}$.
    \item [(C)] Neither $X$ nor $Y$ is a random variable with respect to $\mathcal{G}$.
    \item [(D)] Both $X$ and $Y$ are random variables with respect to $\mathcal{G}$.
\end{enumerate} \hfill (GATE ST 2023)\\
\solution
%\input{gate/ST/2023/14/main.tex}
	\item  A die is loaded in such a way that each odd number is twice as likely to occur as
each even number. Find $P(G)$, where $G$ is the event that a number greater than
3 occurs on a single roll of the die.
\\
\solution
		%\input{exemplar/11/16/3/5/main.tex}
	\item All the jacks, queens and kings are removed from a deck of 52 playing cards. The remaining cards are well shuffled and then one card is drawn at random. Giving ace a value 1 similar value for other cards, find the probability that the card has a value 
		\begin{enumerate}
			\item 7
			\item greater than 7
			\item less than 7
		\end{enumerate}
		%\input{exemplar/10/13/3/30/main.tex}
  \item A Lot consists of 48 mobile phones of which 42 are good, 3 have only minor defects and 3 have major defects.Varnika will buy a phone if it is good but the trader will only buy a mobile if it has no major defects. One phone is selected at random from the lot. What is the probability that it is
\begin{enumerate}
	\item acceptable to Varnika?
            \item acceptable to the trader?
\end{enumerate}
\solution
	%\input{exemplar/10/13/3/40/main.tex}
 \item A student says that if you throw a die, it will show up 1 or not 1. Therefore, the probability of getting 1 and the probability of getting 'not 1' each is equal to $\frac{1}{2}$. Is this correct? Give reasons.\\
 \solution
        %\input{exemplar/10/13/2/9/main.tex}
   \item Four candidates A, B, C, D have ap-
plied for the assignment to coach a school cricket
team. If A is twice as likely to be selected as B, and
B and C are given about the same chance of being
selected, while C is twice as likely to be selected
as D, what are the probabilities that
\begin{enumerate}
\item C will be selected?
\item A will not be selected?
\end{enumerate}
	%\input{exemplar/11/16/3/9/main.tex}
 \item A bag contain 24 balls of which $x$ balls are red, $2x$ are white and $3x$ are blue. A ball is selected at random, What is the probability that it is
\begin{enumerate}[label=\alph*)]
\item not red ?
\item white ?
\end{enumerate}
%\input{exemplar/10/13/3/41/main.tex}
If the letters of the word ASSASSINATION are arranged at random. Find the Probability that
\begin{enumerate}[label=(\alph*)]
\item Four $S's$ come consecutively in the word
\item Two  $I's$ and two $N's$ come together
\item All $A's$ are not coming together
\item No two $A's$ are coming together
\end{enumerate}
%\input{exemplar/11/16/3/14/main.tex}
	\item One urn contains two black balls (labelled B1 and B2) and one white ball. A
	second urn contains one black ball and two white balls (labelled W1 and W2).
	Suppose the following experiment is performed. One of the two urns is chosen
	at random. Next a ball is randomly chosen from the urn. Then a second ball is
	chosen at random from the same urn without replacing the first ball.
	
	\begin{enumerate}
	\item What is the probability that two black balls are chosen?
	
	\item What is the probability that two balls of opposite colour are chosen?
	\end{enumerate}
	\solution
	%\input{exemplar/11/16/3/12/main1.tex}
\end{enumerate}

	\item A card is selected from a pack of 52 cards.
 \begin{enumerate}[label=(\alph*)] 
                 \item How many points are there in the sample space?
                 \item Calculate the probability that the card is an ace of spades.
                 \item Calculate the probability that the card is (i) an ace and (ii) black card.
 \end{enumerate}
\solution
		%\begin{table}[H]
	\centering
\begin{tabular}{|c|c|c|}
\hline
Random variable &Value &Definition\\ \hline
\multirow{3}{*}{X} &0 &Slips of Rs 1\\
&1 &Slips of Rs 5\\
&2 &Slips of Rs 13\\ \hline
\multirow{2}{*}{Y} &0 &Box A\\
&1 &Box B\\\hline
\end{tabular}
\caption{}
\label{tab:Distribution}
\end{table}
See \tabref{tab:Distribution}.
\begin{align}
p_{Y}\brak{k}= \begin{cases} 
      \frac{1}{3} & {k=0} \\
      \frac{2}{3 }& {k=1} 
   \end{cases}
   \\
p_{Y|X}\brak{0|0} = \frac{19}{25}\, 
p_{Y|X}\brak{0|1} = \frac{6}{25}\,
p_{Y|X}\brak{1|0} = \frac{45}{50}\,
p_{Y|X}\brak{1|2} = \frac{5}{50}
\end{align}
The desired probability is the probability that a slip drawn at random is marked other than Rs 1,
\begin{align}
&=1-p_X\brak{0}\\
&= p_X(1) + p_X(2)
\end{align}
Using Bayes theorem,
\begin{align}
&= p_Y\brak{0} \times \pr{Y=0 | X=1} + p_Y\brak{1} \times \pr{Y=1|X=2}\\
&=\frac{1}{3} \times \frac{6}{25} + \frac{2}{3} \times \frac{5}{50}\\
&=\frac{11}{75}
\end{align}

\newpage

%\tableofcontents

\bigskip

\renewcommand{\thefigure}{\theenumi}
\renewcommand{\thetable}{\theenumi}
%\renewcommand{\theequation}{\theenumi}

%\begin{abstract}
%%\boldmath
%In this letter, an algorithm for evaluating the exact analytical bit error rate  (BER)  for the piecewise linear (PL) combiner for  multiple relays is presented. Previous results were available only for upto three relays. The algorithm is unique in the sense that  the actual mathematical expressions, that are prohibitively large, need not be explicitly obtained. The diversity gain due to multiple relays is shown through plots of the analytical BER, well supported by simulations. 
%
%\end{abstract}
% IEEEtran.cls defaults to using nonbold math in the Abstract.
% This preserves the distinction between vectors and scalars. However,
% if the journal you are submitting to favors bold math in the abstract,
% then you can use LaTeX's standard command \boldmath at the very start
% of the abstract to achieve this. Many IEEE journals frown on math
% in the abstract anyway.

% Note that keywords are not normally used for peerreview papers.
%\begin{IEEEkeywords}
%Cooperative diversity, decode and forward, piecewise linear
%\end{IEEEkeywords}



% For peer review papers, you can put extra information on the cover
% page as needed:
% \ifCLASSOPTIONpeerreview
% \begin{center} \bfseries EDICS Category: 3-BBND \end{center}
% \fi
%
% For peerreview papers, this IEEEtran command inserts a page break and
% creates the second title. It will be ignored for other modes.
%\IEEEpeerreviewmaketitle




\item Four cards are drawn from a well-shuffled deck of 52 cards. What is the probability of obtaining 3 diamonds and one spade.
\\
\solution
		%\begin{enumerate}[label=\thesection.\arabic*,ref=\thesection.\theenumi]
	\item One card is drawn from a well-shuffled deck of 52 cards. Find the probability of getting
\begin{enumerate}
\item A king of red colour 
\item A face card 
\item A red face card
\item The jack of hearts
\item A spade
\item The queen of diamonds

\end{enumerate}
\solution
		%\input{ncert/10/15/1/14/main.tex}
	\item Five cards—the ten, jack, queen, king and ace of diamonds, are well-shuffled with their face downwards. One card is then picked up at random.
\begin{enumerate}
\item
What is the probability that the card is the queen? 
\item
If the queen is drawn and put aside, what is the probability that the second card picked up is (a) an ace? (b) a queen?\\
\end{enumerate}
\solution
		%\input{ncert/10/15/1/15/defs.tex}
	\item A bag contains $5$ red balls and some blue balls. If the probability of drawing a blue ball is double that if a red ball, determine the number of blue balls in the bag. 
		\\
\solution
		%\input{ncert/10/15/2/3/defs.tex}
	\item A card is selected from a pack of 52 cards.
 \begin{enumerate}[label=(\alph*)] 
                 \item How many points are there in the sample space?
                 \item Calculate the probability that the card is an ace of spades.
                 \item Calculate the probability that the card is (i) an ace and (ii) black card.
 \end{enumerate}
\solution
		%\input{ncert/11/16/3/4/main.tex}
\item Four cards are drawn from a well-shuffled deck of 52 cards. What is the probability of obtaining 3 diamonds and one spade.
\\
\solution
		%\input{ncert/11/16/4/2/defs.tex}
\item In a certain lottery 10,000 tickets are sold and ten equal prizes are awarded. What is the probability of not getting a prize if you buy (a) one ticket (b) two tickets (c) 10 tickets ?	
\\
\solution
		%\input{ncert/11/16/4/4/defs.tex}
		%
\item 
Out of 100 students, two sections of 40 and 60 are formed. If you and your friend are among the 100 students, what is the probability that
\begin{enumerate}
\item you both enter the same section?
\item you both enter the different sections?
\end{enumerate}
\solution
		%\input{ncert/11/16/4/5/defs.tex}
	\item 
The number lock of a suitcase has 4 wheels each labelled with ten digits i.e. from 0 to 9.The lock opens with a sequence of four digits with no repeats.What is the probability of a person getting the right sequence to open the suitcase.
\\
\solution
		%\input{ncert/11/16/4/10/defs.tex}
		%
\item 
Two cards are drawn at random and without replacement from a pack of 52 playing cards. Find the probability that both the cards are black.
\\
\solution
		%\input{ncert/12/13/2/2/defs.tex}
		\item A box of oranges is inspected by examining three randomly selected oranges drawn without replacement. If all the three oranges are good, the box is approved for sale, otherwise, it is rejected. Find the probability that a box containing 15 oranges out of which 12 are good and 3 are bad ones will be approved for sale.
		\label{ncert/12/13/2/3/defs.tex}
		\item Two balls are drawn at random with replacement from a box containing 10 black and 8 red balls. Find the probability that
		\label{ncert/12/13/2/12}
\begin{enumerate}
\item both balls are red.
\item first ball is black and second is red.
\item one of them is black and other is red.
\end{enumerate}

\item In a hostel, 60\% of the students read Hindi newspaper, 40\% read English newspaper and 20\% read both Hindi and English newspapers. A student is selected at random.
		\label{ncert/12/13/2/15}
\begin{enumerate}
\item Find the probability that she reads neither Hindi nor English newspapers.
\item If she reads Hindi newspaper, find the probability that she reads English newspaper.
\item If she reads English newspaper, find the probability that she reads Hindi newspaper.\\
\end{enumerate}
\item The probability of obtaining an even prime number on each die, when a pair of dice is rolled is 
\begin{enumerate}
    \item $0$ 
    
    \item $\frac{1}{3}$ 
    
    \item $\frac{1}{12}$ 
    
    \item $\frac{1}{36}$ 
\end{enumerate}
\solution
		%\input{ncert/12/13/2/17/defs.tex}
	\item A bag contains 4 red and 4 black balls, another bag contains 2 red and 6 black balls. One of the two bags is selected at random and a ball is drawn from the bag which is found to be red. Find the probability that the ball is drawn from the first bag.
\\
\solution
		%\input{ncert/12/13/3/2/main.tex}
  \item
  Cards with numbers 2 to 101 are placed in a box. A card is selected at random.Find the probability that the card has
\begin{enumerate}[label=(\roman*)]
	\item an even number 
	\item a square number
\end{enumerate}
\solution
%\input{exemplar/10/13/3/32/main.tex}
\item
The king, queen and jack of clubs are removed from a deck of 52 playing cards and then well shuffled. Now one card is drawn at random from the remaining cards.  Determine the probability that the card is
\begin{enumerate}[label=(\roman*)]
\item a club
\item 10 of hearts
\end{enumerate}
\solution
%\input{exemplar/10/13/3/29/main.tex}
\item A team of medical students doing their internship have to assist during surgeries
at a city hospital. The probabilities of surgeries rated as very complex, complex,
routine, simple or very simple are respectively, 0.15, 0.20, 0.31, 0.26, .08. Find
the probabilities that a particular surgery will be rated
\begin{enumerate}
	\item complex or very complex;
	\item neither very complex nor very simple;
	\item routine or complex
	\item routine or simple
\end{enumerate}
\solution
%\input{exemplar/11/16/3/8(1)/main.tex}
\item A card is selected from a pack of 52 cards.
\begin{enumerate}[label=(\alph*)]
    \item How many points are there in the sample space?
    \item Calculate the probability that the card is an ace of spades.
    \item Calculate the probability that the card is (i) an ace and (ii) black card.
\end{enumerate}
\solution
%\input{exemplar/11/16/3/4/main2.tex}
\item The probability that a non leap year selected at random will contain 53 sundays.
\\
\solution
%\input{exemplar/10/13/1/19/main.tex}
\item One of the four persons John, Rita, Aslam or Gurpreet will be promoted next
month. Consequently the sample space consists of four elementary outcomes
S = {John promoted, Rita promoted, Aslam promoted, Gurpreet promoted}
You are told that the chances of John’s promotion is same as that of Gurpreet,
Rita’s chances of promotion are twice as likely as Johns. Aslam’s chances are
four times that of John.
\begin{enumerate}
	\item Determine
	\begin{enumerate}
		\item P (John promoted)
		\item P (Rita promoted)
		\item P (Aslam promoted)
		\item P (Gurpreet promoted)
	\end{enumerate}
	\item If A = {John promoted or Gurpreet promoted}, find P (A).
\end{enumerate}
\solution
%\input{exemplar/11/16/3/10/main.tex}
\item A card is drawn from a deck of 52 cards. Find the probability of getting a king or a heart or a red card.\\
\solution
%\input{exemplar/11/16/3/15/main.tex}
\item The probability that a student will pass his examination is 0.73, the probability of
the student getting a compartment is 0.13, and the probability that the student will
either pass or get compartment is 0.96. State True or False.\\
\solution
%\input{exemplar/11/16/3/31/main.tex}
\item A card is selected from a pack of 52 cards\\
\begin{enumerate}[label=(\alph*)]
\item How many points are there in the sample space?
\item Calculate the probability that the cards is an ace of spades.
\item Calculate the probability that the card is (i) an ace (ii)black card.\\
\end{enumerate}
%\input{ncert/11/16/3/4_1/Prob_4.tex}
\item In a non-leap year, the probability of having 53 tuesdays or 53 wednesdays is\\
\solution
%\input{exemplar/11/16/3/18/main.tex}
\item There are 1000 sealed envelopes in a box, 10 of them contain a cash prize of
Rs 100 each, 100 of them contain a cash prize of Rs 50 each and 200 of them
contain a cash prize of Rs 10 each and rest do not contain any cash prize. If they
are well shuffled and an envelope is picked up out, what is the probability that it
contains no cash prize?\\
\solution
%\input{exemplar/10/13/3/34/main.tex}
\item 
A die is thrown and a card is selected at random from a deck of 52 playing cards. The probability of getting an even number on the die and a spade card.\\
\solution
%\input{exemplar/12/13/3/78/main.tex}
\item
If 4-digit numbers greater than 5,000 are randomly formed from the digits 0, 1, 3, 5, and 7, what is the probability of forming a number divisible by 5 when:
\begin{enumerate}
    \item The digits are repeated?
    \item The repetition of digits is not allowed?
\end{enumerate}
\solution
%\input{ncert/11/16/4/9/main.tex}
\item Consider the probability space $\brak{\Omega, \mathcal{G}, P}$ where $\Omega = [0,2]$ and $\mathcal{G} = \cbrak{\phi, \Omega, [0,1], (1,2]}$. Let $X$ and $Y$ be two functions on $\Omega$ defined as
\begin{align*}
    X(\omega) = 
    \begin{cases}
        1 & \text{if }\omega \in [0, 1]\\
        2 & \text{if }\omega \in (1, 2]
    \end{cases}
\end{align*}
and
\begin{align*}
    Y(\omega) = 
    \begin{cases}
        2 & \text{if }\omega \in [0, 1.5]\\
        3 & \text{if }\omega \in (1.5, 2].
    \end{cases}
\end{align*}
Then which one of the following statements is true?
\begin{enumerate}
    \item [(A)] $X$ is a random variable with respect to $\mathcal{G}$, but $Y$ is not a random variable with respect to $\mathcal{G}$.
    \item [(B)] $Y$ is a random variable with respect to $\mathcal{G}$, but $X$ is not a random variable with respect to $\mathcal{G}$.
    \item [(C)] Neither $X$ nor $Y$ is a random variable with respect to $\mathcal{G}$.
    \item [(D)] Both $X$ and $Y$ are random variables with respect to $\mathcal{G}$.
\end{enumerate} \hfill (GATE ST 2023)\\
\solution
%\input{gate/ST/2023/14/main.tex}
	\item  A die is loaded in such a way that each odd number is twice as likely to occur as
each even number. Find $P(G)$, where $G$ is the event that a number greater than
3 occurs on a single roll of the die.
\\
\solution
		%\input{exemplar/11/16/3/5/main.tex}
	\item All the jacks, queens and kings are removed from a deck of 52 playing cards. The remaining cards are well shuffled and then one card is drawn at random. Giving ace a value 1 similar value for other cards, find the probability that the card has a value 
		\begin{enumerate}
			\item 7
			\item greater than 7
			\item less than 7
		\end{enumerate}
		%\input{exemplar/10/13/3/30/main.tex}
  \item A Lot consists of 48 mobile phones of which 42 are good, 3 have only minor defects and 3 have major defects.Varnika will buy a phone if it is good but the trader will only buy a mobile if it has no major defects. One phone is selected at random from the lot. What is the probability that it is
\begin{enumerate}
	\item acceptable to Varnika?
            \item acceptable to the trader?
\end{enumerate}
\solution
	%\input{exemplar/10/13/3/40/main.tex}
 \item A student says that if you throw a die, it will show up 1 or not 1. Therefore, the probability of getting 1 and the probability of getting 'not 1' each is equal to $\frac{1}{2}$. Is this correct? Give reasons.\\
 \solution
        %\input{exemplar/10/13/2/9/main.tex}
   \item Four candidates A, B, C, D have ap-
plied for the assignment to coach a school cricket
team. If A is twice as likely to be selected as B, and
B and C are given about the same chance of being
selected, while C is twice as likely to be selected
as D, what are the probabilities that
\begin{enumerate}
\item C will be selected?
\item A will not be selected?
\end{enumerate}
	%\input{exemplar/11/16/3/9/main.tex}
 \item A bag contain 24 balls of which $x$ balls are red, $2x$ are white and $3x$ are blue. A ball is selected at random, What is the probability that it is
\begin{enumerate}[label=\alph*)]
\item not red ?
\item white ?
\end{enumerate}
%\input{exemplar/10/13/3/41/main.tex}
If the letters of the word ASSASSINATION are arranged at random. Find the Probability that
\begin{enumerate}[label=(\alph*)]
\item Four $S's$ come consecutively in the word
\item Two  $I's$ and two $N's$ come together
\item All $A's$ are not coming together
\item No two $A's$ are coming together
\end{enumerate}
%\input{exemplar/11/16/3/14/main.tex}
	\item One urn contains two black balls (labelled B1 and B2) and one white ball. A
	second urn contains one black ball and two white balls (labelled W1 and W2).
	Suppose the following experiment is performed. One of the two urns is chosen
	at random. Next a ball is randomly chosen from the urn. Then a second ball is
	chosen at random from the same urn without replacing the first ball.
	
	\begin{enumerate}
	\item What is the probability that two black balls are chosen?
	
	\item What is the probability that two balls of opposite colour are chosen?
	\end{enumerate}
	\solution
	%\input{exemplar/11/16/3/12/main1.tex}
\end{enumerate}

\item In a certain lottery 10,000 tickets are sold and ten equal prizes are awarded. What is the probability of not getting a prize if you buy (a) one ticket (b) two tickets (c) 10 tickets ?	
\\
\solution
		%\begin{enumerate}[label=\thesection.\arabic*,ref=\thesection.\theenumi]
	\item One card is drawn from a well-shuffled deck of 52 cards. Find the probability of getting
\begin{enumerate}
\item A king of red colour 
\item A face card 
\item A red face card
\item The jack of hearts
\item A spade
\item The queen of diamonds

\end{enumerate}
\solution
		%\input{ncert/10/15/1/14/main.tex}
	\item Five cards—the ten, jack, queen, king and ace of diamonds, are well-shuffled with their face downwards. One card is then picked up at random.
\begin{enumerate}
\item
What is the probability that the card is the queen? 
\item
If the queen is drawn and put aside, what is the probability that the second card picked up is (a) an ace? (b) a queen?\\
\end{enumerate}
\solution
		%\input{ncert/10/15/1/15/defs.tex}
	\item A bag contains $5$ red balls and some blue balls. If the probability of drawing a blue ball is double that if a red ball, determine the number of blue balls in the bag. 
		\\
\solution
		%\input{ncert/10/15/2/3/defs.tex}
	\item A card is selected from a pack of 52 cards.
 \begin{enumerate}[label=(\alph*)] 
                 \item How many points are there in the sample space?
                 \item Calculate the probability that the card is an ace of spades.
                 \item Calculate the probability that the card is (i) an ace and (ii) black card.
 \end{enumerate}
\solution
		%\input{ncert/11/16/3/4/main.tex}
\item Four cards are drawn from a well-shuffled deck of 52 cards. What is the probability of obtaining 3 diamonds and one spade.
\\
\solution
		%\input{ncert/11/16/4/2/defs.tex}
\item In a certain lottery 10,000 tickets are sold and ten equal prizes are awarded. What is the probability of not getting a prize if you buy (a) one ticket (b) two tickets (c) 10 tickets ?	
\\
\solution
		%\input{ncert/11/16/4/4/defs.tex}
		%
\item 
Out of 100 students, two sections of 40 and 60 are formed. If you and your friend are among the 100 students, what is the probability that
\begin{enumerate}
\item you both enter the same section?
\item you both enter the different sections?
\end{enumerate}
\solution
		%\input{ncert/11/16/4/5/defs.tex}
	\item 
The number lock of a suitcase has 4 wheels each labelled with ten digits i.e. from 0 to 9.The lock opens with a sequence of four digits with no repeats.What is the probability of a person getting the right sequence to open the suitcase.
\\
\solution
		%\input{ncert/11/16/4/10/defs.tex}
		%
\item 
Two cards are drawn at random and without replacement from a pack of 52 playing cards. Find the probability that both the cards are black.
\\
\solution
		%\input{ncert/12/13/2/2/defs.tex}
		\item A box of oranges is inspected by examining three randomly selected oranges drawn without replacement. If all the three oranges are good, the box is approved for sale, otherwise, it is rejected. Find the probability that a box containing 15 oranges out of which 12 are good and 3 are bad ones will be approved for sale.
		\label{ncert/12/13/2/3/defs.tex}
		\item Two balls are drawn at random with replacement from a box containing 10 black and 8 red balls. Find the probability that
		\label{ncert/12/13/2/12}
\begin{enumerate}
\item both balls are red.
\item first ball is black and second is red.
\item one of them is black and other is red.
\end{enumerate}

\item In a hostel, 60\% of the students read Hindi newspaper, 40\% read English newspaper and 20\% read both Hindi and English newspapers. A student is selected at random.
		\label{ncert/12/13/2/15}
\begin{enumerate}
\item Find the probability that she reads neither Hindi nor English newspapers.
\item If she reads Hindi newspaper, find the probability that she reads English newspaper.
\item If she reads English newspaper, find the probability that she reads Hindi newspaper.\\
\end{enumerate}
\item The probability of obtaining an even prime number on each die, when a pair of dice is rolled is 
\begin{enumerate}
    \item $0$ 
    
    \item $\frac{1}{3}$ 
    
    \item $\frac{1}{12}$ 
    
    \item $\frac{1}{36}$ 
\end{enumerate}
\solution
		%\input{ncert/12/13/2/17/defs.tex}
	\item A bag contains 4 red and 4 black balls, another bag contains 2 red and 6 black balls. One of the two bags is selected at random and a ball is drawn from the bag which is found to be red. Find the probability that the ball is drawn from the first bag.
\\
\solution
		%\input{ncert/12/13/3/2/main.tex}
  \item
  Cards with numbers 2 to 101 are placed in a box. A card is selected at random.Find the probability that the card has
\begin{enumerate}[label=(\roman*)]
	\item an even number 
	\item a square number
\end{enumerate}
\solution
%\input{exemplar/10/13/3/32/main.tex}
\item
The king, queen and jack of clubs are removed from a deck of 52 playing cards and then well shuffled. Now one card is drawn at random from the remaining cards.  Determine the probability that the card is
\begin{enumerate}[label=(\roman*)]
\item a club
\item 10 of hearts
\end{enumerate}
\solution
%\input{exemplar/10/13/3/29/main.tex}
\item A team of medical students doing their internship have to assist during surgeries
at a city hospital. The probabilities of surgeries rated as very complex, complex,
routine, simple or very simple are respectively, 0.15, 0.20, 0.31, 0.26, .08. Find
the probabilities that a particular surgery will be rated
\begin{enumerate}
	\item complex or very complex;
	\item neither very complex nor very simple;
	\item routine or complex
	\item routine or simple
\end{enumerate}
\solution
%\input{exemplar/11/16/3/8(1)/main.tex}
\item A card is selected from a pack of 52 cards.
\begin{enumerate}[label=(\alph*)]
    \item How many points are there in the sample space?
    \item Calculate the probability that the card is an ace of spades.
    \item Calculate the probability that the card is (i) an ace and (ii) black card.
\end{enumerate}
\solution
%\input{exemplar/11/16/3/4/main2.tex}
\item The probability that a non leap year selected at random will contain 53 sundays.
\\
\solution
%\input{exemplar/10/13/1/19/main.tex}
\item One of the four persons John, Rita, Aslam or Gurpreet will be promoted next
month. Consequently the sample space consists of four elementary outcomes
S = {John promoted, Rita promoted, Aslam promoted, Gurpreet promoted}
You are told that the chances of John’s promotion is same as that of Gurpreet,
Rita’s chances of promotion are twice as likely as Johns. Aslam’s chances are
four times that of John.
\begin{enumerate}
	\item Determine
	\begin{enumerate}
		\item P (John promoted)
		\item P (Rita promoted)
		\item P (Aslam promoted)
		\item P (Gurpreet promoted)
	\end{enumerate}
	\item If A = {John promoted or Gurpreet promoted}, find P (A).
\end{enumerate}
\solution
%\input{exemplar/11/16/3/10/main.tex}
\item A card is drawn from a deck of 52 cards. Find the probability of getting a king or a heart or a red card.\\
\solution
%\input{exemplar/11/16/3/15/main.tex}
\item The probability that a student will pass his examination is 0.73, the probability of
the student getting a compartment is 0.13, and the probability that the student will
either pass or get compartment is 0.96. State True or False.\\
\solution
%\input{exemplar/11/16/3/31/main.tex}
\item A card is selected from a pack of 52 cards\\
\begin{enumerate}[label=(\alph*)]
\item How many points are there in the sample space?
\item Calculate the probability that the cards is an ace of spades.
\item Calculate the probability that the card is (i) an ace (ii)black card.\\
\end{enumerate}
%\input{ncert/11/16/3/4_1/Prob_4.tex}
\item In a non-leap year, the probability of having 53 tuesdays or 53 wednesdays is\\
\solution
%\input{exemplar/11/16/3/18/main.tex}
\item There are 1000 sealed envelopes in a box, 10 of them contain a cash prize of
Rs 100 each, 100 of them contain a cash prize of Rs 50 each and 200 of them
contain a cash prize of Rs 10 each and rest do not contain any cash prize. If they
are well shuffled and an envelope is picked up out, what is the probability that it
contains no cash prize?\\
\solution
%\input{exemplar/10/13/3/34/main.tex}
\item 
A die is thrown and a card is selected at random from a deck of 52 playing cards. The probability of getting an even number on the die and a spade card.\\
\solution
%\input{exemplar/12/13/3/78/main.tex}
\item
If 4-digit numbers greater than 5,000 are randomly formed from the digits 0, 1, 3, 5, and 7, what is the probability of forming a number divisible by 5 when:
\begin{enumerate}
    \item The digits are repeated?
    \item The repetition of digits is not allowed?
\end{enumerate}
\solution
%\input{ncert/11/16/4/9/main.tex}
\item Consider the probability space $\brak{\Omega, \mathcal{G}, P}$ where $\Omega = [0,2]$ and $\mathcal{G} = \cbrak{\phi, \Omega, [0,1], (1,2]}$. Let $X$ and $Y$ be two functions on $\Omega$ defined as
\begin{align*}
    X(\omega) = 
    \begin{cases}
        1 & \text{if }\omega \in [0, 1]\\
        2 & \text{if }\omega \in (1, 2]
    \end{cases}
\end{align*}
and
\begin{align*}
    Y(\omega) = 
    \begin{cases}
        2 & \text{if }\omega \in [0, 1.5]\\
        3 & \text{if }\omega \in (1.5, 2].
    \end{cases}
\end{align*}
Then which one of the following statements is true?
\begin{enumerate}
    \item [(A)] $X$ is a random variable with respect to $\mathcal{G}$, but $Y$ is not a random variable with respect to $\mathcal{G}$.
    \item [(B)] $Y$ is a random variable with respect to $\mathcal{G}$, but $X$ is not a random variable with respect to $\mathcal{G}$.
    \item [(C)] Neither $X$ nor $Y$ is a random variable with respect to $\mathcal{G}$.
    \item [(D)] Both $X$ and $Y$ are random variables with respect to $\mathcal{G}$.
\end{enumerate} \hfill (GATE ST 2023)\\
\solution
%\input{gate/ST/2023/14/main.tex}
	\item  A die is loaded in such a way that each odd number is twice as likely to occur as
each even number. Find $P(G)$, where $G$ is the event that a number greater than
3 occurs on a single roll of the die.
\\
\solution
		%\input{exemplar/11/16/3/5/main.tex}
	\item All the jacks, queens and kings are removed from a deck of 52 playing cards. The remaining cards are well shuffled and then one card is drawn at random. Giving ace a value 1 similar value for other cards, find the probability that the card has a value 
		\begin{enumerate}
			\item 7
			\item greater than 7
			\item less than 7
		\end{enumerate}
		%\input{exemplar/10/13/3/30/main.tex}
  \item A Lot consists of 48 mobile phones of which 42 are good, 3 have only minor defects and 3 have major defects.Varnika will buy a phone if it is good but the trader will only buy a mobile if it has no major defects. One phone is selected at random from the lot. What is the probability that it is
\begin{enumerate}
	\item acceptable to Varnika?
            \item acceptable to the trader?
\end{enumerate}
\solution
	%\input{exemplar/10/13/3/40/main.tex}
 \item A student says that if you throw a die, it will show up 1 or not 1. Therefore, the probability of getting 1 and the probability of getting 'not 1' each is equal to $\frac{1}{2}$. Is this correct? Give reasons.\\
 \solution
        %\input{exemplar/10/13/2/9/main.tex}
   \item Four candidates A, B, C, D have ap-
plied for the assignment to coach a school cricket
team. If A is twice as likely to be selected as B, and
B and C are given about the same chance of being
selected, while C is twice as likely to be selected
as D, what are the probabilities that
\begin{enumerate}
\item C will be selected?
\item A will not be selected?
\end{enumerate}
	%\input{exemplar/11/16/3/9/main.tex}
 \item A bag contain 24 balls of which $x$ balls are red, $2x$ are white and $3x$ are blue. A ball is selected at random, What is the probability that it is
\begin{enumerate}[label=\alph*)]
\item not red ?
\item white ?
\end{enumerate}
%\input{exemplar/10/13/3/41/main.tex}
If the letters of the word ASSASSINATION are arranged at random. Find the Probability that
\begin{enumerate}[label=(\alph*)]
\item Four $S's$ come consecutively in the word
\item Two  $I's$ and two $N's$ come together
\item All $A's$ are not coming together
\item No two $A's$ are coming together
\end{enumerate}
%\input{exemplar/11/16/3/14/main.tex}
	\item One urn contains two black balls (labelled B1 and B2) and one white ball. A
	second urn contains one black ball and two white balls (labelled W1 and W2).
	Suppose the following experiment is performed. One of the two urns is chosen
	at random. Next a ball is randomly chosen from the urn. Then a second ball is
	chosen at random from the same urn without replacing the first ball.
	
	\begin{enumerate}
	\item What is the probability that two black balls are chosen?
	
	\item What is the probability that two balls of opposite colour are chosen?
	\end{enumerate}
	\solution
	%\input{exemplar/11/16/3/12/main1.tex}
\end{enumerate}

		%
\item 
Out of 100 students, two sections of 40 and 60 are formed. If you and your friend are among the 100 students, what is the probability that
\begin{enumerate}
\item you both enter the same section?
\item you both enter the different sections?
\end{enumerate}
\solution
		%\begin{enumerate}[label=\thesection.\arabic*,ref=\thesection.\theenumi]
	\item One card is drawn from a well-shuffled deck of 52 cards. Find the probability of getting
\begin{enumerate}
\item A king of red colour 
\item A face card 
\item A red face card
\item The jack of hearts
\item A spade
\item The queen of diamonds

\end{enumerate}
\solution
		%\input{ncert/10/15/1/14/main.tex}
	\item Five cards—the ten, jack, queen, king and ace of diamonds, are well-shuffled with their face downwards. One card is then picked up at random.
\begin{enumerate}
\item
What is the probability that the card is the queen? 
\item
If the queen is drawn and put aside, what is the probability that the second card picked up is (a) an ace? (b) a queen?\\
\end{enumerate}
\solution
		%\input{ncert/10/15/1/15/defs.tex}
	\item A bag contains $5$ red balls and some blue balls. If the probability of drawing a blue ball is double that if a red ball, determine the number of blue balls in the bag. 
		\\
\solution
		%\input{ncert/10/15/2/3/defs.tex}
	\item A card is selected from a pack of 52 cards.
 \begin{enumerate}[label=(\alph*)] 
                 \item How many points are there in the sample space?
                 \item Calculate the probability that the card is an ace of spades.
                 \item Calculate the probability that the card is (i) an ace and (ii) black card.
 \end{enumerate}
\solution
		%\input{ncert/11/16/3/4/main.tex}
\item Four cards are drawn from a well-shuffled deck of 52 cards. What is the probability of obtaining 3 diamonds and one spade.
\\
\solution
		%\input{ncert/11/16/4/2/defs.tex}
\item In a certain lottery 10,000 tickets are sold and ten equal prizes are awarded. What is the probability of not getting a prize if you buy (a) one ticket (b) two tickets (c) 10 tickets ?	
\\
\solution
		%\input{ncert/11/16/4/4/defs.tex}
		%
\item 
Out of 100 students, two sections of 40 and 60 are formed. If you and your friend are among the 100 students, what is the probability that
\begin{enumerate}
\item you both enter the same section?
\item you both enter the different sections?
\end{enumerate}
\solution
		%\input{ncert/11/16/4/5/defs.tex}
	\item 
The number lock of a suitcase has 4 wheels each labelled with ten digits i.e. from 0 to 9.The lock opens with a sequence of four digits with no repeats.What is the probability of a person getting the right sequence to open the suitcase.
\\
\solution
		%\input{ncert/11/16/4/10/defs.tex}
		%
\item 
Two cards are drawn at random and without replacement from a pack of 52 playing cards. Find the probability that both the cards are black.
\\
\solution
		%\input{ncert/12/13/2/2/defs.tex}
		\item A box of oranges is inspected by examining three randomly selected oranges drawn without replacement. If all the three oranges are good, the box is approved for sale, otherwise, it is rejected. Find the probability that a box containing 15 oranges out of which 12 are good and 3 are bad ones will be approved for sale.
		\label{ncert/12/13/2/3/defs.tex}
		\item Two balls are drawn at random with replacement from a box containing 10 black and 8 red balls. Find the probability that
		\label{ncert/12/13/2/12}
\begin{enumerate}
\item both balls are red.
\item first ball is black and second is red.
\item one of them is black and other is red.
\end{enumerate}

\item In a hostel, 60\% of the students read Hindi newspaper, 40\% read English newspaper and 20\% read both Hindi and English newspapers. A student is selected at random.
		\label{ncert/12/13/2/15}
\begin{enumerate}
\item Find the probability that she reads neither Hindi nor English newspapers.
\item If she reads Hindi newspaper, find the probability that she reads English newspaper.
\item If she reads English newspaper, find the probability that she reads Hindi newspaper.\\
\end{enumerate}
\item The probability of obtaining an even prime number on each die, when a pair of dice is rolled is 
\begin{enumerate}
    \item $0$ 
    
    \item $\frac{1}{3}$ 
    
    \item $\frac{1}{12}$ 
    
    \item $\frac{1}{36}$ 
\end{enumerate}
\solution
		%\input{ncert/12/13/2/17/defs.tex}
	\item A bag contains 4 red and 4 black balls, another bag contains 2 red and 6 black balls. One of the two bags is selected at random and a ball is drawn from the bag which is found to be red. Find the probability that the ball is drawn from the first bag.
\\
\solution
		%\input{ncert/12/13/3/2/main.tex}
  \item
  Cards with numbers 2 to 101 are placed in a box. A card is selected at random.Find the probability that the card has
\begin{enumerate}[label=(\roman*)]
	\item an even number 
	\item a square number
\end{enumerate}
\solution
%\input{exemplar/10/13/3/32/main.tex}
\item
The king, queen and jack of clubs are removed from a deck of 52 playing cards and then well shuffled. Now one card is drawn at random from the remaining cards.  Determine the probability that the card is
\begin{enumerate}[label=(\roman*)]
\item a club
\item 10 of hearts
\end{enumerate}
\solution
%\input{exemplar/10/13/3/29/main.tex}
\item A team of medical students doing their internship have to assist during surgeries
at a city hospital. The probabilities of surgeries rated as very complex, complex,
routine, simple or very simple are respectively, 0.15, 0.20, 0.31, 0.26, .08. Find
the probabilities that a particular surgery will be rated
\begin{enumerate}
	\item complex or very complex;
	\item neither very complex nor very simple;
	\item routine or complex
	\item routine or simple
\end{enumerate}
\solution
%\input{exemplar/11/16/3/8(1)/main.tex}
\item A card is selected from a pack of 52 cards.
\begin{enumerate}[label=(\alph*)]
    \item How many points are there in the sample space?
    \item Calculate the probability that the card is an ace of spades.
    \item Calculate the probability that the card is (i) an ace and (ii) black card.
\end{enumerate}
\solution
%\input{exemplar/11/16/3/4/main2.tex}
\item The probability that a non leap year selected at random will contain 53 sundays.
\\
\solution
%\input{exemplar/10/13/1/19/main.tex}
\item One of the four persons John, Rita, Aslam or Gurpreet will be promoted next
month. Consequently the sample space consists of four elementary outcomes
S = {John promoted, Rita promoted, Aslam promoted, Gurpreet promoted}
You are told that the chances of John’s promotion is same as that of Gurpreet,
Rita’s chances of promotion are twice as likely as Johns. Aslam’s chances are
four times that of John.
\begin{enumerate}
	\item Determine
	\begin{enumerate}
		\item P (John promoted)
		\item P (Rita promoted)
		\item P (Aslam promoted)
		\item P (Gurpreet promoted)
	\end{enumerate}
	\item If A = {John promoted or Gurpreet promoted}, find P (A).
\end{enumerate}
\solution
%\input{exemplar/11/16/3/10/main.tex}
\item A card is drawn from a deck of 52 cards. Find the probability of getting a king or a heart or a red card.\\
\solution
%\input{exemplar/11/16/3/15/main.tex}
\item The probability that a student will pass his examination is 0.73, the probability of
the student getting a compartment is 0.13, and the probability that the student will
either pass or get compartment is 0.96. State True or False.\\
\solution
%\input{exemplar/11/16/3/31/main.tex}
\item A card is selected from a pack of 52 cards\\
\begin{enumerate}[label=(\alph*)]
\item How many points are there in the sample space?
\item Calculate the probability that the cards is an ace of spades.
\item Calculate the probability that the card is (i) an ace (ii)black card.\\
\end{enumerate}
%\input{ncert/11/16/3/4_1/Prob_4.tex}
\item In a non-leap year, the probability of having 53 tuesdays or 53 wednesdays is\\
\solution
%\input{exemplar/11/16/3/18/main.tex}
\item There are 1000 sealed envelopes in a box, 10 of them contain a cash prize of
Rs 100 each, 100 of them contain a cash prize of Rs 50 each and 200 of them
contain a cash prize of Rs 10 each and rest do not contain any cash prize. If they
are well shuffled and an envelope is picked up out, what is the probability that it
contains no cash prize?\\
\solution
%\input{exemplar/10/13/3/34/main.tex}
\item 
A die is thrown and a card is selected at random from a deck of 52 playing cards. The probability of getting an even number on the die and a spade card.\\
\solution
%\input{exemplar/12/13/3/78/main.tex}
\item
If 4-digit numbers greater than 5,000 are randomly formed from the digits 0, 1, 3, 5, and 7, what is the probability of forming a number divisible by 5 when:
\begin{enumerate}
    \item The digits are repeated?
    \item The repetition of digits is not allowed?
\end{enumerate}
\solution
%\input{ncert/11/16/4/9/main.tex}
\item Consider the probability space $\brak{\Omega, \mathcal{G}, P}$ where $\Omega = [0,2]$ and $\mathcal{G} = \cbrak{\phi, \Omega, [0,1], (1,2]}$. Let $X$ and $Y$ be two functions on $\Omega$ defined as
\begin{align*}
    X(\omega) = 
    \begin{cases}
        1 & \text{if }\omega \in [0, 1]\\
        2 & \text{if }\omega \in (1, 2]
    \end{cases}
\end{align*}
and
\begin{align*}
    Y(\omega) = 
    \begin{cases}
        2 & \text{if }\omega \in [0, 1.5]\\
        3 & \text{if }\omega \in (1.5, 2].
    \end{cases}
\end{align*}
Then which one of the following statements is true?
\begin{enumerate}
    \item [(A)] $X$ is a random variable with respect to $\mathcal{G}$, but $Y$ is not a random variable with respect to $\mathcal{G}$.
    \item [(B)] $Y$ is a random variable with respect to $\mathcal{G}$, but $X$ is not a random variable with respect to $\mathcal{G}$.
    \item [(C)] Neither $X$ nor $Y$ is a random variable with respect to $\mathcal{G}$.
    \item [(D)] Both $X$ and $Y$ are random variables with respect to $\mathcal{G}$.
\end{enumerate} \hfill (GATE ST 2023)\\
\solution
%\input{gate/ST/2023/14/main.tex}
	\item  A die is loaded in such a way that each odd number is twice as likely to occur as
each even number. Find $P(G)$, where $G$ is the event that a number greater than
3 occurs on a single roll of the die.
\\
\solution
		%\input{exemplar/11/16/3/5/main.tex}
	\item All the jacks, queens and kings are removed from a deck of 52 playing cards. The remaining cards are well shuffled and then one card is drawn at random. Giving ace a value 1 similar value for other cards, find the probability that the card has a value 
		\begin{enumerate}
			\item 7
			\item greater than 7
			\item less than 7
		\end{enumerate}
		%\input{exemplar/10/13/3/30/main.tex}
  \item A Lot consists of 48 mobile phones of which 42 are good, 3 have only minor defects and 3 have major defects.Varnika will buy a phone if it is good but the trader will only buy a mobile if it has no major defects. One phone is selected at random from the lot. What is the probability that it is
\begin{enumerate}
	\item acceptable to Varnika?
            \item acceptable to the trader?
\end{enumerate}
\solution
	%\input{exemplar/10/13/3/40/main.tex}
 \item A student says that if you throw a die, it will show up 1 or not 1. Therefore, the probability of getting 1 and the probability of getting 'not 1' each is equal to $\frac{1}{2}$. Is this correct? Give reasons.\\
 \solution
        %\input{exemplar/10/13/2/9/main.tex}
   \item Four candidates A, B, C, D have ap-
plied for the assignment to coach a school cricket
team. If A is twice as likely to be selected as B, and
B and C are given about the same chance of being
selected, while C is twice as likely to be selected
as D, what are the probabilities that
\begin{enumerate}
\item C will be selected?
\item A will not be selected?
\end{enumerate}
	%\input{exemplar/11/16/3/9/main.tex}
 \item A bag contain 24 balls of which $x$ balls are red, $2x$ are white and $3x$ are blue. A ball is selected at random, What is the probability that it is
\begin{enumerate}[label=\alph*)]
\item not red ?
\item white ?
\end{enumerate}
%\input{exemplar/10/13/3/41/main.tex}
If the letters of the word ASSASSINATION are arranged at random. Find the Probability that
\begin{enumerate}[label=(\alph*)]
\item Four $S's$ come consecutively in the word
\item Two  $I's$ and two $N's$ come together
\item All $A's$ are not coming together
\item No two $A's$ are coming together
\end{enumerate}
%\input{exemplar/11/16/3/14/main.tex}
	\item One urn contains two black balls (labelled B1 and B2) and one white ball. A
	second urn contains one black ball and two white balls (labelled W1 and W2).
	Suppose the following experiment is performed. One of the two urns is chosen
	at random. Next a ball is randomly chosen from the urn. Then a second ball is
	chosen at random from the same urn without replacing the first ball.
	
	\begin{enumerate}
	\item What is the probability that two black balls are chosen?
	
	\item What is the probability that two balls of opposite colour are chosen?
	\end{enumerate}
	\solution
	%\input{exemplar/11/16/3/12/main1.tex}
\end{enumerate}

	\item 
The number lock of a suitcase has 4 wheels each labelled with ten digits i.e. from 0 to 9.The lock opens with a sequence of four digits with no repeats.What is the probability of a person getting the right sequence to open the suitcase.
\\
\solution
		%\begin{enumerate}[label=\thesection.\arabic*,ref=\thesection.\theenumi]
	\item One card is drawn from a well-shuffled deck of 52 cards. Find the probability of getting
\begin{enumerate}
\item A king of red colour 
\item A face card 
\item A red face card
\item The jack of hearts
\item A spade
\item The queen of diamonds

\end{enumerate}
\solution
		%\input{ncert/10/15/1/14/main.tex}
	\item Five cards—the ten, jack, queen, king and ace of diamonds, are well-shuffled with their face downwards. One card is then picked up at random.
\begin{enumerate}
\item
What is the probability that the card is the queen? 
\item
If the queen is drawn and put aside, what is the probability that the second card picked up is (a) an ace? (b) a queen?\\
\end{enumerate}
\solution
		%\input{ncert/10/15/1/15/defs.tex}
	\item A bag contains $5$ red balls and some blue balls. If the probability of drawing a blue ball is double that if a red ball, determine the number of blue balls in the bag. 
		\\
\solution
		%\input{ncert/10/15/2/3/defs.tex}
	\item A card is selected from a pack of 52 cards.
 \begin{enumerate}[label=(\alph*)] 
                 \item How many points are there in the sample space?
                 \item Calculate the probability that the card is an ace of spades.
                 \item Calculate the probability that the card is (i) an ace and (ii) black card.
 \end{enumerate}
\solution
		%\input{ncert/11/16/3/4/main.tex}
\item Four cards are drawn from a well-shuffled deck of 52 cards. What is the probability of obtaining 3 diamonds and one spade.
\\
\solution
		%\input{ncert/11/16/4/2/defs.tex}
\item In a certain lottery 10,000 tickets are sold and ten equal prizes are awarded. What is the probability of not getting a prize if you buy (a) one ticket (b) two tickets (c) 10 tickets ?	
\\
\solution
		%\input{ncert/11/16/4/4/defs.tex}
		%
\item 
Out of 100 students, two sections of 40 and 60 are formed. If you and your friend are among the 100 students, what is the probability that
\begin{enumerate}
\item you both enter the same section?
\item you both enter the different sections?
\end{enumerate}
\solution
		%\input{ncert/11/16/4/5/defs.tex}
	\item 
The number lock of a suitcase has 4 wheels each labelled with ten digits i.e. from 0 to 9.The lock opens with a sequence of four digits with no repeats.What is the probability of a person getting the right sequence to open the suitcase.
\\
\solution
		%\input{ncert/11/16/4/10/defs.tex}
		%
\item 
Two cards are drawn at random and without replacement from a pack of 52 playing cards. Find the probability that both the cards are black.
\\
\solution
		%\input{ncert/12/13/2/2/defs.tex}
		\item A box of oranges is inspected by examining three randomly selected oranges drawn without replacement. If all the three oranges are good, the box is approved for sale, otherwise, it is rejected. Find the probability that a box containing 15 oranges out of which 12 are good and 3 are bad ones will be approved for sale.
		\label{ncert/12/13/2/3/defs.tex}
		\item Two balls are drawn at random with replacement from a box containing 10 black and 8 red balls. Find the probability that
		\label{ncert/12/13/2/12}
\begin{enumerate}
\item both balls are red.
\item first ball is black and second is red.
\item one of them is black and other is red.
\end{enumerate}

\item In a hostel, 60\% of the students read Hindi newspaper, 40\% read English newspaper and 20\% read both Hindi and English newspapers. A student is selected at random.
		\label{ncert/12/13/2/15}
\begin{enumerate}
\item Find the probability that she reads neither Hindi nor English newspapers.
\item If she reads Hindi newspaper, find the probability that she reads English newspaper.
\item If she reads English newspaper, find the probability that she reads Hindi newspaper.\\
\end{enumerate}
\item The probability of obtaining an even prime number on each die, when a pair of dice is rolled is 
\begin{enumerate}
    \item $0$ 
    
    \item $\frac{1}{3}$ 
    
    \item $\frac{1}{12}$ 
    
    \item $\frac{1}{36}$ 
\end{enumerate}
\solution
		%\input{ncert/12/13/2/17/defs.tex}
	\item A bag contains 4 red and 4 black balls, another bag contains 2 red and 6 black balls. One of the two bags is selected at random and a ball is drawn from the bag which is found to be red. Find the probability that the ball is drawn from the first bag.
\\
\solution
		%\input{ncert/12/13/3/2/main.tex}
  \item
  Cards with numbers 2 to 101 are placed in a box. A card is selected at random.Find the probability that the card has
\begin{enumerate}[label=(\roman*)]
	\item an even number 
	\item a square number
\end{enumerate}
\solution
%\input{exemplar/10/13/3/32/main.tex}
\item
The king, queen and jack of clubs are removed from a deck of 52 playing cards and then well shuffled. Now one card is drawn at random from the remaining cards.  Determine the probability that the card is
\begin{enumerate}[label=(\roman*)]
\item a club
\item 10 of hearts
\end{enumerate}
\solution
%\input{exemplar/10/13/3/29/main.tex}
\item A team of medical students doing their internship have to assist during surgeries
at a city hospital. The probabilities of surgeries rated as very complex, complex,
routine, simple or very simple are respectively, 0.15, 0.20, 0.31, 0.26, .08. Find
the probabilities that a particular surgery will be rated
\begin{enumerate}
	\item complex or very complex;
	\item neither very complex nor very simple;
	\item routine or complex
	\item routine or simple
\end{enumerate}
\solution
%\input{exemplar/11/16/3/8(1)/main.tex}
\item A card is selected from a pack of 52 cards.
\begin{enumerate}[label=(\alph*)]
    \item How many points are there in the sample space?
    \item Calculate the probability that the card is an ace of spades.
    \item Calculate the probability that the card is (i) an ace and (ii) black card.
\end{enumerate}
\solution
%\input{exemplar/11/16/3/4/main2.tex}
\item The probability that a non leap year selected at random will contain 53 sundays.
\\
\solution
%\input{exemplar/10/13/1/19/main.tex}
\item One of the four persons John, Rita, Aslam or Gurpreet will be promoted next
month. Consequently the sample space consists of four elementary outcomes
S = {John promoted, Rita promoted, Aslam promoted, Gurpreet promoted}
You are told that the chances of John’s promotion is same as that of Gurpreet,
Rita’s chances of promotion are twice as likely as Johns. Aslam’s chances are
four times that of John.
\begin{enumerate}
	\item Determine
	\begin{enumerate}
		\item P (John promoted)
		\item P (Rita promoted)
		\item P (Aslam promoted)
		\item P (Gurpreet promoted)
	\end{enumerate}
	\item If A = {John promoted or Gurpreet promoted}, find P (A).
\end{enumerate}
\solution
%\input{exemplar/11/16/3/10/main.tex}
\item A card is drawn from a deck of 52 cards. Find the probability of getting a king or a heart or a red card.\\
\solution
%\input{exemplar/11/16/3/15/main.tex}
\item The probability that a student will pass his examination is 0.73, the probability of
the student getting a compartment is 0.13, and the probability that the student will
either pass or get compartment is 0.96. State True or False.\\
\solution
%\input{exemplar/11/16/3/31/main.tex}
\item A card is selected from a pack of 52 cards\\
\begin{enumerate}[label=(\alph*)]
\item How many points are there in the sample space?
\item Calculate the probability that the cards is an ace of spades.
\item Calculate the probability that the card is (i) an ace (ii)black card.\\
\end{enumerate}
%\input{ncert/11/16/3/4_1/Prob_4.tex}
\item In a non-leap year, the probability of having 53 tuesdays or 53 wednesdays is\\
\solution
%\input{exemplar/11/16/3/18/main.tex}
\item There are 1000 sealed envelopes in a box, 10 of them contain a cash prize of
Rs 100 each, 100 of them contain a cash prize of Rs 50 each and 200 of them
contain a cash prize of Rs 10 each and rest do not contain any cash prize. If they
are well shuffled and an envelope is picked up out, what is the probability that it
contains no cash prize?\\
\solution
%\input{exemplar/10/13/3/34/main.tex}
\item 
A die is thrown and a card is selected at random from a deck of 52 playing cards. The probability of getting an even number on the die and a spade card.\\
\solution
%\input{exemplar/12/13/3/78/main.tex}
\item
If 4-digit numbers greater than 5,000 are randomly formed from the digits 0, 1, 3, 5, and 7, what is the probability of forming a number divisible by 5 when:
\begin{enumerate}
    \item The digits are repeated?
    \item The repetition of digits is not allowed?
\end{enumerate}
\solution
%\input{ncert/11/16/4/9/main.tex}
\item Consider the probability space $\brak{\Omega, \mathcal{G}, P}$ where $\Omega = [0,2]$ and $\mathcal{G} = \cbrak{\phi, \Omega, [0,1], (1,2]}$. Let $X$ and $Y$ be two functions on $\Omega$ defined as
\begin{align*}
    X(\omega) = 
    \begin{cases}
        1 & \text{if }\omega \in [0, 1]\\
        2 & \text{if }\omega \in (1, 2]
    \end{cases}
\end{align*}
and
\begin{align*}
    Y(\omega) = 
    \begin{cases}
        2 & \text{if }\omega \in [0, 1.5]\\
        3 & \text{if }\omega \in (1.5, 2].
    \end{cases}
\end{align*}
Then which one of the following statements is true?
\begin{enumerate}
    \item [(A)] $X$ is a random variable with respect to $\mathcal{G}$, but $Y$ is not a random variable with respect to $\mathcal{G}$.
    \item [(B)] $Y$ is a random variable with respect to $\mathcal{G}$, but $X$ is not a random variable with respect to $\mathcal{G}$.
    \item [(C)] Neither $X$ nor $Y$ is a random variable with respect to $\mathcal{G}$.
    \item [(D)] Both $X$ and $Y$ are random variables with respect to $\mathcal{G}$.
\end{enumerate} \hfill (GATE ST 2023)\\
\solution
%\input{gate/ST/2023/14/main.tex}
	\item  A die is loaded in such a way that each odd number is twice as likely to occur as
each even number. Find $P(G)$, where $G$ is the event that a number greater than
3 occurs on a single roll of the die.
\\
\solution
		%\input{exemplar/11/16/3/5/main.tex}
	\item All the jacks, queens and kings are removed from a deck of 52 playing cards. The remaining cards are well shuffled and then one card is drawn at random. Giving ace a value 1 similar value for other cards, find the probability that the card has a value 
		\begin{enumerate}
			\item 7
			\item greater than 7
			\item less than 7
		\end{enumerate}
		%\input{exemplar/10/13/3/30/main.tex}
  \item A Lot consists of 48 mobile phones of which 42 are good, 3 have only minor defects and 3 have major defects.Varnika will buy a phone if it is good but the trader will only buy a mobile if it has no major defects. One phone is selected at random from the lot. What is the probability that it is
\begin{enumerate}
	\item acceptable to Varnika?
            \item acceptable to the trader?
\end{enumerate}
\solution
	%\input{exemplar/10/13/3/40/main.tex}
 \item A student says that if you throw a die, it will show up 1 or not 1. Therefore, the probability of getting 1 and the probability of getting 'not 1' each is equal to $\frac{1}{2}$. Is this correct? Give reasons.\\
 \solution
        %\input{exemplar/10/13/2/9/main.tex}
   \item Four candidates A, B, C, D have ap-
plied for the assignment to coach a school cricket
team. If A is twice as likely to be selected as B, and
B and C are given about the same chance of being
selected, while C is twice as likely to be selected
as D, what are the probabilities that
\begin{enumerate}
\item C will be selected?
\item A will not be selected?
\end{enumerate}
	%\input{exemplar/11/16/3/9/main.tex}
 \item A bag contain 24 balls of which $x$ balls are red, $2x$ are white and $3x$ are blue. A ball is selected at random, What is the probability that it is
\begin{enumerate}[label=\alph*)]
\item not red ?
\item white ?
\end{enumerate}
%\input{exemplar/10/13/3/41/main.tex}
If the letters of the word ASSASSINATION are arranged at random. Find the Probability that
\begin{enumerate}[label=(\alph*)]
\item Four $S's$ come consecutively in the word
\item Two  $I's$ and two $N's$ come together
\item All $A's$ are not coming together
\item No two $A's$ are coming together
\end{enumerate}
%\input{exemplar/11/16/3/14/main.tex}
	\item One urn contains two black balls (labelled B1 and B2) and one white ball. A
	second urn contains one black ball and two white balls (labelled W1 and W2).
	Suppose the following experiment is performed. One of the two urns is chosen
	at random. Next a ball is randomly chosen from the urn. Then a second ball is
	chosen at random from the same urn without replacing the first ball.
	
	\begin{enumerate}
	\item What is the probability that two black balls are chosen?
	
	\item What is the probability that two balls of opposite colour are chosen?
	\end{enumerate}
	\solution
	%\input{exemplar/11/16/3/12/main1.tex}
\end{enumerate}

		%
\item 
Two cards are drawn at random and without replacement from a pack of 52 playing cards. Find the probability that both the cards are black.
\\
\solution
		%\begin{enumerate}[label=\thesection.\arabic*,ref=\thesection.\theenumi]
	\item One card is drawn from a well-shuffled deck of 52 cards. Find the probability of getting
\begin{enumerate}
\item A king of red colour 
\item A face card 
\item A red face card
\item The jack of hearts
\item A spade
\item The queen of diamonds

\end{enumerate}
\solution
		%\input{ncert/10/15/1/14/main.tex}
	\item Five cards—the ten, jack, queen, king and ace of diamonds, are well-shuffled with their face downwards. One card is then picked up at random.
\begin{enumerate}
\item
What is the probability that the card is the queen? 
\item
If the queen is drawn and put aside, what is the probability that the second card picked up is (a) an ace? (b) a queen?\\
\end{enumerate}
\solution
		%\input{ncert/10/15/1/15/defs.tex}
	\item A bag contains $5$ red balls and some blue balls. If the probability of drawing a blue ball is double that if a red ball, determine the number of blue balls in the bag. 
		\\
\solution
		%\input{ncert/10/15/2/3/defs.tex}
	\item A card is selected from a pack of 52 cards.
 \begin{enumerate}[label=(\alph*)] 
                 \item How many points are there in the sample space?
                 \item Calculate the probability that the card is an ace of spades.
                 \item Calculate the probability that the card is (i) an ace and (ii) black card.
 \end{enumerate}
\solution
		%\input{ncert/11/16/3/4/main.tex}
\item Four cards are drawn from a well-shuffled deck of 52 cards. What is the probability of obtaining 3 diamonds and one spade.
\\
\solution
		%\input{ncert/11/16/4/2/defs.tex}
\item In a certain lottery 10,000 tickets are sold and ten equal prizes are awarded. What is the probability of not getting a prize if you buy (a) one ticket (b) two tickets (c) 10 tickets ?	
\\
\solution
		%\input{ncert/11/16/4/4/defs.tex}
		%
\item 
Out of 100 students, two sections of 40 and 60 are formed. If you and your friend are among the 100 students, what is the probability that
\begin{enumerate}
\item you both enter the same section?
\item you both enter the different sections?
\end{enumerate}
\solution
		%\input{ncert/11/16/4/5/defs.tex}
	\item 
The number lock of a suitcase has 4 wheels each labelled with ten digits i.e. from 0 to 9.The lock opens with a sequence of four digits with no repeats.What is the probability of a person getting the right sequence to open the suitcase.
\\
\solution
		%\input{ncert/11/16/4/10/defs.tex}
		%
\item 
Two cards are drawn at random and without replacement from a pack of 52 playing cards. Find the probability that both the cards are black.
\\
\solution
		%\input{ncert/12/13/2/2/defs.tex}
		\item A box of oranges is inspected by examining three randomly selected oranges drawn without replacement. If all the three oranges are good, the box is approved for sale, otherwise, it is rejected. Find the probability that a box containing 15 oranges out of which 12 are good and 3 are bad ones will be approved for sale.
		\label{ncert/12/13/2/3/defs.tex}
		\item Two balls are drawn at random with replacement from a box containing 10 black and 8 red balls. Find the probability that
		\label{ncert/12/13/2/12}
\begin{enumerate}
\item both balls are red.
\item first ball is black and second is red.
\item one of them is black and other is red.
\end{enumerate}

\item In a hostel, 60\% of the students read Hindi newspaper, 40\% read English newspaper and 20\% read both Hindi and English newspapers. A student is selected at random.
		\label{ncert/12/13/2/15}
\begin{enumerate}
\item Find the probability that she reads neither Hindi nor English newspapers.
\item If she reads Hindi newspaper, find the probability that she reads English newspaper.
\item If she reads English newspaper, find the probability that she reads Hindi newspaper.\\
\end{enumerate}
\item The probability of obtaining an even prime number on each die, when a pair of dice is rolled is 
\begin{enumerate}
    \item $0$ 
    
    \item $\frac{1}{3}$ 
    
    \item $\frac{1}{12}$ 
    
    \item $\frac{1}{36}$ 
\end{enumerate}
\solution
		%\input{ncert/12/13/2/17/defs.tex}
	\item A bag contains 4 red and 4 black balls, another bag contains 2 red and 6 black balls. One of the two bags is selected at random and a ball is drawn from the bag which is found to be red. Find the probability that the ball is drawn from the first bag.
\\
\solution
		%\input{ncert/12/13/3/2/main.tex}
  \item
  Cards with numbers 2 to 101 are placed in a box. A card is selected at random.Find the probability that the card has
\begin{enumerate}[label=(\roman*)]
	\item an even number 
	\item a square number
\end{enumerate}
\solution
%\input{exemplar/10/13/3/32/main.tex}
\item
The king, queen and jack of clubs are removed from a deck of 52 playing cards and then well shuffled. Now one card is drawn at random from the remaining cards.  Determine the probability that the card is
\begin{enumerate}[label=(\roman*)]
\item a club
\item 10 of hearts
\end{enumerate}
\solution
%\input{exemplar/10/13/3/29/main.tex}
\item A team of medical students doing their internship have to assist during surgeries
at a city hospital. The probabilities of surgeries rated as very complex, complex,
routine, simple or very simple are respectively, 0.15, 0.20, 0.31, 0.26, .08. Find
the probabilities that a particular surgery will be rated
\begin{enumerate}
	\item complex or very complex;
	\item neither very complex nor very simple;
	\item routine or complex
	\item routine or simple
\end{enumerate}
\solution
%\input{exemplar/11/16/3/8(1)/main.tex}
\item A card is selected from a pack of 52 cards.
\begin{enumerate}[label=(\alph*)]
    \item How many points are there in the sample space?
    \item Calculate the probability that the card is an ace of spades.
    \item Calculate the probability that the card is (i) an ace and (ii) black card.
\end{enumerate}
\solution
%\input{exemplar/11/16/3/4/main2.tex}
\item The probability that a non leap year selected at random will contain 53 sundays.
\\
\solution
%\input{exemplar/10/13/1/19/main.tex}
\item One of the four persons John, Rita, Aslam or Gurpreet will be promoted next
month. Consequently the sample space consists of four elementary outcomes
S = {John promoted, Rita promoted, Aslam promoted, Gurpreet promoted}
You are told that the chances of John’s promotion is same as that of Gurpreet,
Rita’s chances of promotion are twice as likely as Johns. Aslam’s chances are
four times that of John.
\begin{enumerate}
	\item Determine
	\begin{enumerate}
		\item P (John promoted)
		\item P (Rita promoted)
		\item P (Aslam promoted)
		\item P (Gurpreet promoted)
	\end{enumerate}
	\item If A = {John promoted or Gurpreet promoted}, find P (A).
\end{enumerate}
\solution
%\input{exemplar/11/16/3/10/main.tex}
\item A card is drawn from a deck of 52 cards. Find the probability of getting a king or a heart or a red card.\\
\solution
%\input{exemplar/11/16/3/15/main.tex}
\item The probability that a student will pass his examination is 0.73, the probability of
the student getting a compartment is 0.13, and the probability that the student will
either pass or get compartment is 0.96. State True or False.\\
\solution
%\input{exemplar/11/16/3/31/main.tex}
\item A card is selected from a pack of 52 cards\\
\begin{enumerate}[label=(\alph*)]
\item How many points are there in the sample space?
\item Calculate the probability that the cards is an ace of spades.
\item Calculate the probability that the card is (i) an ace (ii)black card.\\
\end{enumerate}
%\input{ncert/11/16/3/4_1/Prob_4.tex}
\item In a non-leap year, the probability of having 53 tuesdays or 53 wednesdays is\\
\solution
%\input{exemplar/11/16/3/18/main.tex}
\item There are 1000 sealed envelopes in a box, 10 of them contain a cash prize of
Rs 100 each, 100 of them contain a cash prize of Rs 50 each and 200 of them
contain a cash prize of Rs 10 each and rest do not contain any cash prize. If they
are well shuffled and an envelope is picked up out, what is the probability that it
contains no cash prize?\\
\solution
%\input{exemplar/10/13/3/34/main.tex}
\item 
A die is thrown and a card is selected at random from a deck of 52 playing cards. The probability of getting an even number on the die and a spade card.\\
\solution
%\input{exemplar/12/13/3/78/main.tex}
\item
If 4-digit numbers greater than 5,000 are randomly formed from the digits 0, 1, 3, 5, and 7, what is the probability of forming a number divisible by 5 when:
\begin{enumerate}
    \item The digits are repeated?
    \item The repetition of digits is not allowed?
\end{enumerate}
\solution
%\input{ncert/11/16/4/9/main.tex}
\item Consider the probability space $\brak{\Omega, \mathcal{G}, P}$ where $\Omega = [0,2]$ and $\mathcal{G} = \cbrak{\phi, \Omega, [0,1], (1,2]}$. Let $X$ and $Y$ be two functions on $\Omega$ defined as
\begin{align*}
    X(\omega) = 
    \begin{cases}
        1 & \text{if }\omega \in [0, 1]\\
        2 & \text{if }\omega \in (1, 2]
    \end{cases}
\end{align*}
and
\begin{align*}
    Y(\omega) = 
    \begin{cases}
        2 & \text{if }\omega \in [0, 1.5]\\
        3 & \text{if }\omega \in (1.5, 2].
    \end{cases}
\end{align*}
Then which one of the following statements is true?
\begin{enumerate}
    \item [(A)] $X$ is a random variable with respect to $\mathcal{G}$, but $Y$ is not a random variable with respect to $\mathcal{G}$.
    \item [(B)] $Y$ is a random variable with respect to $\mathcal{G}$, but $X$ is not a random variable with respect to $\mathcal{G}$.
    \item [(C)] Neither $X$ nor $Y$ is a random variable with respect to $\mathcal{G}$.
    \item [(D)] Both $X$ and $Y$ are random variables with respect to $\mathcal{G}$.
\end{enumerate} \hfill (GATE ST 2023)\\
\solution
%\input{gate/ST/2023/14/main.tex}
	\item  A die is loaded in such a way that each odd number is twice as likely to occur as
each even number. Find $P(G)$, where $G$ is the event that a number greater than
3 occurs on a single roll of the die.
\\
\solution
		%\input{exemplar/11/16/3/5/main.tex}
	\item All the jacks, queens and kings are removed from a deck of 52 playing cards. The remaining cards are well shuffled and then one card is drawn at random. Giving ace a value 1 similar value for other cards, find the probability that the card has a value 
		\begin{enumerate}
			\item 7
			\item greater than 7
			\item less than 7
		\end{enumerate}
		%\input{exemplar/10/13/3/30/main.tex}
  \item A Lot consists of 48 mobile phones of which 42 are good, 3 have only minor defects and 3 have major defects.Varnika will buy a phone if it is good but the trader will only buy a mobile if it has no major defects. One phone is selected at random from the lot. What is the probability that it is
\begin{enumerate}
	\item acceptable to Varnika?
            \item acceptable to the trader?
\end{enumerate}
\solution
	%\input{exemplar/10/13/3/40/main.tex}
 \item A student says that if you throw a die, it will show up 1 or not 1. Therefore, the probability of getting 1 and the probability of getting 'not 1' each is equal to $\frac{1}{2}$. Is this correct? Give reasons.\\
 \solution
        %\input{exemplar/10/13/2/9/main.tex}
   \item Four candidates A, B, C, D have ap-
plied for the assignment to coach a school cricket
team. If A is twice as likely to be selected as B, and
B and C are given about the same chance of being
selected, while C is twice as likely to be selected
as D, what are the probabilities that
\begin{enumerate}
\item C will be selected?
\item A will not be selected?
\end{enumerate}
	%\input{exemplar/11/16/3/9/main.tex}
 \item A bag contain 24 balls of which $x$ balls are red, $2x$ are white and $3x$ are blue. A ball is selected at random, What is the probability that it is
\begin{enumerate}[label=\alph*)]
\item not red ?
\item white ?
\end{enumerate}
%\input{exemplar/10/13/3/41/main.tex}
If the letters of the word ASSASSINATION are arranged at random. Find the Probability that
\begin{enumerate}[label=(\alph*)]
\item Four $S's$ come consecutively in the word
\item Two  $I's$ and two $N's$ come together
\item All $A's$ are not coming together
\item No two $A's$ are coming together
\end{enumerate}
%\input{exemplar/11/16/3/14/main.tex}
	\item One urn contains two black balls (labelled B1 and B2) and one white ball. A
	second urn contains one black ball and two white balls (labelled W1 and W2).
	Suppose the following experiment is performed. One of the two urns is chosen
	at random. Next a ball is randomly chosen from the urn. Then a second ball is
	chosen at random from the same urn without replacing the first ball.
	
	\begin{enumerate}
	\item What is the probability that two black balls are chosen?
	
	\item What is the probability that two balls of opposite colour are chosen?
	\end{enumerate}
	\solution
	%\input{exemplar/11/16/3/12/main1.tex}
\end{enumerate}

		\item A box of oranges is inspected by examining three randomly selected oranges drawn without replacement. If all the three oranges are good, the box is approved for sale, otherwise, it is rejected. Find the probability that a box containing 15 oranges out of which 12 are good and 3 are bad ones will be approved for sale.
		\label{ncert/12/13/2/3/defs.tex}
		\item Two balls are drawn at random with replacement from a box containing 10 black and 8 red balls. Find the probability that
		\label{ncert/12/13/2/12}
\begin{enumerate}
\item both balls are red.
\item first ball is black and second is red.
\item one of them is black and other is red.
\end{enumerate}

\item In a hostel, 60\% of the students read Hindi newspaper, 40\% read English newspaper and 20\% read both Hindi and English newspapers. A student is selected at random.
		\label{ncert/12/13/2/15}
\begin{enumerate}
\item Find the probability that she reads neither Hindi nor English newspapers.
\item If she reads Hindi newspaper, find the probability that she reads English newspaper.
\item If she reads English newspaper, find the probability that she reads Hindi newspaper.\\
\end{enumerate}
\item The probability of obtaining an even prime number on each die, when a pair of dice is rolled is 
\begin{enumerate}
    \item $0$ 
    
    \item $\frac{1}{3}$ 
    
    \item $\frac{1}{12}$ 
    
    \item $\frac{1}{36}$ 
\end{enumerate}
\solution
		%\begin{enumerate}[label=\thesection.\arabic*,ref=\thesection.\theenumi]
	\item One card is drawn from a well-shuffled deck of 52 cards. Find the probability of getting
\begin{enumerate}
\item A king of red colour 
\item A face card 
\item A red face card
\item The jack of hearts
\item A spade
\item The queen of diamonds

\end{enumerate}
\solution
		%\input{ncert/10/15/1/14/main.tex}
	\item Five cards—the ten, jack, queen, king and ace of diamonds, are well-shuffled with their face downwards. One card is then picked up at random.
\begin{enumerate}
\item
What is the probability that the card is the queen? 
\item
If the queen is drawn and put aside, what is the probability that the second card picked up is (a) an ace? (b) a queen?\\
\end{enumerate}
\solution
		%\input{ncert/10/15/1/15/defs.tex}
	\item A bag contains $5$ red balls and some blue balls. If the probability of drawing a blue ball is double that if a red ball, determine the number of blue balls in the bag. 
		\\
\solution
		%\input{ncert/10/15/2/3/defs.tex}
	\item A card is selected from a pack of 52 cards.
 \begin{enumerate}[label=(\alph*)] 
                 \item How many points are there in the sample space?
                 \item Calculate the probability that the card is an ace of spades.
                 \item Calculate the probability that the card is (i) an ace and (ii) black card.
 \end{enumerate}
\solution
		%\input{ncert/11/16/3/4/main.tex}
\item Four cards are drawn from a well-shuffled deck of 52 cards. What is the probability of obtaining 3 diamonds and one spade.
\\
\solution
		%\input{ncert/11/16/4/2/defs.tex}
\item In a certain lottery 10,000 tickets are sold and ten equal prizes are awarded. What is the probability of not getting a prize if you buy (a) one ticket (b) two tickets (c) 10 tickets ?	
\\
\solution
		%\input{ncert/11/16/4/4/defs.tex}
		%
\item 
Out of 100 students, two sections of 40 and 60 are formed. If you and your friend are among the 100 students, what is the probability that
\begin{enumerate}
\item you both enter the same section?
\item you both enter the different sections?
\end{enumerate}
\solution
		%\input{ncert/11/16/4/5/defs.tex}
	\item 
The number lock of a suitcase has 4 wheels each labelled with ten digits i.e. from 0 to 9.The lock opens with a sequence of four digits with no repeats.What is the probability of a person getting the right sequence to open the suitcase.
\\
\solution
		%\input{ncert/11/16/4/10/defs.tex}
		%
\item 
Two cards are drawn at random and without replacement from a pack of 52 playing cards. Find the probability that both the cards are black.
\\
\solution
		%\input{ncert/12/13/2/2/defs.tex}
		\item A box of oranges is inspected by examining three randomly selected oranges drawn without replacement. If all the three oranges are good, the box is approved for sale, otherwise, it is rejected. Find the probability that a box containing 15 oranges out of which 12 are good and 3 are bad ones will be approved for sale.
		\label{ncert/12/13/2/3/defs.tex}
		\item Two balls are drawn at random with replacement from a box containing 10 black and 8 red balls. Find the probability that
		\label{ncert/12/13/2/12}
\begin{enumerate}
\item both balls are red.
\item first ball is black and second is red.
\item one of them is black and other is red.
\end{enumerate}

\item In a hostel, 60\% of the students read Hindi newspaper, 40\% read English newspaper and 20\% read both Hindi and English newspapers. A student is selected at random.
		\label{ncert/12/13/2/15}
\begin{enumerate}
\item Find the probability that she reads neither Hindi nor English newspapers.
\item If she reads Hindi newspaper, find the probability that she reads English newspaper.
\item If she reads English newspaper, find the probability that she reads Hindi newspaper.\\
\end{enumerate}
\item The probability of obtaining an even prime number on each die, when a pair of dice is rolled is 
\begin{enumerate}
    \item $0$ 
    
    \item $\frac{1}{3}$ 
    
    \item $\frac{1}{12}$ 
    
    \item $\frac{1}{36}$ 
\end{enumerate}
\solution
		%\input{ncert/12/13/2/17/defs.tex}
	\item A bag contains 4 red and 4 black balls, another bag contains 2 red and 6 black balls. One of the two bags is selected at random and a ball is drawn from the bag which is found to be red. Find the probability that the ball is drawn from the first bag.
\\
\solution
		%\input{ncert/12/13/3/2/main.tex}
  \item
  Cards with numbers 2 to 101 are placed in a box. A card is selected at random.Find the probability that the card has
\begin{enumerate}[label=(\roman*)]
	\item an even number 
	\item a square number
\end{enumerate}
\solution
%\input{exemplar/10/13/3/32/main.tex}
\item
The king, queen and jack of clubs are removed from a deck of 52 playing cards and then well shuffled. Now one card is drawn at random from the remaining cards.  Determine the probability that the card is
\begin{enumerate}[label=(\roman*)]
\item a club
\item 10 of hearts
\end{enumerate}
\solution
%\input{exemplar/10/13/3/29/main.tex}
\item A team of medical students doing their internship have to assist during surgeries
at a city hospital. The probabilities of surgeries rated as very complex, complex,
routine, simple or very simple are respectively, 0.15, 0.20, 0.31, 0.26, .08. Find
the probabilities that a particular surgery will be rated
\begin{enumerate}
	\item complex or very complex;
	\item neither very complex nor very simple;
	\item routine or complex
	\item routine or simple
\end{enumerate}
\solution
%\input{exemplar/11/16/3/8(1)/main.tex}
\item A card is selected from a pack of 52 cards.
\begin{enumerate}[label=(\alph*)]
    \item How many points are there in the sample space?
    \item Calculate the probability that the card is an ace of spades.
    \item Calculate the probability that the card is (i) an ace and (ii) black card.
\end{enumerate}
\solution
%\input{exemplar/11/16/3/4/main2.tex}
\item The probability that a non leap year selected at random will contain 53 sundays.
\\
\solution
%\input{exemplar/10/13/1/19/main.tex}
\item One of the four persons John, Rita, Aslam or Gurpreet will be promoted next
month. Consequently the sample space consists of four elementary outcomes
S = {John promoted, Rita promoted, Aslam promoted, Gurpreet promoted}
You are told that the chances of John’s promotion is same as that of Gurpreet,
Rita’s chances of promotion are twice as likely as Johns. Aslam’s chances are
four times that of John.
\begin{enumerate}
	\item Determine
	\begin{enumerate}
		\item P (John promoted)
		\item P (Rita promoted)
		\item P (Aslam promoted)
		\item P (Gurpreet promoted)
	\end{enumerate}
	\item If A = {John promoted or Gurpreet promoted}, find P (A).
\end{enumerate}
\solution
%\input{exemplar/11/16/3/10/main.tex}
\item A card is drawn from a deck of 52 cards. Find the probability of getting a king or a heart or a red card.\\
\solution
%\input{exemplar/11/16/3/15/main.tex}
\item The probability that a student will pass his examination is 0.73, the probability of
the student getting a compartment is 0.13, and the probability that the student will
either pass or get compartment is 0.96. State True or False.\\
\solution
%\input{exemplar/11/16/3/31/main.tex}
\item A card is selected from a pack of 52 cards\\
\begin{enumerate}[label=(\alph*)]
\item How many points are there in the sample space?
\item Calculate the probability that the cards is an ace of spades.
\item Calculate the probability that the card is (i) an ace (ii)black card.\\
\end{enumerate}
%\input{ncert/11/16/3/4_1/Prob_4.tex}
\item In a non-leap year, the probability of having 53 tuesdays or 53 wednesdays is\\
\solution
%\input{exemplar/11/16/3/18/main.tex}
\item There are 1000 sealed envelopes in a box, 10 of them contain a cash prize of
Rs 100 each, 100 of them contain a cash prize of Rs 50 each and 200 of them
contain a cash prize of Rs 10 each and rest do not contain any cash prize. If they
are well shuffled and an envelope is picked up out, what is the probability that it
contains no cash prize?\\
\solution
%\input{exemplar/10/13/3/34/main.tex}
\item 
A die is thrown and a card is selected at random from a deck of 52 playing cards. The probability of getting an even number on the die and a spade card.\\
\solution
%\input{exemplar/12/13/3/78/main.tex}
\item
If 4-digit numbers greater than 5,000 are randomly formed from the digits 0, 1, 3, 5, and 7, what is the probability of forming a number divisible by 5 when:
\begin{enumerate}
    \item The digits are repeated?
    \item The repetition of digits is not allowed?
\end{enumerate}
\solution
%\input{ncert/11/16/4/9/main.tex}
\item Consider the probability space $\brak{\Omega, \mathcal{G}, P}$ where $\Omega = [0,2]$ and $\mathcal{G} = \cbrak{\phi, \Omega, [0,1], (1,2]}$. Let $X$ and $Y$ be two functions on $\Omega$ defined as
\begin{align*}
    X(\omega) = 
    \begin{cases}
        1 & \text{if }\omega \in [0, 1]\\
        2 & \text{if }\omega \in (1, 2]
    \end{cases}
\end{align*}
and
\begin{align*}
    Y(\omega) = 
    \begin{cases}
        2 & \text{if }\omega \in [0, 1.5]\\
        3 & \text{if }\omega \in (1.5, 2].
    \end{cases}
\end{align*}
Then which one of the following statements is true?
\begin{enumerate}
    \item [(A)] $X$ is a random variable with respect to $\mathcal{G}$, but $Y$ is not a random variable with respect to $\mathcal{G}$.
    \item [(B)] $Y$ is a random variable with respect to $\mathcal{G}$, but $X$ is not a random variable with respect to $\mathcal{G}$.
    \item [(C)] Neither $X$ nor $Y$ is a random variable with respect to $\mathcal{G}$.
    \item [(D)] Both $X$ and $Y$ are random variables with respect to $\mathcal{G}$.
\end{enumerate} \hfill (GATE ST 2023)\\
\solution
%\input{gate/ST/2023/14/main.tex}
	\item  A die is loaded in such a way that each odd number is twice as likely to occur as
each even number. Find $P(G)$, where $G$ is the event that a number greater than
3 occurs on a single roll of the die.
\\
\solution
		%\input{exemplar/11/16/3/5/main.tex}
	\item All the jacks, queens and kings are removed from a deck of 52 playing cards. The remaining cards are well shuffled and then one card is drawn at random. Giving ace a value 1 similar value for other cards, find the probability that the card has a value 
		\begin{enumerate}
			\item 7
			\item greater than 7
			\item less than 7
		\end{enumerate}
		%\input{exemplar/10/13/3/30/main.tex}
  \item A Lot consists of 48 mobile phones of which 42 are good, 3 have only minor defects and 3 have major defects.Varnika will buy a phone if it is good but the trader will only buy a mobile if it has no major defects. One phone is selected at random from the lot. What is the probability that it is
\begin{enumerate}
	\item acceptable to Varnika?
            \item acceptable to the trader?
\end{enumerate}
\solution
	%\input{exemplar/10/13/3/40/main.tex}
 \item A student says that if you throw a die, it will show up 1 or not 1. Therefore, the probability of getting 1 and the probability of getting 'not 1' each is equal to $\frac{1}{2}$. Is this correct? Give reasons.\\
 \solution
        %\input{exemplar/10/13/2/9/main.tex}
   \item Four candidates A, B, C, D have ap-
plied for the assignment to coach a school cricket
team. If A is twice as likely to be selected as B, and
B and C are given about the same chance of being
selected, while C is twice as likely to be selected
as D, what are the probabilities that
\begin{enumerate}
\item C will be selected?
\item A will not be selected?
\end{enumerate}
	%\input{exemplar/11/16/3/9/main.tex}
 \item A bag contain 24 balls of which $x$ balls are red, $2x$ are white and $3x$ are blue. A ball is selected at random, What is the probability that it is
\begin{enumerate}[label=\alph*)]
\item not red ?
\item white ?
\end{enumerate}
%\input{exemplar/10/13/3/41/main.tex}
If the letters of the word ASSASSINATION are arranged at random. Find the Probability that
\begin{enumerate}[label=(\alph*)]
\item Four $S's$ come consecutively in the word
\item Two  $I's$ and two $N's$ come together
\item All $A's$ are not coming together
\item No two $A's$ are coming together
\end{enumerate}
%\input{exemplar/11/16/3/14/main.tex}
	\item One urn contains two black balls (labelled B1 and B2) and one white ball. A
	second urn contains one black ball and two white balls (labelled W1 and W2).
	Suppose the following experiment is performed. One of the two urns is chosen
	at random. Next a ball is randomly chosen from the urn. Then a second ball is
	chosen at random from the same urn without replacing the first ball.
	
	\begin{enumerate}
	\item What is the probability that two black balls are chosen?
	
	\item What is the probability that two balls of opposite colour are chosen?
	\end{enumerate}
	\solution
	%\input{exemplar/11/16/3/12/main1.tex}
\end{enumerate}

	\item A bag contains 4 red and 4 black balls, another bag contains 2 red and 6 black balls. One of the two bags is selected at random and a ball is drawn from the bag which is found to be red. Find the probability that the ball is drawn from the first bag.
\\
\solution
		%\begin{table}[H]
	\centering
\begin{tabular}{|c|c|c|}
\hline
Random variable &Value &Definition\\ \hline
\multirow{3}{*}{X} &0 &Slips of Rs 1\\
&1 &Slips of Rs 5\\
&2 &Slips of Rs 13\\ \hline
\multirow{2}{*}{Y} &0 &Box A\\
&1 &Box B\\\hline
\end{tabular}
\caption{}
\label{tab:Distribution}
\end{table}
See \tabref{tab:Distribution}.
\begin{align}
p_{Y}\brak{k}= \begin{cases} 
      \frac{1}{3} & {k=0} \\
      \frac{2}{3 }& {k=1} 
   \end{cases}
   \\
p_{Y|X}\brak{0|0} = \frac{19}{25}\, 
p_{Y|X}\brak{0|1} = \frac{6}{25}\,
p_{Y|X}\brak{1|0} = \frac{45}{50}\,
p_{Y|X}\brak{1|2} = \frac{5}{50}
\end{align}
The desired probability is the probability that a slip drawn at random is marked other than Rs 1,
\begin{align}
&=1-p_X\brak{0}\\
&= p_X(1) + p_X(2)
\end{align}
Using Bayes theorem,
\begin{align}
&= p_Y\brak{0} \times \pr{Y=0 | X=1} + p_Y\brak{1} \times \pr{Y=1|X=2}\\
&=\frac{1}{3} \times \frac{6}{25} + \frac{2}{3} \times \frac{5}{50}\\
&=\frac{11}{75}
\end{align}

\newpage

%\tableofcontents

\bigskip

\renewcommand{\thefigure}{\theenumi}
\renewcommand{\thetable}{\theenumi}
%\renewcommand{\theequation}{\theenumi}

%\begin{abstract}
%%\boldmath
%In this letter, an algorithm for evaluating the exact analytical bit error rate  (BER)  for the piecewise linear (PL) combiner for  multiple relays is presented. Previous results were available only for upto three relays. The algorithm is unique in the sense that  the actual mathematical expressions, that are prohibitively large, need not be explicitly obtained. The diversity gain due to multiple relays is shown through plots of the analytical BER, well supported by simulations. 
%
%\end{abstract}
% IEEEtran.cls defaults to using nonbold math in the Abstract.
% This preserves the distinction between vectors and scalars. However,
% if the journal you are submitting to favors bold math in the abstract,
% then you can use LaTeX's standard command \boldmath at the very start
% of the abstract to achieve this. Many IEEE journals frown on math
% in the abstract anyway.

% Note that keywords are not normally used for peerreview papers.
%\begin{IEEEkeywords}
%Cooperative diversity, decode and forward, piecewise linear
%\end{IEEEkeywords}



% For peer review papers, you can put extra information on the cover
% page as needed:
% \ifCLASSOPTIONpeerreview
% \begin{center} \bfseries EDICS Category: 3-BBND \end{center}
% \fi
%
% For peerreview papers, this IEEEtran command inserts a page break and
% creates the second title. It will be ignored for other modes.
%\IEEEpeerreviewmaketitle




  \item
  Cards with numbers 2 to 101 are placed in a box. A card is selected at random.Find the probability that the card has
\begin{enumerate}[label=(\roman*)]
	\item an even number 
	\item a square number
\end{enumerate}
\solution
%\begin{table}[H]
	\centering
\begin{tabular}{|c|c|c|}
\hline
Random variable &Value &Definition\\ \hline
\multirow{3}{*}{X} &0 &Slips of Rs 1\\
&1 &Slips of Rs 5\\
&2 &Slips of Rs 13\\ \hline
\multirow{2}{*}{Y} &0 &Box A\\
&1 &Box B\\\hline
\end{tabular}
\caption{}
\label{tab:Distribution}
\end{table}
See \tabref{tab:Distribution}.
\begin{align}
p_{Y}\brak{k}= \begin{cases} 
      \frac{1}{3} & {k=0} \\
      \frac{2}{3 }& {k=1} 
   \end{cases}
   \\
p_{Y|X}\brak{0|0} = \frac{19}{25}\, 
p_{Y|X}\brak{0|1} = \frac{6}{25}\,
p_{Y|X}\brak{1|0} = \frac{45}{50}\,
p_{Y|X}\brak{1|2} = \frac{5}{50}
\end{align}
The desired probability is the probability that a slip drawn at random is marked other than Rs 1,
\begin{align}
&=1-p_X\brak{0}\\
&= p_X(1) + p_X(2)
\end{align}
Using Bayes theorem,
\begin{align}
&= p_Y\brak{0} \times \pr{Y=0 | X=1} + p_Y\brak{1} \times \pr{Y=1|X=2}\\
&=\frac{1}{3} \times \frac{6}{25} + \frac{2}{3} \times \frac{5}{50}\\
&=\frac{11}{75}
\end{align}

\newpage

%\tableofcontents

\bigskip

\renewcommand{\thefigure}{\theenumi}
\renewcommand{\thetable}{\theenumi}
%\renewcommand{\theequation}{\theenumi}

%\begin{abstract}
%%\boldmath
%In this letter, an algorithm for evaluating the exact analytical bit error rate  (BER)  for the piecewise linear (PL) combiner for  multiple relays is presented. Previous results were available only for upto three relays. The algorithm is unique in the sense that  the actual mathematical expressions, that are prohibitively large, need not be explicitly obtained. The diversity gain due to multiple relays is shown through plots of the analytical BER, well supported by simulations. 
%
%\end{abstract}
% IEEEtran.cls defaults to using nonbold math in the Abstract.
% This preserves the distinction between vectors and scalars. However,
% if the journal you are submitting to favors bold math in the abstract,
% then you can use LaTeX's standard command \boldmath at the very start
% of the abstract to achieve this. Many IEEE journals frown on math
% in the abstract anyway.

% Note that keywords are not normally used for peerreview papers.
%\begin{IEEEkeywords}
%Cooperative diversity, decode and forward, piecewise linear
%\end{IEEEkeywords}



% For peer review papers, you can put extra information on the cover
% page as needed:
% \ifCLASSOPTIONpeerreview
% \begin{center} \bfseries EDICS Category: 3-BBND \end{center}
% \fi
%
% For peerreview papers, this IEEEtran command inserts a page break and
% creates the second title. It will be ignored for other modes.
%\IEEEpeerreviewmaketitle




\item
The king, queen and jack of clubs are removed from a deck of 52 playing cards and then well shuffled. Now one card is drawn at random from the remaining cards.  Determine the probability that the card is
\begin{enumerate}[label=(\roman*)]
\item a club
\item 10 of hearts
\end{enumerate}
\solution
%\begin{table}[H]
	\centering
\begin{tabular}{|c|c|c|}
\hline
Random variable &Value &Definition\\ \hline
\multirow{3}{*}{X} &0 &Slips of Rs 1\\
&1 &Slips of Rs 5\\
&2 &Slips of Rs 13\\ \hline
\multirow{2}{*}{Y} &0 &Box A\\
&1 &Box B\\\hline
\end{tabular}
\caption{}
\label{tab:Distribution}
\end{table}
See \tabref{tab:Distribution}.
\begin{align}
p_{Y}\brak{k}= \begin{cases} 
      \frac{1}{3} & {k=0} \\
      \frac{2}{3 }& {k=1} 
   \end{cases}
   \\
p_{Y|X}\brak{0|0} = \frac{19}{25}\, 
p_{Y|X}\brak{0|1} = \frac{6}{25}\,
p_{Y|X}\brak{1|0} = \frac{45}{50}\,
p_{Y|X}\brak{1|2} = \frac{5}{50}
\end{align}
The desired probability is the probability that a slip drawn at random is marked other than Rs 1,
\begin{align}
&=1-p_X\brak{0}\\
&= p_X(1) + p_X(2)
\end{align}
Using Bayes theorem,
\begin{align}
&= p_Y\brak{0} \times \pr{Y=0 | X=1} + p_Y\brak{1} \times \pr{Y=1|X=2}\\
&=\frac{1}{3} \times \frac{6}{25} + \frac{2}{3} \times \frac{5}{50}\\
&=\frac{11}{75}
\end{align}

\newpage

%\tableofcontents

\bigskip

\renewcommand{\thefigure}{\theenumi}
\renewcommand{\thetable}{\theenumi}
%\renewcommand{\theequation}{\theenumi}

%\begin{abstract}
%%\boldmath
%In this letter, an algorithm for evaluating the exact analytical bit error rate  (BER)  for the piecewise linear (PL) combiner for  multiple relays is presented. Previous results were available only for upto three relays. The algorithm is unique in the sense that  the actual mathematical expressions, that are prohibitively large, need not be explicitly obtained. The diversity gain due to multiple relays is shown through plots of the analytical BER, well supported by simulations. 
%
%\end{abstract}
% IEEEtran.cls defaults to using nonbold math in the Abstract.
% This preserves the distinction between vectors and scalars. However,
% if the journal you are submitting to favors bold math in the abstract,
% then you can use LaTeX's standard command \boldmath at the very start
% of the abstract to achieve this. Many IEEE journals frown on math
% in the abstract anyway.

% Note that keywords are not normally used for peerreview papers.
%\begin{IEEEkeywords}
%Cooperative diversity, decode and forward, piecewise linear
%\end{IEEEkeywords}



% For peer review papers, you can put extra information on the cover
% page as needed:
% \ifCLASSOPTIONpeerreview
% \begin{center} \bfseries EDICS Category: 3-BBND \end{center}
% \fi
%
% For peerreview papers, this IEEEtran command inserts a page break and
% creates the second title. It will be ignored for other modes.
%\IEEEpeerreviewmaketitle




\item A team of medical students doing their internship have to assist during surgeries
at a city hospital. The probabilities of surgeries rated as very complex, complex,
routine, simple or very simple are respectively, 0.15, 0.20, 0.31, 0.26, .08. Find
the probabilities that a particular surgery will be rated
\begin{enumerate}
	\item complex or very complex;
	\item neither very complex nor very simple;
	\item routine or complex
	\item routine or simple
\end{enumerate}
\solution
%\begin{table}[H]
	\centering
\begin{tabular}{|c|c|c|}
\hline
Random variable &Value &Definition\\ \hline
\multirow{3}{*}{X} &0 &Slips of Rs 1\\
&1 &Slips of Rs 5\\
&2 &Slips of Rs 13\\ \hline
\multirow{2}{*}{Y} &0 &Box A\\
&1 &Box B\\\hline
\end{tabular}
\caption{}
\label{tab:Distribution}
\end{table}
See \tabref{tab:Distribution}.
\begin{align}
p_{Y}\brak{k}= \begin{cases} 
      \frac{1}{3} & {k=0} \\
      \frac{2}{3 }& {k=1} 
   \end{cases}
   \\
p_{Y|X}\brak{0|0} = \frac{19}{25}\, 
p_{Y|X}\brak{0|1} = \frac{6}{25}\,
p_{Y|X}\brak{1|0} = \frac{45}{50}\,
p_{Y|X}\brak{1|2} = \frac{5}{50}
\end{align}
The desired probability is the probability that a slip drawn at random is marked other than Rs 1,
\begin{align}
&=1-p_X\brak{0}\\
&= p_X(1) + p_X(2)
\end{align}
Using Bayes theorem,
\begin{align}
&= p_Y\brak{0} \times \pr{Y=0 | X=1} + p_Y\brak{1} \times \pr{Y=1|X=2}\\
&=\frac{1}{3} \times \frac{6}{25} + \frac{2}{3} \times \frac{5}{50}\\
&=\frac{11}{75}
\end{align}

\newpage

%\tableofcontents

\bigskip

\renewcommand{\thefigure}{\theenumi}
\renewcommand{\thetable}{\theenumi}
%\renewcommand{\theequation}{\theenumi}

%\begin{abstract}
%%\boldmath
%In this letter, an algorithm for evaluating the exact analytical bit error rate  (BER)  for the piecewise linear (PL) combiner for  multiple relays is presented. Previous results were available only for upto three relays. The algorithm is unique in the sense that  the actual mathematical expressions, that are prohibitively large, need not be explicitly obtained. The diversity gain due to multiple relays is shown through plots of the analytical BER, well supported by simulations. 
%
%\end{abstract}
% IEEEtran.cls defaults to using nonbold math in the Abstract.
% This preserves the distinction between vectors and scalars. However,
% if the journal you are submitting to favors bold math in the abstract,
% then you can use LaTeX's standard command \boldmath at the very start
% of the abstract to achieve this. Many IEEE journals frown on math
% in the abstract anyway.

% Note that keywords are not normally used for peerreview papers.
%\begin{IEEEkeywords}
%Cooperative diversity, decode and forward, piecewise linear
%\end{IEEEkeywords}



% For peer review papers, you can put extra information on the cover
% page as needed:
% \ifCLASSOPTIONpeerreview
% \begin{center} \bfseries EDICS Category: 3-BBND \end{center}
% \fi
%
% For peerreview papers, this IEEEtran command inserts a page break and
% creates the second title. It will be ignored for other modes.
%\IEEEpeerreviewmaketitle




\item A card is selected from a pack of 52 cards.
\begin{enumerate}[label=(\alph*)]
    \item How many points are there in the sample space?
    \item Calculate the probability that the card is an ace of spades.
    \item Calculate the probability that the card is (i) an ace and (ii) black card.
\end{enumerate}
\solution
%Let $X$ be an bernoulli rv defined as in \tabref{tab:exemplar/11/16/3/26}.  Then, 
\begin{equation}
    p =
        \frac{4}{11} 
\end{equation}
\begin{table}[H]
	\centering
	\input{exemplar/11/16/3/26/tables/Table2.tex}
	\caption{}
        \label{tab:exemplar/11/16/3/26}
\end{table}

\item The probability that a non leap year selected at random will contain 53 sundays.
\\
\solution
%\begin{table}[H]
	\centering
\begin{tabular}{|c|c|c|}
\hline
Random variable &Value &Definition\\ \hline
\multirow{3}{*}{X} &0 &Slips of Rs 1\\
&1 &Slips of Rs 5\\
&2 &Slips of Rs 13\\ \hline
\multirow{2}{*}{Y} &0 &Box A\\
&1 &Box B\\\hline
\end{tabular}
\caption{}
\label{tab:Distribution}
\end{table}
See \tabref{tab:Distribution}.
\begin{align}
p_{Y}\brak{k}= \begin{cases} 
      \frac{1}{3} & {k=0} \\
      \frac{2}{3 }& {k=1} 
   \end{cases}
   \\
p_{Y|X}\brak{0|0} = \frac{19}{25}\, 
p_{Y|X}\brak{0|1} = \frac{6}{25}\,
p_{Y|X}\brak{1|0} = \frac{45}{50}\,
p_{Y|X}\brak{1|2} = \frac{5}{50}
\end{align}
The desired probability is the probability that a slip drawn at random is marked other than Rs 1,
\begin{align}
&=1-p_X\brak{0}\\
&= p_X(1) + p_X(2)
\end{align}
Using Bayes theorem,
\begin{align}
&= p_Y\brak{0} \times \pr{Y=0 | X=1} + p_Y\brak{1} \times \pr{Y=1|X=2}\\
&=\frac{1}{3} \times \frac{6}{25} + \frac{2}{3} \times \frac{5}{50}\\
&=\frac{11}{75}
\end{align}

\newpage

%\tableofcontents

\bigskip

\renewcommand{\thefigure}{\theenumi}
\renewcommand{\thetable}{\theenumi}
%\renewcommand{\theequation}{\theenumi}

%\begin{abstract}
%%\boldmath
%In this letter, an algorithm for evaluating the exact analytical bit error rate  (BER)  for the piecewise linear (PL) combiner for  multiple relays is presented. Previous results were available only for upto three relays. The algorithm is unique in the sense that  the actual mathematical expressions, that are prohibitively large, need not be explicitly obtained. The diversity gain due to multiple relays is shown through plots of the analytical BER, well supported by simulations. 
%
%\end{abstract}
% IEEEtran.cls defaults to using nonbold math in the Abstract.
% This preserves the distinction between vectors and scalars. However,
% if the journal you are submitting to favors bold math in the abstract,
% then you can use LaTeX's standard command \boldmath at the very start
% of the abstract to achieve this. Many IEEE journals frown on math
% in the abstract anyway.

% Note that keywords are not normally used for peerreview papers.
%\begin{IEEEkeywords}
%Cooperative diversity, decode and forward, piecewise linear
%\end{IEEEkeywords}



% For peer review papers, you can put extra information on the cover
% page as needed:
% \ifCLASSOPTIONpeerreview
% \begin{center} \bfseries EDICS Category: 3-BBND \end{center}
% \fi
%
% For peerreview papers, this IEEEtran command inserts a page break and
% creates the second title. It will be ignored for other modes.
%\IEEEpeerreviewmaketitle




\item One of the four persons John, Rita, Aslam or Gurpreet will be promoted next
month. Consequently the sample space consists of four elementary outcomes
S = {John promoted, Rita promoted, Aslam promoted, Gurpreet promoted}
You are told that the chances of John’s promotion is same as that of Gurpreet,
Rita’s chances of promotion are twice as likely as Johns. Aslam’s chances are
four times that of John.
\begin{enumerate}
	\item Determine
	\begin{enumerate}
		\item P (John promoted)
		\item P (Rita promoted)
		\item P (Aslam promoted)
		\item P (Gurpreet promoted)
	\end{enumerate}
	\item If A = {John promoted or Gurpreet promoted}, find P (A).
\end{enumerate}
\solution
%\begin{table}[H]
	\centering
\begin{tabular}{|c|c|c|}
\hline
Random variable &Value &Definition\\ \hline
\multirow{3}{*}{X} &0 &Slips of Rs 1\\
&1 &Slips of Rs 5\\
&2 &Slips of Rs 13\\ \hline
\multirow{2}{*}{Y} &0 &Box A\\
&1 &Box B\\\hline
\end{tabular}
\caption{}
\label{tab:Distribution}
\end{table}
See \tabref{tab:Distribution}.
\begin{align}
p_{Y}\brak{k}= \begin{cases} 
      \frac{1}{3} & {k=0} \\
      \frac{2}{3 }& {k=1} 
   \end{cases}
   \\
p_{Y|X}\brak{0|0} = \frac{19}{25}\, 
p_{Y|X}\brak{0|1} = \frac{6}{25}\,
p_{Y|X}\brak{1|0} = \frac{45}{50}\,
p_{Y|X}\brak{1|2} = \frac{5}{50}
\end{align}
The desired probability is the probability that a slip drawn at random is marked other than Rs 1,
\begin{align}
&=1-p_X\brak{0}\\
&= p_X(1) + p_X(2)
\end{align}
Using Bayes theorem,
\begin{align}
&= p_Y\brak{0} \times \pr{Y=0 | X=1} + p_Y\brak{1} \times \pr{Y=1|X=2}\\
&=\frac{1}{3} \times \frac{6}{25} + \frac{2}{3} \times \frac{5}{50}\\
&=\frac{11}{75}
\end{align}

\newpage

%\tableofcontents

\bigskip

\renewcommand{\thefigure}{\theenumi}
\renewcommand{\thetable}{\theenumi}
%\renewcommand{\theequation}{\theenumi}

%\begin{abstract}
%%\boldmath
%In this letter, an algorithm for evaluating the exact analytical bit error rate  (BER)  for the piecewise linear (PL) combiner for  multiple relays is presented. Previous results were available only for upto three relays. The algorithm is unique in the sense that  the actual mathematical expressions, that are prohibitively large, need not be explicitly obtained. The diversity gain due to multiple relays is shown through plots of the analytical BER, well supported by simulations. 
%
%\end{abstract}
% IEEEtran.cls defaults to using nonbold math in the Abstract.
% This preserves the distinction between vectors and scalars. However,
% if the journal you are submitting to favors bold math in the abstract,
% then you can use LaTeX's standard command \boldmath at the very start
% of the abstract to achieve this. Many IEEE journals frown on math
% in the abstract anyway.

% Note that keywords are not normally used for peerreview papers.
%\begin{IEEEkeywords}
%Cooperative diversity, decode and forward, piecewise linear
%\end{IEEEkeywords}



% For peer review papers, you can put extra information on the cover
% page as needed:
% \ifCLASSOPTIONpeerreview
% \begin{center} \bfseries EDICS Category: 3-BBND \end{center}
% \fi
%
% For peerreview papers, this IEEEtran command inserts a page break and
% creates the second title. It will be ignored for other modes.
%\IEEEpeerreviewmaketitle




\item A card is drawn from a deck of 52 cards. Find the probability of getting a king or a heart or a red card.\\
\solution
%\begin{table}[H]
	\centering
\begin{tabular}{|c|c|c|}
\hline
Random variable &Value &Definition\\ \hline
\multirow{3}{*}{X} &0 &Slips of Rs 1\\
&1 &Slips of Rs 5\\
&2 &Slips of Rs 13\\ \hline
\multirow{2}{*}{Y} &0 &Box A\\
&1 &Box B\\\hline
\end{tabular}
\caption{}
\label{tab:Distribution}
\end{table}
See \tabref{tab:Distribution}.
\begin{align}
p_{Y}\brak{k}= \begin{cases} 
      \frac{1}{3} & {k=0} \\
      \frac{2}{3 }& {k=1} 
   \end{cases}
   \\
p_{Y|X}\brak{0|0} = \frac{19}{25}\, 
p_{Y|X}\brak{0|1} = \frac{6}{25}\,
p_{Y|X}\brak{1|0} = \frac{45}{50}\,
p_{Y|X}\brak{1|2} = \frac{5}{50}
\end{align}
The desired probability is the probability that a slip drawn at random is marked other than Rs 1,
\begin{align}
&=1-p_X\brak{0}\\
&= p_X(1) + p_X(2)
\end{align}
Using Bayes theorem,
\begin{align}
&= p_Y\brak{0} \times \pr{Y=0 | X=1} + p_Y\brak{1} \times \pr{Y=1|X=2}\\
&=\frac{1}{3} \times \frac{6}{25} + \frac{2}{3} \times \frac{5}{50}\\
&=\frac{11}{75}
\end{align}

\newpage

%\tableofcontents

\bigskip

\renewcommand{\thefigure}{\theenumi}
\renewcommand{\thetable}{\theenumi}
%\renewcommand{\theequation}{\theenumi}

%\begin{abstract}
%%\boldmath
%In this letter, an algorithm for evaluating the exact analytical bit error rate  (BER)  for the piecewise linear (PL) combiner for  multiple relays is presented. Previous results were available only for upto three relays. The algorithm is unique in the sense that  the actual mathematical expressions, that are prohibitively large, need not be explicitly obtained. The diversity gain due to multiple relays is shown through plots of the analytical BER, well supported by simulations. 
%
%\end{abstract}
% IEEEtran.cls defaults to using nonbold math in the Abstract.
% This preserves the distinction between vectors and scalars. However,
% if the journal you are submitting to favors bold math in the abstract,
% then you can use LaTeX's standard command \boldmath at the very start
% of the abstract to achieve this. Many IEEE journals frown on math
% in the abstract anyway.

% Note that keywords are not normally used for peerreview papers.
%\begin{IEEEkeywords}
%Cooperative diversity, decode and forward, piecewise linear
%\end{IEEEkeywords}



% For peer review papers, you can put extra information on the cover
% page as needed:
% \ifCLASSOPTIONpeerreview
% \begin{center} \bfseries EDICS Category: 3-BBND \end{center}
% \fi
%
% For peerreview papers, this IEEEtran command inserts a page break and
% creates the second title. It will be ignored for other modes.
%\IEEEpeerreviewmaketitle




\item The probability that a student will pass his examination is 0.73, the probability of
the student getting a compartment is 0.13, and the probability that the student will
either pass or get compartment is 0.96. State True or False.\\
\solution
%\begin{table}[H]
	\centering
\begin{tabular}{|c|c|c|}
\hline
Random variable &Value &Definition\\ \hline
\multirow{3}{*}{X} &0 &Slips of Rs 1\\
&1 &Slips of Rs 5\\
&2 &Slips of Rs 13\\ \hline
\multirow{2}{*}{Y} &0 &Box A\\
&1 &Box B\\\hline
\end{tabular}
\caption{}
\label{tab:Distribution}
\end{table}
See \tabref{tab:Distribution}.
\begin{align}
p_{Y}\brak{k}= \begin{cases} 
      \frac{1}{3} & {k=0} \\
      \frac{2}{3 }& {k=1} 
   \end{cases}
   \\
p_{Y|X}\brak{0|0} = \frac{19}{25}\, 
p_{Y|X}\brak{0|1} = \frac{6}{25}\,
p_{Y|X}\brak{1|0} = \frac{45}{50}\,
p_{Y|X}\brak{1|2} = \frac{5}{50}
\end{align}
The desired probability is the probability that a slip drawn at random is marked other than Rs 1,
\begin{align}
&=1-p_X\brak{0}\\
&= p_X(1) + p_X(2)
\end{align}
Using Bayes theorem,
\begin{align}
&= p_Y\brak{0} \times \pr{Y=0 | X=1} + p_Y\brak{1} \times \pr{Y=1|X=2}\\
&=\frac{1}{3} \times \frac{6}{25} + \frac{2}{3} \times \frac{5}{50}\\
&=\frac{11}{75}
\end{align}

\newpage

%\tableofcontents

\bigskip

\renewcommand{\thefigure}{\theenumi}
\renewcommand{\thetable}{\theenumi}
%\renewcommand{\theequation}{\theenumi}

%\begin{abstract}
%%\boldmath
%In this letter, an algorithm for evaluating the exact analytical bit error rate  (BER)  for the piecewise linear (PL) combiner for  multiple relays is presented. Previous results were available only for upto three relays. The algorithm is unique in the sense that  the actual mathematical expressions, that are prohibitively large, need not be explicitly obtained. The diversity gain due to multiple relays is shown through plots of the analytical BER, well supported by simulations. 
%
%\end{abstract}
% IEEEtran.cls defaults to using nonbold math in the Abstract.
% This preserves the distinction between vectors and scalars. However,
% if the journal you are submitting to favors bold math in the abstract,
% then you can use LaTeX's standard command \boldmath at the very start
% of the abstract to achieve this. Many IEEE journals frown on math
% in the abstract anyway.

% Note that keywords are not normally used for peerreview papers.
%\begin{IEEEkeywords}
%Cooperative diversity, decode and forward, piecewise linear
%\end{IEEEkeywords}



% For peer review papers, you can put extra information on the cover
% page as needed:
% \ifCLASSOPTIONpeerreview
% \begin{center} \bfseries EDICS Category: 3-BBND \end{center}
% \fi
%
% For peerreview papers, this IEEEtran command inserts a page break and
% creates the second title. It will be ignored for other modes.
%\IEEEpeerreviewmaketitle




\item A card is selected from a pack of 52 cards\\
\begin{enumerate}[label=(\alph*)]
\item How many points are there in the sample space?
\item Calculate the probability that the cards is an ace of spades.
\item Calculate the probability that the card is (i) an ace (ii)black card.\\
\end{enumerate}
%\input{ncert/11/16/3/4_1/Prob_4.tex}
\item In a non-leap year, the probability of having 53 tuesdays or 53 wednesdays is\\
\solution
%A non-leap year has a total of 365 days, and a week has 7 days.\\
So it can be expressed as 
\begin{align}
365\text{days} &=52\times 7+1 \text{day}
\end{align}
$\implies$ 52 tuesdays or wednesdays\\
Random variable X denotes the days of a week
\begin{align}
p_X\brak{k}&=\frac{1}{7}; \quad \brak{1<k<7}
\end{align}
So the probability of extra day being tuesday or wednesday is
\begin{align}
p_X\brak{3}+p_X\brak{4}&=\frac{1}{7}+\frac{1}{7}=\frac{2}{7}
\end{align}



\item There are 1000 sealed envelopes in a box, 10 of them contain a cash prize of
Rs 100 each, 100 of them contain a cash prize of Rs 50 each and 200 of them
contain a cash prize of Rs 10 each and rest do not contain any cash prize. If they
are well shuffled and an envelope is picked up out, what is the probability that it
contains no cash prize?\\
\solution
%\begin{table}[H]
	\centering
\begin{tabular}{|c|c|c|}
\hline
Random variable &Value &Definition\\ \hline
\multirow{3}{*}{X} &0 &Slips of Rs 1\\
&1 &Slips of Rs 5\\
&2 &Slips of Rs 13\\ \hline
\multirow{2}{*}{Y} &0 &Box A\\
&1 &Box B\\\hline
\end{tabular}
\caption{}
\label{tab:Distribution}
\end{table}
See \tabref{tab:Distribution}.
\begin{align}
p_{Y}\brak{k}= \begin{cases} 
      \frac{1}{3} & {k=0} \\
      \frac{2}{3 }& {k=1} 
   \end{cases}
   \\
p_{Y|X}\brak{0|0} = \frac{19}{25}\, 
p_{Y|X}\brak{0|1} = \frac{6}{25}\,
p_{Y|X}\brak{1|0} = \frac{45}{50}\,
p_{Y|X}\brak{1|2} = \frac{5}{50}
\end{align}
The desired probability is the probability that a slip drawn at random is marked other than Rs 1,
\begin{align}
&=1-p_X\brak{0}\\
&= p_X(1) + p_X(2)
\end{align}
Using Bayes theorem,
\begin{align}
&= p_Y\brak{0} \times \pr{Y=0 | X=1} + p_Y\brak{1} \times \pr{Y=1|X=2}\\
&=\frac{1}{3} \times \frac{6}{25} + \frac{2}{3} \times \frac{5}{50}\\
&=\frac{11}{75}
\end{align}

\newpage

%\tableofcontents

\bigskip

\renewcommand{\thefigure}{\theenumi}
\renewcommand{\thetable}{\theenumi}
%\renewcommand{\theequation}{\theenumi}

%\begin{abstract}
%%\boldmath
%In this letter, an algorithm for evaluating the exact analytical bit error rate  (BER)  for the piecewise linear (PL) combiner for  multiple relays is presented. Previous results were available only for upto three relays. The algorithm is unique in the sense that  the actual mathematical expressions, that are prohibitively large, need not be explicitly obtained. The diversity gain due to multiple relays is shown through plots of the analytical BER, well supported by simulations. 
%
%\end{abstract}
% IEEEtran.cls defaults to using nonbold math in the Abstract.
% This preserves the distinction between vectors and scalars. However,
% if the journal you are submitting to favors bold math in the abstract,
% then you can use LaTeX's standard command \boldmath at the very start
% of the abstract to achieve this. Many IEEE journals frown on math
% in the abstract anyway.

% Note that keywords are not normally used for peerreview papers.
%\begin{IEEEkeywords}
%Cooperative diversity, decode and forward, piecewise linear
%\end{IEEEkeywords}



% For peer review papers, you can put extra information on the cover
% page as needed:
% \ifCLASSOPTIONpeerreview
% \begin{center} \bfseries EDICS Category: 3-BBND \end{center}
% \fi
%
% For peerreview papers, this IEEEtran command inserts a page break and
% creates the second title. It will be ignored for other modes.
%\IEEEpeerreviewmaketitle




\item 
A die is thrown and a card is selected at random from a deck of 52 playing cards. The probability of getting an even number on the die and a spade card.\\
\solution
%\begin{table}[H]
	\centering
\begin{tabular}{|c|c|c|}
\hline
Random variable &Value &Definition\\ \hline
\multirow{3}{*}{X} &0 &Slips of Rs 1\\
&1 &Slips of Rs 5\\
&2 &Slips of Rs 13\\ \hline
\multirow{2}{*}{Y} &0 &Box A\\
&1 &Box B\\\hline
\end{tabular}
\caption{}
\label{tab:Distribution}
\end{table}
See \tabref{tab:Distribution}.
\begin{align}
p_{Y}\brak{k}= \begin{cases} 
      \frac{1}{3} & {k=0} \\
      \frac{2}{3 }& {k=1} 
   \end{cases}
   \\
p_{Y|X}\brak{0|0} = \frac{19}{25}\, 
p_{Y|X}\brak{0|1} = \frac{6}{25}\,
p_{Y|X}\brak{1|0} = \frac{45}{50}\,
p_{Y|X}\brak{1|2} = \frac{5}{50}
\end{align}
The desired probability is the probability that a slip drawn at random is marked other than Rs 1,
\begin{align}
&=1-p_X\brak{0}\\
&= p_X(1) + p_X(2)
\end{align}
Using Bayes theorem,
\begin{align}
&= p_Y\brak{0} \times \pr{Y=0 | X=1} + p_Y\brak{1} \times \pr{Y=1|X=2}\\
&=\frac{1}{3} \times \frac{6}{25} + \frac{2}{3} \times \frac{5}{50}\\
&=\frac{11}{75}
\end{align}

\newpage

%\tableofcontents

\bigskip

\renewcommand{\thefigure}{\theenumi}
\renewcommand{\thetable}{\theenumi}
%\renewcommand{\theequation}{\theenumi}

%\begin{abstract}
%%\boldmath
%In this letter, an algorithm for evaluating the exact analytical bit error rate  (BER)  for the piecewise linear (PL) combiner for  multiple relays is presented. Previous results were available only for upto three relays. The algorithm is unique in the sense that  the actual mathematical expressions, that are prohibitively large, need not be explicitly obtained. The diversity gain due to multiple relays is shown through plots of the analytical BER, well supported by simulations. 
%
%\end{abstract}
% IEEEtran.cls defaults to using nonbold math in the Abstract.
% This preserves the distinction between vectors and scalars. However,
% if the journal you are submitting to favors bold math in the abstract,
% then you can use LaTeX's standard command \boldmath at the very start
% of the abstract to achieve this. Many IEEE journals frown on math
% in the abstract anyway.

% Note that keywords are not normally used for peerreview papers.
%\begin{IEEEkeywords}
%Cooperative diversity, decode and forward, piecewise linear
%\end{IEEEkeywords}



% For peer review papers, you can put extra information on the cover
% page as needed:
% \ifCLASSOPTIONpeerreview
% \begin{center} \bfseries EDICS Category: 3-BBND \end{center}
% \fi
%
% For peerreview papers, this IEEEtran command inserts a page break and
% creates the second title. It will be ignored for other modes.
%\IEEEpeerreviewmaketitle




\item
If 4-digit numbers greater than 5,000 are randomly formed from the digits 0, 1, 3, 5, and 7, what is the probability of forming a number divisible by 5 when:
\begin{enumerate}
    \item The digits are repeated?
    \item The repetition of digits is not allowed?
\end{enumerate}
\solution
%\begin{table}[H]
	\centering
\begin{tabular}{|c|c|c|}
\hline
Random variable &Value &Definition\\ \hline
\multirow{3}{*}{X} &0 &Slips of Rs 1\\
&1 &Slips of Rs 5\\
&2 &Slips of Rs 13\\ \hline
\multirow{2}{*}{Y} &0 &Box A\\
&1 &Box B\\\hline
\end{tabular}
\caption{}
\label{tab:Distribution}
\end{table}
See \tabref{tab:Distribution}.
\begin{align}
p_{Y}\brak{k}= \begin{cases} 
      \frac{1}{3} & {k=0} \\
      \frac{2}{3 }& {k=1} 
   \end{cases}
   \\
p_{Y|X}\brak{0|0} = \frac{19}{25}\, 
p_{Y|X}\brak{0|1} = \frac{6}{25}\,
p_{Y|X}\brak{1|0} = \frac{45}{50}\,
p_{Y|X}\brak{1|2} = \frac{5}{50}
\end{align}
The desired probability is the probability that a slip drawn at random is marked other than Rs 1,
\begin{align}
&=1-p_X\brak{0}\\
&= p_X(1) + p_X(2)
\end{align}
Using Bayes theorem,
\begin{align}
&= p_Y\brak{0} \times \pr{Y=0 | X=1} + p_Y\brak{1} \times \pr{Y=1|X=2}\\
&=\frac{1}{3} \times \frac{6}{25} + \frac{2}{3} \times \frac{5}{50}\\
&=\frac{11}{75}
\end{align}

\newpage

%\tableofcontents

\bigskip

\renewcommand{\thefigure}{\theenumi}
\renewcommand{\thetable}{\theenumi}
%\renewcommand{\theequation}{\theenumi}

%\begin{abstract}
%%\boldmath
%In this letter, an algorithm for evaluating the exact analytical bit error rate  (BER)  for the piecewise linear (PL) combiner for  multiple relays is presented. Previous results were available only for upto three relays. The algorithm is unique in the sense that  the actual mathematical expressions, that are prohibitively large, need not be explicitly obtained. The diversity gain due to multiple relays is shown through plots of the analytical BER, well supported by simulations. 
%
%\end{abstract}
% IEEEtran.cls defaults to using nonbold math in the Abstract.
% This preserves the distinction between vectors and scalars. However,
% if the journal you are submitting to favors bold math in the abstract,
% then you can use LaTeX's standard command \boldmath at the very start
% of the abstract to achieve this. Many IEEE journals frown on math
% in the abstract anyway.

% Note that keywords are not normally used for peerreview papers.
%\begin{IEEEkeywords}
%Cooperative diversity, decode and forward, piecewise linear
%\end{IEEEkeywords}



% For peer review papers, you can put extra information on the cover
% page as needed:
% \ifCLASSOPTIONpeerreview
% \begin{center} \bfseries EDICS Category: 3-BBND \end{center}
% \fi
%
% For peerreview papers, this IEEEtran command inserts a page break and
% creates the second title. It will be ignored for other modes.
%\IEEEpeerreviewmaketitle




\item Consider the probability space $\brak{\Omega, \mathcal{G}, P}$ where $\Omega = [0,2]$ and $\mathcal{G} = \cbrak{\phi, \Omega, [0,1], (1,2]}$. Let $X$ and $Y$ be two functions on $\Omega$ defined as
\begin{align*}
    X(\omega) = 
    \begin{cases}
        1 & \text{if }\omega \in [0, 1]\\
        2 & \text{if }\omega \in (1, 2]
    \end{cases}
\end{align*}
and
\begin{align*}
    Y(\omega) = 
    \begin{cases}
        2 & \text{if }\omega \in [0, 1.5]\\
        3 & \text{if }\omega \in (1.5, 2].
    \end{cases}
\end{align*}
Then which one of the following statements is true?
\begin{enumerate}
    \item [(A)] $X$ is a random variable with respect to $\mathcal{G}$, but $Y$ is not a random variable with respect to $\mathcal{G}$.
    \item [(B)] $Y$ is a random variable with respect to $\mathcal{G}$, but $X$ is not a random variable with respect to $\mathcal{G}$.
    \item [(C)] Neither $X$ nor $Y$ is a random variable with respect to $\mathcal{G}$.
    \item [(D)] Both $X$ and $Y$ are random variables with respect to $\mathcal{G}$.
\end{enumerate} \hfill (GATE ST 2023)\\
\solution
%\begin{table}[H]
	\centering
\begin{tabular}{|c|c|c|}
\hline
Random variable &Value &Definition\\ \hline
\multirow{3}{*}{X} &0 &Slips of Rs 1\\
&1 &Slips of Rs 5\\
&2 &Slips of Rs 13\\ \hline
\multirow{2}{*}{Y} &0 &Box A\\
&1 &Box B\\\hline
\end{tabular}
\caption{}
\label{tab:Distribution}
\end{table}
See \tabref{tab:Distribution}.
\begin{align}
p_{Y}\brak{k}= \begin{cases} 
      \frac{1}{3} & {k=0} \\
      \frac{2}{3 }& {k=1} 
   \end{cases}
   \\
p_{Y|X}\brak{0|0} = \frac{19}{25}\, 
p_{Y|X}\brak{0|1} = \frac{6}{25}\,
p_{Y|X}\brak{1|0} = \frac{45}{50}\,
p_{Y|X}\brak{1|2} = \frac{5}{50}
\end{align}
The desired probability is the probability that a slip drawn at random is marked other than Rs 1,
\begin{align}
&=1-p_X\brak{0}\\
&= p_X(1) + p_X(2)
\end{align}
Using Bayes theorem,
\begin{align}
&= p_Y\brak{0} \times \pr{Y=0 | X=1} + p_Y\brak{1} \times \pr{Y=1|X=2}\\
&=\frac{1}{3} \times \frac{6}{25} + \frac{2}{3} \times \frac{5}{50}\\
&=\frac{11}{75}
\end{align}

\newpage

%\tableofcontents

\bigskip

\renewcommand{\thefigure}{\theenumi}
\renewcommand{\thetable}{\theenumi}
%\renewcommand{\theequation}{\theenumi}

%\begin{abstract}
%%\boldmath
%In this letter, an algorithm for evaluating the exact analytical bit error rate  (BER)  for the piecewise linear (PL) combiner for  multiple relays is presented. Previous results were available only for upto three relays. The algorithm is unique in the sense that  the actual mathematical expressions, that are prohibitively large, need not be explicitly obtained. The diversity gain due to multiple relays is shown through plots of the analytical BER, well supported by simulations. 
%
%\end{abstract}
% IEEEtran.cls defaults to using nonbold math in the Abstract.
% This preserves the distinction between vectors and scalars. However,
% if the journal you are submitting to favors bold math in the abstract,
% then you can use LaTeX's standard command \boldmath at the very start
% of the abstract to achieve this. Many IEEE journals frown on math
% in the abstract anyway.

% Note that keywords are not normally used for peerreview papers.
%\begin{IEEEkeywords}
%Cooperative diversity, decode and forward, piecewise linear
%\end{IEEEkeywords}



% For peer review papers, you can put extra information on the cover
% page as needed:
% \ifCLASSOPTIONpeerreview
% \begin{center} \bfseries EDICS Category: 3-BBND \end{center}
% \fi
%
% For peerreview papers, this IEEEtran command inserts a page break and
% creates the second title. It will be ignored for other modes.
%\IEEEpeerreviewmaketitle




	\item  A die is loaded in such a way that each odd number is twice as likely to occur as
each even number. Find $P(G)$, where $G$ is the event that a number greater than
3 occurs on a single roll of the die.
\\
\solution
		%\begin{table}[H]
	\centering
\begin{tabular}{|c|c|c|}
\hline
Random variable &Value &Definition\\ \hline
\multirow{3}{*}{X} &0 &Slips of Rs 1\\
&1 &Slips of Rs 5\\
&2 &Slips of Rs 13\\ \hline
\multirow{2}{*}{Y} &0 &Box A\\
&1 &Box B\\\hline
\end{tabular}
\caption{}
\label{tab:Distribution}
\end{table}
See \tabref{tab:Distribution}.
\begin{align}
p_{Y}\brak{k}= \begin{cases} 
      \frac{1}{3} & {k=0} \\
      \frac{2}{3 }& {k=1} 
   \end{cases}
   \\
p_{Y|X}\brak{0|0} = \frac{19}{25}\, 
p_{Y|X}\brak{0|1} = \frac{6}{25}\,
p_{Y|X}\brak{1|0} = \frac{45}{50}\,
p_{Y|X}\brak{1|2} = \frac{5}{50}
\end{align}
The desired probability is the probability that a slip drawn at random is marked other than Rs 1,
\begin{align}
&=1-p_X\brak{0}\\
&= p_X(1) + p_X(2)
\end{align}
Using Bayes theorem,
\begin{align}
&= p_Y\brak{0} \times \pr{Y=0 | X=1} + p_Y\brak{1} \times \pr{Y=1|X=2}\\
&=\frac{1}{3} \times \frac{6}{25} + \frac{2}{3} \times \frac{5}{50}\\
&=\frac{11}{75}
\end{align}

\newpage

%\tableofcontents

\bigskip

\renewcommand{\thefigure}{\theenumi}
\renewcommand{\thetable}{\theenumi}
%\renewcommand{\theequation}{\theenumi}

%\begin{abstract}
%%\boldmath
%In this letter, an algorithm for evaluating the exact analytical bit error rate  (BER)  for the piecewise linear (PL) combiner for  multiple relays is presented. Previous results were available only for upto three relays. The algorithm is unique in the sense that  the actual mathematical expressions, that are prohibitively large, need not be explicitly obtained. The diversity gain due to multiple relays is shown through plots of the analytical BER, well supported by simulations. 
%
%\end{abstract}
% IEEEtran.cls defaults to using nonbold math in the Abstract.
% This preserves the distinction between vectors and scalars. However,
% if the journal you are submitting to favors bold math in the abstract,
% then you can use LaTeX's standard command \boldmath at the very start
% of the abstract to achieve this. Many IEEE journals frown on math
% in the abstract anyway.

% Note that keywords are not normally used for peerreview papers.
%\begin{IEEEkeywords}
%Cooperative diversity, decode and forward, piecewise linear
%\end{IEEEkeywords}



% For peer review papers, you can put extra information on the cover
% page as needed:
% \ifCLASSOPTIONpeerreview
% \begin{center} \bfseries EDICS Category: 3-BBND \end{center}
% \fi
%
% For peerreview papers, this IEEEtran command inserts a page break and
% creates the second title. It will be ignored for other modes.
%\IEEEpeerreviewmaketitle




	\item All the jacks, queens and kings are removed from a deck of 52 playing cards. The remaining cards are well shuffled and then one card is drawn at random. Giving ace a value 1 similar value for other cards, find the probability that the card has a value 
		\begin{enumerate}
			\item 7
			\item greater than 7
			\item less than 7
		\end{enumerate}
		%Number of cards left after removing all jacks, queens and kings 
\begin{align}
N	= 52 - 4\times 3
	= 40
\end{align}
%\begin{table}[H]
%\def\arraystretch{1.2}
%\begin{tabular}{|c|c|c|}
%\hline
%	\textbf{Parameter} &\textbf{Value} &\textbf{Description}\\ \hline
%	$X$ &1-10 &Represents the value of the card picked \\ \hline
%\end{tabular}
%\end{table}
Let $1 \le X \le 10$ be the value of the card picked.  Then,
\begin{align}
	p_X(k) &= \Pr(X=k)\ \forall\ 1 \leq k \leq 10\\
	&= \frac{4\times 1}{40}\\
	&= \frac{1}{10}\\
	\therefore p_X(k) &= 
	\begin{cases}
		\frac{1}{10} & 1 \leq k \leq 10\\
		0 & \text{otherwise}
	\end{cases}
\end{align}
and
\begin{align}
	F_{X}(k) &= \sum_{m=0}^{k}p_{X}(m) \quad 1 \leq k \leq 10\\
	&= \frac{k}{10}\\
	\therefore F_{X}(k) &= 
	\begin{cases}
		0 & k \leq 0\\
		\frac{k}{10} & 1\leq k \leq 10\\
		1 & k > 10 
	\end{cases}
\end{align}
\begin{enumerate}
	\item Probability that card has value equal to 7 is
		\begin{align}
			 p_{X}(7)
			= \frac{1}{10}
		\end{align}
	\item Probability that card has value greater than 7 is
		\begin{align}
			1 - F_X(7)
			&= 1 - \frac{7}{10}
			\\
			&= \frac{3}{10}
		\end{align}
	\item Probability that card has value less than 7 is
		\begin{align}
			 F_{X}(6)
			=\frac{6}{10}
		\end{align}
\end{enumerate}

  \item A Lot consists of 48 mobile phones of which 42 are good, 3 have only minor defects and 3 have major defects.Varnika will buy a phone if it is good but the trader will only buy a mobile if it has no major defects. One phone is selected at random from the lot. What is the probability that it is
\begin{enumerate}
	\item acceptable to Varnika?
            \item acceptable to the trader?
\end{enumerate}
\solution
	%\begin{table}[H]
	\centering
\begin{tabular}{|c|c|c|}
\hline
Random variable &Value &Definition\\ \hline
\multirow{3}{*}{X} &0 &Slips of Rs 1\\
&1 &Slips of Rs 5\\
&2 &Slips of Rs 13\\ \hline
\multirow{2}{*}{Y} &0 &Box A\\
&1 &Box B\\\hline
\end{tabular}
\caption{}
\label{tab:Distribution}
\end{table}
See \tabref{tab:Distribution}.
\begin{align}
p_{Y}\brak{k}= \begin{cases} 
      \frac{1}{3} & {k=0} \\
      \frac{2}{3 }& {k=1} 
   \end{cases}
   \\
p_{Y|X}\brak{0|0} = \frac{19}{25}\, 
p_{Y|X}\brak{0|1} = \frac{6}{25}\,
p_{Y|X}\brak{1|0} = \frac{45}{50}\,
p_{Y|X}\brak{1|2} = \frac{5}{50}
\end{align}
The desired probability is the probability that a slip drawn at random is marked other than Rs 1,
\begin{align}
&=1-p_X\brak{0}\\
&= p_X(1) + p_X(2)
\end{align}
Using Bayes theorem,
\begin{align}
&= p_Y\brak{0} \times \pr{Y=0 | X=1} + p_Y\brak{1} \times \pr{Y=1|X=2}\\
&=\frac{1}{3} \times \frac{6}{25} + \frac{2}{3} \times \frac{5}{50}\\
&=\frac{11}{75}
\end{align}

\newpage

%\tableofcontents

\bigskip

\renewcommand{\thefigure}{\theenumi}
\renewcommand{\thetable}{\theenumi}
%\renewcommand{\theequation}{\theenumi}

%\begin{abstract}
%%\boldmath
%In this letter, an algorithm for evaluating the exact analytical bit error rate  (BER)  for the piecewise linear (PL) combiner for  multiple relays is presented. Previous results were available only for upto three relays. The algorithm is unique in the sense that  the actual mathematical expressions, that are prohibitively large, need not be explicitly obtained. The diversity gain due to multiple relays is shown through plots of the analytical BER, well supported by simulations. 
%
%\end{abstract}
% IEEEtran.cls defaults to using nonbold math in the Abstract.
% This preserves the distinction between vectors and scalars. However,
% if the journal you are submitting to favors bold math in the abstract,
% then you can use LaTeX's standard command \boldmath at the very start
% of the abstract to achieve this. Many IEEE journals frown on math
% in the abstract anyway.

% Note that keywords are not normally used for peerreview papers.
%\begin{IEEEkeywords}
%Cooperative diversity, decode and forward, piecewise linear
%\end{IEEEkeywords}



% For peer review papers, you can put extra information on the cover
% page as needed:
% \ifCLASSOPTIONpeerreview
% \begin{center} \bfseries EDICS Category: 3-BBND \end{center}
% \fi
%
% For peerreview papers, this IEEEtran command inserts a page break and
% creates the second title. It will be ignored for other modes.
%\IEEEpeerreviewmaketitle




 \item A student says that if you throw a die, it will show up 1 or not 1. Therefore, the probability of getting 1 and the probability of getting 'not 1' each is equal to $\frac{1}{2}$. Is this correct? Give reasons.\\
 \solution
        %\begin{table}[H]
	\centering
\begin{tabular}{|c|c|c|}
\hline
Random variable &Value &Definition\\ \hline
\multirow{3}{*}{X} &0 &Slips of Rs 1\\
&1 &Slips of Rs 5\\
&2 &Slips of Rs 13\\ \hline
\multirow{2}{*}{Y} &0 &Box A\\
&1 &Box B\\\hline
\end{tabular}
\caption{}
\label{tab:Distribution}
\end{table}
See \tabref{tab:Distribution}.
\begin{align}
p_{Y}\brak{k}= \begin{cases} 
      \frac{1}{3} & {k=0} \\
      \frac{2}{3 }& {k=1} 
   \end{cases}
   \\
p_{Y|X}\brak{0|0} = \frac{19}{25}\, 
p_{Y|X}\brak{0|1} = \frac{6}{25}\,
p_{Y|X}\brak{1|0} = \frac{45}{50}\,
p_{Y|X}\brak{1|2} = \frac{5}{50}
\end{align}
The desired probability is the probability that a slip drawn at random is marked other than Rs 1,
\begin{align}
&=1-p_X\brak{0}\\
&= p_X(1) + p_X(2)
\end{align}
Using Bayes theorem,
\begin{align}
&= p_Y\brak{0} \times \pr{Y=0 | X=1} + p_Y\brak{1} \times \pr{Y=1|X=2}\\
&=\frac{1}{3} \times \frac{6}{25} + \frac{2}{3} \times \frac{5}{50}\\
&=\frac{11}{75}
\end{align}

\newpage

%\tableofcontents

\bigskip

\renewcommand{\thefigure}{\theenumi}
\renewcommand{\thetable}{\theenumi}
%\renewcommand{\theequation}{\theenumi}

%\begin{abstract}
%%\boldmath
%In this letter, an algorithm for evaluating the exact analytical bit error rate  (BER)  for the piecewise linear (PL) combiner for  multiple relays is presented. Previous results were available only for upto three relays. The algorithm is unique in the sense that  the actual mathematical expressions, that are prohibitively large, need not be explicitly obtained. The diversity gain due to multiple relays is shown through plots of the analytical BER, well supported by simulations. 
%
%\end{abstract}
% IEEEtran.cls defaults to using nonbold math in the Abstract.
% This preserves the distinction between vectors and scalars. However,
% if the journal you are submitting to favors bold math in the abstract,
% then you can use LaTeX's standard command \boldmath at the very start
% of the abstract to achieve this. Many IEEE journals frown on math
% in the abstract anyway.

% Note that keywords are not normally used for peerreview papers.
%\begin{IEEEkeywords}
%Cooperative diversity, decode and forward, piecewise linear
%\end{IEEEkeywords}



% For peer review papers, you can put extra information on the cover
% page as needed:
% \ifCLASSOPTIONpeerreview
% \begin{center} \bfseries EDICS Category: 3-BBND \end{center}
% \fi
%
% For peerreview papers, this IEEEtran command inserts a page break and
% creates the second title. It will be ignored for other modes.
%\IEEEpeerreviewmaketitle




   \item Four candidates A, B, C, D have ap-
plied for the assignment to coach a school cricket
team. If A is twice as likely to be selected as B, and
B and C are given about the same chance of being
selected, while C is twice as likely to be selected
as D, what are the probabilities that
\begin{enumerate}
\item C will be selected?
\item A will not be selected?
\end{enumerate}
	%\begin{table}[H]
	\centering
\begin{tabular}{|c|c|c|}
\hline
Random variable &Value &Definition\\ \hline
\multirow{3}{*}{X} &0 &Slips of Rs 1\\
&1 &Slips of Rs 5\\
&2 &Slips of Rs 13\\ \hline
\multirow{2}{*}{Y} &0 &Box A\\
&1 &Box B\\\hline
\end{tabular}
\caption{}
\label{tab:Distribution}
\end{table}
See \tabref{tab:Distribution}.
\begin{align}
p_{Y}\brak{k}= \begin{cases} 
      \frac{1}{3} & {k=0} \\
      \frac{2}{3 }& {k=1} 
   \end{cases}
   \\
p_{Y|X}\brak{0|0} = \frac{19}{25}\, 
p_{Y|X}\brak{0|1} = \frac{6}{25}\,
p_{Y|X}\brak{1|0} = \frac{45}{50}\,
p_{Y|X}\brak{1|2} = \frac{5}{50}
\end{align}
The desired probability is the probability that a slip drawn at random is marked other than Rs 1,
\begin{align}
&=1-p_X\brak{0}\\
&= p_X(1) + p_X(2)
\end{align}
Using Bayes theorem,
\begin{align}
&= p_Y\brak{0} \times \pr{Y=0 | X=1} + p_Y\brak{1} \times \pr{Y=1|X=2}\\
&=\frac{1}{3} \times \frac{6}{25} + \frac{2}{3} \times \frac{5}{50}\\
&=\frac{11}{75}
\end{align}

\newpage

%\tableofcontents

\bigskip

\renewcommand{\thefigure}{\theenumi}
\renewcommand{\thetable}{\theenumi}
%\renewcommand{\theequation}{\theenumi}

%\begin{abstract}
%%\boldmath
%In this letter, an algorithm for evaluating the exact analytical bit error rate  (BER)  for the piecewise linear (PL) combiner for  multiple relays is presented. Previous results were available only for upto three relays. The algorithm is unique in the sense that  the actual mathematical expressions, that are prohibitively large, need not be explicitly obtained. The diversity gain due to multiple relays is shown through plots of the analytical BER, well supported by simulations. 
%
%\end{abstract}
% IEEEtran.cls defaults to using nonbold math in the Abstract.
% This preserves the distinction between vectors and scalars. However,
% if the journal you are submitting to favors bold math in the abstract,
% then you can use LaTeX's standard command \boldmath at the very start
% of the abstract to achieve this. Many IEEE journals frown on math
% in the abstract anyway.

% Note that keywords are not normally used for peerreview papers.
%\begin{IEEEkeywords}
%Cooperative diversity, decode and forward, piecewise linear
%\end{IEEEkeywords}



% For peer review papers, you can put extra information on the cover
% page as needed:
% \ifCLASSOPTIONpeerreview
% \begin{center} \bfseries EDICS Category: 3-BBND \end{center}
% \fi
%
% For peerreview papers, this IEEEtran command inserts a page break and
% creates the second title. It will be ignored for other modes.
%\IEEEpeerreviewmaketitle




 \item A bag contain 24 balls of which $x$ balls are red, $2x$ are white and $3x$ are blue. A ball is selected at random, What is the probability that it is
\begin{enumerate}[label=\alph*)]
\item not red ?
\item white ?
\end{enumerate}
%\begin{table}[H]
	\centering
\begin{tabular}{|c|c|c|}
\hline
Random variable &Value &Definition\\ \hline
\multirow{3}{*}{X} &0 &Slips of Rs 1\\
&1 &Slips of Rs 5\\
&2 &Slips of Rs 13\\ \hline
\multirow{2}{*}{Y} &0 &Box A\\
&1 &Box B\\\hline
\end{tabular}
\caption{}
\label{tab:Distribution}
\end{table}
See \tabref{tab:Distribution}.
\begin{align}
p_{Y}\brak{k}= \begin{cases} 
      \frac{1}{3} & {k=0} \\
      \frac{2}{3 }& {k=1} 
   \end{cases}
   \\
p_{Y|X}\brak{0|0} = \frac{19}{25}\, 
p_{Y|X}\brak{0|1} = \frac{6}{25}\,
p_{Y|X}\brak{1|0} = \frac{45}{50}\,
p_{Y|X}\brak{1|2} = \frac{5}{50}
\end{align}
The desired probability is the probability that a slip drawn at random is marked other than Rs 1,
\begin{align}
&=1-p_X\brak{0}\\
&= p_X(1) + p_X(2)
\end{align}
Using Bayes theorem,
\begin{align}
&= p_Y\brak{0} \times \pr{Y=0 | X=1} + p_Y\brak{1} \times \pr{Y=1|X=2}\\
&=\frac{1}{3} \times \frac{6}{25} + \frac{2}{3} \times \frac{5}{50}\\
&=\frac{11}{75}
\end{align}

\newpage

%\tableofcontents

\bigskip

\renewcommand{\thefigure}{\theenumi}
\renewcommand{\thetable}{\theenumi}
%\renewcommand{\theequation}{\theenumi}

%\begin{abstract}
%%\boldmath
%In this letter, an algorithm for evaluating the exact analytical bit error rate  (BER)  for the piecewise linear (PL) combiner for  multiple relays is presented. Previous results were available only for upto three relays. The algorithm is unique in the sense that  the actual mathematical expressions, that are prohibitively large, need not be explicitly obtained. The diversity gain due to multiple relays is shown through plots of the analytical BER, well supported by simulations. 
%
%\end{abstract}
% IEEEtran.cls defaults to using nonbold math in the Abstract.
% This preserves the distinction between vectors and scalars. However,
% if the journal you are submitting to favors bold math in the abstract,
% then you can use LaTeX's standard command \boldmath at the very start
% of the abstract to achieve this. Many IEEE journals frown on math
% in the abstract anyway.

% Note that keywords are not normally used for peerreview papers.
%\begin{IEEEkeywords}
%Cooperative diversity, decode and forward, piecewise linear
%\end{IEEEkeywords}



% For peer review papers, you can put extra information on the cover
% page as needed:
% \ifCLASSOPTIONpeerreview
% \begin{center} \bfseries EDICS Category: 3-BBND \end{center}
% \fi
%
% For peerreview papers, this IEEEtran command inserts a page break and
% creates the second title. It will be ignored for other modes.
%\IEEEpeerreviewmaketitle




If the letters of the word ASSASSINATION are arranged at random. Find the Probability that
\begin{enumerate}[label=(\alph*)]
\item Four $S's$ come consecutively in the word
\item Two  $I's$ and two $N's$ come together
\item All $A's$ are not coming together
\item No two $A's$ are coming together
\end{enumerate}
%\begin{table}[H]
	\centering
\begin{tabular}{|c|c|c|}
\hline
Random variable &Value &Definition\\ \hline
\multirow{3}{*}{X} &0 &Slips of Rs 1\\
&1 &Slips of Rs 5\\
&2 &Slips of Rs 13\\ \hline
\multirow{2}{*}{Y} &0 &Box A\\
&1 &Box B\\\hline
\end{tabular}
\caption{}
\label{tab:Distribution}
\end{table}
See \tabref{tab:Distribution}.
\begin{align}
p_{Y}\brak{k}= \begin{cases} 
      \frac{1}{3} & {k=0} \\
      \frac{2}{3 }& {k=1} 
   \end{cases}
   \\
p_{Y|X}\brak{0|0} = \frac{19}{25}\, 
p_{Y|X}\brak{0|1} = \frac{6}{25}\,
p_{Y|X}\brak{1|0} = \frac{45}{50}\,
p_{Y|X}\brak{1|2} = \frac{5}{50}
\end{align}
The desired probability is the probability that a slip drawn at random is marked other than Rs 1,
\begin{align}
&=1-p_X\brak{0}\\
&= p_X(1) + p_X(2)
\end{align}
Using Bayes theorem,
\begin{align}
&= p_Y\brak{0} \times \pr{Y=0 | X=1} + p_Y\brak{1} \times \pr{Y=1|X=2}\\
&=\frac{1}{3} \times \frac{6}{25} + \frac{2}{3} \times \frac{5}{50}\\
&=\frac{11}{75}
\end{align}

\newpage

%\tableofcontents

\bigskip

\renewcommand{\thefigure}{\theenumi}
\renewcommand{\thetable}{\theenumi}
%\renewcommand{\theequation}{\theenumi}

%\begin{abstract}
%%\boldmath
%In this letter, an algorithm for evaluating the exact analytical bit error rate  (BER)  for the piecewise linear (PL) combiner for  multiple relays is presented. Previous results were available only for upto three relays. The algorithm is unique in the sense that  the actual mathematical expressions, that are prohibitively large, need not be explicitly obtained. The diversity gain due to multiple relays is shown through plots of the analytical BER, well supported by simulations. 
%
%\end{abstract}
% IEEEtran.cls defaults to using nonbold math in the Abstract.
% This preserves the distinction between vectors and scalars. However,
% if the journal you are submitting to favors bold math in the abstract,
% then you can use LaTeX's standard command \boldmath at the very start
% of the abstract to achieve this. Many IEEE journals frown on math
% in the abstract anyway.

% Note that keywords are not normally used for peerreview papers.
%\begin{IEEEkeywords}
%Cooperative diversity, decode and forward, piecewise linear
%\end{IEEEkeywords}



% For peer review papers, you can put extra information on the cover
% page as needed:
% \ifCLASSOPTIONpeerreview
% \begin{center} \bfseries EDICS Category: 3-BBND \end{center}
% \fi
%
% For peerreview papers, this IEEEtran command inserts a page break and
% creates the second title. It will be ignored for other modes.
%\IEEEpeerreviewmaketitle




	\item One urn contains two black balls (labelled B1 and B2) and one white ball. A
	second urn contains one black ball and two white balls (labelled W1 and W2).
	Suppose the following experiment is performed. One of the two urns is chosen
	at random. Next a ball is randomly chosen from the urn. Then a second ball is
	chosen at random from the same urn without replacing the first ball.
	
	\begin{enumerate}
	\item What is the probability that two black balls are chosen?
	
	\item What is the probability that two balls of opposite colour are chosen?
	\end{enumerate}
	\solution
	%\begin{align}
    \label{eq:12.13.6.18.1}
	\because	\pr{A|B} &> \pr{A},\
\frac{\pr{AB}}{\pr{B}} > \pr{A}
\\
    \label{eq:12.13.6.18.2}
	\implies \pr{AB} &> \pr{A}\pr{B}
	\\
	\text{or, } \frac{\pr{AB}}{\pr{A}} &=\pr{B|A} > \pr{A}
\end{align}

\end{enumerate}

	\item A card is selected from a pack of 52 cards.
 \begin{enumerate}[label=(\alph*)] 
                 \item How many points are there in the sample space?
                 \item Calculate the probability that the card is an ace of spades.
                 \item Calculate the probability that the card is (i) an ace and (ii) black card.
 \end{enumerate}
\solution
		%\begin{table}[H]
	\centering
\begin{tabular}{|c|c|c|}
\hline
Random variable &Value &Definition\\ \hline
\multirow{3}{*}{X} &0 &Slips of Rs 1\\
&1 &Slips of Rs 5\\
&2 &Slips of Rs 13\\ \hline
\multirow{2}{*}{Y} &0 &Box A\\
&1 &Box B\\\hline
\end{tabular}
\caption{}
\label{tab:Distribution}
\end{table}
See \tabref{tab:Distribution}.
\begin{align}
p_{Y}\brak{k}= \begin{cases} 
      \frac{1}{3} & {k=0} \\
      \frac{2}{3 }& {k=1} 
   \end{cases}
   \\
p_{Y|X}\brak{0|0} = \frac{19}{25}\, 
p_{Y|X}\brak{0|1} = \frac{6}{25}\,
p_{Y|X}\brak{1|0} = \frac{45}{50}\,
p_{Y|X}\brak{1|2} = \frac{5}{50}
\end{align}
The desired probability is the probability that a slip drawn at random is marked other than Rs 1,
\begin{align}
&=1-p_X\brak{0}\\
&= p_X(1) + p_X(2)
\end{align}
Using Bayes theorem,
\begin{align}
&= p_Y\brak{0} \times \pr{Y=0 | X=1} + p_Y\brak{1} \times \pr{Y=1|X=2}\\
&=\frac{1}{3} \times \frac{6}{25} + \frac{2}{3} \times \frac{5}{50}\\
&=\frac{11}{75}
\end{align}

\newpage

%\tableofcontents

\bigskip

\renewcommand{\thefigure}{\theenumi}
\renewcommand{\thetable}{\theenumi}
%\renewcommand{\theequation}{\theenumi}

%\begin{abstract}
%%\boldmath
%In this letter, an algorithm for evaluating the exact analytical bit error rate  (BER)  for the piecewise linear (PL) combiner for  multiple relays is presented. Previous results were available only for upto three relays. The algorithm is unique in the sense that  the actual mathematical expressions, that are prohibitively large, need not be explicitly obtained. The diversity gain due to multiple relays is shown through plots of the analytical BER, well supported by simulations. 
%
%\end{abstract}
% IEEEtran.cls defaults to using nonbold math in the Abstract.
% This preserves the distinction between vectors and scalars. However,
% if the journal you are submitting to favors bold math in the abstract,
% then you can use LaTeX's standard command \boldmath at the very start
% of the abstract to achieve this. Many IEEE journals frown on math
% in the abstract anyway.

% Note that keywords are not normally used for peerreview papers.
%\begin{IEEEkeywords}
%Cooperative diversity, decode and forward, piecewise linear
%\end{IEEEkeywords}



% For peer review papers, you can put extra information on the cover
% page as needed:
% \ifCLASSOPTIONpeerreview
% \begin{center} \bfseries EDICS Category: 3-BBND \end{center}
% \fi
%
% For peerreview papers, this IEEEtran command inserts a page break and
% creates the second title. It will be ignored for other modes.
%\IEEEpeerreviewmaketitle




\item Four cards are drawn from a well-shuffled deck of 52 cards. What is the probability of obtaining 3 diamonds and one spade.
\\
\solution
		%\begin{enumerate}[label=\thesection.\arabic*,ref=\thesection.\theenumi]
	\item One card is drawn from a well-shuffled deck of 52 cards. Find the probability of getting
\begin{enumerate}
\item A king of red colour 
\item A face card 
\item A red face card
\item The jack of hearts
\item A spade
\item The queen of diamonds

\end{enumerate}
\solution
		%\begin{table}[H]
	\centering
\begin{tabular}{|c|c|c|}
\hline
Random variable &Value &Definition\\ \hline
\multirow{3}{*}{X} &0 &Slips of Rs 1\\
&1 &Slips of Rs 5\\
&2 &Slips of Rs 13\\ \hline
\multirow{2}{*}{Y} &0 &Box A\\
&1 &Box B\\\hline
\end{tabular}
\caption{}
\label{tab:Distribution}
\end{table}
See \tabref{tab:Distribution}.
\begin{align}
p_{Y}\brak{k}= \begin{cases} 
      \frac{1}{3} & {k=0} \\
      \frac{2}{3 }& {k=1} 
   \end{cases}
   \\
p_{Y|X}\brak{0|0} = \frac{19}{25}\, 
p_{Y|X}\brak{0|1} = \frac{6}{25}\,
p_{Y|X}\brak{1|0} = \frac{45}{50}\,
p_{Y|X}\brak{1|2} = \frac{5}{50}
\end{align}
The desired probability is the probability that a slip drawn at random is marked other than Rs 1,
\begin{align}
&=1-p_X\brak{0}\\
&= p_X(1) + p_X(2)
\end{align}
Using Bayes theorem,
\begin{align}
&= p_Y\brak{0} \times \pr{Y=0 | X=1} + p_Y\brak{1} \times \pr{Y=1|X=2}\\
&=\frac{1}{3} \times \frac{6}{25} + \frac{2}{3} \times \frac{5}{50}\\
&=\frac{11}{75}
\end{align}

\newpage

%\tableofcontents

\bigskip

\renewcommand{\thefigure}{\theenumi}
\renewcommand{\thetable}{\theenumi}
%\renewcommand{\theequation}{\theenumi}

%\begin{abstract}
%%\boldmath
%In this letter, an algorithm for evaluating the exact analytical bit error rate  (BER)  for the piecewise linear (PL) combiner for  multiple relays is presented. Previous results were available only for upto three relays. The algorithm is unique in the sense that  the actual mathematical expressions, that are prohibitively large, need not be explicitly obtained. The diversity gain due to multiple relays is shown through plots of the analytical BER, well supported by simulations. 
%
%\end{abstract}
% IEEEtran.cls defaults to using nonbold math in the Abstract.
% This preserves the distinction between vectors and scalars. However,
% if the journal you are submitting to favors bold math in the abstract,
% then you can use LaTeX's standard command \boldmath at the very start
% of the abstract to achieve this. Many IEEE journals frown on math
% in the abstract anyway.

% Note that keywords are not normally used for peerreview papers.
%\begin{IEEEkeywords}
%Cooperative diversity, decode and forward, piecewise linear
%\end{IEEEkeywords}



% For peer review papers, you can put extra information on the cover
% page as needed:
% \ifCLASSOPTIONpeerreview
% \begin{center} \bfseries EDICS Category: 3-BBND \end{center}
% \fi
%
% For peerreview papers, this IEEEtran command inserts a page break and
% creates the second title. It will be ignored for other modes.
%\IEEEpeerreviewmaketitle




	\item Five cards—the ten, jack, queen, king and ace of diamonds, are well-shuffled with their face downwards. One card is then picked up at random.
\begin{enumerate}
\item
What is the probability that the card is the queen? 
\item
If the queen is drawn and put aside, what is the probability that the second card picked up is (a) an ace? (b) a queen?\\
\end{enumerate}
\solution
		%\begin{enumerate}[label=\thesection.\arabic*,ref=\thesection.\theenumi]
	\item One card is drawn from a well-shuffled deck of 52 cards. Find the probability of getting
\begin{enumerate}
\item A king of red colour 
\item A face card 
\item A red face card
\item The jack of hearts
\item A spade
\item The queen of diamonds

\end{enumerate}
\solution
		%\input{ncert/10/15/1/14/main.tex}
	\item Five cards—the ten, jack, queen, king and ace of diamonds, are well-shuffled with their face downwards. One card is then picked up at random.
\begin{enumerate}
\item
What is the probability that the card is the queen? 
\item
If the queen is drawn and put aside, what is the probability that the second card picked up is (a) an ace? (b) a queen?\\
\end{enumerate}
\solution
		%\input{ncert/10/15/1/15/defs.tex}
	\item A bag contains $5$ red balls and some blue balls. If the probability of drawing a blue ball is double that if a red ball, determine the number of blue balls in the bag. 
		\\
\solution
		%\input{ncert/10/15/2/3/defs.tex}
	\item A card is selected from a pack of 52 cards.
 \begin{enumerate}[label=(\alph*)] 
                 \item How many points are there in the sample space?
                 \item Calculate the probability that the card is an ace of spades.
                 \item Calculate the probability that the card is (i) an ace and (ii) black card.
 \end{enumerate}
\solution
		%\input{ncert/11/16/3/4/main.tex}
\item Four cards are drawn from a well-shuffled deck of 52 cards. What is the probability of obtaining 3 diamonds and one spade.
\\
\solution
		%\input{ncert/11/16/4/2/defs.tex}
\item In a certain lottery 10,000 tickets are sold and ten equal prizes are awarded. What is the probability of not getting a prize if you buy (a) one ticket (b) two tickets (c) 10 tickets ?	
\\
\solution
		%\input{ncert/11/16/4/4/defs.tex}
		%
\item 
Out of 100 students, two sections of 40 and 60 are formed. If you and your friend are among the 100 students, what is the probability that
\begin{enumerate}
\item you both enter the same section?
\item you both enter the different sections?
\end{enumerate}
\solution
		%\input{ncert/11/16/4/5/defs.tex}
	\item 
The number lock of a suitcase has 4 wheels each labelled with ten digits i.e. from 0 to 9.The lock opens with a sequence of four digits with no repeats.What is the probability of a person getting the right sequence to open the suitcase.
\\
\solution
		%\input{ncert/11/16/4/10/defs.tex}
		%
\item 
Two cards are drawn at random and without replacement from a pack of 52 playing cards. Find the probability that both the cards are black.
\\
\solution
		%\input{ncert/12/13/2/2/defs.tex}
		\item A box of oranges is inspected by examining three randomly selected oranges drawn without replacement. If all the three oranges are good, the box is approved for sale, otherwise, it is rejected. Find the probability that a box containing 15 oranges out of which 12 are good and 3 are bad ones will be approved for sale.
		\label{ncert/12/13/2/3/defs.tex}
		\item Two balls are drawn at random with replacement from a box containing 10 black and 8 red balls. Find the probability that
		\label{ncert/12/13/2/12}
\begin{enumerate}
\item both balls are red.
\item first ball is black and second is red.
\item one of them is black and other is red.
\end{enumerate}

\item In a hostel, 60\% of the students read Hindi newspaper, 40\% read English newspaper and 20\% read both Hindi and English newspapers. A student is selected at random.
		\label{ncert/12/13/2/15}
\begin{enumerate}
\item Find the probability that she reads neither Hindi nor English newspapers.
\item If she reads Hindi newspaper, find the probability that she reads English newspaper.
\item If she reads English newspaper, find the probability that she reads Hindi newspaper.\\
\end{enumerate}
\item The probability of obtaining an even prime number on each die, when a pair of dice is rolled is 
\begin{enumerate}
    \item $0$ 
    
    \item $\frac{1}{3}$ 
    
    \item $\frac{1}{12}$ 
    
    \item $\frac{1}{36}$ 
\end{enumerate}
\solution
		%\input{ncert/12/13/2/17/defs.tex}
	\item A bag contains 4 red and 4 black balls, another bag contains 2 red and 6 black balls. One of the two bags is selected at random and a ball is drawn from the bag which is found to be red. Find the probability that the ball is drawn from the first bag.
\\
\solution
		%\input{ncert/12/13/3/2/main.tex}
  \item
  Cards with numbers 2 to 101 are placed in a box. A card is selected at random.Find the probability that the card has
\begin{enumerate}[label=(\roman*)]
	\item an even number 
	\item a square number
\end{enumerate}
\solution
%\input{exemplar/10/13/3/32/main.tex}
\item
The king, queen and jack of clubs are removed from a deck of 52 playing cards and then well shuffled. Now one card is drawn at random from the remaining cards.  Determine the probability that the card is
\begin{enumerate}[label=(\roman*)]
\item a club
\item 10 of hearts
\end{enumerate}
\solution
%\input{exemplar/10/13/3/29/main.tex}
\item A team of medical students doing their internship have to assist during surgeries
at a city hospital. The probabilities of surgeries rated as very complex, complex,
routine, simple or very simple are respectively, 0.15, 0.20, 0.31, 0.26, .08. Find
the probabilities that a particular surgery will be rated
\begin{enumerate}
	\item complex or very complex;
	\item neither very complex nor very simple;
	\item routine or complex
	\item routine or simple
\end{enumerate}
\solution
%\input{exemplar/11/16/3/8(1)/main.tex}
\item A card is selected from a pack of 52 cards.
\begin{enumerate}[label=(\alph*)]
    \item How many points are there in the sample space?
    \item Calculate the probability that the card is an ace of spades.
    \item Calculate the probability that the card is (i) an ace and (ii) black card.
\end{enumerate}
\solution
%\input{exemplar/11/16/3/4/main2.tex}
\item The probability that a non leap year selected at random will contain 53 sundays.
\\
\solution
%\input{exemplar/10/13/1/19/main.tex}
\item One of the four persons John, Rita, Aslam or Gurpreet will be promoted next
month. Consequently the sample space consists of four elementary outcomes
S = {John promoted, Rita promoted, Aslam promoted, Gurpreet promoted}
You are told that the chances of John’s promotion is same as that of Gurpreet,
Rita’s chances of promotion are twice as likely as Johns. Aslam’s chances are
four times that of John.
\begin{enumerate}
	\item Determine
	\begin{enumerate}
		\item P (John promoted)
		\item P (Rita promoted)
		\item P (Aslam promoted)
		\item P (Gurpreet promoted)
	\end{enumerate}
	\item If A = {John promoted or Gurpreet promoted}, find P (A).
\end{enumerate}
\solution
%\input{exemplar/11/16/3/10/main.tex}
\item A card is drawn from a deck of 52 cards. Find the probability of getting a king or a heart or a red card.\\
\solution
%\input{exemplar/11/16/3/15/main.tex}
\item The probability that a student will pass his examination is 0.73, the probability of
the student getting a compartment is 0.13, and the probability that the student will
either pass or get compartment is 0.96. State True or False.\\
\solution
%\input{exemplar/11/16/3/31/main.tex}
\item A card is selected from a pack of 52 cards\\
\begin{enumerate}[label=(\alph*)]
\item How many points are there in the sample space?
\item Calculate the probability that the cards is an ace of spades.
\item Calculate the probability that the card is (i) an ace (ii)black card.\\
\end{enumerate}
%\input{ncert/11/16/3/4_1/Prob_4.tex}
\item In a non-leap year, the probability of having 53 tuesdays or 53 wednesdays is\\
\solution
%\input{exemplar/11/16/3/18/main.tex}
\item There are 1000 sealed envelopes in a box, 10 of them contain a cash prize of
Rs 100 each, 100 of them contain a cash prize of Rs 50 each and 200 of them
contain a cash prize of Rs 10 each and rest do not contain any cash prize. If they
are well shuffled and an envelope is picked up out, what is the probability that it
contains no cash prize?\\
\solution
%\input{exemplar/10/13/3/34/main.tex}
\item 
A die is thrown and a card is selected at random from a deck of 52 playing cards. The probability of getting an even number on the die and a spade card.\\
\solution
%\input{exemplar/12/13/3/78/main.tex}
\item
If 4-digit numbers greater than 5,000 are randomly formed from the digits 0, 1, 3, 5, and 7, what is the probability of forming a number divisible by 5 when:
\begin{enumerate}
    \item The digits are repeated?
    \item The repetition of digits is not allowed?
\end{enumerate}
\solution
%\input{ncert/11/16/4/9/main.tex}
\item Consider the probability space $\brak{\Omega, \mathcal{G}, P}$ where $\Omega = [0,2]$ and $\mathcal{G} = \cbrak{\phi, \Omega, [0,1], (1,2]}$. Let $X$ and $Y$ be two functions on $\Omega$ defined as
\begin{align*}
    X(\omega) = 
    \begin{cases}
        1 & \text{if }\omega \in [0, 1]\\
        2 & \text{if }\omega \in (1, 2]
    \end{cases}
\end{align*}
and
\begin{align*}
    Y(\omega) = 
    \begin{cases}
        2 & \text{if }\omega \in [0, 1.5]\\
        3 & \text{if }\omega \in (1.5, 2].
    \end{cases}
\end{align*}
Then which one of the following statements is true?
\begin{enumerate}
    \item [(A)] $X$ is a random variable with respect to $\mathcal{G}$, but $Y$ is not a random variable with respect to $\mathcal{G}$.
    \item [(B)] $Y$ is a random variable with respect to $\mathcal{G}$, but $X$ is not a random variable with respect to $\mathcal{G}$.
    \item [(C)] Neither $X$ nor $Y$ is a random variable with respect to $\mathcal{G}$.
    \item [(D)] Both $X$ and $Y$ are random variables with respect to $\mathcal{G}$.
\end{enumerate} \hfill (GATE ST 2023)\\
\solution
%\input{gate/ST/2023/14/main.tex}
	\item  A die is loaded in such a way that each odd number is twice as likely to occur as
each even number. Find $P(G)$, where $G$ is the event that a number greater than
3 occurs on a single roll of the die.
\\
\solution
		%\input{exemplar/11/16/3/5/main.tex}
	\item All the jacks, queens and kings are removed from a deck of 52 playing cards. The remaining cards are well shuffled and then one card is drawn at random. Giving ace a value 1 similar value for other cards, find the probability that the card has a value 
		\begin{enumerate}
			\item 7
			\item greater than 7
			\item less than 7
		\end{enumerate}
		%\input{exemplar/10/13/3/30/main.tex}
  \item A Lot consists of 48 mobile phones of which 42 are good, 3 have only minor defects and 3 have major defects.Varnika will buy a phone if it is good but the trader will only buy a mobile if it has no major defects. One phone is selected at random from the lot. What is the probability that it is
\begin{enumerate}
	\item acceptable to Varnika?
            \item acceptable to the trader?
\end{enumerate}
\solution
	%\input{exemplar/10/13/3/40/main.tex}
 \item A student says that if you throw a die, it will show up 1 or not 1. Therefore, the probability of getting 1 and the probability of getting 'not 1' each is equal to $\frac{1}{2}$. Is this correct? Give reasons.\\
 \solution
        %\input{exemplar/10/13/2/9/main.tex}
   \item Four candidates A, B, C, D have ap-
plied for the assignment to coach a school cricket
team. If A is twice as likely to be selected as B, and
B and C are given about the same chance of being
selected, while C is twice as likely to be selected
as D, what are the probabilities that
\begin{enumerate}
\item C will be selected?
\item A will not be selected?
\end{enumerate}
	%\input{exemplar/11/16/3/9/main.tex}
 \item A bag contain 24 balls of which $x$ balls are red, $2x$ are white and $3x$ are blue. A ball is selected at random, What is the probability that it is
\begin{enumerate}[label=\alph*)]
\item not red ?
\item white ?
\end{enumerate}
%\input{exemplar/10/13/3/41/main.tex}
If the letters of the word ASSASSINATION are arranged at random. Find the Probability that
\begin{enumerate}[label=(\alph*)]
\item Four $S's$ come consecutively in the word
\item Two  $I's$ and two $N's$ come together
\item All $A's$ are not coming together
\item No two $A's$ are coming together
\end{enumerate}
%\input{exemplar/11/16/3/14/main.tex}
	\item One urn contains two black balls (labelled B1 and B2) and one white ball. A
	second urn contains one black ball and two white balls (labelled W1 and W2).
	Suppose the following experiment is performed. One of the two urns is chosen
	at random. Next a ball is randomly chosen from the urn. Then a second ball is
	chosen at random from the same urn without replacing the first ball.
	
	\begin{enumerate}
	\item What is the probability that two black balls are chosen?
	
	\item What is the probability that two balls of opposite colour are chosen?
	\end{enumerate}
	\solution
	%\input{exemplar/11/16/3/12/main1.tex}
\end{enumerate}

	\item A bag contains $5$ red balls and some blue balls. If the probability of drawing a blue ball is double that if a red ball, determine the number of blue balls in the bag. 
		\\
\solution
		%\begin{enumerate}[label=\thesection.\arabic*,ref=\thesection.\theenumi]
	\item One card is drawn from a well-shuffled deck of 52 cards. Find the probability of getting
\begin{enumerate}
\item A king of red colour 
\item A face card 
\item A red face card
\item The jack of hearts
\item A spade
\item The queen of diamonds

\end{enumerate}
\solution
		%\input{ncert/10/15/1/14/main.tex}
	\item Five cards—the ten, jack, queen, king and ace of diamonds, are well-shuffled with their face downwards. One card is then picked up at random.
\begin{enumerate}
\item
What is the probability that the card is the queen? 
\item
If the queen is drawn and put aside, what is the probability that the second card picked up is (a) an ace? (b) a queen?\\
\end{enumerate}
\solution
		%\input{ncert/10/15/1/15/defs.tex}
	\item A bag contains $5$ red balls and some blue balls. If the probability of drawing a blue ball is double that if a red ball, determine the number of blue balls in the bag. 
		\\
\solution
		%\input{ncert/10/15/2/3/defs.tex}
	\item A card is selected from a pack of 52 cards.
 \begin{enumerate}[label=(\alph*)] 
                 \item How many points are there in the sample space?
                 \item Calculate the probability that the card is an ace of spades.
                 \item Calculate the probability that the card is (i) an ace and (ii) black card.
 \end{enumerate}
\solution
		%\input{ncert/11/16/3/4/main.tex}
\item Four cards are drawn from a well-shuffled deck of 52 cards. What is the probability of obtaining 3 diamonds and one spade.
\\
\solution
		%\input{ncert/11/16/4/2/defs.tex}
\item In a certain lottery 10,000 tickets are sold and ten equal prizes are awarded. What is the probability of not getting a prize if you buy (a) one ticket (b) two tickets (c) 10 tickets ?	
\\
\solution
		%\input{ncert/11/16/4/4/defs.tex}
		%
\item 
Out of 100 students, two sections of 40 and 60 are formed. If you and your friend are among the 100 students, what is the probability that
\begin{enumerate}
\item you both enter the same section?
\item you both enter the different sections?
\end{enumerate}
\solution
		%\input{ncert/11/16/4/5/defs.tex}
	\item 
The number lock of a suitcase has 4 wheels each labelled with ten digits i.e. from 0 to 9.The lock opens with a sequence of four digits with no repeats.What is the probability of a person getting the right sequence to open the suitcase.
\\
\solution
		%\input{ncert/11/16/4/10/defs.tex}
		%
\item 
Two cards are drawn at random and without replacement from a pack of 52 playing cards. Find the probability that both the cards are black.
\\
\solution
		%\input{ncert/12/13/2/2/defs.tex}
		\item A box of oranges is inspected by examining three randomly selected oranges drawn without replacement. If all the three oranges are good, the box is approved for sale, otherwise, it is rejected. Find the probability that a box containing 15 oranges out of which 12 are good and 3 are bad ones will be approved for sale.
		\label{ncert/12/13/2/3/defs.tex}
		\item Two balls are drawn at random with replacement from a box containing 10 black and 8 red balls. Find the probability that
		\label{ncert/12/13/2/12}
\begin{enumerate}
\item both balls are red.
\item first ball is black and second is red.
\item one of them is black and other is red.
\end{enumerate}

\item In a hostel, 60\% of the students read Hindi newspaper, 40\% read English newspaper and 20\% read both Hindi and English newspapers. A student is selected at random.
		\label{ncert/12/13/2/15}
\begin{enumerate}
\item Find the probability that she reads neither Hindi nor English newspapers.
\item If she reads Hindi newspaper, find the probability that she reads English newspaper.
\item If she reads English newspaper, find the probability that she reads Hindi newspaper.\\
\end{enumerate}
\item The probability of obtaining an even prime number on each die, when a pair of dice is rolled is 
\begin{enumerate}
    \item $0$ 
    
    \item $\frac{1}{3}$ 
    
    \item $\frac{1}{12}$ 
    
    \item $\frac{1}{36}$ 
\end{enumerate}
\solution
		%\input{ncert/12/13/2/17/defs.tex}
	\item A bag contains 4 red and 4 black balls, another bag contains 2 red and 6 black balls. One of the two bags is selected at random and a ball is drawn from the bag which is found to be red. Find the probability that the ball is drawn from the first bag.
\\
\solution
		%\input{ncert/12/13/3/2/main.tex}
  \item
  Cards with numbers 2 to 101 are placed in a box. A card is selected at random.Find the probability that the card has
\begin{enumerate}[label=(\roman*)]
	\item an even number 
	\item a square number
\end{enumerate}
\solution
%\input{exemplar/10/13/3/32/main.tex}
\item
The king, queen and jack of clubs are removed from a deck of 52 playing cards and then well shuffled. Now one card is drawn at random from the remaining cards.  Determine the probability that the card is
\begin{enumerate}[label=(\roman*)]
\item a club
\item 10 of hearts
\end{enumerate}
\solution
%\input{exemplar/10/13/3/29/main.tex}
\item A team of medical students doing their internship have to assist during surgeries
at a city hospital. The probabilities of surgeries rated as very complex, complex,
routine, simple or very simple are respectively, 0.15, 0.20, 0.31, 0.26, .08. Find
the probabilities that a particular surgery will be rated
\begin{enumerate}
	\item complex or very complex;
	\item neither very complex nor very simple;
	\item routine or complex
	\item routine or simple
\end{enumerate}
\solution
%\input{exemplar/11/16/3/8(1)/main.tex}
\item A card is selected from a pack of 52 cards.
\begin{enumerate}[label=(\alph*)]
    \item How many points are there in the sample space?
    \item Calculate the probability that the card is an ace of spades.
    \item Calculate the probability that the card is (i) an ace and (ii) black card.
\end{enumerate}
\solution
%\input{exemplar/11/16/3/4/main2.tex}
\item The probability that a non leap year selected at random will contain 53 sundays.
\\
\solution
%\input{exemplar/10/13/1/19/main.tex}
\item One of the four persons John, Rita, Aslam or Gurpreet will be promoted next
month. Consequently the sample space consists of four elementary outcomes
S = {John promoted, Rita promoted, Aslam promoted, Gurpreet promoted}
You are told that the chances of John’s promotion is same as that of Gurpreet,
Rita’s chances of promotion are twice as likely as Johns. Aslam’s chances are
four times that of John.
\begin{enumerate}
	\item Determine
	\begin{enumerate}
		\item P (John promoted)
		\item P (Rita promoted)
		\item P (Aslam promoted)
		\item P (Gurpreet promoted)
	\end{enumerate}
	\item If A = {John promoted or Gurpreet promoted}, find P (A).
\end{enumerate}
\solution
%\input{exemplar/11/16/3/10/main.tex}
\item A card is drawn from a deck of 52 cards. Find the probability of getting a king or a heart or a red card.\\
\solution
%\input{exemplar/11/16/3/15/main.tex}
\item The probability that a student will pass his examination is 0.73, the probability of
the student getting a compartment is 0.13, and the probability that the student will
either pass or get compartment is 0.96. State True or False.\\
\solution
%\input{exemplar/11/16/3/31/main.tex}
\item A card is selected from a pack of 52 cards\\
\begin{enumerate}[label=(\alph*)]
\item How many points are there in the sample space?
\item Calculate the probability that the cards is an ace of spades.
\item Calculate the probability that the card is (i) an ace (ii)black card.\\
\end{enumerate}
%\input{ncert/11/16/3/4_1/Prob_4.tex}
\item In a non-leap year, the probability of having 53 tuesdays or 53 wednesdays is\\
\solution
%\input{exemplar/11/16/3/18/main.tex}
\item There are 1000 sealed envelopes in a box, 10 of them contain a cash prize of
Rs 100 each, 100 of them contain a cash prize of Rs 50 each and 200 of them
contain a cash prize of Rs 10 each and rest do not contain any cash prize. If they
are well shuffled and an envelope is picked up out, what is the probability that it
contains no cash prize?\\
\solution
%\input{exemplar/10/13/3/34/main.tex}
\item 
A die is thrown and a card is selected at random from a deck of 52 playing cards. The probability of getting an even number on the die and a spade card.\\
\solution
%\input{exemplar/12/13/3/78/main.tex}
\item
If 4-digit numbers greater than 5,000 are randomly formed from the digits 0, 1, 3, 5, and 7, what is the probability of forming a number divisible by 5 when:
\begin{enumerate}
    \item The digits are repeated?
    \item The repetition of digits is not allowed?
\end{enumerate}
\solution
%\input{ncert/11/16/4/9/main.tex}
\item Consider the probability space $\brak{\Omega, \mathcal{G}, P}$ where $\Omega = [0,2]$ and $\mathcal{G} = \cbrak{\phi, \Omega, [0,1], (1,2]}$. Let $X$ and $Y$ be two functions on $\Omega$ defined as
\begin{align*}
    X(\omega) = 
    \begin{cases}
        1 & \text{if }\omega \in [0, 1]\\
        2 & \text{if }\omega \in (1, 2]
    \end{cases}
\end{align*}
and
\begin{align*}
    Y(\omega) = 
    \begin{cases}
        2 & \text{if }\omega \in [0, 1.5]\\
        3 & \text{if }\omega \in (1.5, 2].
    \end{cases}
\end{align*}
Then which one of the following statements is true?
\begin{enumerate}
    \item [(A)] $X$ is a random variable with respect to $\mathcal{G}$, but $Y$ is not a random variable with respect to $\mathcal{G}$.
    \item [(B)] $Y$ is a random variable with respect to $\mathcal{G}$, but $X$ is not a random variable with respect to $\mathcal{G}$.
    \item [(C)] Neither $X$ nor $Y$ is a random variable with respect to $\mathcal{G}$.
    \item [(D)] Both $X$ and $Y$ are random variables with respect to $\mathcal{G}$.
\end{enumerate} \hfill (GATE ST 2023)\\
\solution
%\input{gate/ST/2023/14/main.tex}
	\item  A die is loaded in such a way that each odd number is twice as likely to occur as
each even number. Find $P(G)$, where $G$ is the event that a number greater than
3 occurs on a single roll of the die.
\\
\solution
		%\input{exemplar/11/16/3/5/main.tex}
	\item All the jacks, queens and kings are removed from a deck of 52 playing cards. The remaining cards are well shuffled and then one card is drawn at random. Giving ace a value 1 similar value for other cards, find the probability that the card has a value 
		\begin{enumerate}
			\item 7
			\item greater than 7
			\item less than 7
		\end{enumerate}
		%\input{exemplar/10/13/3/30/main.tex}
  \item A Lot consists of 48 mobile phones of which 42 are good, 3 have only minor defects and 3 have major defects.Varnika will buy a phone if it is good but the trader will only buy a mobile if it has no major defects. One phone is selected at random from the lot. What is the probability that it is
\begin{enumerate}
	\item acceptable to Varnika?
            \item acceptable to the trader?
\end{enumerate}
\solution
	%\input{exemplar/10/13/3/40/main.tex}
 \item A student says that if you throw a die, it will show up 1 or not 1. Therefore, the probability of getting 1 and the probability of getting 'not 1' each is equal to $\frac{1}{2}$. Is this correct? Give reasons.\\
 \solution
        %\input{exemplar/10/13/2/9/main.tex}
   \item Four candidates A, B, C, D have ap-
plied for the assignment to coach a school cricket
team. If A is twice as likely to be selected as B, and
B and C are given about the same chance of being
selected, while C is twice as likely to be selected
as D, what are the probabilities that
\begin{enumerate}
\item C will be selected?
\item A will not be selected?
\end{enumerate}
	%\input{exemplar/11/16/3/9/main.tex}
 \item A bag contain 24 balls of which $x$ balls are red, $2x$ are white and $3x$ are blue. A ball is selected at random, What is the probability that it is
\begin{enumerate}[label=\alph*)]
\item not red ?
\item white ?
\end{enumerate}
%\input{exemplar/10/13/3/41/main.tex}
If the letters of the word ASSASSINATION are arranged at random. Find the Probability that
\begin{enumerate}[label=(\alph*)]
\item Four $S's$ come consecutively in the word
\item Two  $I's$ and two $N's$ come together
\item All $A's$ are not coming together
\item No two $A's$ are coming together
\end{enumerate}
%\input{exemplar/11/16/3/14/main.tex}
	\item One urn contains two black balls (labelled B1 and B2) and one white ball. A
	second urn contains one black ball and two white balls (labelled W1 and W2).
	Suppose the following experiment is performed. One of the two urns is chosen
	at random. Next a ball is randomly chosen from the urn. Then a second ball is
	chosen at random from the same urn without replacing the first ball.
	
	\begin{enumerate}
	\item What is the probability that two black balls are chosen?
	
	\item What is the probability that two balls of opposite colour are chosen?
	\end{enumerate}
	\solution
	%\input{exemplar/11/16/3/12/main1.tex}
\end{enumerate}

	\item A card is selected from a pack of 52 cards.
 \begin{enumerate}[label=(\alph*)] 
                 \item How many points are there in the sample space?
                 \item Calculate the probability that the card is an ace of spades.
                 \item Calculate the probability that the card is (i) an ace and (ii) black card.
 \end{enumerate}
\solution
		%\begin{table}[H]
	\centering
\begin{tabular}{|c|c|c|}
\hline
Random variable &Value &Definition\\ \hline
\multirow{3}{*}{X} &0 &Slips of Rs 1\\
&1 &Slips of Rs 5\\
&2 &Slips of Rs 13\\ \hline
\multirow{2}{*}{Y} &0 &Box A\\
&1 &Box B\\\hline
\end{tabular}
\caption{}
\label{tab:Distribution}
\end{table}
See \tabref{tab:Distribution}.
\begin{align}
p_{Y}\brak{k}= \begin{cases} 
      \frac{1}{3} & {k=0} \\
      \frac{2}{3 }& {k=1} 
   \end{cases}
   \\
p_{Y|X}\brak{0|0} = \frac{19}{25}\, 
p_{Y|X}\brak{0|1} = \frac{6}{25}\,
p_{Y|X}\brak{1|0} = \frac{45}{50}\,
p_{Y|X}\brak{1|2} = \frac{5}{50}
\end{align}
The desired probability is the probability that a slip drawn at random is marked other than Rs 1,
\begin{align}
&=1-p_X\brak{0}\\
&= p_X(1) + p_X(2)
\end{align}
Using Bayes theorem,
\begin{align}
&= p_Y\brak{0} \times \pr{Y=0 | X=1} + p_Y\brak{1} \times \pr{Y=1|X=2}\\
&=\frac{1}{3} \times \frac{6}{25} + \frac{2}{3} \times \frac{5}{50}\\
&=\frac{11}{75}
\end{align}

\newpage

%\tableofcontents

\bigskip

\renewcommand{\thefigure}{\theenumi}
\renewcommand{\thetable}{\theenumi}
%\renewcommand{\theequation}{\theenumi}

%\begin{abstract}
%%\boldmath
%In this letter, an algorithm for evaluating the exact analytical bit error rate  (BER)  for the piecewise linear (PL) combiner for  multiple relays is presented. Previous results were available only for upto three relays. The algorithm is unique in the sense that  the actual mathematical expressions, that are prohibitively large, need not be explicitly obtained. The diversity gain due to multiple relays is shown through plots of the analytical BER, well supported by simulations. 
%
%\end{abstract}
% IEEEtran.cls defaults to using nonbold math in the Abstract.
% This preserves the distinction between vectors and scalars. However,
% if the journal you are submitting to favors bold math in the abstract,
% then you can use LaTeX's standard command \boldmath at the very start
% of the abstract to achieve this. Many IEEE journals frown on math
% in the abstract anyway.

% Note that keywords are not normally used for peerreview papers.
%\begin{IEEEkeywords}
%Cooperative diversity, decode and forward, piecewise linear
%\end{IEEEkeywords}



% For peer review papers, you can put extra information on the cover
% page as needed:
% \ifCLASSOPTIONpeerreview
% \begin{center} \bfseries EDICS Category: 3-BBND \end{center}
% \fi
%
% For peerreview papers, this IEEEtran command inserts a page break and
% creates the second title. It will be ignored for other modes.
%\IEEEpeerreviewmaketitle




\item Four cards are drawn from a well-shuffled deck of 52 cards. What is the probability of obtaining 3 diamonds and one spade.
\\
\solution
		%\begin{enumerate}[label=\thesection.\arabic*,ref=\thesection.\theenumi]
	\item One card is drawn from a well-shuffled deck of 52 cards. Find the probability of getting
\begin{enumerate}
\item A king of red colour 
\item A face card 
\item A red face card
\item The jack of hearts
\item A spade
\item The queen of diamonds

\end{enumerate}
\solution
		%\input{ncert/10/15/1/14/main.tex}
	\item Five cards—the ten, jack, queen, king and ace of diamonds, are well-shuffled with their face downwards. One card is then picked up at random.
\begin{enumerate}
\item
What is the probability that the card is the queen? 
\item
If the queen is drawn and put aside, what is the probability that the second card picked up is (a) an ace? (b) a queen?\\
\end{enumerate}
\solution
		%\input{ncert/10/15/1/15/defs.tex}
	\item A bag contains $5$ red balls and some blue balls. If the probability of drawing a blue ball is double that if a red ball, determine the number of blue balls in the bag. 
		\\
\solution
		%\input{ncert/10/15/2/3/defs.tex}
	\item A card is selected from a pack of 52 cards.
 \begin{enumerate}[label=(\alph*)] 
                 \item How many points are there in the sample space?
                 \item Calculate the probability that the card is an ace of spades.
                 \item Calculate the probability that the card is (i) an ace and (ii) black card.
 \end{enumerate}
\solution
		%\input{ncert/11/16/3/4/main.tex}
\item Four cards are drawn from a well-shuffled deck of 52 cards. What is the probability of obtaining 3 diamonds and one spade.
\\
\solution
		%\input{ncert/11/16/4/2/defs.tex}
\item In a certain lottery 10,000 tickets are sold and ten equal prizes are awarded. What is the probability of not getting a prize if you buy (a) one ticket (b) two tickets (c) 10 tickets ?	
\\
\solution
		%\input{ncert/11/16/4/4/defs.tex}
		%
\item 
Out of 100 students, two sections of 40 and 60 are formed. If you and your friend are among the 100 students, what is the probability that
\begin{enumerate}
\item you both enter the same section?
\item you both enter the different sections?
\end{enumerate}
\solution
		%\input{ncert/11/16/4/5/defs.tex}
	\item 
The number lock of a suitcase has 4 wheels each labelled with ten digits i.e. from 0 to 9.The lock opens with a sequence of four digits with no repeats.What is the probability of a person getting the right sequence to open the suitcase.
\\
\solution
		%\input{ncert/11/16/4/10/defs.tex}
		%
\item 
Two cards are drawn at random and without replacement from a pack of 52 playing cards. Find the probability that both the cards are black.
\\
\solution
		%\input{ncert/12/13/2/2/defs.tex}
		\item A box of oranges is inspected by examining three randomly selected oranges drawn without replacement. If all the three oranges are good, the box is approved for sale, otherwise, it is rejected. Find the probability that a box containing 15 oranges out of which 12 are good and 3 are bad ones will be approved for sale.
		\label{ncert/12/13/2/3/defs.tex}
		\item Two balls are drawn at random with replacement from a box containing 10 black and 8 red balls. Find the probability that
		\label{ncert/12/13/2/12}
\begin{enumerate}
\item both balls are red.
\item first ball is black and second is red.
\item one of them is black and other is red.
\end{enumerate}

\item In a hostel, 60\% of the students read Hindi newspaper, 40\% read English newspaper and 20\% read both Hindi and English newspapers. A student is selected at random.
		\label{ncert/12/13/2/15}
\begin{enumerate}
\item Find the probability that she reads neither Hindi nor English newspapers.
\item If she reads Hindi newspaper, find the probability that she reads English newspaper.
\item If she reads English newspaper, find the probability that she reads Hindi newspaper.\\
\end{enumerate}
\item The probability of obtaining an even prime number on each die, when a pair of dice is rolled is 
\begin{enumerate}
    \item $0$ 
    
    \item $\frac{1}{3}$ 
    
    \item $\frac{1}{12}$ 
    
    \item $\frac{1}{36}$ 
\end{enumerate}
\solution
		%\input{ncert/12/13/2/17/defs.tex}
	\item A bag contains 4 red and 4 black balls, another bag contains 2 red and 6 black balls. One of the two bags is selected at random and a ball is drawn from the bag which is found to be red. Find the probability that the ball is drawn from the first bag.
\\
\solution
		%\input{ncert/12/13/3/2/main.tex}
  \item
  Cards with numbers 2 to 101 are placed in a box. A card is selected at random.Find the probability that the card has
\begin{enumerate}[label=(\roman*)]
	\item an even number 
	\item a square number
\end{enumerate}
\solution
%\input{exemplar/10/13/3/32/main.tex}
\item
The king, queen and jack of clubs are removed from a deck of 52 playing cards and then well shuffled. Now one card is drawn at random from the remaining cards.  Determine the probability that the card is
\begin{enumerate}[label=(\roman*)]
\item a club
\item 10 of hearts
\end{enumerate}
\solution
%\input{exemplar/10/13/3/29/main.tex}
\item A team of medical students doing their internship have to assist during surgeries
at a city hospital. The probabilities of surgeries rated as very complex, complex,
routine, simple or very simple are respectively, 0.15, 0.20, 0.31, 0.26, .08. Find
the probabilities that a particular surgery will be rated
\begin{enumerate}
	\item complex or very complex;
	\item neither very complex nor very simple;
	\item routine or complex
	\item routine or simple
\end{enumerate}
\solution
%\input{exemplar/11/16/3/8(1)/main.tex}
\item A card is selected from a pack of 52 cards.
\begin{enumerate}[label=(\alph*)]
    \item How many points are there in the sample space?
    \item Calculate the probability that the card is an ace of spades.
    \item Calculate the probability that the card is (i) an ace and (ii) black card.
\end{enumerate}
\solution
%\input{exemplar/11/16/3/4/main2.tex}
\item The probability that a non leap year selected at random will contain 53 sundays.
\\
\solution
%\input{exemplar/10/13/1/19/main.tex}
\item One of the four persons John, Rita, Aslam or Gurpreet will be promoted next
month. Consequently the sample space consists of four elementary outcomes
S = {John promoted, Rita promoted, Aslam promoted, Gurpreet promoted}
You are told that the chances of John’s promotion is same as that of Gurpreet,
Rita’s chances of promotion are twice as likely as Johns. Aslam’s chances are
four times that of John.
\begin{enumerate}
	\item Determine
	\begin{enumerate}
		\item P (John promoted)
		\item P (Rita promoted)
		\item P (Aslam promoted)
		\item P (Gurpreet promoted)
	\end{enumerate}
	\item If A = {John promoted or Gurpreet promoted}, find P (A).
\end{enumerate}
\solution
%\input{exemplar/11/16/3/10/main.tex}
\item A card is drawn from a deck of 52 cards. Find the probability of getting a king or a heart or a red card.\\
\solution
%\input{exemplar/11/16/3/15/main.tex}
\item The probability that a student will pass his examination is 0.73, the probability of
the student getting a compartment is 0.13, and the probability that the student will
either pass or get compartment is 0.96. State True or False.\\
\solution
%\input{exemplar/11/16/3/31/main.tex}
\item A card is selected from a pack of 52 cards\\
\begin{enumerate}[label=(\alph*)]
\item How many points are there in the sample space?
\item Calculate the probability that the cards is an ace of spades.
\item Calculate the probability that the card is (i) an ace (ii)black card.\\
\end{enumerate}
%\input{ncert/11/16/3/4_1/Prob_4.tex}
\item In a non-leap year, the probability of having 53 tuesdays or 53 wednesdays is\\
\solution
%\input{exemplar/11/16/3/18/main.tex}
\item There are 1000 sealed envelopes in a box, 10 of them contain a cash prize of
Rs 100 each, 100 of them contain a cash prize of Rs 50 each and 200 of them
contain a cash prize of Rs 10 each and rest do not contain any cash prize. If they
are well shuffled and an envelope is picked up out, what is the probability that it
contains no cash prize?\\
\solution
%\input{exemplar/10/13/3/34/main.tex}
\item 
A die is thrown and a card is selected at random from a deck of 52 playing cards. The probability of getting an even number on the die and a spade card.\\
\solution
%\input{exemplar/12/13/3/78/main.tex}
\item
If 4-digit numbers greater than 5,000 are randomly formed from the digits 0, 1, 3, 5, and 7, what is the probability of forming a number divisible by 5 when:
\begin{enumerate}
    \item The digits are repeated?
    \item The repetition of digits is not allowed?
\end{enumerate}
\solution
%\input{ncert/11/16/4/9/main.tex}
\item Consider the probability space $\brak{\Omega, \mathcal{G}, P}$ where $\Omega = [0,2]$ and $\mathcal{G} = \cbrak{\phi, \Omega, [0,1], (1,2]}$. Let $X$ and $Y$ be two functions on $\Omega$ defined as
\begin{align*}
    X(\omega) = 
    \begin{cases}
        1 & \text{if }\omega \in [0, 1]\\
        2 & \text{if }\omega \in (1, 2]
    \end{cases}
\end{align*}
and
\begin{align*}
    Y(\omega) = 
    \begin{cases}
        2 & \text{if }\omega \in [0, 1.5]\\
        3 & \text{if }\omega \in (1.5, 2].
    \end{cases}
\end{align*}
Then which one of the following statements is true?
\begin{enumerate}
    \item [(A)] $X$ is a random variable with respect to $\mathcal{G}$, but $Y$ is not a random variable with respect to $\mathcal{G}$.
    \item [(B)] $Y$ is a random variable with respect to $\mathcal{G}$, but $X$ is not a random variable with respect to $\mathcal{G}$.
    \item [(C)] Neither $X$ nor $Y$ is a random variable with respect to $\mathcal{G}$.
    \item [(D)] Both $X$ and $Y$ are random variables with respect to $\mathcal{G}$.
\end{enumerate} \hfill (GATE ST 2023)\\
\solution
%\input{gate/ST/2023/14/main.tex}
	\item  A die is loaded in such a way that each odd number is twice as likely to occur as
each even number. Find $P(G)$, where $G$ is the event that a number greater than
3 occurs on a single roll of the die.
\\
\solution
		%\input{exemplar/11/16/3/5/main.tex}
	\item All the jacks, queens and kings are removed from a deck of 52 playing cards. The remaining cards are well shuffled and then one card is drawn at random. Giving ace a value 1 similar value for other cards, find the probability that the card has a value 
		\begin{enumerate}
			\item 7
			\item greater than 7
			\item less than 7
		\end{enumerate}
		%\input{exemplar/10/13/3/30/main.tex}
  \item A Lot consists of 48 mobile phones of which 42 are good, 3 have only minor defects and 3 have major defects.Varnika will buy a phone if it is good but the trader will only buy a mobile if it has no major defects. One phone is selected at random from the lot. What is the probability that it is
\begin{enumerate}
	\item acceptable to Varnika?
            \item acceptable to the trader?
\end{enumerate}
\solution
	%\input{exemplar/10/13/3/40/main.tex}
 \item A student says that if you throw a die, it will show up 1 or not 1. Therefore, the probability of getting 1 and the probability of getting 'not 1' each is equal to $\frac{1}{2}$. Is this correct? Give reasons.\\
 \solution
        %\input{exemplar/10/13/2/9/main.tex}
   \item Four candidates A, B, C, D have ap-
plied for the assignment to coach a school cricket
team. If A is twice as likely to be selected as B, and
B and C are given about the same chance of being
selected, while C is twice as likely to be selected
as D, what are the probabilities that
\begin{enumerate}
\item C will be selected?
\item A will not be selected?
\end{enumerate}
	%\input{exemplar/11/16/3/9/main.tex}
 \item A bag contain 24 balls of which $x$ balls are red, $2x$ are white and $3x$ are blue. A ball is selected at random, What is the probability that it is
\begin{enumerate}[label=\alph*)]
\item not red ?
\item white ?
\end{enumerate}
%\input{exemplar/10/13/3/41/main.tex}
If the letters of the word ASSASSINATION are arranged at random. Find the Probability that
\begin{enumerate}[label=(\alph*)]
\item Four $S's$ come consecutively in the word
\item Two  $I's$ and two $N's$ come together
\item All $A's$ are not coming together
\item No two $A's$ are coming together
\end{enumerate}
%\input{exemplar/11/16/3/14/main.tex}
	\item One urn contains two black balls (labelled B1 and B2) and one white ball. A
	second urn contains one black ball and two white balls (labelled W1 and W2).
	Suppose the following experiment is performed. One of the two urns is chosen
	at random. Next a ball is randomly chosen from the urn. Then a second ball is
	chosen at random from the same urn without replacing the first ball.
	
	\begin{enumerate}
	\item What is the probability that two black balls are chosen?
	
	\item What is the probability that two balls of opposite colour are chosen?
	\end{enumerate}
	\solution
	%\input{exemplar/11/16/3/12/main1.tex}
\end{enumerate}

\item In a certain lottery 10,000 tickets are sold and ten equal prizes are awarded. What is the probability of not getting a prize if you buy (a) one ticket (b) two tickets (c) 10 tickets ?	
\\
\solution
		%\begin{enumerate}[label=\thesection.\arabic*,ref=\thesection.\theenumi]
	\item One card is drawn from a well-shuffled deck of 52 cards. Find the probability of getting
\begin{enumerate}
\item A king of red colour 
\item A face card 
\item A red face card
\item The jack of hearts
\item A spade
\item The queen of diamonds

\end{enumerate}
\solution
		%\input{ncert/10/15/1/14/main.tex}
	\item Five cards—the ten, jack, queen, king and ace of diamonds, are well-shuffled with their face downwards. One card is then picked up at random.
\begin{enumerate}
\item
What is the probability that the card is the queen? 
\item
If the queen is drawn and put aside, what is the probability that the second card picked up is (a) an ace? (b) a queen?\\
\end{enumerate}
\solution
		%\input{ncert/10/15/1/15/defs.tex}
	\item A bag contains $5$ red balls and some blue balls. If the probability of drawing a blue ball is double that if a red ball, determine the number of blue balls in the bag. 
		\\
\solution
		%\input{ncert/10/15/2/3/defs.tex}
	\item A card is selected from a pack of 52 cards.
 \begin{enumerate}[label=(\alph*)] 
                 \item How many points are there in the sample space?
                 \item Calculate the probability that the card is an ace of spades.
                 \item Calculate the probability that the card is (i) an ace and (ii) black card.
 \end{enumerate}
\solution
		%\input{ncert/11/16/3/4/main.tex}
\item Four cards are drawn from a well-shuffled deck of 52 cards. What is the probability of obtaining 3 diamonds and one spade.
\\
\solution
		%\input{ncert/11/16/4/2/defs.tex}
\item In a certain lottery 10,000 tickets are sold and ten equal prizes are awarded. What is the probability of not getting a prize if you buy (a) one ticket (b) two tickets (c) 10 tickets ?	
\\
\solution
		%\input{ncert/11/16/4/4/defs.tex}
		%
\item 
Out of 100 students, two sections of 40 and 60 are formed. If you and your friend are among the 100 students, what is the probability that
\begin{enumerate}
\item you both enter the same section?
\item you both enter the different sections?
\end{enumerate}
\solution
		%\input{ncert/11/16/4/5/defs.tex}
	\item 
The number lock of a suitcase has 4 wheels each labelled with ten digits i.e. from 0 to 9.The lock opens with a sequence of four digits with no repeats.What is the probability of a person getting the right sequence to open the suitcase.
\\
\solution
		%\input{ncert/11/16/4/10/defs.tex}
		%
\item 
Two cards are drawn at random and without replacement from a pack of 52 playing cards. Find the probability that both the cards are black.
\\
\solution
		%\input{ncert/12/13/2/2/defs.tex}
		\item A box of oranges is inspected by examining three randomly selected oranges drawn without replacement. If all the three oranges are good, the box is approved for sale, otherwise, it is rejected. Find the probability that a box containing 15 oranges out of which 12 are good and 3 are bad ones will be approved for sale.
		\label{ncert/12/13/2/3/defs.tex}
		\item Two balls are drawn at random with replacement from a box containing 10 black and 8 red balls. Find the probability that
		\label{ncert/12/13/2/12}
\begin{enumerate}
\item both balls are red.
\item first ball is black and second is red.
\item one of them is black and other is red.
\end{enumerate}

\item In a hostel, 60\% of the students read Hindi newspaper, 40\% read English newspaper and 20\% read both Hindi and English newspapers. A student is selected at random.
		\label{ncert/12/13/2/15}
\begin{enumerate}
\item Find the probability that she reads neither Hindi nor English newspapers.
\item If she reads Hindi newspaper, find the probability that she reads English newspaper.
\item If she reads English newspaper, find the probability that she reads Hindi newspaper.\\
\end{enumerate}
\item The probability of obtaining an even prime number on each die, when a pair of dice is rolled is 
\begin{enumerate}
    \item $0$ 
    
    \item $\frac{1}{3}$ 
    
    \item $\frac{1}{12}$ 
    
    \item $\frac{1}{36}$ 
\end{enumerate}
\solution
		%\input{ncert/12/13/2/17/defs.tex}
	\item A bag contains 4 red and 4 black balls, another bag contains 2 red and 6 black balls. One of the two bags is selected at random and a ball is drawn from the bag which is found to be red. Find the probability that the ball is drawn from the first bag.
\\
\solution
		%\input{ncert/12/13/3/2/main.tex}
  \item
  Cards with numbers 2 to 101 are placed in a box. A card is selected at random.Find the probability that the card has
\begin{enumerate}[label=(\roman*)]
	\item an even number 
	\item a square number
\end{enumerate}
\solution
%\input{exemplar/10/13/3/32/main.tex}
\item
The king, queen and jack of clubs are removed from a deck of 52 playing cards and then well shuffled. Now one card is drawn at random from the remaining cards.  Determine the probability that the card is
\begin{enumerate}[label=(\roman*)]
\item a club
\item 10 of hearts
\end{enumerate}
\solution
%\input{exemplar/10/13/3/29/main.tex}
\item A team of medical students doing their internship have to assist during surgeries
at a city hospital. The probabilities of surgeries rated as very complex, complex,
routine, simple or very simple are respectively, 0.15, 0.20, 0.31, 0.26, .08. Find
the probabilities that a particular surgery will be rated
\begin{enumerate}
	\item complex or very complex;
	\item neither very complex nor very simple;
	\item routine or complex
	\item routine or simple
\end{enumerate}
\solution
%\input{exemplar/11/16/3/8(1)/main.tex}
\item A card is selected from a pack of 52 cards.
\begin{enumerate}[label=(\alph*)]
    \item How many points are there in the sample space?
    \item Calculate the probability that the card is an ace of spades.
    \item Calculate the probability that the card is (i) an ace and (ii) black card.
\end{enumerate}
\solution
%\input{exemplar/11/16/3/4/main2.tex}
\item The probability that a non leap year selected at random will contain 53 sundays.
\\
\solution
%\input{exemplar/10/13/1/19/main.tex}
\item One of the four persons John, Rita, Aslam or Gurpreet will be promoted next
month. Consequently the sample space consists of four elementary outcomes
S = {John promoted, Rita promoted, Aslam promoted, Gurpreet promoted}
You are told that the chances of John’s promotion is same as that of Gurpreet,
Rita’s chances of promotion are twice as likely as Johns. Aslam’s chances are
four times that of John.
\begin{enumerate}
	\item Determine
	\begin{enumerate}
		\item P (John promoted)
		\item P (Rita promoted)
		\item P (Aslam promoted)
		\item P (Gurpreet promoted)
	\end{enumerate}
	\item If A = {John promoted or Gurpreet promoted}, find P (A).
\end{enumerate}
\solution
%\input{exemplar/11/16/3/10/main.tex}
\item A card is drawn from a deck of 52 cards. Find the probability of getting a king or a heart or a red card.\\
\solution
%\input{exemplar/11/16/3/15/main.tex}
\item The probability that a student will pass his examination is 0.73, the probability of
the student getting a compartment is 0.13, and the probability that the student will
either pass or get compartment is 0.96. State True or False.\\
\solution
%\input{exemplar/11/16/3/31/main.tex}
\item A card is selected from a pack of 52 cards\\
\begin{enumerate}[label=(\alph*)]
\item How many points are there in the sample space?
\item Calculate the probability that the cards is an ace of spades.
\item Calculate the probability that the card is (i) an ace (ii)black card.\\
\end{enumerate}
%\input{ncert/11/16/3/4_1/Prob_4.tex}
\item In a non-leap year, the probability of having 53 tuesdays or 53 wednesdays is\\
\solution
%\input{exemplar/11/16/3/18/main.tex}
\item There are 1000 sealed envelopes in a box, 10 of them contain a cash prize of
Rs 100 each, 100 of them contain a cash prize of Rs 50 each and 200 of them
contain a cash prize of Rs 10 each and rest do not contain any cash prize. If they
are well shuffled and an envelope is picked up out, what is the probability that it
contains no cash prize?\\
\solution
%\input{exemplar/10/13/3/34/main.tex}
\item 
A die is thrown and a card is selected at random from a deck of 52 playing cards. The probability of getting an even number on the die and a spade card.\\
\solution
%\input{exemplar/12/13/3/78/main.tex}
\item
If 4-digit numbers greater than 5,000 are randomly formed from the digits 0, 1, 3, 5, and 7, what is the probability of forming a number divisible by 5 when:
\begin{enumerate}
    \item The digits are repeated?
    \item The repetition of digits is not allowed?
\end{enumerate}
\solution
%\input{ncert/11/16/4/9/main.tex}
\item Consider the probability space $\brak{\Omega, \mathcal{G}, P}$ where $\Omega = [0,2]$ and $\mathcal{G} = \cbrak{\phi, \Omega, [0,1], (1,2]}$. Let $X$ and $Y$ be two functions on $\Omega$ defined as
\begin{align*}
    X(\omega) = 
    \begin{cases}
        1 & \text{if }\omega \in [0, 1]\\
        2 & \text{if }\omega \in (1, 2]
    \end{cases}
\end{align*}
and
\begin{align*}
    Y(\omega) = 
    \begin{cases}
        2 & \text{if }\omega \in [0, 1.5]\\
        3 & \text{if }\omega \in (1.5, 2].
    \end{cases}
\end{align*}
Then which one of the following statements is true?
\begin{enumerate}
    \item [(A)] $X$ is a random variable with respect to $\mathcal{G}$, but $Y$ is not a random variable with respect to $\mathcal{G}$.
    \item [(B)] $Y$ is a random variable with respect to $\mathcal{G}$, but $X$ is not a random variable with respect to $\mathcal{G}$.
    \item [(C)] Neither $X$ nor $Y$ is a random variable with respect to $\mathcal{G}$.
    \item [(D)] Both $X$ and $Y$ are random variables with respect to $\mathcal{G}$.
\end{enumerate} \hfill (GATE ST 2023)\\
\solution
%\input{gate/ST/2023/14/main.tex}
	\item  A die is loaded in such a way that each odd number is twice as likely to occur as
each even number. Find $P(G)$, where $G$ is the event that a number greater than
3 occurs on a single roll of the die.
\\
\solution
		%\input{exemplar/11/16/3/5/main.tex}
	\item All the jacks, queens and kings are removed from a deck of 52 playing cards. The remaining cards are well shuffled and then one card is drawn at random. Giving ace a value 1 similar value for other cards, find the probability that the card has a value 
		\begin{enumerate}
			\item 7
			\item greater than 7
			\item less than 7
		\end{enumerate}
		%\input{exemplar/10/13/3/30/main.tex}
  \item A Lot consists of 48 mobile phones of which 42 are good, 3 have only minor defects and 3 have major defects.Varnika will buy a phone if it is good but the trader will only buy a mobile if it has no major defects. One phone is selected at random from the lot. What is the probability that it is
\begin{enumerate}
	\item acceptable to Varnika?
            \item acceptable to the trader?
\end{enumerate}
\solution
	%\input{exemplar/10/13/3/40/main.tex}
 \item A student says that if you throw a die, it will show up 1 or not 1. Therefore, the probability of getting 1 and the probability of getting 'not 1' each is equal to $\frac{1}{2}$. Is this correct? Give reasons.\\
 \solution
        %\input{exemplar/10/13/2/9/main.tex}
   \item Four candidates A, B, C, D have ap-
plied for the assignment to coach a school cricket
team. If A is twice as likely to be selected as B, and
B and C are given about the same chance of being
selected, while C is twice as likely to be selected
as D, what are the probabilities that
\begin{enumerate}
\item C will be selected?
\item A will not be selected?
\end{enumerate}
	%\input{exemplar/11/16/3/9/main.tex}
 \item A bag contain 24 balls of which $x$ balls are red, $2x$ are white and $3x$ are blue. A ball is selected at random, What is the probability that it is
\begin{enumerate}[label=\alph*)]
\item not red ?
\item white ?
\end{enumerate}
%\input{exemplar/10/13/3/41/main.tex}
If the letters of the word ASSASSINATION are arranged at random. Find the Probability that
\begin{enumerate}[label=(\alph*)]
\item Four $S's$ come consecutively in the word
\item Two  $I's$ and two $N's$ come together
\item All $A's$ are not coming together
\item No two $A's$ are coming together
\end{enumerate}
%\input{exemplar/11/16/3/14/main.tex}
	\item One urn contains two black balls (labelled B1 and B2) and one white ball. A
	second urn contains one black ball and two white balls (labelled W1 and W2).
	Suppose the following experiment is performed. One of the two urns is chosen
	at random. Next a ball is randomly chosen from the urn. Then a second ball is
	chosen at random from the same urn without replacing the first ball.
	
	\begin{enumerate}
	\item What is the probability that two black balls are chosen?
	
	\item What is the probability that two balls of opposite colour are chosen?
	\end{enumerate}
	\solution
	%\input{exemplar/11/16/3/12/main1.tex}
\end{enumerate}

		%
\item 
Out of 100 students, two sections of 40 and 60 are formed. If you and your friend are among the 100 students, what is the probability that
\begin{enumerate}
\item you both enter the same section?
\item you both enter the different sections?
\end{enumerate}
\solution
		%\begin{enumerate}[label=\thesection.\arabic*,ref=\thesection.\theenumi]
	\item One card is drawn from a well-shuffled deck of 52 cards. Find the probability of getting
\begin{enumerate}
\item A king of red colour 
\item A face card 
\item A red face card
\item The jack of hearts
\item A spade
\item The queen of diamonds

\end{enumerate}
\solution
		%\input{ncert/10/15/1/14/main.tex}
	\item Five cards—the ten, jack, queen, king and ace of diamonds, are well-shuffled with their face downwards. One card is then picked up at random.
\begin{enumerate}
\item
What is the probability that the card is the queen? 
\item
If the queen is drawn and put aside, what is the probability that the second card picked up is (a) an ace? (b) a queen?\\
\end{enumerate}
\solution
		%\input{ncert/10/15/1/15/defs.tex}
	\item A bag contains $5$ red balls and some blue balls. If the probability of drawing a blue ball is double that if a red ball, determine the number of blue balls in the bag. 
		\\
\solution
		%\input{ncert/10/15/2/3/defs.tex}
	\item A card is selected from a pack of 52 cards.
 \begin{enumerate}[label=(\alph*)] 
                 \item How many points are there in the sample space?
                 \item Calculate the probability that the card is an ace of spades.
                 \item Calculate the probability that the card is (i) an ace and (ii) black card.
 \end{enumerate}
\solution
		%\input{ncert/11/16/3/4/main.tex}
\item Four cards are drawn from a well-shuffled deck of 52 cards. What is the probability of obtaining 3 diamonds and one spade.
\\
\solution
		%\input{ncert/11/16/4/2/defs.tex}
\item In a certain lottery 10,000 tickets are sold and ten equal prizes are awarded. What is the probability of not getting a prize if you buy (a) one ticket (b) two tickets (c) 10 tickets ?	
\\
\solution
		%\input{ncert/11/16/4/4/defs.tex}
		%
\item 
Out of 100 students, two sections of 40 and 60 are formed. If you and your friend are among the 100 students, what is the probability that
\begin{enumerate}
\item you both enter the same section?
\item you both enter the different sections?
\end{enumerate}
\solution
		%\input{ncert/11/16/4/5/defs.tex}
	\item 
The number lock of a suitcase has 4 wheels each labelled with ten digits i.e. from 0 to 9.The lock opens with a sequence of four digits with no repeats.What is the probability of a person getting the right sequence to open the suitcase.
\\
\solution
		%\input{ncert/11/16/4/10/defs.tex}
		%
\item 
Two cards are drawn at random and without replacement from a pack of 52 playing cards. Find the probability that both the cards are black.
\\
\solution
		%\input{ncert/12/13/2/2/defs.tex}
		\item A box of oranges is inspected by examining three randomly selected oranges drawn without replacement. If all the three oranges are good, the box is approved for sale, otherwise, it is rejected. Find the probability that a box containing 15 oranges out of which 12 are good and 3 are bad ones will be approved for sale.
		\label{ncert/12/13/2/3/defs.tex}
		\item Two balls are drawn at random with replacement from a box containing 10 black and 8 red balls. Find the probability that
		\label{ncert/12/13/2/12}
\begin{enumerate}
\item both balls are red.
\item first ball is black and second is red.
\item one of them is black and other is red.
\end{enumerate}

\item In a hostel, 60\% of the students read Hindi newspaper, 40\% read English newspaper and 20\% read both Hindi and English newspapers. A student is selected at random.
		\label{ncert/12/13/2/15}
\begin{enumerate}
\item Find the probability that she reads neither Hindi nor English newspapers.
\item If she reads Hindi newspaper, find the probability that she reads English newspaper.
\item If she reads English newspaper, find the probability that she reads Hindi newspaper.\\
\end{enumerate}
\item The probability of obtaining an even prime number on each die, when a pair of dice is rolled is 
\begin{enumerate}
    \item $0$ 
    
    \item $\frac{1}{3}$ 
    
    \item $\frac{1}{12}$ 
    
    \item $\frac{1}{36}$ 
\end{enumerate}
\solution
		%\input{ncert/12/13/2/17/defs.tex}
	\item A bag contains 4 red and 4 black balls, another bag contains 2 red and 6 black balls. One of the two bags is selected at random and a ball is drawn from the bag which is found to be red. Find the probability that the ball is drawn from the first bag.
\\
\solution
		%\input{ncert/12/13/3/2/main.tex}
  \item
  Cards with numbers 2 to 101 are placed in a box. A card is selected at random.Find the probability that the card has
\begin{enumerate}[label=(\roman*)]
	\item an even number 
	\item a square number
\end{enumerate}
\solution
%\input{exemplar/10/13/3/32/main.tex}
\item
The king, queen and jack of clubs are removed from a deck of 52 playing cards and then well shuffled. Now one card is drawn at random from the remaining cards.  Determine the probability that the card is
\begin{enumerate}[label=(\roman*)]
\item a club
\item 10 of hearts
\end{enumerate}
\solution
%\input{exemplar/10/13/3/29/main.tex}
\item A team of medical students doing their internship have to assist during surgeries
at a city hospital. The probabilities of surgeries rated as very complex, complex,
routine, simple or very simple are respectively, 0.15, 0.20, 0.31, 0.26, .08. Find
the probabilities that a particular surgery will be rated
\begin{enumerate}
	\item complex or very complex;
	\item neither very complex nor very simple;
	\item routine or complex
	\item routine or simple
\end{enumerate}
\solution
%\input{exemplar/11/16/3/8(1)/main.tex}
\item A card is selected from a pack of 52 cards.
\begin{enumerate}[label=(\alph*)]
    \item How many points are there in the sample space?
    \item Calculate the probability that the card is an ace of spades.
    \item Calculate the probability that the card is (i) an ace and (ii) black card.
\end{enumerate}
\solution
%\input{exemplar/11/16/3/4/main2.tex}
\item The probability that a non leap year selected at random will contain 53 sundays.
\\
\solution
%\input{exemplar/10/13/1/19/main.tex}
\item One of the four persons John, Rita, Aslam or Gurpreet will be promoted next
month. Consequently the sample space consists of four elementary outcomes
S = {John promoted, Rita promoted, Aslam promoted, Gurpreet promoted}
You are told that the chances of John’s promotion is same as that of Gurpreet,
Rita’s chances of promotion are twice as likely as Johns. Aslam’s chances are
four times that of John.
\begin{enumerate}
	\item Determine
	\begin{enumerate}
		\item P (John promoted)
		\item P (Rita promoted)
		\item P (Aslam promoted)
		\item P (Gurpreet promoted)
	\end{enumerate}
	\item If A = {John promoted or Gurpreet promoted}, find P (A).
\end{enumerate}
\solution
%\input{exemplar/11/16/3/10/main.tex}
\item A card is drawn from a deck of 52 cards. Find the probability of getting a king or a heart or a red card.\\
\solution
%\input{exemplar/11/16/3/15/main.tex}
\item The probability that a student will pass his examination is 0.73, the probability of
the student getting a compartment is 0.13, and the probability that the student will
either pass or get compartment is 0.96. State True or False.\\
\solution
%\input{exemplar/11/16/3/31/main.tex}
\item A card is selected from a pack of 52 cards\\
\begin{enumerate}[label=(\alph*)]
\item How many points are there in the sample space?
\item Calculate the probability that the cards is an ace of spades.
\item Calculate the probability that the card is (i) an ace (ii)black card.\\
\end{enumerate}
%\input{ncert/11/16/3/4_1/Prob_4.tex}
\item In a non-leap year, the probability of having 53 tuesdays or 53 wednesdays is\\
\solution
%\input{exemplar/11/16/3/18/main.tex}
\item There are 1000 sealed envelopes in a box, 10 of them contain a cash prize of
Rs 100 each, 100 of them contain a cash prize of Rs 50 each and 200 of them
contain a cash prize of Rs 10 each and rest do not contain any cash prize. If they
are well shuffled and an envelope is picked up out, what is the probability that it
contains no cash prize?\\
\solution
%\input{exemplar/10/13/3/34/main.tex}
\item 
A die is thrown and a card is selected at random from a deck of 52 playing cards. The probability of getting an even number on the die and a spade card.\\
\solution
%\input{exemplar/12/13/3/78/main.tex}
\item
If 4-digit numbers greater than 5,000 are randomly formed from the digits 0, 1, 3, 5, and 7, what is the probability of forming a number divisible by 5 when:
\begin{enumerate}
    \item The digits are repeated?
    \item The repetition of digits is not allowed?
\end{enumerate}
\solution
%\input{ncert/11/16/4/9/main.tex}
\item Consider the probability space $\brak{\Omega, \mathcal{G}, P}$ where $\Omega = [0,2]$ and $\mathcal{G} = \cbrak{\phi, \Omega, [0,1], (1,2]}$. Let $X$ and $Y$ be two functions on $\Omega$ defined as
\begin{align*}
    X(\omega) = 
    \begin{cases}
        1 & \text{if }\omega \in [0, 1]\\
        2 & \text{if }\omega \in (1, 2]
    \end{cases}
\end{align*}
and
\begin{align*}
    Y(\omega) = 
    \begin{cases}
        2 & \text{if }\omega \in [0, 1.5]\\
        3 & \text{if }\omega \in (1.5, 2].
    \end{cases}
\end{align*}
Then which one of the following statements is true?
\begin{enumerate}
    \item [(A)] $X$ is a random variable with respect to $\mathcal{G}$, but $Y$ is not a random variable with respect to $\mathcal{G}$.
    \item [(B)] $Y$ is a random variable with respect to $\mathcal{G}$, but $X$ is not a random variable with respect to $\mathcal{G}$.
    \item [(C)] Neither $X$ nor $Y$ is a random variable with respect to $\mathcal{G}$.
    \item [(D)] Both $X$ and $Y$ are random variables with respect to $\mathcal{G}$.
\end{enumerate} \hfill (GATE ST 2023)\\
\solution
%\input{gate/ST/2023/14/main.tex}
	\item  A die is loaded in such a way that each odd number is twice as likely to occur as
each even number. Find $P(G)$, where $G$ is the event that a number greater than
3 occurs on a single roll of the die.
\\
\solution
		%\input{exemplar/11/16/3/5/main.tex}
	\item All the jacks, queens and kings are removed from a deck of 52 playing cards. The remaining cards are well shuffled and then one card is drawn at random. Giving ace a value 1 similar value for other cards, find the probability that the card has a value 
		\begin{enumerate}
			\item 7
			\item greater than 7
			\item less than 7
		\end{enumerate}
		%\input{exemplar/10/13/3/30/main.tex}
  \item A Lot consists of 48 mobile phones of which 42 are good, 3 have only minor defects and 3 have major defects.Varnika will buy a phone if it is good but the trader will only buy a mobile if it has no major defects. One phone is selected at random from the lot. What is the probability that it is
\begin{enumerate}
	\item acceptable to Varnika?
            \item acceptable to the trader?
\end{enumerate}
\solution
	%\input{exemplar/10/13/3/40/main.tex}
 \item A student says that if you throw a die, it will show up 1 or not 1. Therefore, the probability of getting 1 and the probability of getting 'not 1' each is equal to $\frac{1}{2}$. Is this correct? Give reasons.\\
 \solution
        %\input{exemplar/10/13/2/9/main.tex}
   \item Four candidates A, B, C, D have ap-
plied for the assignment to coach a school cricket
team. If A is twice as likely to be selected as B, and
B and C are given about the same chance of being
selected, while C is twice as likely to be selected
as D, what are the probabilities that
\begin{enumerate}
\item C will be selected?
\item A will not be selected?
\end{enumerate}
	%\input{exemplar/11/16/3/9/main.tex}
 \item A bag contain 24 balls of which $x$ balls are red, $2x$ are white and $3x$ are blue. A ball is selected at random, What is the probability that it is
\begin{enumerate}[label=\alph*)]
\item not red ?
\item white ?
\end{enumerate}
%\input{exemplar/10/13/3/41/main.tex}
If the letters of the word ASSASSINATION are arranged at random. Find the Probability that
\begin{enumerate}[label=(\alph*)]
\item Four $S's$ come consecutively in the word
\item Two  $I's$ and two $N's$ come together
\item All $A's$ are not coming together
\item No two $A's$ are coming together
\end{enumerate}
%\input{exemplar/11/16/3/14/main.tex}
	\item One urn contains two black balls (labelled B1 and B2) and one white ball. A
	second urn contains one black ball and two white balls (labelled W1 and W2).
	Suppose the following experiment is performed. One of the two urns is chosen
	at random. Next a ball is randomly chosen from the urn. Then a second ball is
	chosen at random from the same urn without replacing the first ball.
	
	\begin{enumerate}
	\item What is the probability that two black balls are chosen?
	
	\item What is the probability that two balls of opposite colour are chosen?
	\end{enumerate}
	\solution
	%\input{exemplar/11/16/3/12/main1.tex}
\end{enumerate}

	\item 
The number lock of a suitcase has 4 wheels each labelled with ten digits i.e. from 0 to 9.The lock opens with a sequence of four digits with no repeats.What is the probability of a person getting the right sequence to open the suitcase.
\\
\solution
		%\begin{enumerate}[label=\thesection.\arabic*,ref=\thesection.\theenumi]
	\item One card is drawn from a well-shuffled deck of 52 cards. Find the probability of getting
\begin{enumerate}
\item A king of red colour 
\item A face card 
\item A red face card
\item The jack of hearts
\item A spade
\item The queen of diamonds

\end{enumerate}
\solution
		%\input{ncert/10/15/1/14/main.tex}
	\item Five cards—the ten, jack, queen, king and ace of diamonds, are well-shuffled with their face downwards. One card is then picked up at random.
\begin{enumerate}
\item
What is the probability that the card is the queen? 
\item
If the queen is drawn and put aside, what is the probability that the second card picked up is (a) an ace? (b) a queen?\\
\end{enumerate}
\solution
		%\input{ncert/10/15/1/15/defs.tex}
	\item A bag contains $5$ red balls and some blue balls. If the probability of drawing a blue ball is double that if a red ball, determine the number of blue balls in the bag. 
		\\
\solution
		%\input{ncert/10/15/2/3/defs.tex}
	\item A card is selected from a pack of 52 cards.
 \begin{enumerate}[label=(\alph*)] 
                 \item How many points are there in the sample space?
                 \item Calculate the probability that the card is an ace of spades.
                 \item Calculate the probability that the card is (i) an ace and (ii) black card.
 \end{enumerate}
\solution
		%\input{ncert/11/16/3/4/main.tex}
\item Four cards are drawn from a well-shuffled deck of 52 cards. What is the probability of obtaining 3 diamonds and one spade.
\\
\solution
		%\input{ncert/11/16/4/2/defs.tex}
\item In a certain lottery 10,000 tickets are sold and ten equal prizes are awarded. What is the probability of not getting a prize if you buy (a) one ticket (b) two tickets (c) 10 tickets ?	
\\
\solution
		%\input{ncert/11/16/4/4/defs.tex}
		%
\item 
Out of 100 students, two sections of 40 and 60 are formed. If you and your friend are among the 100 students, what is the probability that
\begin{enumerate}
\item you both enter the same section?
\item you both enter the different sections?
\end{enumerate}
\solution
		%\input{ncert/11/16/4/5/defs.tex}
	\item 
The number lock of a suitcase has 4 wheels each labelled with ten digits i.e. from 0 to 9.The lock opens with a sequence of four digits with no repeats.What is the probability of a person getting the right sequence to open the suitcase.
\\
\solution
		%\input{ncert/11/16/4/10/defs.tex}
		%
\item 
Two cards are drawn at random and without replacement from a pack of 52 playing cards. Find the probability that both the cards are black.
\\
\solution
		%\input{ncert/12/13/2/2/defs.tex}
		\item A box of oranges is inspected by examining three randomly selected oranges drawn without replacement. If all the three oranges are good, the box is approved for sale, otherwise, it is rejected. Find the probability that a box containing 15 oranges out of which 12 are good and 3 are bad ones will be approved for sale.
		\label{ncert/12/13/2/3/defs.tex}
		\item Two balls are drawn at random with replacement from a box containing 10 black and 8 red balls. Find the probability that
		\label{ncert/12/13/2/12}
\begin{enumerate}
\item both balls are red.
\item first ball is black and second is red.
\item one of them is black and other is red.
\end{enumerate}

\item In a hostel, 60\% of the students read Hindi newspaper, 40\% read English newspaper and 20\% read both Hindi and English newspapers. A student is selected at random.
		\label{ncert/12/13/2/15}
\begin{enumerate}
\item Find the probability that she reads neither Hindi nor English newspapers.
\item If she reads Hindi newspaper, find the probability that she reads English newspaper.
\item If she reads English newspaper, find the probability that she reads Hindi newspaper.\\
\end{enumerate}
\item The probability of obtaining an even prime number on each die, when a pair of dice is rolled is 
\begin{enumerate}
    \item $0$ 
    
    \item $\frac{1}{3}$ 
    
    \item $\frac{1}{12}$ 
    
    \item $\frac{1}{36}$ 
\end{enumerate}
\solution
		%\input{ncert/12/13/2/17/defs.tex}
	\item A bag contains 4 red and 4 black balls, another bag contains 2 red and 6 black balls. One of the two bags is selected at random and a ball is drawn from the bag which is found to be red. Find the probability that the ball is drawn from the first bag.
\\
\solution
		%\input{ncert/12/13/3/2/main.tex}
  \item
  Cards with numbers 2 to 101 are placed in a box. A card is selected at random.Find the probability that the card has
\begin{enumerate}[label=(\roman*)]
	\item an even number 
	\item a square number
\end{enumerate}
\solution
%\input{exemplar/10/13/3/32/main.tex}
\item
The king, queen and jack of clubs are removed from a deck of 52 playing cards and then well shuffled. Now one card is drawn at random from the remaining cards.  Determine the probability that the card is
\begin{enumerate}[label=(\roman*)]
\item a club
\item 10 of hearts
\end{enumerate}
\solution
%\input{exemplar/10/13/3/29/main.tex}
\item A team of medical students doing their internship have to assist during surgeries
at a city hospital. The probabilities of surgeries rated as very complex, complex,
routine, simple or very simple are respectively, 0.15, 0.20, 0.31, 0.26, .08. Find
the probabilities that a particular surgery will be rated
\begin{enumerate}
	\item complex or very complex;
	\item neither very complex nor very simple;
	\item routine or complex
	\item routine or simple
\end{enumerate}
\solution
%\input{exemplar/11/16/3/8(1)/main.tex}
\item A card is selected from a pack of 52 cards.
\begin{enumerate}[label=(\alph*)]
    \item How many points are there in the sample space?
    \item Calculate the probability that the card is an ace of spades.
    \item Calculate the probability that the card is (i) an ace and (ii) black card.
\end{enumerate}
\solution
%\input{exemplar/11/16/3/4/main2.tex}
\item The probability that a non leap year selected at random will contain 53 sundays.
\\
\solution
%\input{exemplar/10/13/1/19/main.tex}
\item One of the four persons John, Rita, Aslam or Gurpreet will be promoted next
month. Consequently the sample space consists of four elementary outcomes
S = {John promoted, Rita promoted, Aslam promoted, Gurpreet promoted}
You are told that the chances of John’s promotion is same as that of Gurpreet,
Rita’s chances of promotion are twice as likely as Johns. Aslam’s chances are
four times that of John.
\begin{enumerate}
	\item Determine
	\begin{enumerate}
		\item P (John promoted)
		\item P (Rita promoted)
		\item P (Aslam promoted)
		\item P (Gurpreet promoted)
	\end{enumerate}
	\item If A = {John promoted or Gurpreet promoted}, find P (A).
\end{enumerate}
\solution
%\input{exemplar/11/16/3/10/main.tex}
\item A card is drawn from a deck of 52 cards. Find the probability of getting a king or a heart or a red card.\\
\solution
%\input{exemplar/11/16/3/15/main.tex}
\item The probability that a student will pass his examination is 0.73, the probability of
the student getting a compartment is 0.13, and the probability that the student will
either pass or get compartment is 0.96. State True or False.\\
\solution
%\input{exemplar/11/16/3/31/main.tex}
\item A card is selected from a pack of 52 cards\\
\begin{enumerate}[label=(\alph*)]
\item How many points are there in the sample space?
\item Calculate the probability that the cards is an ace of spades.
\item Calculate the probability that the card is (i) an ace (ii)black card.\\
\end{enumerate}
%\input{ncert/11/16/3/4_1/Prob_4.tex}
\item In a non-leap year, the probability of having 53 tuesdays or 53 wednesdays is\\
\solution
%\input{exemplar/11/16/3/18/main.tex}
\item There are 1000 sealed envelopes in a box, 10 of them contain a cash prize of
Rs 100 each, 100 of them contain a cash prize of Rs 50 each and 200 of them
contain a cash prize of Rs 10 each and rest do not contain any cash prize. If they
are well shuffled and an envelope is picked up out, what is the probability that it
contains no cash prize?\\
\solution
%\input{exemplar/10/13/3/34/main.tex}
\item 
A die is thrown and a card is selected at random from a deck of 52 playing cards. The probability of getting an even number on the die and a spade card.\\
\solution
%\input{exemplar/12/13/3/78/main.tex}
\item
If 4-digit numbers greater than 5,000 are randomly formed from the digits 0, 1, 3, 5, and 7, what is the probability of forming a number divisible by 5 when:
\begin{enumerate}
    \item The digits are repeated?
    \item The repetition of digits is not allowed?
\end{enumerate}
\solution
%\input{ncert/11/16/4/9/main.tex}
\item Consider the probability space $\brak{\Omega, \mathcal{G}, P}$ where $\Omega = [0,2]$ and $\mathcal{G} = \cbrak{\phi, \Omega, [0,1], (1,2]}$. Let $X$ and $Y$ be two functions on $\Omega$ defined as
\begin{align*}
    X(\omega) = 
    \begin{cases}
        1 & \text{if }\omega \in [0, 1]\\
        2 & \text{if }\omega \in (1, 2]
    \end{cases}
\end{align*}
and
\begin{align*}
    Y(\omega) = 
    \begin{cases}
        2 & \text{if }\omega \in [0, 1.5]\\
        3 & \text{if }\omega \in (1.5, 2].
    \end{cases}
\end{align*}
Then which one of the following statements is true?
\begin{enumerate}
    \item [(A)] $X$ is a random variable with respect to $\mathcal{G}$, but $Y$ is not a random variable with respect to $\mathcal{G}$.
    \item [(B)] $Y$ is a random variable with respect to $\mathcal{G}$, but $X$ is not a random variable with respect to $\mathcal{G}$.
    \item [(C)] Neither $X$ nor $Y$ is a random variable with respect to $\mathcal{G}$.
    \item [(D)] Both $X$ and $Y$ are random variables with respect to $\mathcal{G}$.
\end{enumerate} \hfill (GATE ST 2023)\\
\solution
%\input{gate/ST/2023/14/main.tex}
	\item  A die is loaded in such a way that each odd number is twice as likely to occur as
each even number. Find $P(G)$, where $G$ is the event that a number greater than
3 occurs on a single roll of the die.
\\
\solution
		%\input{exemplar/11/16/3/5/main.tex}
	\item All the jacks, queens and kings are removed from a deck of 52 playing cards. The remaining cards are well shuffled and then one card is drawn at random. Giving ace a value 1 similar value for other cards, find the probability that the card has a value 
		\begin{enumerate}
			\item 7
			\item greater than 7
			\item less than 7
		\end{enumerate}
		%\input{exemplar/10/13/3/30/main.tex}
  \item A Lot consists of 48 mobile phones of which 42 are good, 3 have only minor defects and 3 have major defects.Varnika will buy a phone if it is good but the trader will only buy a mobile if it has no major defects. One phone is selected at random from the lot. What is the probability that it is
\begin{enumerate}
	\item acceptable to Varnika?
            \item acceptable to the trader?
\end{enumerate}
\solution
	%\input{exemplar/10/13/3/40/main.tex}
 \item A student says that if you throw a die, it will show up 1 or not 1. Therefore, the probability of getting 1 and the probability of getting 'not 1' each is equal to $\frac{1}{2}$. Is this correct? Give reasons.\\
 \solution
        %\input{exemplar/10/13/2/9/main.tex}
   \item Four candidates A, B, C, D have ap-
plied for the assignment to coach a school cricket
team. If A is twice as likely to be selected as B, and
B and C are given about the same chance of being
selected, while C is twice as likely to be selected
as D, what are the probabilities that
\begin{enumerate}
\item C will be selected?
\item A will not be selected?
\end{enumerate}
	%\input{exemplar/11/16/3/9/main.tex}
 \item A bag contain 24 balls of which $x$ balls are red, $2x$ are white and $3x$ are blue. A ball is selected at random, What is the probability that it is
\begin{enumerate}[label=\alph*)]
\item not red ?
\item white ?
\end{enumerate}
%\input{exemplar/10/13/3/41/main.tex}
If the letters of the word ASSASSINATION are arranged at random. Find the Probability that
\begin{enumerate}[label=(\alph*)]
\item Four $S's$ come consecutively in the word
\item Two  $I's$ and two $N's$ come together
\item All $A's$ are not coming together
\item No two $A's$ are coming together
\end{enumerate}
%\input{exemplar/11/16/3/14/main.tex}
	\item One urn contains two black balls (labelled B1 and B2) and one white ball. A
	second urn contains one black ball and two white balls (labelled W1 and W2).
	Suppose the following experiment is performed. One of the two urns is chosen
	at random. Next a ball is randomly chosen from the urn. Then a second ball is
	chosen at random from the same urn without replacing the first ball.
	
	\begin{enumerate}
	\item What is the probability that two black balls are chosen?
	
	\item What is the probability that two balls of opposite colour are chosen?
	\end{enumerate}
	\solution
	%\input{exemplar/11/16/3/12/main1.tex}
\end{enumerate}

		%
\item 
Two cards are drawn at random and without replacement from a pack of 52 playing cards. Find the probability that both the cards are black.
\\
\solution
		%\begin{enumerate}[label=\thesection.\arabic*,ref=\thesection.\theenumi]
	\item One card is drawn from a well-shuffled deck of 52 cards. Find the probability of getting
\begin{enumerate}
\item A king of red colour 
\item A face card 
\item A red face card
\item The jack of hearts
\item A spade
\item The queen of diamonds

\end{enumerate}
\solution
		%\input{ncert/10/15/1/14/main.tex}
	\item Five cards—the ten, jack, queen, king and ace of diamonds, are well-shuffled with their face downwards. One card is then picked up at random.
\begin{enumerate}
\item
What is the probability that the card is the queen? 
\item
If the queen is drawn and put aside, what is the probability that the second card picked up is (a) an ace? (b) a queen?\\
\end{enumerate}
\solution
		%\input{ncert/10/15/1/15/defs.tex}
	\item A bag contains $5$ red balls and some blue balls. If the probability of drawing a blue ball is double that if a red ball, determine the number of blue balls in the bag. 
		\\
\solution
		%\input{ncert/10/15/2/3/defs.tex}
	\item A card is selected from a pack of 52 cards.
 \begin{enumerate}[label=(\alph*)] 
                 \item How many points are there in the sample space?
                 \item Calculate the probability that the card is an ace of spades.
                 \item Calculate the probability that the card is (i) an ace and (ii) black card.
 \end{enumerate}
\solution
		%\input{ncert/11/16/3/4/main.tex}
\item Four cards are drawn from a well-shuffled deck of 52 cards. What is the probability of obtaining 3 diamonds and one spade.
\\
\solution
		%\input{ncert/11/16/4/2/defs.tex}
\item In a certain lottery 10,000 tickets are sold and ten equal prizes are awarded. What is the probability of not getting a prize if you buy (a) one ticket (b) two tickets (c) 10 tickets ?	
\\
\solution
		%\input{ncert/11/16/4/4/defs.tex}
		%
\item 
Out of 100 students, two sections of 40 and 60 are formed. If you and your friend are among the 100 students, what is the probability that
\begin{enumerate}
\item you both enter the same section?
\item you both enter the different sections?
\end{enumerate}
\solution
		%\input{ncert/11/16/4/5/defs.tex}
	\item 
The number lock of a suitcase has 4 wheels each labelled with ten digits i.e. from 0 to 9.The lock opens with a sequence of four digits with no repeats.What is the probability of a person getting the right sequence to open the suitcase.
\\
\solution
		%\input{ncert/11/16/4/10/defs.tex}
		%
\item 
Two cards are drawn at random and without replacement from a pack of 52 playing cards. Find the probability that both the cards are black.
\\
\solution
		%\input{ncert/12/13/2/2/defs.tex}
		\item A box of oranges is inspected by examining three randomly selected oranges drawn without replacement. If all the three oranges are good, the box is approved for sale, otherwise, it is rejected. Find the probability that a box containing 15 oranges out of which 12 are good and 3 are bad ones will be approved for sale.
		\label{ncert/12/13/2/3/defs.tex}
		\item Two balls are drawn at random with replacement from a box containing 10 black and 8 red balls. Find the probability that
		\label{ncert/12/13/2/12}
\begin{enumerate}
\item both balls are red.
\item first ball is black and second is red.
\item one of them is black and other is red.
\end{enumerate}

\item In a hostel, 60\% of the students read Hindi newspaper, 40\% read English newspaper and 20\% read both Hindi and English newspapers. A student is selected at random.
		\label{ncert/12/13/2/15}
\begin{enumerate}
\item Find the probability that she reads neither Hindi nor English newspapers.
\item If she reads Hindi newspaper, find the probability that she reads English newspaper.
\item If she reads English newspaper, find the probability that she reads Hindi newspaper.\\
\end{enumerate}
\item The probability of obtaining an even prime number on each die, when a pair of dice is rolled is 
\begin{enumerate}
    \item $0$ 
    
    \item $\frac{1}{3}$ 
    
    \item $\frac{1}{12}$ 
    
    \item $\frac{1}{36}$ 
\end{enumerate}
\solution
		%\input{ncert/12/13/2/17/defs.tex}
	\item A bag contains 4 red and 4 black balls, another bag contains 2 red and 6 black balls. One of the two bags is selected at random and a ball is drawn from the bag which is found to be red. Find the probability that the ball is drawn from the first bag.
\\
\solution
		%\input{ncert/12/13/3/2/main.tex}
  \item
  Cards with numbers 2 to 101 are placed in a box. A card is selected at random.Find the probability that the card has
\begin{enumerate}[label=(\roman*)]
	\item an even number 
	\item a square number
\end{enumerate}
\solution
%\input{exemplar/10/13/3/32/main.tex}
\item
The king, queen and jack of clubs are removed from a deck of 52 playing cards and then well shuffled. Now one card is drawn at random from the remaining cards.  Determine the probability that the card is
\begin{enumerate}[label=(\roman*)]
\item a club
\item 10 of hearts
\end{enumerate}
\solution
%\input{exemplar/10/13/3/29/main.tex}
\item A team of medical students doing their internship have to assist during surgeries
at a city hospital. The probabilities of surgeries rated as very complex, complex,
routine, simple or very simple are respectively, 0.15, 0.20, 0.31, 0.26, .08. Find
the probabilities that a particular surgery will be rated
\begin{enumerate}
	\item complex or very complex;
	\item neither very complex nor very simple;
	\item routine or complex
	\item routine or simple
\end{enumerate}
\solution
%\input{exemplar/11/16/3/8(1)/main.tex}
\item A card is selected from a pack of 52 cards.
\begin{enumerate}[label=(\alph*)]
    \item How many points are there in the sample space?
    \item Calculate the probability that the card is an ace of spades.
    \item Calculate the probability that the card is (i) an ace and (ii) black card.
\end{enumerate}
\solution
%\input{exemplar/11/16/3/4/main2.tex}
\item The probability that a non leap year selected at random will contain 53 sundays.
\\
\solution
%\input{exemplar/10/13/1/19/main.tex}
\item One of the four persons John, Rita, Aslam or Gurpreet will be promoted next
month. Consequently the sample space consists of four elementary outcomes
S = {John promoted, Rita promoted, Aslam promoted, Gurpreet promoted}
You are told that the chances of John’s promotion is same as that of Gurpreet,
Rita’s chances of promotion are twice as likely as Johns. Aslam’s chances are
four times that of John.
\begin{enumerate}
	\item Determine
	\begin{enumerate}
		\item P (John promoted)
		\item P (Rita promoted)
		\item P (Aslam promoted)
		\item P (Gurpreet promoted)
	\end{enumerate}
	\item If A = {John promoted or Gurpreet promoted}, find P (A).
\end{enumerate}
\solution
%\input{exemplar/11/16/3/10/main.tex}
\item A card is drawn from a deck of 52 cards. Find the probability of getting a king or a heart or a red card.\\
\solution
%\input{exemplar/11/16/3/15/main.tex}
\item The probability that a student will pass his examination is 0.73, the probability of
the student getting a compartment is 0.13, and the probability that the student will
either pass or get compartment is 0.96. State True or False.\\
\solution
%\input{exemplar/11/16/3/31/main.tex}
\item A card is selected from a pack of 52 cards\\
\begin{enumerate}[label=(\alph*)]
\item How many points are there in the sample space?
\item Calculate the probability that the cards is an ace of spades.
\item Calculate the probability that the card is (i) an ace (ii)black card.\\
\end{enumerate}
%\input{ncert/11/16/3/4_1/Prob_4.tex}
\item In a non-leap year, the probability of having 53 tuesdays or 53 wednesdays is\\
\solution
%\input{exemplar/11/16/3/18/main.tex}
\item There are 1000 sealed envelopes in a box, 10 of them contain a cash prize of
Rs 100 each, 100 of them contain a cash prize of Rs 50 each and 200 of them
contain a cash prize of Rs 10 each and rest do not contain any cash prize. If they
are well shuffled and an envelope is picked up out, what is the probability that it
contains no cash prize?\\
\solution
%\input{exemplar/10/13/3/34/main.tex}
\item 
A die is thrown and a card is selected at random from a deck of 52 playing cards. The probability of getting an even number on the die and a spade card.\\
\solution
%\input{exemplar/12/13/3/78/main.tex}
\item
If 4-digit numbers greater than 5,000 are randomly formed from the digits 0, 1, 3, 5, and 7, what is the probability of forming a number divisible by 5 when:
\begin{enumerate}
    \item The digits are repeated?
    \item The repetition of digits is not allowed?
\end{enumerate}
\solution
%\input{ncert/11/16/4/9/main.tex}
\item Consider the probability space $\brak{\Omega, \mathcal{G}, P}$ where $\Omega = [0,2]$ and $\mathcal{G} = \cbrak{\phi, \Omega, [0,1], (1,2]}$. Let $X$ and $Y$ be two functions on $\Omega$ defined as
\begin{align*}
    X(\omega) = 
    \begin{cases}
        1 & \text{if }\omega \in [0, 1]\\
        2 & \text{if }\omega \in (1, 2]
    \end{cases}
\end{align*}
and
\begin{align*}
    Y(\omega) = 
    \begin{cases}
        2 & \text{if }\omega \in [0, 1.5]\\
        3 & \text{if }\omega \in (1.5, 2].
    \end{cases}
\end{align*}
Then which one of the following statements is true?
\begin{enumerate}
    \item [(A)] $X$ is a random variable with respect to $\mathcal{G}$, but $Y$ is not a random variable with respect to $\mathcal{G}$.
    \item [(B)] $Y$ is a random variable with respect to $\mathcal{G}$, but $X$ is not a random variable with respect to $\mathcal{G}$.
    \item [(C)] Neither $X$ nor $Y$ is a random variable with respect to $\mathcal{G}$.
    \item [(D)] Both $X$ and $Y$ are random variables with respect to $\mathcal{G}$.
\end{enumerate} \hfill (GATE ST 2023)\\
\solution
%\input{gate/ST/2023/14/main.tex}
	\item  A die is loaded in such a way that each odd number is twice as likely to occur as
each even number. Find $P(G)$, where $G$ is the event that a number greater than
3 occurs on a single roll of the die.
\\
\solution
		%\input{exemplar/11/16/3/5/main.tex}
	\item All the jacks, queens and kings are removed from a deck of 52 playing cards. The remaining cards are well shuffled and then one card is drawn at random. Giving ace a value 1 similar value for other cards, find the probability that the card has a value 
		\begin{enumerate}
			\item 7
			\item greater than 7
			\item less than 7
		\end{enumerate}
		%\input{exemplar/10/13/3/30/main.tex}
  \item A Lot consists of 48 mobile phones of which 42 are good, 3 have only minor defects and 3 have major defects.Varnika will buy a phone if it is good but the trader will only buy a mobile if it has no major defects. One phone is selected at random from the lot. What is the probability that it is
\begin{enumerate}
	\item acceptable to Varnika?
            \item acceptable to the trader?
\end{enumerate}
\solution
	%\input{exemplar/10/13/3/40/main.tex}
 \item A student says that if you throw a die, it will show up 1 or not 1. Therefore, the probability of getting 1 and the probability of getting 'not 1' each is equal to $\frac{1}{2}$. Is this correct? Give reasons.\\
 \solution
        %\input{exemplar/10/13/2/9/main.tex}
   \item Four candidates A, B, C, D have ap-
plied for the assignment to coach a school cricket
team. If A is twice as likely to be selected as B, and
B and C are given about the same chance of being
selected, while C is twice as likely to be selected
as D, what are the probabilities that
\begin{enumerate}
\item C will be selected?
\item A will not be selected?
\end{enumerate}
	%\input{exemplar/11/16/3/9/main.tex}
 \item A bag contain 24 balls of which $x$ balls are red, $2x$ are white and $3x$ are blue. A ball is selected at random, What is the probability that it is
\begin{enumerate}[label=\alph*)]
\item not red ?
\item white ?
\end{enumerate}
%\input{exemplar/10/13/3/41/main.tex}
If the letters of the word ASSASSINATION are arranged at random. Find the Probability that
\begin{enumerate}[label=(\alph*)]
\item Four $S's$ come consecutively in the word
\item Two  $I's$ and two $N's$ come together
\item All $A's$ are not coming together
\item No two $A's$ are coming together
\end{enumerate}
%\input{exemplar/11/16/3/14/main.tex}
	\item One urn contains two black balls (labelled B1 and B2) and one white ball. A
	second urn contains one black ball and two white balls (labelled W1 and W2).
	Suppose the following experiment is performed. One of the two urns is chosen
	at random. Next a ball is randomly chosen from the urn. Then a second ball is
	chosen at random from the same urn without replacing the first ball.
	
	\begin{enumerate}
	\item What is the probability that two black balls are chosen?
	
	\item What is the probability that two balls of opposite colour are chosen?
	\end{enumerate}
	\solution
	%\input{exemplar/11/16/3/12/main1.tex}
\end{enumerate}

		\item A box of oranges is inspected by examining three randomly selected oranges drawn without replacement. If all the three oranges are good, the box is approved for sale, otherwise, it is rejected. Find the probability that a box containing 15 oranges out of which 12 are good and 3 are bad ones will be approved for sale.
		\label{ncert/12/13/2/3/defs.tex}
		\item Two balls are drawn at random with replacement from a box containing 10 black and 8 red balls. Find the probability that
		\label{ncert/12/13/2/12}
\begin{enumerate}
\item both balls are red.
\item first ball is black and second is red.
\item one of them is black and other is red.
\end{enumerate}

\item In a hostel, 60\% of the students read Hindi newspaper, 40\% read English newspaper and 20\% read both Hindi and English newspapers. A student is selected at random.
		\label{ncert/12/13/2/15}
\begin{enumerate}
\item Find the probability that she reads neither Hindi nor English newspapers.
\item If she reads Hindi newspaper, find the probability that she reads English newspaper.
\item If she reads English newspaper, find the probability that she reads Hindi newspaper.\\
\end{enumerate}
\item The probability of obtaining an even prime number on each die, when a pair of dice is rolled is 
\begin{enumerate}
    \item $0$ 
    
    \item $\frac{1}{3}$ 
    
    \item $\frac{1}{12}$ 
    
    \item $\frac{1}{36}$ 
\end{enumerate}
\solution
		%\begin{enumerate}[label=\thesection.\arabic*,ref=\thesection.\theenumi]
	\item One card is drawn from a well-shuffled deck of 52 cards. Find the probability of getting
\begin{enumerate}
\item A king of red colour 
\item A face card 
\item A red face card
\item The jack of hearts
\item A spade
\item The queen of diamonds

\end{enumerate}
\solution
		%\input{ncert/10/15/1/14/main.tex}
	\item Five cards—the ten, jack, queen, king and ace of diamonds, are well-shuffled with their face downwards. One card is then picked up at random.
\begin{enumerate}
\item
What is the probability that the card is the queen? 
\item
If the queen is drawn and put aside, what is the probability that the second card picked up is (a) an ace? (b) a queen?\\
\end{enumerate}
\solution
		%\input{ncert/10/15/1/15/defs.tex}
	\item A bag contains $5$ red balls and some blue balls. If the probability of drawing a blue ball is double that if a red ball, determine the number of blue balls in the bag. 
		\\
\solution
		%\input{ncert/10/15/2/3/defs.tex}
	\item A card is selected from a pack of 52 cards.
 \begin{enumerate}[label=(\alph*)] 
                 \item How many points are there in the sample space?
                 \item Calculate the probability that the card is an ace of spades.
                 \item Calculate the probability that the card is (i) an ace and (ii) black card.
 \end{enumerate}
\solution
		%\input{ncert/11/16/3/4/main.tex}
\item Four cards are drawn from a well-shuffled deck of 52 cards. What is the probability of obtaining 3 diamonds and one spade.
\\
\solution
		%\input{ncert/11/16/4/2/defs.tex}
\item In a certain lottery 10,000 tickets are sold and ten equal prizes are awarded. What is the probability of not getting a prize if you buy (a) one ticket (b) two tickets (c) 10 tickets ?	
\\
\solution
		%\input{ncert/11/16/4/4/defs.tex}
		%
\item 
Out of 100 students, two sections of 40 and 60 are formed. If you and your friend are among the 100 students, what is the probability that
\begin{enumerate}
\item you both enter the same section?
\item you both enter the different sections?
\end{enumerate}
\solution
		%\input{ncert/11/16/4/5/defs.tex}
	\item 
The number lock of a suitcase has 4 wheels each labelled with ten digits i.e. from 0 to 9.The lock opens with a sequence of four digits with no repeats.What is the probability of a person getting the right sequence to open the suitcase.
\\
\solution
		%\input{ncert/11/16/4/10/defs.tex}
		%
\item 
Two cards are drawn at random and without replacement from a pack of 52 playing cards. Find the probability that both the cards are black.
\\
\solution
		%\input{ncert/12/13/2/2/defs.tex}
		\item A box of oranges is inspected by examining three randomly selected oranges drawn without replacement. If all the three oranges are good, the box is approved for sale, otherwise, it is rejected. Find the probability that a box containing 15 oranges out of which 12 are good and 3 are bad ones will be approved for sale.
		\label{ncert/12/13/2/3/defs.tex}
		\item Two balls are drawn at random with replacement from a box containing 10 black and 8 red balls. Find the probability that
		\label{ncert/12/13/2/12}
\begin{enumerate}
\item both balls are red.
\item first ball is black and second is red.
\item one of them is black and other is red.
\end{enumerate}

\item In a hostel, 60\% of the students read Hindi newspaper, 40\% read English newspaper and 20\% read both Hindi and English newspapers. A student is selected at random.
		\label{ncert/12/13/2/15}
\begin{enumerate}
\item Find the probability that she reads neither Hindi nor English newspapers.
\item If she reads Hindi newspaper, find the probability that she reads English newspaper.
\item If she reads English newspaper, find the probability that she reads Hindi newspaper.\\
\end{enumerate}
\item The probability of obtaining an even prime number on each die, when a pair of dice is rolled is 
\begin{enumerate}
    \item $0$ 
    
    \item $\frac{1}{3}$ 
    
    \item $\frac{1}{12}$ 
    
    \item $\frac{1}{36}$ 
\end{enumerate}
\solution
		%\input{ncert/12/13/2/17/defs.tex}
	\item A bag contains 4 red and 4 black balls, another bag contains 2 red and 6 black balls. One of the two bags is selected at random and a ball is drawn from the bag which is found to be red. Find the probability that the ball is drawn from the first bag.
\\
\solution
		%\input{ncert/12/13/3/2/main.tex}
  \item
  Cards with numbers 2 to 101 are placed in a box. A card is selected at random.Find the probability that the card has
\begin{enumerate}[label=(\roman*)]
	\item an even number 
	\item a square number
\end{enumerate}
\solution
%\input{exemplar/10/13/3/32/main.tex}
\item
The king, queen and jack of clubs are removed from a deck of 52 playing cards and then well shuffled. Now one card is drawn at random from the remaining cards.  Determine the probability that the card is
\begin{enumerate}[label=(\roman*)]
\item a club
\item 10 of hearts
\end{enumerate}
\solution
%\input{exemplar/10/13/3/29/main.tex}
\item A team of medical students doing their internship have to assist during surgeries
at a city hospital. The probabilities of surgeries rated as very complex, complex,
routine, simple or very simple are respectively, 0.15, 0.20, 0.31, 0.26, .08. Find
the probabilities that a particular surgery will be rated
\begin{enumerate}
	\item complex or very complex;
	\item neither very complex nor very simple;
	\item routine or complex
	\item routine or simple
\end{enumerate}
\solution
%\input{exemplar/11/16/3/8(1)/main.tex}
\item A card is selected from a pack of 52 cards.
\begin{enumerate}[label=(\alph*)]
    \item How many points are there in the sample space?
    \item Calculate the probability that the card is an ace of spades.
    \item Calculate the probability that the card is (i) an ace and (ii) black card.
\end{enumerate}
\solution
%\input{exemplar/11/16/3/4/main2.tex}
\item The probability that a non leap year selected at random will contain 53 sundays.
\\
\solution
%\input{exemplar/10/13/1/19/main.tex}
\item One of the four persons John, Rita, Aslam or Gurpreet will be promoted next
month. Consequently the sample space consists of four elementary outcomes
S = {John promoted, Rita promoted, Aslam promoted, Gurpreet promoted}
You are told that the chances of John’s promotion is same as that of Gurpreet,
Rita’s chances of promotion are twice as likely as Johns. Aslam’s chances are
four times that of John.
\begin{enumerate}
	\item Determine
	\begin{enumerate}
		\item P (John promoted)
		\item P (Rita promoted)
		\item P (Aslam promoted)
		\item P (Gurpreet promoted)
	\end{enumerate}
	\item If A = {John promoted or Gurpreet promoted}, find P (A).
\end{enumerate}
\solution
%\input{exemplar/11/16/3/10/main.tex}
\item A card is drawn from a deck of 52 cards. Find the probability of getting a king or a heart or a red card.\\
\solution
%\input{exemplar/11/16/3/15/main.tex}
\item The probability that a student will pass his examination is 0.73, the probability of
the student getting a compartment is 0.13, and the probability that the student will
either pass or get compartment is 0.96. State True or False.\\
\solution
%\input{exemplar/11/16/3/31/main.tex}
\item A card is selected from a pack of 52 cards\\
\begin{enumerate}[label=(\alph*)]
\item How many points are there in the sample space?
\item Calculate the probability that the cards is an ace of spades.
\item Calculate the probability that the card is (i) an ace (ii)black card.\\
\end{enumerate}
%\input{ncert/11/16/3/4_1/Prob_4.tex}
\item In a non-leap year, the probability of having 53 tuesdays or 53 wednesdays is\\
\solution
%\input{exemplar/11/16/3/18/main.tex}
\item There are 1000 sealed envelopes in a box, 10 of them contain a cash prize of
Rs 100 each, 100 of them contain a cash prize of Rs 50 each and 200 of them
contain a cash prize of Rs 10 each and rest do not contain any cash prize. If they
are well shuffled and an envelope is picked up out, what is the probability that it
contains no cash prize?\\
\solution
%\input{exemplar/10/13/3/34/main.tex}
\item 
A die is thrown and a card is selected at random from a deck of 52 playing cards. The probability of getting an even number on the die and a spade card.\\
\solution
%\input{exemplar/12/13/3/78/main.tex}
\item
If 4-digit numbers greater than 5,000 are randomly formed from the digits 0, 1, 3, 5, and 7, what is the probability of forming a number divisible by 5 when:
\begin{enumerate}
    \item The digits are repeated?
    \item The repetition of digits is not allowed?
\end{enumerate}
\solution
%\input{ncert/11/16/4/9/main.tex}
\item Consider the probability space $\brak{\Omega, \mathcal{G}, P}$ where $\Omega = [0,2]$ and $\mathcal{G} = \cbrak{\phi, \Omega, [0,1], (1,2]}$. Let $X$ and $Y$ be two functions on $\Omega$ defined as
\begin{align*}
    X(\omega) = 
    \begin{cases}
        1 & \text{if }\omega \in [0, 1]\\
        2 & \text{if }\omega \in (1, 2]
    \end{cases}
\end{align*}
and
\begin{align*}
    Y(\omega) = 
    \begin{cases}
        2 & \text{if }\omega \in [0, 1.5]\\
        3 & \text{if }\omega \in (1.5, 2].
    \end{cases}
\end{align*}
Then which one of the following statements is true?
\begin{enumerate}
    \item [(A)] $X$ is a random variable with respect to $\mathcal{G}$, but $Y$ is not a random variable with respect to $\mathcal{G}$.
    \item [(B)] $Y$ is a random variable with respect to $\mathcal{G}$, but $X$ is not a random variable with respect to $\mathcal{G}$.
    \item [(C)] Neither $X$ nor $Y$ is a random variable with respect to $\mathcal{G}$.
    \item [(D)] Both $X$ and $Y$ are random variables with respect to $\mathcal{G}$.
\end{enumerate} \hfill (GATE ST 2023)\\
\solution
%\input{gate/ST/2023/14/main.tex}
	\item  A die is loaded in such a way that each odd number is twice as likely to occur as
each even number. Find $P(G)$, where $G$ is the event that a number greater than
3 occurs on a single roll of the die.
\\
\solution
		%\input{exemplar/11/16/3/5/main.tex}
	\item All the jacks, queens and kings are removed from a deck of 52 playing cards. The remaining cards are well shuffled and then one card is drawn at random. Giving ace a value 1 similar value for other cards, find the probability that the card has a value 
		\begin{enumerate}
			\item 7
			\item greater than 7
			\item less than 7
		\end{enumerate}
		%\input{exemplar/10/13/3/30/main.tex}
  \item A Lot consists of 48 mobile phones of which 42 are good, 3 have only minor defects and 3 have major defects.Varnika will buy a phone if it is good but the trader will only buy a mobile if it has no major defects. One phone is selected at random from the lot. What is the probability that it is
\begin{enumerate}
	\item acceptable to Varnika?
            \item acceptable to the trader?
\end{enumerate}
\solution
	%\input{exemplar/10/13/3/40/main.tex}
 \item A student says that if you throw a die, it will show up 1 or not 1. Therefore, the probability of getting 1 and the probability of getting 'not 1' each is equal to $\frac{1}{2}$. Is this correct? Give reasons.\\
 \solution
        %\input{exemplar/10/13/2/9/main.tex}
   \item Four candidates A, B, C, D have ap-
plied for the assignment to coach a school cricket
team. If A is twice as likely to be selected as B, and
B and C are given about the same chance of being
selected, while C is twice as likely to be selected
as D, what are the probabilities that
\begin{enumerate}
\item C will be selected?
\item A will not be selected?
\end{enumerate}
	%\input{exemplar/11/16/3/9/main.tex}
 \item A bag contain 24 balls of which $x$ balls are red, $2x$ are white and $3x$ are blue. A ball is selected at random, What is the probability that it is
\begin{enumerate}[label=\alph*)]
\item not red ?
\item white ?
\end{enumerate}
%\input{exemplar/10/13/3/41/main.tex}
If the letters of the word ASSASSINATION are arranged at random. Find the Probability that
\begin{enumerate}[label=(\alph*)]
\item Four $S's$ come consecutively in the word
\item Two  $I's$ and two $N's$ come together
\item All $A's$ are not coming together
\item No two $A's$ are coming together
\end{enumerate}
%\input{exemplar/11/16/3/14/main.tex}
	\item One urn contains two black balls (labelled B1 and B2) and one white ball. A
	second urn contains one black ball and two white balls (labelled W1 and W2).
	Suppose the following experiment is performed. One of the two urns is chosen
	at random. Next a ball is randomly chosen from the urn. Then a second ball is
	chosen at random from the same urn without replacing the first ball.
	
	\begin{enumerate}
	\item What is the probability that two black balls are chosen?
	
	\item What is the probability that two balls of opposite colour are chosen?
	\end{enumerate}
	\solution
	%\input{exemplar/11/16/3/12/main1.tex}
\end{enumerate}

	\item A bag contains 4 red and 4 black balls, another bag contains 2 red and 6 black balls. One of the two bags is selected at random and a ball is drawn from the bag which is found to be red. Find the probability that the ball is drawn from the first bag.
\\
\solution
		%\begin{table}[H]
	\centering
\begin{tabular}{|c|c|c|}
\hline
Random variable &Value &Definition\\ \hline
\multirow{3}{*}{X} &0 &Slips of Rs 1\\
&1 &Slips of Rs 5\\
&2 &Slips of Rs 13\\ \hline
\multirow{2}{*}{Y} &0 &Box A\\
&1 &Box B\\\hline
\end{tabular}
\caption{}
\label{tab:Distribution}
\end{table}
See \tabref{tab:Distribution}.
\begin{align}
p_{Y}\brak{k}= \begin{cases} 
      \frac{1}{3} & {k=0} \\
      \frac{2}{3 }& {k=1} 
   \end{cases}
   \\
p_{Y|X}\brak{0|0} = \frac{19}{25}\, 
p_{Y|X}\brak{0|1} = \frac{6}{25}\,
p_{Y|X}\brak{1|0} = \frac{45}{50}\,
p_{Y|X}\brak{1|2} = \frac{5}{50}
\end{align}
The desired probability is the probability that a slip drawn at random is marked other than Rs 1,
\begin{align}
&=1-p_X\brak{0}\\
&= p_X(1) + p_X(2)
\end{align}
Using Bayes theorem,
\begin{align}
&= p_Y\brak{0} \times \pr{Y=0 | X=1} + p_Y\brak{1} \times \pr{Y=1|X=2}\\
&=\frac{1}{3} \times \frac{6}{25} + \frac{2}{3} \times \frac{5}{50}\\
&=\frac{11}{75}
\end{align}

\newpage

%\tableofcontents

\bigskip

\renewcommand{\thefigure}{\theenumi}
\renewcommand{\thetable}{\theenumi}
%\renewcommand{\theequation}{\theenumi}

%\begin{abstract}
%%\boldmath
%In this letter, an algorithm for evaluating the exact analytical bit error rate  (BER)  for the piecewise linear (PL) combiner for  multiple relays is presented. Previous results were available only for upto three relays. The algorithm is unique in the sense that  the actual mathematical expressions, that are prohibitively large, need not be explicitly obtained. The diversity gain due to multiple relays is shown through plots of the analytical BER, well supported by simulations. 
%
%\end{abstract}
% IEEEtran.cls defaults to using nonbold math in the Abstract.
% This preserves the distinction between vectors and scalars. However,
% if the journal you are submitting to favors bold math in the abstract,
% then you can use LaTeX's standard command \boldmath at the very start
% of the abstract to achieve this. Many IEEE journals frown on math
% in the abstract anyway.

% Note that keywords are not normally used for peerreview papers.
%\begin{IEEEkeywords}
%Cooperative diversity, decode and forward, piecewise linear
%\end{IEEEkeywords}



% For peer review papers, you can put extra information on the cover
% page as needed:
% \ifCLASSOPTIONpeerreview
% \begin{center} \bfseries EDICS Category: 3-BBND \end{center}
% \fi
%
% For peerreview papers, this IEEEtran command inserts a page break and
% creates the second title. It will be ignored for other modes.
%\IEEEpeerreviewmaketitle




  \item
  Cards with numbers 2 to 101 are placed in a box. A card is selected at random.Find the probability that the card has
\begin{enumerate}[label=(\roman*)]
	\item an even number 
	\item a square number
\end{enumerate}
\solution
%\begin{table}[H]
	\centering
\begin{tabular}{|c|c|c|}
\hline
Random variable &Value &Definition\\ \hline
\multirow{3}{*}{X} &0 &Slips of Rs 1\\
&1 &Slips of Rs 5\\
&2 &Slips of Rs 13\\ \hline
\multirow{2}{*}{Y} &0 &Box A\\
&1 &Box B\\\hline
\end{tabular}
\caption{}
\label{tab:Distribution}
\end{table}
See \tabref{tab:Distribution}.
\begin{align}
p_{Y}\brak{k}= \begin{cases} 
      \frac{1}{3} & {k=0} \\
      \frac{2}{3 }& {k=1} 
   \end{cases}
   \\
p_{Y|X}\brak{0|0} = \frac{19}{25}\, 
p_{Y|X}\brak{0|1} = \frac{6}{25}\,
p_{Y|X}\brak{1|0} = \frac{45}{50}\,
p_{Y|X}\brak{1|2} = \frac{5}{50}
\end{align}
The desired probability is the probability that a slip drawn at random is marked other than Rs 1,
\begin{align}
&=1-p_X\brak{0}\\
&= p_X(1) + p_X(2)
\end{align}
Using Bayes theorem,
\begin{align}
&= p_Y\brak{0} \times \pr{Y=0 | X=1} + p_Y\brak{1} \times \pr{Y=1|X=2}\\
&=\frac{1}{3} \times \frac{6}{25} + \frac{2}{3} \times \frac{5}{50}\\
&=\frac{11}{75}
\end{align}

\newpage

%\tableofcontents

\bigskip

\renewcommand{\thefigure}{\theenumi}
\renewcommand{\thetable}{\theenumi}
%\renewcommand{\theequation}{\theenumi}

%\begin{abstract}
%%\boldmath
%In this letter, an algorithm for evaluating the exact analytical bit error rate  (BER)  for the piecewise linear (PL) combiner for  multiple relays is presented. Previous results were available only for upto three relays. The algorithm is unique in the sense that  the actual mathematical expressions, that are prohibitively large, need not be explicitly obtained. The diversity gain due to multiple relays is shown through plots of the analytical BER, well supported by simulations. 
%
%\end{abstract}
% IEEEtran.cls defaults to using nonbold math in the Abstract.
% This preserves the distinction between vectors and scalars. However,
% if the journal you are submitting to favors bold math in the abstract,
% then you can use LaTeX's standard command \boldmath at the very start
% of the abstract to achieve this. Many IEEE journals frown on math
% in the abstract anyway.

% Note that keywords are not normally used for peerreview papers.
%\begin{IEEEkeywords}
%Cooperative diversity, decode and forward, piecewise linear
%\end{IEEEkeywords}



% For peer review papers, you can put extra information on the cover
% page as needed:
% \ifCLASSOPTIONpeerreview
% \begin{center} \bfseries EDICS Category: 3-BBND \end{center}
% \fi
%
% For peerreview papers, this IEEEtran command inserts a page break and
% creates the second title. It will be ignored for other modes.
%\IEEEpeerreviewmaketitle




\item
The king, queen and jack of clubs are removed from a deck of 52 playing cards and then well shuffled. Now one card is drawn at random from the remaining cards.  Determine the probability that the card is
\begin{enumerate}[label=(\roman*)]
\item a club
\item 10 of hearts
\end{enumerate}
\solution
%\begin{table}[H]
	\centering
\begin{tabular}{|c|c|c|}
\hline
Random variable &Value &Definition\\ \hline
\multirow{3}{*}{X} &0 &Slips of Rs 1\\
&1 &Slips of Rs 5\\
&2 &Slips of Rs 13\\ \hline
\multirow{2}{*}{Y} &0 &Box A\\
&1 &Box B\\\hline
\end{tabular}
\caption{}
\label{tab:Distribution}
\end{table}
See \tabref{tab:Distribution}.
\begin{align}
p_{Y}\brak{k}= \begin{cases} 
      \frac{1}{3} & {k=0} \\
      \frac{2}{3 }& {k=1} 
   \end{cases}
   \\
p_{Y|X}\brak{0|0} = \frac{19}{25}\, 
p_{Y|X}\brak{0|1} = \frac{6}{25}\,
p_{Y|X}\brak{1|0} = \frac{45}{50}\,
p_{Y|X}\brak{1|2} = \frac{5}{50}
\end{align}
The desired probability is the probability that a slip drawn at random is marked other than Rs 1,
\begin{align}
&=1-p_X\brak{0}\\
&= p_X(1) + p_X(2)
\end{align}
Using Bayes theorem,
\begin{align}
&= p_Y\brak{0} \times \pr{Y=0 | X=1} + p_Y\brak{1} \times \pr{Y=1|X=2}\\
&=\frac{1}{3} \times \frac{6}{25} + \frac{2}{3} \times \frac{5}{50}\\
&=\frac{11}{75}
\end{align}

\newpage

%\tableofcontents

\bigskip

\renewcommand{\thefigure}{\theenumi}
\renewcommand{\thetable}{\theenumi}
%\renewcommand{\theequation}{\theenumi}

%\begin{abstract}
%%\boldmath
%In this letter, an algorithm for evaluating the exact analytical bit error rate  (BER)  for the piecewise linear (PL) combiner for  multiple relays is presented. Previous results were available only for upto three relays. The algorithm is unique in the sense that  the actual mathematical expressions, that are prohibitively large, need not be explicitly obtained. The diversity gain due to multiple relays is shown through plots of the analytical BER, well supported by simulations. 
%
%\end{abstract}
% IEEEtran.cls defaults to using nonbold math in the Abstract.
% This preserves the distinction between vectors and scalars. However,
% if the journal you are submitting to favors bold math in the abstract,
% then you can use LaTeX's standard command \boldmath at the very start
% of the abstract to achieve this. Many IEEE journals frown on math
% in the abstract anyway.

% Note that keywords are not normally used for peerreview papers.
%\begin{IEEEkeywords}
%Cooperative diversity, decode and forward, piecewise linear
%\end{IEEEkeywords}



% For peer review papers, you can put extra information on the cover
% page as needed:
% \ifCLASSOPTIONpeerreview
% \begin{center} \bfseries EDICS Category: 3-BBND \end{center}
% \fi
%
% For peerreview papers, this IEEEtran command inserts a page break and
% creates the second title. It will be ignored for other modes.
%\IEEEpeerreviewmaketitle




\item A team of medical students doing their internship have to assist during surgeries
at a city hospital. The probabilities of surgeries rated as very complex, complex,
routine, simple or very simple are respectively, 0.15, 0.20, 0.31, 0.26, .08. Find
the probabilities that a particular surgery will be rated
\begin{enumerate}
	\item complex or very complex;
	\item neither very complex nor very simple;
	\item routine or complex
	\item routine or simple
\end{enumerate}
\solution
%\begin{table}[H]
	\centering
\begin{tabular}{|c|c|c|}
\hline
Random variable &Value &Definition\\ \hline
\multirow{3}{*}{X} &0 &Slips of Rs 1\\
&1 &Slips of Rs 5\\
&2 &Slips of Rs 13\\ \hline
\multirow{2}{*}{Y} &0 &Box A\\
&1 &Box B\\\hline
\end{tabular}
\caption{}
\label{tab:Distribution}
\end{table}
See \tabref{tab:Distribution}.
\begin{align}
p_{Y}\brak{k}= \begin{cases} 
      \frac{1}{3} & {k=0} \\
      \frac{2}{3 }& {k=1} 
   \end{cases}
   \\
p_{Y|X}\brak{0|0} = \frac{19}{25}\, 
p_{Y|X}\brak{0|1} = \frac{6}{25}\,
p_{Y|X}\brak{1|0} = \frac{45}{50}\,
p_{Y|X}\brak{1|2} = \frac{5}{50}
\end{align}
The desired probability is the probability that a slip drawn at random is marked other than Rs 1,
\begin{align}
&=1-p_X\brak{0}\\
&= p_X(1) + p_X(2)
\end{align}
Using Bayes theorem,
\begin{align}
&= p_Y\brak{0} \times \pr{Y=0 | X=1} + p_Y\brak{1} \times \pr{Y=1|X=2}\\
&=\frac{1}{3} \times \frac{6}{25} + \frac{2}{3} \times \frac{5}{50}\\
&=\frac{11}{75}
\end{align}

\newpage

%\tableofcontents

\bigskip

\renewcommand{\thefigure}{\theenumi}
\renewcommand{\thetable}{\theenumi}
%\renewcommand{\theequation}{\theenumi}

%\begin{abstract}
%%\boldmath
%In this letter, an algorithm for evaluating the exact analytical bit error rate  (BER)  for the piecewise linear (PL) combiner for  multiple relays is presented. Previous results were available only for upto three relays. The algorithm is unique in the sense that  the actual mathematical expressions, that are prohibitively large, need not be explicitly obtained. The diversity gain due to multiple relays is shown through plots of the analytical BER, well supported by simulations. 
%
%\end{abstract}
% IEEEtran.cls defaults to using nonbold math in the Abstract.
% This preserves the distinction between vectors and scalars. However,
% if the journal you are submitting to favors bold math in the abstract,
% then you can use LaTeX's standard command \boldmath at the very start
% of the abstract to achieve this. Many IEEE journals frown on math
% in the abstract anyway.

% Note that keywords are not normally used for peerreview papers.
%\begin{IEEEkeywords}
%Cooperative diversity, decode and forward, piecewise linear
%\end{IEEEkeywords}



% For peer review papers, you can put extra information on the cover
% page as needed:
% \ifCLASSOPTIONpeerreview
% \begin{center} \bfseries EDICS Category: 3-BBND \end{center}
% \fi
%
% For peerreview papers, this IEEEtran command inserts a page break and
% creates the second title. It will be ignored for other modes.
%\IEEEpeerreviewmaketitle




\item A card is selected from a pack of 52 cards.
\begin{enumerate}[label=(\alph*)]
    \item How many points are there in the sample space?
    \item Calculate the probability that the card is an ace of spades.
    \item Calculate the probability that the card is (i) an ace and (ii) black card.
\end{enumerate}
\solution
%Let $X$ be an bernoulli rv defined as in \tabref{tab:exemplar/11/16/3/26}.  Then, 
\begin{equation}
    p =
        \frac{4}{11} 
\end{equation}
\begin{table}[H]
	\centering
	\input{exemplar/11/16/3/26/tables/Table2.tex}
	\caption{}
        \label{tab:exemplar/11/16/3/26}
\end{table}

\item The probability that a non leap year selected at random will contain 53 sundays.
\\
\solution
%\begin{table}[H]
	\centering
\begin{tabular}{|c|c|c|}
\hline
Random variable &Value &Definition\\ \hline
\multirow{3}{*}{X} &0 &Slips of Rs 1\\
&1 &Slips of Rs 5\\
&2 &Slips of Rs 13\\ \hline
\multirow{2}{*}{Y} &0 &Box A\\
&1 &Box B\\\hline
\end{tabular}
\caption{}
\label{tab:Distribution}
\end{table}
See \tabref{tab:Distribution}.
\begin{align}
p_{Y}\brak{k}= \begin{cases} 
      \frac{1}{3} & {k=0} \\
      \frac{2}{3 }& {k=1} 
   \end{cases}
   \\
p_{Y|X}\brak{0|0} = \frac{19}{25}\, 
p_{Y|X}\brak{0|1} = \frac{6}{25}\,
p_{Y|X}\brak{1|0} = \frac{45}{50}\,
p_{Y|X}\brak{1|2} = \frac{5}{50}
\end{align}
The desired probability is the probability that a slip drawn at random is marked other than Rs 1,
\begin{align}
&=1-p_X\brak{0}\\
&= p_X(1) + p_X(2)
\end{align}
Using Bayes theorem,
\begin{align}
&= p_Y\brak{0} \times \pr{Y=0 | X=1} + p_Y\brak{1} \times \pr{Y=1|X=2}\\
&=\frac{1}{3} \times \frac{6}{25} + \frac{2}{3} \times \frac{5}{50}\\
&=\frac{11}{75}
\end{align}

\newpage

%\tableofcontents

\bigskip

\renewcommand{\thefigure}{\theenumi}
\renewcommand{\thetable}{\theenumi}
%\renewcommand{\theequation}{\theenumi}

%\begin{abstract}
%%\boldmath
%In this letter, an algorithm for evaluating the exact analytical bit error rate  (BER)  for the piecewise linear (PL) combiner for  multiple relays is presented. Previous results were available only for upto three relays. The algorithm is unique in the sense that  the actual mathematical expressions, that are prohibitively large, need not be explicitly obtained. The diversity gain due to multiple relays is shown through plots of the analytical BER, well supported by simulations. 
%
%\end{abstract}
% IEEEtran.cls defaults to using nonbold math in the Abstract.
% This preserves the distinction between vectors and scalars. However,
% if the journal you are submitting to favors bold math in the abstract,
% then you can use LaTeX's standard command \boldmath at the very start
% of the abstract to achieve this. Many IEEE journals frown on math
% in the abstract anyway.

% Note that keywords are not normally used for peerreview papers.
%\begin{IEEEkeywords}
%Cooperative diversity, decode and forward, piecewise linear
%\end{IEEEkeywords}



% For peer review papers, you can put extra information on the cover
% page as needed:
% \ifCLASSOPTIONpeerreview
% \begin{center} \bfseries EDICS Category: 3-BBND \end{center}
% \fi
%
% For peerreview papers, this IEEEtran command inserts a page break and
% creates the second title. It will be ignored for other modes.
%\IEEEpeerreviewmaketitle




\item One of the four persons John, Rita, Aslam or Gurpreet will be promoted next
month. Consequently the sample space consists of four elementary outcomes
S = {John promoted, Rita promoted, Aslam promoted, Gurpreet promoted}
You are told that the chances of John’s promotion is same as that of Gurpreet,
Rita’s chances of promotion are twice as likely as Johns. Aslam’s chances are
four times that of John.
\begin{enumerate}
	\item Determine
	\begin{enumerate}
		\item P (John promoted)
		\item P (Rita promoted)
		\item P (Aslam promoted)
		\item P (Gurpreet promoted)
	\end{enumerate}
	\item If A = {John promoted or Gurpreet promoted}, find P (A).
\end{enumerate}
\solution
%\begin{table}[H]
	\centering
\begin{tabular}{|c|c|c|}
\hline
Random variable &Value &Definition\\ \hline
\multirow{3}{*}{X} &0 &Slips of Rs 1\\
&1 &Slips of Rs 5\\
&2 &Slips of Rs 13\\ \hline
\multirow{2}{*}{Y} &0 &Box A\\
&1 &Box B\\\hline
\end{tabular}
\caption{}
\label{tab:Distribution}
\end{table}
See \tabref{tab:Distribution}.
\begin{align}
p_{Y}\brak{k}= \begin{cases} 
      \frac{1}{3} & {k=0} \\
      \frac{2}{3 }& {k=1} 
   \end{cases}
   \\
p_{Y|X}\brak{0|0} = \frac{19}{25}\, 
p_{Y|X}\brak{0|1} = \frac{6}{25}\,
p_{Y|X}\brak{1|0} = \frac{45}{50}\,
p_{Y|X}\brak{1|2} = \frac{5}{50}
\end{align}
The desired probability is the probability that a slip drawn at random is marked other than Rs 1,
\begin{align}
&=1-p_X\brak{0}\\
&= p_X(1) + p_X(2)
\end{align}
Using Bayes theorem,
\begin{align}
&= p_Y\brak{0} \times \pr{Y=0 | X=1} + p_Y\brak{1} \times \pr{Y=1|X=2}\\
&=\frac{1}{3} \times \frac{6}{25} + \frac{2}{3} \times \frac{5}{50}\\
&=\frac{11}{75}
\end{align}

\newpage

%\tableofcontents

\bigskip

\renewcommand{\thefigure}{\theenumi}
\renewcommand{\thetable}{\theenumi}
%\renewcommand{\theequation}{\theenumi}

%\begin{abstract}
%%\boldmath
%In this letter, an algorithm for evaluating the exact analytical bit error rate  (BER)  for the piecewise linear (PL) combiner for  multiple relays is presented. Previous results were available only for upto three relays. The algorithm is unique in the sense that  the actual mathematical expressions, that are prohibitively large, need not be explicitly obtained. The diversity gain due to multiple relays is shown through plots of the analytical BER, well supported by simulations. 
%
%\end{abstract}
% IEEEtran.cls defaults to using nonbold math in the Abstract.
% This preserves the distinction between vectors and scalars. However,
% if the journal you are submitting to favors bold math in the abstract,
% then you can use LaTeX's standard command \boldmath at the very start
% of the abstract to achieve this. Many IEEE journals frown on math
% in the abstract anyway.

% Note that keywords are not normally used for peerreview papers.
%\begin{IEEEkeywords}
%Cooperative diversity, decode and forward, piecewise linear
%\end{IEEEkeywords}



% For peer review papers, you can put extra information on the cover
% page as needed:
% \ifCLASSOPTIONpeerreview
% \begin{center} \bfseries EDICS Category: 3-BBND \end{center}
% \fi
%
% For peerreview papers, this IEEEtran command inserts a page break and
% creates the second title. It will be ignored for other modes.
%\IEEEpeerreviewmaketitle




\item A card is drawn from a deck of 52 cards. Find the probability of getting a king or a heart or a red card.\\
\solution
%\begin{table}[H]
	\centering
\begin{tabular}{|c|c|c|}
\hline
Random variable &Value &Definition\\ \hline
\multirow{3}{*}{X} &0 &Slips of Rs 1\\
&1 &Slips of Rs 5\\
&2 &Slips of Rs 13\\ \hline
\multirow{2}{*}{Y} &0 &Box A\\
&1 &Box B\\\hline
\end{tabular}
\caption{}
\label{tab:Distribution}
\end{table}
See \tabref{tab:Distribution}.
\begin{align}
p_{Y}\brak{k}= \begin{cases} 
      \frac{1}{3} & {k=0} \\
      \frac{2}{3 }& {k=1} 
   \end{cases}
   \\
p_{Y|X}\brak{0|0} = \frac{19}{25}\, 
p_{Y|X}\brak{0|1} = \frac{6}{25}\,
p_{Y|X}\brak{1|0} = \frac{45}{50}\,
p_{Y|X}\brak{1|2} = \frac{5}{50}
\end{align}
The desired probability is the probability that a slip drawn at random is marked other than Rs 1,
\begin{align}
&=1-p_X\brak{0}\\
&= p_X(1) + p_X(2)
\end{align}
Using Bayes theorem,
\begin{align}
&= p_Y\brak{0} \times \pr{Y=0 | X=1} + p_Y\brak{1} \times \pr{Y=1|X=2}\\
&=\frac{1}{3} \times \frac{6}{25} + \frac{2}{3} \times \frac{5}{50}\\
&=\frac{11}{75}
\end{align}

\newpage

%\tableofcontents

\bigskip

\renewcommand{\thefigure}{\theenumi}
\renewcommand{\thetable}{\theenumi}
%\renewcommand{\theequation}{\theenumi}

%\begin{abstract}
%%\boldmath
%In this letter, an algorithm for evaluating the exact analytical bit error rate  (BER)  for the piecewise linear (PL) combiner for  multiple relays is presented. Previous results were available only for upto three relays. The algorithm is unique in the sense that  the actual mathematical expressions, that are prohibitively large, need not be explicitly obtained. The diversity gain due to multiple relays is shown through plots of the analytical BER, well supported by simulations. 
%
%\end{abstract}
% IEEEtran.cls defaults to using nonbold math in the Abstract.
% This preserves the distinction between vectors and scalars. However,
% if the journal you are submitting to favors bold math in the abstract,
% then you can use LaTeX's standard command \boldmath at the very start
% of the abstract to achieve this. Many IEEE journals frown on math
% in the abstract anyway.

% Note that keywords are not normally used for peerreview papers.
%\begin{IEEEkeywords}
%Cooperative diversity, decode and forward, piecewise linear
%\end{IEEEkeywords}



% For peer review papers, you can put extra information on the cover
% page as needed:
% \ifCLASSOPTIONpeerreview
% \begin{center} \bfseries EDICS Category: 3-BBND \end{center}
% \fi
%
% For peerreview papers, this IEEEtran command inserts a page break and
% creates the second title. It will be ignored for other modes.
%\IEEEpeerreviewmaketitle




\item The probability that a student will pass his examination is 0.73, the probability of
the student getting a compartment is 0.13, and the probability that the student will
either pass or get compartment is 0.96. State True or False.\\
\solution
%\begin{table}[H]
	\centering
\begin{tabular}{|c|c|c|}
\hline
Random variable &Value &Definition\\ \hline
\multirow{3}{*}{X} &0 &Slips of Rs 1\\
&1 &Slips of Rs 5\\
&2 &Slips of Rs 13\\ \hline
\multirow{2}{*}{Y} &0 &Box A\\
&1 &Box B\\\hline
\end{tabular}
\caption{}
\label{tab:Distribution}
\end{table}
See \tabref{tab:Distribution}.
\begin{align}
p_{Y}\brak{k}= \begin{cases} 
      \frac{1}{3} & {k=0} \\
      \frac{2}{3 }& {k=1} 
   \end{cases}
   \\
p_{Y|X}\brak{0|0} = \frac{19}{25}\, 
p_{Y|X}\brak{0|1} = \frac{6}{25}\,
p_{Y|X}\brak{1|0} = \frac{45}{50}\,
p_{Y|X}\brak{1|2} = \frac{5}{50}
\end{align}
The desired probability is the probability that a slip drawn at random is marked other than Rs 1,
\begin{align}
&=1-p_X\brak{0}\\
&= p_X(1) + p_X(2)
\end{align}
Using Bayes theorem,
\begin{align}
&= p_Y\brak{0} \times \pr{Y=0 | X=1} + p_Y\brak{1} \times \pr{Y=1|X=2}\\
&=\frac{1}{3} \times \frac{6}{25} + \frac{2}{3} \times \frac{5}{50}\\
&=\frac{11}{75}
\end{align}

\newpage

%\tableofcontents

\bigskip

\renewcommand{\thefigure}{\theenumi}
\renewcommand{\thetable}{\theenumi}
%\renewcommand{\theequation}{\theenumi}

%\begin{abstract}
%%\boldmath
%In this letter, an algorithm for evaluating the exact analytical bit error rate  (BER)  for the piecewise linear (PL) combiner for  multiple relays is presented. Previous results were available only for upto three relays. The algorithm is unique in the sense that  the actual mathematical expressions, that are prohibitively large, need not be explicitly obtained. The diversity gain due to multiple relays is shown through plots of the analytical BER, well supported by simulations. 
%
%\end{abstract}
% IEEEtran.cls defaults to using nonbold math in the Abstract.
% This preserves the distinction between vectors and scalars. However,
% if the journal you are submitting to favors bold math in the abstract,
% then you can use LaTeX's standard command \boldmath at the very start
% of the abstract to achieve this. Many IEEE journals frown on math
% in the abstract anyway.

% Note that keywords are not normally used for peerreview papers.
%\begin{IEEEkeywords}
%Cooperative diversity, decode and forward, piecewise linear
%\end{IEEEkeywords}



% For peer review papers, you can put extra information on the cover
% page as needed:
% \ifCLASSOPTIONpeerreview
% \begin{center} \bfseries EDICS Category: 3-BBND \end{center}
% \fi
%
% For peerreview papers, this IEEEtran command inserts a page break and
% creates the second title. It will be ignored for other modes.
%\IEEEpeerreviewmaketitle




\item A card is selected from a pack of 52 cards\\
\begin{enumerate}[label=(\alph*)]
\item How many points are there in the sample space?
\item Calculate the probability that the cards is an ace of spades.
\item Calculate the probability that the card is (i) an ace (ii)black card.\\
\end{enumerate}
%\input{ncert/11/16/3/4_1/Prob_4.tex}
\item In a non-leap year, the probability of having 53 tuesdays or 53 wednesdays is\\
\solution
%A non-leap year has a total of 365 days, and a week has 7 days.\\
So it can be expressed as 
\begin{align}
365\text{days} &=52\times 7+1 \text{day}
\end{align}
$\implies$ 52 tuesdays or wednesdays\\
Random variable X denotes the days of a week
\begin{align}
p_X\brak{k}&=\frac{1}{7}; \quad \brak{1<k<7}
\end{align}
So the probability of extra day being tuesday or wednesday is
\begin{align}
p_X\brak{3}+p_X\brak{4}&=\frac{1}{7}+\frac{1}{7}=\frac{2}{7}
\end{align}



\item There are 1000 sealed envelopes in a box, 10 of them contain a cash prize of
Rs 100 each, 100 of them contain a cash prize of Rs 50 each and 200 of them
contain a cash prize of Rs 10 each and rest do not contain any cash prize. If they
are well shuffled and an envelope is picked up out, what is the probability that it
contains no cash prize?\\
\solution
%\begin{table}[H]
	\centering
\begin{tabular}{|c|c|c|}
\hline
Random variable &Value &Definition\\ \hline
\multirow{3}{*}{X} &0 &Slips of Rs 1\\
&1 &Slips of Rs 5\\
&2 &Slips of Rs 13\\ \hline
\multirow{2}{*}{Y} &0 &Box A\\
&1 &Box B\\\hline
\end{tabular}
\caption{}
\label{tab:Distribution}
\end{table}
See \tabref{tab:Distribution}.
\begin{align}
p_{Y}\brak{k}= \begin{cases} 
      \frac{1}{3} & {k=0} \\
      \frac{2}{3 }& {k=1} 
   \end{cases}
   \\
p_{Y|X}\brak{0|0} = \frac{19}{25}\, 
p_{Y|X}\brak{0|1} = \frac{6}{25}\,
p_{Y|X}\brak{1|0} = \frac{45}{50}\,
p_{Y|X}\brak{1|2} = \frac{5}{50}
\end{align}
The desired probability is the probability that a slip drawn at random is marked other than Rs 1,
\begin{align}
&=1-p_X\brak{0}\\
&= p_X(1) + p_X(2)
\end{align}
Using Bayes theorem,
\begin{align}
&= p_Y\brak{0} \times \pr{Y=0 | X=1} + p_Y\brak{1} \times \pr{Y=1|X=2}\\
&=\frac{1}{3} \times \frac{6}{25} + \frac{2}{3} \times \frac{5}{50}\\
&=\frac{11}{75}
\end{align}

\newpage

%\tableofcontents

\bigskip

\renewcommand{\thefigure}{\theenumi}
\renewcommand{\thetable}{\theenumi}
%\renewcommand{\theequation}{\theenumi}

%\begin{abstract}
%%\boldmath
%In this letter, an algorithm for evaluating the exact analytical bit error rate  (BER)  for the piecewise linear (PL) combiner for  multiple relays is presented. Previous results were available only for upto three relays. The algorithm is unique in the sense that  the actual mathematical expressions, that are prohibitively large, need not be explicitly obtained. The diversity gain due to multiple relays is shown through plots of the analytical BER, well supported by simulations. 
%
%\end{abstract}
% IEEEtran.cls defaults to using nonbold math in the Abstract.
% This preserves the distinction between vectors and scalars. However,
% if the journal you are submitting to favors bold math in the abstract,
% then you can use LaTeX's standard command \boldmath at the very start
% of the abstract to achieve this. Many IEEE journals frown on math
% in the abstract anyway.

% Note that keywords are not normally used for peerreview papers.
%\begin{IEEEkeywords}
%Cooperative diversity, decode and forward, piecewise linear
%\end{IEEEkeywords}



% For peer review papers, you can put extra information on the cover
% page as needed:
% \ifCLASSOPTIONpeerreview
% \begin{center} \bfseries EDICS Category: 3-BBND \end{center}
% \fi
%
% For peerreview papers, this IEEEtran command inserts a page break and
% creates the second title. It will be ignored for other modes.
%\IEEEpeerreviewmaketitle




\item 
A die is thrown and a card is selected at random from a deck of 52 playing cards. The probability of getting an even number on the die and a spade card.\\
\solution
%\begin{table}[H]
	\centering
\begin{tabular}{|c|c|c|}
\hline
Random variable &Value &Definition\\ \hline
\multirow{3}{*}{X} &0 &Slips of Rs 1\\
&1 &Slips of Rs 5\\
&2 &Slips of Rs 13\\ \hline
\multirow{2}{*}{Y} &0 &Box A\\
&1 &Box B\\\hline
\end{tabular}
\caption{}
\label{tab:Distribution}
\end{table}
See \tabref{tab:Distribution}.
\begin{align}
p_{Y}\brak{k}= \begin{cases} 
      \frac{1}{3} & {k=0} \\
      \frac{2}{3 }& {k=1} 
   \end{cases}
   \\
p_{Y|X}\brak{0|0} = \frac{19}{25}\, 
p_{Y|X}\brak{0|1} = \frac{6}{25}\,
p_{Y|X}\brak{1|0} = \frac{45}{50}\,
p_{Y|X}\brak{1|2} = \frac{5}{50}
\end{align}
The desired probability is the probability that a slip drawn at random is marked other than Rs 1,
\begin{align}
&=1-p_X\brak{0}\\
&= p_X(1) + p_X(2)
\end{align}
Using Bayes theorem,
\begin{align}
&= p_Y\brak{0} \times \pr{Y=0 | X=1} + p_Y\brak{1} \times \pr{Y=1|X=2}\\
&=\frac{1}{3} \times \frac{6}{25} + \frac{2}{3} \times \frac{5}{50}\\
&=\frac{11}{75}
\end{align}

\newpage

%\tableofcontents

\bigskip

\renewcommand{\thefigure}{\theenumi}
\renewcommand{\thetable}{\theenumi}
%\renewcommand{\theequation}{\theenumi}

%\begin{abstract}
%%\boldmath
%In this letter, an algorithm for evaluating the exact analytical bit error rate  (BER)  for the piecewise linear (PL) combiner for  multiple relays is presented. Previous results were available only for upto three relays. The algorithm is unique in the sense that  the actual mathematical expressions, that are prohibitively large, need not be explicitly obtained. The diversity gain due to multiple relays is shown through plots of the analytical BER, well supported by simulations. 
%
%\end{abstract}
% IEEEtran.cls defaults to using nonbold math in the Abstract.
% This preserves the distinction between vectors and scalars. However,
% if the journal you are submitting to favors bold math in the abstract,
% then you can use LaTeX's standard command \boldmath at the very start
% of the abstract to achieve this. Many IEEE journals frown on math
% in the abstract anyway.

% Note that keywords are not normally used for peerreview papers.
%\begin{IEEEkeywords}
%Cooperative diversity, decode and forward, piecewise linear
%\end{IEEEkeywords}



% For peer review papers, you can put extra information on the cover
% page as needed:
% \ifCLASSOPTIONpeerreview
% \begin{center} \bfseries EDICS Category: 3-BBND \end{center}
% \fi
%
% For peerreview papers, this IEEEtran command inserts a page break and
% creates the second title. It will be ignored for other modes.
%\IEEEpeerreviewmaketitle




\item
If 4-digit numbers greater than 5,000 are randomly formed from the digits 0, 1, 3, 5, and 7, what is the probability of forming a number divisible by 5 when:
\begin{enumerate}
    \item The digits are repeated?
    \item The repetition of digits is not allowed?
\end{enumerate}
\solution
%\begin{table}[H]
	\centering
\begin{tabular}{|c|c|c|}
\hline
Random variable &Value &Definition\\ \hline
\multirow{3}{*}{X} &0 &Slips of Rs 1\\
&1 &Slips of Rs 5\\
&2 &Slips of Rs 13\\ \hline
\multirow{2}{*}{Y} &0 &Box A\\
&1 &Box B\\\hline
\end{tabular}
\caption{}
\label{tab:Distribution}
\end{table}
See \tabref{tab:Distribution}.
\begin{align}
p_{Y}\brak{k}= \begin{cases} 
      \frac{1}{3} & {k=0} \\
      \frac{2}{3 }& {k=1} 
   \end{cases}
   \\
p_{Y|X}\brak{0|0} = \frac{19}{25}\, 
p_{Y|X}\brak{0|1} = \frac{6}{25}\,
p_{Y|X}\brak{1|0} = \frac{45}{50}\,
p_{Y|X}\brak{1|2} = \frac{5}{50}
\end{align}
The desired probability is the probability that a slip drawn at random is marked other than Rs 1,
\begin{align}
&=1-p_X\brak{0}\\
&= p_X(1) + p_X(2)
\end{align}
Using Bayes theorem,
\begin{align}
&= p_Y\brak{0} \times \pr{Y=0 | X=1} + p_Y\brak{1} \times \pr{Y=1|X=2}\\
&=\frac{1}{3} \times \frac{6}{25} + \frac{2}{3} \times \frac{5}{50}\\
&=\frac{11}{75}
\end{align}

\newpage

%\tableofcontents

\bigskip

\renewcommand{\thefigure}{\theenumi}
\renewcommand{\thetable}{\theenumi}
%\renewcommand{\theequation}{\theenumi}

%\begin{abstract}
%%\boldmath
%In this letter, an algorithm for evaluating the exact analytical bit error rate  (BER)  for the piecewise linear (PL) combiner for  multiple relays is presented. Previous results were available only for upto three relays. The algorithm is unique in the sense that  the actual mathematical expressions, that are prohibitively large, need not be explicitly obtained. The diversity gain due to multiple relays is shown through plots of the analytical BER, well supported by simulations. 
%
%\end{abstract}
% IEEEtran.cls defaults to using nonbold math in the Abstract.
% This preserves the distinction between vectors and scalars. However,
% if the journal you are submitting to favors bold math in the abstract,
% then you can use LaTeX's standard command \boldmath at the very start
% of the abstract to achieve this. Many IEEE journals frown on math
% in the abstract anyway.

% Note that keywords are not normally used for peerreview papers.
%\begin{IEEEkeywords}
%Cooperative diversity, decode and forward, piecewise linear
%\end{IEEEkeywords}



% For peer review papers, you can put extra information on the cover
% page as needed:
% \ifCLASSOPTIONpeerreview
% \begin{center} \bfseries EDICS Category: 3-BBND \end{center}
% \fi
%
% For peerreview papers, this IEEEtran command inserts a page break and
% creates the second title. It will be ignored for other modes.
%\IEEEpeerreviewmaketitle




\item Consider the probability space $\brak{\Omega, \mathcal{G}, P}$ where $\Omega = [0,2]$ and $\mathcal{G} = \cbrak{\phi, \Omega, [0,1], (1,2]}$. Let $X$ and $Y$ be two functions on $\Omega$ defined as
\begin{align*}
    X(\omega) = 
    \begin{cases}
        1 & \text{if }\omega \in [0, 1]\\
        2 & \text{if }\omega \in (1, 2]
    \end{cases}
\end{align*}
and
\begin{align*}
    Y(\omega) = 
    \begin{cases}
        2 & \text{if }\omega \in [0, 1.5]\\
        3 & \text{if }\omega \in (1.5, 2].
    \end{cases}
\end{align*}
Then which one of the following statements is true?
\begin{enumerate}
    \item [(A)] $X$ is a random variable with respect to $\mathcal{G}$, but $Y$ is not a random variable with respect to $\mathcal{G}$.
    \item [(B)] $Y$ is a random variable with respect to $\mathcal{G}$, but $X$ is not a random variable with respect to $\mathcal{G}$.
    \item [(C)] Neither $X$ nor $Y$ is a random variable with respect to $\mathcal{G}$.
    \item [(D)] Both $X$ and $Y$ are random variables with respect to $\mathcal{G}$.
\end{enumerate} \hfill (GATE ST 2023)\\
\solution
%\begin{table}[H]
	\centering
\begin{tabular}{|c|c|c|}
\hline
Random variable &Value &Definition\\ \hline
\multirow{3}{*}{X} &0 &Slips of Rs 1\\
&1 &Slips of Rs 5\\
&2 &Slips of Rs 13\\ \hline
\multirow{2}{*}{Y} &0 &Box A\\
&1 &Box B\\\hline
\end{tabular}
\caption{}
\label{tab:Distribution}
\end{table}
See \tabref{tab:Distribution}.
\begin{align}
p_{Y}\brak{k}= \begin{cases} 
      \frac{1}{3} & {k=0} \\
      \frac{2}{3 }& {k=1} 
   \end{cases}
   \\
p_{Y|X}\brak{0|0} = \frac{19}{25}\, 
p_{Y|X}\brak{0|1} = \frac{6}{25}\,
p_{Y|X}\brak{1|0} = \frac{45}{50}\,
p_{Y|X}\brak{1|2} = \frac{5}{50}
\end{align}
The desired probability is the probability that a slip drawn at random is marked other than Rs 1,
\begin{align}
&=1-p_X\brak{0}\\
&= p_X(1) + p_X(2)
\end{align}
Using Bayes theorem,
\begin{align}
&= p_Y\brak{0} \times \pr{Y=0 | X=1} + p_Y\brak{1} \times \pr{Y=1|X=2}\\
&=\frac{1}{3} \times \frac{6}{25} + \frac{2}{3} \times \frac{5}{50}\\
&=\frac{11}{75}
\end{align}

\newpage

%\tableofcontents

\bigskip

\renewcommand{\thefigure}{\theenumi}
\renewcommand{\thetable}{\theenumi}
%\renewcommand{\theequation}{\theenumi}

%\begin{abstract}
%%\boldmath
%In this letter, an algorithm for evaluating the exact analytical bit error rate  (BER)  for the piecewise linear (PL) combiner for  multiple relays is presented. Previous results were available only for upto three relays. The algorithm is unique in the sense that  the actual mathematical expressions, that are prohibitively large, need not be explicitly obtained. The diversity gain due to multiple relays is shown through plots of the analytical BER, well supported by simulations. 
%
%\end{abstract}
% IEEEtran.cls defaults to using nonbold math in the Abstract.
% This preserves the distinction between vectors and scalars. However,
% if the journal you are submitting to favors bold math in the abstract,
% then you can use LaTeX's standard command \boldmath at the very start
% of the abstract to achieve this. Many IEEE journals frown on math
% in the abstract anyway.

% Note that keywords are not normally used for peerreview papers.
%\begin{IEEEkeywords}
%Cooperative diversity, decode and forward, piecewise linear
%\end{IEEEkeywords}



% For peer review papers, you can put extra information on the cover
% page as needed:
% \ifCLASSOPTIONpeerreview
% \begin{center} \bfseries EDICS Category: 3-BBND \end{center}
% \fi
%
% For peerreview papers, this IEEEtran command inserts a page break and
% creates the second title. It will be ignored for other modes.
%\IEEEpeerreviewmaketitle




	\item  A die is loaded in such a way that each odd number is twice as likely to occur as
each even number. Find $P(G)$, where $G$ is the event that a number greater than
3 occurs on a single roll of the die.
\\
\solution
		%\begin{table}[H]
	\centering
\begin{tabular}{|c|c|c|}
\hline
Random variable &Value &Definition\\ \hline
\multirow{3}{*}{X} &0 &Slips of Rs 1\\
&1 &Slips of Rs 5\\
&2 &Slips of Rs 13\\ \hline
\multirow{2}{*}{Y} &0 &Box A\\
&1 &Box B\\\hline
\end{tabular}
\caption{}
\label{tab:Distribution}
\end{table}
See \tabref{tab:Distribution}.
\begin{align}
p_{Y}\brak{k}= \begin{cases} 
      \frac{1}{3} & {k=0} \\
      \frac{2}{3 }& {k=1} 
   \end{cases}
   \\
p_{Y|X}\brak{0|0} = \frac{19}{25}\, 
p_{Y|X}\brak{0|1} = \frac{6}{25}\,
p_{Y|X}\brak{1|0} = \frac{45}{50}\,
p_{Y|X}\brak{1|2} = \frac{5}{50}
\end{align}
The desired probability is the probability that a slip drawn at random is marked other than Rs 1,
\begin{align}
&=1-p_X\brak{0}\\
&= p_X(1) + p_X(2)
\end{align}
Using Bayes theorem,
\begin{align}
&= p_Y\brak{0} \times \pr{Y=0 | X=1} + p_Y\brak{1} \times \pr{Y=1|X=2}\\
&=\frac{1}{3} \times \frac{6}{25} + \frac{2}{3} \times \frac{5}{50}\\
&=\frac{11}{75}
\end{align}

\newpage

%\tableofcontents

\bigskip

\renewcommand{\thefigure}{\theenumi}
\renewcommand{\thetable}{\theenumi}
%\renewcommand{\theequation}{\theenumi}

%\begin{abstract}
%%\boldmath
%In this letter, an algorithm for evaluating the exact analytical bit error rate  (BER)  for the piecewise linear (PL) combiner for  multiple relays is presented. Previous results were available only for upto three relays. The algorithm is unique in the sense that  the actual mathematical expressions, that are prohibitively large, need not be explicitly obtained. The diversity gain due to multiple relays is shown through plots of the analytical BER, well supported by simulations. 
%
%\end{abstract}
% IEEEtran.cls defaults to using nonbold math in the Abstract.
% This preserves the distinction between vectors and scalars. However,
% if the journal you are submitting to favors bold math in the abstract,
% then you can use LaTeX's standard command \boldmath at the very start
% of the abstract to achieve this. Many IEEE journals frown on math
% in the abstract anyway.

% Note that keywords are not normally used for peerreview papers.
%\begin{IEEEkeywords}
%Cooperative diversity, decode and forward, piecewise linear
%\end{IEEEkeywords}



% For peer review papers, you can put extra information on the cover
% page as needed:
% \ifCLASSOPTIONpeerreview
% \begin{center} \bfseries EDICS Category: 3-BBND \end{center}
% \fi
%
% For peerreview papers, this IEEEtran command inserts a page break and
% creates the second title. It will be ignored for other modes.
%\IEEEpeerreviewmaketitle




	\item All the jacks, queens and kings are removed from a deck of 52 playing cards. The remaining cards are well shuffled and then one card is drawn at random. Giving ace a value 1 similar value for other cards, find the probability that the card has a value 
		\begin{enumerate}
			\item 7
			\item greater than 7
			\item less than 7
		\end{enumerate}
		%Number of cards left after removing all jacks, queens and kings 
\begin{align}
N	= 52 - 4\times 3
	= 40
\end{align}
%\begin{table}[H]
%\def\arraystretch{1.2}
%\begin{tabular}{|c|c|c|}
%\hline
%	\textbf{Parameter} &\textbf{Value} &\textbf{Description}\\ \hline
%	$X$ &1-10 &Represents the value of the card picked \\ \hline
%\end{tabular}
%\end{table}
Let $1 \le X \le 10$ be the value of the card picked.  Then,
\begin{align}
	p_X(k) &= \Pr(X=k)\ \forall\ 1 \leq k \leq 10\\
	&= \frac{4\times 1}{40}\\
	&= \frac{1}{10}\\
	\therefore p_X(k) &= 
	\begin{cases}
		\frac{1}{10} & 1 \leq k \leq 10\\
		0 & \text{otherwise}
	\end{cases}
\end{align}
and
\begin{align}
	F_{X}(k) &= \sum_{m=0}^{k}p_{X}(m) \quad 1 \leq k \leq 10\\
	&= \frac{k}{10}\\
	\therefore F_{X}(k) &= 
	\begin{cases}
		0 & k \leq 0\\
		\frac{k}{10} & 1\leq k \leq 10\\
		1 & k > 10 
	\end{cases}
\end{align}
\begin{enumerate}
	\item Probability that card has value equal to 7 is
		\begin{align}
			 p_{X}(7)
			= \frac{1}{10}
		\end{align}
	\item Probability that card has value greater than 7 is
		\begin{align}
			1 - F_X(7)
			&= 1 - \frac{7}{10}
			\\
			&= \frac{3}{10}
		\end{align}
	\item Probability that card has value less than 7 is
		\begin{align}
			 F_{X}(6)
			=\frac{6}{10}
		\end{align}
\end{enumerate}

  \item A Lot consists of 48 mobile phones of which 42 are good, 3 have only minor defects and 3 have major defects.Varnika will buy a phone if it is good but the trader will only buy a mobile if it has no major defects. One phone is selected at random from the lot. What is the probability that it is
\begin{enumerate}
	\item acceptable to Varnika?
            \item acceptable to the trader?
\end{enumerate}
\solution
	%\begin{table}[H]
	\centering
\begin{tabular}{|c|c|c|}
\hline
Random variable &Value &Definition\\ \hline
\multirow{3}{*}{X} &0 &Slips of Rs 1\\
&1 &Slips of Rs 5\\
&2 &Slips of Rs 13\\ \hline
\multirow{2}{*}{Y} &0 &Box A\\
&1 &Box B\\\hline
\end{tabular}
\caption{}
\label{tab:Distribution}
\end{table}
See \tabref{tab:Distribution}.
\begin{align}
p_{Y}\brak{k}= \begin{cases} 
      \frac{1}{3} & {k=0} \\
      \frac{2}{3 }& {k=1} 
   \end{cases}
   \\
p_{Y|X}\brak{0|0} = \frac{19}{25}\, 
p_{Y|X}\brak{0|1} = \frac{6}{25}\,
p_{Y|X}\brak{1|0} = \frac{45}{50}\,
p_{Y|X}\brak{1|2} = \frac{5}{50}
\end{align}
The desired probability is the probability that a slip drawn at random is marked other than Rs 1,
\begin{align}
&=1-p_X\brak{0}\\
&= p_X(1) + p_X(2)
\end{align}
Using Bayes theorem,
\begin{align}
&= p_Y\brak{0} \times \pr{Y=0 | X=1} + p_Y\brak{1} \times \pr{Y=1|X=2}\\
&=\frac{1}{3} \times \frac{6}{25} + \frac{2}{3} \times \frac{5}{50}\\
&=\frac{11}{75}
\end{align}

\newpage

%\tableofcontents

\bigskip

\renewcommand{\thefigure}{\theenumi}
\renewcommand{\thetable}{\theenumi}
%\renewcommand{\theequation}{\theenumi}

%\begin{abstract}
%%\boldmath
%In this letter, an algorithm for evaluating the exact analytical bit error rate  (BER)  for the piecewise linear (PL) combiner for  multiple relays is presented. Previous results were available only for upto three relays. The algorithm is unique in the sense that  the actual mathematical expressions, that are prohibitively large, need not be explicitly obtained. The diversity gain due to multiple relays is shown through plots of the analytical BER, well supported by simulations. 
%
%\end{abstract}
% IEEEtran.cls defaults to using nonbold math in the Abstract.
% This preserves the distinction between vectors and scalars. However,
% if the journal you are submitting to favors bold math in the abstract,
% then you can use LaTeX's standard command \boldmath at the very start
% of the abstract to achieve this. Many IEEE journals frown on math
% in the abstract anyway.

% Note that keywords are not normally used for peerreview papers.
%\begin{IEEEkeywords}
%Cooperative diversity, decode and forward, piecewise linear
%\end{IEEEkeywords}



% For peer review papers, you can put extra information on the cover
% page as needed:
% \ifCLASSOPTIONpeerreview
% \begin{center} \bfseries EDICS Category: 3-BBND \end{center}
% \fi
%
% For peerreview papers, this IEEEtran command inserts a page break and
% creates the second title. It will be ignored for other modes.
%\IEEEpeerreviewmaketitle




 \item A student says that if you throw a die, it will show up 1 or not 1. Therefore, the probability of getting 1 and the probability of getting 'not 1' each is equal to $\frac{1}{2}$. Is this correct? Give reasons.\\
 \solution
        %\begin{table}[H]
	\centering
\begin{tabular}{|c|c|c|}
\hline
Random variable &Value &Definition\\ \hline
\multirow{3}{*}{X} &0 &Slips of Rs 1\\
&1 &Slips of Rs 5\\
&2 &Slips of Rs 13\\ \hline
\multirow{2}{*}{Y} &0 &Box A\\
&1 &Box B\\\hline
\end{tabular}
\caption{}
\label{tab:Distribution}
\end{table}
See \tabref{tab:Distribution}.
\begin{align}
p_{Y}\brak{k}= \begin{cases} 
      \frac{1}{3} & {k=0} \\
      \frac{2}{3 }& {k=1} 
   \end{cases}
   \\
p_{Y|X}\brak{0|0} = \frac{19}{25}\, 
p_{Y|X}\brak{0|1} = \frac{6}{25}\,
p_{Y|X}\brak{1|0} = \frac{45}{50}\,
p_{Y|X}\brak{1|2} = \frac{5}{50}
\end{align}
The desired probability is the probability that a slip drawn at random is marked other than Rs 1,
\begin{align}
&=1-p_X\brak{0}\\
&= p_X(1) + p_X(2)
\end{align}
Using Bayes theorem,
\begin{align}
&= p_Y\brak{0} \times \pr{Y=0 | X=1} + p_Y\brak{1} \times \pr{Y=1|X=2}\\
&=\frac{1}{3} \times \frac{6}{25} + \frac{2}{3} \times \frac{5}{50}\\
&=\frac{11}{75}
\end{align}

\newpage

%\tableofcontents

\bigskip

\renewcommand{\thefigure}{\theenumi}
\renewcommand{\thetable}{\theenumi}
%\renewcommand{\theequation}{\theenumi}

%\begin{abstract}
%%\boldmath
%In this letter, an algorithm for evaluating the exact analytical bit error rate  (BER)  for the piecewise linear (PL) combiner for  multiple relays is presented. Previous results were available only for upto three relays. The algorithm is unique in the sense that  the actual mathematical expressions, that are prohibitively large, need not be explicitly obtained. The diversity gain due to multiple relays is shown through plots of the analytical BER, well supported by simulations. 
%
%\end{abstract}
% IEEEtran.cls defaults to using nonbold math in the Abstract.
% This preserves the distinction between vectors and scalars. However,
% if the journal you are submitting to favors bold math in the abstract,
% then you can use LaTeX's standard command \boldmath at the very start
% of the abstract to achieve this. Many IEEE journals frown on math
% in the abstract anyway.

% Note that keywords are not normally used for peerreview papers.
%\begin{IEEEkeywords}
%Cooperative diversity, decode and forward, piecewise linear
%\end{IEEEkeywords}



% For peer review papers, you can put extra information on the cover
% page as needed:
% \ifCLASSOPTIONpeerreview
% \begin{center} \bfseries EDICS Category: 3-BBND \end{center}
% \fi
%
% For peerreview papers, this IEEEtran command inserts a page break and
% creates the second title. It will be ignored for other modes.
%\IEEEpeerreviewmaketitle




   \item Four candidates A, B, C, D have ap-
plied for the assignment to coach a school cricket
team. If A is twice as likely to be selected as B, and
B and C are given about the same chance of being
selected, while C is twice as likely to be selected
as D, what are the probabilities that
\begin{enumerate}
\item C will be selected?
\item A will not be selected?
\end{enumerate}
	%\begin{table}[H]
	\centering
\begin{tabular}{|c|c|c|}
\hline
Random variable &Value &Definition\\ \hline
\multirow{3}{*}{X} &0 &Slips of Rs 1\\
&1 &Slips of Rs 5\\
&2 &Slips of Rs 13\\ \hline
\multirow{2}{*}{Y} &0 &Box A\\
&1 &Box B\\\hline
\end{tabular}
\caption{}
\label{tab:Distribution}
\end{table}
See \tabref{tab:Distribution}.
\begin{align}
p_{Y}\brak{k}= \begin{cases} 
      \frac{1}{3} & {k=0} \\
      \frac{2}{3 }& {k=1} 
   \end{cases}
   \\
p_{Y|X}\brak{0|0} = \frac{19}{25}\, 
p_{Y|X}\brak{0|1} = \frac{6}{25}\,
p_{Y|X}\brak{1|0} = \frac{45}{50}\,
p_{Y|X}\brak{1|2} = \frac{5}{50}
\end{align}
The desired probability is the probability that a slip drawn at random is marked other than Rs 1,
\begin{align}
&=1-p_X\brak{0}\\
&= p_X(1) + p_X(2)
\end{align}
Using Bayes theorem,
\begin{align}
&= p_Y\brak{0} \times \pr{Y=0 | X=1} + p_Y\brak{1} \times \pr{Y=1|X=2}\\
&=\frac{1}{3} \times \frac{6}{25} + \frac{2}{3} \times \frac{5}{50}\\
&=\frac{11}{75}
\end{align}

\newpage

%\tableofcontents

\bigskip

\renewcommand{\thefigure}{\theenumi}
\renewcommand{\thetable}{\theenumi}
%\renewcommand{\theequation}{\theenumi}

%\begin{abstract}
%%\boldmath
%In this letter, an algorithm for evaluating the exact analytical bit error rate  (BER)  for the piecewise linear (PL) combiner for  multiple relays is presented. Previous results were available only for upto three relays. The algorithm is unique in the sense that  the actual mathematical expressions, that are prohibitively large, need not be explicitly obtained. The diversity gain due to multiple relays is shown through plots of the analytical BER, well supported by simulations. 
%
%\end{abstract}
% IEEEtran.cls defaults to using nonbold math in the Abstract.
% This preserves the distinction between vectors and scalars. However,
% if the journal you are submitting to favors bold math in the abstract,
% then you can use LaTeX's standard command \boldmath at the very start
% of the abstract to achieve this. Many IEEE journals frown on math
% in the abstract anyway.

% Note that keywords are not normally used for peerreview papers.
%\begin{IEEEkeywords}
%Cooperative diversity, decode and forward, piecewise linear
%\end{IEEEkeywords}



% For peer review papers, you can put extra information on the cover
% page as needed:
% \ifCLASSOPTIONpeerreview
% \begin{center} \bfseries EDICS Category: 3-BBND \end{center}
% \fi
%
% For peerreview papers, this IEEEtran command inserts a page break and
% creates the second title. It will be ignored for other modes.
%\IEEEpeerreviewmaketitle




 \item A bag contain 24 balls of which $x$ balls are red, $2x$ are white and $3x$ are blue. A ball is selected at random, What is the probability that it is
\begin{enumerate}[label=\alph*)]
\item not red ?
\item white ?
\end{enumerate}
%\begin{table}[H]
	\centering
\begin{tabular}{|c|c|c|}
\hline
Random variable &Value &Definition\\ \hline
\multirow{3}{*}{X} &0 &Slips of Rs 1\\
&1 &Slips of Rs 5\\
&2 &Slips of Rs 13\\ \hline
\multirow{2}{*}{Y} &0 &Box A\\
&1 &Box B\\\hline
\end{tabular}
\caption{}
\label{tab:Distribution}
\end{table}
See \tabref{tab:Distribution}.
\begin{align}
p_{Y}\brak{k}= \begin{cases} 
      \frac{1}{3} & {k=0} \\
      \frac{2}{3 }& {k=1} 
   \end{cases}
   \\
p_{Y|X}\brak{0|0} = \frac{19}{25}\, 
p_{Y|X}\brak{0|1} = \frac{6}{25}\,
p_{Y|X}\brak{1|0} = \frac{45}{50}\,
p_{Y|X}\brak{1|2} = \frac{5}{50}
\end{align}
The desired probability is the probability that a slip drawn at random is marked other than Rs 1,
\begin{align}
&=1-p_X\brak{0}\\
&= p_X(1) + p_X(2)
\end{align}
Using Bayes theorem,
\begin{align}
&= p_Y\brak{0} \times \pr{Y=0 | X=1} + p_Y\brak{1} \times \pr{Y=1|X=2}\\
&=\frac{1}{3} \times \frac{6}{25} + \frac{2}{3} \times \frac{5}{50}\\
&=\frac{11}{75}
\end{align}

\newpage

%\tableofcontents

\bigskip

\renewcommand{\thefigure}{\theenumi}
\renewcommand{\thetable}{\theenumi}
%\renewcommand{\theequation}{\theenumi}

%\begin{abstract}
%%\boldmath
%In this letter, an algorithm for evaluating the exact analytical bit error rate  (BER)  for the piecewise linear (PL) combiner for  multiple relays is presented. Previous results were available only for upto three relays. The algorithm is unique in the sense that  the actual mathematical expressions, that are prohibitively large, need not be explicitly obtained. The diversity gain due to multiple relays is shown through plots of the analytical BER, well supported by simulations. 
%
%\end{abstract}
% IEEEtran.cls defaults to using nonbold math in the Abstract.
% This preserves the distinction between vectors and scalars. However,
% if the journal you are submitting to favors bold math in the abstract,
% then you can use LaTeX's standard command \boldmath at the very start
% of the abstract to achieve this. Many IEEE journals frown on math
% in the abstract anyway.

% Note that keywords are not normally used for peerreview papers.
%\begin{IEEEkeywords}
%Cooperative diversity, decode and forward, piecewise linear
%\end{IEEEkeywords}



% For peer review papers, you can put extra information on the cover
% page as needed:
% \ifCLASSOPTIONpeerreview
% \begin{center} \bfseries EDICS Category: 3-BBND \end{center}
% \fi
%
% For peerreview papers, this IEEEtran command inserts a page break and
% creates the second title. It will be ignored for other modes.
%\IEEEpeerreviewmaketitle




If the letters of the word ASSASSINATION are arranged at random. Find the Probability that
\begin{enumerate}[label=(\alph*)]
\item Four $S's$ come consecutively in the word
\item Two  $I's$ and two $N's$ come together
\item All $A's$ are not coming together
\item No two $A's$ are coming together
\end{enumerate}
%\begin{table}[H]
	\centering
\begin{tabular}{|c|c|c|}
\hline
Random variable &Value &Definition\\ \hline
\multirow{3}{*}{X} &0 &Slips of Rs 1\\
&1 &Slips of Rs 5\\
&2 &Slips of Rs 13\\ \hline
\multirow{2}{*}{Y} &0 &Box A\\
&1 &Box B\\\hline
\end{tabular}
\caption{}
\label{tab:Distribution}
\end{table}
See \tabref{tab:Distribution}.
\begin{align}
p_{Y}\brak{k}= \begin{cases} 
      \frac{1}{3} & {k=0} \\
      \frac{2}{3 }& {k=1} 
   \end{cases}
   \\
p_{Y|X}\brak{0|0} = \frac{19}{25}\, 
p_{Y|X}\brak{0|1} = \frac{6}{25}\,
p_{Y|X}\brak{1|0} = \frac{45}{50}\,
p_{Y|X}\brak{1|2} = \frac{5}{50}
\end{align}
The desired probability is the probability that a slip drawn at random is marked other than Rs 1,
\begin{align}
&=1-p_X\brak{0}\\
&= p_X(1) + p_X(2)
\end{align}
Using Bayes theorem,
\begin{align}
&= p_Y\brak{0} \times \pr{Y=0 | X=1} + p_Y\brak{1} \times \pr{Y=1|X=2}\\
&=\frac{1}{3} \times \frac{6}{25} + \frac{2}{3} \times \frac{5}{50}\\
&=\frac{11}{75}
\end{align}

\newpage

%\tableofcontents

\bigskip

\renewcommand{\thefigure}{\theenumi}
\renewcommand{\thetable}{\theenumi}
%\renewcommand{\theequation}{\theenumi}

%\begin{abstract}
%%\boldmath
%In this letter, an algorithm for evaluating the exact analytical bit error rate  (BER)  for the piecewise linear (PL) combiner for  multiple relays is presented. Previous results were available only for upto three relays. The algorithm is unique in the sense that  the actual mathematical expressions, that are prohibitively large, need not be explicitly obtained. The diversity gain due to multiple relays is shown through plots of the analytical BER, well supported by simulations. 
%
%\end{abstract}
% IEEEtran.cls defaults to using nonbold math in the Abstract.
% This preserves the distinction between vectors and scalars. However,
% if the journal you are submitting to favors bold math in the abstract,
% then you can use LaTeX's standard command \boldmath at the very start
% of the abstract to achieve this. Many IEEE journals frown on math
% in the abstract anyway.

% Note that keywords are not normally used for peerreview papers.
%\begin{IEEEkeywords}
%Cooperative diversity, decode and forward, piecewise linear
%\end{IEEEkeywords}



% For peer review papers, you can put extra information on the cover
% page as needed:
% \ifCLASSOPTIONpeerreview
% \begin{center} \bfseries EDICS Category: 3-BBND \end{center}
% \fi
%
% For peerreview papers, this IEEEtran command inserts a page break and
% creates the second title. It will be ignored for other modes.
%\IEEEpeerreviewmaketitle




	\item One urn contains two black balls (labelled B1 and B2) and one white ball. A
	second urn contains one black ball and two white balls (labelled W1 and W2).
	Suppose the following experiment is performed. One of the two urns is chosen
	at random. Next a ball is randomly chosen from the urn. Then a second ball is
	chosen at random from the same urn without replacing the first ball.
	
	\begin{enumerate}
	\item What is the probability that two black balls are chosen?
	
	\item What is the probability that two balls of opposite colour are chosen?
	\end{enumerate}
	\solution
	%\begin{align}
    \label{eq:12.13.6.18.1}
	\because	\pr{A|B} &> \pr{A},\
\frac{\pr{AB}}{\pr{B}} > \pr{A}
\\
    \label{eq:12.13.6.18.2}
	\implies \pr{AB} &> \pr{A}\pr{B}
	\\
	\text{or, } \frac{\pr{AB}}{\pr{A}} &=\pr{B|A} > \pr{A}
\end{align}

\end{enumerate}

\item In a certain lottery 10,000 tickets are sold and ten equal prizes are awarded. What is the probability of not getting a prize if you buy (a) one ticket (b) two tickets (c) 10 tickets ?	
\\
\solution
		%\begin{enumerate}[label=\thesection.\arabic*,ref=\thesection.\theenumi]
	\item One card is drawn from a well-shuffled deck of 52 cards. Find the probability of getting
\begin{enumerate}
\item A king of red colour 
\item A face card 
\item A red face card
\item The jack of hearts
\item A spade
\item The queen of diamonds

\end{enumerate}
\solution
		%\begin{table}[H]
	\centering
\begin{tabular}{|c|c|c|}
\hline
Random variable &Value &Definition\\ \hline
\multirow{3}{*}{X} &0 &Slips of Rs 1\\
&1 &Slips of Rs 5\\
&2 &Slips of Rs 13\\ \hline
\multirow{2}{*}{Y} &0 &Box A\\
&1 &Box B\\\hline
\end{tabular}
\caption{}
\label{tab:Distribution}
\end{table}
See \tabref{tab:Distribution}.
\begin{align}
p_{Y}\brak{k}= \begin{cases} 
      \frac{1}{3} & {k=0} \\
      \frac{2}{3 }& {k=1} 
   \end{cases}
   \\
p_{Y|X}\brak{0|0} = \frac{19}{25}\, 
p_{Y|X}\brak{0|1} = \frac{6}{25}\,
p_{Y|X}\brak{1|0} = \frac{45}{50}\,
p_{Y|X}\brak{1|2} = \frac{5}{50}
\end{align}
The desired probability is the probability that a slip drawn at random is marked other than Rs 1,
\begin{align}
&=1-p_X\brak{0}\\
&= p_X(1) + p_X(2)
\end{align}
Using Bayes theorem,
\begin{align}
&= p_Y\brak{0} \times \pr{Y=0 | X=1} + p_Y\brak{1} \times \pr{Y=1|X=2}\\
&=\frac{1}{3} \times \frac{6}{25} + \frac{2}{3} \times \frac{5}{50}\\
&=\frac{11}{75}
\end{align}

\newpage

%\tableofcontents

\bigskip

\renewcommand{\thefigure}{\theenumi}
\renewcommand{\thetable}{\theenumi}
%\renewcommand{\theequation}{\theenumi}

%\begin{abstract}
%%\boldmath
%In this letter, an algorithm for evaluating the exact analytical bit error rate  (BER)  for the piecewise linear (PL) combiner for  multiple relays is presented. Previous results were available only for upto three relays. The algorithm is unique in the sense that  the actual mathematical expressions, that are prohibitively large, need not be explicitly obtained. The diversity gain due to multiple relays is shown through plots of the analytical BER, well supported by simulations. 
%
%\end{abstract}
% IEEEtran.cls defaults to using nonbold math in the Abstract.
% This preserves the distinction between vectors and scalars. However,
% if the journal you are submitting to favors bold math in the abstract,
% then you can use LaTeX's standard command \boldmath at the very start
% of the abstract to achieve this. Many IEEE journals frown on math
% in the abstract anyway.

% Note that keywords are not normally used for peerreview papers.
%\begin{IEEEkeywords}
%Cooperative diversity, decode and forward, piecewise linear
%\end{IEEEkeywords}



% For peer review papers, you can put extra information on the cover
% page as needed:
% \ifCLASSOPTIONpeerreview
% \begin{center} \bfseries EDICS Category: 3-BBND \end{center}
% \fi
%
% For peerreview papers, this IEEEtran command inserts a page break and
% creates the second title. It will be ignored for other modes.
%\IEEEpeerreviewmaketitle




	\item Five cards—the ten, jack, queen, king and ace of diamonds, are well-shuffled with their face downwards. One card is then picked up at random.
\begin{enumerate}
\item
What is the probability that the card is the queen? 
\item
If the queen is drawn and put aside, what is the probability that the second card picked up is (a) an ace? (b) a queen?\\
\end{enumerate}
\solution
		%\begin{enumerate}[label=\thesection.\arabic*,ref=\thesection.\theenumi]
	\item One card is drawn from a well-shuffled deck of 52 cards. Find the probability of getting
\begin{enumerate}
\item A king of red colour 
\item A face card 
\item A red face card
\item The jack of hearts
\item A spade
\item The queen of diamonds

\end{enumerate}
\solution
		%\input{ncert/10/15/1/14/main.tex}
	\item Five cards—the ten, jack, queen, king and ace of diamonds, are well-shuffled with their face downwards. One card is then picked up at random.
\begin{enumerate}
\item
What is the probability that the card is the queen? 
\item
If the queen is drawn and put aside, what is the probability that the second card picked up is (a) an ace? (b) a queen?\\
\end{enumerate}
\solution
		%\input{ncert/10/15/1/15/defs.tex}
	\item A bag contains $5$ red balls and some blue balls. If the probability of drawing a blue ball is double that if a red ball, determine the number of blue balls in the bag. 
		\\
\solution
		%\input{ncert/10/15/2/3/defs.tex}
	\item A card is selected from a pack of 52 cards.
 \begin{enumerate}[label=(\alph*)] 
                 \item How many points are there in the sample space?
                 \item Calculate the probability that the card is an ace of spades.
                 \item Calculate the probability that the card is (i) an ace and (ii) black card.
 \end{enumerate}
\solution
		%\input{ncert/11/16/3/4/main.tex}
\item Four cards are drawn from a well-shuffled deck of 52 cards. What is the probability of obtaining 3 diamonds and one spade.
\\
\solution
		%\input{ncert/11/16/4/2/defs.tex}
\item In a certain lottery 10,000 tickets are sold and ten equal prizes are awarded. What is the probability of not getting a prize if you buy (a) one ticket (b) two tickets (c) 10 tickets ?	
\\
\solution
		%\input{ncert/11/16/4/4/defs.tex}
		%
\item 
Out of 100 students, two sections of 40 and 60 are formed. If you and your friend are among the 100 students, what is the probability that
\begin{enumerate}
\item you both enter the same section?
\item you both enter the different sections?
\end{enumerate}
\solution
		%\input{ncert/11/16/4/5/defs.tex}
	\item 
The number lock of a suitcase has 4 wheels each labelled with ten digits i.e. from 0 to 9.The lock opens with a sequence of four digits with no repeats.What is the probability of a person getting the right sequence to open the suitcase.
\\
\solution
		%\input{ncert/11/16/4/10/defs.tex}
		%
\item 
Two cards are drawn at random and without replacement from a pack of 52 playing cards. Find the probability that both the cards are black.
\\
\solution
		%\input{ncert/12/13/2/2/defs.tex}
		\item A box of oranges is inspected by examining three randomly selected oranges drawn without replacement. If all the three oranges are good, the box is approved for sale, otherwise, it is rejected. Find the probability that a box containing 15 oranges out of which 12 are good and 3 are bad ones will be approved for sale.
		\label{ncert/12/13/2/3/defs.tex}
		\item Two balls are drawn at random with replacement from a box containing 10 black and 8 red balls. Find the probability that
		\label{ncert/12/13/2/12}
\begin{enumerate}
\item both balls are red.
\item first ball is black and second is red.
\item one of them is black and other is red.
\end{enumerate}

\item In a hostel, 60\% of the students read Hindi newspaper, 40\% read English newspaper and 20\% read both Hindi and English newspapers. A student is selected at random.
		\label{ncert/12/13/2/15}
\begin{enumerate}
\item Find the probability that she reads neither Hindi nor English newspapers.
\item If she reads Hindi newspaper, find the probability that she reads English newspaper.
\item If she reads English newspaper, find the probability that she reads Hindi newspaper.\\
\end{enumerate}
\item The probability of obtaining an even prime number on each die, when a pair of dice is rolled is 
\begin{enumerate}
    \item $0$ 
    
    \item $\frac{1}{3}$ 
    
    \item $\frac{1}{12}$ 
    
    \item $\frac{1}{36}$ 
\end{enumerate}
\solution
		%\input{ncert/12/13/2/17/defs.tex}
	\item A bag contains 4 red and 4 black balls, another bag contains 2 red and 6 black balls. One of the two bags is selected at random and a ball is drawn from the bag which is found to be red. Find the probability that the ball is drawn from the first bag.
\\
\solution
		%\input{ncert/12/13/3/2/main.tex}
  \item
  Cards with numbers 2 to 101 are placed in a box. A card is selected at random.Find the probability that the card has
\begin{enumerate}[label=(\roman*)]
	\item an even number 
	\item a square number
\end{enumerate}
\solution
%\input{exemplar/10/13/3/32/main.tex}
\item
The king, queen and jack of clubs are removed from a deck of 52 playing cards and then well shuffled. Now one card is drawn at random from the remaining cards.  Determine the probability that the card is
\begin{enumerate}[label=(\roman*)]
\item a club
\item 10 of hearts
\end{enumerate}
\solution
%\input{exemplar/10/13/3/29/main.tex}
\item A team of medical students doing their internship have to assist during surgeries
at a city hospital. The probabilities of surgeries rated as very complex, complex,
routine, simple or very simple are respectively, 0.15, 0.20, 0.31, 0.26, .08. Find
the probabilities that a particular surgery will be rated
\begin{enumerate}
	\item complex or very complex;
	\item neither very complex nor very simple;
	\item routine or complex
	\item routine or simple
\end{enumerate}
\solution
%\input{exemplar/11/16/3/8(1)/main.tex}
\item A card is selected from a pack of 52 cards.
\begin{enumerate}[label=(\alph*)]
    \item How many points are there in the sample space?
    \item Calculate the probability that the card is an ace of spades.
    \item Calculate the probability that the card is (i) an ace and (ii) black card.
\end{enumerate}
\solution
%\input{exemplar/11/16/3/4/main2.tex}
\item The probability that a non leap year selected at random will contain 53 sundays.
\\
\solution
%\input{exemplar/10/13/1/19/main.tex}
\item One of the four persons John, Rita, Aslam or Gurpreet will be promoted next
month. Consequently the sample space consists of four elementary outcomes
S = {John promoted, Rita promoted, Aslam promoted, Gurpreet promoted}
You are told that the chances of John’s promotion is same as that of Gurpreet,
Rita’s chances of promotion are twice as likely as Johns. Aslam’s chances are
four times that of John.
\begin{enumerate}
	\item Determine
	\begin{enumerate}
		\item P (John promoted)
		\item P (Rita promoted)
		\item P (Aslam promoted)
		\item P (Gurpreet promoted)
	\end{enumerate}
	\item If A = {John promoted or Gurpreet promoted}, find P (A).
\end{enumerate}
\solution
%\input{exemplar/11/16/3/10/main.tex}
\item A card is drawn from a deck of 52 cards. Find the probability of getting a king or a heart or a red card.\\
\solution
%\input{exemplar/11/16/3/15/main.tex}
\item The probability that a student will pass his examination is 0.73, the probability of
the student getting a compartment is 0.13, and the probability that the student will
either pass or get compartment is 0.96. State True or False.\\
\solution
%\input{exemplar/11/16/3/31/main.tex}
\item A card is selected from a pack of 52 cards\\
\begin{enumerate}[label=(\alph*)]
\item How many points are there in the sample space?
\item Calculate the probability that the cards is an ace of spades.
\item Calculate the probability that the card is (i) an ace (ii)black card.\\
\end{enumerate}
%\input{ncert/11/16/3/4_1/Prob_4.tex}
\item In a non-leap year, the probability of having 53 tuesdays or 53 wednesdays is\\
\solution
%\input{exemplar/11/16/3/18/main.tex}
\item There are 1000 sealed envelopes in a box, 10 of them contain a cash prize of
Rs 100 each, 100 of them contain a cash prize of Rs 50 each and 200 of them
contain a cash prize of Rs 10 each and rest do not contain any cash prize. If they
are well shuffled and an envelope is picked up out, what is the probability that it
contains no cash prize?\\
\solution
%\input{exemplar/10/13/3/34/main.tex}
\item 
A die is thrown and a card is selected at random from a deck of 52 playing cards. The probability of getting an even number on the die and a spade card.\\
\solution
%\input{exemplar/12/13/3/78/main.tex}
\item
If 4-digit numbers greater than 5,000 are randomly formed from the digits 0, 1, 3, 5, and 7, what is the probability of forming a number divisible by 5 when:
\begin{enumerate}
    \item The digits are repeated?
    \item The repetition of digits is not allowed?
\end{enumerate}
\solution
%\input{ncert/11/16/4/9/main.tex}
\item Consider the probability space $\brak{\Omega, \mathcal{G}, P}$ where $\Omega = [0,2]$ and $\mathcal{G} = \cbrak{\phi, \Omega, [0,1], (1,2]}$. Let $X$ and $Y$ be two functions on $\Omega$ defined as
\begin{align*}
    X(\omega) = 
    \begin{cases}
        1 & \text{if }\omega \in [0, 1]\\
        2 & \text{if }\omega \in (1, 2]
    \end{cases}
\end{align*}
and
\begin{align*}
    Y(\omega) = 
    \begin{cases}
        2 & \text{if }\omega \in [0, 1.5]\\
        3 & \text{if }\omega \in (1.5, 2].
    \end{cases}
\end{align*}
Then which one of the following statements is true?
\begin{enumerate}
    \item [(A)] $X$ is a random variable with respect to $\mathcal{G}$, but $Y$ is not a random variable with respect to $\mathcal{G}$.
    \item [(B)] $Y$ is a random variable with respect to $\mathcal{G}$, but $X$ is not a random variable with respect to $\mathcal{G}$.
    \item [(C)] Neither $X$ nor $Y$ is a random variable with respect to $\mathcal{G}$.
    \item [(D)] Both $X$ and $Y$ are random variables with respect to $\mathcal{G}$.
\end{enumerate} \hfill (GATE ST 2023)\\
\solution
%\input{gate/ST/2023/14/main.tex}
	\item  A die is loaded in such a way that each odd number is twice as likely to occur as
each even number. Find $P(G)$, where $G$ is the event that a number greater than
3 occurs on a single roll of the die.
\\
\solution
		%\input{exemplar/11/16/3/5/main.tex}
	\item All the jacks, queens and kings are removed from a deck of 52 playing cards. The remaining cards are well shuffled and then one card is drawn at random. Giving ace a value 1 similar value for other cards, find the probability that the card has a value 
		\begin{enumerate}
			\item 7
			\item greater than 7
			\item less than 7
		\end{enumerate}
		%\input{exemplar/10/13/3/30/main.tex}
  \item A Lot consists of 48 mobile phones of which 42 are good, 3 have only minor defects and 3 have major defects.Varnika will buy a phone if it is good but the trader will only buy a mobile if it has no major defects. One phone is selected at random from the lot. What is the probability that it is
\begin{enumerate}
	\item acceptable to Varnika?
            \item acceptable to the trader?
\end{enumerate}
\solution
	%\input{exemplar/10/13/3/40/main.tex}
 \item A student says that if you throw a die, it will show up 1 or not 1. Therefore, the probability of getting 1 and the probability of getting 'not 1' each is equal to $\frac{1}{2}$. Is this correct? Give reasons.\\
 \solution
        %\input{exemplar/10/13/2/9/main.tex}
   \item Four candidates A, B, C, D have ap-
plied for the assignment to coach a school cricket
team. If A is twice as likely to be selected as B, and
B and C are given about the same chance of being
selected, while C is twice as likely to be selected
as D, what are the probabilities that
\begin{enumerate}
\item C will be selected?
\item A will not be selected?
\end{enumerate}
	%\input{exemplar/11/16/3/9/main.tex}
 \item A bag contain 24 balls of which $x$ balls are red, $2x$ are white and $3x$ are blue. A ball is selected at random, What is the probability that it is
\begin{enumerate}[label=\alph*)]
\item not red ?
\item white ?
\end{enumerate}
%\input{exemplar/10/13/3/41/main.tex}
If the letters of the word ASSASSINATION are arranged at random. Find the Probability that
\begin{enumerate}[label=(\alph*)]
\item Four $S's$ come consecutively in the word
\item Two  $I's$ and two $N's$ come together
\item All $A's$ are not coming together
\item No two $A's$ are coming together
\end{enumerate}
%\input{exemplar/11/16/3/14/main.tex}
	\item One urn contains two black balls (labelled B1 and B2) and one white ball. A
	second urn contains one black ball and two white balls (labelled W1 and W2).
	Suppose the following experiment is performed. One of the two urns is chosen
	at random. Next a ball is randomly chosen from the urn. Then a second ball is
	chosen at random from the same urn without replacing the first ball.
	
	\begin{enumerate}
	\item What is the probability that two black balls are chosen?
	
	\item What is the probability that two balls of opposite colour are chosen?
	\end{enumerate}
	\solution
	%\input{exemplar/11/16/3/12/main1.tex}
\end{enumerate}

	\item A bag contains $5$ red balls and some blue balls. If the probability of drawing a blue ball is double that if a red ball, determine the number of blue balls in the bag. 
		\\
\solution
		%\begin{enumerate}[label=\thesection.\arabic*,ref=\thesection.\theenumi]
	\item One card is drawn from a well-shuffled deck of 52 cards. Find the probability of getting
\begin{enumerate}
\item A king of red colour 
\item A face card 
\item A red face card
\item The jack of hearts
\item A spade
\item The queen of diamonds

\end{enumerate}
\solution
		%\input{ncert/10/15/1/14/main.tex}
	\item Five cards—the ten, jack, queen, king and ace of diamonds, are well-shuffled with their face downwards. One card is then picked up at random.
\begin{enumerate}
\item
What is the probability that the card is the queen? 
\item
If the queen is drawn and put aside, what is the probability that the second card picked up is (a) an ace? (b) a queen?\\
\end{enumerate}
\solution
		%\input{ncert/10/15/1/15/defs.tex}
	\item A bag contains $5$ red balls and some blue balls. If the probability of drawing a blue ball is double that if a red ball, determine the number of blue balls in the bag. 
		\\
\solution
		%\input{ncert/10/15/2/3/defs.tex}
	\item A card is selected from a pack of 52 cards.
 \begin{enumerate}[label=(\alph*)] 
                 \item How many points are there in the sample space?
                 \item Calculate the probability that the card is an ace of spades.
                 \item Calculate the probability that the card is (i) an ace and (ii) black card.
 \end{enumerate}
\solution
		%\input{ncert/11/16/3/4/main.tex}
\item Four cards are drawn from a well-shuffled deck of 52 cards. What is the probability of obtaining 3 diamonds and one spade.
\\
\solution
		%\input{ncert/11/16/4/2/defs.tex}
\item In a certain lottery 10,000 tickets are sold and ten equal prizes are awarded. What is the probability of not getting a prize if you buy (a) one ticket (b) two tickets (c) 10 tickets ?	
\\
\solution
		%\input{ncert/11/16/4/4/defs.tex}
		%
\item 
Out of 100 students, two sections of 40 and 60 are formed. If you and your friend are among the 100 students, what is the probability that
\begin{enumerate}
\item you both enter the same section?
\item you both enter the different sections?
\end{enumerate}
\solution
		%\input{ncert/11/16/4/5/defs.tex}
	\item 
The number lock of a suitcase has 4 wheels each labelled with ten digits i.e. from 0 to 9.The lock opens with a sequence of four digits with no repeats.What is the probability of a person getting the right sequence to open the suitcase.
\\
\solution
		%\input{ncert/11/16/4/10/defs.tex}
		%
\item 
Two cards are drawn at random and without replacement from a pack of 52 playing cards. Find the probability that both the cards are black.
\\
\solution
		%\input{ncert/12/13/2/2/defs.tex}
		\item A box of oranges is inspected by examining three randomly selected oranges drawn without replacement. If all the three oranges are good, the box is approved for sale, otherwise, it is rejected. Find the probability that a box containing 15 oranges out of which 12 are good and 3 are bad ones will be approved for sale.
		\label{ncert/12/13/2/3/defs.tex}
		\item Two balls are drawn at random with replacement from a box containing 10 black and 8 red balls. Find the probability that
		\label{ncert/12/13/2/12}
\begin{enumerate}
\item both balls are red.
\item first ball is black and second is red.
\item one of them is black and other is red.
\end{enumerate}

\item In a hostel, 60\% of the students read Hindi newspaper, 40\% read English newspaper and 20\% read both Hindi and English newspapers. A student is selected at random.
		\label{ncert/12/13/2/15}
\begin{enumerate}
\item Find the probability that she reads neither Hindi nor English newspapers.
\item If she reads Hindi newspaper, find the probability that she reads English newspaper.
\item If she reads English newspaper, find the probability that she reads Hindi newspaper.\\
\end{enumerate}
\item The probability of obtaining an even prime number on each die, when a pair of dice is rolled is 
\begin{enumerate}
    \item $0$ 
    
    \item $\frac{1}{3}$ 
    
    \item $\frac{1}{12}$ 
    
    \item $\frac{1}{36}$ 
\end{enumerate}
\solution
		%\input{ncert/12/13/2/17/defs.tex}
	\item A bag contains 4 red and 4 black balls, another bag contains 2 red and 6 black balls. One of the two bags is selected at random and a ball is drawn from the bag which is found to be red. Find the probability that the ball is drawn from the first bag.
\\
\solution
		%\input{ncert/12/13/3/2/main.tex}
  \item
  Cards with numbers 2 to 101 are placed in a box. A card is selected at random.Find the probability that the card has
\begin{enumerate}[label=(\roman*)]
	\item an even number 
	\item a square number
\end{enumerate}
\solution
%\input{exemplar/10/13/3/32/main.tex}
\item
The king, queen and jack of clubs are removed from a deck of 52 playing cards and then well shuffled. Now one card is drawn at random from the remaining cards.  Determine the probability that the card is
\begin{enumerate}[label=(\roman*)]
\item a club
\item 10 of hearts
\end{enumerate}
\solution
%\input{exemplar/10/13/3/29/main.tex}
\item A team of medical students doing their internship have to assist during surgeries
at a city hospital. The probabilities of surgeries rated as very complex, complex,
routine, simple or very simple are respectively, 0.15, 0.20, 0.31, 0.26, .08. Find
the probabilities that a particular surgery will be rated
\begin{enumerate}
	\item complex or very complex;
	\item neither very complex nor very simple;
	\item routine or complex
	\item routine or simple
\end{enumerate}
\solution
%\input{exemplar/11/16/3/8(1)/main.tex}
\item A card is selected from a pack of 52 cards.
\begin{enumerate}[label=(\alph*)]
    \item How many points are there in the sample space?
    \item Calculate the probability that the card is an ace of spades.
    \item Calculate the probability that the card is (i) an ace and (ii) black card.
\end{enumerate}
\solution
%\input{exemplar/11/16/3/4/main2.tex}
\item The probability that a non leap year selected at random will contain 53 sundays.
\\
\solution
%\input{exemplar/10/13/1/19/main.tex}
\item One of the four persons John, Rita, Aslam or Gurpreet will be promoted next
month. Consequently the sample space consists of four elementary outcomes
S = {John promoted, Rita promoted, Aslam promoted, Gurpreet promoted}
You are told that the chances of John’s promotion is same as that of Gurpreet,
Rita’s chances of promotion are twice as likely as Johns. Aslam’s chances are
four times that of John.
\begin{enumerate}
	\item Determine
	\begin{enumerate}
		\item P (John promoted)
		\item P (Rita promoted)
		\item P (Aslam promoted)
		\item P (Gurpreet promoted)
	\end{enumerate}
	\item If A = {John promoted or Gurpreet promoted}, find P (A).
\end{enumerate}
\solution
%\input{exemplar/11/16/3/10/main.tex}
\item A card is drawn from a deck of 52 cards. Find the probability of getting a king or a heart or a red card.\\
\solution
%\input{exemplar/11/16/3/15/main.tex}
\item The probability that a student will pass his examination is 0.73, the probability of
the student getting a compartment is 0.13, and the probability that the student will
either pass or get compartment is 0.96. State True or False.\\
\solution
%\input{exemplar/11/16/3/31/main.tex}
\item A card is selected from a pack of 52 cards\\
\begin{enumerate}[label=(\alph*)]
\item How many points are there in the sample space?
\item Calculate the probability that the cards is an ace of spades.
\item Calculate the probability that the card is (i) an ace (ii)black card.\\
\end{enumerate}
%\input{ncert/11/16/3/4_1/Prob_4.tex}
\item In a non-leap year, the probability of having 53 tuesdays or 53 wednesdays is\\
\solution
%\input{exemplar/11/16/3/18/main.tex}
\item There are 1000 sealed envelopes in a box, 10 of them contain a cash prize of
Rs 100 each, 100 of them contain a cash prize of Rs 50 each and 200 of them
contain a cash prize of Rs 10 each and rest do not contain any cash prize. If they
are well shuffled and an envelope is picked up out, what is the probability that it
contains no cash prize?\\
\solution
%\input{exemplar/10/13/3/34/main.tex}
\item 
A die is thrown and a card is selected at random from a deck of 52 playing cards. The probability of getting an even number on the die and a spade card.\\
\solution
%\input{exemplar/12/13/3/78/main.tex}
\item
If 4-digit numbers greater than 5,000 are randomly formed from the digits 0, 1, 3, 5, and 7, what is the probability of forming a number divisible by 5 when:
\begin{enumerate}
    \item The digits are repeated?
    \item The repetition of digits is not allowed?
\end{enumerate}
\solution
%\input{ncert/11/16/4/9/main.tex}
\item Consider the probability space $\brak{\Omega, \mathcal{G}, P}$ where $\Omega = [0,2]$ and $\mathcal{G} = \cbrak{\phi, \Omega, [0,1], (1,2]}$. Let $X$ and $Y$ be two functions on $\Omega$ defined as
\begin{align*}
    X(\omega) = 
    \begin{cases}
        1 & \text{if }\omega \in [0, 1]\\
        2 & \text{if }\omega \in (1, 2]
    \end{cases}
\end{align*}
and
\begin{align*}
    Y(\omega) = 
    \begin{cases}
        2 & \text{if }\omega \in [0, 1.5]\\
        3 & \text{if }\omega \in (1.5, 2].
    \end{cases}
\end{align*}
Then which one of the following statements is true?
\begin{enumerate}
    \item [(A)] $X$ is a random variable with respect to $\mathcal{G}$, but $Y$ is not a random variable with respect to $\mathcal{G}$.
    \item [(B)] $Y$ is a random variable with respect to $\mathcal{G}$, but $X$ is not a random variable with respect to $\mathcal{G}$.
    \item [(C)] Neither $X$ nor $Y$ is a random variable with respect to $\mathcal{G}$.
    \item [(D)] Both $X$ and $Y$ are random variables with respect to $\mathcal{G}$.
\end{enumerate} \hfill (GATE ST 2023)\\
\solution
%\input{gate/ST/2023/14/main.tex}
	\item  A die is loaded in such a way that each odd number is twice as likely to occur as
each even number. Find $P(G)$, where $G$ is the event that a number greater than
3 occurs on a single roll of the die.
\\
\solution
		%\input{exemplar/11/16/3/5/main.tex}
	\item All the jacks, queens and kings are removed from a deck of 52 playing cards. The remaining cards are well shuffled and then one card is drawn at random. Giving ace a value 1 similar value for other cards, find the probability that the card has a value 
		\begin{enumerate}
			\item 7
			\item greater than 7
			\item less than 7
		\end{enumerate}
		%\input{exemplar/10/13/3/30/main.tex}
  \item A Lot consists of 48 mobile phones of which 42 are good, 3 have only minor defects and 3 have major defects.Varnika will buy a phone if it is good but the trader will only buy a mobile if it has no major defects. One phone is selected at random from the lot. What is the probability that it is
\begin{enumerate}
	\item acceptable to Varnika?
            \item acceptable to the trader?
\end{enumerate}
\solution
	%\input{exemplar/10/13/3/40/main.tex}
 \item A student says that if you throw a die, it will show up 1 or not 1. Therefore, the probability of getting 1 and the probability of getting 'not 1' each is equal to $\frac{1}{2}$. Is this correct? Give reasons.\\
 \solution
        %\input{exemplar/10/13/2/9/main.tex}
   \item Four candidates A, B, C, D have ap-
plied for the assignment to coach a school cricket
team. If A is twice as likely to be selected as B, and
B and C are given about the same chance of being
selected, while C is twice as likely to be selected
as D, what are the probabilities that
\begin{enumerate}
\item C will be selected?
\item A will not be selected?
\end{enumerate}
	%\input{exemplar/11/16/3/9/main.tex}
 \item A bag contain 24 balls of which $x$ balls are red, $2x$ are white and $3x$ are blue. A ball is selected at random, What is the probability that it is
\begin{enumerate}[label=\alph*)]
\item not red ?
\item white ?
\end{enumerate}
%\input{exemplar/10/13/3/41/main.tex}
If the letters of the word ASSASSINATION are arranged at random. Find the Probability that
\begin{enumerate}[label=(\alph*)]
\item Four $S's$ come consecutively in the word
\item Two  $I's$ and two $N's$ come together
\item All $A's$ are not coming together
\item No two $A's$ are coming together
\end{enumerate}
%\input{exemplar/11/16/3/14/main.tex}
	\item One urn contains two black balls (labelled B1 and B2) and one white ball. A
	second urn contains one black ball and two white balls (labelled W1 and W2).
	Suppose the following experiment is performed. One of the two urns is chosen
	at random. Next a ball is randomly chosen from the urn. Then a second ball is
	chosen at random from the same urn without replacing the first ball.
	
	\begin{enumerate}
	\item What is the probability that two black balls are chosen?
	
	\item What is the probability that two balls of opposite colour are chosen?
	\end{enumerate}
	\solution
	%\input{exemplar/11/16/3/12/main1.tex}
\end{enumerate}

	\item A card is selected from a pack of 52 cards.
 \begin{enumerate}[label=(\alph*)] 
                 \item How many points are there in the sample space?
                 \item Calculate the probability that the card is an ace of spades.
                 \item Calculate the probability that the card is (i) an ace and (ii) black card.
 \end{enumerate}
\solution
		%\begin{table}[H]
	\centering
\begin{tabular}{|c|c|c|}
\hline
Random variable &Value &Definition\\ \hline
\multirow{3}{*}{X} &0 &Slips of Rs 1\\
&1 &Slips of Rs 5\\
&2 &Slips of Rs 13\\ \hline
\multirow{2}{*}{Y} &0 &Box A\\
&1 &Box B\\\hline
\end{tabular}
\caption{}
\label{tab:Distribution}
\end{table}
See \tabref{tab:Distribution}.
\begin{align}
p_{Y}\brak{k}= \begin{cases} 
      \frac{1}{3} & {k=0} \\
      \frac{2}{3 }& {k=1} 
   \end{cases}
   \\
p_{Y|X}\brak{0|0} = \frac{19}{25}\, 
p_{Y|X}\brak{0|1} = \frac{6}{25}\,
p_{Y|X}\brak{1|0} = \frac{45}{50}\,
p_{Y|X}\brak{1|2} = \frac{5}{50}
\end{align}
The desired probability is the probability that a slip drawn at random is marked other than Rs 1,
\begin{align}
&=1-p_X\brak{0}\\
&= p_X(1) + p_X(2)
\end{align}
Using Bayes theorem,
\begin{align}
&= p_Y\brak{0} \times \pr{Y=0 | X=1} + p_Y\brak{1} \times \pr{Y=1|X=2}\\
&=\frac{1}{3} \times \frac{6}{25} + \frac{2}{3} \times \frac{5}{50}\\
&=\frac{11}{75}
\end{align}

\newpage

%\tableofcontents

\bigskip

\renewcommand{\thefigure}{\theenumi}
\renewcommand{\thetable}{\theenumi}
%\renewcommand{\theequation}{\theenumi}

%\begin{abstract}
%%\boldmath
%In this letter, an algorithm for evaluating the exact analytical bit error rate  (BER)  for the piecewise linear (PL) combiner for  multiple relays is presented. Previous results were available only for upto three relays. The algorithm is unique in the sense that  the actual mathematical expressions, that are prohibitively large, need not be explicitly obtained. The diversity gain due to multiple relays is shown through plots of the analytical BER, well supported by simulations. 
%
%\end{abstract}
% IEEEtran.cls defaults to using nonbold math in the Abstract.
% This preserves the distinction between vectors and scalars. However,
% if the journal you are submitting to favors bold math in the abstract,
% then you can use LaTeX's standard command \boldmath at the very start
% of the abstract to achieve this. Many IEEE journals frown on math
% in the abstract anyway.

% Note that keywords are not normally used for peerreview papers.
%\begin{IEEEkeywords}
%Cooperative diversity, decode and forward, piecewise linear
%\end{IEEEkeywords}



% For peer review papers, you can put extra information on the cover
% page as needed:
% \ifCLASSOPTIONpeerreview
% \begin{center} \bfseries EDICS Category: 3-BBND \end{center}
% \fi
%
% For peerreview papers, this IEEEtran command inserts a page break and
% creates the second title. It will be ignored for other modes.
%\IEEEpeerreviewmaketitle




\item Four cards are drawn from a well-shuffled deck of 52 cards. What is the probability of obtaining 3 diamonds and one spade.
\\
\solution
		%\begin{enumerate}[label=\thesection.\arabic*,ref=\thesection.\theenumi]
	\item One card is drawn from a well-shuffled deck of 52 cards. Find the probability of getting
\begin{enumerate}
\item A king of red colour 
\item A face card 
\item A red face card
\item The jack of hearts
\item A spade
\item The queen of diamonds

\end{enumerate}
\solution
		%\input{ncert/10/15/1/14/main.tex}
	\item Five cards—the ten, jack, queen, king and ace of diamonds, are well-shuffled with their face downwards. One card is then picked up at random.
\begin{enumerate}
\item
What is the probability that the card is the queen? 
\item
If the queen is drawn and put aside, what is the probability that the second card picked up is (a) an ace? (b) a queen?\\
\end{enumerate}
\solution
		%\input{ncert/10/15/1/15/defs.tex}
	\item A bag contains $5$ red balls and some blue balls. If the probability of drawing a blue ball is double that if a red ball, determine the number of blue balls in the bag. 
		\\
\solution
		%\input{ncert/10/15/2/3/defs.tex}
	\item A card is selected from a pack of 52 cards.
 \begin{enumerate}[label=(\alph*)] 
                 \item How many points are there in the sample space?
                 \item Calculate the probability that the card is an ace of spades.
                 \item Calculate the probability that the card is (i) an ace and (ii) black card.
 \end{enumerate}
\solution
		%\input{ncert/11/16/3/4/main.tex}
\item Four cards are drawn from a well-shuffled deck of 52 cards. What is the probability of obtaining 3 diamonds and one spade.
\\
\solution
		%\input{ncert/11/16/4/2/defs.tex}
\item In a certain lottery 10,000 tickets are sold and ten equal prizes are awarded. What is the probability of not getting a prize if you buy (a) one ticket (b) two tickets (c) 10 tickets ?	
\\
\solution
		%\input{ncert/11/16/4/4/defs.tex}
		%
\item 
Out of 100 students, two sections of 40 and 60 are formed. If you and your friend are among the 100 students, what is the probability that
\begin{enumerate}
\item you both enter the same section?
\item you both enter the different sections?
\end{enumerate}
\solution
		%\input{ncert/11/16/4/5/defs.tex}
	\item 
The number lock of a suitcase has 4 wheels each labelled with ten digits i.e. from 0 to 9.The lock opens with a sequence of four digits with no repeats.What is the probability of a person getting the right sequence to open the suitcase.
\\
\solution
		%\input{ncert/11/16/4/10/defs.tex}
		%
\item 
Two cards are drawn at random and without replacement from a pack of 52 playing cards. Find the probability that both the cards are black.
\\
\solution
		%\input{ncert/12/13/2/2/defs.tex}
		\item A box of oranges is inspected by examining three randomly selected oranges drawn without replacement. If all the three oranges are good, the box is approved for sale, otherwise, it is rejected. Find the probability that a box containing 15 oranges out of which 12 are good and 3 are bad ones will be approved for sale.
		\label{ncert/12/13/2/3/defs.tex}
		\item Two balls are drawn at random with replacement from a box containing 10 black and 8 red balls. Find the probability that
		\label{ncert/12/13/2/12}
\begin{enumerate}
\item both balls are red.
\item first ball is black and second is red.
\item one of them is black and other is red.
\end{enumerate}

\item In a hostel, 60\% of the students read Hindi newspaper, 40\% read English newspaper and 20\% read both Hindi and English newspapers. A student is selected at random.
		\label{ncert/12/13/2/15}
\begin{enumerate}
\item Find the probability that she reads neither Hindi nor English newspapers.
\item If she reads Hindi newspaper, find the probability that she reads English newspaper.
\item If she reads English newspaper, find the probability that she reads Hindi newspaper.\\
\end{enumerate}
\item The probability of obtaining an even prime number on each die, when a pair of dice is rolled is 
\begin{enumerate}
    \item $0$ 
    
    \item $\frac{1}{3}$ 
    
    \item $\frac{1}{12}$ 
    
    \item $\frac{1}{36}$ 
\end{enumerate}
\solution
		%\input{ncert/12/13/2/17/defs.tex}
	\item A bag contains 4 red and 4 black balls, another bag contains 2 red and 6 black balls. One of the two bags is selected at random and a ball is drawn from the bag which is found to be red. Find the probability that the ball is drawn from the first bag.
\\
\solution
		%\input{ncert/12/13/3/2/main.tex}
  \item
  Cards with numbers 2 to 101 are placed in a box. A card is selected at random.Find the probability that the card has
\begin{enumerate}[label=(\roman*)]
	\item an even number 
	\item a square number
\end{enumerate}
\solution
%\input{exemplar/10/13/3/32/main.tex}
\item
The king, queen and jack of clubs are removed from a deck of 52 playing cards and then well shuffled. Now one card is drawn at random from the remaining cards.  Determine the probability that the card is
\begin{enumerate}[label=(\roman*)]
\item a club
\item 10 of hearts
\end{enumerate}
\solution
%\input{exemplar/10/13/3/29/main.tex}
\item A team of medical students doing their internship have to assist during surgeries
at a city hospital. The probabilities of surgeries rated as very complex, complex,
routine, simple or very simple are respectively, 0.15, 0.20, 0.31, 0.26, .08. Find
the probabilities that a particular surgery will be rated
\begin{enumerate}
	\item complex or very complex;
	\item neither very complex nor very simple;
	\item routine or complex
	\item routine or simple
\end{enumerate}
\solution
%\input{exemplar/11/16/3/8(1)/main.tex}
\item A card is selected from a pack of 52 cards.
\begin{enumerate}[label=(\alph*)]
    \item How many points are there in the sample space?
    \item Calculate the probability that the card is an ace of spades.
    \item Calculate the probability that the card is (i) an ace and (ii) black card.
\end{enumerate}
\solution
%\input{exemplar/11/16/3/4/main2.tex}
\item The probability that a non leap year selected at random will contain 53 sundays.
\\
\solution
%\input{exemplar/10/13/1/19/main.tex}
\item One of the four persons John, Rita, Aslam or Gurpreet will be promoted next
month. Consequently the sample space consists of four elementary outcomes
S = {John promoted, Rita promoted, Aslam promoted, Gurpreet promoted}
You are told that the chances of John’s promotion is same as that of Gurpreet,
Rita’s chances of promotion are twice as likely as Johns. Aslam’s chances are
four times that of John.
\begin{enumerate}
	\item Determine
	\begin{enumerate}
		\item P (John promoted)
		\item P (Rita promoted)
		\item P (Aslam promoted)
		\item P (Gurpreet promoted)
	\end{enumerate}
	\item If A = {John promoted or Gurpreet promoted}, find P (A).
\end{enumerate}
\solution
%\input{exemplar/11/16/3/10/main.tex}
\item A card is drawn from a deck of 52 cards. Find the probability of getting a king or a heart or a red card.\\
\solution
%\input{exemplar/11/16/3/15/main.tex}
\item The probability that a student will pass his examination is 0.73, the probability of
the student getting a compartment is 0.13, and the probability that the student will
either pass or get compartment is 0.96. State True or False.\\
\solution
%\input{exemplar/11/16/3/31/main.tex}
\item A card is selected from a pack of 52 cards\\
\begin{enumerate}[label=(\alph*)]
\item How many points are there in the sample space?
\item Calculate the probability that the cards is an ace of spades.
\item Calculate the probability that the card is (i) an ace (ii)black card.\\
\end{enumerate}
%\input{ncert/11/16/3/4_1/Prob_4.tex}
\item In a non-leap year, the probability of having 53 tuesdays or 53 wednesdays is\\
\solution
%\input{exemplar/11/16/3/18/main.tex}
\item There are 1000 sealed envelopes in a box, 10 of them contain a cash prize of
Rs 100 each, 100 of them contain a cash prize of Rs 50 each and 200 of them
contain a cash prize of Rs 10 each and rest do not contain any cash prize. If they
are well shuffled and an envelope is picked up out, what is the probability that it
contains no cash prize?\\
\solution
%\input{exemplar/10/13/3/34/main.tex}
\item 
A die is thrown and a card is selected at random from a deck of 52 playing cards. The probability of getting an even number on the die and a spade card.\\
\solution
%\input{exemplar/12/13/3/78/main.tex}
\item
If 4-digit numbers greater than 5,000 are randomly formed from the digits 0, 1, 3, 5, and 7, what is the probability of forming a number divisible by 5 when:
\begin{enumerate}
    \item The digits are repeated?
    \item The repetition of digits is not allowed?
\end{enumerate}
\solution
%\input{ncert/11/16/4/9/main.tex}
\item Consider the probability space $\brak{\Omega, \mathcal{G}, P}$ where $\Omega = [0,2]$ and $\mathcal{G} = \cbrak{\phi, \Omega, [0,1], (1,2]}$. Let $X$ and $Y$ be two functions on $\Omega$ defined as
\begin{align*}
    X(\omega) = 
    \begin{cases}
        1 & \text{if }\omega \in [0, 1]\\
        2 & \text{if }\omega \in (1, 2]
    \end{cases}
\end{align*}
and
\begin{align*}
    Y(\omega) = 
    \begin{cases}
        2 & \text{if }\omega \in [0, 1.5]\\
        3 & \text{if }\omega \in (1.5, 2].
    \end{cases}
\end{align*}
Then which one of the following statements is true?
\begin{enumerate}
    \item [(A)] $X$ is a random variable with respect to $\mathcal{G}$, but $Y$ is not a random variable with respect to $\mathcal{G}$.
    \item [(B)] $Y$ is a random variable with respect to $\mathcal{G}$, but $X$ is not a random variable with respect to $\mathcal{G}$.
    \item [(C)] Neither $X$ nor $Y$ is a random variable with respect to $\mathcal{G}$.
    \item [(D)] Both $X$ and $Y$ are random variables with respect to $\mathcal{G}$.
\end{enumerate} \hfill (GATE ST 2023)\\
\solution
%\input{gate/ST/2023/14/main.tex}
	\item  A die is loaded in such a way that each odd number is twice as likely to occur as
each even number. Find $P(G)$, where $G$ is the event that a number greater than
3 occurs on a single roll of the die.
\\
\solution
		%\input{exemplar/11/16/3/5/main.tex}
	\item All the jacks, queens and kings are removed from a deck of 52 playing cards. The remaining cards are well shuffled and then one card is drawn at random. Giving ace a value 1 similar value for other cards, find the probability that the card has a value 
		\begin{enumerate}
			\item 7
			\item greater than 7
			\item less than 7
		\end{enumerate}
		%\input{exemplar/10/13/3/30/main.tex}
  \item A Lot consists of 48 mobile phones of which 42 are good, 3 have only minor defects and 3 have major defects.Varnika will buy a phone if it is good but the trader will only buy a mobile if it has no major defects. One phone is selected at random from the lot. What is the probability that it is
\begin{enumerate}
	\item acceptable to Varnika?
            \item acceptable to the trader?
\end{enumerate}
\solution
	%\input{exemplar/10/13/3/40/main.tex}
 \item A student says that if you throw a die, it will show up 1 or not 1. Therefore, the probability of getting 1 and the probability of getting 'not 1' each is equal to $\frac{1}{2}$. Is this correct? Give reasons.\\
 \solution
        %\input{exemplar/10/13/2/9/main.tex}
   \item Four candidates A, B, C, D have ap-
plied for the assignment to coach a school cricket
team. If A is twice as likely to be selected as B, and
B and C are given about the same chance of being
selected, while C is twice as likely to be selected
as D, what are the probabilities that
\begin{enumerate}
\item C will be selected?
\item A will not be selected?
\end{enumerate}
	%\input{exemplar/11/16/3/9/main.tex}
 \item A bag contain 24 balls of which $x$ balls are red, $2x$ are white and $3x$ are blue. A ball is selected at random, What is the probability that it is
\begin{enumerate}[label=\alph*)]
\item not red ?
\item white ?
\end{enumerate}
%\input{exemplar/10/13/3/41/main.tex}
If the letters of the word ASSASSINATION are arranged at random. Find the Probability that
\begin{enumerate}[label=(\alph*)]
\item Four $S's$ come consecutively in the word
\item Two  $I's$ and two $N's$ come together
\item All $A's$ are not coming together
\item No two $A's$ are coming together
\end{enumerate}
%\input{exemplar/11/16/3/14/main.tex}
	\item One urn contains two black balls (labelled B1 and B2) and one white ball. A
	second urn contains one black ball and two white balls (labelled W1 and W2).
	Suppose the following experiment is performed. One of the two urns is chosen
	at random. Next a ball is randomly chosen from the urn. Then a second ball is
	chosen at random from the same urn without replacing the first ball.
	
	\begin{enumerate}
	\item What is the probability that two black balls are chosen?
	
	\item What is the probability that two balls of opposite colour are chosen?
	\end{enumerate}
	\solution
	%\input{exemplar/11/16/3/12/main1.tex}
\end{enumerate}

\item In a certain lottery 10,000 tickets are sold and ten equal prizes are awarded. What is the probability of not getting a prize if you buy (a) one ticket (b) two tickets (c) 10 tickets ?	
\\
\solution
		%\begin{enumerate}[label=\thesection.\arabic*,ref=\thesection.\theenumi]
	\item One card is drawn from a well-shuffled deck of 52 cards. Find the probability of getting
\begin{enumerate}
\item A king of red colour 
\item A face card 
\item A red face card
\item The jack of hearts
\item A spade
\item The queen of diamonds

\end{enumerate}
\solution
		%\input{ncert/10/15/1/14/main.tex}
	\item Five cards—the ten, jack, queen, king and ace of diamonds, are well-shuffled with their face downwards. One card is then picked up at random.
\begin{enumerate}
\item
What is the probability that the card is the queen? 
\item
If the queen is drawn and put aside, what is the probability that the second card picked up is (a) an ace? (b) a queen?\\
\end{enumerate}
\solution
		%\input{ncert/10/15/1/15/defs.tex}
	\item A bag contains $5$ red balls and some blue balls. If the probability of drawing a blue ball is double that if a red ball, determine the number of blue balls in the bag. 
		\\
\solution
		%\input{ncert/10/15/2/3/defs.tex}
	\item A card is selected from a pack of 52 cards.
 \begin{enumerate}[label=(\alph*)] 
                 \item How many points are there in the sample space?
                 \item Calculate the probability that the card is an ace of spades.
                 \item Calculate the probability that the card is (i) an ace and (ii) black card.
 \end{enumerate}
\solution
		%\input{ncert/11/16/3/4/main.tex}
\item Four cards are drawn from a well-shuffled deck of 52 cards. What is the probability of obtaining 3 diamonds and one spade.
\\
\solution
		%\input{ncert/11/16/4/2/defs.tex}
\item In a certain lottery 10,000 tickets are sold and ten equal prizes are awarded. What is the probability of not getting a prize if you buy (a) one ticket (b) two tickets (c) 10 tickets ?	
\\
\solution
		%\input{ncert/11/16/4/4/defs.tex}
		%
\item 
Out of 100 students, two sections of 40 and 60 are formed. If you and your friend are among the 100 students, what is the probability that
\begin{enumerate}
\item you both enter the same section?
\item you both enter the different sections?
\end{enumerate}
\solution
		%\input{ncert/11/16/4/5/defs.tex}
	\item 
The number lock of a suitcase has 4 wheels each labelled with ten digits i.e. from 0 to 9.The lock opens with a sequence of four digits with no repeats.What is the probability of a person getting the right sequence to open the suitcase.
\\
\solution
		%\input{ncert/11/16/4/10/defs.tex}
		%
\item 
Two cards are drawn at random and without replacement from a pack of 52 playing cards. Find the probability that both the cards are black.
\\
\solution
		%\input{ncert/12/13/2/2/defs.tex}
		\item A box of oranges is inspected by examining three randomly selected oranges drawn without replacement. If all the three oranges are good, the box is approved for sale, otherwise, it is rejected. Find the probability that a box containing 15 oranges out of which 12 are good and 3 are bad ones will be approved for sale.
		\label{ncert/12/13/2/3/defs.tex}
		\item Two balls are drawn at random with replacement from a box containing 10 black and 8 red balls. Find the probability that
		\label{ncert/12/13/2/12}
\begin{enumerate}
\item both balls are red.
\item first ball is black and second is red.
\item one of them is black and other is red.
\end{enumerate}

\item In a hostel, 60\% of the students read Hindi newspaper, 40\% read English newspaper and 20\% read both Hindi and English newspapers. A student is selected at random.
		\label{ncert/12/13/2/15}
\begin{enumerate}
\item Find the probability that she reads neither Hindi nor English newspapers.
\item If she reads Hindi newspaper, find the probability that she reads English newspaper.
\item If she reads English newspaper, find the probability that she reads Hindi newspaper.\\
\end{enumerate}
\item The probability of obtaining an even prime number on each die, when a pair of dice is rolled is 
\begin{enumerate}
    \item $0$ 
    
    \item $\frac{1}{3}$ 
    
    \item $\frac{1}{12}$ 
    
    \item $\frac{1}{36}$ 
\end{enumerate}
\solution
		%\input{ncert/12/13/2/17/defs.tex}
	\item A bag contains 4 red and 4 black balls, another bag contains 2 red and 6 black balls. One of the two bags is selected at random and a ball is drawn from the bag which is found to be red. Find the probability that the ball is drawn from the first bag.
\\
\solution
		%\input{ncert/12/13/3/2/main.tex}
  \item
  Cards with numbers 2 to 101 are placed in a box. A card is selected at random.Find the probability that the card has
\begin{enumerate}[label=(\roman*)]
	\item an even number 
	\item a square number
\end{enumerate}
\solution
%\input{exemplar/10/13/3/32/main.tex}
\item
The king, queen and jack of clubs are removed from a deck of 52 playing cards and then well shuffled. Now one card is drawn at random from the remaining cards.  Determine the probability that the card is
\begin{enumerate}[label=(\roman*)]
\item a club
\item 10 of hearts
\end{enumerate}
\solution
%\input{exemplar/10/13/3/29/main.tex}
\item A team of medical students doing their internship have to assist during surgeries
at a city hospital. The probabilities of surgeries rated as very complex, complex,
routine, simple or very simple are respectively, 0.15, 0.20, 0.31, 0.26, .08. Find
the probabilities that a particular surgery will be rated
\begin{enumerate}
	\item complex or very complex;
	\item neither very complex nor very simple;
	\item routine or complex
	\item routine or simple
\end{enumerate}
\solution
%\input{exemplar/11/16/3/8(1)/main.tex}
\item A card is selected from a pack of 52 cards.
\begin{enumerate}[label=(\alph*)]
    \item How many points are there in the sample space?
    \item Calculate the probability that the card is an ace of spades.
    \item Calculate the probability that the card is (i) an ace and (ii) black card.
\end{enumerate}
\solution
%\input{exemplar/11/16/3/4/main2.tex}
\item The probability that a non leap year selected at random will contain 53 sundays.
\\
\solution
%\input{exemplar/10/13/1/19/main.tex}
\item One of the four persons John, Rita, Aslam or Gurpreet will be promoted next
month. Consequently the sample space consists of four elementary outcomes
S = {John promoted, Rita promoted, Aslam promoted, Gurpreet promoted}
You are told that the chances of John’s promotion is same as that of Gurpreet,
Rita’s chances of promotion are twice as likely as Johns. Aslam’s chances are
four times that of John.
\begin{enumerate}
	\item Determine
	\begin{enumerate}
		\item P (John promoted)
		\item P (Rita promoted)
		\item P (Aslam promoted)
		\item P (Gurpreet promoted)
	\end{enumerate}
	\item If A = {John promoted or Gurpreet promoted}, find P (A).
\end{enumerate}
\solution
%\input{exemplar/11/16/3/10/main.tex}
\item A card is drawn from a deck of 52 cards. Find the probability of getting a king or a heart or a red card.\\
\solution
%\input{exemplar/11/16/3/15/main.tex}
\item The probability that a student will pass his examination is 0.73, the probability of
the student getting a compartment is 0.13, and the probability that the student will
either pass or get compartment is 0.96. State True or False.\\
\solution
%\input{exemplar/11/16/3/31/main.tex}
\item A card is selected from a pack of 52 cards\\
\begin{enumerate}[label=(\alph*)]
\item How many points are there in the sample space?
\item Calculate the probability that the cards is an ace of spades.
\item Calculate the probability that the card is (i) an ace (ii)black card.\\
\end{enumerate}
%\input{ncert/11/16/3/4_1/Prob_4.tex}
\item In a non-leap year, the probability of having 53 tuesdays or 53 wednesdays is\\
\solution
%\input{exemplar/11/16/3/18/main.tex}
\item There are 1000 sealed envelopes in a box, 10 of them contain a cash prize of
Rs 100 each, 100 of them contain a cash prize of Rs 50 each and 200 of them
contain a cash prize of Rs 10 each and rest do not contain any cash prize. If they
are well shuffled and an envelope is picked up out, what is the probability that it
contains no cash prize?\\
\solution
%\input{exemplar/10/13/3/34/main.tex}
\item 
A die is thrown and a card is selected at random from a deck of 52 playing cards. The probability of getting an even number on the die and a spade card.\\
\solution
%\input{exemplar/12/13/3/78/main.tex}
\item
If 4-digit numbers greater than 5,000 are randomly formed from the digits 0, 1, 3, 5, and 7, what is the probability of forming a number divisible by 5 when:
\begin{enumerate}
    \item The digits are repeated?
    \item The repetition of digits is not allowed?
\end{enumerate}
\solution
%\input{ncert/11/16/4/9/main.tex}
\item Consider the probability space $\brak{\Omega, \mathcal{G}, P}$ where $\Omega = [0,2]$ and $\mathcal{G} = \cbrak{\phi, \Omega, [0,1], (1,2]}$. Let $X$ and $Y$ be two functions on $\Omega$ defined as
\begin{align*}
    X(\omega) = 
    \begin{cases}
        1 & \text{if }\omega \in [0, 1]\\
        2 & \text{if }\omega \in (1, 2]
    \end{cases}
\end{align*}
and
\begin{align*}
    Y(\omega) = 
    \begin{cases}
        2 & \text{if }\omega \in [0, 1.5]\\
        3 & \text{if }\omega \in (1.5, 2].
    \end{cases}
\end{align*}
Then which one of the following statements is true?
\begin{enumerate}
    \item [(A)] $X$ is a random variable with respect to $\mathcal{G}$, but $Y$ is not a random variable with respect to $\mathcal{G}$.
    \item [(B)] $Y$ is a random variable with respect to $\mathcal{G}$, but $X$ is not a random variable with respect to $\mathcal{G}$.
    \item [(C)] Neither $X$ nor $Y$ is a random variable with respect to $\mathcal{G}$.
    \item [(D)] Both $X$ and $Y$ are random variables with respect to $\mathcal{G}$.
\end{enumerate} \hfill (GATE ST 2023)\\
\solution
%\input{gate/ST/2023/14/main.tex}
	\item  A die is loaded in such a way that each odd number is twice as likely to occur as
each even number. Find $P(G)$, where $G$ is the event that a number greater than
3 occurs on a single roll of the die.
\\
\solution
		%\input{exemplar/11/16/3/5/main.tex}
	\item All the jacks, queens and kings are removed from a deck of 52 playing cards. The remaining cards are well shuffled and then one card is drawn at random. Giving ace a value 1 similar value for other cards, find the probability that the card has a value 
		\begin{enumerate}
			\item 7
			\item greater than 7
			\item less than 7
		\end{enumerate}
		%\input{exemplar/10/13/3/30/main.tex}
  \item A Lot consists of 48 mobile phones of which 42 are good, 3 have only minor defects and 3 have major defects.Varnika will buy a phone if it is good but the trader will only buy a mobile if it has no major defects. One phone is selected at random from the lot. What is the probability that it is
\begin{enumerate}
	\item acceptable to Varnika?
            \item acceptable to the trader?
\end{enumerate}
\solution
	%\input{exemplar/10/13/3/40/main.tex}
 \item A student says that if you throw a die, it will show up 1 or not 1. Therefore, the probability of getting 1 and the probability of getting 'not 1' each is equal to $\frac{1}{2}$. Is this correct? Give reasons.\\
 \solution
        %\input{exemplar/10/13/2/9/main.tex}
   \item Four candidates A, B, C, D have ap-
plied for the assignment to coach a school cricket
team. If A is twice as likely to be selected as B, and
B and C are given about the same chance of being
selected, while C is twice as likely to be selected
as D, what are the probabilities that
\begin{enumerate}
\item C will be selected?
\item A will not be selected?
\end{enumerate}
	%\input{exemplar/11/16/3/9/main.tex}
 \item A bag contain 24 balls of which $x$ balls are red, $2x$ are white and $3x$ are blue. A ball is selected at random, What is the probability that it is
\begin{enumerate}[label=\alph*)]
\item not red ?
\item white ?
\end{enumerate}
%\input{exemplar/10/13/3/41/main.tex}
If the letters of the word ASSASSINATION are arranged at random. Find the Probability that
\begin{enumerate}[label=(\alph*)]
\item Four $S's$ come consecutively in the word
\item Two  $I's$ and two $N's$ come together
\item All $A's$ are not coming together
\item No two $A's$ are coming together
\end{enumerate}
%\input{exemplar/11/16/3/14/main.tex}
	\item One urn contains two black balls (labelled B1 and B2) and one white ball. A
	second urn contains one black ball and two white balls (labelled W1 and W2).
	Suppose the following experiment is performed. One of the two urns is chosen
	at random. Next a ball is randomly chosen from the urn. Then a second ball is
	chosen at random from the same urn without replacing the first ball.
	
	\begin{enumerate}
	\item What is the probability that two black balls are chosen?
	
	\item What is the probability that two balls of opposite colour are chosen?
	\end{enumerate}
	\solution
	%\input{exemplar/11/16/3/12/main1.tex}
\end{enumerate}

		%
\item 
Out of 100 students, two sections of 40 and 60 are formed. If you and your friend are among the 100 students, what is the probability that
\begin{enumerate}
\item you both enter the same section?
\item you both enter the different sections?
\end{enumerate}
\solution
		%\begin{enumerate}[label=\thesection.\arabic*,ref=\thesection.\theenumi]
	\item One card is drawn from a well-shuffled deck of 52 cards. Find the probability of getting
\begin{enumerate}
\item A king of red colour 
\item A face card 
\item A red face card
\item The jack of hearts
\item A spade
\item The queen of diamonds

\end{enumerate}
\solution
		%\input{ncert/10/15/1/14/main.tex}
	\item Five cards—the ten, jack, queen, king and ace of diamonds, are well-shuffled with their face downwards. One card is then picked up at random.
\begin{enumerate}
\item
What is the probability that the card is the queen? 
\item
If the queen is drawn and put aside, what is the probability that the second card picked up is (a) an ace? (b) a queen?\\
\end{enumerate}
\solution
		%\input{ncert/10/15/1/15/defs.tex}
	\item A bag contains $5$ red balls and some blue balls. If the probability of drawing a blue ball is double that if a red ball, determine the number of blue balls in the bag. 
		\\
\solution
		%\input{ncert/10/15/2/3/defs.tex}
	\item A card is selected from a pack of 52 cards.
 \begin{enumerate}[label=(\alph*)] 
                 \item How many points are there in the sample space?
                 \item Calculate the probability that the card is an ace of spades.
                 \item Calculate the probability that the card is (i) an ace and (ii) black card.
 \end{enumerate}
\solution
		%\input{ncert/11/16/3/4/main.tex}
\item Four cards are drawn from a well-shuffled deck of 52 cards. What is the probability of obtaining 3 diamonds and one spade.
\\
\solution
		%\input{ncert/11/16/4/2/defs.tex}
\item In a certain lottery 10,000 tickets are sold and ten equal prizes are awarded. What is the probability of not getting a prize if you buy (a) one ticket (b) two tickets (c) 10 tickets ?	
\\
\solution
		%\input{ncert/11/16/4/4/defs.tex}
		%
\item 
Out of 100 students, two sections of 40 and 60 are formed. If you and your friend are among the 100 students, what is the probability that
\begin{enumerate}
\item you both enter the same section?
\item you both enter the different sections?
\end{enumerate}
\solution
		%\input{ncert/11/16/4/5/defs.tex}
	\item 
The number lock of a suitcase has 4 wheels each labelled with ten digits i.e. from 0 to 9.The lock opens with a sequence of four digits with no repeats.What is the probability of a person getting the right sequence to open the suitcase.
\\
\solution
		%\input{ncert/11/16/4/10/defs.tex}
		%
\item 
Two cards are drawn at random and without replacement from a pack of 52 playing cards. Find the probability that both the cards are black.
\\
\solution
		%\input{ncert/12/13/2/2/defs.tex}
		\item A box of oranges is inspected by examining three randomly selected oranges drawn without replacement. If all the three oranges are good, the box is approved for sale, otherwise, it is rejected. Find the probability that a box containing 15 oranges out of which 12 are good and 3 are bad ones will be approved for sale.
		\label{ncert/12/13/2/3/defs.tex}
		\item Two balls are drawn at random with replacement from a box containing 10 black and 8 red balls. Find the probability that
		\label{ncert/12/13/2/12}
\begin{enumerate}
\item both balls are red.
\item first ball is black and second is red.
\item one of them is black and other is red.
\end{enumerate}

\item In a hostel, 60\% of the students read Hindi newspaper, 40\% read English newspaper and 20\% read both Hindi and English newspapers. A student is selected at random.
		\label{ncert/12/13/2/15}
\begin{enumerate}
\item Find the probability that she reads neither Hindi nor English newspapers.
\item If she reads Hindi newspaper, find the probability that she reads English newspaper.
\item If she reads English newspaper, find the probability that she reads Hindi newspaper.\\
\end{enumerate}
\item The probability of obtaining an even prime number on each die, when a pair of dice is rolled is 
\begin{enumerate}
    \item $0$ 
    
    \item $\frac{1}{3}$ 
    
    \item $\frac{1}{12}$ 
    
    \item $\frac{1}{36}$ 
\end{enumerate}
\solution
		%\input{ncert/12/13/2/17/defs.tex}
	\item A bag contains 4 red and 4 black balls, another bag contains 2 red and 6 black balls. One of the two bags is selected at random and a ball is drawn from the bag which is found to be red. Find the probability that the ball is drawn from the first bag.
\\
\solution
		%\input{ncert/12/13/3/2/main.tex}
  \item
  Cards with numbers 2 to 101 are placed in a box. A card is selected at random.Find the probability that the card has
\begin{enumerate}[label=(\roman*)]
	\item an even number 
	\item a square number
\end{enumerate}
\solution
%\input{exemplar/10/13/3/32/main.tex}
\item
The king, queen and jack of clubs are removed from a deck of 52 playing cards and then well shuffled. Now one card is drawn at random from the remaining cards.  Determine the probability that the card is
\begin{enumerate}[label=(\roman*)]
\item a club
\item 10 of hearts
\end{enumerate}
\solution
%\input{exemplar/10/13/3/29/main.tex}
\item A team of medical students doing their internship have to assist during surgeries
at a city hospital. The probabilities of surgeries rated as very complex, complex,
routine, simple or very simple are respectively, 0.15, 0.20, 0.31, 0.26, .08. Find
the probabilities that a particular surgery will be rated
\begin{enumerate}
	\item complex or very complex;
	\item neither very complex nor very simple;
	\item routine or complex
	\item routine or simple
\end{enumerate}
\solution
%\input{exemplar/11/16/3/8(1)/main.tex}
\item A card is selected from a pack of 52 cards.
\begin{enumerate}[label=(\alph*)]
    \item How many points are there in the sample space?
    \item Calculate the probability that the card is an ace of spades.
    \item Calculate the probability that the card is (i) an ace and (ii) black card.
\end{enumerate}
\solution
%\input{exemplar/11/16/3/4/main2.tex}
\item The probability that a non leap year selected at random will contain 53 sundays.
\\
\solution
%\input{exemplar/10/13/1/19/main.tex}
\item One of the four persons John, Rita, Aslam or Gurpreet will be promoted next
month. Consequently the sample space consists of four elementary outcomes
S = {John promoted, Rita promoted, Aslam promoted, Gurpreet promoted}
You are told that the chances of John’s promotion is same as that of Gurpreet,
Rita’s chances of promotion are twice as likely as Johns. Aslam’s chances are
four times that of John.
\begin{enumerate}
	\item Determine
	\begin{enumerate}
		\item P (John promoted)
		\item P (Rita promoted)
		\item P (Aslam promoted)
		\item P (Gurpreet promoted)
	\end{enumerate}
	\item If A = {John promoted or Gurpreet promoted}, find P (A).
\end{enumerate}
\solution
%\input{exemplar/11/16/3/10/main.tex}
\item A card is drawn from a deck of 52 cards. Find the probability of getting a king or a heart or a red card.\\
\solution
%\input{exemplar/11/16/3/15/main.tex}
\item The probability that a student will pass his examination is 0.73, the probability of
the student getting a compartment is 0.13, and the probability that the student will
either pass or get compartment is 0.96. State True or False.\\
\solution
%\input{exemplar/11/16/3/31/main.tex}
\item A card is selected from a pack of 52 cards\\
\begin{enumerate}[label=(\alph*)]
\item How many points are there in the sample space?
\item Calculate the probability that the cards is an ace of spades.
\item Calculate the probability that the card is (i) an ace (ii)black card.\\
\end{enumerate}
%\input{ncert/11/16/3/4_1/Prob_4.tex}
\item In a non-leap year, the probability of having 53 tuesdays or 53 wednesdays is\\
\solution
%\input{exemplar/11/16/3/18/main.tex}
\item There are 1000 sealed envelopes in a box, 10 of them contain a cash prize of
Rs 100 each, 100 of them contain a cash prize of Rs 50 each and 200 of them
contain a cash prize of Rs 10 each and rest do not contain any cash prize. If they
are well shuffled and an envelope is picked up out, what is the probability that it
contains no cash prize?\\
\solution
%\input{exemplar/10/13/3/34/main.tex}
\item 
A die is thrown and a card is selected at random from a deck of 52 playing cards. The probability of getting an even number on the die and a spade card.\\
\solution
%\input{exemplar/12/13/3/78/main.tex}
\item
If 4-digit numbers greater than 5,000 are randomly formed from the digits 0, 1, 3, 5, and 7, what is the probability of forming a number divisible by 5 when:
\begin{enumerate}
    \item The digits are repeated?
    \item The repetition of digits is not allowed?
\end{enumerate}
\solution
%\input{ncert/11/16/4/9/main.tex}
\item Consider the probability space $\brak{\Omega, \mathcal{G}, P}$ where $\Omega = [0,2]$ and $\mathcal{G} = \cbrak{\phi, \Omega, [0,1], (1,2]}$. Let $X$ and $Y$ be two functions on $\Omega$ defined as
\begin{align*}
    X(\omega) = 
    \begin{cases}
        1 & \text{if }\omega \in [0, 1]\\
        2 & \text{if }\omega \in (1, 2]
    \end{cases}
\end{align*}
and
\begin{align*}
    Y(\omega) = 
    \begin{cases}
        2 & \text{if }\omega \in [0, 1.5]\\
        3 & \text{if }\omega \in (1.5, 2].
    \end{cases}
\end{align*}
Then which one of the following statements is true?
\begin{enumerate}
    \item [(A)] $X$ is a random variable with respect to $\mathcal{G}$, but $Y$ is not a random variable with respect to $\mathcal{G}$.
    \item [(B)] $Y$ is a random variable with respect to $\mathcal{G}$, but $X$ is not a random variable with respect to $\mathcal{G}$.
    \item [(C)] Neither $X$ nor $Y$ is a random variable with respect to $\mathcal{G}$.
    \item [(D)] Both $X$ and $Y$ are random variables with respect to $\mathcal{G}$.
\end{enumerate} \hfill (GATE ST 2023)\\
\solution
%\input{gate/ST/2023/14/main.tex}
	\item  A die is loaded in such a way that each odd number is twice as likely to occur as
each even number. Find $P(G)$, where $G$ is the event that a number greater than
3 occurs on a single roll of the die.
\\
\solution
		%\input{exemplar/11/16/3/5/main.tex}
	\item All the jacks, queens and kings are removed from a deck of 52 playing cards. The remaining cards are well shuffled and then one card is drawn at random. Giving ace a value 1 similar value for other cards, find the probability that the card has a value 
		\begin{enumerate}
			\item 7
			\item greater than 7
			\item less than 7
		\end{enumerate}
		%\input{exemplar/10/13/3/30/main.tex}
  \item A Lot consists of 48 mobile phones of which 42 are good, 3 have only minor defects and 3 have major defects.Varnika will buy a phone if it is good but the trader will only buy a mobile if it has no major defects. One phone is selected at random from the lot. What is the probability that it is
\begin{enumerate}
	\item acceptable to Varnika?
            \item acceptable to the trader?
\end{enumerate}
\solution
	%\input{exemplar/10/13/3/40/main.tex}
 \item A student says that if you throw a die, it will show up 1 or not 1. Therefore, the probability of getting 1 and the probability of getting 'not 1' each is equal to $\frac{1}{2}$. Is this correct? Give reasons.\\
 \solution
        %\input{exemplar/10/13/2/9/main.tex}
   \item Four candidates A, B, C, D have ap-
plied for the assignment to coach a school cricket
team. If A is twice as likely to be selected as B, and
B and C are given about the same chance of being
selected, while C is twice as likely to be selected
as D, what are the probabilities that
\begin{enumerate}
\item C will be selected?
\item A will not be selected?
\end{enumerate}
	%\input{exemplar/11/16/3/9/main.tex}
 \item A bag contain 24 balls of which $x$ balls are red, $2x$ are white and $3x$ are blue. A ball is selected at random, What is the probability that it is
\begin{enumerate}[label=\alph*)]
\item not red ?
\item white ?
\end{enumerate}
%\input{exemplar/10/13/3/41/main.tex}
If the letters of the word ASSASSINATION are arranged at random. Find the Probability that
\begin{enumerate}[label=(\alph*)]
\item Four $S's$ come consecutively in the word
\item Two  $I's$ and two $N's$ come together
\item All $A's$ are not coming together
\item No two $A's$ are coming together
\end{enumerate}
%\input{exemplar/11/16/3/14/main.tex}
	\item One urn contains two black balls (labelled B1 and B2) and one white ball. A
	second urn contains one black ball and two white balls (labelled W1 and W2).
	Suppose the following experiment is performed. One of the two urns is chosen
	at random. Next a ball is randomly chosen from the urn. Then a second ball is
	chosen at random from the same urn without replacing the first ball.
	
	\begin{enumerate}
	\item What is the probability that two black balls are chosen?
	
	\item What is the probability that two balls of opposite colour are chosen?
	\end{enumerate}
	\solution
	%\input{exemplar/11/16/3/12/main1.tex}
\end{enumerate}

	\item 
The number lock of a suitcase has 4 wheels each labelled with ten digits i.e. from 0 to 9.The lock opens with a sequence of four digits with no repeats.What is the probability of a person getting the right sequence to open the suitcase.
\\
\solution
		%\begin{enumerate}[label=\thesection.\arabic*,ref=\thesection.\theenumi]
	\item One card is drawn from a well-shuffled deck of 52 cards. Find the probability of getting
\begin{enumerate}
\item A king of red colour 
\item A face card 
\item A red face card
\item The jack of hearts
\item A spade
\item The queen of diamonds

\end{enumerate}
\solution
		%\input{ncert/10/15/1/14/main.tex}
	\item Five cards—the ten, jack, queen, king and ace of diamonds, are well-shuffled with their face downwards. One card is then picked up at random.
\begin{enumerate}
\item
What is the probability that the card is the queen? 
\item
If the queen is drawn and put aside, what is the probability that the second card picked up is (a) an ace? (b) a queen?\\
\end{enumerate}
\solution
		%\input{ncert/10/15/1/15/defs.tex}
	\item A bag contains $5$ red balls and some blue balls. If the probability of drawing a blue ball is double that if a red ball, determine the number of blue balls in the bag. 
		\\
\solution
		%\input{ncert/10/15/2/3/defs.tex}
	\item A card is selected from a pack of 52 cards.
 \begin{enumerate}[label=(\alph*)] 
                 \item How many points are there in the sample space?
                 \item Calculate the probability that the card is an ace of spades.
                 \item Calculate the probability that the card is (i) an ace and (ii) black card.
 \end{enumerate}
\solution
		%\input{ncert/11/16/3/4/main.tex}
\item Four cards are drawn from a well-shuffled deck of 52 cards. What is the probability of obtaining 3 diamonds and one spade.
\\
\solution
		%\input{ncert/11/16/4/2/defs.tex}
\item In a certain lottery 10,000 tickets are sold and ten equal prizes are awarded. What is the probability of not getting a prize if you buy (a) one ticket (b) two tickets (c) 10 tickets ?	
\\
\solution
		%\input{ncert/11/16/4/4/defs.tex}
		%
\item 
Out of 100 students, two sections of 40 and 60 are formed. If you and your friend are among the 100 students, what is the probability that
\begin{enumerate}
\item you both enter the same section?
\item you both enter the different sections?
\end{enumerate}
\solution
		%\input{ncert/11/16/4/5/defs.tex}
	\item 
The number lock of a suitcase has 4 wheels each labelled with ten digits i.e. from 0 to 9.The lock opens with a sequence of four digits with no repeats.What is the probability of a person getting the right sequence to open the suitcase.
\\
\solution
		%\input{ncert/11/16/4/10/defs.tex}
		%
\item 
Two cards are drawn at random and without replacement from a pack of 52 playing cards. Find the probability that both the cards are black.
\\
\solution
		%\input{ncert/12/13/2/2/defs.tex}
		\item A box of oranges is inspected by examining three randomly selected oranges drawn without replacement. If all the three oranges are good, the box is approved for sale, otherwise, it is rejected. Find the probability that a box containing 15 oranges out of which 12 are good and 3 are bad ones will be approved for sale.
		\label{ncert/12/13/2/3/defs.tex}
		\item Two balls are drawn at random with replacement from a box containing 10 black and 8 red balls. Find the probability that
		\label{ncert/12/13/2/12}
\begin{enumerate}
\item both balls are red.
\item first ball is black and second is red.
\item one of them is black and other is red.
\end{enumerate}

\item In a hostel, 60\% of the students read Hindi newspaper, 40\% read English newspaper and 20\% read both Hindi and English newspapers. A student is selected at random.
		\label{ncert/12/13/2/15}
\begin{enumerate}
\item Find the probability that she reads neither Hindi nor English newspapers.
\item If she reads Hindi newspaper, find the probability that she reads English newspaper.
\item If she reads English newspaper, find the probability that she reads Hindi newspaper.\\
\end{enumerate}
\item The probability of obtaining an even prime number on each die, when a pair of dice is rolled is 
\begin{enumerate}
    \item $0$ 
    
    \item $\frac{1}{3}$ 
    
    \item $\frac{1}{12}$ 
    
    \item $\frac{1}{36}$ 
\end{enumerate}
\solution
		%\input{ncert/12/13/2/17/defs.tex}
	\item A bag contains 4 red and 4 black balls, another bag contains 2 red and 6 black balls. One of the two bags is selected at random and a ball is drawn from the bag which is found to be red. Find the probability that the ball is drawn from the first bag.
\\
\solution
		%\input{ncert/12/13/3/2/main.tex}
  \item
  Cards with numbers 2 to 101 are placed in a box. A card is selected at random.Find the probability that the card has
\begin{enumerate}[label=(\roman*)]
	\item an even number 
	\item a square number
\end{enumerate}
\solution
%\input{exemplar/10/13/3/32/main.tex}
\item
The king, queen and jack of clubs are removed from a deck of 52 playing cards and then well shuffled. Now one card is drawn at random from the remaining cards.  Determine the probability that the card is
\begin{enumerate}[label=(\roman*)]
\item a club
\item 10 of hearts
\end{enumerate}
\solution
%\input{exemplar/10/13/3/29/main.tex}
\item A team of medical students doing their internship have to assist during surgeries
at a city hospital. The probabilities of surgeries rated as very complex, complex,
routine, simple or very simple are respectively, 0.15, 0.20, 0.31, 0.26, .08. Find
the probabilities that a particular surgery will be rated
\begin{enumerate}
	\item complex or very complex;
	\item neither very complex nor very simple;
	\item routine or complex
	\item routine or simple
\end{enumerate}
\solution
%\input{exemplar/11/16/3/8(1)/main.tex}
\item A card is selected from a pack of 52 cards.
\begin{enumerate}[label=(\alph*)]
    \item How many points are there in the sample space?
    \item Calculate the probability that the card is an ace of spades.
    \item Calculate the probability that the card is (i) an ace and (ii) black card.
\end{enumerate}
\solution
%\input{exemplar/11/16/3/4/main2.tex}
\item The probability that a non leap year selected at random will contain 53 sundays.
\\
\solution
%\input{exemplar/10/13/1/19/main.tex}
\item One of the four persons John, Rita, Aslam or Gurpreet will be promoted next
month. Consequently the sample space consists of four elementary outcomes
S = {John promoted, Rita promoted, Aslam promoted, Gurpreet promoted}
You are told that the chances of John’s promotion is same as that of Gurpreet,
Rita’s chances of promotion are twice as likely as Johns. Aslam’s chances are
four times that of John.
\begin{enumerate}
	\item Determine
	\begin{enumerate}
		\item P (John promoted)
		\item P (Rita promoted)
		\item P (Aslam promoted)
		\item P (Gurpreet promoted)
	\end{enumerate}
	\item If A = {John promoted or Gurpreet promoted}, find P (A).
\end{enumerate}
\solution
%\input{exemplar/11/16/3/10/main.tex}
\item A card is drawn from a deck of 52 cards. Find the probability of getting a king or a heart or a red card.\\
\solution
%\input{exemplar/11/16/3/15/main.tex}
\item The probability that a student will pass his examination is 0.73, the probability of
the student getting a compartment is 0.13, and the probability that the student will
either pass or get compartment is 0.96. State True or False.\\
\solution
%\input{exemplar/11/16/3/31/main.tex}
\item A card is selected from a pack of 52 cards\\
\begin{enumerate}[label=(\alph*)]
\item How many points are there in the sample space?
\item Calculate the probability that the cards is an ace of spades.
\item Calculate the probability that the card is (i) an ace (ii)black card.\\
\end{enumerate}
%\input{ncert/11/16/3/4_1/Prob_4.tex}
\item In a non-leap year, the probability of having 53 tuesdays or 53 wednesdays is\\
\solution
%\input{exemplar/11/16/3/18/main.tex}
\item There are 1000 sealed envelopes in a box, 10 of them contain a cash prize of
Rs 100 each, 100 of them contain a cash prize of Rs 50 each and 200 of them
contain a cash prize of Rs 10 each and rest do not contain any cash prize. If they
are well shuffled and an envelope is picked up out, what is the probability that it
contains no cash prize?\\
\solution
%\input{exemplar/10/13/3/34/main.tex}
\item 
A die is thrown and a card is selected at random from a deck of 52 playing cards. The probability of getting an even number on the die and a spade card.\\
\solution
%\input{exemplar/12/13/3/78/main.tex}
\item
If 4-digit numbers greater than 5,000 are randomly formed from the digits 0, 1, 3, 5, and 7, what is the probability of forming a number divisible by 5 when:
\begin{enumerate}
    \item The digits are repeated?
    \item The repetition of digits is not allowed?
\end{enumerate}
\solution
%\input{ncert/11/16/4/9/main.tex}
\item Consider the probability space $\brak{\Omega, \mathcal{G}, P}$ where $\Omega = [0,2]$ and $\mathcal{G} = \cbrak{\phi, \Omega, [0,1], (1,2]}$. Let $X$ and $Y$ be two functions on $\Omega$ defined as
\begin{align*}
    X(\omega) = 
    \begin{cases}
        1 & \text{if }\omega \in [0, 1]\\
        2 & \text{if }\omega \in (1, 2]
    \end{cases}
\end{align*}
and
\begin{align*}
    Y(\omega) = 
    \begin{cases}
        2 & \text{if }\omega \in [0, 1.5]\\
        3 & \text{if }\omega \in (1.5, 2].
    \end{cases}
\end{align*}
Then which one of the following statements is true?
\begin{enumerate}
    \item [(A)] $X$ is a random variable with respect to $\mathcal{G}$, but $Y$ is not a random variable with respect to $\mathcal{G}$.
    \item [(B)] $Y$ is a random variable with respect to $\mathcal{G}$, but $X$ is not a random variable with respect to $\mathcal{G}$.
    \item [(C)] Neither $X$ nor $Y$ is a random variable with respect to $\mathcal{G}$.
    \item [(D)] Both $X$ and $Y$ are random variables with respect to $\mathcal{G}$.
\end{enumerate} \hfill (GATE ST 2023)\\
\solution
%\input{gate/ST/2023/14/main.tex}
	\item  A die is loaded in such a way that each odd number is twice as likely to occur as
each even number. Find $P(G)$, where $G$ is the event that a number greater than
3 occurs on a single roll of the die.
\\
\solution
		%\input{exemplar/11/16/3/5/main.tex}
	\item All the jacks, queens and kings are removed from a deck of 52 playing cards. The remaining cards are well shuffled and then one card is drawn at random. Giving ace a value 1 similar value for other cards, find the probability that the card has a value 
		\begin{enumerate}
			\item 7
			\item greater than 7
			\item less than 7
		\end{enumerate}
		%\input{exemplar/10/13/3/30/main.tex}
  \item A Lot consists of 48 mobile phones of which 42 are good, 3 have only minor defects and 3 have major defects.Varnika will buy a phone if it is good but the trader will only buy a mobile if it has no major defects. One phone is selected at random from the lot. What is the probability that it is
\begin{enumerate}
	\item acceptable to Varnika?
            \item acceptable to the trader?
\end{enumerate}
\solution
	%\input{exemplar/10/13/3/40/main.tex}
 \item A student says that if you throw a die, it will show up 1 or not 1. Therefore, the probability of getting 1 and the probability of getting 'not 1' each is equal to $\frac{1}{2}$. Is this correct? Give reasons.\\
 \solution
        %\input{exemplar/10/13/2/9/main.tex}
   \item Four candidates A, B, C, D have ap-
plied for the assignment to coach a school cricket
team. If A is twice as likely to be selected as B, and
B and C are given about the same chance of being
selected, while C is twice as likely to be selected
as D, what are the probabilities that
\begin{enumerate}
\item C will be selected?
\item A will not be selected?
\end{enumerate}
	%\input{exemplar/11/16/3/9/main.tex}
 \item A bag contain 24 balls of which $x$ balls are red, $2x$ are white and $3x$ are blue. A ball is selected at random, What is the probability that it is
\begin{enumerate}[label=\alph*)]
\item not red ?
\item white ?
\end{enumerate}
%\input{exemplar/10/13/3/41/main.tex}
If the letters of the word ASSASSINATION are arranged at random. Find the Probability that
\begin{enumerate}[label=(\alph*)]
\item Four $S's$ come consecutively in the word
\item Two  $I's$ and two $N's$ come together
\item All $A's$ are not coming together
\item No two $A's$ are coming together
\end{enumerate}
%\input{exemplar/11/16/3/14/main.tex}
	\item One urn contains two black balls (labelled B1 and B2) and one white ball. A
	second urn contains one black ball and two white balls (labelled W1 and W2).
	Suppose the following experiment is performed. One of the two urns is chosen
	at random. Next a ball is randomly chosen from the urn. Then a second ball is
	chosen at random from the same urn without replacing the first ball.
	
	\begin{enumerate}
	\item What is the probability that two black balls are chosen?
	
	\item What is the probability that two balls of opposite colour are chosen?
	\end{enumerate}
	\solution
	%\input{exemplar/11/16/3/12/main1.tex}
\end{enumerate}

		%
\item 
Two cards are drawn at random and without replacement from a pack of 52 playing cards. Find the probability that both the cards are black.
\\
\solution
		%\begin{enumerate}[label=\thesection.\arabic*,ref=\thesection.\theenumi]
	\item One card is drawn from a well-shuffled deck of 52 cards. Find the probability of getting
\begin{enumerate}
\item A king of red colour 
\item A face card 
\item A red face card
\item The jack of hearts
\item A spade
\item The queen of diamonds

\end{enumerate}
\solution
		%\input{ncert/10/15/1/14/main.tex}
	\item Five cards—the ten, jack, queen, king and ace of diamonds, are well-shuffled with their face downwards. One card is then picked up at random.
\begin{enumerate}
\item
What is the probability that the card is the queen? 
\item
If the queen is drawn and put aside, what is the probability that the second card picked up is (a) an ace? (b) a queen?\\
\end{enumerate}
\solution
		%\input{ncert/10/15/1/15/defs.tex}
	\item A bag contains $5$ red balls and some blue balls. If the probability of drawing a blue ball is double that if a red ball, determine the number of blue balls in the bag. 
		\\
\solution
		%\input{ncert/10/15/2/3/defs.tex}
	\item A card is selected from a pack of 52 cards.
 \begin{enumerate}[label=(\alph*)] 
                 \item How many points are there in the sample space?
                 \item Calculate the probability that the card is an ace of spades.
                 \item Calculate the probability that the card is (i) an ace and (ii) black card.
 \end{enumerate}
\solution
		%\input{ncert/11/16/3/4/main.tex}
\item Four cards are drawn from a well-shuffled deck of 52 cards. What is the probability of obtaining 3 diamonds and one spade.
\\
\solution
		%\input{ncert/11/16/4/2/defs.tex}
\item In a certain lottery 10,000 tickets are sold and ten equal prizes are awarded. What is the probability of not getting a prize if you buy (a) one ticket (b) two tickets (c) 10 tickets ?	
\\
\solution
		%\input{ncert/11/16/4/4/defs.tex}
		%
\item 
Out of 100 students, two sections of 40 and 60 are formed. If you and your friend are among the 100 students, what is the probability that
\begin{enumerate}
\item you both enter the same section?
\item you both enter the different sections?
\end{enumerate}
\solution
		%\input{ncert/11/16/4/5/defs.tex}
	\item 
The number lock of a suitcase has 4 wheels each labelled with ten digits i.e. from 0 to 9.The lock opens with a sequence of four digits with no repeats.What is the probability of a person getting the right sequence to open the suitcase.
\\
\solution
		%\input{ncert/11/16/4/10/defs.tex}
		%
\item 
Two cards are drawn at random and without replacement from a pack of 52 playing cards. Find the probability that both the cards are black.
\\
\solution
		%\input{ncert/12/13/2/2/defs.tex}
		\item A box of oranges is inspected by examining three randomly selected oranges drawn without replacement. If all the three oranges are good, the box is approved for sale, otherwise, it is rejected. Find the probability that a box containing 15 oranges out of which 12 are good and 3 are bad ones will be approved for sale.
		\label{ncert/12/13/2/3/defs.tex}
		\item Two balls are drawn at random with replacement from a box containing 10 black and 8 red balls. Find the probability that
		\label{ncert/12/13/2/12}
\begin{enumerate}
\item both balls are red.
\item first ball is black and second is red.
\item one of them is black and other is red.
\end{enumerate}

\item In a hostel, 60\% of the students read Hindi newspaper, 40\% read English newspaper and 20\% read both Hindi and English newspapers. A student is selected at random.
		\label{ncert/12/13/2/15}
\begin{enumerate}
\item Find the probability that she reads neither Hindi nor English newspapers.
\item If she reads Hindi newspaper, find the probability that she reads English newspaper.
\item If she reads English newspaper, find the probability that she reads Hindi newspaper.\\
\end{enumerate}
\item The probability of obtaining an even prime number on each die, when a pair of dice is rolled is 
\begin{enumerate}
    \item $0$ 
    
    \item $\frac{1}{3}$ 
    
    \item $\frac{1}{12}$ 
    
    \item $\frac{1}{36}$ 
\end{enumerate}
\solution
		%\input{ncert/12/13/2/17/defs.tex}
	\item A bag contains 4 red and 4 black balls, another bag contains 2 red and 6 black balls. One of the two bags is selected at random and a ball is drawn from the bag which is found to be red. Find the probability that the ball is drawn from the first bag.
\\
\solution
		%\input{ncert/12/13/3/2/main.tex}
  \item
  Cards with numbers 2 to 101 are placed in a box. A card is selected at random.Find the probability that the card has
\begin{enumerate}[label=(\roman*)]
	\item an even number 
	\item a square number
\end{enumerate}
\solution
%\input{exemplar/10/13/3/32/main.tex}
\item
The king, queen and jack of clubs are removed from a deck of 52 playing cards and then well shuffled. Now one card is drawn at random from the remaining cards.  Determine the probability that the card is
\begin{enumerate}[label=(\roman*)]
\item a club
\item 10 of hearts
\end{enumerate}
\solution
%\input{exemplar/10/13/3/29/main.tex}
\item A team of medical students doing their internship have to assist during surgeries
at a city hospital. The probabilities of surgeries rated as very complex, complex,
routine, simple or very simple are respectively, 0.15, 0.20, 0.31, 0.26, .08. Find
the probabilities that a particular surgery will be rated
\begin{enumerate}
	\item complex or very complex;
	\item neither very complex nor very simple;
	\item routine or complex
	\item routine or simple
\end{enumerate}
\solution
%\input{exemplar/11/16/3/8(1)/main.tex}
\item A card is selected from a pack of 52 cards.
\begin{enumerate}[label=(\alph*)]
    \item How many points are there in the sample space?
    \item Calculate the probability that the card is an ace of spades.
    \item Calculate the probability that the card is (i) an ace and (ii) black card.
\end{enumerate}
\solution
%\input{exemplar/11/16/3/4/main2.tex}
\item The probability that a non leap year selected at random will contain 53 sundays.
\\
\solution
%\input{exemplar/10/13/1/19/main.tex}
\item One of the four persons John, Rita, Aslam or Gurpreet will be promoted next
month. Consequently the sample space consists of four elementary outcomes
S = {John promoted, Rita promoted, Aslam promoted, Gurpreet promoted}
You are told that the chances of John’s promotion is same as that of Gurpreet,
Rita’s chances of promotion are twice as likely as Johns. Aslam’s chances are
four times that of John.
\begin{enumerate}
	\item Determine
	\begin{enumerate}
		\item P (John promoted)
		\item P (Rita promoted)
		\item P (Aslam promoted)
		\item P (Gurpreet promoted)
	\end{enumerate}
	\item If A = {John promoted or Gurpreet promoted}, find P (A).
\end{enumerate}
\solution
%\input{exemplar/11/16/3/10/main.tex}
\item A card is drawn from a deck of 52 cards. Find the probability of getting a king or a heart or a red card.\\
\solution
%\input{exemplar/11/16/3/15/main.tex}
\item The probability that a student will pass his examination is 0.73, the probability of
the student getting a compartment is 0.13, and the probability that the student will
either pass or get compartment is 0.96. State True or False.\\
\solution
%\input{exemplar/11/16/3/31/main.tex}
\item A card is selected from a pack of 52 cards\\
\begin{enumerate}[label=(\alph*)]
\item How many points are there in the sample space?
\item Calculate the probability that the cards is an ace of spades.
\item Calculate the probability that the card is (i) an ace (ii)black card.\\
\end{enumerate}
%\input{ncert/11/16/3/4_1/Prob_4.tex}
\item In a non-leap year, the probability of having 53 tuesdays or 53 wednesdays is\\
\solution
%\input{exemplar/11/16/3/18/main.tex}
\item There are 1000 sealed envelopes in a box, 10 of them contain a cash prize of
Rs 100 each, 100 of them contain a cash prize of Rs 50 each and 200 of them
contain a cash prize of Rs 10 each and rest do not contain any cash prize. If they
are well shuffled and an envelope is picked up out, what is the probability that it
contains no cash prize?\\
\solution
%\input{exemplar/10/13/3/34/main.tex}
\item 
A die is thrown and a card is selected at random from a deck of 52 playing cards. The probability of getting an even number on the die and a spade card.\\
\solution
%\input{exemplar/12/13/3/78/main.tex}
\item
If 4-digit numbers greater than 5,000 are randomly formed from the digits 0, 1, 3, 5, and 7, what is the probability of forming a number divisible by 5 when:
\begin{enumerate}
    \item The digits are repeated?
    \item The repetition of digits is not allowed?
\end{enumerate}
\solution
%\input{ncert/11/16/4/9/main.tex}
\item Consider the probability space $\brak{\Omega, \mathcal{G}, P}$ where $\Omega = [0,2]$ and $\mathcal{G} = \cbrak{\phi, \Omega, [0,1], (1,2]}$. Let $X$ and $Y$ be two functions on $\Omega$ defined as
\begin{align*}
    X(\omega) = 
    \begin{cases}
        1 & \text{if }\omega \in [0, 1]\\
        2 & \text{if }\omega \in (1, 2]
    \end{cases}
\end{align*}
and
\begin{align*}
    Y(\omega) = 
    \begin{cases}
        2 & \text{if }\omega \in [0, 1.5]\\
        3 & \text{if }\omega \in (1.5, 2].
    \end{cases}
\end{align*}
Then which one of the following statements is true?
\begin{enumerate}
    \item [(A)] $X$ is a random variable with respect to $\mathcal{G}$, but $Y$ is not a random variable with respect to $\mathcal{G}$.
    \item [(B)] $Y$ is a random variable with respect to $\mathcal{G}$, but $X$ is not a random variable with respect to $\mathcal{G}$.
    \item [(C)] Neither $X$ nor $Y$ is a random variable with respect to $\mathcal{G}$.
    \item [(D)] Both $X$ and $Y$ are random variables with respect to $\mathcal{G}$.
\end{enumerate} \hfill (GATE ST 2023)\\
\solution
%\input{gate/ST/2023/14/main.tex}
	\item  A die is loaded in such a way that each odd number is twice as likely to occur as
each even number. Find $P(G)$, where $G$ is the event that a number greater than
3 occurs on a single roll of the die.
\\
\solution
		%\input{exemplar/11/16/3/5/main.tex}
	\item All the jacks, queens and kings are removed from a deck of 52 playing cards. The remaining cards are well shuffled and then one card is drawn at random. Giving ace a value 1 similar value for other cards, find the probability that the card has a value 
		\begin{enumerate}
			\item 7
			\item greater than 7
			\item less than 7
		\end{enumerate}
		%\input{exemplar/10/13/3/30/main.tex}
  \item A Lot consists of 48 mobile phones of which 42 are good, 3 have only minor defects and 3 have major defects.Varnika will buy a phone if it is good but the trader will only buy a mobile if it has no major defects. One phone is selected at random from the lot. What is the probability that it is
\begin{enumerate}
	\item acceptable to Varnika?
            \item acceptable to the trader?
\end{enumerate}
\solution
	%\input{exemplar/10/13/3/40/main.tex}
 \item A student says that if you throw a die, it will show up 1 or not 1. Therefore, the probability of getting 1 and the probability of getting 'not 1' each is equal to $\frac{1}{2}$. Is this correct? Give reasons.\\
 \solution
        %\input{exemplar/10/13/2/9/main.tex}
   \item Four candidates A, B, C, D have ap-
plied for the assignment to coach a school cricket
team. If A is twice as likely to be selected as B, and
B and C are given about the same chance of being
selected, while C is twice as likely to be selected
as D, what are the probabilities that
\begin{enumerate}
\item C will be selected?
\item A will not be selected?
\end{enumerate}
	%\input{exemplar/11/16/3/9/main.tex}
 \item A bag contain 24 balls of which $x$ balls are red, $2x$ are white and $3x$ are blue. A ball is selected at random, What is the probability that it is
\begin{enumerate}[label=\alph*)]
\item not red ?
\item white ?
\end{enumerate}
%\input{exemplar/10/13/3/41/main.tex}
If the letters of the word ASSASSINATION are arranged at random. Find the Probability that
\begin{enumerate}[label=(\alph*)]
\item Four $S's$ come consecutively in the word
\item Two  $I's$ and two $N's$ come together
\item All $A's$ are not coming together
\item No two $A's$ are coming together
\end{enumerate}
%\input{exemplar/11/16/3/14/main.tex}
	\item One urn contains two black balls (labelled B1 and B2) and one white ball. A
	second urn contains one black ball and two white balls (labelled W1 and W2).
	Suppose the following experiment is performed. One of the two urns is chosen
	at random. Next a ball is randomly chosen from the urn. Then a second ball is
	chosen at random from the same urn without replacing the first ball.
	
	\begin{enumerate}
	\item What is the probability that two black balls are chosen?
	
	\item What is the probability that two balls of opposite colour are chosen?
	\end{enumerate}
	\solution
	%\input{exemplar/11/16/3/12/main1.tex}
\end{enumerate}

		\item A box of oranges is inspected by examining three randomly selected oranges drawn without replacement. If all the three oranges are good, the box is approved for sale, otherwise, it is rejected. Find the probability that a box containing 15 oranges out of which 12 are good and 3 are bad ones will be approved for sale.
		\label{ncert/12/13/2/3/defs.tex}
		\item Two balls are drawn at random with replacement from a box containing 10 black and 8 red balls. Find the probability that
		\label{ncert/12/13/2/12}
\begin{enumerate}
\item both balls are red.
\item first ball is black and second is red.
\item one of them is black and other is red.
\end{enumerate}

\item In a hostel, 60\% of the students read Hindi newspaper, 40\% read English newspaper and 20\% read both Hindi and English newspapers. A student is selected at random.
		\label{ncert/12/13/2/15}
\begin{enumerate}
\item Find the probability that she reads neither Hindi nor English newspapers.
\item If she reads Hindi newspaper, find the probability that she reads English newspaper.
\item If she reads English newspaper, find the probability that she reads Hindi newspaper.\\
\end{enumerate}
\item The probability of obtaining an even prime number on each die, when a pair of dice is rolled is 
\begin{enumerate}
    \item $0$ 
    
    \item $\frac{1}{3}$ 
    
    \item $\frac{1}{12}$ 
    
    \item $\frac{1}{36}$ 
\end{enumerate}
\solution
		%\begin{enumerate}[label=\thesection.\arabic*,ref=\thesection.\theenumi]
	\item One card is drawn from a well-shuffled deck of 52 cards. Find the probability of getting
\begin{enumerate}
\item A king of red colour 
\item A face card 
\item A red face card
\item The jack of hearts
\item A spade
\item The queen of diamonds

\end{enumerate}
\solution
		%\input{ncert/10/15/1/14/main.tex}
	\item Five cards—the ten, jack, queen, king and ace of diamonds, are well-shuffled with their face downwards. One card is then picked up at random.
\begin{enumerate}
\item
What is the probability that the card is the queen? 
\item
If the queen is drawn and put aside, what is the probability that the second card picked up is (a) an ace? (b) a queen?\\
\end{enumerate}
\solution
		%\input{ncert/10/15/1/15/defs.tex}
	\item A bag contains $5$ red balls and some blue balls. If the probability of drawing a blue ball is double that if a red ball, determine the number of blue balls in the bag. 
		\\
\solution
		%\input{ncert/10/15/2/3/defs.tex}
	\item A card is selected from a pack of 52 cards.
 \begin{enumerate}[label=(\alph*)] 
                 \item How many points are there in the sample space?
                 \item Calculate the probability that the card is an ace of spades.
                 \item Calculate the probability that the card is (i) an ace and (ii) black card.
 \end{enumerate}
\solution
		%\input{ncert/11/16/3/4/main.tex}
\item Four cards are drawn from a well-shuffled deck of 52 cards. What is the probability of obtaining 3 diamonds and one spade.
\\
\solution
		%\input{ncert/11/16/4/2/defs.tex}
\item In a certain lottery 10,000 tickets are sold and ten equal prizes are awarded. What is the probability of not getting a prize if you buy (a) one ticket (b) two tickets (c) 10 tickets ?	
\\
\solution
		%\input{ncert/11/16/4/4/defs.tex}
		%
\item 
Out of 100 students, two sections of 40 and 60 are formed. If you and your friend are among the 100 students, what is the probability that
\begin{enumerate}
\item you both enter the same section?
\item you both enter the different sections?
\end{enumerate}
\solution
		%\input{ncert/11/16/4/5/defs.tex}
	\item 
The number lock of a suitcase has 4 wheels each labelled with ten digits i.e. from 0 to 9.The lock opens with a sequence of four digits with no repeats.What is the probability of a person getting the right sequence to open the suitcase.
\\
\solution
		%\input{ncert/11/16/4/10/defs.tex}
		%
\item 
Two cards are drawn at random and without replacement from a pack of 52 playing cards. Find the probability that both the cards are black.
\\
\solution
		%\input{ncert/12/13/2/2/defs.tex}
		\item A box of oranges is inspected by examining three randomly selected oranges drawn without replacement. If all the three oranges are good, the box is approved for sale, otherwise, it is rejected. Find the probability that a box containing 15 oranges out of which 12 are good and 3 are bad ones will be approved for sale.
		\label{ncert/12/13/2/3/defs.tex}
		\item Two balls are drawn at random with replacement from a box containing 10 black and 8 red balls. Find the probability that
		\label{ncert/12/13/2/12}
\begin{enumerate}
\item both balls are red.
\item first ball is black and second is red.
\item one of them is black and other is red.
\end{enumerate}

\item In a hostel, 60\% of the students read Hindi newspaper, 40\% read English newspaper and 20\% read both Hindi and English newspapers. A student is selected at random.
		\label{ncert/12/13/2/15}
\begin{enumerate}
\item Find the probability that she reads neither Hindi nor English newspapers.
\item If she reads Hindi newspaper, find the probability that she reads English newspaper.
\item If she reads English newspaper, find the probability that she reads Hindi newspaper.\\
\end{enumerate}
\item The probability of obtaining an even prime number on each die, when a pair of dice is rolled is 
\begin{enumerate}
    \item $0$ 
    
    \item $\frac{1}{3}$ 
    
    \item $\frac{1}{12}$ 
    
    \item $\frac{1}{36}$ 
\end{enumerate}
\solution
		%\input{ncert/12/13/2/17/defs.tex}
	\item A bag contains 4 red and 4 black balls, another bag contains 2 red and 6 black balls. One of the two bags is selected at random and a ball is drawn from the bag which is found to be red. Find the probability that the ball is drawn from the first bag.
\\
\solution
		%\input{ncert/12/13/3/2/main.tex}
  \item
  Cards with numbers 2 to 101 are placed in a box. A card is selected at random.Find the probability that the card has
\begin{enumerate}[label=(\roman*)]
	\item an even number 
	\item a square number
\end{enumerate}
\solution
%\input{exemplar/10/13/3/32/main.tex}
\item
The king, queen and jack of clubs are removed from a deck of 52 playing cards and then well shuffled. Now one card is drawn at random from the remaining cards.  Determine the probability that the card is
\begin{enumerate}[label=(\roman*)]
\item a club
\item 10 of hearts
\end{enumerate}
\solution
%\input{exemplar/10/13/3/29/main.tex}
\item A team of medical students doing their internship have to assist during surgeries
at a city hospital. The probabilities of surgeries rated as very complex, complex,
routine, simple or very simple are respectively, 0.15, 0.20, 0.31, 0.26, .08. Find
the probabilities that a particular surgery will be rated
\begin{enumerate}
	\item complex or very complex;
	\item neither very complex nor very simple;
	\item routine or complex
	\item routine or simple
\end{enumerate}
\solution
%\input{exemplar/11/16/3/8(1)/main.tex}
\item A card is selected from a pack of 52 cards.
\begin{enumerate}[label=(\alph*)]
    \item How many points are there in the sample space?
    \item Calculate the probability that the card is an ace of spades.
    \item Calculate the probability that the card is (i) an ace and (ii) black card.
\end{enumerate}
\solution
%\input{exemplar/11/16/3/4/main2.tex}
\item The probability that a non leap year selected at random will contain 53 sundays.
\\
\solution
%\input{exemplar/10/13/1/19/main.tex}
\item One of the four persons John, Rita, Aslam or Gurpreet will be promoted next
month. Consequently the sample space consists of four elementary outcomes
S = {John promoted, Rita promoted, Aslam promoted, Gurpreet promoted}
You are told that the chances of John’s promotion is same as that of Gurpreet,
Rita’s chances of promotion are twice as likely as Johns. Aslam’s chances are
four times that of John.
\begin{enumerate}
	\item Determine
	\begin{enumerate}
		\item P (John promoted)
		\item P (Rita promoted)
		\item P (Aslam promoted)
		\item P (Gurpreet promoted)
	\end{enumerate}
	\item If A = {John promoted or Gurpreet promoted}, find P (A).
\end{enumerate}
\solution
%\input{exemplar/11/16/3/10/main.tex}
\item A card is drawn from a deck of 52 cards. Find the probability of getting a king or a heart or a red card.\\
\solution
%\input{exemplar/11/16/3/15/main.tex}
\item The probability that a student will pass his examination is 0.73, the probability of
the student getting a compartment is 0.13, and the probability that the student will
either pass or get compartment is 0.96. State True or False.\\
\solution
%\input{exemplar/11/16/3/31/main.tex}
\item A card is selected from a pack of 52 cards\\
\begin{enumerate}[label=(\alph*)]
\item How many points are there in the sample space?
\item Calculate the probability that the cards is an ace of spades.
\item Calculate the probability that the card is (i) an ace (ii)black card.\\
\end{enumerate}
%\input{ncert/11/16/3/4_1/Prob_4.tex}
\item In a non-leap year, the probability of having 53 tuesdays or 53 wednesdays is\\
\solution
%\input{exemplar/11/16/3/18/main.tex}
\item There are 1000 sealed envelopes in a box, 10 of them contain a cash prize of
Rs 100 each, 100 of them contain a cash prize of Rs 50 each and 200 of them
contain a cash prize of Rs 10 each and rest do not contain any cash prize. If they
are well shuffled and an envelope is picked up out, what is the probability that it
contains no cash prize?\\
\solution
%\input{exemplar/10/13/3/34/main.tex}
\item 
A die is thrown and a card is selected at random from a deck of 52 playing cards. The probability of getting an even number on the die and a spade card.\\
\solution
%\input{exemplar/12/13/3/78/main.tex}
\item
If 4-digit numbers greater than 5,000 are randomly formed from the digits 0, 1, 3, 5, and 7, what is the probability of forming a number divisible by 5 when:
\begin{enumerate}
    \item The digits are repeated?
    \item The repetition of digits is not allowed?
\end{enumerate}
\solution
%\input{ncert/11/16/4/9/main.tex}
\item Consider the probability space $\brak{\Omega, \mathcal{G}, P}$ where $\Omega = [0,2]$ and $\mathcal{G} = \cbrak{\phi, \Omega, [0,1], (1,2]}$. Let $X$ and $Y$ be two functions on $\Omega$ defined as
\begin{align*}
    X(\omega) = 
    \begin{cases}
        1 & \text{if }\omega \in [0, 1]\\
        2 & \text{if }\omega \in (1, 2]
    \end{cases}
\end{align*}
and
\begin{align*}
    Y(\omega) = 
    \begin{cases}
        2 & \text{if }\omega \in [0, 1.5]\\
        3 & \text{if }\omega \in (1.5, 2].
    \end{cases}
\end{align*}
Then which one of the following statements is true?
\begin{enumerate}
    \item [(A)] $X$ is a random variable with respect to $\mathcal{G}$, but $Y$ is not a random variable with respect to $\mathcal{G}$.
    \item [(B)] $Y$ is a random variable with respect to $\mathcal{G}$, but $X$ is not a random variable with respect to $\mathcal{G}$.
    \item [(C)] Neither $X$ nor $Y$ is a random variable with respect to $\mathcal{G}$.
    \item [(D)] Both $X$ and $Y$ are random variables with respect to $\mathcal{G}$.
\end{enumerate} \hfill (GATE ST 2023)\\
\solution
%\input{gate/ST/2023/14/main.tex}
	\item  A die is loaded in such a way that each odd number is twice as likely to occur as
each even number. Find $P(G)$, where $G$ is the event that a number greater than
3 occurs on a single roll of the die.
\\
\solution
		%\input{exemplar/11/16/3/5/main.tex}
	\item All the jacks, queens and kings are removed from a deck of 52 playing cards. The remaining cards are well shuffled and then one card is drawn at random. Giving ace a value 1 similar value for other cards, find the probability that the card has a value 
		\begin{enumerate}
			\item 7
			\item greater than 7
			\item less than 7
		\end{enumerate}
		%\input{exemplar/10/13/3/30/main.tex}
  \item A Lot consists of 48 mobile phones of which 42 are good, 3 have only minor defects and 3 have major defects.Varnika will buy a phone if it is good but the trader will only buy a mobile if it has no major defects. One phone is selected at random from the lot. What is the probability that it is
\begin{enumerate}
	\item acceptable to Varnika?
            \item acceptable to the trader?
\end{enumerate}
\solution
	%\input{exemplar/10/13/3/40/main.tex}
 \item A student says that if you throw a die, it will show up 1 or not 1. Therefore, the probability of getting 1 and the probability of getting 'not 1' each is equal to $\frac{1}{2}$. Is this correct? Give reasons.\\
 \solution
        %\input{exemplar/10/13/2/9/main.tex}
   \item Four candidates A, B, C, D have ap-
plied for the assignment to coach a school cricket
team. If A is twice as likely to be selected as B, and
B and C are given about the same chance of being
selected, while C is twice as likely to be selected
as D, what are the probabilities that
\begin{enumerate}
\item C will be selected?
\item A will not be selected?
\end{enumerate}
	%\input{exemplar/11/16/3/9/main.tex}
 \item A bag contain 24 balls of which $x$ balls are red, $2x$ are white and $3x$ are blue. A ball is selected at random, What is the probability that it is
\begin{enumerate}[label=\alph*)]
\item not red ?
\item white ?
\end{enumerate}
%\input{exemplar/10/13/3/41/main.tex}
If the letters of the word ASSASSINATION are arranged at random. Find the Probability that
\begin{enumerate}[label=(\alph*)]
\item Four $S's$ come consecutively in the word
\item Two  $I's$ and two $N's$ come together
\item All $A's$ are not coming together
\item No two $A's$ are coming together
\end{enumerate}
%\input{exemplar/11/16/3/14/main.tex}
	\item One urn contains two black balls (labelled B1 and B2) and one white ball. A
	second urn contains one black ball and two white balls (labelled W1 and W2).
	Suppose the following experiment is performed. One of the two urns is chosen
	at random. Next a ball is randomly chosen from the urn. Then a second ball is
	chosen at random from the same urn without replacing the first ball.
	
	\begin{enumerate}
	\item What is the probability that two black balls are chosen?
	
	\item What is the probability that two balls of opposite colour are chosen?
	\end{enumerate}
	\solution
	%\input{exemplar/11/16/3/12/main1.tex}
\end{enumerate}

	\item A bag contains 4 red and 4 black balls, another bag contains 2 red and 6 black balls. One of the two bags is selected at random and a ball is drawn from the bag which is found to be red. Find the probability that the ball is drawn from the first bag.
\\
\solution
		%\begin{table}[H]
	\centering
\begin{tabular}{|c|c|c|}
\hline
Random variable &Value &Definition\\ \hline
\multirow{3}{*}{X} &0 &Slips of Rs 1\\
&1 &Slips of Rs 5\\
&2 &Slips of Rs 13\\ \hline
\multirow{2}{*}{Y} &0 &Box A\\
&1 &Box B\\\hline
\end{tabular}
\caption{}
\label{tab:Distribution}
\end{table}
See \tabref{tab:Distribution}.
\begin{align}
p_{Y}\brak{k}= \begin{cases} 
      \frac{1}{3} & {k=0} \\
      \frac{2}{3 }& {k=1} 
   \end{cases}
   \\
p_{Y|X}\brak{0|0} = \frac{19}{25}\, 
p_{Y|X}\brak{0|1} = \frac{6}{25}\,
p_{Y|X}\brak{1|0} = \frac{45}{50}\,
p_{Y|X}\brak{1|2} = \frac{5}{50}
\end{align}
The desired probability is the probability that a slip drawn at random is marked other than Rs 1,
\begin{align}
&=1-p_X\brak{0}\\
&= p_X(1) + p_X(2)
\end{align}
Using Bayes theorem,
\begin{align}
&= p_Y\brak{0} \times \pr{Y=0 | X=1} + p_Y\brak{1} \times \pr{Y=1|X=2}\\
&=\frac{1}{3} \times \frac{6}{25} + \frac{2}{3} \times \frac{5}{50}\\
&=\frac{11}{75}
\end{align}

\newpage

%\tableofcontents

\bigskip

\renewcommand{\thefigure}{\theenumi}
\renewcommand{\thetable}{\theenumi}
%\renewcommand{\theequation}{\theenumi}

%\begin{abstract}
%%\boldmath
%In this letter, an algorithm for evaluating the exact analytical bit error rate  (BER)  for the piecewise linear (PL) combiner for  multiple relays is presented. Previous results were available only for upto three relays. The algorithm is unique in the sense that  the actual mathematical expressions, that are prohibitively large, need not be explicitly obtained. The diversity gain due to multiple relays is shown through plots of the analytical BER, well supported by simulations. 
%
%\end{abstract}
% IEEEtran.cls defaults to using nonbold math in the Abstract.
% This preserves the distinction between vectors and scalars. However,
% if the journal you are submitting to favors bold math in the abstract,
% then you can use LaTeX's standard command \boldmath at the very start
% of the abstract to achieve this. Many IEEE journals frown on math
% in the abstract anyway.

% Note that keywords are not normally used for peerreview papers.
%\begin{IEEEkeywords}
%Cooperative diversity, decode and forward, piecewise linear
%\end{IEEEkeywords}



% For peer review papers, you can put extra information on the cover
% page as needed:
% \ifCLASSOPTIONpeerreview
% \begin{center} \bfseries EDICS Category: 3-BBND \end{center}
% \fi
%
% For peerreview papers, this IEEEtran command inserts a page break and
% creates the second title. It will be ignored for other modes.
%\IEEEpeerreviewmaketitle




  \item
  Cards with numbers 2 to 101 are placed in a box. A card is selected at random.Find the probability that the card has
\begin{enumerate}[label=(\roman*)]
	\item an even number 
	\item a square number
\end{enumerate}
\solution
%\begin{table}[H]
	\centering
\begin{tabular}{|c|c|c|}
\hline
Random variable &Value &Definition\\ \hline
\multirow{3}{*}{X} &0 &Slips of Rs 1\\
&1 &Slips of Rs 5\\
&2 &Slips of Rs 13\\ \hline
\multirow{2}{*}{Y} &0 &Box A\\
&1 &Box B\\\hline
\end{tabular}
\caption{}
\label{tab:Distribution}
\end{table}
See \tabref{tab:Distribution}.
\begin{align}
p_{Y}\brak{k}= \begin{cases} 
      \frac{1}{3} & {k=0} \\
      \frac{2}{3 }& {k=1} 
   \end{cases}
   \\
p_{Y|X}\brak{0|0} = \frac{19}{25}\, 
p_{Y|X}\brak{0|1} = \frac{6}{25}\,
p_{Y|X}\brak{1|0} = \frac{45}{50}\,
p_{Y|X}\brak{1|2} = \frac{5}{50}
\end{align}
The desired probability is the probability that a slip drawn at random is marked other than Rs 1,
\begin{align}
&=1-p_X\brak{0}\\
&= p_X(1) + p_X(2)
\end{align}
Using Bayes theorem,
\begin{align}
&= p_Y\brak{0} \times \pr{Y=0 | X=1} + p_Y\brak{1} \times \pr{Y=1|X=2}\\
&=\frac{1}{3} \times \frac{6}{25} + \frac{2}{3} \times \frac{5}{50}\\
&=\frac{11}{75}
\end{align}

\newpage

%\tableofcontents

\bigskip

\renewcommand{\thefigure}{\theenumi}
\renewcommand{\thetable}{\theenumi}
%\renewcommand{\theequation}{\theenumi}

%\begin{abstract}
%%\boldmath
%In this letter, an algorithm for evaluating the exact analytical bit error rate  (BER)  for the piecewise linear (PL) combiner for  multiple relays is presented. Previous results were available only for upto three relays. The algorithm is unique in the sense that  the actual mathematical expressions, that are prohibitively large, need not be explicitly obtained. The diversity gain due to multiple relays is shown through plots of the analytical BER, well supported by simulations. 
%
%\end{abstract}
% IEEEtran.cls defaults to using nonbold math in the Abstract.
% This preserves the distinction between vectors and scalars. However,
% if the journal you are submitting to favors bold math in the abstract,
% then you can use LaTeX's standard command \boldmath at the very start
% of the abstract to achieve this. Many IEEE journals frown on math
% in the abstract anyway.

% Note that keywords are not normally used for peerreview papers.
%\begin{IEEEkeywords}
%Cooperative diversity, decode and forward, piecewise linear
%\end{IEEEkeywords}



% For peer review papers, you can put extra information on the cover
% page as needed:
% \ifCLASSOPTIONpeerreview
% \begin{center} \bfseries EDICS Category: 3-BBND \end{center}
% \fi
%
% For peerreview papers, this IEEEtran command inserts a page break and
% creates the second title. It will be ignored for other modes.
%\IEEEpeerreviewmaketitle




\item
The king, queen and jack of clubs are removed from a deck of 52 playing cards and then well shuffled. Now one card is drawn at random from the remaining cards.  Determine the probability that the card is
\begin{enumerate}[label=(\roman*)]
\item a club
\item 10 of hearts
\end{enumerate}
\solution
%\begin{table}[H]
	\centering
\begin{tabular}{|c|c|c|}
\hline
Random variable &Value &Definition\\ \hline
\multirow{3}{*}{X} &0 &Slips of Rs 1\\
&1 &Slips of Rs 5\\
&2 &Slips of Rs 13\\ \hline
\multirow{2}{*}{Y} &0 &Box A\\
&1 &Box B\\\hline
\end{tabular}
\caption{}
\label{tab:Distribution}
\end{table}
See \tabref{tab:Distribution}.
\begin{align}
p_{Y}\brak{k}= \begin{cases} 
      \frac{1}{3} & {k=0} \\
      \frac{2}{3 }& {k=1} 
   \end{cases}
   \\
p_{Y|X}\brak{0|0} = \frac{19}{25}\, 
p_{Y|X}\brak{0|1} = \frac{6}{25}\,
p_{Y|X}\brak{1|0} = \frac{45}{50}\,
p_{Y|X}\brak{1|2} = \frac{5}{50}
\end{align}
The desired probability is the probability that a slip drawn at random is marked other than Rs 1,
\begin{align}
&=1-p_X\brak{0}\\
&= p_X(1) + p_X(2)
\end{align}
Using Bayes theorem,
\begin{align}
&= p_Y\brak{0} \times \pr{Y=0 | X=1} + p_Y\brak{1} \times \pr{Y=1|X=2}\\
&=\frac{1}{3} \times \frac{6}{25} + \frac{2}{3} \times \frac{5}{50}\\
&=\frac{11}{75}
\end{align}

\newpage

%\tableofcontents

\bigskip

\renewcommand{\thefigure}{\theenumi}
\renewcommand{\thetable}{\theenumi}
%\renewcommand{\theequation}{\theenumi}

%\begin{abstract}
%%\boldmath
%In this letter, an algorithm for evaluating the exact analytical bit error rate  (BER)  for the piecewise linear (PL) combiner for  multiple relays is presented. Previous results were available only for upto three relays. The algorithm is unique in the sense that  the actual mathematical expressions, that are prohibitively large, need not be explicitly obtained. The diversity gain due to multiple relays is shown through plots of the analytical BER, well supported by simulations. 
%
%\end{abstract}
% IEEEtran.cls defaults to using nonbold math in the Abstract.
% This preserves the distinction between vectors and scalars. However,
% if the journal you are submitting to favors bold math in the abstract,
% then you can use LaTeX's standard command \boldmath at the very start
% of the abstract to achieve this. Many IEEE journals frown on math
% in the abstract anyway.

% Note that keywords are not normally used for peerreview papers.
%\begin{IEEEkeywords}
%Cooperative diversity, decode and forward, piecewise linear
%\end{IEEEkeywords}



% For peer review papers, you can put extra information on the cover
% page as needed:
% \ifCLASSOPTIONpeerreview
% \begin{center} \bfseries EDICS Category: 3-BBND \end{center}
% \fi
%
% For peerreview papers, this IEEEtran command inserts a page break and
% creates the second title. It will be ignored for other modes.
%\IEEEpeerreviewmaketitle




\item A team of medical students doing their internship have to assist during surgeries
at a city hospital. The probabilities of surgeries rated as very complex, complex,
routine, simple or very simple are respectively, 0.15, 0.20, 0.31, 0.26, .08. Find
the probabilities that a particular surgery will be rated
\begin{enumerate}
	\item complex or very complex;
	\item neither very complex nor very simple;
	\item routine or complex
	\item routine or simple
\end{enumerate}
\solution
%\begin{table}[H]
	\centering
\begin{tabular}{|c|c|c|}
\hline
Random variable &Value &Definition\\ \hline
\multirow{3}{*}{X} &0 &Slips of Rs 1\\
&1 &Slips of Rs 5\\
&2 &Slips of Rs 13\\ \hline
\multirow{2}{*}{Y} &0 &Box A\\
&1 &Box B\\\hline
\end{tabular}
\caption{}
\label{tab:Distribution}
\end{table}
See \tabref{tab:Distribution}.
\begin{align}
p_{Y}\brak{k}= \begin{cases} 
      \frac{1}{3} & {k=0} \\
      \frac{2}{3 }& {k=1} 
   \end{cases}
   \\
p_{Y|X}\brak{0|0} = \frac{19}{25}\, 
p_{Y|X}\brak{0|1} = \frac{6}{25}\,
p_{Y|X}\brak{1|0} = \frac{45}{50}\,
p_{Y|X}\brak{1|2} = \frac{5}{50}
\end{align}
The desired probability is the probability that a slip drawn at random is marked other than Rs 1,
\begin{align}
&=1-p_X\brak{0}\\
&= p_X(1) + p_X(2)
\end{align}
Using Bayes theorem,
\begin{align}
&= p_Y\brak{0} \times \pr{Y=0 | X=1} + p_Y\brak{1} \times \pr{Y=1|X=2}\\
&=\frac{1}{3} \times \frac{6}{25} + \frac{2}{3} \times \frac{5}{50}\\
&=\frac{11}{75}
\end{align}

\newpage

%\tableofcontents

\bigskip

\renewcommand{\thefigure}{\theenumi}
\renewcommand{\thetable}{\theenumi}
%\renewcommand{\theequation}{\theenumi}

%\begin{abstract}
%%\boldmath
%In this letter, an algorithm for evaluating the exact analytical bit error rate  (BER)  for the piecewise linear (PL) combiner for  multiple relays is presented. Previous results were available only for upto three relays. The algorithm is unique in the sense that  the actual mathematical expressions, that are prohibitively large, need not be explicitly obtained. The diversity gain due to multiple relays is shown through plots of the analytical BER, well supported by simulations. 
%
%\end{abstract}
% IEEEtran.cls defaults to using nonbold math in the Abstract.
% This preserves the distinction between vectors and scalars. However,
% if the journal you are submitting to favors bold math in the abstract,
% then you can use LaTeX's standard command \boldmath at the very start
% of the abstract to achieve this. Many IEEE journals frown on math
% in the abstract anyway.

% Note that keywords are not normally used for peerreview papers.
%\begin{IEEEkeywords}
%Cooperative diversity, decode and forward, piecewise linear
%\end{IEEEkeywords}



% For peer review papers, you can put extra information on the cover
% page as needed:
% \ifCLASSOPTIONpeerreview
% \begin{center} \bfseries EDICS Category: 3-BBND \end{center}
% \fi
%
% For peerreview papers, this IEEEtran command inserts a page break and
% creates the second title. It will be ignored for other modes.
%\IEEEpeerreviewmaketitle




\item A card is selected from a pack of 52 cards.
\begin{enumerate}[label=(\alph*)]
    \item How many points are there in the sample space?
    \item Calculate the probability that the card is an ace of spades.
    \item Calculate the probability that the card is (i) an ace and (ii) black card.
\end{enumerate}
\solution
%Let $X$ be an bernoulli rv defined as in \tabref{tab:exemplar/11/16/3/26}.  Then, 
\begin{equation}
    p =
        \frac{4}{11} 
\end{equation}
\begin{table}[H]
	\centering
	\input{exemplar/11/16/3/26/tables/Table2.tex}
	\caption{}
        \label{tab:exemplar/11/16/3/26}
\end{table}

\item The probability that a non leap year selected at random will contain 53 sundays.
\\
\solution
%\begin{table}[H]
	\centering
\begin{tabular}{|c|c|c|}
\hline
Random variable &Value &Definition\\ \hline
\multirow{3}{*}{X} &0 &Slips of Rs 1\\
&1 &Slips of Rs 5\\
&2 &Slips of Rs 13\\ \hline
\multirow{2}{*}{Y} &0 &Box A\\
&1 &Box B\\\hline
\end{tabular}
\caption{}
\label{tab:Distribution}
\end{table}
See \tabref{tab:Distribution}.
\begin{align}
p_{Y}\brak{k}= \begin{cases} 
      \frac{1}{3} & {k=0} \\
      \frac{2}{3 }& {k=1} 
   \end{cases}
   \\
p_{Y|X}\brak{0|0} = \frac{19}{25}\, 
p_{Y|X}\brak{0|1} = \frac{6}{25}\,
p_{Y|X}\brak{1|0} = \frac{45}{50}\,
p_{Y|X}\brak{1|2} = \frac{5}{50}
\end{align}
The desired probability is the probability that a slip drawn at random is marked other than Rs 1,
\begin{align}
&=1-p_X\brak{0}\\
&= p_X(1) + p_X(2)
\end{align}
Using Bayes theorem,
\begin{align}
&= p_Y\brak{0} \times \pr{Y=0 | X=1} + p_Y\brak{1} \times \pr{Y=1|X=2}\\
&=\frac{1}{3} \times \frac{6}{25} + \frac{2}{3} \times \frac{5}{50}\\
&=\frac{11}{75}
\end{align}

\newpage

%\tableofcontents

\bigskip

\renewcommand{\thefigure}{\theenumi}
\renewcommand{\thetable}{\theenumi}
%\renewcommand{\theequation}{\theenumi}

%\begin{abstract}
%%\boldmath
%In this letter, an algorithm for evaluating the exact analytical bit error rate  (BER)  for the piecewise linear (PL) combiner for  multiple relays is presented. Previous results were available only for upto three relays. The algorithm is unique in the sense that  the actual mathematical expressions, that are prohibitively large, need not be explicitly obtained. The diversity gain due to multiple relays is shown through plots of the analytical BER, well supported by simulations. 
%
%\end{abstract}
% IEEEtran.cls defaults to using nonbold math in the Abstract.
% This preserves the distinction between vectors and scalars. However,
% if the journal you are submitting to favors bold math in the abstract,
% then you can use LaTeX's standard command \boldmath at the very start
% of the abstract to achieve this. Many IEEE journals frown on math
% in the abstract anyway.

% Note that keywords are not normally used for peerreview papers.
%\begin{IEEEkeywords}
%Cooperative diversity, decode and forward, piecewise linear
%\end{IEEEkeywords}



% For peer review papers, you can put extra information on the cover
% page as needed:
% \ifCLASSOPTIONpeerreview
% \begin{center} \bfseries EDICS Category: 3-BBND \end{center}
% \fi
%
% For peerreview papers, this IEEEtran command inserts a page break and
% creates the second title. It will be ignored for other modes.
%\IEEEpeerreviewmaketitle




\item One of the four persons John, Rita, Aslam or Gurpreet will be promoted next
month. Consequently the sample space consists of four elementary outcomes
S = {John promoted, Rita promoted, Aslam promoted, Gurpreet promoted}
You are told that the chances of John’s promotion is same as that of Gurpreet,
Rita’s chances of promotion are twice as likely as Johns. Aslam’s chances are
four times that of John.
\begin{enumerate}
	\item Determine
	\begin{enumerate}
		\item P (John promoted)
		\item P (Rita promoted)
		\item P (Aslam promoted)
		\item P (Gurpreet promoted)
	\end{enumerate}
	\item If A = {John promoted or Gurpreet promoted}, find P (A).
\end{enumerate}
\solution
%\begin{table}[H]
	\centering
\begin{tabular}{|c|c|c|}
\hline
Random variable &Value &Definition\\ \hline
\multirow{3}{*}{X} &0 &Slips of Rs 1\\
&1 &Slips of Rs 5\\
&2 &Slips of Rs 13\\ \hline
\multirow{2}{*}{Y} &0 &Box A\\
&1 &Box B\\\hline
\end{tabular}
\caption{}
\label{tab:Distribution}
\end{table}
See \tabref{tab:Distribution}.
\begin{align}
p_{Y}\brak{k}= \begin{cases} 
      \frac{1}{3} & {k=0} \\
      \frac{2}{3 }& {k=1} 
   \end{cases}
   \\
p_{Y|X}\brak{0|0} = \frac{19}{25}\, 
p_{Y|X}\brak{0|1} = \frac{6}{25}\,
p_{Y|X}\brak{1|0} = \frac{45}{50}\,
p_{Y|X}\brak{1|2} = \frac{5}{50}
\end{align}
The desired probability is the probability that a slip drawn at random is marked other than Rs 1,
\begin{align}
&=1-p_X\brak{0}\\
&= p_X(1) + p_X(2)
\end{align}
Using Bayes theorem,
\begin{align}
&= p_Y\brak{0} \times \pr{Y=0 | X=1} + p_Y\brak{1} \times \pr{Y=1|X=2}\\
&=\frac{1}{3} \times \frac{6}{25} + \frac{2}{3} \times \frac{5}{50}\\
&=\frac{11}{75}
\end{align}

\newpage

%\tableofcontents

\bigskip

\renewcommand{\thefigure}{\theenumi}
\renewcommand{\thetable}{\theenumi}
%\renewcommand{\theequation}{\theenumi}

%\begin{abstract}
%%\boldmath
%In this letter, an algorithm for evaluating the exact analytical bit error rate  (BER)  for the piecewise linear (PL) combiner for  multiple relays is presented. Previous results were available only for upto three relays. The algorithm is unique in the sense that  the actual mathematical expressions, that are prohibitively large, need not be explicitly obtained. The diversity gain due to multiple relays is shown through plots of the analytical BER, well supported by simulations. 
%
%\end{abstract}
% IEEEtran.cls defaults to using nonbold math in the Abstract.
% This preserves the distinction between vectors and scalars. However,
% if the journal you are submitting to favors bold math in the abstract,
% then you can use LaTeX's standard command \boldmath at the very start
% of the abstract to achieve this. Many IEEE journals frown on math
% in the abstract anyway.

% Note that keywords are not normally used for peerreview papers.
%\begin{IEEEkeywords}
%Cooperative diversity, decode and forward, piecewise linear
%\end{IEEEkeywords}



% For peer review papers, you can put extra information on the cover
% page as needed:
% \ifCLASSOPTIONpeerreview
% \begin{center} \bfseries EDICS Category: 3-BBND \end{center}
% \fi
%
% For peerreview papers, this IEEEtran command inserts a page break and
% creates the second title. It will be ignored for other modes.
%\IEEEpeerreviewmaketitle




\item A card is drawn from a deck of 52 cards. Find the probability of getting a king or a heart or a red card.\\
\solution
%\begin{table}[H]
	\centering
\begin{tabular}{|c|c|c|}
\hline
Random variable &Value &Definition\\ \hline
\multirow{3}{*}{X} &0 &Slips of Rs 1\\
&1 &Slips of Rs 5\\
&2 &Slips of Rs 13\\ \hline
\multirow{2}{*}{Y} &0 &Box A\\
&1 &Box B\\\hline
\end{tabular}
\caption{}
\label{tab:Distribution}
\end{table}
See \tabref{tab:Distribution}.
\begin{align}
p_{Y}\brak{k}= \begin{cases} 
      \frac{1}{3} & {k=0} \\
      \frac{2}{3 }& {k=1} 
   \end{cases}
   \\
p_{Y|X}\brak{0|0} = \frac{19}{25}\, 
p_{Y|X}\brak{0|1} = \frac{6}{25}\,
p_{Y|X}\brak{1|0} = \frac{45}{50}\,
p_{Y|X}\brak{1|2} = \frac{5}{50}
\end{align}
The desired probability is the probability that a slip drawn at random is marked other than Rs 1,
\begin{align}
&=1-p_X\brak{0}\\
&= p_X(1) + p_X(2)
\end{align}
Using Bayes theorem,
\begin{align}
&= p_Y\brak{0} \times \pr{Y=0 | X=1} + p_Y\brak{1} \times \pr{Y=1|X=2}\\
&=\frac{1}{3} \times \frac{6}{25} + \frac{2}{3} \times \frac{5}{50}\\
&=\frac{11}{75}
\end{align}

\newpage

%\tableofcontents

\bigskip

\renewcommand{\thefigure}{\theenumi}
\renewcommand{\thetable}{\theenumi}
%\renewcommand{\theequation}{\theenumi}

%\begin{abstract}
%%\boldmath
%In this letter, an algorithm for evaluating the exact analytical bit error rate  (BER)  for the piecewise linear (PL) combiner for  multiple relays is presented. Previous results were available only for upto three relays. The algorithm is unique in the sense that  the actual mathematical expressions, that are prohibitively large, need not be explicitly obtained. The diversity gain due to multiple relays is shown through plots of the analytical BER, well supported by simulations. 
%
%\end{abstract}
% IEEEtran.cls defaults to using nonbold math in the Abstract.
% This preserves the distinction between vectors and scalars. However,
% if the journal you are submitting to favors bold math in the abstract,
% then you can use LaTeX's standard command \boldmath at the very start
% of the abstract to achieve this. Many IEEE journals frown on math
% in the abstract anyway.

% Note that keywords are not normally used for peerreview papers.
%\begin{IEEEkeywords}
%Cooperative diversity, decode and forward, piecewise linear
%\end{IEEEkeywords}



% For peer review papers, you can put extra information on the cover
% page as needed:
% \ifCLASSOPTIONpeerreview
% \begin{center} \bfseries EDICS Category: 3-BBND \end{center}
% \fi
%
% For peerreview papers, this IEEEtran command inserts a page break and
% creates the second title. It will be ignored for other modes.
%\IEEEpeerreviewmaketitle




\item The probability that a student will pass his examination is 0.73, the probability of
the student getting a compartment is 0.13, and the probability that the student will
either pass or get compartment is 0.96. State True or False.\\
\solution
%\begin{table}[H]
	\centering
\begin{tabular}{|c|c|c|}
\hline
Random variable &Value &Definition\\ \hline
\multirow{3}{*}{X} &0 &Slips of Rs 1\\
&1 &Slips of Rs 5\\
&2 &Slips of Rs 13\\ \hline
\multirow{2}{*}{Y} &0 &Box A\\
&1 &Box B\\\hline
\end{tabular}
\caption{}
\label{tab:Distribution}
\end{table}
See \tabref{tab:Distribution}.
\begin{align}
p_{Y}\brak{k}= \begin{cases} 
      \frac{1}{3} & {k=0} \\
      \frac{2}{3 }& {k=1} 
   \end{cases}
   \\
p_{Y|X}\brak{0|0} = \frac{19}{25}\, 
p_{Y|X}\brak{0|1} = \frac{6}{25}\,
p_{Y|X}\brak{1|0} = \frac{45}{50}\,
p_{Y|X}\brak{1|2} = \frac{5}{50}
\end{align}
The desired probability is the probability that a slip drawn at random is marked other than Rs 1,
\begin{align}
&=1-p_X\brak{0}\\
&= p_X(1) + p_X(2)
\end{align}
Using Bayes theorem,
\begin{align}
&= p_Y\brak{0} \times \pr{Y=0 | X=1} + p_Y\brak{1} \times \pr{Y=1|X=2}\\
&=\frac{1}{3} \times \frac{6}{25} + \frac{2}{3} \times \frac{5}{50}\\
&=\frac{11}{75}
\end{align}

\newpage

%\tableofcontents

\bigskip

\renewcommand{\thefigure}{\theenumi}
\renewcommand{\thetable}{\theenumi}
%\renewcommand{\theequation}{\theenumi}

%\begin{abstract}
%%\boldmath
%In this letter, an algorithm for evaluating the exact analytical bit error rate  (BER)  for the piecewise linear (PL) combiner for  multiple relays is presented. Previous results were available only for upto three relays. The algorithm is unique in the sense that  the actual mathematical expressions, that are prohibitively large, need not be explicitly obtained. The diversity gain due to multiple relays is shown through plots of the analytical BER, well supported by simulations. 
%
%\end{abstract}
% IEEEtran.cls defaults to using nonbold math in the Abstract.
% This preserves the distinction between vectors and scalars. However,
% if the journal you are submitting to favors bold math in the abstract,
% then you can use LaTeX's standard command \boldmath at the very start
% of the abstract to achieve this. Many IEEE journals frown on math
% in the abstract anyway.

% Note that keywords are not normally used for peerreview papers.
%\begin{IEEEkeywords}
%Cooperative diversity, decode and forward, piecewise linear
%\end{IEEEkeywords}



% For peer review papers, you can put extra information on the cover
% page as needed:
% \ifCLASSOPTIONpeerreview
% \begin{center} \bfseries EDICS Category: 3-BBND \end{center}
% \fi
%
% For peerreview papers, this IEEEtran command inserts a page break and
% creates the second title. It will be ignored for other modes.
%\IEEEpeerreviewmaketitle




\item A card is selected from a pack of 52 cards\\
\begin{enumerate}[label=(\alph*)]
\item How many points are there in the sample space?
\item Calculate the probability that the cards is an ace of spades.
\item Calculate the probability that the card is (i) an ace (ii)black card.\\
\end{enumerate}
%\input{ncert/11/16/3/4_1/Prob_4.tex}
\item In a non-leap year, the probability of having 53 tuesdays or 53 wednesdays is\\
\solution
%A non-leap year has a total of 365 days, and a week has 7 days.\\
So it can be expressed as 
\begin{align}
365\text{days} &=52\times 7+1 \text{day}
\end{align}
$\implies$ 52 tuesdays or wednesdays\\
Random variable X denotes the days of a week
\begin{align}
p_X\brak{k}&=\frac{1}{7}; \quad \brak{1<k<7}
\end{align}
So the probability of extra day being tuesday or wednesday is
\begin{align}
p_X\brak{3}+p_X\brak{4}&=\frac{1}{7}+\frac{1}{7}=\frac{2}{7}
\end{align}



\item There are 1000 sealed envelopes in a box, 10 of them contain a cash prize of
Rs 100 each, 100 of them contain a cash prize of Rs 50 each and 200 of them
contain a cash prize of Rs 10 each and rest do not contain any cash prize. If they
are well shuffled and an envelope is picked up out, what is the probability that it
contains no cash prize?\\
\solution
%\begin{table}[H]
	\centering
\begin{tabular}{|c|c|c|}
\hline
Random variable &Value &Definition\\ \hline
\multirow{3}{*}{X} &0 &Slips of Rs 1\\
&1 &Slips of Rs 5\\
&2 &Slips of Rs 13\\ \hline
\multirow{2}{*}{Y} &0 &Box A\\
&1 &Box B\\\hline
\end{tabular}
\caption{}
\label{tab:Distribution}
\end{table}
See \tabref{tab:Distribution}.
\begin{align}
p_{Y}\brak{k}= \begin{cases} 
      \frac{1}{3} & {k=0} \\
      \frac{2}{3 }& {k=1} 
   \end{cases}
   \\
p_{Y|X}\brak{0|0} = \frac{19}{25}\, 
p_{Y|X}\brak{0|1} = \frac{6}{25}\,
p_{Y|X}\brak{1|0} = \frac{45}{50}\,
p_{Y|X}\brak{1|2} = \frac{5}{50}
\end{align}
The desired probability is the probability that a slip drawn at random is marked other than Rs 1,
\begin{align}
&=1-p_X\brak{0}\\
&= p_X(1) + p_X(2)
\end{align}
Using Bayes theorem,
\begin{align}
&= p_Y\brak{0} \times \pr{Y=0 | X=1} + p_Y\brak{1} \times \pr{Y=1|X=2}\\
&=\frac{1}{3} \times \frac{6}{25} + \frac{2}{3} \times \frac{5}{50}\\
&=\frac{11}{75}
\end{align}

\newpage

%\tableofcontents

\bigskip

\renewcommand{\thefigure}{\theenumi}
\renewcommand{\thetable}{\theenumi}
%\renewcommand{\theequation}{\theenumi}

%\begin{abstract}
%%\boldmath
%In this letter, an algorithm for evaluating the exact analytical bit error rate  (BER)  for the piecewise linear (PL) combiner for  multiple relays is presented. Previous results were available only for upto three relays. The algorithm is unique in the sense that  the actual mathematical expressions, that are prohibitively large, need not be explicitly obtained. The diversity gain due to multiple relays is shown through plots of the analytical BER, well supported by simulations. 
%
%\end{abstract}
% IEEEtran.cls defaults to using nonbold math in the Abstract.
% This preserves the distinction between vectors and scalars. However,
% if the journal you are submitting to favors bold math in the abstract,
% then you can use LaTeX's standard command \boldmath at the very start
% of the abstract to achieve this. Many IEEE journals frown on math
% in the abstract anyway.

% Note that keywords are not normally used for peerreview papers.
%\begin{IEEEkeywords}
%Cooperative diversity, decode and forward, piecewise linear
%\end{IEEEkeywords}



% For peer review papers, you can put extra information on the cover
% page as needed:
% \ifCLASSOPTIONpeerreview
% \begin{center} \bfseries EDICS Category: 3-BBND \end{center}
% \fi
%
% For peerreview papers, this IEEEtran command inserts a page break and
% creates the second title. It will be ignored for other modes.
%\IEEEpeerreviewmaketitle




\item 
A die is thrown and a card is selected at random from a deck of 52 playing cards. The probability of getting an even number on the die and a spade card.\\
\solution
%\begin{table}[H]
	\centering
\begin{tabular}{|c|c|c|}
\hline
Random variable &Value &Definition\\ \hline
\multirow{3}{*}{X} &0 &Slips of Rs 1\\
&1 &Slips of Rs 5\\
&2 &Slips of Rs 13\\ \hline
\multirow{2}{*}{Y} &0 &Box A\\
&1 &Box B\\\hline
\end{tabular}
\caption{}
\label{tab:Distribution}
\end{table}
See \tabref{tab:Distribution}.
\begin{align}
p_{Y}\brak{k}= \begin{cases} 
      \frac{1}{3} & {k=0} \\
      \frac{2}{3 }& {k=1} 
   \end{cases}
   \\
p_{Y|X}\brak{0|0} = \frac{19}{25}\, 
p_{Y|X}\brak{0|1} = \frac{6}{25}\,
p_{Y|X}\brak{1|0} = \frac{45}{50}\,
p_{Y|X}\brak{1|2} = \frac{5}{50}
\end{align}
The desired probability is the probability that a slip drawn at random is marked other than Rs 1,
\begin{align}
&=1-p_X\brak{0}\\
&= p_X(1) + p_X(2)
\end{align}
Using Bayes theorem,
\begin{align}
&= p_Y\brak{0} \times \pr{Y=0 | X=1} + p_Y\brak{1} \times \pr{Y=1|X=2}\\
&=\frac{1}{3} \times \frac{6}{25} + \frac{2}{3} \times \frac{5}{50}\\
&=\frac{11}{75}
\end{align}

\newpage

%\tableofcontents

\bigskip

\renewcommand{\thefigure}{\theenumi}
\renewcommand{\thetable}{\theenumi}
%\renewcommand{\theequation}{\theenumi}

%\begin{abstract}
%%\boldmath
%In this letter, an algorithm for evaluating the exact analytical bit error rate  (BER)  for the piecewise linear (PL) combiner for  multiple relays is presented. Previous results were available only for upto three relays. The algorithm is unique in the sense that  the actual mathematical expressions, that are prohibitively large, need not be explicitly obtained. The diversity gain due to multiple relays is shown through plots of the analytical BER, well supported by simulations. 
%
%\end{abstract}
% IEEEtran.cls defaults to using nonbold math in the Abstract.
% This preserves the distinction between vectors and scalars. However,
% if the journal you are submitting to favors bold math in the abstract,
% then you can use LaTeX's standard command \boldmath at the very start
% of the abstract to achieve this. Many IEEE journals frown on math
% in the abstract anyway.

% Note that keywords are not normally used for peerreview papers.
%\begin{IEEEkeywords}
%Cooperative diversity, decode and forward, piecewise linear
%\end{IEEEkeywords}



% For peer review papers, you can put extra information on the cover
% page as needed:
% \ifCLASSOPTIONpeerreview
% \begin{center} \bfseries EDICS Category: 3-BBND \end{center}
% \fi
%
% For peerreview papers, this IEEEtran command inserts a page break and
% creates the second title. It will be ignored for other modes.
%\IEEEpeerreviewmaketitle




\item
If 4-digit numbers greater than 5,000 are randomly formed from the digits 0, 1, 3, 5, and 7, what is the probability of forming a number divisible by 5 when:
\begin{enumerate}
    \item The digits are repeated?
    \item The repetition of digits is not allowed?
\end{enumerate}
\solution
%\begin{table}[H]
	\centering
\begin{tabular}{|c|c|c|}
\hline
Random variable &Value &Definition\\ \hline
\multirow{3}{*}{X} &0 &Slips of Rs 1\\
&1 &Slips of Rs 5\\
&2 &Slips of Rs 13\\ \hline
\multirow{2}{*}{Y} &0 &Box A\\
&1 &Box B\\\hline
\end{tabular}
\caption{}
\label{tab:Distribution}
\end{table}
See \tabref{tab:Distribution}.
\begin{align}
p_{Y}\brak{k}= \begin{cases} 
      \frac{1}{3} & {k=0} \\
      \frac{2}{3 }& {k=1} 
   \end{cases}
   \\
p_{Y|X}\brak{0|0} = \frac{19}{25}\, 
p_{Y|X}\brak{0|1} = \frac{6}{25}\,
p_{Y|X}\brak{1|0} = \frac{45}{50}\,
p_{Y|X}\brak{1|2} = \frac{5}{50}
\end{align}
The desired probability is the probability that a slip drawn at random is marked other than Rs 1,
\begin{align}
&=1-p_X\brak{0}\\
&= p_X(1) + p_X(2)
\end{align}
Using Bayes theorem,
\begin{align}
&= p_Y\brak{0} \times \pr{Y=0 | X=1} + p_Y\brak{1} \times \pr{Y=1|X=2}\\
&=\frac{1}{3} \times \frac{6}{25} + \frac{2}{3} \times \frac{5}{50}\\
&=\frac{11}{75}
\end{align}

\newpage

%\tableofcontents

\bigskip

\renewcommand{\thefigure}{\theenumi}
\renewcommand{\thetable}{\theenumi}
%\renewcommand{\theequation}{\theenumi}

%\begin{abstract}
%%\boldmath
%In this letter, an algorithm for evaluating the exact analytical bit error rate  (BER)  for the piecewise linear (PL) combiner for  multiple relays is presented. Previous results were available only for upto three relays. The algorithm is unique in the sense that  the actual mathematical expressions, that are prohibitively large, need not be explicitly obtained. The diversity gain due to multiple relays is shown through plots of the analytical BER, well supported by simulations. 
%
%\end{abstract}
% IEEEtran.cls defaults to using nonbold math in the Abstract.
% This preserves the distinction between vectors and scalars. However,
% if the journal you are submitting to favors bold math in the abstract,
% then you can use LaTeX's standard command \boldmath at the very start
% of the abstract to achieve this. Many IEEE journals frown on math
% in the abstract anyway.

% Note that keywords are not normally used for peerreview papers.
%\begin{IEEEkeywords}
%Cooperative diversity, decode and forward, piecewise linear
%\end{IEEEkeywords}



% For peer review papers, you can put extra information on the cover
% page as needed:
% \ifCLASSOPTIONpeerreview
% \begin{center} \bfseries EDICS Category: 3-BBND \end{center}
% \fi
%
% For peerreview papers, this IEEEtran command inserts a page break and
% creates the second title. It will be ignored for other modes.
%\IEEEpeerreviewmaketitle




\item Consider the probability space $\brak{\Omega, \mathcal{G}, P}$ where $\Omega = [0,2]$ and $\mathcal{G} = \cbrak{\phi, \Omega, [0,1], (1,2]}$. Let $X$ and $Y$ be two functions on $\Omega$ defined as
\begin{align*}
    X(\omega) = 
    \begin{cases}
        1 & \text{if }\omega \in [0, 1]\\
        2 & \text{if }\omega \in (1, 2]
    \end{cases}
\end{align*}
and
\begin{align*}
    Y(\omega) = 
    \begin{cases}
        2 & \text{if }\omega \in [0, 1.5]\\
        3 & \text{if }\omega \in (1.5, 2].
    \end{cases}
\end{align*}
Then which one of the following statements is true?
\begin{enumerate}
    \item [(A)] $X$ is a random variable with respect to $\mathcal{G}$, but $Y$ is not a random variable with respect to $\mathcal{G}$.
    \item [(B)] $Y$ is a random variable with respect to $\mathcal{G}$, but $X$ is not a random variable with respect to $\mathcal{G}$.
    \item [(C)] Neither $X$ nor $Y$ is a random variable with respect to $\mathcal{G}$.
    \item [(D)] Both $X$ and $Y$ are random variables with respect to $\mathcal{G}$.
\end{enumerate} \hfill (GATE ST 2023)\\
\solution
%\begin{table}[H]
	\centering
\begin{tabular}{|c|c|c|}
\hline
Random variable &Value &Definition\\ \hline
\multirow{3}{*}{X} &0 &Slips of Rs 1\\
&1 &Slips of Rs 5\\
&2 &Slips of Rs 13\\ \hline
\multirow{2}{*}{Y} &0 &Box A\\
&1 &Box B\\\hline
\end{tabular}
\caption{}
\label{tab:Distribution}
\end{table}
See \tabref{tab:Distribution}.
\begin{align}
p_{Y}\brak{k}= \begin{cases} 
      \frac{1}{3} & {k=0} \\
      \frac{2}{3 }& {k=1} 
   \end{cases}
   \\
p_{Y|X}\brak{0|0} = \frac{19}{25}\, 
p_{Y|X}\brak{0|1} = \frac{6}{25}\,
p_{Y|X}\brak{1|0} = \frac{45}{50}\,
p_{Y|X}\brak{1|2} = \frac{5}{50}
\end{align}
The desired probability is the probability that a slip drawn at random is marked other than Rs 1,
\begin{align}
&=1-p_X\brak{0}\\
&= p_X(1) + p_X(2)
\end{align}
Using Bayes theorem,
\begin{align}
&= p_Y\brak{0} \times \pr{Y=0 | X=1} + p_Y\brak{1} \times \pr{Y=1|X=2}\\
&=\frac{1}{3} \times \frac{6}{25} + \frac{2}{3} \times \frac{5}{50}\\
&=\frac{11}{75}
\end{align}

\newpage

%\tableofcontents

\bigskip

\renewcommand{\thefigure}{\theenumi}
\renewcommand{\thetable}{\theenumi}
%\renewcommand{\theequation}{\theenumi}

%\begin{abstract}
%%\boldmath
%In this letter, an algorithm for evaluating the exact analytical bit error rate  (BER)  for the piecewise linear (PL) combiner for  multiple relays is presented. Previous results were available only for upto three relays. The algorithm is unique in the sense that  the actual mathematical expressions, that are prohibitively large, need not be explicitly obtained. The diversity gain due to multiple relays is shown through plots of the analytical BER, well supported by simulations. 
%
%\end{abstract}
% IEEEtran.cls defaults to using nonbold math in the Abstract.
% This preserves the distinction between vectors and scalars. However,
% if the journal you are submitting to favors bold math in the abstract,
% then you can use LaTeX's standard command \boldmath at the very start
% of the abstract to achieve this. Many IEEE journals frown on math
% in the abstract anyway.

% Note that keywords are not normally used for peerreview papers.
%\begin{IEEEkeywords}
%Cooperative diversity, decode and forward, piecewise linear
%\end{IEEEkeywords}



% For peer review papers, you can put extra information on the cover
% page as needed:
% \ifCLASSOPTIONpeerreview
% \begin{center} \bfseries EDICS Category: 3-BBND \end{center}
% \fi
%
% For peerreview papers, this IEEEtran command inserts a page break and
% creates the second title. It will be ignored for other modes.
%\IEEEpeerreviewmaketitle




	\item  A die is loaded in such a way that each odd number is twice as likely to occur as
each even number. Find $P(G)$, where $G$ is the event that a number greater than
3 occurs on a single roll of the die.
\\
\solution
		%\begin{table}[H]
	\centering
\begin{tabular}{|c|c|c|}
\hline
Random variable &Value &Definition\\ \hline
\multirow{3}{*}{X} &0 &Slips of Rs 1\\
&1 &Slips of Rs 5\\
&2 &Slips of Rs 13\\ \hline
\multirow{2}{*}{Y} &0 &Box A\\
&1 &Box B\\\hline
\end{tabular}
\caption{}
\label{tab:Distribution}
\end{table}
See \tabref{tab:Distribution}.
\begin{align}
p_{Y}\brak{k}= \begin{cases} 
      \frac{1}{3} & {k=0} \\
      \frac{2}{3 }& {k=1} 
   \end{cases}
   \\
p_{Y|X}\brak{0|0} = \frac{19}{25}\, 
p_{Y|X}\brak{0|1} = \frac{6}{25}\,
p_{Y|X}\brak{1|0} = \frac{45}{50}\,
p_{Y|X}\brak{1|2} = \frac{5}{50}
\end{align}
The desired probability is the probability that a slip drawn at random is marked other than Rs 1,
\begin{align}
&=1-p_X\brak{0}\\
&= p_X(1) + p_X(2)
\end{align}
Using Bayes theorem,
\begin{align}
&= p_Y\brak{0} \times \pr{Y=0 | X=1} + p_Y\brak{1} \times \pr{Y=1|X=2}\\
&=\frac{1}{3} \times \frac{6}{25} + \frac{2}{3} \times \frac{5}{50}\\
&=\frac{11}{75}
\end{align}

\newpage

%\tableofcontents

\bigskip

\renewcommand{\thefigure}{\theenumi}
\renewcommand{\thetable}{\theenumi}
%\renewcommand{\theequation}{\theenumi}

%\begin{abstract}
%%\boldmath
%In this letter, an algorithm for evaluating the exact analytical bit error rate  (BER)  for the piecewise linear (PL) combiner for  multiple relays is presented. Previous results were available only for upto three relays. The algorithm is unique in the sense that  the actual mathematical expressions, that are prohibitively large, need not be explicitly obtained. The diversity gain due to multiple relays is shown through plots of the analytical BER, well supported by simulations. 
%
%\end{abstract}
% IEEEtran.cls defaults to using nonbold math in the Abstract.
% This preserves the distinction between vectors and scalars. However,
% if the journal you are submitting to favors bold math in the abstract,
% then you can use LaTeX's standard command \boldmath at the very start
% of the abstract to achieve this. Many IEEE journals frown on math
% in the abstract anyway.

% Note that keywords are not normally used for peerreview papers.
%\begin{IEEEkeywords}
%Cooperative diversity, decode and forward, piecewise linear
%\end{IEEEkeywords}



% For peer review papers, you can put extra information on the cover
% page as needed:
% \ifCLASSOPTIONpeerreview
% \begin{center} \bfseries EDICS Category: 3-BBND \end{center}
% \fi
%
% For peerreview papers, this IEEEtran command inserts a page break and
% creates the second title. It will be ignored for other modes.
%\IEEEpeerreviewmaketitle




	\item All the jacks, queens and kings are removed from a deck of 52 playing cards. The remaining cards are well shuffled and then one card is drawn at random. Giving ace a value 1 similar value for other cards, find the probability that the card has a value 
		\begin{enumerate}
			\item 7
			\item greater than 7
			\item less than 7
		\end{enumerate}
		%Number of cards left after removing all jacks, queens and kings 
\begin{align}
N	= 52 - 4\times 3
	= 40
\end{align}
%\begin{table}[H]
%\def\arraystretch{1.2}
%\begin{tabular}{|c|c|c|}
%\hline
%	\textbf{Parameter} &\textbf{Value} &\textbf{Description}\\ \hline
%	$X$ &1-10 &Represents the value of the card picked \\ \hline
%\end{tabular}
%\end{table}
Let $1 \le X \le 10$ be the value of the card picked.  Then,
\begin{align}
	p_X(k) &= \Pr(X=k)\ \forall\ 1 \leq k \leq 10\\
	&= \frac{4\times 1}{40}\\
	&= \frac{1}{10}\\
	\therefore p_X(k) &= 
	\begin{cases}
		\frac{1}{10} & 1 \leq k \leq 10\\
		0 & \text{otherwise}
	\end{cases}
\end{align}
and
\begin{align}
	F_{X}(k) &= \sum_{m=0}^{k}p_{X}(m) \quad 1 \leq k \leq 10\\
	&= \frac{k}{10}\\
	\therefore F_{X}(k) &= 
	\begin{cases}
		0 & k \leq 0\\
		\frac{k}{10} & 1\leq k \leq 10\\
		1 & k > 10 
	\end{cases}
\end{align}
\begin{enumerate}
	\item Probability that card has value equal to 7 is
		\begin{align}
			 p_{X}(7)
			= \frac{1}{10}
		\end{align}
	\item Probability that card has value greater than 7 is
		\begin{align}
			1 - F_X(7)
			&= 1 - \frac{7}{10}
			\\
			&= \frac{3}{10}
		\end{align}
	\item Probability that card has value less than 7 is
		\begin{align}
			 F_{X}(6)
			=\frac{6}{10}
		\end{align}
\end{enumerate}

  \item A Lot consists of 48 mobile phones of which 42 are good, 3 have only minor defects and 3 have major defects.Varnika will buy a phone if it is good but the trader will only buy a mobile if it has no major defects. One phone is selected at random from the lot. What is the probability that it is
\begin{enumerate}
	\item acceptable to Varnika?
            \item acceptable to the trader?
\end{enumerate}
\solution
	%\begin{table}[H]
	\centering
\begin{tabular}{|c|c|c|}
\hline
Random variable &Value &Definition\\ \hline
\multirow{3}{*}{X} &0 &Slips of Rs 1\\
&1 &Slips of Rs 5\\
&2 &Slips of Rs 13\\ \hline
\multirow{2}{*}{Y} &0 &Box A\\
&1 &Box B\\\hline
\end{tabular}
\caption{}
\label{tab:Distribution}
\end{table}
See \tabref{tab:Distribution}.
\begin{align}
p_{Y}\brak{k}= \begin{cases} 
      \frac{1}{3} & {k=0} \\
      \frac{2}{3 }& {k=1} 
   \end{cases}
   \\
p_{Y|X}\brak{0|0} = \frac{19}{25}\, 
p_{Y|X}\brak{0|1} = \frac{6}{25}\,
p_{Y|X}\brak{1|0} = \frac{45}{50}\,
p_{Y|X}\brak{1|2} = \frac{5}{50}
\end{align}
The desired probability is the probability that a slip drawn at random is marked other than Rs 1,
\begin{align}
&=1-p_X\brak{0}\\
&= p_X(1) + p_X(2)
\end{align}
Using Bayes theorem,
\begin{align}
&= p_Y\brak{0} \times \pr{Y=0 | X=1} + p_Y\brak{1} \times \pr{Y=1|X=2}\\
&=\frac{1}{3} \times \frac{6}{25} + \frac{2}{3} \times \frac{5}{50}\\
&=\frac{11}{75}
\end{align}

\newpage

%\tableofcontents

\bigskip

\renewcommand{\thefigure}{\theenumi}
\renewcommand{\thetable}{\theenumi}
%\renewcommand{\theequation}{\theenumi}

%\begin{abstract}
%%\boldmath
%In this letter, an algorithm for evaluating the exact analytical bit error rate  (BER)  for the piecewise linear (PL) combiner for  multiple relays is presented. Previous results were available only for upto three relays. The algorithm is unique in the sense that  the actual mathematical expressions, that are prohibitively large, need not be explicitly obtained. The diversity gain due to multiple relays is shown through plots of the analytical BER, well supported by simulations. 
%
%\end{abstract}
% IEEEtran.cls defaults to using nonbold math in the Abstract.
% This preserves the distinction between vectors and scalars. However,
% if the journal you are submitting to favors bold math in the abstract,
% then you can use LaTeX's standard command \boldmath at the very start
% of the abstract to achieve this. Many IEEE journals frown on math
% in the abstract anyway.

% Note that keywords are not normally used for peerreview papers.
%\begin{IEEEkeywords}
%Cooperative diversity, decode and forward, piecewise linear
%\end{IEEEkeywords}



% For peer review papers, you can put extra information on the cover
% page as needed:
% \ifCLASSOPTIONpeerreview
% \begin{center} \bfseries EDICS Category: 3-BBND \end{center}
% \fi
%
% For peerreview papers, this IEEEtran command inserts a page break and
% creates the second title. It will be ignored for other modes.
%\IEEEpeerreviewmaketitle




 \item A student says that if you throw a die, it will show up 1 or not 1. Therefore, the probability of getting 1 and the probability of getting 'not 1' each is equal to $\frac{1}{2}$. Is this correct? Give reasons.\\
 \solution
        %\begin{table}[H]
	\centering
\begin{tabular}{|c|c|c|}
\hline
Random variable &Value &Definition\\ \hline
\multirow{3}{*}{X} &0 &Slips of Rs 1\\
&1 &Slips of Rs 5\\
&2 &Slips of Rs 13\\ \hline
\multirow{2}{*}{Y} &0 &Box A\\
&1 &Box B\\\hline
\end{tabular}
\caption{}
\label{tab:Distribution}
\end{table}
See \tabref{tab:Distribution}.
\begin{align}
p_{Y}\brak{k}= \begin{cases} 
      \frac{1}{3} & {k=0} \\
      \frac{2}{3 }& {k=1} 
   \end{cases}
   \\
p_{Y|X}\brak{0|0} = \frac{19}{25}\, 
p_{Y|X}\brak{0|1} = \frac{6}{25}\,
p_{Y|X}\brak{1|0} = \frac{45}{50}\,
p_{Y|X}\brak{1|2} = \frac{5}{50}
\end{align}
The desired probability is the probability that a slip drawn at random is marked other than Rs 1,
\begin{align}
&=1-p_X\brak{0}\\
&= p_X(1) + p_X(2)
\end{align}
Using Bayes theorem,
\begin{align}
&= p_Y\brak{0} \times \pr{Y=0 | X=1} + p_Y\brak{1} \times \pr{Y=1|X=2}\\
&=\frac{1}{3} \times \frac{6}{25} + \frac{2}{3} \times \frac{5}{50}\\
&=\frac{11}{75}
\end{align}

\newpage

%\tableofcontents

\bigskip

\renewcommand{\thefigure}{\theenumi}
\renewcommand{\thetable}{\theenumi}
%\renewcommand{\theequation}{\theenumi}

%\begin{abstract}
%%\boldmath
%In this letter, an algorithm for evaluating the exact analytical bit error rate  (BER)  for the piecewise linear (PL) combiner for  multiple relays is presented. Previous results were available only for upto three relays. The algorithm is unique in the sense that  the actual mathematical expressions, that are prohibitively large, need not be explicitly obtained. The diversity gain due to multiple relays is shown through plots of the analytical BER, well supported by simulations. 
%
%\end{abstract}
% IEEEtran.cls defaults to using nonbold math in the Abstract.
% This preserves the distinction between vectors and scalars. However,
% if the journal you are submitting to favors bold math in the abstract,
% then you can use LaTeX's standard command \boldmath at the very start
% of the abstract to achieve this. Many IEEE journals frown on math
% in the abstract anyway.

% Note that keywords are not normally used for peerreview papers.
%\begin{IEEEkeywords}
%Cooperative diversity, decode and forward, piecewise linear
%\end{IEEEkeywords}



% For peer review papers, you can put extra information on the cover
% page as needed:
% \ifCLASSOPTIONpeerreview
% \begin{center} \bfseries EDICS Category: 3-BBND \end{center}
% \fi
%
% For peerreview papers, this IEEEtran command inserts a page break and
% creates the second title. It will be ignored for other modes.
%\IEEEpeerreviewmaketitle




   \item Four candidates A, B, C, D have ap-
plied for the assignment to coach a school cricket
team. If A is twice as likely to be selected as B, and
B and C are given about the same chance of being
selected, while C is twice as likely to be selected
as D, what are the probabilities that
\begin{enumerate}
\item C will be selected?
\item A will not be selected?
\end{enumerate}
	%\begin{table}[H]
	\centering
\begin{tabular}{|c|c|c|}
\hline
Random variable &Value &Definition\\ \hline
\multirow{3}{*}{X} &0 &Slips of Rs 1\\
&1 &Slips of Rs 5\\
&2 &Slips of Rs 13\\ \hline
\multirow{2}{*}{Y} &0 &Box A\\
&1 &Box B\\\hline
\end{tabular}
\caption{}
\label{tab:Distribution}
\end{table}
See \tabref{tab:Distribution}.
\begin{align}
p_{Y}\brak{k}= \begin{cases} 
      \frac{1}{3} & {k=0} \\
      \frac{2}{3 }& {k=1} 
   \end{cases}
   \\
p_{Y|X}\brak{0|0} = \frac{19}{25}\, 
p_{Y|X}\brak{0|1} = \frac{6}{25}\,
p_{Y|X}\brak{1|0} = \frac{45}{50}\,
p_{Y|X}\brak{1|2} = \frac{5}{50}
\end{align}
The desired probability is the probability that a slip drawn at random is marked other than Rs 1,
\begin{align}
&=1-p_X\brak{0}\\
&= p_X(1) + p_X(2)
\end{align}
Using Bayes theorem,
\begin{align}
&= p_Y\brak{0} \times \pr{Y=0 | X=1} + p_Y\brak{1} \times \pr{Y=1|X=2}\\
&=\frac{1}{3} \times \frac{6}{25} + \frac{2}{3} \times \frac{5}{50}\\
&=\frac{11}{75}
\end{align}

\newpage

%\tableofcontents

\bigskip

\renewcommand{\thefigure}{\theenumi}
\renewcommand{\thetable}{\theenumi}
%\renewcommand{\theequation}{\theenumi}

%\begin{abstract}
%%\boldmath
%In this letter, an algorithm for evaluating the exact analytical bit error rate  (BER)  for the piecewise linear (PL) combiner for  multiple relays is presented. Previous results were available only for upto three relays. The algorithm is unique in the sense that  the actual mathematical expressions, that are prohibitively large, need not be explicitly obtained. The diversity gain due to multiple relays is shown through plots of the analytical BER, well supported by simulations. 
%
%\end{abstract}
% IEEEtran.cls defaults to using nonbold math in the Abstract.
% This preserves the distinction between vectors and scalars. However,
% if the journal you are submitting to favors bold math in the abstract,
% then you can use LaTeX's standard command \boldmath at the very start
% of the abstract to achieve this. Many IEEE journals frown on math
% in the abstract anyway.

% Note that keywords are not normally used for peerreview papers.
%\begin{IEEEkeywords}
%Cooperative diversity, decode and forward, piecewise linear
%\end{IEEEkeywords}



% For peer review papers, you can put extra information on the cover
% page as needed:
% \ifCLASSOPTIONpeerreview
% \begin{center} \bfseries EDICS Category: 3-BBND \end{center}
% \fi
%
% For peerreview papers, this IEEEtran command inserts a page break and
% creates the second title. It will be ignored for other modes.
%\IEEEpeerreviewmaketitle




 \item A bag contain 24 balls of which $x$ balls are red, $2x$ are white and $3x$ are blue. A ball is selected at random, What is the probability that it is
\begin{enumerate}[label=\alph*)]
\item not red ?
\item white ?
\end{enumerate}
%\begin{table}[H]
	\centering
\begin{tabular}{|c|c|c|}
\hline
Random variable &Value &Definition\\ \hline
\multirow{3}{*}{X} &0 &Slips of Rs 1\\
&1 &Slips of Rs 5\\
&2 &Slips of Rs 13\\ \hline
\multirow{2}{*}{Y} &0 &Box A\\
&1 &Box B\\\hline
\end{tabular}
\caption{}
\label{tab:Distribution}
\end{table}
See \tabref{tab:Distribution}.
\begin{align}
p_{Y}\brak{k}= \begin{cases} 
      \frac{1}{3} & {k=0} \\
      \frac{2}{3 }& {k=1} 
   \end{cases}
   \\
p_{Y|X}\brak{0|0} = \frac{19}{25}\, 
p_{Y|X}\brak{0|1} = \frac{6}{25}\,
p_{Y|X}\brak{1|0} = \frac{45}{50}\,
p_{Y|X}\brak{1|2} = \frac{5}{50}
\end{align}
The desired probability is the probability that a slip drawn at random is marked other than Rs 1,
\begin{align}
&=1-p_X\brak{0}\\
&= p_X(1) + p_X(2)
\end{align}
Using Bayes theorem,
\begin{align}
&= p_Y\brak{0} \times \pr{Y=0 | X=1} + p_Y\brak{1} \times \pr{Y=1|X=2}\\
&=\frac{1}{3} \times \frac{6}{25} + \frac{2}{3} \times \frac{5}{50}\\
&=\frac{11}{75}
\end{align}

\newpage

%\tableofcontents

\bigskip

\renewcommand{\thefigure}{\theenumi}
\renewcommand{\thetable}{\theenumi}
%\renewcommand{\theequation}{\theenumi}

%\begin{abstract}
%%\boldmath
%In this letter, an algorithm for evaluating the exact analytical bit error rate  (BER)  for the piecewise linear (PL) combiner for  multiple relays is presented. Previous results were available only for upto three relays. The algorithm is unique in the sense that  the actual mathematical expressions, that are prohibitively large, need not be explicitly obtained. The diversity gain due to multiple relays is shown through plots of the analytical BER, well supported by simulations. 
%
%\end{abstract}
% IEEEtran.cls defaults to using nonbold math in the Abstract.
% This preserves the distinction between vectors and scalars. However,
% if the journal you are submitting to favors bold math in the abstract,
% then you can use LaTeX's standard command \boldmath at the very start
% of the abstract to achieve this. Many IEEE journals frown on math
% in the abstract anyway.

% Note that keywords are not normally used for peerreview papers.
%\begin{IEEEkeywords}
%Cooperative diversity, decode and forward, piecewise linear
%\end{IEEEkeywords}



% For peer review papers, you can put extra information on the cover
% page as needed:
% \ifCLASSOPTIONpeerreview
% \begin{center} \bfseries EDICS Category: 3-BBND \end{center}
% \fi
%
% For peerreview papers, this IEEEtran command inserts a page break and
% creates the second title. It will be ignored for other modes.
%\IEEEpeerreviewmaketitle




If the letters of the word ASSASSINATION are arranged at random. Find the Probability that
\begin{enumerate}[label=(\alph*)]
\item Four $S's$ come consecutively in the word
\item Two  $I's$ and two $N's$ come together
\item All $A's$ are not coming together
\item No two $A's$ are coming together
\end{enumerate}
%\begin{table}[H]
	\centering
\begin{tabular}{|c|c|c|}
\hline
Random variable &Value &Definition\\ \hline
\multirow{3}{*}{X} &0 &Slips of Rs 1\\
&1 &Slips of Rs 5\\
&2 &Slips of Rs 13\\ \hline
\multirow{2}{*}{Y} &0 &Box A\\
&1 &Box B\\\hline
\end{tabular}
\caption{}
\label{tab:Distribution}
\end{table}
See \tabref{tab:Distribution}.
\begin{align}
p_{Y}\brak{k}= \begin{cases} 
      \frac{1}{3} & {k=0} \\
      \frac{2}{3 }& {k=1} 
   \end{cases}
   \\
p_{Y|X}\brak{0|0} = \frac{19}{25}\, 
p_{Y|X}\brak{0|1} = \frac{6}{25}\,
p_{Y|X}\brak{1|0} = \frac{45}{50}\,
p_{Y|X}\brak{1|2} = \frac{5}{50}
\end{align}
The desired probability is the probability that a slip drawn at random is marked other than Rs 1,
\begin{align}
&=1-p_X\brak{0}\\
&= p_X(1) + p_X(2)
\end{align}
Using Bayes theorem,
\begin{align}
&= p_Y\brak{0} \times \pr{Y=0 | X=1} + p_Y\brak{1} \times \pr{Y=1|X=2}\\
&=\frac{1}{3} \times \frac{6}{25} + \frac{2}{3} \times \frac{5}{50}\\
&=\frac{11}{75}
\end{align}

\newpage

%\tableofcontents

\bigskip

\renewcommand{\thefigure}{\theenumi}
\renewcommand{\thetable}{\theenumi}
%\renewcommand{\theequation}{\theenumi}

%\begin{abstract}
%%\boldmath
%In this letter, an algorithm for evaluating the exact analytical bit error rate  (BER)  for the piecewise linear (PL) combiner for  multiple relays is presented. Previous results were available only for upto three relays. The algorithm is unique in the sense that  the actual mathematical expressions, that are prohibitively large, need not be explicitly obtained. The diversity gain due to multiple relays is shown through plots of the analytical BER, well supported by simulations. 
%
%\end{abstract}
% IEEEtran.cls defaults to using nonbold math in the Abstract.
% This preserves the distinction between vectors and scalars. However,
% if the journal you are submitting to favors bold math in the abstract,
% then you can use LaTeX's standard command \boldmath at the very start
% of the abstract to achieve this. Many IEEE journals frown on math
% in the abstract anyway.

% Note that keywords are not normally used for peerreview papers.
%\begin{IEEEkeywords}
%Cooperative diversity, decode and forward, piecewise linear
%\end{IEEEkeywords}



% For peer review papers, you can put extra information on the cover
% page as needed:
% \ifCLASSOPTIONpeerreview
% \begin{center} \bfseries EDICS Category: 3-BBND \end{center}
% \fi
%
% For peerreview papers, this IEEEtran command inserts a page break and
% creates the second title. It will be ignored for other modes.
%\IEEEpeerreviewmaketitle




	\item One urn contains two black balls (labelled B1 and B2) and one white ball. A
	second urn contains one black ball and two white balls (labelled W1 and W2).
	Suppose the following experiment is performed. One of the two urns is chosen
	at random. Next a ball is randomly chosen from the urn. Then a second ball is
	chosen at random from the same urn without replacing the first ball.
	
	\begin{enumerate}
	\item What is the probability that two black balls are chosen?
	
	\item What is the probability that two balls of opposite colour are chosen?
	\end{enumerate}
	\solution
	%\begin{align}
    \label{eq:12.13.6.18.1}
	\because	\pr{A|B} &> \pr{A},\
\frac{\pr{AB}}{\pr{B}} > \pr{A}
\\
    \label{eq:12.13.6.18.2}
	\implies \pr{AB} &> \pr{A}\pr{B}
	\\
	\text{or, } \frac{\pr{AB}}{\pr{A}} &=\pr{B|A} > \pr{A}
\end{align}

\end{enumerate}

		%
\item 
Out of 100 students, two sections of 40 and 60 are formed. If you and your friend are among the 100 students, what is the probability that
\begin{enumerate}
\item you both enter the same section?
\item you both enter the different sections?
\end{enumerate}
\solution
		%\begin{enumerate}[label=\thesection.\arabic*,ref=\thesection.\theenumi]
	\item One card is drawn from a well-shuffled deck of 52 cards. Find the probability of getting
\begin{enumerate}
\item A king of red colour 
\item A face card 
\item A red face card
\item The jack of hearts
\item A spade
\item The queen of diamonds

\end{enumerate}
\solution
		%\begin{table}[H]
	\centering
\begin{tabular}{|c|c|c|}
\hline
Random variable &Value &Definition\\ \hline
\multirow{3}{*}{X} &0 &Slips of Rs 1\\
&1 &Slips of Rs 5\\
&2 &Slips of Rs 13\\ \hline
\multirow{2}{*}{Y} &0 &Box A\\
&1 &Box B\\\hline
\end{tabular}
\caption{}
\label{tab:Distribution}
\end{table}
See \tabref{tab:Distribution}.
\begin{align}
p_{Y}\brak{k}= \begin{cases} 
      \frac{1}{3} & {k=0} \\
      \frac{2}{3 }& {k=1} 
   \end{cases}
   \\
p_{Y|X}\brak{0|0} = \frac{19}{25}\, 
p_{Y|X}\brak{0|1} = \frac{6}{25}\,
p_{Y|X}\brak{1|0} = \frac{45}{50}\,
p_{Y|X}\brak{1|2} = \frac{5}{50}
\end{align}
The desired probability is the probability that a slip drawn at random is marked other than Rs 1,
\begin{align}
&=1-p_X\brak{0}\\
&= p_X(1) + p_X(2)
\end{align}
Using Bayes theorem,
\begin{align}
&= p_Y\brak{0} \times \pr{Y=0 | X=1} + p_Y\brak{1} \times \pr{Y=1|X=2}\\
&=\frac{1}{3} \times \frac{6}{25} + \frac{2}{3} \times \frac{5}{50}\\
&=\frac{11}{75}
\end{align}

\newpage

%\tableofcontents

\bigskip

\renewcommand{\thefigure}{\theenumi}
\renewcommand{\thetable}{\theenumi}
%\renewcommand{\theequation}{\theenumi}

%\begin{abstract}
%%\boldmath
%In this letter, an algorithm for evaluating the exact analytical bit error rate  (BER)  for the piecewise linear (PL) combiner for  multiple relays is presented. Previous results were available only for upto three relays. The algorithm is unique in the sense that  the actual mathematical expressions, that are prohibitively large, need not be explicitly obtained. The diversity gain due to multiple relays is shown through plots of the analytical BER, well supported by simulations. 
%
%\end{abstract}
% IEEEtran.cls defaults to using nonbold math in the Abstract.
% This preserves the distinction between vectors and scalars. However,
% if the journal you are submitting to favors bold math in the abstract,
% then you can use LaTeX's standard command \boldmath at the very start
% of the abstract to achieve this. Many IEEE journals frown on math
% in the abstract anyway.

% Note that keywords are not normally used for peerreview papers.
%\begin{IEEEkeywords}
%Cooperative diversity, decode and forward, piecewise linear
%\end{IEEEkeywords}



% For peer review papers, you can put extra information on the cover
% page as needed:
% \ifCLASSOPTIONpeerreview
% \begin{center} \bfseries EDICS Category: 3-BBND \end{center}
% \fi
%
% For peerreview papers, this IEEEtran command inserts a page break and
% creates the second title. It will be ignored for other modes.
%\IEEEpeerreviewmaketitle




	\item Five cards—the ten, jack, queen, king and ace of diamonds, are well-shuffled with their face downwards. One card is then picked up at random.
\begin{enumerate}
\item
What is the probability that the card is the queen? 
\item
If the queen is drawn and put aside, what is the probability that the second card picked up is (a) an ace? (b) a queen?\\
\end{enumerate}
\solution
		%\begin{enumerate}[label=\thesection.\arabic*,ref=\thesection.\theenumi]
	\item One card is drawn from a well-shuffled deck of 52 cards. Find the probability of getting
\begin{enumerate}
\item A king of red colour 
\item A face card 
\item A red face card
\item The jack of hearts
\item A spade
\item The queen of diamonds

\end{enumerate}
\solution
		%\input{ncert/10/15/1/14/main.tex}
	\item Five cards—the ten, jack, queen, king and ace of diamonds, are well-shuffled with their face downwards. One card is then picked up at random.
\begin{enumerate}
\item
What is the probability that the card is the queen? 
\item
If the queen is drawn and put aside, what is the probability that the second card picked up is (a) an ace? (b) a queen?\\
\end{enumerate}
\solution
		%\input{ncert/10/15/1/15/defs.tex}
	\item A bag contains $5$ red balls and some blue balls. If the probability of drawing a blue ball is double that if a red ball, determine the number of blue balls in the bag. 
		\\
\solution
		%\input{ncert/10/15/2/3/defs.tex}
	\item A card is selected from a pack of 52 cards.
 \begin{enumerate}[label=(\alph*)] 
                 \item How many points are there in the sample space?
                 \item Calculate the probability that the card is an ace of spades.
                 \item Calculate the probability that the card is (i) an ace and (ii) black card.
 \end{enumerate}
\solution
		%\input{ncert/11/16/3/4/main.tex}
\item Four cards are drawn from a well-shuffled deck of 52 cards. What is the probability of obtaining 3 diamonds and one spade.
\\
\solution
		%\input{ncert/11/16/4/2/defs.tex}
\item In a certain lottery 10,000 tickets are sold and ten equal prizes are awarded. What is the probability of not getting a prize if you buy (a) one ticket (b) two tickets (c) 10 tickets ?	
\\
\solution
		%\input{ncert/11/16/4/4/defs.tex}
		%
\item 
Out of 100 students, two sections of 40 and 60 are formed. If you and your friend are among the 100 students, what is the probability that
\begin{enumerate}
\item you both enter the same section?
\item you both enter the different sections?
\end{enumerate}
\solution
		%\input{ncert/11/16/4/5/defs.tex}
	\item 
The number lock of a suitcase has 4 wheels each labelled with ten digits i.e. from 0 to 9.The lock opens with a sequence of four digits with no repeats.What is the probability of a person getting the right sequence to open the suitcase.
\\
\solution
		%\input{ncert/11/16/4/10/defs.tex}
		%
\item 
Two cards are drawn at random and without replacement from a pack of 52 playing cards. Find the probability that both the cards are black.
\\
\solution
		%\input{ncert/12/13/2/2/defs.tex}
		\item A box of oranges is inspected by examining three randomly selected oranges drawn without replacement. If all the three oranges are good, the box is approved for sale, otherwise, it is rejected. Find the probability that a box containing 15 oranges out of which 12 are good and 3 are bad ones will be approved for sale.
		\label{ncert/12/13/2/3/defs.tex}
		\item Two balls are drawn at random with replacement from a box containing 10 black and 8 red balls. Find the probability that
		\label{ncert/12/13/2/12}
\begin{enumerate}
\item both balls are red.
\item first ball is black and second is red.
\item one of them is black and other is red.
\end{enumerate}

\item In a hostel, 60\% of the students read Hindi newspaper, 40\% read English newspaper and 20\% read both Hindi and English newspapers. A student is selected at random.
		\label{ncert/12/13/2/15}
\begin{enumerate}
\item Find the probability that she reads neither Hindi nor English newspapers.
\item If she reads Hindi newspaper, find the probability that she reads English newspaper.
\item If she reads English newspaper, find the probability that she reads Hindi newspaper.\\
\end{enumerate}
\item The probability of obtaining an even prime number on each die, when a pair of dice is rolled is 
\begin{enumerate}
    \item $0$ 
    
    \item $\frac{1}{3}$ 
    
    \item $\frac{1}{12}$ 
    
    \item $\frac{1}{36}$ 
\end{enumerate}
\solution
		%\input{ncert/12/13/2/17/defs.tex}
	\item A bag contains 4 red and 4 black balls, another bag contains 2 red and 6 black balls. One of the two bags is selected at random and a ball is drawn from the bag which is found to be red. Find the probability that the ball is drawn from the first bag.
\\
\solution
		%\input{ncert/12/13/3/2/main.tex}
  \item
  Cards with numbers 2 to 101 are placed in a box. A card is selected at random.Find the probability that the card has
\begin{enumerate}[label=(\roman*)]
	\item an even number 
	\item a square number
\end{enumerate}
\solution
%\input{exemplar/10/13/3/32/main.tex}
\item
The king, queen and jack of clubs are removed from a deck of 52 playing cards and then well shuffled. Now one card is drawn at random from the remaining cards.  Determine the probability that the card is
\begin{enumerate}[label=(\roman*)]
\item a club
\item 10 of hearts
\end{enumerate}
\solution
%\input{exemplar/10/13/3/29/main.tex}
\item A team of medical students doing their internship have to assist during surgeries
at a city hospital. The probabilities of surgeries rated as very complex, complex,
routine, simple or very simple are respectively, 0.15, 0.20, 0.31, 0.26, .08. Find
the probabilities that a particular surgery will be rated
\begin{enumerate}
	\item complex or very complex;
	\item neither very complex nor very simple;
	\item routine or complex
	\item routine or simple
\end{enumerate}
\solution
%\input{exemplar/11/16/3/8(1)/main.tex}
\item A card is selected from a pack of 52 cards.
\begin{enumerate}[label=(\alph*)]
    \item How many points are there in the sample space?
    \item Calculate the probability that the card is an ace of spades.
    \item Calculate the probability that the card is (i) an ace and (ii) black card.
\end{enumerate}
\solution
%\input{exemplar/11/16/3/4/main2.tex}
\item The probability that a non leap year selected at random will contain 53 sundays.
\\
\solution
%\input{exemplar/10/13/1/19/main.tex}
\item One of the four persons John, Rita, Aslam or Gurpreet will be promoted next
month. Consequently the sample space consists of four elementary outcomes
S = {John promoted, Rita promoted, Aslam promoted, Gurpreet promoted}
You are told that the chances of John’s promotion is same as that of Gurpreet,
Rita’s chances of promotion are twice as likely as Johns. Aslam’s chances are
four times that of John.
\begin{enumerate}
	\item Determine
	\begin{enumerate}
		\item P (John promoted)
		\item P (Rita promoted)
		\item P (Aslam promoted)
		\item P (Gurpreet promoted)
	\end{enumerate}
	\item If A = {John promoted or Gurpreet promoted}, find P (A).
\end{enumerate}
\solution
%\input{exemplar/11/16/3/10/main.tex}
\item A card is drawn from a deck of 52 cards. Find the probability of getting a king or a heart or a red card.\\
\solution
%\input{exemplar/11/16/3/15/main.tex}
\item The probability that a student will pass his examination is 0.73, the probability of
the student getting a compartment is 0.13, and the probability that the student will
either pass or get compartment is 0.96. State True or False.\\
\solution
%\input{exemplar/11/16/3/31/main.tex}
\item A card is selected from a pack of 52 cards\\
\begin{enumerate}[label=(\alph*)]
\item How many points are there in the sample space?
\item Calculate the probability that the cards is an ace of spades.
\item Calculate the probability that the card is (i) an ace (ii)black card.\\
\end{enumerate}
%\input{ncert/11/16/3/4_1/Prob_4.tex}
\item In a non-leap year, the probability of having 53 tuesdays or 53 wednesdays is\\
\solution
%\input{exemplar/11/16/3/18/main.tex}
\item There are 1000 sealed envelopes in a box, 10 of them contain a cash prize of
Rs 100 each, 100 of them contain a cash prize of Rs 50 each and 200 of them
contain a cash prize of Rs 10 each and rest do not contain any cash prize. If they
are well shuffled and an envelope is picked up out, what is the probability that it
contains no cash prize?\\
\solution
%\input{exemplar/10/13/3/34/main.tex}
\item 
A die is thrown and a card is selected at random from a deck of 52 playing cards. The probability of getting an even number on the die and a spade card.\\
\solution
%\input{exemplar/12/13/3/78/main.tex}
\item
If 4-digit numbers greater than 5,000 are randomly formed from the digits 0, 1, 3, 5, and 7, what is the probability of forming a number divisible by 5 when:
\begin{enumerate}
    \item The digits are repeated?
    \item The repetition of digits is not allowed?
\end{enumerate}
\solution
%\input{ncert/11/16/4/9/main.tex}
\item Consider the probability space $\brak{\Omega, \mathcal{G}, P}$ where $\Omega = [0,2]$ and $\mathcal{G} = \cbrak{\phi, \Omega, [0,1], (1,2]}$. Let $X$ and $Y$ be two functions on $\Omega$ defined as
\begin{align*}
    X(\omega) = 
    \begin{cases}
        1 & \text{if }\omega \in [0, 1]\\
        2 & \text{if }\omega \in (1, 2]
    \end{cases}
\end{align*}
and
\begin{align*}
    Y(\omega) = 
    \begin{cases}
        2 & \text{if }\omega \in [0, 1.5]\\
        3 & \text{if }\omega \in (1.5, 2].
    \end{cases}
\end{align*}
Then which one of the following statements is true?
\begin{enumerate}
    \item [(A)] $X$ is a random variable with respect to $\mathcal{G}$, but $Y$ is not a random variable with respect to $\mathcal{G}$.
    \item [(B)] $Y$ is a random variable with respect to $\mathcal{G}$, but $X$ is not a random variable with respect to $\mathcal{G}$.
    \item [(C)] Neither $X$ nor $Y$ is a random variable with respect to $\mathcal{G}$.
    \item [(D)] Both $X$ and $Y$ are random variables with respect to $\mathcal{G}$.
\end{enumerate} \hfill (GATE ST 2023)\\
\solution
%\input{gate/ST/2023/14/main.tex}
	\item  A die is loaded in such a way that each odd number is twice as likely to occur as
each even number. Find $P(G)$, where $G$ is the event that a number greater than
3 occurs on a single roll of the die.
\\
\solution
		%\input{exemplar/11/16/3/5/main.tex}
	\item All the jacks, queens and kings are removed from a deck of 52 playing cards. The remaining cards are well shuffled and then one card is drawn at random. Giving ace a value 1 similar value for other cards, find the probability that the card has a value 
		\begin{enumerate}
			\item 7
			\item greater than 7
			\item less than 7
		\end{enumerate}
		%\input{exemplar/10/13/3/30/main.tex}
  \item A Lot consists of 48 mobile phones of which 42 are good, 3 have only minor defects and 3 have major defects.Varnika will buy a phone if it is good but the trader will only buy a mobile if it has no major defects. One phone is selected at random from the lot. What is the probability that it is
\begin{enumerate}
	\item acceptable to Varnika?
            \item acceptable to the trader?
\end{enumerate}
\solution
	%\input{exemplar/10/13/3/40/main.tex}
 \item A student says that if you throw a die, it will show up 1 or not 1. Therefore, the probability of getting 1 and the probability of getting 'not 1' each is equal to $\frac{1}{2}$. Is this correct? Give reasons.\\
 \solution
        %\input{exemplar/10/13/2/9/main.tex}
   \item Four candidates A, B, C, D have ap-
plied for the assignment to coach a school cricket
team. If A is twice as likely to be selected as B, and
B and C are given about the same chance of being
selected, while C is twice as likely to be selected
as D, what are the probabilities that
\begin{enumerate}
\item C will be selected?
\item A will not be selected?
\end{enumerate}
	%\input{exemplar/11/16/3/9/main.tex}
 \item A bag contain 24 balls of which $x$ balls are red, $2x$ are white and $3x$ are blue. A ball is selected at random, What is the probability that it is
\begin{enumerate}[label=\alph*)]
\item not red ?
\item white ?
\end{enumerate}
%\input{exemplar/10/13/3/41/main.tex}
If the letters of the word ASSASSINATION are arranged at random. Find the Probability that
\begin{enumerate}[label=(\alph*)]
\item Four $S's$ come consecutively in the word
\item Two  $I's$ and two $N's$ come together
\item All $A's$ are not coming together
\item No two $A's$ are coming together
\end{enumerate}
%\input{exemplar/11/16/3/14/main.tex}
	\item One urn contains two black balls (labelled B1 and B2) and one white ball. A
	second urn contains one black ball and two white balls (labelled W1 and W2).
	Suppose the following experiment is performed. One of the two urns is chosen
	at random. Next a ball is randomly chosen from the urn. Then a second ball is
	chosen at random from the same urn without replacing the first ball.
	
	\begin{enumerate}
	\item What is the probability that two black balls are chosen?
	
	\item What is the probability that two balls of opposite colour are chosen?
	\end{enumerate}
	\solution
	%\input{exemplar/11/16/3/12/main1.tex}
\end{enumerate}

	\item A bag contains $5$ red balls and some blue balls. If the probability of drawing a blue ball is double that if a red ball, determine the number of blue balls in the bag. 
		\\
\solution
		%\begin{enumerate}[label=\thesection.\arabic*,ref=\thesection.\theenumi]
	\item One card is drawn from a well-shuffled deck of 52 cards. Find the probability of getting
\begin{enumerate}
\item A king of red colour 
\item A face card 
\item A red face card
\item The jack of hearts
\item A spade
\item The queen of diamonds

\end{enumerate}
\solution
		%\input{ncert/10/15/1/14/main.tex}
	\item Five cards—the ten, jack, queen, king and ace of diamonds, are well-shuffled with their face downwards. One card is then picked up at random.
\begin{enumerate}
\item
What is the probability that the card is the queen? 
\item
If the queen is drawn and put aside, what is the probability that the second card picked up is (a) an ace? (b) a queen?\\
\end{enumerate}
\solution
		%\input{ncert/10/15/1/15/defs.tex}
	\item A bag contains $5$ red balls and some blue balls. If the probability of drawing a blue ball is double that if a red ball, determine the number of blue balls in the bag. 
		\\
\solution
		%\input{ncert/10/15/2/3/defs.tex}
	\item A card is selected from a pack of 52 cards.
 \begin{enumerate}[label=(\alph*)] 
                 \item How many points are there in the sample space?
                 \item Calculate the probability that the card is an ace of spades.
                 \item Calculate the probability that the card is (i) an ace and (ii) black card.
 \end{enumerate}
\solution
		%\input{ncert/11/16/3/4/main.tex}
\item Four cards are drawn from a well-shuffled deck of 52 cards. What is the probability of obtaining 3 diamonds and one spade.
\\
\solution
		%\input{ncert/11/16/4/2/defs.tex}
\item In a certain lottery 10,000 tickets are sold and ten equal prizes are awarded. What is the probability of not getting a prize if you buy (a) one ticket (b) two tickets (c) 10 tickets ?	
\\
\solution
		%\input{ncert/11/16/4/4/defs.tex}
		%
\item 
Out of 100 students, two sections of 40 and 60 are formed. If you and your friend are among the 100 students, what is the probability that
\begin{enumerate}
\item you both enter the same section?
\item you both enter the different sections?
\end{enumerate}
\solution
		%\input{ncert/11/16/4/5/defs.tex}
	\item 
The number lock of a suitcase has 4 wheels each labelled with ten digits i.e. from 0 to 9.The lock opens with a sequence of four digits with no repeats.What is the probability of a person getting the right sequence to open the suitcase.
\\
\solution
		%\input{ncert/11/16/4/10/defs.tex}
		%
\item 
Two cards are drawn at random and without replacement from a pack of 52 playing cards. Find the probability that both the cards are black.
\\
\solution
		%\input{ncert/12/13/2/2/defs.tex}
		\item A box of oranges is inspected by examining three randomly selected oranges drawn without replacement. If all the three oranges are good, the box is approved for sale, otherwise, it is rejected. Find the probability that a box containing 15 oranges out of which 12 are good and 3 are bad ones will be approved for sale.
		\label{ncert/12/13/2/3/defs.tex}
		\item Two balls are drawn at random with replacement from a box containing 10 black and 8 red balls. Find the probability that
		\label{ncert/12/13/2/12}
\begin{enumerate}
\item both balls are red.
\item first ball is black and second is red.
\item one of them is black and other is red.
\end{enumerate}

\item In a hostel, 60\% of the students read Hindi newspaper, 40\% read English newspaper and 20\% read both Hindi and English newspapers. A student is selected at random.
		\label{ncert/12/13/2/15}
\begin{enumerate}
\item Find the probability that she reads neither Hindi nor English newspapers.
\item If she reads Hindi newspaper, find the probability that she reads English newspaper.
\item If she reads English newspaper, find the probability that she reads Hindi newspaper.\\
\end{enumerate}
\item The probability of obtaining an even prime number on each die, when a pair of dice is rolled is 
\begin{enumerate}
    \item $0$ 
    
    \item $\frac{1}{3}$ 
    
    \item $\frac{1}{12}$ 
    
    \item $\frac{1}{36}$ 
\end{enumerate}
\solution
		%\input{ncert/12/13/2/17/defs.tex}
	\item A bag contains 4 red and 4 black balls, another bag contains 2 red and 6 black balls. One of the two bags is selected at random and a ball is drawn from the bag which is found to be red. Find the probability that the ball is drawn from the first bag.
\\
\solution
		%\input{ncert/12/13/3/2/main.tex}
  \item
  Cards with numbers 2 to 101 are placed in a box. A card is selected at random.Find the probability that the card has
\begin{enumerate}[label=(\roman*)]
	\item an even number 
	\item a square number
\end{enumerate}
\solution
%\input{exemplar/10/13/3/32/main.tex}
\item
The king, queen and jack of clubs are removed from a deck of 52 playing cards and then well shuffled. Now one card is drawn at random from the remaining cards.  Determine the probability that the card is
\begin{enumerate}[label=(\roman*)]
\item a club
\item 10 of hearts
\end{enumerate}
\solution
%\input{exemplar/10/13/3/29/main.tex}
\item A team of medical students doing their internship have to assist during surgeries
at a city hospital. The probabilities of surgeries rated as very complex, complex,
routine, simple or very simple are respectively, 0.15, 0.20, 0.31, 0.26, .08. Find
the probabilities that a particular surgery will be rated
\begin{enumerate}
	\item complex or very complex;
	\item neither very complex nor very simple;
	\item routine or complex
	\item routine or simple
\end{enumerate}
\solution
%\input{exemplar/11/16/3/8(1)/main.tex}
\item A card is selected from a pack of 52 cards.
\begin{enumerate}[label=(\alph*)]
    \item How many points are there in the sample space?
    \item Calculate the probability that the card is an ace of spades.
    \item Calculate the probability that the card is (i) an ace and (ii) black card.
\end{enumerate}
\solution
%\input{exemplar/11/16/3/4/main2.tex}
\item The probability that a non leap year selected at random will contain 53 sundays.
\\
\solution
%\input{exemplar/10/13/1/19/main.tex}
\item One of the four persons John, Rita, Aslam or Gurpreet will be promoted next
month. Consequently the sample space consists of four elementary outcomes
S = {John promoted, Rita promoted, Aslam promoted, Gurpreet promoted}
You are told that the chances of John’s promotion is same as that of Gurpreet,
Rita’s chances of promotion are twice as likely as Johns. Aslam’s chances are
four times that of John.
\begin{enumerate}
	\item Determine
	\begin{enumerate}
		\item P (John promoted)
		\item P (Rita promoted)
		\item P (Aslam promoted)
		\item P (Gurpreet promoted)
	\end{enumerate}
	\item If A = {John promoted or Gurpreet promoted}, find P (A).
\end{enumerate}
\solution
%\input{exemplar/11/16/3/10/main.tex}
\item A card is drawn from a deck of 52 cards. Find the probability of getting a king or a heart or a red card.\\
\solution
%\input{exemplar/11/16/3/15/main.tex}
\item The probability that a student will pass his examination is 0.73, the probability of
the student getting a compartment is 0.13, and the probability that the student will
either pass or get compartment is 0.96. State True or False.\\
\solution
%\input{exemplar/11/16/3/31/main.tex}
\item A card is selected from a pack of 52 cards\\
\begin{enumerate}[label=(\alph*)]
\item How many points are there in the sample space?
\item Calculate the probability that the cards is an ace of spades.
\item Calculate the probability that the card is (i) an ace (ii)black card.\\
\end{enumerate}
%\input{ncert/11/16/3/4_1/Prob_4.tex}
\item In a non-leap year, the probability of having 53 tuesdays or 53 wednesdays is\\
\solution
%\input{exemplar/11/16/3/18/main.tex}
\item There are 1000 sealed envelopes in a box, 10 of them contain a cash prize of
Rs 100 each, 100 of them contain a cash prize of Rs 50 each and 200 of them
contain a cash prize of Rs 10 each and rest do not contain any cash prize. If they
are well shuffled and an envelope is picked up out, what is the probability that it
contains no cash prize?\\
\solution
%\input{exemplar/10/13/3/34/main.tex}
\item 
A die is thrown and a card is selected at random from a deck of 52 playing cards. The probability of getting an even number on the die and a spade card.\\
\solution
%\input{exemplar/12/13/3/78/main.tex}
\item
If 4-digit numbers greater than 5,000 are randomly formed from the digits 0, 1, 3, 5, and 7, what is the probability of forming a number divisible by 5 when:
\begin{enumerate}
    \item The digits are repeated?
    \item The repetition of digits is not allowed?
\end{enumerate}
\solution
%\input{ncert/11/16/4/9/main.tex}
\item Consider the probability space $\brak{\Omega, \mathcal{G}, P}$ where $\Omega = [0,2]$ and $\mathcal{G} = \cbrak{\phi, \Omega, [0,1], (1,2]}$. Let $X$ and $Y$ be two functions on $\Omega$ defined as
\begin{align*}
    X(\omega) = 
    \begin{cases}
        1 & \text{if }\omega \in [0, 1]\\
        2 & \text{if }\omega \in (1, 2]
    \end{cases}
\end{align*}
and
\begin{align*}
    Y(\omega) = 
    \begin{cases}
        2 & \text{if }\omega \in [0, 1.5]\\
        3 & \text{if }\omega \in (1.5, 2].
    \end{cases}
\end{align*}
Then which one of the following statements is true?
\begin{enumerate}
    \item [(A)] $X$ is a random variable with respect to $\mathcal{G}$, but $Y$ is not a random variable with respect to $\mathcal{G}$.
    \item [(B)] $Y$ is a random variable with respect to $\mathcal{G}$, but $X$ is not a random variable with respect to $\mathcal{G}$.
    \item [(C)] Neither $X$ nor $Y$ is a random variable with respect to $\mathcal{G}$.
    \item [(D)] Both $X$ and $Y$ are random variables with respect to $\mathcal{G}$.
\end{enumerate} \hfill (GATE ST 2023)\\
\solution
%\input{gate/ST/2023/14/main.tex}
	\item  A die is loaded in such a way that each odd number is twice as likely to occur as
each even number. Find $P(G)$, where $G$ is the event that a number greater than
3 occurs on a single roll of the die.
\\
\solution
		%\input{exemplar/11/16/3/5/main.tex}
	\item All the jacks, queens and kings are removed from a deck of 52 playing cards. The remaining cards are well shuffled and then one card is drawn at random. Giving ace a value 1 similar value for other cards, find the probability that the card has a value 
		\begin{enumerate}
			\item 7
			\item greater than 7
			\item less than 7
		\end{enumerate}
		%\input{exemplar/10/13/3/30/main.tex}
  \item A Lot consists of 48 mobile phones of which 42 are good, 3 have only minor defects and 3 have major defects.Varnika will buy a phone if it is good but the trader will only buy a mobile if it has no major defects. One phone is selected at random from the lot. What is the probability that it is
\begin{enumerate}
	\item acceptable to Varnika?
            \item acceptable to the trader?
\end{enumerate}
\solution
	%\input{exemplar/10/13/3/40/main.tex}
 \item A student says that if you throw a die, it will show up 1 or not 1. Therefore, the probability of getting 1 and the probability of getting 'not 1' each is equal to $\frac{1}{2}$. Is this correct? Give reasons.\\
 \solution
        %\input{exemplar/10/13/2/9/main.tex}
   \item Four candidates A, B, C, D have ap-
plied for the assignment to coach a school cricket
team. If A is twice as likely to be selected as B, and
B and C are given about the same chance of being
selected, while C is twice as likely to be selected
as D, what are the probabilities that
\begin{enumerate}
\item C will be selected?
\item A will not be selected?
\end{enumerate}
	%\input{exemplar/11/16/3/9/main.tex}
 \item A bag contain 24 balls of which $x$ balls are red, $2x$ are white and $3x$ are blue. A ball is selected at random, What is the probability that it is
\begin{enumerate}[label=\alph*)]
\item not red ?
\item white ?
\end{enumerate}
%\input{exemplar/10/13/3/41/main.tex}
If the letters of the word ASSASSINATION are arranged at random. Find the Probability that
\begin{enumerate}[label=(\alph*)]
\item Four $S's$ come consecutively in the word
\item Two  $I's$ and two $N's$ come together
\item All $A's$ are not coming together
\item No two $A's$ are coming together
\end{enumerate}
%\input{exemplar/11/16/3/14/main.tex}
	\item One urn contains two black balls (labelled B1 and B2) and one white ball. A
	second urn contains one black ball and two white balls (labelled W1 and W2).
	Suppose the following experiment is performed. One of the two urns is chosen
	at random. Next a ball is randomly chosen from the urn. Then a second ball is
	chosen at random from the same urn without replacing the first ball.
	
	\begin{enumerate}
	\item What is the probability that two black balls are chosen?
	
	\item What is the probability that two balls of opposite colour are chosen?
	\end{enumerate}
	\solution
	%\input{exemplar/11/16/3/12/main1.tex}
\end{enumerate}

	\item A card is selected from a pack of 52 cards.
 \begin{enumerate}[label=(\alph*)] 
                 \item How many points are there in the sample space?
                 \item Calculate the probability that the card is an ace of spades.
                 \item Calculate the probability that the card is (i) an ace and (ii) black card.
 \end{enumerate}
\solution
		%\begin{table}[H]
	\centering
\begin{tabular}{|c|c|c|}
\hline
Random variable &Value &Definition\\ \hline
\multirow{3}{*}{X} &0 &Slips of Rs 1\\
&1 &Slips of Rs 5\\
&2 &Slips of Rs 13\\ \hline
\multirow{2}{*}{Y} &0 &Box A\\
&1 &Box B\\\hline
\end{tabular}
\caption{}
\label{tab:Distribution}
\end{table}
See \tabref{tab:Distribution}.
\begin{align}
p_{Y}\brak{k}= \begin{cases} 
      \frac{1}{3} & {k=0} \\
      \frac{2}{3 }& {k=1} 
   \end{cases}
   \\
p_{Y|X}\brak{0|0} = \frac{19}{25}\, 
p_{Y|X}\brak{0|1} = \frac{6}{25}\,
p_{Y|X}\brak{1|0} = \frac{45}{50}\,
p_{Y|X}\brak{1|2} = \frac{5}{50}
\end{align}
The desired probability is the probability that a slip drawn at random is marked other than Rs 1,
\begin{align}
&=1-p_X\brak{0}\\
&= p_X(1) + p_X(2)
\end{align}
Using Bayes theorem,
\begin{align}
&= p_Y\brak{0} \times \pr{Y=0 | X=1} + p_Y\brak{1} \times \pr{Y=1|X=2}\\
&=\frac{1}{3} \times \frac{6}{25} + \frac{2}{3} \times \frac{5}{50}\\
&=\frac{11}{75}
\end{align}

\newpage

%\tableofcontents

\bigskip

\renewcommand{\thefigure}{\theenumi}
\renewcommand{\thetable}{\theenumi}
%\renewcommand{\theequation}{\theenumi}

%\begin{abstract}
%%\boldmath
%In this letter, an algorithm for evaluating the exact analytical bit error rate  (BER)  for the piecewise linear (PL) combiner for  multiple relays is presented. Previous results were available only for upto three relays. The algorithm is unique in the sense that  the actual mathematical expressions, that are prohibitively large, need not be explicitly obtained. The diversity gain due to multiple relays is shown through plots of the analytical BER, well supported by simulations. 
%
%\end{abstract}
% IEEEtran.cls defaults to using nonbold math in the Abstract.
% This preserves the distinction between vectors and scalars. However,
% if the journal you are submitting to favors bold math in the abstract,
% then you can use LaTeX's standard command \boldmath at the very start
% of the abstract to achieve this. Many IEEE journals frown on math
% in the abstract anyway.

% Note that keywords are not normally used for peerreview papers.
%\begin{IEEEkeywords}
%Cooperative diversity, decode and forward, piecewise linear
%\end{IEEEkeywords}



% For peer review papers, you can put extra information on the cover
% page as needed:
% \ifCLASSOPTIONpeerreview
% \begin{center} \bfseries EDICS Category: 3-BBND \end{center}
% \fi
%
% For peerreview papers, this IEEEtran command inserts a page break and
% creates the second title. It will be ignored for other modes.
%\IEEEpeerreviewmaketitle




\item Four cards are drawn from a well-shuffled deck of 52 cards. What is the probability of obtaining 3 diamonds and one spade.
\\
\solution
		%\begin{enumerate}[label=\thesection.\arabic*,ref=\thesection.\theenumi]
	\item One card is drawn from a well-shuffled deck of 52 cards. Find the probability of getting
\begin{enumerate}
\item A king of red colour 
\item A face card 
\item A red face card
\item The jack of hearts
\item A spade
\item The queen of diamonds

\end{enumerate}
\solution
		%\input{ncert/10/15/1/14/main.tex}
	\item Five cards—the ten, jack, queen, king and ace of diamonds, are well-shuffled with their face downwards. One card is then picked up at random.
\begin{enumerate}
\item
What is the probability that the card is the queen? 
\item
If the queen is drawn and put aside, what is the probability that the second card picked up is (a) an ace? (b) a queen?\\
\end{enumerate}
\solution
		%\input{ncert/10/15/1/15/defs.tex}
	\item A bag contains $5$ red balls and some blue balls. If the probability of drawing a blue ball is double that if a red ball, determine the number of blue balls in the bag. 
		\\
\solution
		%\input{ncert/10/15/2/3/defs.tex}
	\item A card is selected from a pack of 52 cards.
 \begin{enumerate}[label=(\alph*)] 
                 \item How many points are there in the sample space?
                 \item Calculate the probability that the card is an ace of spades.
                 \item Calculate the probability that the card is (i) an ace and (ii) black card.
 \end{enumerate}
\solution
		%\input{ncert/11/16/3/4/main.tex}
\item Four cards are drawn from a well-shuffled deck of 52 cards. What is the probability of obtaining 3 diamonds and one spade.
\\
\solution
		%\input{ncert/11/16/4/2/defs.tex}
\item In a certain lottery 10,000 tickets are sold and ten equal prizes are awarded. What is the probability of not getting a prize if you buy (a) one ticket (b) two tickets (c) 10 tickets ?	
\\
\solution
		%\input{ncert/11/16/4/4/defs.tex}
		%
\item 
Out of 100 students, two sections of 40 and 60 are formed. If you and your friend are among the 100 students, what is the probability that
\begin{enumerate}
\item you both enter the same section?
\item you both enter the different sections?
\end{enumerate}
\solution
		%\input{ncert/11/16/4/5/defs.tex}
	\item 
The number lock of a suitcase has 4 wheels each labelled with ten digits i.e. from 0 to 9.The lock opens with a sequence of four digits with no repeats.What is the probability of a person getting the right sequence to open the suitcase.
\\
\solution
		%\input{ncert/11/16/4/10/defs.tex}
		%
\item 
Two cards are drawn at random and without replacement from a pack of 52 playing cards. Find the probability that both the cards are black.
\\
\solution
		%\input{ncert/12/13/2/2/defs.tex}
		\item A box of oranges is inspected by examining three randomly selected oranges drawn without replacement. If all the three oranges are good, the box is approved for sale, otherwise, it is rejected. Find the probability that a box containing 15 oranges out of which 12 are good and 3 are bad ones will be approved for sale.
		\label{ncert/12/13/2/3/defs.tex}
		\item Two balls are drawn at random with replacement from a box containing 10 black and 8 red balls. Find the probability that
		\label{ncert/12/13/2/12}
\begin{enumerate}
\item both balls are red.
\item first ball is black and second is red.
\item one of them is black and other is red.
\end{enumerate}

\item In a hostel, 60\% of the students read Hindi newspaper, 40\% read English newspaper and 20\% read both Hindi and English newspapers. A student is selected at random.
		\label{ncert/12/13/2/15}
\begin{enumerate}
\item Find the probability that she reads neither Hindi nor English newspapers.
\item If she reads Hindi newspaper, find the probability that she reads English newspaper.
\item If she reads English newspaper, find the probability that she reads Hindi newspaper.\\
\end{enumerate}
\item The probability of obtaining an even prime number on each die, when a pair of dice is rolled is 
\begin{enumerate}
    \item $0$ 
    
    \item $\frac{1}{3}$ 
    
    \item $\frac{1}{12}$ 
    
    \item $\frac{1}{36}$ 
\end{enumerate}
\solution
		%\input{ncert/12/13/2/17/defs.tex}
	\item A bag contains 4 red and 4 black balls, another bag contains 2 red and 6 black balls. One of the two bags is selected at random and a ball is drawn from the bag which is found to be red. Find the probability that the ball is drawn from the first bag.
\\
\solution
		%\input{ncert/12/13/3/2/main.tex}
  \item
  Cards with numbers 2 to 101 are placed in a box. A card is selected at random.Find the probability that the card has
\begin{enumerate}[label=(\roman*)]
	\item an even number 
	\item a square number
\end{enumerate}
\solution
%\input{exemplar/10/13/3/32/main.tex}
\item
The king, queen and jack of clubs are removed from a deck of 52 playing cards and then well shuffled. Now one card is drawn at random from the remaining cards.  Determine the probability that the card is
\begin{enumerate}[label=(\roman*)]
\item a club
\item 10 of hearts
\end{enumerate}
\solution
%\input{exemplar/10/13/3/29/main.tex}
\item A team of medical students doing their internship have to assist during surgeries
at a city hospital. The probabilities of surgeries rated as very complex, complex,
routine, simple or very simple are respectively, 0.15, 0.20, 0.31, 0.26, .08. Find
the probabilities that a particular surgery will be rated
\begin{enumerate}
	\item complex or very complex;
	\item neither very complex nor very simple;
	\item routine or complex
	\item routine or simple
\end{enumerate}
\solution
%\input{exemplar/11/16/3/8(1)/main.tex}
\item A card is selected from a pack of 52 cards.
\begin{enumerate}[label=(\alph*)]
    \item How many points are there in the sample space?
    \item Calculate the probability that the card is an ace of spades.
    \item Calculate the probability that the card is (i) an ace and (ii) black card.
\end{enumerate}
\solution
%\input{exemplar/11/16/3/4/main2.tex}
\item The probability that a non leap year selected at random will contain 53 sundays.
\\
\solution
%\input{exemplar/10/13/1/19/main.tex}
\item One of the four persons John, Rita, Aslam or Gurpreet will be promoted next
month. Consequently the sample space consists of four elementary outcomes
S = {John promoted, Rita promoted, Aslam promoted, Gurpreet promoted}
You are told that the chances of John’s promotion is same as that of Gurpreet,
Rita’s chances of promotion are twice as likely as Johns. Aslam’s chances are
four times that of John.
\begin{enumerate}
	\item Determine
	\begin{enumerate}
		\item P (John promoted)
		\item P (Rita promoted)
		\item P (Aslam promoted)
		\item P (Gurpreet promoted)
	\end{enumerate}
	\item If A = {John promoted or Gurpreet promoted}, find P (A).
\end{enumerate}
\solution
%\input{exemplar/11/16/3/10/main.tex}
\item A card is drawn from a deck of 52 cards. Find the probability of getting a king or a heart or a red card.\\
\solution
%\input{exemplar/11/16/3/15/main.tex}
\item The probability that a student will pass his examination is 0.73, the probability of
the student getting a compartment is 0.13, and the probability that the student will
either pass or get compartment is 0.96. State True or False.\\
\solution
%\input{exemplar/11/16/3/31/main.tex}
\item A card is selected from a pack of 52 cards\\
\begin{enumerate}[label=(\alph*)]
\item How many points are there in the sample space?
\item Calculate the probability that the cards is an ace of spades.
\item Calculate the probability that the card is (i) an ace (ii)black card.\\
\end{enumerate}
%\input{ncert/11/16/3/4_1/Prob_4.tex}
\item In a non-leap year, the probability of having 53 tuesdays or 53 wednesdays is\\
\solution
%\input{exemplar/11/16/3/18/main.tex}
\item There are 1000 sealed envelopes in a box, 10 of them contain a cash prize of
Rs 100 each, 100 of them contain a cash prize of Rs 50 each and 200 of them
contain a cash prize of Rs 10 each and rest do not contain any cash prize. If they
are well shuffled and an envelope is picked up out, what is the probability that it
contains no cash prize?\\
\solution
%\input{exemplar/10/13/3/34/main.tex}
\item 
A die is thrown and a card is selected at random from a deck of 52 playing cards. The probability of getting an even number on the die and a spade card.\\
\solution
%\input{exemplar/12/13/3/78/main.tex}
\item
If 4-digit numbers greater than 5,000 are randomly formed from the digits 0, 1, 3, 5, and 7, what is the probability of forming a number divisible by 5 when:
\begin{enumerate}
    \item The digits are repeated?
    \item The repetition of digits is not allowed?
\end{enumerate}
\solution
%\input{ncert/11/16/4/9/main.tex}
\item Consider the probability space $\brak{\Omega, \mathcal{G}, P}$ where $\Omega = [0,2]$ and $\mathcal{G} = \cbrak{\phi, \Omega, [0,1], (1,2]}$. Let $X$ and $Y$ be two functions on $\Omega$ defined as
\begin{align*}
    X(\omega) = 
    \begin{cases}
        1 & \text{if }\omega \in [0, 1]\\
        2 & \text{if }\omega \in (1, 2]
    \end{cases}
\end{align*}
and
\begin{align*}
    Y(\omega) = 
    \begin{cases}
        2 & \text{if }\omega \in [0, 1.5]\\
        3 & \text{if }\omega \in (1.5, 2].
    \end{cases}
\end{align*}
Then which one of the following statements is true?
\begin{enumerate}
    \item [(A)] $X$ is a random variable with respect to $\mathcal{G}$, but $Y$ is not a random variable with respect to $\mathcal{G}$.
    \item [(B)] $Y$ is a random variable with respect to $\mathcal{G}$, but $X$ is not a random variable with respect to $\mathcal{G}$.
    \item [(C)] Neither $X$ nor $Y$ is a random variable with respect to $\mathcal{G}$.
    \item [(D)] Both $X$ and $Y$ are random variables with respect to $\mathcal{G}$.
\end{enumerate} \hfill (GATE ST 2023)\\
\solution
%\input{gate/ST/2023/14/main.tex}
	\item  A die is loaded in such a way that each odd number is twice as likely to occur as
each even number. Find $P(G)$, where $G$ is the event that a number greater than
3 occurs on a single roll of the die.
\\
\solution
		%\input{exemplar/11/16/3/5/main.tex}
	\item All the jacks, queens and kings are removed from a deck of 52 playing cards. The remaining cards are well shuffled and then one card is drawn at random. Giving ace a value 1 similar value for other cards, find the probability that the card has a value 
		\begin{enumerate}
			\item 7
			\item greater than 7
			\item less than 7
		\end{enumerate}
		%\input{exemplar/10/13/3/30/main.tex}
  \item A Lot consists of 48 mobile phones of which 42 are good, 3 have only minor defects and 3 have major defects.Varnika will buy a phone if it is good but the trader will only buy a mobile if it has no major defects. One phone is selected at random from the lot. What is the probability that it is
\begin{enumerate}
	\item acceptable to Varnika?
            \item acceptable to the trader?
\end{enumerate}
\solution
	%\input{exemplar/10/13/3/40/main.tex}
 \item A student says that if you throw a die, it will show up 1 or not 1. Therefore, the probability of getting 1 and the probability of getting 'not 1' each is equal to $\frac{1}{2}$. Is this correct? Give reasons.\\
 \solution
        %\input{exemplar/10/13/2/9/main.tex}
   \item Four candidates A, B, C, D have ap-
plied for the assignment to coach a school cricket
team. If A is twice as likely to be selected as B, and
B and C are given about the same chance of being
selected, while C is twice as likely to be selected
as D, what are the probabilities that
\begin{enumerate}
\item C will be selected?
\item A will not be selected?
\end{enumerate}
	%\input{exemplar/11/16/3/9/main.tex}
 \item A bag contain 24 balls of which $x$ balls are red, $2x$ are white and $3x$ are blue. A ball is selected at random, What is the probability that it is
\begin{enumerate}[label=\alph*)]
\item not red ?
\item white ?
\end{enumerate}
%\input{exemplar/10/13/3/41/main.tex}
If the letters of the word ASSASSINATION are arranged at random. Find the Probability that
\begin{enumerate}[label=(\alph*)]
\item Four $S's$ come consecutively in the word
\item Two  $I's$ and two $N's$ come together
\item All $A's$ are not coming together
\item No two $A's$ are coming together
\end{enumerate}
%\input{exemplar/11/16/3/14/main.tex}
	\item One urn contains two black balls (labelled B1 and B2) and one white ball. A
	second urn contains one black ball and two white balls (labelled W1 and W2).
	Suppose the following experiment is performed. One of the two urns is chosen
	at random. Next a ball is randomly chosen from the urn. Then a second ball is
	chosen at random from the same urn without replacing the first ball.
	
	\begin{enumerate}
	\item What is the probability that two black balls are chosen?
	
	\item What is the probability that two balls of opposite colour are chosen?
	\end{enumerate}
	\solution
	%\input{exemplar/11/16/3/12/main1.tex}
\end{enumerate}

\item In a certain lottery 10,000 tickets are sold and ten equal prizes are awarded. What is the probability of not getting a prize if you buy (a) one ticket (b) two tickets (c) 10 tickets ?	
\\
\solution
		%\begin{enumerate}[label=\thesection.\arabic*,ref=\thesection.\theenumi]
	\item One card is drawn from a well-shuffled deck of 52 cards. Find the probability of getting
\begin{enumerate}
\item A king of red colour 
\item A face card 
\item A red face card
\item The jack of hearts
\item A spade
\item The queen of diamonds

\end{enumerate}
\solution
		%\input{ncert/10/15/1/14/main.tex}
	\item Five cards—the ten, jack, queen, king and ace of diamonds, are well-shuffled with their face downwards. One card is then picked up at random.
\begin{enumerate}
\item
What is the probability that the card is the queen? 
\item
If the queen is drawn and put aside, what is the probability that the second card picked up is (a) an ace? (b) a queen?\\
\end{enumerate}
\solution
		%\input{ncert/10/15/1/15/defs.tex}
	\item A bag contains $5$ red balls and some blue balls. If the probability of drawing a blue ball is double that if a red ball, determine the number of blue balls in the bag. 
		\\
\solution
		%\input{ncert/10/15/2/3/defs.tex}
	\item A card is selected from a pack of 52 cards.
 \begin{enumerate}[label=(\alph*)] 
                 \item How many points are there in the sample space?
                 \item Calculate the probability that the card is an ace of spades.
                 \item Calculate the probability that the card is (i) an ace and (ii) black card.
 \end{enumerate}
\solution
		%\input{ncert/11/16/3/4/main.tex}
\item Four cards are drawn from a well-shuffled deck of 52 cards. What is the probability of obtaining 3 diamonds and one spade.
\\
\solution
		%\input{ncert/11/16/4/2/defs.tex}
\item In a certain lottery 10,000 tickets are sold and ten equal prizes are awarded. What is the probability of not getting a prize if you buy (a) one ticket (b) two tickets (c) 10 tickets ?	
\\
\solution
		%\input{ncert/11/16/4/4/defs.tex}
		%
\item 
Out of 100 students, two sections of 40 and 60 are formed. If you and your friend are among the 100 students, what is the probability that
\begin{enumerate}
\item you both enter the same section?
\item you both enter the different sections?
\end{enumerate}
\solution
		%\input{ncert/11/16/4/5/defs.tex}
	\item 
The number lock of a suitcase has 4 wheels each labelled with ten digits i.e. from 0 to 9.The lock opens with a sequence of four digits with no repeats.What is the probability of a person getting the right sequence to open the suitcase.
\\
\solution
		%\input{ncert/11/16/4/10/defs.tex}
		%
\item 
Two cards are drawn at random and without replacement from a pack of 52 playing cards. Find the probability that both the cards are black.
\\
\solution
		%\input{ncert/12/13/2/2/defs.tex}
		\item A box of oranges is inspected by examining three randomly selected oranges drawn without replacement. If all the three oranges are good, the box is approved for sale, otherwise, it is rejected. Find the probability that a box containing 15 oranges out of which 12 are good and 3 are bad ones will be approved for sale.
		\label{ncert/12/13/2/3/defs.tex}
		\item Two balls are drawn at random with replacement from a box containing 10 black and 8 red balls. Find the probability that
		\label{ncert/12/13/2/12}
\begin{enumerate}
\item both balls are red.
\item first ball is black and second is red.
\item one of them is black and other is red.
\end{enumerate}

\item In a hostel, 60\% of the students read Hindi newspaper, 40\% read English newspaper and 20\% read both Hindi and English newspapers. A student is selected at random.
		\label{ncert/12/13/2/15}
\begin{enumerate}
\item Find the probability that she reads neither Hindi nor English newspapers.
\item If she reads Hindi newspaper, find the probability that she reads English newspaper.
\item If she reads English newspaper, find the probability that she reads Hindi newspaper.\\
\end{enumerate}
\item The probability of obtaining an even prime number on each die, when a pair of dice is rolled is 
\begin{enumerate}
    \item $0$ 
    
    \item $\frac{1}{3}$ 
    
    \item $\frac{1}{12}$ 
    
    \item $\frac{1}{36}$ 
\end{enumerate}
\solution
		%\input{ncert/12/13/2/17/defs.tex}
	\item A bag contains 4 red and 4 black balls, another bag contains 2 red and 6 black balls. One of the two bags is selected at random and a ball is drawn from the bag which is found to be red. Find the probability that the ball is drawn from the first bag.
\\
\solution
		%\input{ncert/12/13/3/2/main.tex}
  \item
  Cards with numbers 2 to 101 are placed in a box. A card is selected at random.Find the probability that the card has
\begin{enumerate}[label=(\roman*)]
	\item an even number 
	\item a square number
\end{enumerate}
\solution
%\input{exemplar/10/13/3/32/main.tex}
\item
The king, queen and jack of clubs are removed from a deck of 52 playing cards and then well shuffled. Now one card is drawn at random from the remaining cards.  Determine the probability that the card is
\begin{enumerate}[label=(\roman*)]
\item a club
\item 10 of hearts
\end{enumerate}
\solution
%\input{exemplar/10/13/3/29/main.tex}
\item A team of medical students doing their internship have to assist during surgeries
at a city hospital. The probabilities of surgeries rated as very complex, complex,
routine, simple or very simple are respectively, 0.15, 0.20, 0.31, 0.26, .08. Find
the probabilities that a particular surgery will be rated
\begin{enumerate}
	\item complex or very complex;
	\item neither very complex nor very simple;
	\item routine or complex
	\item routine or simple
\end{enumerate}
\solution
%\input{exemplar/11/16/3/8(1)/main.tex}
\item A card is selected from a pack of 52 cards.
\begin{enumerate}[label=(\alph*)]
    \item How many points are there in the sample space?
    \item Calculate the probability that the card is an ace of spades.
    \item Calculate the probability that the card is (i) an ace and (ii) black card.
\end{enumerate}
\solution
%\input{exemplar/11/16/3/4/main2.tex}
\item The probability that a non leap year selected at random will contain 53 sundays.
\\
\solution
%\input{exemplar/10/13/1/19/main.tex}
\item One of the four persons John, Rita, Aslam or Gurpreet will be promoted next
month. Consequently the sample space consists of four elementary outcomes
S = {John promoted, Rita promoted, Aslam promoted, Gurpreet promoted}
You are told that the chances of John’s promotion is same as that of Gurpreet,
Rita’s chances of promotion are twice as likely as Johns. Aslam’s chances are
four times that of John.
\begin{enumerate}
	\item Determine
	\begin{enumerate}
		\item P (John promoted)
		\item P (Rita promoted)
		\item P (Aslam promoted)
		\item P (Gurpreet promoted)
	\end{enumerate}
	\item If A = {John promoted or Gurpreet promoted}, find P (A).
\end{enumerate}
\solution
%\input{exemplar/11/16/3/10/main.tex}
\item A card is drawn from a deck of 52 cards. Find the probability of getting a king or a heart or a red card.\\
\solution
%\input{exemplar/11/16/3/15/main.tex}
\item The probability that a student will pass his examination is 0.73, the probability of
the student getting a compartment is 0.13, and the probability that the student will
either pass or get compartment is 0.96. State True or False.\\
\solution
%\input{exemplar/11/16/3/31/main.tex}
\item A card is selected from a pack of 52 cards\\
\begin{enumerate}[label=(\alph*)]
\item How many points are there in the sample space?
\item Calculate the probability that the cards is an ace of spades.
\item Calculate the probability that the card is (i) an ace (ii)black card.\\
\end{enumerate}
%\input{ncert/11/16/3/4_1/Prob_4.tex}
\item In a non-leap year, the probability of having 53 tuesdays or 53 wednesdays is\\
\solution
%\input{exemplar/11/16/3/18/main.tex}
\item There are 1000 sealed envelopes in a box, 10 of them contain a cash prize of
Rs 100 each, 100 of them contain a cash prize of Rs 50 each and 200 of them
contain a cash prize of Rs 10 each and rest do not contain any cash prize. If they
are well shuffled and an envelope is picked up out, what is the probability that it
contains no cash prize?\\
\solution
%\input{exemplar/10/13/3/34/main.tex}
\item 
A die is thrown and a card is selected at random from a deck of 52 playing cards. The probability of getting an even number on the die and a spade card.\\
\solution
%\input{exemplar/12/13/3/78/main.tex}
\item
If 4-digit numbers greater than 5,000 are randomly formed from the digits 0, 1, 3, 5, and 7, what is the probability of forming a number divisible by 5 when:
\begin{enumerate}
    \item The digits are repeated?
    \item The repetition of digits is not allowed?
\end{enumerate}
\solution
%\input{ncert/11/16/4/9/main.tex}
\item Consider the probability space $\brak{\Omega, \mathcal{G}, P}$ where $\Omega = [0,2]$ and $\mathcal{G} = \cbrak{\phi, \Omega, [0,1], (1,2]}$. Let $X$ and $Y$ be two functions on $\Omega$ defined as
\begin{align*}
    X(\omega) = 
    \begin{cases}
        1 & \text{if }\omega \in [0, 1]\\
        2 & \text{if }\omega \in (1, 2]
    \end{cases}
\end{align*}
and
\begin{align*}
    Y(\omega) = 
    \begin{cases}
        2 & \text{if }\omega \in [0, 1.5]\\
        3 & \text{if }\omega \in (1.5, 2].
    \end{cases}
\end{align*}
Then which one of the following statements is true?
\begin{enumerate}
    \item [(A)] $X$ is a random variable with respect to $\mathcal{G}$, but $Y$ is not a random variable with respect to $\mathcal{G}$.
    \item [(B)] $Y$ is a random variable with respect to $\mathcal{G}$, but $X$ is not a random variable with respect to $\mathcal{G}$.
    \item [(C)] Neither $X$ nor $Y$ is a random variable with respect to $\mathcal{G}$.
    \item [(D)] Both $X$ and $Y$ are random variables with respect to $\mathcal{G}$.
\end{enumerate} \hfill (GATE ST 2023)\\
\solution
%\input{gate/ST/2023/14/main.tex}
	\item  A die is loaded in such a way that each odd number is twice as likely to occur as
each even number. Find $P(G)$, where $G$ is the event that a number greater than
3 occurs on a single roll of the die.
\\
\solution
		%\input{exemplar/11/16/3/5/main.tex}
	\item All the jacks, queens and kings are removed from a deck of 52 playing cards. The remaining cards are well shuffled and then one card is drawn at random. Giving ace a value 1 similar value for other cards, find the probability that the card has a value 
		\begin{enumerate}
			\item 7
			\item greater than 7
			\item less than 7
		\end{enumerate}
		%\input{exemplar/10/13/3/30/main.tex}
  \item A Lot consists of 48 mobile phones of which 42 are good, 3 have only minor defects and 3 have major defects.Varnika will buy a phone if it is good but the trader will only buy a mobile if it has no major defects. One phone is selected at random from the lot. What is the probability that it is
\begin{enumerate}
	\item acceptable to Varnika?
            \item acceptable to the trader?
\end{enumerate}
\solution
	%\input{exemplar/10/13/3/40/main.tex}
 \item A student says that if you throw a die, it will show up 1 or not 1. Therefore, the probability of getting 1 and the probability of getting 'not 1' each is equal to $\frac{1}{2}$. Is this correct? Give reasons.\\
 \solution
        %\input{exemplar/10/13/2/9/main.tex}
   \item Four candidates A, B, C, D have ap-
plied for the assignment to coach a school cricket
team. If A is twice as likely to be selected as B, and
B and C are given about the same chance of being
selected, while C is twice as likely to be selected
as D, what are the probabilities that
\begin{enumerate}
\item C will be selected?
\item A will not be selected?
\end{enumerate}
	%\input{exemplar/11/16/3/9/main.tex}
 \item A bag contain 24 balls of which $x$ balls are red, $2x$ are white and $3x$ are blue. A ball is selected at random, What is the probability that it is
\begin{enumerate}[label=\alph*)]
\item not red ?
\item white ?
\end{enumerate}
%\input{exemplar/10/13/3/41/main.tex}
If the letters of the word ASSASSINATION are arranged at random. Find the Probability that
\begin{enumerate}[label=(\alph*)]
\item Four $S's$ come consecutively in the word
\item Two  $I's$ and two $N's$ come together
\item All $A's$ are not coming together
\item No two $A's$ are coming together
\end{enumerate}
%\input{exemplar/11/16/3/14/main.tex}
	\item One urn contains two black balls (labelled B1 and B2) and one white ball. A
	second urn contains one black ball and two white balls (labelled W1 and W2).
	Suppose the following experiment is performed. One of the two urns is chosen
	at random. Next a ball is randomly chosen from the urn. Then a second ball is
	chosen at random from the same urn without replacing the first ball.
	
	\begin{enumerate}
	\item What is the probability that two black balls are chosen?
	
	\item What is the probability that two balls of opposite colour are chosen?
	\end{enumerate}
	\solution
	%\input{exemplar/11/16/3/12/main1.tex}
\end{enumerate}

		%
\item 
Out of 100 students, two sections of 40 and 60 are formed. If you and your friend are among the 100 students, what is the probability that
\begin{enumerate}
\item you both enter the same section?
\item you both enter the different sections?
\end{enumerate}
\solution
		%\begin{enumerate}[label=\thesection.\arabic*,ref=\thesection.\theenumi]
	\item One card is drawn from a well-shuffled deck of 52 cards. Find the probability of getting
\begin{enumerate}
\item A king of red colour 
\item A face card 
\item A red face card
\item The jack of hearts
\item A spade
\item The queen of diamonds

\end{enumerate}
\solution
		%\input{ncert/10/15/1/14/main.tex}
	\item Five cards—the ten, jack, queen, king and ace of diamonds, are well-shuffled with their face downwards. One card is then picked up at random.
\begin{enumerate}
\item
What is the probability that the card is the queen? 
\item
If the queen is drawn and put aside, what is the probability that the second card picked up is (a) an ace? (b) a queen?\\
\end{enumerate}
\solution
		%\input{ncert/10/15/1/15/defs.tex}
	\item A bag contains $5$ red balls and some blue balls. If the probability of drawing a blue ball is double that if a red ball, determine the number of blue balls in the bag. 
		\\
\solution
		%\input{ncert/10/15/2/3/defs.tex}
	\item A card is selected from a pack of 52 cards.
 \begin{enumerate}[label=(\alph*)] 
                 \item How many points are there in the sample space?
                 \item Calculate the probability that the card is an ace of spades.
                 \item Calculate the probability that the card is (i) an ace and (ii) black card.
 \end{enumerate}
\solution
		%\input{ncert/11/16/3/4/main.tex}
\item Four cards are drawn from a well-shuffled deck of 52 cards. What is the probability of obtaining 3 diamonds and one spade.
\\
\solution
		%\input{ncert/11/16/4/2/defs.tex}
\item In a certain lottery 10,000 tickets are sold and ten equal prizes are awarded. What is the probability of not getting a prize if you buy (a) one ticket (b) two tickets (c) 10 tickets ?	
\\
\solution
		%\input{ncert/11/16/4/4/defs.tex}
		%
\item 
Out of 100 students, two sections of 40 and 60 are formed. If you and your friend are among the 100 students, what is the probability that
\begin{enumerate}
\item you both enter the same section?
\item you both enter the different sections?
\end{enumerate}
\solution
		%\input{ncert/11/16/4/5/defs.tex}
	\item 
The number lock of a suitcase has 4 wheels each labelled with ten digits i.e. from 0 to 9.The lock opens with a sequence of four digits with no repeats.What is the probability of a person getting the right sequence to open the suitcase.
\\
\solution
		%\input{ncert/11/16/4/10/defs.tex}
		%
\item 
Two cards are drawn at random and without replacement from a pack of 52 playing cards. Find the probability that both the cards are black.
\\
\solution
		%\input{ncert/12/13/2/2/defs.tex}
		\item A box of oranges is inspected by examining three randomly selected oranges drawn without replacement. If all the three oranges are good, the box is approved for sale, otherwise, it is rejected. Find the probability that a box containing 15 oranges out of which 12 are good and 3 are bad ones will be approved for sale.
		\label{ncert/12/13/2/3/defs.tex}
		\item Two balls are drawn at random with replacement from a box containing 10 black and 8 red balls. Find the probability that
		\label{ncert/12/13/2/12}
\begin{enumerate}
\item both balls are red.
\item first ball is black and second is red.
\item one of them is black and other is red.
\end{enumerate}

\item In a hostel, 60\% of the students read Hindi newspaper, 40\% read English newspaper and 20\% read both Hindi and English newspapers. A student is selected at random.
		\label{ncert/12/13/2/15}
\begin{enumerate}
\item Find the probability that she reads neither Hindi nor English newspapers.
\item If she reads Hindi newspaper, find the probability that she reads English newspaper.
\item If she reads English newspaper, find the probability that she reads Hindi newspaper.\\
\end{enumerate}
\item The probability of obtaining an even prime number on each die, when a pair of dice is rolled is 
\begin{enumerate}
    \item $0$ 
    
    \item $\frac{1}{3}$ 
    
    \item $\frac{1}{12}$ 
    
    \item $\frac{1}{36}$ 
\end{enumerate}
\solution
		%\input{ncert/12/13/2/17/defs.tex}
	\item A bag contains 4 red and 4 black balls, another bag contains 2 red and 6 black balls. One of the two bags is selected at random and a ball is drawn from the bag which is found to be red. Find the probability that the ball is drawn from the first bag.
\\
\solution
		%\input{ncert/12/13/3/2/main.tex}
  \item
  Cards with numbers 2 to 101 are placed in a box. A card is selected at random.Find the probability that the card has
\begin{enumerate}[label=(\roman*)]
	\item an even number 
	\item a square number
\end{enumerate}
\solution
%\input{exemplar/10/13/3/32/main.tex}
\item
The king, queen and jack of clubs are removed from a deck of 52 playing cards and then well shuffled. Now one card is drawn at random from the remaining cards.  Determine the probability that the card is
\begin{enumerate}[label=(\roman*)]
\item a club
\item 10 of hearts
\end{enumerate}
\solution
%\input{exemplar/10/13/3/29/main.tex}
\item A team of medical students doing their internship have to assist during surgeries
at a city hospital. The probabilities of surgeries rated as very complex, complex,
routine, simple or very simple are respectively, 0.15, 0.20, 0.31, 0.26, .08. Find
the probabilities that a particular surgery will be rated
\begin{enumerate}
	\item complex or very complex;
	\item neither very complex nor very simple;
	\item routine or complex
	\item routine or simple
\end{enumerate}
\solution
%\input{exemplar/11/16/3/8(1)/main.tex}
\item A card is selected from a pack of 52 cards.
\begin{enumerate}[label=(\alph*)]
    \item How many points are there in the sample space?
    \item Calculate the probability that the card is an ace of spades.
    \item Calculate the probability that the card is (i) an ace and (ii) black card.
\end{enumerate}
\solution
%\input{exemplar/11/16/3/4/main2.tex}
\item The probability that a non leap year selected at random will contain 53 sundays.
\\
\solution
%\input{exemplar/10/13/1/19/main.tex}
\item One of the four persons John, Rita, Aslam or Gurpreet will be promoted next
month. Consequently the sample space consists of four elementary outcomes
S = {John promoted, Rita promoted, Aslam promoted, Gurpreet promoted}
You are told that the chances of John’s promotion is same as that of Gurpreet,
Rita’s chances of promotion are twice as likely as Johns. Aslam’s chances are
four times that of John.
\begin{enumerate}
	\item Determine
	\begin{enumerate}
		\item P (John promoted)
		\item P (Rita promoted)
		\item P (Aslam promoted)
		\item P (Gurpreet promoted)
	\end{enumerate}
	\item If A = {John promoted or Gurpreet promoted}, find P (A).
\end{enumerate}
\solution
%\input{exemplar/11/16/3/10/main.tex}
\item A card is drawn from a deck of 52 cards. Find the probability of getting a king or a heart or a red card.\\
\solution
%\input{exemplar/11/16/3/15/main.tex}
\item The probability that a student will pass his examination is 0.73, the probability of
the student getting a compartment is 0.13, and the probability that the student will
either pass or get compartment is 0.96. State True or False.\\
\solution
%\input{exemplar/11/16/3/31/main.tex}
\item A card is selected from a pack of 52 cards\\
\begin{enumerate}[label=(\alph*)]
\item How many points are there in the sample space?
\item Calculate the probability that the cards is an ace of spades.
\item Calculate the probability that the card is (i) an ace (ii)black card.\\
\end{enumerate}
%\input{ncert/11/16/3/4_1/Prob_4.tex}
\item In a non-leap year, the probability of having 53 tuesdays or 53 wednesdays is\\
\solution
%\input{exemplar/11/16/3/18/main.tex}
\item There are 1000 sealed envelopes in a box, 10 of them contain a cash prize of
Rs 100 each, 100 of them contain a cash prize of Rs 50 each and 200 of them
contain a cash prize of Rs 10 each and rest do not contain any cash prize. If they
are well shuffled and an envelope is picked up out, what is the probability that it
contains no cash prize?\\
\solution
%\input{exemplar/10/13/3/34/main.tex}
\item 
A die is thrown and a card is selected at random from a deck of 52 playing cards. The probability of getting an even number on the die and a spade card.\\
\solution
%\input{exemplar/12/13/3/78/main.tex}
\item
If 4-digit numbers greater than 5,000 are randomly formed from the digits 0, 1, 3, 5, and 7, what is the probability of forming a number divisible by 5 when:
\begin{enumerate}
    \item The digits are repeated?
    \item The repetition of digits is not allowed?
\end{enumerate}
\solution
%\input{ncert/11/16/4/9/main.tex}
\item Consider the probability space $\brak{\Omega, \mathcal{G}, P}$ where $\Omega = [0,2]$ and $\mathcal{G} = \cbrak{\phi, \Omega, [0,1], (1,2]}$. Let $X$ and $Y$ be two functions on $\Omega$ defined as
\begin{align*}
    X(\omega) = 
    \begin{cases}
        1 & \text{if }\omega \in [0, 1]\\
        2 & \text{if }\omega \in (1, 2]
    \end{cases}
\end{align*}
and
\begin{align*}
    Y(\omega) = 
    \begin{cases}
        2 & \text{if }\omega \in [0, 1.5]\\
        3 & \text{if }\omega \in (1.5, 2].
    \end{cases}
\end{align*}
Then which one of the following statements is true?
\begin{enumerate}
    \item [(A)] $X$ is a random variable with respect to $\mathcal{G}$, but $Y$ is not a random variable with respect to $\mathcal{G}$.
    \item [(B)] $Y$ is a random variable with respect to $\mathcal{G}$, but $X$ is not a random variable with respect to $\mathcal{G}$.
    \item [(C)] Neither $X$ nor $Y$ is a random variable with respect to $\mathcal{G}$.
    \item [(D)] Both $X$ and $Y$ are random variables with respect to $\mathcal{G}$.
\end{enumerate} \hfill (GATE ST 2023)\\
\solution
%\input{gate/ST/2023/14/main.tex}
	\item  A die is loaded in such a way that each odd number is twice as likely to occur as
each even number. Find $P(G)$, where $G$ is the event that a number greater than
3 occurs on a single roll of the die.
\\
\solution
		%\input{exemplar/11/16/3/5/main.tex}
	\item All the jacks, queens and kings are removed from a deck of 52 playing cards. The remaining cards are well shuffled and then one card is drawn at random. Giving ace a value 1 similar value for other cards, find the probability that the card has a value 
		\begin{enumerate}
			\item 7
			\item greater than 7
			\item less than 7
		\end{enumerate}
		%\input{exemplar/10/13/3/30/main.tex}
  \item A Lot consists of 48 mobile phones of which 42 are good, 3 have only minor defects and 3 have major defects.Varnika will buy a phone if it is good but the trader will only buy a mobile if it has no major defects. One phone is selected at random from the lot. What is the probability that it is
\begin{enumerate}
	\item acceptable to Varnika?
            \item acceptable to the trader?
\end{enumerate}
\solution
	%\input{exemplar/10/13/3/40/main.tex}
 \item A student says that if you throw a die, it will show up 1 or not 1. Therefore, the probability of getting 1 and the probability of getting 'not 1' each is equal to $\frac{1}{2}$. Is this correct? Give reasons.\\
 \solution
        %\input{exemplar/10/13/2/9/main.tex}
   \item Four candidates A, B, C, D have ap-
plied for the assignment to coach a school cricket
team. If A is twice as likely to be selected as B, and
B and C are given about the same chance of being
selected, while C is twice as likely to be selected
as D, what are the probabilities that
\begin{enumerate}
\item C will be selected?
\item A will not be selected?
\end{enumerate}
	%\input{exemplar/11/16/3/9/main.tex}
 \item A bag contain 24 balls of which $x$ balls are red, $2x$ are white and $3x$ are blue. A ball is selected at random, What is the probability that it is
\begin{enumerate}[label=\alph*)]
\item not red ?
\item white ?
\end{enumerate}
%\input{exemplar/10/13/3/41/main.tex}
If the letters of the word ASSASSINATION are arranged at random. Find the Probability that
\begin{enumerate}[label=(\alph*)]
\item Four $S's$ come consecutively in the word
\item Two  $I's$ and two $N's$ come together
\item All $A's$ are not coming together
\item No two $A's$ are coming together
\end{enumerate}
%\input{exemplar/11/16/3/14/main.tex}
	\item One urn contains two black balls (labelled B1 and B2) and one white ball. A
	second urn contains one black ball and two white balls (labelled W1 and W2).
	Suppose the following experiment is performed. One of the two urns is chosen
	at random. Next a ball is randomly chosen from the urn. Then a second ball is
	chosen at random from the same urn without replacing the first ball.
	
	\begin{enumerate}
	\item What is the probability that two black balls are chosen?
	
	\item What is the probability that two balls of opposite colour are chosen?
	\end{enumerate}
	\solution
	%\input{exemplar/11/16/3/12/main1.tex}
\end{enumerate}

	\item 
The number lock of a suitcase has 4 wheels each labelled with ten digits i.e. from 0 to 9.The lock opens with a sequence of four digits with no repeats.What is the probability of a person getting the right sequence to open the suitcase.
\\
\solution
		%\begin{enumerate}[label=\thesection.\arabic*,ref=\thesection.\theenumi]
	\item One card is drawn from a well-shuffled deck of 52 cards. Find the probability of getting
\begin{enumerate}
\item A king of red colour 
\item A face card 
\item A red face card
\item The jack of hearts
\item A spade
\item The queen of diamonds

\end{enumerate}
\solution
		%\input{ncert/10/15/1/14/main.tex}
	\item Five cards—the ten, jack, queen, king and ace of diamonds, are well-shuffled with their face downwards. One card is then picked up at random.
\begin{enumerate}
\item
What is the probability that the card is the queen? 
\item
If the queen is drawn and put aside, what is the probability that the second card picked up is (a) an ace? (b) a queen?\\
\end{enumerate}
\solution
		%\input{ncert/10/15/1/15/defs.tex}
	\item A bag contains $5$ red balls and some blue balls. If the probability of drawing a blue ball is double that if a red ball, determine the number of blue balls in the bag. 
		\\
\solution
		%\input{ncert/10/15/2/3/defs.tex}
	\item A card is selected from a pack of 52 cards.
 \begin{enumerate}[label=(\alph*)] 
                 \item How many points are there in the sample space?
                 \item Calculate the probability that the card is an ace of spades.
                 \item Calculate the probability that the card is (i) an ace and (ii) black card.
 \end{enumerate}
\solution
		%\input{ncert/11/16/3/4/main.tex}
\item Four cards are drawn from a well-shuffled deck of 52 cards. What is the probability of obtaining 3 diamonds and one spade.
\\
\solution
		%\input{ncert/11/16/4/2/defs.tex}
\item In a certain lottery 10,000 tickets are sold and ten equal prizes are awarded. What is the probability of not getting a prize if you buy (a) one ticket (b) two tickets (c) 10 tickets ?	
\\
\solution
		%\input{ncert/11/16/4/4/defs.tex}
		%
\item 
Out of 100 students, two sections of 40 and 60 are formed. If you and your friend are among the 100 students, what is the probability that
\begin{enumerate}
\item you both enter the same section?
\item you both enter the different sections?
\end{enumerate}
\solution
		%\input{ncert/11/16/4/5/defs.tex}
	\item 
The number lock of a suitcase has 4 wheels each labelled with ten digits i.e. from 0 to 9.The lock opens with a sequence of four digits with no repeats.What is the probability of a person getting the right sequence to open the suitcase.
\\
\solution
		%\input{ncert/11/16/4/10/defs.tex}
		%
\item 
Two cards are drawn at random and without replacement from a pack of 52 playing cards. Find the probability that both the cards are black.
\\
\solution
		%\input{ncert/12/13/2/2/defs.tex}
		\item A box of oranges is inspected by examining three randomly selected oranges drawn without replacement. If all the three oranges are good, the box is approved for sale, otherwise, it is rejected. Find the probability that a box containing 15 oranges out of which 12 are good and 3 are bad ones will be approved for sale.
		\label{ncert/12/13/2/3/defs.tex}
		\item Two balls are drawn at random with replacement from a box containing 10 black and 8 red balls. Find the probability that
		\label{ncert/12/13/2/12}
\begin{enumerate}
\item both balls are red.
\item first ball is black and second is red.
\item one of them is black and other is red.
\end{enumerate}

\item In a hostel, 60\% of the students read Hindi newspaper, 40\% read English newspaper and 20\% read both Hindi and English newspapers. A student is selected at random.
		\label{ncert/12/13/2/15}
\begin{enumerate}
\item Find the probability that she reads neither Hindi nor English newspapers.
\item If she reads Hindi newspaper, find the probability that she reads English newspaper.
\item If she reads English newspaper, find the probability that she reads Hindi newspaper.\\
\end{enumerate}
\item The probability of obtaining an even prime number on each die, when a pair of dice is rolled is 
\begin{enumerate}
    \item $0$ 
    
    \item $\frac{1}{3}$ 
    
    \item $\frac{1}{12}$ 
    
    \item $\frac{1}{36}$ 
\end{enumerate}
\solution
		%\input{ncert/12/13/2/17/defs.tex}
	\item A bag contains 4 red and 4 black balls, another bag contains 2 red and 6 black balls. One of the two bags is selected at random and a ball is drawn from the bag which is found to be red. Find the probability that the ball is drawn from the first bag.
\\
\solution
		%\input{ncert/12/13/3/2/main.tex}
  \item
  Cards with numbers 2 to 101 are placed in a box. A card is selected at random.Find the probability that the card has
\begin{enumerate}[label=(\roman*)]
	\item an even number 
	\item a square number
\end{enumerate}
\solution
%\input{exemplar/10/13/3/32/main.tex}
\item
The king, queen and jack of clubs are removed from a deck of 52 playing cards and then well shuffled. Now one card is drawn at random from the remaining cards.  Determine the probability that the card is
\begin{enumerate}[label=(\roman*)]
\item a club
\item 10 of hearts
\end{enumerate}
\solution
%\input{exemplar/10/13/3/29/main.tex}
\item A team of medical students doing their internship have to assist during surgeries
at a city hospital. The probabilities of surgeries rated as very complex, complex,
routine, simple or very simple are respectively, 0.15, 0.20, 0.31, 0.26, .08. Find
the probabilities that a particular surgery will be rated
\begin{enumerate}
	\item complex or very complex;
	\item neither very complex nor very simple;
	\item routine or complex
	\item routine or simple
\end{enumerate}
\solution
%\input{exemplar/11/16/3/8(1)/main.tex}
\item A card is selected from a pack of 52 cards.
\begin{enumerate}[label=(\alph*)]
    \item How many points are there in the sample space?
    \item Calculate the probability that the card is an ace of spades.
    \item Calculate the probability that the card is (i) an ace and (ii) black card.
\end{enumerate}
\solution
%\input{exemplar/11/16/3/4/main2.tex}
\item The probability that a non leap year selected at random will contain 53 sundays.
\\
\solution
%\input{exemplar/10/13/1/19/main.tex}
\item One of the four persons John, Rita, Aslam or Gurpreet will be promoted next
month. Consequently the sample space consists of four elementary outcomes
S = {John promoted, Rita promoted, Aslam promoted, Gurpreet promoted}
You are told that the chances of John’s promotion is same as that of Gurpreet,
Rita’s chances of promotion are twice as likely as Johns. Aslam’s chances are
four times that of John.
\begin{enumerate}
	\item Determine
	\begin{enumerate}
		\item P (John promoted)
		\item P (Rita promoted)
		\item P (Aslam promoted)
		\item P (Gurpreet promoted)
	\end{enumerate}
	\item If A = {John promoted or Gurpreet promoted}, find P (A).
\end{enumerate}
\solution
%\input{exemplar/11/16/3/10/main.tex}
\item A card is drawn from a deck of 52 cards. Find the probability of getting a king or a heart or a red card.\\
\solution
%\input{exemplar/11/16/3/15/main.tex}
\item The probability that a student will pass his examination is 0.73, the probability of
the student getting a compartment is 0.13, and the probability that the student will
either pass or get compartment is 0.96. State True or False.\\
\solution
%\input{exemplar/11/16/3/31/main.tex}
\item A card is selected from a pack of 52 cards\\
\begin{enumerate}[label=(\alph*)]
\item How many points are there in the sample space?
\item Calculate the probability that the cards is an ace of spades.
\item Calculate the probability that the card is (i) an ace (ii)black card.\\
\end{enumerate}
%\input{ncert/11/16/3/4_1/Prob_4.tex}
\item In a non-leap year, the probability of having 53 tuesdays or 53 wednesdays is\\
\solution
%\input{exemplar/11/16/3/18/main.tex}
\item There are 1000 sealed envelopes in a box, 10 of them contain a cash prize of
Rs 100 each, 100 of them contain a cash prize of Rs 50 each and 200 of them
contain a cash prize of Rs 10 each and rest do not contain any cash prize. If they
are well shuffled and an envelope is picked up out, what is the probability that it
contains no cash prize?\\
\solution
%\input{exemplar/10/13/3/34/main.tex}
\item 
A die is thrown and a card is selected at random from a deck of 52 playing cards. The probability of getting an even number on the die and a spade card.\\
\solution
%\input{exemplar/12/13/3/78/main.tex}
\item
If 4-digit numbers greater than 5,000 are randomly formed from the digits 0, 1, 3, 5, and 7, what is the probability of forming a number divisible by 5 when:
\begin{enumerate}
    \item The digits are repeated?
    \item The repetition of digits is not allowed?
\end{enumerate}
\solution
%\input{ncert/11/16/4/9/main.tex}
\item Consider the probability space $\brak{\Omega, \mathcal{G}, P}$ where $\Omega = [0,2]$ and $\mathcal{G} = \cbrak{\phi, \Omega, [0,1], (1,2]}$. Let $X$ and $Y$ be two functions on $\Omega$ defined as
\begin{align*}
    X(\omega) = 
    \begin{cases}
        1 & \text{if }\omega \in [0, 1]\\
        2 & \text{if }\omega \in (1, 2]
    \end{cases}
\end{align*}
and
\begin{align*}
    Y(\omega) = 
    \begin{cases}
        2 & \text{if }\omega \in [0, 1.5]\\
        3 & \text{if }\omega \in (1.5, 2].
    \end{cases}
\end{align*}
Then which one of the following statements is true?
\begin{enumerate}
    \item [(A)] $X$ is a random variable with respect to $\mathcal{G}$, but $Y$ is not a random variable with respect to $\mathcal{G}$.
    \item [(B)] $Y$ is a random variable with respect to $\mathcal{G}$, but $X$ is not a random variable with respect to $\mathcal{G}$.
    \item [(C)] Neither $X$ nor $Y$ is a random variable with respect to $\mathcal{G}$.
    \item [(D)] Both $X$ and $Y$ are random variables with respect to $\mathcal{G}$.
\end{enumerate} \hfill (GATE ST 2023)\\
\solution
%\input{gate/ST/2023/14/main.tex}
	\item  A die is loaded in such a way that each odd number is twice as likely to occur as
each even number. Find $P(G)$, where $G$ is the event that a number greater than
3 occurs on a single roll of the die.
\\
\solution
		%\input{exemplar/11/16/3/5/main.tex}
	\item All the jacks, queens and kings are removed from a deck of 52 playing cards. The remaining cards are well shuffled and then one card is drawn at random. Giving ace a value 1 similar value for other cards, find the probability that the card has a value 
		\begin{enumerate}
			\item 7
			\item greater than 7
			\item less than 7
		\end{enumerate}
		%\input{exemplar/10/13/3/30/main.tex}
  \item A Lot consists of 48 mobile phones of which 42 are good, 3 have only minor defects and 3 have major defects.Varnika will buy a phone if it is good but the trader will only buy a mobile if it has no major defects. One phone is selected at random from the lot. What is the probability that it is
\begin{enumerate}
	\item acceptable to Varnika?
            \item acceptable to the trader?
\end{enumerate}
\solution
	%\input{exemplar/10/13/3/40/main.tex}
 \item A student says that if you throw a die, it will show up 1 or not 1. Therefore, the probability of getting 1 and the probability of getting 'not 1' each is equal to $\frac{1}{2}$. Is this correct? Give reasons.\\
 \solution
        %\input{exemplar/10/13/2/9/main.tex}
   \item Four candidates A, B, C, D have ap-
plied for the assignment to coach a school cricket
team. If A is twice as likely to be selected as B, and
B and C are given about the same chance of being
selected, while C is twice as likely to be selected
as D, what are the probabilities that
\begin{enumerate}
\item C will be selected?
\item A will not be selected?
\end{enumerate}
	%\input{exemplar/11/16/3/9/main.tex}
 \item A bag contain 24 balls of which $x$ balls are red, $2x$ are white and $3x$ are blue. A ball is selected at random, What is the probability that it is
\begin{enumerate}[label=\alph*)]
\item not red ?
\item white ?
\end{enumerate}
%\input{exemplar/10/13/3/41/main.tex}
If the letters of the word ASSASSINATION are arranged at random. Find the Probability that
\begin{enumerate}[label=(\alph*)]
\item Four $S's$ come consecutively in the word
\item Two  $I's$ and two $N's$ come together
\item All $A's$ are not coming together
\item No two $A's$ are coming together
\end{enumerate}
%\input{exemplar/11/16/3/14/main.tex}
	\item One urn contains two black balls (labelled B1 and B2) and one white ball. A
	second urn contains one black ball and two white balls (labelled W1 and W2).
	Suppose the following experiment is performed. One of the two urns is chosen
	at random. Next a ball is randomly chosen from the urn. Then a second ball is
	chosen at random from the same urn without replacing the first ball.
	
	\begin{enumerate}
	\item What is the probability that two black balls are chosen?
	
	\item What is the probability that two balls of opposite colour are chosen?
	\end{enumerate}
	\solution
	%\input{exemplar/11/16/3/12/main1.tex}
\end{enumerate}

		%
\item 
Two cards are drawn at random and without replacement from a pack of 52 playing cards. Find the probability that both the cards are black.
\\
\solution
		%\begin{enumerate}[label=\thesection.\arabic*,ref=\thesection.\theenumi]
	\item One card is drawn from a well-shuffled deck of 52 cards. Find the probability of getting
\begin{enumerate}
\item A king of red colour 
\item A face card 
\item A red face card
\item The jack of hearts
\item A spade
\item The queen of diamonds

\end{enumerate}
\solution
		%\input{ncert/10/15/1/14/main.tex}
	\item Five cards—the ten, jack, queen, king and ace of diamonds, are well-shuffled with their face downwards. One card is then picked up at random.
\begin{enumerate}
\item
What is the probability that the card is the queen? 
\item
If the queen is drawn and put aside, what is the probability that the second card picked up is (a) an ace? (b) a queen?\\
\end{enumerate}
\solution
		%\input{ncert/10/15/1/15/defs.tex}
	\item A bag contains $5$ red balls and some blue balls. If the probability of drawing a blue ball is double that if a red ball, determine the number of blue balls in the bag. 
		\\
\solution
		%\input{ncert/10/15/2/3/defs.tex}
	\item A card is selected from a pack of 52 cards.
 \begin{enumerate}[label=(\alph*)] 
                 \item How many points are there in the sample space?
                 \item Calculate the probability that the card is an ace of spades.
                 \item Calculate the probability that the card is (i) an ace and (ii) black card.
 \end{enumerate}
\solution
		%\input{ncert/11/16/3/4/main.tex}
\item Four cards are drawn from a well-shuffled deck of 52 cards. What is the probability of obtaining 3 diamonds and one spade.
\\
\solution
		%\input{ncert/11/16/4/2/defs.tex}
\item In a certain lottery 10,000 tickets are sold and ten equal prizes are awarded. What is the probability of not getting a prize if you buy (a) one ticket (b) two tickets (c) 10 tickets ?	
\\
\solution
		%\input{ncert/11/16/4/4/defs.tex}
		%
\item 
Out of 100 students, two sections of 40 and 60 are formed. If you and your friend are among the 100 students, what is the probability that
\begin{enumerate}
\item you both enter the same section?
\item you both enter the different sections?
\end{enumerate}
\solution
		%\input{ncert/11/16/4/5/defs.tex}
	\item 
The number lock of a suitcase has 4 wheels each labelled with ten digits i.e. from 0 to 9.The lock opens with a sequence of four digits with no repeats.What is the probability of a person getting the right sequence to open the suitcase.
\\
\solution
		%\input{ncert/11/16/4/10/defs.tex}
		%
\item 
Two cards are drawn at random and without replacement from a pack of 52 playing cards. Find the probability that both the cards are black.
\\
\solution
		%\input{ncert/12/13/2/2/defs.tex}
		\item A box of oranges is inspected by examining three randomly selected oranges drawn without replacement. If all the three oranges are good, the box is approved for sale, otherwise, it is rejected. Find the probability that a box containing 15 oranges out of which 12 are good and 3 are bad ones will be approved for sale.
		\label{ncert/12/13/2/3/defs.tex}
		\item Two balls are drawn at random with replacement from a box containing 10 black and 8 red balls. Find the probability that
		\label{ncert/12/13/2/12}
\begin{enumerate}
\item both balls are red.
\item first ball is black and second is red.
\item one of them is black and other is red.
\end{enumerate}

\item In a hostel, 60\% of the students read Hindi newspaper, 40\% read English newspaper and 20\% read both Hindi and English newspapers. A student is selected at random.
		\label{ncert/12/13/2/15}
\begin{enumerate}
\item Find the probability that she reads neither Hindi nor English newspapers.
\item If she reads Hindi newspaper, find the probability that she reads English newspaper.
\item If she reads English newspaper, find the probability that she reads Hindi newspaper.\\
\end{enumerate}
\item The probability of obtaining an even prime number on each die, when a pair of dice is rolled is 
\begin{enumerate}
    \item $0$ 
    
    \item $\frac{1}{3}$ 
    
    \item $\frac{1}{12}$ 
    
    \item $\frac{1}{36}$ 
\end{enumerate}
\solution
		%\input{ncert/12/13/2/17/defs.tex}
	\item A bag contains 4 red and 4 black balls, another bag contains 2 red and 6 black balls. One of the two bags is selected at random and a ball is drawn from the bag which is found to be red. Find the probability that the ball is drawn from the first bag.
\\
\solution
		%\input{ncert/12/13/3/2/main.tex}
  \item
  Cards with numbers 2 to 101 are placed in a box. A card is selected at random.Find the probability that the card has
\begin{enumerate}[label=(\roman*)]
	\item an even number 
	\item a square number
\end{enumerate}
\solution
%\input{exemplar/10/13/3/32/main.tex}
\item
The king, queen and jack of clubs are removed from a deck of 52 playing cards and then well shuffled. Now one card is drawn at random from the remaining cards.  Determine the probability that the card is
\begin{enumerate}[label=(\roman*)]
\item a club
\item 10 of hearts
\end{enumerate}
\solution
%\input{exemplar/10/13/3/29/main.tex}
\item A team of medical students doing their internship have to assist during surgeries
at a city hospital. The probabilities of surgeries rated as very complex, complex,
routine, simple or very simple are respectively, 0.15, 0.20, 0.31, 0.26, .08. Find
the probabilities that a particular surgery will be rated
\begin{enumerate}
	\item complex or very complex;
	\item neither very complex nor very simple;
	\item routine or complex
	\item routine or simple
\end{enumerate}
\solution
%\input{exemplar/11/16/3/8(1)/main.tex}
\item A card is selected from a pack of 52 cards.
\begin{enumerate}[label=(\alph*)]
    \item How many points are there in the sample space?
    \item Calculate the probability that the card is an ace of spades.
    \item Calculate the probability that the card is (i) an ace and (ii) black card.
\end{enumerate}
\solution
%\input{exemplar/11/16/3/4/main2.tex}
\item The probability that a non leap year selected at random will contain 53 sundays.
\\
\solution
%\input{exemplar/10/13/1/19/main.tex}
\item One of the four persons John, Rita, Aslam or Gurpreet will be promoted next
month. Consequently the sample space consists of four elementary outcomes
S = {John promoted, Rita promoted, Aslam promoted, Gurpreet promoted}
You are told that the chances of John’s promotion is same as that of Gurpreet,
Rita’s chances of promotion are twice as likely as Johns. Aslam’s chances are
four times that of John.
\begin{enumerate}
	\item Determine
	\begin{enumerate}
		\item P (John promoted)
		\item P (Rita promoted)
		\item P (Aslam promoted)
		\item P (Gurpreet promoted)
	\end{enumerate}
	\item If A = {John promoted or Gurpreet promoted}, find P (A).
\end{enumerate}
\solution
%\input{exemplar/11/16/3/10/main.tex}
\item A card is drawn from a deck of 52 cards. Find the probability of getting a king or a heart or a red card.\\
\solution
%\input{exemplar/11/16/3/15/main.tex}
\item The probability that a student will pass his examination is 0.73, the probability of
the student getting a compartment is 0.13, and the probability that the student will
either pass or get compartment is 0.96. State True or False.\\
\solution
%\input{exemplar/11/16/3/31/main.tex}
\item A card is selected from a pack of 52 cards\\
\begin{enumerate}[label=(\alph*)]
\item How many points are there in the sample space?
\item Calculate the probability that the cards is an ace of spades.
\item Calculate the probability that the card is (i) an ace (ii)black card.\\
\end{enumerate}
%\input{ncert/11/16/3/4_1/Prob_4.tex}
\item In a non-leap year, the probability of having 53 tuesdays or 53 wednesdays is\\
\solution
%\input{exemplar/11/16/3/18/main.tex}
\item There are 1000 sealed envelopes in a box, 10 of them contain a cash prize of
Rs 100 each, 100 of them contain a cash prize of Rs 50 each and 200 of them
contain a cash prize of Rs 10 each and rest do not contain any cash prize. If they
are well shuffled and an envelope is picked up out, what is the probability that it
contains no cash prize?\\
\solution
%\input{exemplar/10/13/3/34/main.tex}
\item 
A die is thrown and a card is selected at random from a deck of 52 playing cards. The probability of getting an even number on the die and a spade card.\\
\solution
%\input{exemplar/12/13/3/78/main.tex}
\item
If 4-digit numbers greater than 5,000 are randomly formed from the digits 0, 1, 3, 5, and 7, what is the probability of forming a number divisible by 5 when:
\begin{enumerate}
    \item The digits are repeated?
    \item The repetition of digits is not allowed?
\end{enumerate}
\solution
%\input{ncert/11/16/4/9/main.tex}
\item Consider the probability space $\brak{\Omega, \mathcal{G}, P}$ where $\Omega = [0,2]$ and $\mathcal{G} = \cbrak{\phi, \Omega, [0,1], (1,2]}$. Let $X$ and $Y$ be two functions on $\Omega$ defined as
\begin{align*}
    X(\omega) = 
    \begin{cases}
        1 & \text{if }\omega \in [0, 1]\\
        2 & \text{if }\omega \in (1, 2]
    \end{cases}
\end{align*}
and
\begin{align*}
    Y(\omega) = 
    \begin{cases}
        2 & \text{if }\omega \in [0, 1.5]\\
        3 & \text{if }\omega \in (1.5, 2].
    \end{cases}
\end{align*}
Then which one of the following statements is true?
\begin{enumerate}
    \item [(A)] $X$ is a random variable with respect to $\mathcal{G}$, but $Y$ is not a random variable with respect to $\mathcal{G}$.
    \item [(B)] $Y$ is a random variable with respect to $\mathcal{G}$, but $X$ is not a random variable with respect to $\mathcal{G}$.
    \item [(C)] Neither $X$ nor $Y$ is a random variable with respect to $\mathcal{G}$.
    \item [(D)] Both $X$ and $Y$ are random variables with respect to $\mathcal{G}$.
\end{enumerate} \hfill (GATE ST 2023)\\
\solution
%\input{gate/ST/2023/14/main.tex}
	\item  A die is loaded in such a way that each odd number is twice as likely to occur as
each even number. Find $P(G)$, where $G$ is the event that a number greater than
3 occurs on a single roll of the die.
\\
\solution
		%\input{exemplar/11/16/3/5/main.tex}
	\item All the jacks, queens and kings are removed from a deck of 52 playing cards. The remaining cards are well shuffled and then one card is drawn at random. Giving ace a value 1 similar value for other cards, find the probability that the card has a value 
		\begin{enumerate}
			\item 7
			\item greater than 7
			\item less than 7
		\end{enumerate}
		%\input{exemplar/10/13/3/30/main.tex}
  \item A Lot consists of 48 mobile phones of which 42 are good, 3 have only minor defects and 3 have major defects.Varnika will buy a phone if it is good but the trader will only buy a mobile if it has no major defects. One phone is selected at random from the lot. What is the probability that it is
\begin{enumerate}
	\item acceptable to Varnika?
            \item acceptable to the trader?
\end{enumerate}
\solution
	%\input{exemplar/10/13/3/40/main.tex}
 \item A student says that if you throw a die, it will show up 1 or not 1. Therefore, the probability of getting 1 and the probability of getting 'not 1' each is equal to $\frac{1}{2}$. Is this correct? Give reasons.\\
 \solution
        %\input{exemplar/10/13/2/9/main.tex}
   \item Four candidates A, B, C, D have ap-
plied for the assignment to coach a school cricket
team. If A is twice as likely to be selected as B, and
B and C are given about the same chance of being
selected, while C is twice as likely to be selected
as D, what are the probabilities that
\begin{enumerate}
\item C will be selected?
\item A will not be selected?
\end{enumerate}
	%\input{exemplar/11/16/3/9/main.tex}
 \item A bag contain 24 balls of which $x$ balls are red, $2x$ are white and $3x$ are blue. A ball is selected at random, What is the probability that it is
\begin{enumerate}[label=\alph*)]
\item not red ?
\item white ?
\end{enumerate}
%\input{exemplar/10/13/3/41/main.tex}
If the letters of the word ASSASSINATION are arranged at random. Find the Probability that
\begin{enumerate}[label=(\alph*)]
\item Four $S's$ come consecutively in the word
\item Two  $I's$ and two $N's$ come together
\item All $A's$ are not coming together
\item No two $A's$ are coming together
\end{enumerate}
%\input{exemplar/11/16/3/14/main.tex}
	\item One urn contains two black balls (labelled B1 and B2) and one white ball. A
	second urn contains one black ball and two white balls (labelled W1 and W2).
	Suppose the following experiment is performed. One of the two urns is chosen
	at random. Next a ball is randomly chosen from the urn. Then a second ball is
	chosen at random from the same urn without replacing the first ball.
	
	\begin{enumerate}
	\item What is the probability that two black balls are chosen?
	
	\item What is the probability that two balls of opposite colour are chosen?
	\end{enumerate}
	\solution
	%\input{exemplar/11/16/3/12/main1.tex}
\end{enumerate}

		\item A box of oranges is inspected by examining three randomly selected oranges drawn without replacement. If all the three oranges are good, the box is approved for sale, otherwise, it is rejected. Find the probability that a box containing 15 oranges out of which 12 are good and 3 are bad ones will be approved for sale.
		\label{ncert/12/13/2/3/defs.tex}
		\item Two balls are drawn at random with replacement from a box containing 10 black and 8 red balls. Find the probability that
		\label{ncert/12/13/2/12}
\begin{enumerate}
\item both balls are red.
\item first ball is black and second is red.
\item one of them is black and other is red.
\end{enumerate}

\item In a hostel, 60\% of the students read Hindi newspaper, 40\% read English newspaper and 20\% read both Hindi and English newspapers. A student is selected at random.
		\label{ncert/12/13/2/15}
\begin{enumerate}
\item Find the probability that she reads neither Hindi nor English newspapers.
\item If she reads Hindi newspaper, find the probability that she reads English newspaper.
\item If she reads English newspaper, find the probability that she reads Hindi newspaper.\\
\end{enumerate}
\item The probability of obtaining an even prime number on each die, when a pair of dice is rolled is 
\begin{enumerate}
    \item $0$ 
    
    \item $\frac{1}{3}$ 
    
    \item $\frac{1}{12}$ 
    
    \item $\frac{1}{36}$ 
\end{enumerate}
\solution
		%\begin{enumerate}[label=\thesection.\arabic*,ref=\thesection.\theenumi]
	\item One card is drawn from a well-shuffled deck of 52 cards. Find the probability of getting
\begin{enumerate}
\item A king of red colour 
\item A face card 
\item A red face card
\item The jack of hearts
\item A spade
\item The queen of diamonds

\end{enumerate}
\solution
		%\input{ncert/10/15/1/14/main.tex}
	\item Five cards—the ten, jack, queen, king and ace of diamonds, are well-shuffled with their face downwards. One card is then picked up at random.
\begin{enumerate}
\item
What is the probability that the card is the queen? 
\item
If the queen is drawn and put aside, what is the probability that the second card picked up is (a) an ace? (b) a queen?\\
\end{enumerate}
\solution
		%\input{ncert/10/15/1/15/defs.tex}
	\item A bag contains $5$ red balls and some blue balls. If the probability of drawing a blue ball is double that if a red ball, determine the number of blue balls in the bag. 
		\\
\solution
		%\input{ncert/10/15/2/3/defs.tex}
	\item A card is selected from a pack of 52 cards.
 \begin{enumerate}[label=(\alph*)] 
                 \item How many points are there in the sample space?
                 \item Calculate the probability that the card is an ace of spades.
                 \item Calculate the probability that the card is (i) an ace and (ii) black card.
 \end{enumerate}
\solution
		%\input{ncert/11/16/3/4/main.tex}
\item Four cards are drawn from a well-shuffled deck of 52 cards. What is the probability of obtaining 3 diamonds and one spade.
\\
\solution
		%\input{ncert/11/16/4/2/defs.tex}
\item In a certain lottery 10,000 tickets are sold and ten equal prizes are awarded. What is the probability of not getting a prize if you buy (a) one ticket (b) two tickets (c) 10 tickets ?	
\\
\solution
		%\input{ncert/11/16/4/4/defs.tex}
		%
\item 
Out of 100 students, two sections of 40 and 60 are formed. If you and your friend are among the 100 students, what is the probability that
\begin{enumerate}
\item you both enter the same section?
\item you both enter the different sections?
\end{enumerate}
\solution
		%\input{ncert/11/16/4/5/defs.tex}
	\item 
The number lock of a suitcase has 4 wheels each labelled with ten digits i.e. from 0 to 9.The lock opens with a sequence of four digits with no repeats.What is the probability of a person getting the right sequence to open the suitcase.
\\
\solution
		%\input{ncert/11/16/4/10/defs.tex}
		%
\item 
Two cards are drawn at random and without replacement from a pack of 52 playing cards. Find the probability that both the cards are black.
\\
\solution
		%\input{ncert/12/13/2/2/defs.tex}
		\item A box of oranges is inspected by examining three randomly selected oranges drawn without replacement. If all the three oranges are good, the box is approved for sale, otherwise, it is rejected. Find the probability that a box containing 15 oranges out of which 12 are good and 3 are bad ones will be approved for sale.
		\label{ncert/12/13/2/3/defs.tex}
		\item Two balls are drawn at random with replacement from a box containing 10 black and 8 red balls. Find the probability that
		\label{ncert/12/13/2/12}
\begin{enumerate}
\item both balls are red.
\item first ball is black and second is red.
\item one of them is black and other is red.
\end{enumerate}

\item In a hostel, 60\% of the students read Hindi newspaper, 40\% read English newspaper and 20\% read both Hindi and English newspapers. A student is selected at random.
		\label{ncert/12/13/2/15}
\begin{enumerate}
\item Find the probability that she reads neither Hindi nor English newspapers.
\item If she reads Hindi newspaper, find the probability that she reads English newspaper.
\item If she reads English newspaper, find the probability that she reads Hindi newspaper.\\
\end{enumerate}
\item The probability of obtaining an even prime number on each die, when a pair of dice is rolled is 
\begin{enumerate}
    \item $0$ 
    
    \item $\frac{1}{3}$ 
    
    \item $\frac{1}{12}$ 
    
    \item $\frac{1}{36}$ 
\end{enumerate}
\solution
		%\input{ncert/12/13/2/17/defs.tex}
	\item A bag contains 4 red and 4 black balls, another bag contains 2 red and 6 black balls. One of the two bags is selected at random and a ball is drawn from the bag which is found to be red. Find the probability that the ball is drawn from the first bag.
\\
\solution
		%\input{ncert/12/13/3/2/main.tex}
  \item
  Cards with numbers 2 to 101 are placed in a box. A card is selected at random.Find the probability that the card has
\begin{enumerate}[label=(\roman*)]
	\item an even number 
	\item a square number
\end{enumerate}
\solution
%\input{exemplar/10/13/3/32/main.tex}
\item
The king, queen and jack of clubs are removed from a deck of 52 playing cards and then well shuffled. Now one card is drawn at random from the remaining cards.  Determine the probability that the card is
\begin{enumerate}[label=(\roman*)]
\item a club
\item 10 of hearts
\end{enumerate}
\solution
%\input{exemplar/10/13/3/29/main.tex}
\item A team of medical students doing their internship have to assist during surgeries
at a city hospital. The probabilities of surgeries rated as very complex, complex,
routine, simple or very simple are respectively, 0.15, 0.20, 0.31, 0.26, .08. Find
the probabilities that a particular surgery will be rated
\begin{enumerate}
	\item complex or very complex;
	\item neither very complex nor very simple;
	\item routine or complex
	\item routine or simple
\end{enumerate}
\solution
%\input{exemplar/11/16/3/8(1)/main.tex}
\item A card is selected from a pack of 52 cards.
\begin{enumerate}[label=(\alph*)]
    \item How many points are there in the sample space?
    \item Calculate the probability that the card is an ace of spades.
    \item Calculate the probability that the card is (i) an ace and (ii) black card.
\end{enumerate}
\solution
%\input{exemplar/11/16/3/4/main2.tex}
\item The probability that a non leap year selected at random will contain 53 sundays.
\\
\solution
%\input{exemplar/10/13/1/19/main.tex}
\item One of the four persons John, Rita, Aslam or Gurpreet will be promoted next
month. Consequently the sample space consists of four elementary outcomes
S = {John promoted, Rita promoted, Aslam promoted, Gurpreet promoted}
You are told that the chances of John’s promotion is same as that of Gurpreet,
Rita’s chances of promotion are twice as likely as Johns. Aslam’s chances are
four times that of John.
\begin{enumerate}
	\item Determine
	\begin{enumerate}
		\item P (John promoted)
		\item P (Rita promoted)
		\item P (Aslam promoted)
		\item P (Gurpreet promoted)
	\end{enumerate}
	\item If A = {John promoted or Gurpreet promoted}, find P (A).
\end{enumerate}
\solution
%\input{exemplar/11/16/3/10/main.tex}
\item A card is drawn from a deck of 52 cards. Find the probability of getting a king or a heart or a red card.\\
\solution
%\input{exemplar/11/16/3/15/main.tex}
\item The probability that a student will pass his examination is 0.73, the probability of
the student getting a compartment is 0.13, and the probability that the student will
either pass or get compartment is 0.96. State True or False.\\
\solution
%\input{exemplar/11/16/3/31/main.tex}
\item A card is selected from a pack of 52 cards\\
\begin{enumerate}[label=(\alph*)]
\item How many points are there in the sample space?
\item Calculate the probability that the cards is an ace of spades.
\item Calculate the probability that the card is (i) an ace (ii)black card.\\
\end{enumerate}
%\input{ncert/11/16/3/4_1/Prob_4.tex}
\item In a non-leap year, the probability of having 53 tuesdays or 53 wednesdays is\\
\solution
%\input{exemplar/11/16/3/18/main.tex}
\item There are 1000 sealed envelopes in a box, 10 of them contain a cash prize of
Rs 100 each, 100 of them contain a cash prize of Rs 50 each and 200 of them
contain a cash prize of Rs 10 each and rest do not contain any cash prize. If they
are well shuffled and an envelope is picked up out, what is the probability that it
contains no cash prize?\\
\solution
%\input{exemplar/10/13/3/34/main.tex}
\item 
A die is thrown and a card is selected at random from a deck of 52 playing cards. The probability of getting an even number on the die and a spade card.\\
\solution
%\input{exemplar/12/13/3/78/main.tex}
\item
If 4-digit numbers greater than 5,000 are randomly formed from the digits 0, 1, 3, 5, and 7, what is the probability of forming a number divisible by 5 when:
\begin{enumerate}
    \item The digits are repeated?
    \item The repetition of digits is not allowed?
\end{enumerate}
\solution
%\input{ncert/11/16/4/9/main.tex}
\item Consider the probability space $\brak{\Omega, \mathcal{G}, P}$ where $\Omega = [0,2]$ and $\mathcal{G} = \cbrak{\phi, \Omega, [0,1], (1,2]}$. Let $X$ and $Y$ be two functions on $\Omega$ defined as
\begin{align*}
    X(\omega) = 
    \begin{cases}
        1 & \text{if }\omega \in [0, 1]\\
        2 & \text{if }\omega \in (1, 2]
    \end{cases}
\end{align*}
and
\begin{align*}
    Y(\omega) = 
    \begin{cases}
        2 & \text{if }\omega \in [0, 1.5]\\
        3 & \text{if }\omega \in (1.5, 2].
    \end{cases}
\end{align*}
Then which one of the following statements is true?
\begin{enumerate}
    \item [(A)] $X$ is a random variable with respect to $\mathcal{G}$, but $Y$ is not a random variable with respect to $\mathcal{G}$.
    \item [(B)] $Y$ is a random variable with respect to $\mathcal{G}$, but $X$ is not a random variable with respect to $\mathcal{G}$.
    \item [(C)] Neither $X$ nor $Y$ is a random variable with respect to $\mathcal{G}$.
    \item [(D)] Both $X$ and $Y$ are random variables with respect to $\mathcal{G}$.
\end{enumerate} \hfill (GATE ST 2023)\\
\solution
%\input{gate/ST/2023/14/main.tex}
	\item  A die is loaded in such a way that each odd number is twice as likely to occur as
each even number. Find $P(G)$, where $G$ is the event that a number greater than
3 occurs on a single roll of the die.
\\
\solution
		%\input{exemplar/11/16/3/5/main.tex}
	\item All the jacks, queens and kings are removed from a deck of 52 playing cards. The remaining cards are well shuffled and then one card is drawn at random. Giving ace a value 1 similar value for other cards, find the probability that the card has a value 
		\begin{enumerate}
			\item 7
			\item greater than 7
			\item less than 7
		\end{enumerate}
		%\input{exemplar/10/13/3/30/main.tex}
  \item A Lot consists of 48 mobile phones of which 42 are good, 3 have only minor defects and 3 have major defects.Varnika will buy a phone if it is good but the trader will only buy a mobile if it has no major defects. One phone is selected at random from the lot. What is the probability that it is
\begin{enumerate}
	\item acceptable to Varnika?
            \item acceptable to the trader?
\end{enumerate}
\solution
	%\input{exemplar/10/13/3/40/main.tex}
 \item A student says that if you throw a die, it will show up 1 or not 1. Therefore, the probability of getting 1 and the probability of getting 'not 1' each is equal to $\frac{1}{2}$. Is this correct? Give reasons.\\
 \solution
        %\input{exemplar/10/13/2/9/main.tex}
   \item Four candidates A, B, C, D have ap-
plied for the assignment to coach a school cricket
team. If A is twice as likely to be selected as B, and
B and C are given about the same chance of being
selected, while C is twice as likely to be selected
as D, what are the probabilities that
\begin{enumerate}
\item C will be selected?
\item A will not be selected?
\end{enumerate}
	%\input{exemplar/11/16/3/9/main.tex}
 \item A bag contain 24 balls of which $x$ balls are red, $2x$ are white and $3x$ are blue. A ball is selected at random, What is the probability that it is
\begin{enumerate}[label=\alph*)]
\item not red ?
\item white ?
\end{enumerate}
%\input{exemplar/10/13/3/41/main.tex}
If the letters of the word ASSASSINATION are arranged at random. Find the Probability that
\begin{enumerate}[label=(\alph*)]
\item Four $S's$ come consecutively in the word
\item Two  $I's$ and two $N's$ come together
\item All $A's$ are not coming together
\item No two $A's$ are coming together
\end{enumerate}
%\input{exemplar/11/16/3/14/main.tex}
	\item One urn contains two black balls (labelled B1 and B2) and one white ball. A
	second urn contains one black ball and two white balls (labelled W1 and W2).
	Suppose the following experiment is performed. One of the two urns is chosen
	at random. Next a ball is randomly chosen from the urn. Then a second ball is
	chosen at random from the same urn without replacing the first ball.
	
	\begin{enumerate}
	\item What is the probability that two black balls are chosen?
	
	\item What is the probability that two balls of opposite colour are chosen?
	\end{enumerate}
	\solution
	%\input{exemplar/11/16/3/12/main1.tex}
\end{enumerate}

	\item A bag contains 4 red and 4 black balls, another bag contains 2 red and 6 black balls. One of the two bags is selected at random and a ball is drawn from the bag which is found to be red. Find the probability that the ball is drawn from the first bag.
\\
\solution
		%\begin{table}[H]
	\centering
\begin{tabular}{|c|c|c|}
\hline
Random variable &Value &Definition\\ \hline
\multirow{3}{*}{X} &0 &Slips of Rs 1\\
&1 &Slips of Rs 5\\
&2 &Slips of Rs 13\\ \hline
\multirow{2}{*}{Y} &0 &Box A\\
&1 &Box B\\\hline
\end{tabular}
\caption{}
\label{tab:Distribution}
\end{table}
See \tabref{tab:Distribution}.
\begin{align}
p_{Y}\brak{k}= \begin{cases} 
      \frac{1}{3} & {k=0} \\
      \frac{2}{3 }& {k=1} 
   \end{cases}
   \\
p_{Y|X}\brak{0|0} = \frac{19}{25}\, 
p_{Y|X}\brak{0|1} = \frac{6}{25}\,
p_{Y|X}\brak{1|0} = \frac{45}{50}\,
p_{Y|X}\brak{1|2} = \frac{5}{50}
\end{align}
The desired probability is the probability that a slip drawn at random is marked other than Rs 1,
\begin{align}
&=1-p_X\brak{0}\\
&= p_X(1) + p_X(2)
\end{align}
Using Bayes theorem,
\begin{align}
&= p_Y\brak{0} \times \pr{Y=0 | X=1} + p_Y\brak{1} \times \pr{Y=1|X=2}\\
&=\frac{1}{3} \times \frac{6}{25} + \frac{2}{3} \times \frac{5}{50}\\
&=\frac{11}{75}
\end{align}

\newpage

%\tableofcontents

\bigskip

\renewcommand{\thefigure}{\theenumi}
\renewcommand{\thetable}{\theenumi}
%\renewcommand{\theequation}{\theenumi}

%\begin{abstract}
%%\boldmath
%In this letter, an algorithm for evaluating the exact analytical bit error rate  (BER)  for the piecewise linear (PL) combiner for  multiple relays is presented. Previous results were available only for upto three relays. The algorithm is unique in the sense that  the actual mathematical expressions, that are prohibitively large, need not be explicitly obtained. The diversity gain due to multiple relays is shown through plots of the analytical BER, well supported by simulations. 
%
%\end{abstract}
% IEEEtran.cls defaults to using nonbold math in the Abstract.
% This preserves the distinction between vectors and scalars. However,
% if the journal you are submitting to favors bold math in the abstract,
% then you can use LaTeX's standard command \boldmath at the very start
% of the abstract to achieve this. Many IEEE journals frown on math
% in the abstract anyway.

% Note that keywords are not normally used for peerreview papers.
%\begin{IEEEkeywords}
%Cooperative diversity, decode and forward, piecewise linear
%\end{IEEEkeywords}



% For peer review papers, you can put extra information on the cover
% page as needed:
% \ifCLASSOPTIONpeerreview
% \begin{center} \bfseries EDICS Category: 3-BBND \end{center}
% \fi
%
% For peerreview papers, this IEEEtran command inserts a page break and
% creates the second title. It will be ignored for other modes.
%\IEEEpeerreviewmaketitle




  \item
  Cards with numbers 2 to 101 are placed in a box. A card is selected at random.Find the probability that the card has
\begin{enumerate}[label=(\roman*)]
	\item an even number 
	\item a square number
\end{enumerate}
\solution
%\begin{table}[H]
	\centering
\begin{tabular}{|c|c|c|}
\hline
Random variable &Value &Definition\\ \hline
\multirow{3}{*}{X} &0 &Slips of Rs 1\\
&1 &Slips of Rs 5\\
&2 &Slips of Rs 13\\ \hline
\multirow{2}{*}{Y} &0 &Box A\\
&1 &Box B\\\hline
\end{tabular}
\caption{}
\label{tab:Distribution}
\end{table}
See \tabref{tab:Distribution}.
\begin{align}
p_{Y}\brak{k}= \begin{cases} 
      \frac{1}{3} & {k=0} \\
      \frac{2}{3 }& {k=1} 
   \end{cases}
   \\
p_{Y|X}\brak{0|0} = \frac{19}{25}\, 
p_{Y|X}\brak{0|1} = \frac{6}{25}\,
p_{Y|X}\brak{1|0} = \frac{45}{50}\,
p_{Y|X}\brak{1|2} = \frac{5}{50}
\end{align}
The desired probability is the probability that a slip drawn at random is marked other than Rs 1,
\begin{align}
&=1-p_X\brak{0}\\
&= p_X(1) + p_X(2)
\end{align}
Using Bayes theorem,
\begin{align}
&= p_Y\brak{0} \times \pr{Y=0 | X=1} + p_Y\brak{1} \times \pr{Y=1|X=2}\\
&=\frac{1}{3} \times \frac{6}{25} + \frac{2}{3} \times \frac{5}{50}\\
&=\frac{11}{75}
\end{align}

\newpage

%\tableofcontents

\bigskip

\renewcommand{\thefigure}{\theenumi}
\renewcommand{\thetable}{\theenumi}
%\renewcommand{\theequation}{\theenumi}

%\begin{abstract}
%%\boldmath
%In this letter, an algorithm for evaluating the exact analytical bit error rate  (BER)  for the piecewise linear (PL) combiner for  multiple relays is presented. Previous results were available only for upto three relays. The algorithm is unique in the sense that  the actual mathematical expressions, that are prohibitively large, need not be explicitly obtained. The diversity gain due to multiple relays is shown through plots of the analytical BER, well supported by simulations. 
%
%\end{abstract}
% IEEEtran.cls defaults to using nonbold math in the Abstract.
% This preserves the distinction between vectors and scalars. However,
% if the journal you are submitting to favors bold math in the abstract,
% then you can use LaTeX's standard command \boldmath at the very start
% of the abstract to achieve this. Many IEEE journals frown on math
% in the abstract anyway.

% Note that keywords are not normally used for peerreview papers.
%\begin{IEEEkeywords}
%Cooperative diversity, decode and forward, piecewise linear
%\end{IEEEkeywords}



% For peer review papers, you can put extra information on the cover
% page as needed:
% \ifCLASSOPTIONpeerreview
% \begin{center} \bfseries EDICS Category: 3-BBND \end{center}
% \fi
%
% For peerreview papers, this IEEEtran command inserts a page break and
% creates the second title. It will be ignored for other modes.
%\IEEEpeerreviewmaketitle




\item
The king, queen and jack of clubs are removed from a deck of 52 playing cards and then well shuffled. Now one card is drawn at random from the remaining cards.  Determine the probability that the card is
\begin{enumerate}[label=(\roman*)]
\item a club
\item 10 of hearts
\end{enumerate}
\solution
%\begin{table}[H]
	\centering
\begin{tabular}{|c|c|c|}
\hline
Random variable &Value &Definition\\ \hline
\multirow{3}{*}{X} &0 &Slips of Rs 1\\
&1 &Slips of Rs 5\\
&2 &Slips of Rs 13\\ \hline
\multirow{2}{*}{Y} &0 &Box A\\
&1 &Box B\\\hline
\end{tabular}
\caption{}
\label{tab:Distribution}
\end{table}
See \tabref{tab:Distribution}.
\begin{align}
p_{Y}\brak{k}= \begin{cases} 
      \frac{1}{3} & {k=0} \\
      \frac{2}{3 }& {k=1} 
   \end{cases}
   \\
p_{Y|X}\brak{0|0} = \frac{19}{25}\, 
p_{Y|X}\brak{0|1} = \frac{6}{25}\,
p_{Y|X}\brak{1|0} = \frac{45}{50}\,
p_{Y|X}\brak{1|2} = \frac{5}{50}
\end{align}
The desired probability is the probability that a slip drawn at random is marked other than Rs 1,
\begin{align}
&=1-p_X\brak{0}\\
&= p_X(1) + p_X(2)
\end{align}
Using Bayes theorem,
\begin{align}
&= p_Y\brak{0} \times \pr{Y=0 | X=1} + p_Y\brak{1} \times \pr{Y=1|X=2}\\
&=\frac{1}{3} \times \frac{6}{25} + \frac{2}{3} \times \frac{5}{50}\\
&=\frac{11}{75}
\end{align}

\newpage

%\tableofcontents

\bigskip

\renewcommand{\thefigure}{\theenumi}
\renewcommand{\thetable}{\theenumi}
%\renewcommand{\theequation}{\theenumi}

%\begin{abstract}
%%\boldmath
%In this letter, an algorithm for evaluating the exact analytical bit error rate  (BER)  for the piecewise linear (PL) combiner for  multiple relays is presented. Previous results were available only for upto three relays. The algorithm is unique in the sense that  the actual mathematical expressions, that are prohibitively large, need not be explicitly obtained. The diversity gain due to multiple relays is shown through plots of the analytical BER, well supported by simulations. 
%
%\end{abstract}
% IEEEtran.cls defaults to using nonbold math in the Abstract.
% This preserves the distinction between vectors and scalars. However,
% if the journal you are submitting to favors bold math in the abstract,
% then you can use LaTeX's standard command \boldmath at the very start
% of the abstract to achieve this. Many IEEE journals frown on math
% in the abstract anyway.

% Note that keywords are not normally used for peerreview papers.
%\begin{IEEEkeywords}
%Cooperative diversity, decode and forward, piecewise linear
%\end{IEEEkeywords}



% For peer review papers, you can put extra information on the cover
% page as needed:
% \ifCLASSOPTIONpeerreview
% \begin{center} \bfseries EDICS Category: 3-BBND \end{center}
% \fi
%
% For peerreview papers, this IEEEtran command inserts a page break and
% creates the second title. It will be ignored for other modes.
%\IEEEpeerreviewmaketitle




\item A team of medical students doing their internship have to assist during surgeries
at a city hospital. The probabilities of surgeries rated as very complex, complex,
routine, simple or very simple are respectively, 0.15, 0.20, 0.31, 0.26, .08. Find
the probabilities that a particular surgery will be rated
\begin{enumerate}
	\item complex or very complex;
	\item neither very complex nor very simple;
	\item routine or complex
	\item routine or simple
\end{enumerate}
\solution
%\begin{table}[H]
	\centering
\begin{tabular}{|c|c|c|}
\hline
Random variable &Value &Definition\\ \hline
\multirow{3}{*}{X} &0 &Slips of Rs 1\\
&1 &Slips of Rs 5\\
&2 &Slips of Rs 13\\ \hline
\multirow{2}{*}{Y} &0 &Box A\\
&1 &Box B\\\hline
\end{tabular}
\caption{}
\label{tab:Distribution}
\end{table}
See \tabref{tab:Distribution}.
\begin{align}
p_{Y}\brak{k}= \begin{cases} 
      \frac{1}{3} & {k=0} \\
      \frac{2}{3 }& {k=1} 
   \end{cases}
   \\
p_{Y|X}\brak{0|0} = \frac{19}{25}\, 
p_{Y|X}\brak{0|1} = \frac{6}{25}\,
p_{Y|X}\brak{1|0} = \frac{45}{50}\,
p_{Y|X}\brak{1|2} = \frac{5}{50}
\end{align}
The desired probability is the probability that a slip drawn at random is marked other than Rs 1,
\begin{align}
&=1-p_X\brak{0}\\
&= p_X(1) + p_X(2)
\end{align}
Using Bayes theorem,
\begin{align}
&= p_Y\brak{0} \times \pr{Y=0 | X=1} + p_Y\brak{1} \times \pr{Y=1|X=2}\\
&=\frac{1}{3} \times \frac{6}{25} + \frac{2}{3} \times \frac{5}{50}\\
&=\frac{11}{75}
\end{align}

\newpage

%\tableofcontents

\bigskip

\renewcommand{\thefigure}{\theenumi}
\renewcommand{\thetable}{\theenumi}
%\renewcommand{\theequation}{\theenumi}

%\begin{abstract}
%%\boldmath
%In this letter, an algorithm for evaluating the exact analytical bit error rate  (BER)  for the piecewise linear (PL) combiner for  multiple relays is presented. Previous results were available only for upto three relays. The algorithm is unique in the sense that  the actual mathematical expressions, that are prohibitively large, need not be explicitly obtained. The diversity gain due to multiple relays is shown through plots of the analytical BER, well supported by simulations. 
%
%\end{abstract}
% IEEEtran.cls defaults to using nonbold math in the Abstract.
% This preserves the distinction between vectors and scalars. However,
% if the journal you are submitting to favors bold math in the abstract,
% then you can use LaTeX's standard command \boldmath at the very start
% of the abstract to achieve this. Many IEEE journals frown on math
% in the abstract anyway.

% Note that keywords are not normally used for peerreview papers.
%\begin{IEEEkeywords}
%Cooperative diversity, decode and forward, piecewise linear
%\end{IEEEkeywords}



% For peer review papers, you can put extra information on the cover
% page as needed:
% \ifCLASSOPTIONpeerreview
% \begin{center} \bfseries EDICS Category: 3-BBND \end{center}
% \fi
%
% For peerreview papers, this IEEEtran command inserts a page break and
% creates the second title. It will be ignored for other modes.
%\IEEEpeerreviewmaketitle




\item A card is selected from a pack of 52 cards.
\begin{enumerate}[label=(\alph*)]
    \item How many points are there in the sample space?
    \item Calculate the probability that the card is an ace of spades.
    \item Calculate the probability that the card is (i) an ace and (ii) black card.
\end{enumerate}
\solution
%Let $X$ be an bernoulli rv defined as in \tabref{tab:exemplar/11/16/3/26}.  Then, 
\begin{equation}
    p =
        \frac{4}{11} 
\end{equation}
\begin{table}[H]
	\centering
	\input{exemplar/11/16/3/26/tables/Table2.tex}
	\caption{}
        \label{tab:exemplar/11/16/3/26}
\end{table}

\item The probability that a non leap year selected at random will contain 53 sundays.
\\
\solution
%\begin{table}[H]
	\centering
\begin{tabular}{|c|c|c|}
\hline
Random variable &Value &Definition\\ \hline
\multirow{3}{*}{X} &0 &Slips of Rs 1\\
&1 &Slips of Rs 5\\
&2 &Slips of Rs 13\\ \hline
\multirow{2}{*}{Y} &0 &Box A\\
&1 &Box B\\\hline
\end{tabular}
\caption{}
\label{tab:Distribution}
\end{table}
See \tabref{tab:Distribution}.
\begin{align}
p_{Y}\brak{k}= \begin{cases} 
      \frac{1}{3} & {k=0} \\
      \frac{2}{3 }& {k=1} 
   \end{cases}
   \\
p_{Y|X}\brak{0|0} = \frac{19}{25}\, 
p_{Y|X}\brak{0|1} = \frac{6}{25}\,
p_{Y|X}\brak{1|0} = \frac{45}{50}\,
p_{Y|X}\brak{1|2} = \frac{5}{50}
\end{align}
The desired probability is the probability that a slip drawn at random is marked other than Rs 1,
\begin{align}
&=1-p_X\brak{0}\\
&= p_X(1) + p_X(2)
\end{align}
Using Bayes theorem,
\begin{align}
&= p_Y\brak{0} \times \pr{Y=0 | X=1} + p_Y\brak{1} \times \pr{Y=1|X=2}\\
&=\frac{1}{3} \times \frac{6}{25} + \frac{2}{3} \times \frac{5}{50}\\
&=\frac{11}{75}
\end{align}

\newpage

%\tableofcontents

\bigskip

\renewcommand{\thefigure}{\theenumi}
\renewcommand{\thetable}{\theenumi}
%\renewcommand{\theequation}{\theenumi}

%\begin{abstract}
%%\boldmath
%In this letter, an algorithm for evaluating the exact analytical bit error rate  (BER)  for the piecewise linear (PL) combiner for  multiple relays is presented. Previous results were available only for upto three relays. The algorithm is unique in the sense that  the actual mathematical expressions, that are prohibitively large, need not be explicitly obtained. The diversity gain due to multiple relays is shown through plots of the analytical BER, well supported by simulations. 
%
%\end{abstract}
% IEEEtran.cls defaults to using nonbold math in the Abstract.
% This preserves the distinction between vectors and scalars. However,
% if the journal you are submitting to favors bold math in the abstract,
% then you can use LaTeX's standard command \boldmath at the very start
% of the abstract to achieve this. Many IEEE journals frown on math
% in the abstract anyway.

% Note that keywords are not normally used for peerreview papers.
%\begin{IEEEkeywords}
%Cooperative diversity, decode and forward, piecewise linear
%\end{IEEEkeywords}



% For peer review papers, you can put extra information on the cover
% page as needed:
% \ifCLASSOPTIONpeerreview
% \begin{center} \bfseries EDICS Category: 3-BBND \end{center}
% \fi
%
% For peerreview papers, this IEEEtran command inserts a page break and
% creates the second title. It will be ignored for other modes.
%\IEEEpeerreviewmaketitle




\item One of the four persons John, Rita, Aslam or Gurpreet will be promoted next
month. Consequently the sample space consists of four elementary outcomes
S = {John promoted, Rita promoted, Aslam promoted, Gurpreet promoted}
You are told that the chances of John’s promotion is same as that of Gurpreet,
Rita’s chances of promotion are twice as likely as Johns. Aslam’s chances are
four times that of John.
\begin{enumerate}
	\item Determine
	\begin{enumerate}
		\item P (John promoted)
		\item P (Rita promoted)
		\item P (Aslam promoted)
		\item P (Gurpreet promoted)
	\end{enumerate}
	\item If A = {John promoted or Gurpreet promoted}, find P (A).
\end{enumerate}
\solution
%\begin{table}[H]
	\centering
\begin{tabular}{|c|c|c|}
\hline
Random variable &Value &Definition\\ \hline
\multirow{3}{*}{X} &0 &Slips of Rs 1\\
&1 &Slips of Rs 5\\
&2 &Slips of Rs 13\\ \hline
\multirow{2}{*}{Y} &0 &Box A\\
&1 &Box B\\\hline
\end{tabular}
\caption{}
\label{tab:Distribution}
\end{table}
See \tabref{tab:Distribution}.
\begin{align}
p_{Y}\brak{k}= \begin{cases} 
      \frac{1}{3} & {k=0} \\
      \frac{2}{3 }& {k=1} 
   \end{cases}
   \\
p_{Y|X}\brak{0|0} = \frac{19}{25}\, 
p_{Y|X}\brak{0|1} = \frac{6}{25}\,
p_{Y|X}\brak{1|0} = \frac{45}{50}\,
p_{Y|X}\brak{1|2} = \frac{5}{50}
\end{align}
The desired probability is the probability that a slip drawn at random is marked other than Rs 1,
\begin{align}
&=1-p_X\brak{0}\\
&= p_X(1) + p_X(2)
\end{align}
Using Bayes theorem,
\begin{align}
&= p_Y\brak{0} \times \pr{Y=0 | X=1} + p_Y\brak{1} \times \pr{Y=1|X=2}\\
&=\frac{1}{3} \times \frac{6}{25} + \frac{2}{3} \times \frac{5}{50}\\
&=\frac{11}{75}
\end{align}

\newpage

%\tableofcontents

\bigskip

\renewcommand{\thefigure}{\theenumi}
\renewcommand{\thetable}{\theenumi}
%\renewcommand{\theequation}{\theenumi}

%\begin{abstract}
%%\boldmath
%In this letter, an algorithm for evaluating the exact analytical bit error rate  (BER)  for the piecewise linear (PL) combiner for  multiple relays is presented. Previous results were available only for upto three relays. The algorithm is unique in the sense that  the actual mathematical expressions, that are prohibitively large, need not be explicitly obtained. The diversity gain due to multiple relays is shown through plots of the analytical BER, well supported by simulations. 
%
%\end{abstract}
% IEEEtran.cls defaults to using nonbold math in the Abstract.
% This preserves the distinction between vectors and scalars. However,
% if the journal you are submitting to favors bold math in the abstract,
% then you can use LaTeX's standard command \boldmath at the very start
% of the abstract to achieve this. Many IEEE journals frown on math
% in the abstract anyway.

% Note that keywords are not normally used for peerreview papers.
%\begin{IEEEkeywords}
%Cooperative diversity, decode and forward, piecewise linear
%\end{IEEEkeywords}



% For peer review papers, you can put extra information on the cover
% page as needed:
% \ifCLASSOPTIONpeerreview
% \begin{center} \bfseries EDICS Category: 3-BBND \end{center}
% \fi
%
% For peerreview papers, this IEEEtran command inserts a page break and
% creates the second title. It will be ignored for other modes.
%\IEEEpeerreviewmaketitle




\item A card is drawn from a deck of 52 cards. Find the probability of getting a king or a heart or a red card.\\
\solution
%\begin{table}[H]
	\centering
\begin{tabular}{|c|c|c|}
\hline
Random variable &Value &Definition\\ \hline
\multirow{3}{*}{X} &0 &Slips of Rs 1\\
&1 &Slips of Rs 5\\
&2 &Slips of Rs 13\\ \hline
\multirow{2}{*}{Y} &0 &Box A\\
&1 &Box B\\\hline
\end{tabular}
\caption{}
\label{tab:Distribution}
\end{table}
See \tabref{tab:Distribution}.
\begin{align}
p_{Y}\brak{k}= \begin{cases} 
      \frac{1}{3} & {k=0} \\
      \frac{2}{3 }& {k=1} 
   \end{cases}
   \\
p_{Y|X}\brak{0|0} = \frac{19}{25}\, 
p_{Y|X}\brak{0|1} = \frac{6}{25}\,
p_{Y|X}\brak{1|0} = \frac{45}{50}\,
p_{Y|X}\brak{1|2} = \frac{5}{50}
\end{align}
The desired probability is the probability that a slip drawn at random is marked other than Rs 1,
\begin{align}
&=1-p_X\brak{0}\\
&= p_X(1) + p_X(2)
\end{align}
Using Bayes theorem,
\begin{align}
&= p_Y\brak{0} \times \pr{Y=0 | X=1} + p_Y\brak{1} \times \pr{Y=1|X=2}\\
&=\frac{1}{3} \times \frac{6}{25} + \frac{2}{3} \times \frac{5}{50}\\
&=\frac{11}{75}
\end{align}

\newpage

%\tableofcontents

\bigskip

\renewcommand{\thefigure}{\theenumi}
\renewcommand{\thetable}{\theenumi}
%\renewcommand{\theequation}{\theenumi}

%\begin{abstract}
%%\boldmath
%In this letter, an algorithm for evaluating the exact analytical bit error rate  (BER)  for the piecewise linear (PL) combiner for  multiple relays is presented. Previous results were available only for upto three relays. The algorithm is unique in the sense that  the actual mathematical expressions, that are prohibitively large, need not be explicitly obtained. The diversity gain due to multiple relays is shown through plots of the analytical BER, well supported by simulations. 
%
%\end{abstract}
% IEEEtran.cls defaults to using nonbold math in the Abstract.
% This preserves the distinction between vectors and scalars. However,
% if the journal you are submitting to favors bold math in the abstract,
% then you can use LaTeX's standard command \boldmath at the very start
% of the abstract to achieve this. Many IEEE journals frown on math
% in the abstract anyway.

% Note that keywords are not normally used for peerreview papers.
%\begin{IEEEkeywords}
%Cooperative diversity, decode and forward, piecewise linear
%\end{IEEEkeywords}



% For peer review papers, you can put extra information on the cover
% page as needed:
% \ifCLASSOPTIONpeerreview
% \begin{center} \bfseries EDICS Category: 3-BBND \end{center}
% \fi
%
% For peerreview papers, this IEEEtran command inserts a page break and
% creates the second title. It will be ignored for other modes.
%\IEEEpeerreviewmaketitle




\item The probability that a student will pass his examination is 0.73, the probability of
the student getting a compartment is 0.13, and the probability that the student will
either pass or get compartment is 0.96. State True or False.\\
\solution
%\begin{table}[H]
	\centering
\begin{tabular}{|c|c|c|}
\hline
Random variable &Value &Definition\\ \hline
\multirow{3}{*}{X} &0 &Slips of Rs 1\\
&1 &Slips of Rs 5\\
&2 &Slips of Rs 13\\ \hline
\multirow{2}{*}{Y} &0 &Box A\\
&1 &Box B\\\hline
\end{tabular}
\caption{}
\label{tab:Distribution}
\end{table}
See \tabref{tab:Distribution}.
\begin{align}
p_{Y}\brak{k}= \begin{cases} 
      \frac{1}{3} & {k=0} \\
      \frac{2}{3 }& {k=1} 
   \end{cases}
   \\
p_{Y|X}\brak{0|0} = \frac{19}{25}\, 
p_{Y|X}\brak{0|1} = \frac{6}{25}\,
p_{Y|X}\brak{1|0} = \frac{45}{50}\,
p_{Y|X}\brak{1|2} = \frac{5}{50}
\end{align}
The desired probability is the probability that a slip drawn at random is marked other than Rs 1,
\begin{align}
&=1-p_X\brak{0}\\
&= p_X(1) + p_X(2)
\end{align}
Using Bayes theorem,
\begin{align}
&= p_Y\brak{0} \times \pr{Y=0 | X=1} + p_Y\brak{1} \times \pr{Y=1|X=2}\\
&=\frac{1}{3} \times \frac{6}{25} + \frac{2}{3} \times \frac{5}{50}\\
&=\frac{11}{75}
\end{align}

\newpage

%\tableofcontents

\bigskip

\renewcommand{\thefigure}{\theenumi}
\renewcommand{\thetable}{\theenumi}
%\renewcommand{\theequation}{\theenumi}

%\begin{abstract}
%%\boldmath
%In this letter, an algorithm for evaluating the exact analytical bit error rate  (BER)  for the piecewise linear (PL) combiner for  multiple relays is presented. Previous results were available only for upto three relays. The algorithm is unique in the sense that  the actual mathematical expressions, that are prohibitively large, need not be explicitly obtained. The diversity gain due to multiple relays is shown through plots of the analytical BER, well supported by simulations. 
%
%\end{abstract}
% IEEEtran.cls defaults to using nonbold math in the Abstract.
% This preserves the distinction between vectors and scalars. However,
% if the journal you are submitting to favors bold math in the abstract,
% then you can use LaTeX's standard command \boldmath at the very start
% of the abstract to achieve this. Many IEEE journals frown on math
% in the abstract anyway.

% Note that keywords are not normally used for peerreview papers.
%\begin{IEEEkeywords}
%Cooperative diversity, decode and forward, piecewise linear
%\end{IEEEkeywords}



% For peer review papers, you can put extra information on the cover
% page as needed:
% \ifCLASSOPTIONpeerreview
% \begin{center} \bfseries EDICS Category: 3-BBND \end{center}
% \fi
%
% For peerreview papers, this IEEEtran command inserts a page break and
% creates the second title. It will be ignored for other modes.
%\IEEEpeerreviewmaketitle




\item A card is selected from a pack of 52 cards\\
\begin{enumerate}[label=(\alph*)]
\item How many points are there in the sample space?
\item Calculate the probability that the cards is an ace of spades.
\item Calculate the probability that the card is (i) an ace (ii)black card.\\
\end{enumerate}
%\input{ncert/11/16/3/4_1/Prob_4.tex}
\item In a non-leap year, the probability of having 53 tuesdays or 53 wednesdays is\\
\solution
%A non-leap year has a total of 365 days, and a week has 7 days.\\
So it can be expressed as 
\begin{align}
365\text{days} &=52\times 7+1 \text{day}
\end{align}
$\implies$ 52 tuesdays or wednesdays\\
Random variable X denotes the days of a week
\begin{align}
p_X\brak{k}&=\frac{1}{7}; \quad \brak{1<k<7}
\end{align}
So the probability of extra day being tuesday or wednesday is
\begin{align}
p_X\brak{3}+p_X\brak{4}&=\frac{1}{7}+\frac{1}{7}=\frac{2}{7}
\end{align}



\item There are 1000 sealed envelopes in a box, 10 of them contain a cash prize of
Rs 100 each, 100 of them contain a cash prize of Rs 50 each and 200 of them
contain a cash prize of Rs 10 each and rest do not contain any cash prize. If they
are well shuffled and an envelope is picked up out, what is the probability that it
contains no cash prize?\\
\solution
%\begin{table}[H]
	\centering
\begin{tabular}{|c|c|c|}
\hline
Random variable &Value &Definition\\ \hline
\multirow{3}{*}{X} &0 &Slips of Rs 1\\
&1 &Slips of Rs 5\\
&2 &Slips of Rs 13\\ \hline
\multirow{2}{*}{Y} &0 &Box A\\
&1 &Box B\\\hline
\end{tabular}
\caption{}
\label{tab:Distribution}
\end{table}
See \tabref{tab:Distribution}.
\begin{align}
p_{Y}\brak{k}= \begin{cases} 
      \frac{1}{3} & {k=0} \\
      \frac{2}{3 }& {k=1} 
   \end{cases}
   \\
p_{Y|X}\brak{0|0} = \frac{19}{25}\, 
p_{Y|X}\brak{0|1} = \frac{6}{25}\,
p_{Y|X}\brak{1|0} = \frac{45}{50}\,
p_{Y|X}\brak{1|2} = \frac{5}{50}
\end{align}
The desired probability is the probability that a slip drawn at random is marked other than Rs 1,
\begin{align}
&=1-p_X\brak{0}\\
&= p_X(1) + p_X(2)
\end{align}
Using Bayes theorem,
\begin{align}
&= p_Y\brak{0} \times \pr{Y=0 | X=1} + p_Y\brak{1} \times \pr{Y=1|X=2}\\
&=\frac{1}{3} \times \frac{6}{25} + \frac{2}{3} \times \frac{5}{50}\\
&=\frac{11}{75}
\end{align}

\newpage

%\tableofcontents

\bigskip

\renewcommand{\thefigure}{\theenumi}
\renewcommand{\thetable}{\theenumi}
%\renewcommand{\theequation}{\theenumi}

%\begin{abstract}
%%\boldmath
%In this letter, an algorithm for evaluating the exact analytical bit error rate  (BER)  for the piecewise linear (PL) combiner for  multiple relays is presented. Previous results were available only for upto three relays. The algorithm is unique in the sense that  the actual mathematical expressions, that are prohibitively large, need not be explicitly obtained. The diversity gain due to multiple relays is shown through plots of the analytical BER, well supported by simulations. 
%
%\end{abstract}
% IEEEtran.cls defaults to using nonbold math in the Abstract.
% This preserves the distinction between vectors and scalars. However,
% if the journal you are submitting to favors bold math in the abstract,
% then you can use LaTeX's standard command \boldmath at the very start
% of the abstract to achieve this. Many IEEE journals frown on math
% in the abstract anyway.

% Note that keywords are not normally used for peerreview papers.
%\begin{IEEEkeywords}
%Cooperative diversity, decode and forward, piecewise linear
%\end{IEEEkeywords}



% For peer review papers, you can put extra information on the cover
% page as needed:
% \ifCLASSOPTIONpeerreview
% \begin{center} \bfseries EDICS Category: 3-BBND \end{center}
% \fi
%
% For peerreview papers, this IEEEtran command inserts a page break and
% creates the second title. It will be ignored for other modes.
%\IEEEpeerreviewmaketitle




\item 
A die is thrown and a card is selected at random from a deck of 52 playing cards. The probability of getting an even number on the die and a spade card.\\
\solution
%\begin{table}[H]
	\centering
\begin{tabular}{|c|c|c|}
\hline
Random variable &Value &Definition\\ \hline
\multirow{3}{*}{X} &0 &Slips of Rs 1\\
&1 &Slips of Rs 5\\
&2 &Slips of Rs 13\\ \hline
\multirow{2}{*}{Y} &0 &Box A\\
&1 &Box B\\\hline
\end{tabular}
\caption{}
\label{tab:Distribution}
\end{table}
See \tabref{tab:Distribution}.
\begin{align}
p_{Y}\brak{k}= \begin{cases} 
      \frac{1}{3} & {k=0} \\
      \frac{2}{3 }& {k=1} 
   \end{cases}
   \\
p_{Y|X}\brak{0|0} = \frac{19}{25}\, 
p_{Y|X}\brak{0|1} = \frac{6}{25}\,
p_{Y|X}\brak{1|0} = \frac{45}{50}\,
p_{Y|X}\brak{1|2} = \frac{5}{50}
\end{align}
The desired probability is the probability that a slip drawn at random is marked other than Rs 1,
\begin{align}
&=1-p_X\brak{0}\\
&= p_X(1) + p_X(2)
\end{align}
Using Bayes theorem,
\begin{align}
&= p_Y\brak{0} \times \pr{Y=0 | X=1} + p_Y\brak{1} \times \pr{Y=1|X=2}\\
&=\frac{1}{3} \times \frac{6}{25} + \frac{2}{3} \times \frac{5}{50}\\
&=\frac{11}{75}
\end{align}

\newpage

%\tableofcontents

\bigskip

\renewcommand{\thefigure}{\theenumi}
\renewcommand{\thetable}{\theenumi}
%\renewcommand{\theequation}{\theenumi}

%\begin{abstract}
%%\boldmath
%In this letter, an algorithm for evaluating the exact analytical bit error rate  (BER)  for the piecewise linear (PL) combiner for  multiple relays is presented. Previous results were available only for upto three relays. The algorithm is unique in the sense that  the actual mathematical expressions, that are prohibitively large, need not be explicitly obtained. The diversity gain due to multiple relays is shown through plots of the analytical BER, well supported by simulations. 
%
%\end{abstract}
% IEEEtran.cls defaults to using nonbold math in the Abstract.
% This preserves the distinction between vectors and scalars. However,
% if the journal you are submitting to favors bold math in the abstract,
% then you can use LaTeX's standard command \boldmath at the very start
% of the abstract to achieve this. Many IEEE journals frown on math
% in the abstract anyway.

% Note that keywords are not normally used for peerreview papers.
%\begin{IEEEkeywords}
%Cooperative diversity, decode and forward, piecewise linear
%\end{IEEEkeywords}



% For peer review papers, you can put extra information on the cover
% page as needed:
% \ifCLASSOPTIONpeerreview
% \begin{center} \bfseries EDICS Category: 3-BBND \end{center}
% \fi
%
% For peerreview papers, this IEEEtran command inserts a page break and
% creates the second title. It will be ignored for other modes.
%\IEEEpeerreviewmaketitle




\item
If 4-digit numbers greater than 5,000 are randomly formed from the digits 0, 1, 3, 5, and 7, what is the probability of forming a number divisible by 5 when:
\begin{enumerate}
    \item The digits are repeated?
    \item The repetition of digits is not allowed?
\end{enumerate}
\solution
%\begin{table}[H]
	\centering
\begin{tabular}{|c|c|c|}
\hline
Random variable &Value &Definition\\ \hline
\multirow{3}{*}{X} &0 &Slips of Rs 1\\
&1 &Slips of Rs 5\\
&2 &Slips of Rs 13\\ \hline
\multirow{2}{*}{Y} &0 &Box A\\
&1 &Box B\\\hline
\end{tabular}
\caption{}
\label{tab:Distribution}
\end{table}
See \tabref{tab:Distribution}.
\begin{align}
p_{Y}\brak{k}= \begin{cases} 
      \frac{1}{3} & {k=0} \\
      \frac{2}{3 }& {k=1} 
   \end{cases}
   \\
p_{Y|X}\brak{0|0} = \frac{19}{25}\, 
p_{Y|X}\brak{0|1} = \frac{6}{25}\,
p_{Y|X}\brak{1|0} = \frac{45}{50}\,
p_{Y|X}\brak{1|2} = \frac{5}{50}
\end{align}
The desired probability is the probability that a slip drawn at random is marked other than Rs 1,
\begin{align}
&=1-p_X\brak{0}\\
&= p_X(1) + p_X(2)
\end{align}
Using Bayes theorem,
\begin{align}
&= p_Y\brak{0} \times \pr{Y=0 | X=1} + p_Y\brak{1} \times \pr{Y=1|X=2}\\
&=\frac{1}{3} \times \frac{6}{25} + \frac{2}{3} \times \frac{5}{50}\\
&=\frac{11}{75}
\end{align}

\newpage

%\tableofcontents

\bigskip

\renewcommand{\thefigure}{\theenumi}
\renewcommand{\thetable}{\theenumi}
%\renewcommand{\theequation}{\theenumi}

%\begin{abstract}
%%\boldmath
%In this letter, an algorithm for evaluating the exact analytical bit error rate  (BER)  for the piecewise linear (PL) combiner for  multiple relays is presented. Previous results were available only for upto three relays. The algorithm is unique in the sense that  the actual mathematical expressions, that are prohibitively large, need not be explicitly obtained. The diversity gain due to multiple relays is shown through plots of the analytical BER, well supported by simulations. 
%
%\end{abstract}
% IEEEtran.cls defaults to using nonbold math in the Abstract.
% This preserves the distinction between vectors and scalars. However,
% if the journal you are submitting to favors bold math in the abstract,
% then you can use LaTeX's standard command \boldmath at the very start
% of the abstract to achieve this. Many IEEE journals frown on math
% in the abstract anyway.

% Note that keywords are not normally used for peerreview papers.
%\begin{IEEEkeywords}
%Cooperative diversity, decode and forward, piecewise linear
%\end{IEEEkeywords}



% For peer review papers, you can put extra information on the cover
% page as needed:
% \ifCLASSOPTIONpeerreview
% \begin{center} \bfseries EDICS Category: 3-BBND \end{center}
% \fi
%
% For peerreview papers, this IEEEtran command inserts a page break and
% creates the second title. It will be ignored for other modes.
%\IEEEpeerreviewmaketitle




\item Consider the probability space $\brak{\Omega, \mathcal{G}, P}$ where $\Omega = [0,2]$ and $\mathcal{G} = \cbrak{\phi, \Omega, [0,1], (1,2]}$. Let $X$ and $Y$ be two functions on $\Omega$ defined as
\begin{align*}
    X(\omega) = 
    \begin{cases}
        1 & \text{if }\omega \in [0, 1]\\
        2 & \text{if }\omega \in (1, 2]
    \end{cases}
\end{align*}
and
\begin{align*}
    Y(\omega) = 
    \begin{cases}
        2 & \text{if }\omega \in [0, 1.5]\\
        3 & \text{if }\omega \in (1.5, 2].
    \end{cases}
\end{align*}
Then which one of the following statements is true?
\begin{enumerate}
    \item [(A)] $X$ is a random variable with respect to $\mathcal{G}$, but $Y$ is not a random variable with respect to $\mathcal{G}$.
    \item [(B)] $Y$ is a random variable with respect to $\mathcal{G}$, but $X$ is not a random variable with respect to $\mathcal{G}$.
    \item [(C)] Neither $X$ nor $Y$ is a random variable with respect to $\mathcal{G}$.
    \item [(D)] Both $X$ and $Y$ are random variables with respect to $\mathcal{G}$.
\end{enumerate} \hfill (GATE ST 2023)\\
\solution
%\begin{table}[H]
	\centering
\begin{tabular}{|c|c|c|}
\hline
Random variable &Value &Definition\\ \hline
\multirow{3}{*}{X} &0 &Slips of Rs 1\\
&1 &Slips of Rs 5\\
&2 &Slips of Rs 13\\ \hline
\multirow{2}{*}{Y} &0 &Box A\\
&1 &Box B\\\hline
\end{tabular}
\caption{}
\label{tab:Distribution}
\end{table}
See \tabref{tab:Distribution}.
\begin{align}
p_{Y}\brak{k}= \begin{cases} 
      \frac{1}{3} & {k=0} \\
      \frac{2}{3 }& {k=1} 
   \end{cases}
   \\
p_{Y|X}\brak{0|0} = \frac{19}{25}\, 
p_{Y|X}\brak{0|1} = \frac{6}{25}\,
p_{Y|X}\brak{1|0} = \frac{45}{50}\,
p_{Y|X}\brak{1|2} = \frac{5}{50}
\end{align}
The desired probability is the probability that a slip drawn at random is marked other than Rs 1,
\begin{align}
&=1-p_X\brak{0}\\
&= p_X(1) + p_X(2)
\end{align}
Using Bayes theorem,
\begin{align}
&= p_Y\brak{0} \times \pr{Y=0 | X=1} + p_Y\brak{1} \times \pr{Y=1|X=2}\\
&=\frac{1}{3} \times \frac{6}{25} + \frac{2}{3} \times \frac{5}{50}\\
&=\frac{11}{75}
\end{align}

\newpage

%\tableofcontents

\bigskip

\renewcommand{\thefigure}{\theenumi}
\renewcommand{\thetable}{\theenumi}
%\renewcommand{\theequation}{\theenumi}

%\begin{abstract}
%%\boldmath
%In this letter, an algorithm for evaluating the exact analytical bit error rate  (BER)  for the piecewise linear (PL) combiner for  multiple relays is presented. Previous results were available only for upto three relays. The algorithm is unique in the sense that  the actual mathematical expressions, that are prohibitively large, need not be explicitly obtained. The diversity gain due to multiple relays is shown through plots of the analytical BER, well supported by simulations. 
%
%\end{abstract}
% IEEEtran.cls defaults to using nonbold math in the Abstract.
% This preserves the distinction between vectors and scalars. However,
% if the journal you are submitting to favors bold math in the abstract,
% then you can use LaTeX's standard command \boldmath at the very start
% of the abstract to achieve this. Many IEEE journals frown on math
% in the abstract anyway.

% Note that keywords are not normally used for peerreview papers.
%\begin{IEEEkeywords}
%Cooperative diversity, decode and forward, piecewise linear
%\end{IEEEkeywords}



% For peer review papers, you can put extra information on the cover
% page as needed:
% \ifCLASSOPTIONpeerreview
% \begin{center} \bfseries EDICS Category: 3-BBND \end{center}
% \fi
%
% For peerreview papers, this IEEEtran command inserts a page break and
% creates the second title. It will be ignored for other modes.
%\IEEEpeerreviewmaketitle




	\item  A die is loaded in such a way that each odd number is twice as likely to occur as
each even number. Find $P(G)$, where $G$ is the event that a number greater than
3 occurs on a single roll of the die.
\\
\solution
		%\begin{table}[H]
	\centering
\begin{tabular}{|c|c|c|}
\hline
Random variable &Value &Definition\\ \hline
\multirow{3}{*}{X} &0 &Slips of Rs 1\\
&1 &Slips of Rs 5\\
&2 &Slips of Rs 13\\ \hline
\multirow{2}{*}{Y} &0 &Box A\\
&1 &Box B\\\hline
\end{tabular}
\caption{}
\label{tab:Distribution}
\end{table}
See \tabref{tab:Distribution}.
\begin{align}
p_{Y}\brak{k}= \begin{cases} 
      \frac{1}{3} & {k=0} \\
      \frac{2}{3 }& {k=1} 
   \end{cases}
   \\
p_{Y|X}\brak{0|0} = \frac{19}{25}\, 
p_{Y|X}\brak{0|1} = \frac{6}{25}\,
p_{Y|X}\brak{1|0} = \frac{45}{50}\,
p_{Y|X}\brak{1|2} = \frac{5}{50}
\end{align}
The desired probability is the probability that a slip drawn at random is marked other than Rs 1,
\begin{align}
&=1-p_X\brak{0}\\
&= p_X(1) + p_X(2)
\end{align}
Using Bayes theorem,
\begin{align}
&= p_Y\brak{0} \times \pr{Y=0 | X=1} + p_Y\brak{1} \times \pr{Y=1|X=2}\\
&=\frac{1}{3} \times \frac{6}{25} + \frac{2}{3} \times \frac{5}{50}\\
&=\frac{11}{75}
\end{align}

\newpage

%\tableofcontents

\bigskip

\renewcommand{\thefigure}{\theenumi}
\renewcommand{\thetable}{\theenumi}
%\renewcommand{\theequation}{\theenumi}

%\begin{abstract}
%%\boldmath
%In this letter, an algorithm for evaluating the exact analytical bit error rate  (BER)  for the piecewise linear (PL) combiner for  multiple relays is presented. Previous results were available only for upto three relays. The algorithm is unique in the sense that  the actual mathematical expressions, that are prohibitively large, need not be explicitly obtained. The diversity gain due to multiple relays is shown through plots of the analytical BER, well supported by simulations. 
%
%\end{abstract}
% IEEEtran.cls defaults to using nonbold math in the Abstract.
% This preserves the distinction between vectors and scalars. However,
% if the journal you are submitting to favors bold math in the abstract,
% then you can use LaTeX's standard command \boldmath at the very start
% of the abstract to achieve this. Many IEEE journals frown on math
% in the abstract anyway.

% Note that keywords are not normally used for peerreview papers.
%\begin{IEEEkeywords}
%Cooperative diversity, decode and forward, piecewise linear
%\end{IEEEkeywords}



% For peer review papers, you can put extra information on the cover
% page as needed:
% \ifCLASSOPTIONpeerreview
% \begin{center} \bfseries EDICS Category: 3-BBND \end{center}
% \fi
%
% For peerreview papers, this IEEEtran command inserts a page break and
% creates the second title. It will be ignored for other modes.
%\IEEEpeerreviewmaketitle




	\item All the jacks, queens and kings are removed from a deck of 52 playing cards. The remaining cards are well shuffled and then one card is drawn at random. Giving ace a value 1 similar value for other cards, find the probability that the card has a value 
		\begin{enumerate}
			\item 7
			\item greater than 7
			\item less than 7
		\end{enumerate}
		%Number of cards left after removing all jacks, queens and kings 
\begin{align}
N	= 52 - 4\times 3
	= 40
\end{align}
%\begin{table}[H]
%\def\arraystretch{1.2}
%\begin{tabular}{|c|c|c|}
%\hline
%	\textbf{Parameter} &\textbf{Value} &\textbf{Description}\\ \hline
%	$X$ &1-10 &Represents the value of the card picked \\ \hline
%\end{tabular}
%\end{table}
Let $1 \le X \le 10$ be the value of the card picked.  Then,
\begin{align}
	p_X(k) &= \Pr(X=k)\ \forall\ 1 \leq k \leq 10\\
	&= \frac{4\times 1}{40}\\
	&= \frac{1}{10}\\
	\therefore p_X(k) &= 
	\begin{cases}
		\frac{1}{10} & 1 \leq k \leq 10\\
		0 & \text{otherwise}
	\end{cases}
\end{align}
and
\begin{align}
	F_{X}(k) &= \sum_{m=0}^{k}p_{X}(m) \quad 1 \leq k \leq 10\\
	&= \frac{k}{10}\\
	\therefore F_{X}(k) &= 
	\begin{cases}
		0 & k \leq 0\\
		\frac{k}{10} & 1\leq k \leq 10\\
		1 & k > 10 
	\end{cases}
\end{align}
\begin{enumerate}
	\item Probability that card has value equal to 7 is
		\begin{align}
			 p_{X}(7)
			= \frac{1}{10}
		\end{align}
	\item Probability that card has value greater than 7 is
		\begin{align}
			1 - F_X(7)
			&= 1 - \frac{7}{10}
			\\
			&= \frac{3}{10}
		\end{align}
	\item Probability that card has value less than 7 is
		\begin{align}
			 F_{X}(6)
			=\frac{6}{10}
		\end{align}
\end{enumerate}

  \item A Lot consists of 48 mobile phones of which 42 are good, 3 have only minor defects and 3 have major defects.Varnika will buy a phone if it is good but the trader will only buy a mobile if it has no major defects. One phone is selected at random from the lot. What is the probability that it is
\begin{enumerate}
	\item acceptable to Varnika?
            \item acceptable to the trader?
\end{enumerate}
\solution
	%\begin{table}[H]
	\centering
\begin{tabular}{|c|c|c|}
\hline
Random variable &Value &Definition\\ \hline
\multirow{3}{*}{X} &0 &Slips of Rs 1\\
&1 &Slips of Rs 5\\
&2 &Slips of Rs 13\\ \hline
\multirow{2}{*}{Y} &0 &Box A\\
&1 &Box B\\\hline
\end{tabular}
\caption{}
\label{tab:Distribution}
\end{table}
See \tabref{tab:Distribution}.
\begin{align}
p_{Y}\brak{k}= \begin{cases} 
      \frac{1}{3} & {k=0} \\
      \frac{2}{3 }& {k=1} 
   \end{cases}
   \\
p_{Y|X}\brak{0|0} = \frac{19}{25}\, 
p_{Y|X}\brak{0|1} = \frac{6}{25}\,
p_{Y|X}\brak{1|0} = \frac{45}{50}\,
p_{Y|X}\brak{1|2} = \frac{5}{50}
\end{align}
The desired probability is the probability that a slip drawn at random is marked other than Rs 1,
\begin{align}
&=1-p_X\brak{0}\\
&= p_X(1) + p_X(2)
\end{align}
Using Bayes theorem,
\begin{align}
&= p_Y\brak{0} \times \pr{Y=0 | X=1} + p_Y\brak{1} \times \pr{Y=1|X=2}\\
&=\frac{1}{3} \times \frac{6}{25} + \frac{2}{3} \times \frac{5}{50}\\
&=\frac{11}{75}
\end{align}

\newpage

%\tableofcontents

\bigskip

\renewcommand{\thefigure}{\theenumi}
\renewcommand{\thetable}{\theenumi}
%\renewcommand{\theequation}{\theenumi}

%\begin{abstract}
%%\boldmath
%In this letter, an algorithm for evaluating the exact analytical bit error rate  (BER)  for the piecewise linear (PL) combiner for  multiple relays is presented. Previous results were available only for upto three relays. The algorithm is unique in the sense that  the actual mathematical expressions, that are prohibitively large, need not be explicitly obtained. The diversity gain due to multiple relays is shown through plots of the analytical BER, well supported by simulations. 
%
%\end{abstract}
% IEEEtran.cls defaults to using nonbold math in the Abstract.
% This preserves the distinction between vectors and scalars. However,
% if the journal you are submitting to favors bold math in the abstract,
% then you can use LaTeX's standard command \boldmath at the very start
% of the abstract to achieve this. Many IEEE journals frown on math
% in the abstract anyway.

% Note that keywords are not normally used for peerreview papers.
%\begin{IEEEkeywords}
%Cooperative diversity, decode and forward, piecewise linear
%\end{IEEEkeywords}



% For peer review papers, you can put extra information on the cover
% page as needed:
% \ifCLASSOPTIONpeerreview
% \begin{center} \bfseries EDICS Category: 3-BBND \end{center}
% \fi
%
% For peerreview papers, this IEEEtran command inserts a page break and
% creates the second title. It will be ignored for other modes.
%\IEEEpeerreviewmaketitle




 \item A student says that if you throw a die, it will show up 1 or not 1. Therefore, the probability of getting 1 and the probability of getting 'not 1' each is equal to $\frac{1}{2}$. Is this correct? Give reasons.\\
 \solution
        %\begin{table}[H]
	\centering
\begin{tabular}{|c|c|c|}
\hline
Random variable &Value &Definition\\ \hline
\multirow{3}{*}{X} &0 &Slips of Rs 1\\
&1 &Slips of Rs 5\\
&2 &Slips of Rs 13\\ \hline
\multirow{2}{*}{Y} &0 &Box A\\
&1 &Box B\\\hline
\end{tabular}
\caption{}
\label{tab:Distribution}
\end{table}
See \tabref{tab:Distribution}.
\begin{align}
p_{Y}\brak{k}= \begin{cases} 
      \frac{1}{3} & {k=0} \\
      \frac{2}{3 }& {k=1} 
   \end{cases}
   \\
p_{Y|X}\brak{0|0} = \frac{19}{25}\, 
p_{Y|X}\brak{0|1} = \frac{6}{25}\,
p_{Y|X}\brak{1|0} = \frac{45}{50}\,
p_{Y|X}\brak{1|2} = \frac{5}{50}
\end{align}
The desired probability is the probability that a slip drawn at random is marked other than Rs 1,
\begin{align}
&=1-p_X\brak{0}\\
&= p_X(1) + p_X(2)
\end{align}
Using Bayes theorem,
\begin{align}
&= p_Y\brak{0} \times \pr{Y=0 | X=1} + p_Y\brak{1} \times \pr{Y=1|X=2}\\
&=\frac{1}{3} \times \frac{6}{25} + \frac{2}{3} \times \frac{5}{50}\\
&=\frac{11}{75}
\end{align}

\newpage

%\tableofcontents

\bigskip

\renewcommand{\thefigure}{\theenumi}
\renewcommand{\thetable}{\theenumi}
%\renewcommand{\theequation}{\theenumi}

%\begin{abstract}
%%\boldmath
%In this letter, an algorithm for evaluating the exact analytical bit error rate  (BER)  for the piecewise linear (PL) combiner for  multiple relays is presented. Previous results were available only for upto three relays. The algorithm is unique in the sense that  the actual mathematical expressions, that are prohibitively large, need not be explicitly obtained. The diversity gain due to multiple relays is shown through plots of the analytical BER, well supported by simulations. 
%
%\end{abstract}
% IEEEtran.cls defaults to using nonbold math in the Abstract.
% This preserves the distinction between vectors and scalars. However,
% if the journal you are submitting to favors bold math in the abstract,
% then you can use LaTeX's standard command \boldmath at the very start
% of the abstract to achieve this. Many IEEE journals frown on math
% in the abstract anyway.

% Note that keywords are not normally used for peerreview papers.
%\begin{IEEEkeywords}
%Cooperative diversity, decode and forward, piecewise linear
%\end{IEEEkeywords}



% For peer review papers, you can put extra information on the cover
% page as needed:
% \ifCLASSOPTIONpeerreview
% \begin{center} \bfseries EDICS Category: 3-BBND \end{center}
% \fi
%
% For peerreview papers, this IEEEtran command inserts a page break and
% creates the second title. It will be ignored for other modes.
%\IEEEpeerreviewmaketitle




   \item Four candidates A, B, C, D have ap-
plied for the assignment to coach a school cricket
team. If A is twice as likely to be selected as B, and
B and C are given about the same chance of being
selected, while C is twice as likely to be selected
as D, what are the probabilities that
\begin{enumerate}
\item C will be selected?
\item A will not be selected?
\end{enumerate}
	%\begin{table}[H]
	\centering
\begin{tabular}{|c|c|c|}
\hline
Random variable &Value &Definition\\ \hline
\multirow{3}{*}{X} &0 &Slips of Rs 1\\
&1 &Slips of Rs 5\\
&2 &Slips of Rs 13\\ \hline
\multirow{2}{*}{Y} &0 &Box A\\
&1 &Box B\\\hline
\end{tabular}
\caption{}
\label{tab:Distribution}
\end{table}
See \tabref{tab:Distribution}.
\begin{align}
p_{Y}\brak{k}= \begin{cases} 
      \frac{1}{3} & {k=0} \\
      \frac{2}{3 }& {k=1} 
   \end{cases}
   \\
p_{Y|X}\brak{0|0} = \frac{19}{25}\, 
p_{Y|X}\brak{0|1} = \frac{6}{25}\,
p_{Y|X}\brak{1|0} = \frac{45}{50}\,
p_{Y|X}\brak{1|2} = \frac{5}{50}
\end{align}
The desired probability is the probability that a slip drawn at random is marked other than Rs 1,
\begin{align}
&=1-p_X\brak{0}\\
&= p_X(1) + p_X(2)
\end{align}
Using Bayes theorem,
\begin{align}
&= p_Y\brak{0} \times \pr{Y=0 | X=1} + p_Y\brak{1} \times \pr{Y=1|X=2}\\
&=\frac{1}{3} \times \frac{6}{25} + \frac{2}{3} \times \frac{5}{50}\\
&=\frac{11}{75}
\end{align}

\newpage

%\tableofcontents

\bigskip

\renewcommand{\thefigure}{\theenumi}
\renewcommand{\thetable}{\theenumi}
%\renewcommand{\theequation}{\theenumi}

%\begin{abstract}
%%\boldmath
%In this letter, an algorithm for evaluating the exact analytical bit error rate  (BER)  for the piecewise linear (PL) combiner for  multiple relays is presented. Previous results were available only for upto three relays. The algorithm is unique in the sense that  the actual mathematical expressions, that are prohibitively large, need not be explicitly obtained. The diversity gain due to multiple relays is shown through plots of the analytical BER, well supported by simulations. 
%
%\end{abstract}
% IEEEtran.cls defaults to using nonbold math in the Abstract.
% This preserves the distinction between vectors and scalars. However,
% if the journal you are submitting to favors bold math in the abstract,
% then you can use LaTeX's standard command \boldmath at the very start
% of the abstract to achieve this. Many IEEE journals frown on math
% in the abstract anyway.

% Note that keywords are not normally used for peerreview papers.
%\begin{IEEEkeywords}
%Cooperative diversity, decode and forward, piecewise linear
%\end{IEEEkeywords}



% For peer review papers, you can put extra information on the cover
% page as needed:
% \ifCLASSOPTIONpeerreview
% \begin{center} \bfseries EDICS Category: 3-BBND \end{center}
% \fi
%
% For peerreview papers, this IEEEtran command inserts a page break and
% creates the second title. It will be ignored for other modes.
%\IEEEpeerreviewmaketitle




 \item A bag contain 24 balls of which $x$ balls are red, $2x$ are white and $3x$ are blue. A ball is selected at random, What is the probability that it is
\begin{enumerate}[label=\alph*)]
\item not red ?
\item white ?
\end{enumerate}
%\begin{table}[H]
	\centering
\begin{tabular}{|c|c|c|}
\hline
Random variable &Value &Definition\\ \hline
\multirow{3}{*}{X} &0 &Slips of Rs 1\\
&1 &Slips of Rs 5\\
&2 &Slips of Rs 13\\ \hline
\multirow{2}{*}{Y} &0 &Box A\\
&1 &Box B\\\hline
\end{tabular}
\caption{}
\label{tab:Distribution}
\end{table}
See \tabref{tab:Distribution}.
\begin{align}
p_{Y}\brak{k}= \begin{cases} 
      \frac{1}{3} & {k=0} \\
      \frac{2}{3 }& {k=1} 
   \end{cases}
   \\
p_{Y|X}\brak{0|0} = \frac{19}{25}\, 
p_{Y|X}\brak{0|1} = \frac{6}{25}\,
p_{Y|X}\brak{1|0} = \frac{45}{50}\,
p_{Y|X}\brak{1|2} = \frac{5}{50}
\end{align}
The desired probability is the probability that a slip drawn at random is marked other than Rs 1,
\begin{align}
&=1-p_X\brak{0}\\
&= p_X(1) + p_X(2)
\end{align}
Using Bayes theorem,
\begin{align}
&= p_Y\brak{0} \times \pr{Y=0 | X=1} + p_Y\brak{1} \times \pr{Y=1|X=2}\\
&=\frac{1}{3} \times \frac{6}{25} + \frac{2}{3} \times \frac{5}{50}\\
&=\frac{11}{75}
\end{align}

\newpage

%\tableofcontents

\bigskip

\renewcommand{\thefigure}{\theenumi}
\renewcommand{\thetable}{\theenumi}
%\renewcommand{\theequation}{\theenumi}

%\begin{abstract}
%%\boldmath
%In this letter, an algorithm for evaluating the exact analytical bit error rate  (BER)  for the piecewise linear (PL) combiner for  multiple relays is presented. Previous results were available only for upto three relays. The algorithm is unique in the sense that  the actual mathematical expressions, that are prohibitively large, need not be explicitly obtained. The diversity gain due to multiple relays is shown through plots of the analytical BER, well supported by simulations. 
%
%\end{abstract}
% IEEEtran.cls defaults to using nonbold math in the Abstract.
% This preserves the distinction between vectors and scalars. However,
% if the journal you are submitting to favors bold math in the abstract,
% then you can use LaTeX's standard command \boldmath at the very start
% of the abstract to achieve this. Many IEEE journals frown on math
% in the abstract anyway.

% Note that keywords are not normally used for peerreview papers.
%\begin{IEEEkeywords}
%Cooperative diversity, decode and forward, piecewise linear
%\end{IEEEkeywords}



% For peer review papers, you can put extra information on the cover
% page as needed:
% \ifCLASSOPTIONpeerreview
% \begin{center} \bfseries EDICS Category: 3-BBND \end{center}
% \fi
%
% For peerreview papers, this IEEEtran command inserts a page break and
% creates the second title. It will be ignored for other modes.
%\IEEEpeerreviewmaketitle




If the letters of the word ASSASSINATION are arranged at random. Find the Probability that
\begin{enumerate}[label=(\alph*)]
\item Four $S's$ come consecutively in the word
\item Two  $I's$ and two $N's$ come together
\item All $A's$ are not coming together
\item No two $A's$ are coming together
\end{enumerate}
%\begin{table}[H]
	\centering
\begin{tabular}{|c|c|c|}
\hline
Random variable &Value &Definition\\ \hline
\multirow{3}{*}{X} &0 &Slips of Rs 1\\
&1 &Slips of Rs 5\\
&2 &Slips of Rs 13\\ \hline
\multirow{2}{*}{Y} &0 &Box A\\
&1 &Box B\\\hline
\end{tabular}
\caption{}
\label{tab:Distribution}
\end{table}
See \tabref{tab:Distribution}.
\begin{align}
p_{Y}\brak{k}= \begin{cases} 
      \frac{1}{3} & {k=0} \\
      \frac{2}{3 }& {k=1} 
   \end{cases}
   \\
p_{Y|X}\brak{0|0} = \frac{19}{25}\, 
p_{Y|X}\brak{0|1} = \frac{6}{25}\,
p_{Y|X}\brak{1|0} = \frac{45}{50}\,
p_{Y|X}\brak{1|2} = \frac{5}{50}
\end{align}
The desired probability is the probability that a slip drawn at random is marked other than Rs 1,
\begin{align}
&=1-p_X\brak{0}\\
&= p_X(1) + p_X(2)
\end{align}
Using Bayes theorem,
\begin{align}
&= p_Y\brak{0} \times \pr{Y=0 | X=1} + p_Y\brak{1} \times \pr{Y=1|X=2}\\
&=\frac{1}{3} \times \frac{6}{25} + \frac{2}{3} \times \frac{5}{50}\\
&=\frac{11}{75}
\end{align}

\newpage

%\tableofcontents

\bigskip

\renewcommand{\thefigure}{\theenumi}
\renewcommand{\thetable}{\theenumi}
%\renewcommand{\theequation}{\theenumi}

%\begin{abstract}
%%\boldmath
%In this letter, an algorithm for evaluating the exact analytical bit error rate  (BER)  for the piecewise linear (PL) combiner for  multiple relays is presented. Previous results were available only for upto three relays. The algorithm is unique in the sense that  the actual mathematical expressions, that are prohibitively large, need not be explicitly obtained. The diversity gain due to multiple relays is shown through plots of the analytical BER, well supported by simulations. 
%
%\end{abstract}
% IEEEtran.cls defaults to using nonbold math in the Abstract.
% This preserves the distinction between vectors and scalars. However,
% if the journal you are submitting to favors bold math in the abstract,
% then you can use LaTeX's standard command \boldmath at the very start
% of the abstract to achieve this. Many IEEE journals frown on math
% in the abstract anyway.

% Note that keywords are not normally used for peerreview papers.
%\begin{IEEEkeywords}
%Cooperative diversity, decode and forward, piecewise linear
%\end{IEEEkeywords}



% For peer review papers, you can put extra information on the cover
% page as needed:
% \ifCLASSOPTIONpeerreview
% \begin{center} \bfseries EDICS Category: 3-BBND \end{center}
% \fi
%
% For peerreview papers, this IEEEtran command inserts a page break and
% creates the second title. It will be ignored for other modes.
%\IEEEpeerreviewmaketitle




	\item One urn contains two black balls (labelled B1 and B2) and one white ball. A
	second urn contains one black ball and two white balls (labelled W1 and W2).
	Suppose the following experiment is performed. One of the two urns is chosen
	at random. Next a ball is randomly chosen from the urn. Then a second ball is
	chosen at random from the same urn without replacing the first ball.
	
	\begin{enumerate}
	\item What is the probability that two black balls are chosen?
	
	\item What is the probability that two balls of opposite colour are chosen?
	\end{enumerate}
	\solution
	%\begin{align}
    \label{eq:12.13.6.18.1}
	\because	\pr{A|B} &> \pr{A},\
\frac{\pr{AB}}{\pr{B}} > \pr{A}
\\
    \label{eq:12.13.6.18.2}
	\implies \pr{AB} &> \pr{A}\pr{B}
	\\
	\text{or, } \frac{\pr{AB}}{\pr{A}} &=\pr{B|A} > \pr{A}
\end{align}

\end{enumerate}

	\item 
The number lock of a suitcase has 4 wheels each labelled with ten digits i.e. from 0 to 9.The lock opens with a sequence of four digits with no repeats.What is the probability of a person getting the right sequence to open the suitcase.
\\
\solution
		%\begin{enumerate}[label=\thesection.\arabic*,ref=\thesection.\theenumi]
	\item One card is drawn from a well-shuffled deck of 52 cards. Find the probability of getting
\begin{enumerate}
\item A king of red colour 
\item A face card 
\item A red face card
\item The jack of hearts
\item A spade
\item The queen of diamonds

\end{enumerate}
\solution
		%\begin{table}[H]
	\centering
\begin{tabular}{|c|c|c|}
\hline
Random variable &Value &Definition\\ \hline
\multirow{3}{*}{X} &0 &Slips of Rs 1\\
&1 &Slips of Rs 5\\
&2 &Slips of Rs 13\\ \hline
\multirow{2}{*}{Y} &0 &Box A\\
&1 &Box B\\\hline
\end{tabular}
\caption{}
\label{tab:Distribution}
\end{table}
See \tabref{tab:Distribution}.
\begin{align}
p_{Y}\brak{k}= \begin{cases} 
      \frac{1}{3} & {k=0} \\
      \frac{2}{3 }& {k=1} 
   \end{cases}
   \\
p_{Y|X}\brak{0|0} = \frac{19}{25}\, 
p_{Y|X}\brak{0|1} = \frac{6}{25}\,
p_{Y|X}\brak{1|0} = \frac{45}{50}\,
p_{Y|X}\brak{1|2} = \frac{5}{50}
\end{align}
The desired probability is the probability that a slip drawn at random is marked other than Rs 1,
\begin{align}
&=1-p_X\brak{0}\\
&= p_X(1) + p_X(2)
\end{align}
Using Bayes theorem,
\begin{align}
&= p_Y\brak{0} \times \pr{Y=0 | X=1} + p_Y\brak{1} \times \pr{Y=1|X=2}\\
&=\frac{1}{3} \times \frac{6}{25} + \frac{2}{3} \times \frac{5}{50}\\
&=\frac{11}{75}
\end{align}

\newpage

%\tableofcontents

\bigskip

\renewcommand{\thefigure}{\theenumi}
\renewcommand{\thetable}{\theenumi}
%\renewcommand{\theequation}{\theenumi}

%\begin{abstract}
%%\boldmath
%In this letter, an algorithm for evaluating the exact analytical bit error rate  (BER)  for the piecewise linear (PL) combiner for  multiple relays is presented. Previous results were available only for upto three relays. The algorithm is unique in the sense that  the actual mathematical expressions, that are prohibitively large, need not be explicitly obtained. The diversity gain due to multiple relays is shown through plots of the analytical BER, well supported by simulations. 
%
%\end{abstract}
% IEEEtran.cls defaults to using nonbold math in the Abstract.
% This preserves the distinction between vectors and scalars. However,
% if the journal you are submitting to favors bold math in the abstract,
% then you can use LaTeX's standard command \boldmath at the very start
% of the abstract to achieve this. Many IEEE journals frown on math
% in the abstract anyway.

% Note that keywords are not normally used for peerreview papers.
%\begin{IEEEkeywords}
%Cooperative diversity, decode and forward, piecewise linear
%\end{IEEEkeywords}



% For peer review papers, you can put extra information on the cover
% page as needed:
% \ifCLASSOPTIONpeerreview
% \begin{center} \bfseries EDICS Category: 3-BBND \end{center}
% \fi
%
% For peerreview papers, this IEEEtran command inserts a page break and
% creates the second title. It will be ignored for other modes.
%\IEEEpeerreviewmaketitle




	\item Five cards—the ten, jack, queen, king and ace of diamonds, are well-shuffled with their face downwards. One card is then picked up at random.
\begin{enumerate}
\item
What is the probability that the card is the queen? 
\item
If the queen is drawn and put aside, what is the probability that the second card picked up is (a) an ace? (b) a queen?\\
\end{enumerate}
\solution
		%\begin{enumerate}[label=\thesection.\arabic*,ref=\thesection.\theenumi]
	\item One card is drawn from a well-shuffled deck of 52 cards. Find the probability of getting
\begin{enumerate}
\item A king of red colour 
\item A face card 
\item A red face card
\item The jack of hearts
\item A spade
\item The queen of diamonds

\end{enumerate}
\solution
		%\input{ncert/10/15/1/14/main.tex}
	\item Five cards—the ten, jack, queen, king and ace of diamonds, are well-shuffled with their face downwards. One card is then picked up at random.
\begin{enumerate}
\item
What is the probability that the card is the queen? 
\item
If the queen is drawn and put aside, what is the probability that the second card picked up is (a) an ace? (b) a queen?\\
\end{enumerate}
\solution
		%\input{ncert/10/15/1/15/defs.tex}
	\item A bag contains $5$ red balls and some blue balls. If the probability of drawing a blue ball is double that if a red ball, determine the number of blue balls in the bag. 
		\\
\solution
		%\input{ncert/10/15/2/3/defs.tex}
	\item A card is selected from a pack of 52 cards.
 \begin{enumerate}[label=(\alph*)] 
                 \item How many points are there in the sample space?
                 \item Calculate the probability that the card is an ace of spades.
                 \item Calculate the probability that the card is (i) an ace and (ii) black card.
 \end{enumerate}
\solution
		%\input{ncert/11/16/3/4/main.tex}
\item Four cards are drawn from a well-shuffled deck of 52 cards. What is the probability of obtaining 3 diamonds and one spade.
\\
\solution
		%\input{ncert/11/16/4/2/defs.tex}
\item In a certain lottery 10,000 tickets are sold and ten equal prizes are awarded. What is the probability of not getting a prize if you buy (a) one ticket (b) two tickets (c) 10 tickets ?	
\\
\solution
		%\input{ncert/11/16/4/4/defs.tex}
		%
\item 
Out of 100 students, two sections of 40 and 60 are formed. If you and your friend are among the 100 students, what is the probability that
\begin{enumerate}
\item you both enter the same section?
\item you both enter the different sections?
\end{enumerate}
\solution
		%\input{ncert/11/16/4/5/defs.tex}
	\item 
The number lock of a suitcase has 4 wheels each labelled with ten digits i.e. from 0 to 9.The lock opens with a sequence of four digits with no repeats.What is the probability of a person getting the right sequence to open the suitcase.
\\
\solution
		%\input{ncert/11/16/4/10/defs.tex}
		%
\item 
Two cards are drawn at random and without replacement from a pack of 52 playing cards. Find the probability that both the cards are black.
\\
\solution
		%\input{ncert/12/13/2/2/defs.tex}
		\item A box of oranges is inspected by examining three randomly selected oranges drawn without replacement. If all the three oranges are good, the box is approved for sale, otherwise, it is rejected. Find the probability that a box containing 15 oranges out of which 12 are good and 3 are bad ones will be approved for sale.
		\label{ncert/12/13/2/3/defs.tex}
		\item Two balls are drawn at random with replacement from a box containing 10 black and 8 red balls. Find the probability that
		\label{ncert/12/13/2/12}
\begin{enumerate}
\item both balls are red.
\item first ball is black and second is red.
\item one of them is black and other is red.
\end{enumerate}

\item In a hostel, 60\% of the students read Hindi newspaper, 40\% read English newspaper and 20\% read both Hindi and English newspapers. A student is selected at random.
		\label{ncert/12/13/2/15}
\begin{enumerate}
\item Find the probability that she reads neither Hindi nor English newspapers.
\item If she reads Hindi newspaper, find the probability that she reads English newspaper.
\item If she reads English newspaper, find the probability that she reads Hindi newspaper.\\
\end{enumerate}
\item The probability of obtaining an even prime number on each die, when a pair of dice is rolled is 
\begin{enumerate}
    \item $0$ 
    
    \item $\frac{1}{3}$ 
    
    \item $\frac{1}{12}$ 
    
    \item $\frac{1}{36}$ 
\end{enumerate}
\solution
		%\input{ncert/12/13/2/17/defs.tex}
	\item A bag contains 4 red and 4 black balls, another bag contains 2 red and 6 black balls. One of the two bags is selected at random and a ball is drawn from the bag which is found to be red. Find the probability that the ball is drawn from the first bag.
\\
\solution
		%\input{ncert/12/13/3/2/main.tex}
  \item
  Cards with numbers 2 to 101 are placed in a box. A card is selected at random.Find the probability that the card has
\begin{enumerate}[label=(\roman*)]
	\item an even number 
	\item a square number
\end{enumerate}
\solution
%\input{exemplar/10/13/3/32/main.tex}
\item
The king, queen and jack of clubs are removed from a deck of 52 playing cards and then well shuffled. Now one card is drawn at random from the remaining cards.  Determine the probability that the card is
\begin{enumerate}[label=(\roman*)]
\item a club
\item 10 of hearts
\end{enumerate}
\solution
%\input{exemplar/10/13/3/29/main.tex}
\item A team of medical students doing their internship have to assist during surgeries
at a city hospital. The probabilities of surgeries rated as very complex, complex,
routine, simple or very simple are respectively, 0.15, 0.20, 0.31, 0.26, .08. Find
the probabilities that a particular surgery will be rated
\begin{enumerate}
	\item complex or very complex;
	\item neither very complex nor very simple;
	\item routine or complex
	\item routine or simple
\end{enumerate}
\solution
%\input{exemplar/11/16/3/8(1)/main.tex}
\item A card is selected from a pack of 52 cards.
\begin{enumerate}[label=(\alph*)]
    \item How many points are there in the sample space?
    \item Calculate the probability that the card is an ace of spades.
    \item Calculate the probability that the card is (i) an ace and (ii) black card.
\end{enumerate}
\solution
%\input{exemplar/11/16/3/4/main2.tex}
\item The probability that a non leap year selected at random will contain 53 sundays.
\\
\solution
%\input{exemplar/10/13/1/19/main.tex}
\item One of the four persons John, Rita, Aslam or Gurpreet will be promoted next
month. Consequently the sample space consists of four elementary outcomes
S = {John promoted, Rita promoted, Aslam promoted, Gurpreet promoted}
You are told that the chances of John’s promotion is same as that of Gurpreet,
Rita’s chances of promotion are twice as likely as Johns. Aslam’s chances are
four times that of John.
\begin{enumerate}
	\item Determine
	\begin{enumerate}
		\item P (John promoted)
		\item P (Rita promoted)
		\item P (Aslam promoted)
		\item P (Gurpreet promoted)
	\end{enumerate}
	\item If A = {John promoted or Gurpreet promoted}, find P (A).
\end{enumerate}
\solution
%\input{exemplar/11/16/3/10/main.tex}
\item A card is drawn from a deck of 52 cards. Find the probability of getting a king or a heart or a red card.\\
\solution
%\input{exemplar/11/16/3/15/main.tex}
\item The probability that a student will pass his examination is 0.73, the probability of
the student getting a compartment is 0.13, and the probability that the student will
either pass or get compartment is 0.96. State True or False.\\
\solution
%\input{exemplar/11/16/3/31/main.tex}
\item A card is selected from a pack of 52 cards\\
\begin{enumerate}[label=(\alph*)]
\item How many points are there in the sample space?
\item Calculate the probability that the cards is an ace of spades.
\item Calculate the probability that the card is (i) an ace (ii)black card.\\
\end{enumerate}
%\input{ncert/11/16/3/4_1/Prob_4.tex}
\item In a non-leap year, the probability of having 53 tuesdays or 53 wednesdays is\\
\solution
%\input{exemplar/11/16/3/18/main.tex}
\item There are 1000 sealed envelopes in a box, 10 of them contain a cash prize of
Rs 100 each, 100 of them contain a cash prize of Rs 50 each and 200 of them
contain a cash prize of Rs 10 each and rest do not contain any cash prize. If they
are well shuffled and an envelope is picked up out, what is the probability that it
contains no cash prize?\\
\solution
%\input{exemplar/10/13/3/34/main.tex}
\item 
A die is thrown and a card is selected at random from a deck of 52 playing cards. The probability of getting an even number on the die and a spade card.\\
\solution
%\input{exemplar/12/13/3/78/main.tex}
\item
If 4-digit numbers greater than 5,000 are randomly formed from the digits 0, 1, 3, 5, and 7, what is the probability of forming a number divisible by 5 when:
\begin{enumerate}
    \item The digits are repeated?
    \item The repetition of digits is not allowed?
\end{enumerate}
\solution
%\input{ncert/11/16/4/9/main.tex}
\item Consider the probability space $\brak{\Omega, \mathcal{G}, P}$ where $\Omega = [0,2]$ and $\mathcal{G} = \cbrak{\phi, \Omega, [0,1], (1,2]}$. Let $X$ and $Y$ be two functions on $\Omega$ defined as
\begin{align*}
    X(\omega) = 
    \begin{cases}
        1 & \text{if }\omega \in [0, 1]\\
        2 & \text{if }\omega \in (1, 2]
    \end{cases}
\end{align*}
and
\begin{align*}
    Y(\omega) = 
    \begin{cases}
        2 & \text{if }\omega \in [0, 1.5]\\
        3 & \text{if }\omega \in (1.5, 2].
    \end{cases}
\end{align*}
Then which one of the following statements is true?
\begin{enumerate}
    \item [(A)] $X$ is a random variable with respect to $\mathcal{G}$, but $Y$ is not a random variable with respect to $\mathcal{G}$.
    \item [(B)] $Y$ is a random variable with respect to $\mathcal{G}$, but $X$ is not a random variable with respect to $\mathcal{G}$.
    \item [(C)] Neither $X$ nor $Y$ is a random variable with respect to $\mathcal{G}$.
    \item [(D)] Both $X$ and $Y$ are random variables with respect to $\mathcal{G}$.
\end{enumerate} \hfill (GATE ST 2023)\\
\solution
%\input{gate/ST/2023/14/main.tex}
	\item  A die is loaded in such a way that each odd number is twice as likely to occur as
each even number. Find $P(G)$, where $G$ is the event that a number greater than
3 occurs on a single roll of the die.
\\
\solution
		%\input{exemplar/11/16/3/5/main.tex}
	\item All the jacks, queens and kings are removed from a deck of 52 playing cards. The remaining cards are well shuffled and then one card is drawn at random. Giving ace a value 1 similar value for other cards, find the probability that the card has a value 
		\begin{enumerate}
			\item 7
			\item greater than 7
			\item less than 7
		\end{enumerate}
		%\input{exemplar/10/13/3/30/main.tex}
  \item A Lot consists of 48 mobile phones of which 42 are good, 3 have only minor defects and 3 have major defects.Varnika will buy a phone if it is good but the trader will only buy a mobile if it has no major defects. One phone is selected at random from the lot. What is the probability that it is
\begin{enumerate}
	\item acceptable to Varnika?
            \item acceptable to the trader?
\end{enumerate}
\solution
	%\input{exemplar/10/13/3/40/main.tex}
 \item A student says that if you throw a die, it will show up 1 or not 1. Therefore, the probability of getting 1 and the probability of getting 'not 1' each is equal to $\frac{1}{2}$. Is this correct? Give reasons.\\
 \solution
        %\input{exemplar/10/13/2/9/main.tex}
   \item Four candidates A, B, C, D have ap-
plied for the assignment to coach a school cricket
team. If A is twice as likely to be selected as B, and
B and C are given about the same chance of being
selected, while C is twice as likely to be selected
as D, what are the probabilities that
\begin{enumerate}
\item C will be selected?
\item A will not be selected?
\end{enumerate}
	%\input{exemplar/11/16/3/9/main.tex}
 \item A bag contain 24 balls of which $x$ balls are red, $2x$ are white and $3x$ are blue. A ball is selected at random, What is the probability that it is
\begin{enumerate}[label=\alph*)]
\item not red ?
\item white ?
\end{enumerate}
%\input{exemplar/10/13/3/41/main.tex}
If the letters of the word ASSASSINATION are arranged at random. Find the Probability that
\begin{enumerate}[label=(\alph*)]
\item Four $S's$ come consecutively in the word
\item Two  $I's$ and two $N's$ come together
\item All $A's$ are not coming together
\item No two $A's$ are coming together
\end{enumerate}
%\input{exemplar/11/16/3/14/main.tex}
	\item One urn contains two black balls (labelled B1 and B2) and one white ball. A
	second urn contains one black ball and two white balls (labelled W1 and W2).
	Suppose the following experiment is performed. One of the two urns is chosen
	at random. Next a ball is randomly chosen from the urn. Then a second ball is
	chosen at random from the same urn without replacing the first ball.
	
	\begin{enumerate}
	\item What is the probability that two black balls are chosen?
	
	\item What is the probability that two balls of opposite colour are chosen?
	\end{enumerate}
	\solution
	%\input{exemplar/11/16/3/12/main1.tex}
\end{enumerate}

	\item A bag contains $5$ red balls and some blue balls. If the probability of drawing a blue ball is double that if a red ball, determine the number of blue balls in the bag. 
		\\
\solution
		%\begin{enumerate}[label=\thesection.\arabic*,ref=\thesection.\theenumi]
	\item One card is drawn from a well-shuffled deck of 52 cards. Find the probability of getting
\begin{enumerate}
\item A king of red colour 
\item A face card 
\item A red face card
\item The jack of hearts
\item A spade
\item The queen of diamonds

\end{enumerate}
\solution
		%\input{ncert/10/15/1/14/main.tex}
	\item Five cards—the ten, jack, queen, king and ace of diamonds, are well-shuffled with their face downwards. One card is then picked up at random.
\begin{enumerate}
\item
What is the probability that the card is the queen? 
\item
If the queen is drawn and put aside, what is the probability that the second card picked up is (a) an ace? (b) a queen?\\
\end{enumerate}
\solution
		%\input{ncert/10/15/1/15/defs.tex}
	\item A bag contains $5$ red balls and some blue balls. If the probability of drawing a blue ball is double that if a red ball, determine the number of blue balls in the bag. 
		\\
\solution
		%\input{ncert/10/15/2/3/defs.tex}
	\item A card is selected from a pack of 52 cards.
 \begin{enumerate}[label=(\alph*)] 
                 \item How many points are there in the sample space?
                 \item Calculate the probability that the card is an ace of spades.
                 \item Calculate the probability that the card is (i) an ace and (ii) black card.
 \end{enumerate}
\solution
		%\input{ncert/11/16/3/4/main.tex}
\item Four cards are drawn from a well-shuffled deck of 52 cards. What is the probability of obtaining 3 diamonds and one spade.
\\
\solution
		%\input{ncert/11/16/4/2/defs.tex}
\item In a certain lottery 10,000 tickets are sold and ten equal prizes are awarded. What is the probability of not getting a prize if you buy (a) one ticket (b) two tickets (c) 10 tickets ?	
\\
\solution
		%\input{ncert/11/16/4/4/defs.tex}
		%
\item 
Out of 100 students, two sections of 40 and 60 are formed. If you and your friend are among the 100 students, what is the probability that
\begin{enumerate}
\item you both enter the same section?
\item you both enter the different sections?
\end{enumerate}
\solution
		%\input{ncert/11/16/4/5/defs.tex}
	\item 
The number lock of a suitcase has 4 wheels each labelled with ten digits i.e. from 0 to 9.The lock opens with a sequence of four digits with no repeats.What is the probability of a person getting the right sequence to open the suitcase.
\\
\solution
		%\input{ncert/11/16/4/10/defs.tex}
		%
\item 
Two cards are drawn at random and without replacement from a pack of 52 playing cards. Find the probability that both the cards are black.
\\
\solution
		%\input{ncert/12/13/2/2/defs.tex}
		\item A box of oranges is inspected by examining three randomly selected oranges drawn without replacement. If all the three oranges are good, the box is approved for sale, otherwise, it is rejected. Find the probability that a box containing 15 oranges out of which 12 are good and 3 are bad ones will be approved for sale.
		\label{ncert/12/13/2/3/defs.tex}
		\item Two balls are drawn at random with replacement from a box containing 10 black and 8 red balls. Find the probability that
		\label{ncert/12/13/2/12}
\begin{enumerate}
\item both balls are red.
\item first ball is black and second is red.
\item one of them is black and other is red.
\end{enumerate}

\item In a hostel, 60\% of the students read Hindi newspaper, 40\% read English newspaper and 20\% read both Hindi and English newspapers. A student is selected at random.
		\label{ncert/12/13/2/15}
\begin{enumerate}
\item Find the probability that she reads neither Hindi nor English newspapers.
\item If she reads Hindi newspaper, find the probability that she reads English newspaper.
\item If she reads English newspaper, find the probability that she reads Hindi newspaper.\\
\end{enumerate}
\item The probability of obtaining an even prime number on each die, when a pair of dice is rolled is 
\begin{enumerate}
    \item $0$ 
    
    \item $\frac{1}{3}$ 
    
    \item $\frac{1}{12}$ 
    
    \item $\frac{1}{36}$ 
\end{enumerate}
\solution
		%\input{ncert/12/13/2/17/defs.tex}
	\item A bag contains 4 red and 4 black balls, another bag contains 2 red and 6 black balls. One of the two bags is selected at random and a ball is drawn from the bag which is found to be red. Find the probability that the ball is drawn from the first bag.
\\
\solution
		%\input{ncert/12/13/3/2/main.tex}
  \item
  Cards with numbers 2 to 101 are placed in a box. A card is selected at random.Find the probability that the card has
\begin{enumerate}[label=(\roman*)]
	\item an even number 
	\item a square number
\end{enumerate}
\solution
%\input{exemplar/10/13/3/32/main.tex}
\item
The king, queen and jack of clubs are removed from a deck of 52 playing cards and then well shuffled. Now one card is drawn at random from the remaining cards.  Determine the probability that the card is
\begin{enumerate}[label=(\roman*)]
\item a club
\item 10 of hearts
\end{enumerate}
\solution
%\input{exemplar/10/13/3/29/main.tex}
\item A team of medical students doing their internship have to assist during surgeries
at a city hospital. The probabilities of surgeries rated as very complex, complex,
routine, simple or very simple are respectively, 0.15, 0.20, 0.31, 0.26, .08. Find
the probabilities that a particular surgery will be rated
\begin{enumerate}
	\item complex or very complex;
	\item neither very complex nor very simple;
	\item routine or complex
	\item routine or simple
\end{enumerate}
\solution
%\input{exemplar/11/16/3/8(1)/main.tex}
\item A card is selected from a pack of 52 cards.
\begin{enumerate}[label=(\alph*)]
    \item How many points are there in the sample space?
    \item Calculate the probability that the card is an ace of spades.
    \item Calculate the probability that the card is (i) an ace and (ii) black card.
\end{enumerate}
\solution
%\input{exemplar/11/16/3/4/main2.tex}
\item The probability that a non leap year selected at random will contain 53 sundays.
\\
\solution
%\input{exemplar/10/13/1/19/main.tex}
\item One of the four persons John, Rita, Aslam or Gurpreet will be promoted next
month. Consequently the sample space consists of four elementary outcomes
S = {John promoted, Rita promoted, Aslam promoted, Gurpreet promoted}
You are told that the chances of John’s promotion is same as that of Gurpreet,
Rita’s chances of promotion are twice as likely as Johns. Aslam’s chances are
four times that of John.
\begin{enumerate}
	\item Determine
	\begin{enumerate}
		\item P (John promoted)
		\item P (Rita promoted)
		\item P (Aslam promoted)
		\item P (Gurpreet promoted)
	\end{enumerate}
	\item If A = {John promoted or Gurpreet promoted}, find P (A).
\end{enumerate}
\solution
%\input{exemplar/11/16/3/10/main.tex}
\item A card is drawn from a deck of 52 cards. Find the probability of getting a king or a heart or a red card.\\
\solution
%\input{exemplar/11/16/3/15/main.tex}
\item The probability that a student will pass his examination is 0.73, the probability of
the student getting a compartment is 0.13, and the probability that the student will
either pass or get compartment is 0.96. State True or False.\\
\solution
%\input{exemplar/11/16/3/31/main.tex}
\item A card is selected from a pack of 52 cards\\
\begin{enumerate}[label=(\alph*)]
\item How many points are there in the sample space?
\item Calculate the probability that the cards is an ace of spades.
\item Calculate the probability that the card is (i) an ace (ii)black card.\\
\end{enumerate}
%\input{ncert/11/16/3/4_1/Prob_4.tex}
\item In a non-leap year, the probability of having 53 tuesdays or 53 wednesdays is\\
\solution
%\input{exemplar/11/16/3/18/main.tex}
\item There are 1000 sealed envelopes in a box, 10 of them contain a cash prize of
Rs 100 each, 100 of them contain a cash prize of Rs 50 each and 200 of them
contain a cash prize of Rs 10 each and rest do not contain any cash prize. If they
are well shuffled and an envelope is picked up out, what is the probability that it
contains no cash prize?\\
\solution
%\input{exemplar/10/13/3/34/main.tex}
\item 
A die is thrown and a card is selected at random from a deck of 52 playing cards. The probability of getting an even number on the die and a spade card.\\
\solution
%\input{exemplar/12/13/3/78/main.tex}
\item
If 4-digit numbers greater than 5,000 are randomly formed from the digits 0, 1, 3, 5, and 7, what is the probability of forming a number divisible by 5 when:
\begin{enumerate}
    \item The digits are repeated?
    \item The repetition of digits is not allowed?
\end{enumerate}
\solution
%\input{ncert/11/16/4/9/main.tex}
\item Consider the probability space $\brak{\Omega, \mathcal{G}, P}$ where $\Omega = [0,2]$ and $\mathcal{G} = \cbrak{\phi, \Omega, [0,1], (1,2]}$. Let $X$ and $Y$ be two functions on $\Omega$ defined as
\begin{align*}
    X(\omega) = 
    \begin{cases}
        1 & \text{if }\omega \in [0, 1]\\
        2 & \text{if }\omega \in (1, 2]
    \end{cases}
\end{align*}
and
\begin{align*}
    Y(\omega) = 
    \begin{cases}
        2 & \text{if }\omega \in [0, 1.5]\\
        3 & \text{if }\omega \in (1.5, 2].
    \end{cases}
\end{align*}
Then which one of the following statements is true?
\begin{enumerate}
    \item [(A)] $X$ is a random variable with respect to $\mathcal{G}$, but $Y$ is not a random variable with respect to $\mathcal{G}$.
    \item [(B)] $Y$ is a random variable with respect to $\mathcal{G}$, but $X$ is not a random variable with respect to $\mathcal{G}$.
    \item [(C)] Neither $X$ nor $Y$ is a random variable with respect to $\mathcal{G}$.
    \item [(D)] Both $X$ and $Y$ are random variables with respect to $\mathcal{G}$.
\end{enumerate} \hfill (GATE ST 2023)\\
\solution
%\input{gate/ST/2023/14/main.tex}
	\item  A die is loaded in such a way that each odd number is twice as likely to occur as
each even number. Find $P(G)$, where $G$ is the event that a number greater than
3 occurs on a single roll of the die.
\\
\solution
		%\input{exemplar/11/16/3/5/main.tex}
	\item All the jacks, queens and kings are removed from a deck of 52 playing cards. The remaining cards are well shuffled and then one card is drawn at random. Giving ace a value 1 similar value for other cards, find the probability that the card has a value 
		\begin{enumerate}
			\item 7
			\item greater than 7
			\item less than 7
		\end{enumerate}
		%\input{exemplar/10/13/3/30/main.tex}
  \item A Lot consists of 48 mobile phones of which 42 are good, 3 have only minor defects and 3 have major defects.Varnika will buy a phone if it is good but the trader will only buy a mobile if it has no major defects. One phone is selected at random from the lot. What is the probability that it is
\begin{enumerate}
	\item acceptable to Varnika?
            \item acceptable to the trader?
\end{enumerate}
\solution
	%\input{exemplar/10/13/3/40/main.tex}
 \item A student says that if you throw a die, it will show up 1 or not 1. Therefore, the probability of getting 1 and the probability of getting 'not 1' each is equal to $\frac{1}{2}$. Is this correct? Give reasons.\\
 \solution
        %\input{exemplar/10/13/2/9/main.tex}
   \item Four candidates A, B, C, D have ap-
plied for the assignment to coach a school cricket
team. If A is twice as likely to be selected as B, and
B and C are given about the same chance of being
selected, while C is twice as likely to be selected
as D, what are the probabilities that
\begin{enumerate}
\item C will be selected?
\item A will not be selected?
\end{enumerate}
	%\input{exemplar/11/16/3/9/main.tex}
 \item A bag contain 24 balls of which $x$ balls are red, $2x$ are white and $3x$ are blue. A ball is selected at random, What is the probability that it is
\begin{enumerate}[label=\alph*)]
\item not red ?
\item white ?
\end{enumerate}
%\input{exemplar/10/13/3/41/main.tex}
If the letters of the word ASSASSINATION are arranged at random. Find the Probability that
\begin{enumerate}[label=(\alph*)]
\item Four $S's$ come consecutively in the word
\item Two  $I's$ and two $N's$ come together
\item All $A's$ are not coming together
\item No two $A's$ are coming together
\end{enumerate}
%\input{exemplar/11/16/3/14/main.tex}
	\item One urn contains two black balls (labelled B1 and B2) and one white ball. A
	second urn contains one black ball and two white balls (labelled W1 and W2).
	Suppose the following experiment is performed. One of the two urns is chosen
	at random. Next a ball is randomly chosen from the urn. Then a second ball is
	chosen at random from the same urn without replacing the first ball.
	
	\begin{enumerate}
	\item What is the probability that two black balls are chosen?
	
	\item What is the probability that two balls of opposite colour are chosen?
	\end{enumerate}
	\solution
	%\input{exemplar/11/16/3/12/main1.tex}
\end{enumerate}

	\item A card is selected from a pack of 52 cards.
 \begin{enumerate}[label=(\alph*)] 
                 \item How many points are there in the sample space?
                 \item Calculate the probability that the card is an ace of spades.
                 \item Calculate the probability that the card is (i) an ace and (ii) black card.
 \end{enumerate}
\solution
		%\begin{table}[H]
	\centering
\begin{tabular}{|c|c|c|}
\hline
Random variable &Value &Definition\\ \hline
\multirow{3}{*}{X} &0 &Slips of Rs 1\\
&1 &Slips of Rs 5\\
&2 &Slips of Rs 13\\ \hline
\multirow{2}{*}{Y} &0 &Box A\\
&1 &Box B\\\hline
\end{tabular}
\caption{}
\label{tab:Distribution}
\end{table}
See \tabref{tab:Distribution}.
\begin{align}
p_{Y}\brak{k}= \begin{cases} 
      \frac{1}{3} & {k=0} \\
      \frac{2}{3 }& {k=1} 
   \end{cases}
   \\
p_{Y|X}\brak{0|0} = \frac{19}{25}\, 
p_{Y|X}\brak{0|1} = \frac{6}{25}\,
p_{Y|X}\brak{1|0} = \frac{45}{50}\,
p_{Y|X}\brak{1|2} = \frac{5}{50}
\end{align}
The desired probability is the probability that a slip drawn at random is marked other than Rs 1,
\begin{align}
&=1-p_X\brak{0}\\
&= p_X(1) + p_X(2)
\end{align}
Using Bayes theorem,
\begin{align}
&= p_Y\brak{0} \times \pr{Y=0 | X=1} + p_Y\brak{1} \times \pr{Y=1|X=2}\\
&=\frac{1}{3} \times \frac{6}{25} + \frac{2}{3} \times \frac{5}{50}\\
&=\frac{11}{75}
\end{align}

\newpage

%\tableofcontents

\bigskip

\renewcommand{\thefigure}{\theenumi}
\renewcommand{\thetable}{\theenumi}
%\renewcommand{\theequation}{\theenumi}

%\begin{abstract}
%%\boldmath
%In this letter, an algorithm for evaluating the exact analytical bit error rate  (BER)  for the piecewise linear (PL) combiner for  multiple relays is presented. Previous results were available only for upto three relays. The algorithm is unique in the sense that  the actual mathematical expressions, that are prohibitively large, need not be explicitly obtained. The diversity gain due to multiple relays is shown through plots of the analytical BER, well supported by simulations. 
%
%\end{abstract}
% IEEEtran.cls defaults to using nonbold math in the Abstract.
% This preserves the distinction between vectors and scalars. However,
% if the journal you are submitting to favors bold math in the abstract,
% then you can use LaTeX's standard command \boldmath at the very start
% of the abstract to achieve this. Many IEEE journals frown on math
% in the abstract anyway.

% Note that keywords are not normally used for peerreview papers.
%\begin{IEEEkeywords}
%Cooperative diversity, decode and forward, piecewise linear
%\end{IEEEkeywords}



% For peer review papers, you can put extra information on the cover
% page as needed:
% \ifCLASSOPTIONpeerreview
% \begin{center} \bfseries EDICS Category: 3-BBND \end{center}
% \fi
%
% For peerreview papers, this IEEEtran command inserts a page break and
% creates the second title. It will be ignored for other modes.
%\IEEEpeerreviewmaketitle




\item Four cards are drawn from a well-shuffled deck of 52 cards. What is the probability of obtaining 3 diamonds and one spade.
\\
\solution
		%\begin{enumerate}[label=\thesection.\arabic*,ref=\thesection.\theenumi]
	\item One card is drawn from a well-shuffled deck of 52 cards. Find the probability of getting
\begin{enumerate}
\item A king of red colour 
\item A face card 
\item A red face card
\item The jack of hearts
\item A spade
\item The queen of diamonds

\end{enumerate}
\solution
		%\input{ncert/10/15/1/14/main.tex}
	\item Five cards—the ten, jack, queen, king and ace of diamonds, are well-shuffled with their face downwards. One card is then picked up at random.
\begin{enumerate}
\item
What is the probability that the card is the queen? 
\item
If the queen is drawn and put aside, what is the probability that the second card picked up is (a) an ace? (b) a queen?\\
\end{enumerate}
\solution
		%\input{ncert/10/15/1/15/defs.tex}
	\item A bag contains $5$ red balls and some blue balls. If the probability of drawing a blue ball is double that if a red ball, determine the number of blue balls in the bag. 
		\\
\solution
		%\input{ncert/10/15/2/3/defs.tex}
	\item A card is selected from a pack of 52 cards.
 \begin{enumerate}[label=(\alph*)] 
                 \item How many points are there in the sample space?
                 \item Calculate the probability that the card is an ace of spades.
                 \item Calculate the probability that the card is (i) an ace and (ii) black card.
 \end{enumerate}
\solution
		%\input{ncert/11/16/3/4/main.tex}
\item Four cards are drawn from a well-shuffled deck of 52 cards. What is the probability of obtaining 3 diamonds and one spade.
\\
\solution
		%\input{ncert/11/16/4/2/defs.tex}
\item In a certain lottery 10,000 tickets are sold and ten equal prizes are awarded. What is the probability of not getting a prize if you buy (a) one ticket (b) two tickets (c) 10 tickets ?	
\\
\solution
		%\input{ncert/11/16/4/4/defs.tex}
		%
\item 
Out of 100 students, two sections of 40 and 60 are formed. If you and your friend are among the 100 students, what is the probability that
\begin{enumerate}
\item you both enter the same section?
\item you both enter the different sections?
\end{enumerate}
\solution
		%\input{ncert/11/16/4/5/defs.tex}
	\item 
The number lock of a suitcase has 4 wheels each labelled with ten digits i.e. from 0 to 9.The lock opens with a sequence of four digits with no repeats.What is the probability of a person getting the right sequence to open the suitcase.
\\
\solution
		%\input{ncert/11/16/4/10/defs.tex}
		%
\item 
Two cards are drawn at random and without replacement from a pack of 52 playing cards. Find the probability that both the cards are black.
\\
\solution
		%\input{ncert/12/13/2/2/defs.tex}
		\item A box of oranges is inspected by examining three randomly selected oranges drawn without replacement. If all the three oranges are good, the box is approved for sale, otherwise, it is rejected. Find the probability that a box containing 15 oranges out of which 12 are good and 3 are bad ones will be approved for sale.
		\label{ncert/12/13/2/3/defs.tex}
		\item Two balls are drawn at random with replacement from a box containing 10 black and 8 red balls. Find the probability that
		\label{ncert/12/13/2/12}
\begin{enumerate}
\item both balls are red.
\item first ball is black and second is red.
\item one of them is black and other is red.
\end{enumerate}

\item In a hostel, 60\% of the students read Hindi newspaper, 40\% read English newspaper and 20\% read both Hindi and English newspapers. A student is selected at random.
		\label{ncert/12/13/2/15}
\begin{enumerate}
\item Find the probability that she reads neither Hindi nor English newspapers.
\item If she reads Hindi newspaper, find the probability that she reads English newspaper.
\item If she reads English newspaper, find the probability that she reads Hindi newspaper.\\
\end{enumerate}
\item The probability of obtaining an even prime number on each die, when a pair of dice is rolled is 
\begin{enumerate}
    \item $0$ 
    
    \item $\frac{1}{3}$ 
    
    \item $\frac{1}{12}$ 
    
    \item $\frac{1}{36}$ 
\end{enumerate}
\solution
		%\input{ncert/12/13/2/17/defs.tex}
	\item A bag contains 4 red and 4 black balls, another bag contains 2 red and 6 black balls. One of the two bags is selected at random and a ball is drawn from the bag which is found to be red. Find the probability that the ball is drawn from the first bag.
\\
\solution
		%\input{ncert/12/13/3/2/main.tex}
  \item
  Cards with numbers 2 to 101 are placed in a box. A card is selected at random.Find the probability that the card has
\begin{enumerate}[label=(\roman*)]
	\item an even number 
	\item a square number
\end{enumerate}
\solution
%\input{exemplar/10/13/3/32/main.tex}
\item
The king, queen and jack of clubs are removed from a deck of 52 playing cards and then well shuffled. Now one card is drawn at random from the remaining cards.  Determine the probability that the card is
\begin{enumerate}[label=(\roman*)]
\item a club
\item 10 of hearts
\end{enumerate}
\solution
%\input{exemplar/10/13/3/29/main.tex}
\item A team of medical students doing their internship have to assist during surgeries
at a city hospital. The probabilities of surgeries rated as very complex, complex,
routine, simple or very simple are respectively, 0.15, 0.20, 0.31, 0.26, .08. Find
the probabilities that a particular surgery will be rated
\begin{enumerate}
	\item complex or very complex;
	\item neither very complex nor very simple;
	\item routine or complex
	\item routine or simple
\end{enumerate}
\solution
%\input{exemplar/11/16/3/8(1)/main.tex}
\item A card is selected from a pack of 52 cards.
\begin{enumerate}[label=(\alph*)]
    \item How many points are there in the sample space?
    \item Calculate the probability that the card is an ace of spades.
    \item Calculate the probability that the card is (i) an ace and (ii) black card.
\end{enumerate}
\solution
%\input{exemplar/11/16/3/4/main2.tex}
\item The probability that a non leap year selected at random will contain 53 sundays.
\\
\solution
%\input{exemplar/10/13/1/19/main.tex}
\item One of the four persons John, Rita, Aslam or Gurpreet will be promoted next
month. Consequently the sample space consists of four elementary outcomes
S = {John promoted, Rita promoted, Aslam promoted, Gurpreet promoted}
You are told that the chances of John’s promotion is same as that of Gurpreet,
Rita’s chances of promotion are twice as likely as Johns. Aslam’s chances are
four times that of John.
\begin{enumerate}
	\item Determine
	\begin{enumerate}
		\item P (John promoted)
		\item P (Rita promoted)
		\item P (Aslam promoted)
		\item P (Gurpreet promoted)
	\end{enumerate}
	\item If A = {John promoted or Gurpreet promoted}, find P (A).
\end{enumerate}
\solution
%\input{exemplar/11/16/3/10/main.tex}
\item A card is drawn from a deck of 52 cards. Find the probability of getting a king or a heart or a red card.\\
\solution
%\input{exemplar/11/16/3/15/main.tex}
\item The probability that a student will pass his examination is 0.73, the probability of
the student getting a compartment is 0.13, and the probability that the student will
either pass or get compartment is 0.96. State True or False.\\
\solution
%\input{exemplar/11/16/3/31/main.tex}
\item A card is selected from a pack of 52 cards\\
\begin{enumerate}[label=(\alph*)]
\item How many points are there in the sample space?
\item Calculate the probability that the cards is an ace of spades.
\item Calculate the probability that the card is (i) an ace (ii)black card.\\
\end{enumerate}
%\input{ncert/11/16/3/4_1/Prob_4.tex}
\item In a non-leap year, the probability of having 53 tuesdays or 53 wednesdays is\\
\solution
%\input{exemplar/11/16/3/18/main.tex}
\item There are 1000 sealed envelopes in a box, 10 of them contain a cash prize of
Rs 100 each, 100 of them contain a cash prize of Rs 50 each and 200 of them
contain a cash prize of Rs 10 each and rest do not contain any cash prize. If they
are well shuffled and an envelope is picked up out, what is the probability that it
contains no cash prize?\\
\solution
%\input{exemplar/10/13/3/34/main.tex}
\item 
A die is thrown and a card is selected at random from a deck of 52 playing cards. The probability of getting an even number on the die and a spade card.\\
\solution
%\input{exemplar/12/13/3/78/main.tex}
\item
If 4-digit numbers greater than 5,000 are randomly formed from the digits 0, 1, 3, 5, and 7, what is the probability of forming a number divisible by 5 when:
\begin{enumerate}
    \item The digits are repeated?
    \item The repetition of digits is not allowed?
\end{enumerate}
\solution
%\input{ncert/11/16/4/9/main.tex}
\item Consider the probability space $\brak{\Omega, \mathcal{G}, P}$ where $\Omega = [0,2]$ and $\mathcal{G} = \cbrak{\phi, \Omega, [0,1], (1,2]}$. Let $X$ and $Y$ be two functions on $\Omega$ defined as
\begin{align*}
    X(\omega) = 
    \begin{cases}
        1 & \text{if }\omega \in [0, 1]\\
        2 & \text{if }\omega \in (1, 2]
    \end{cases}
\end{align*}
and
\begin{align*}
    Y(\omega) = 
    \begin{cases}
        2 & \text{if }\omega \in [0, 1.5]\\
        3 & \text{if }\omega \in (1.5, 2].
    \end{cases}
\end{align*}
Then which one of the following statements is true?
\begin{enumerate}
    \item [(A)] $X$ is a random variable with respect to $\mathcal{G}$, but $Y$ is not a random variable with respect to $\mathcal{G}$.
    \item [(B)] $Y$ is a random variable with respect to $\mathcal{G}$, but $X$ is not a random variable with respect to $\mathcal{G}$.
    \item [(C)] Neither $X$ nor $Y$ is a random variable with respect to $\mathcal{G}$.
    \item [(D)] Both $X$ and $Y$ are random variables with respect to $\mathcal{G}$.
\end{enumerate} \hfill (GATE ST 2023)\\
\solution
%\input{gate/ST/2023/14/main.tex}
	\item  A die is loaded in such a way that each odd number is twice as likely to occur as
each even number. Find $P(G)$, where $G$ is the event that a number greater than
3 occurs on a single roll of the die.
\\
\solution
		%\input{exemplar/11/16/3/5/main.tex}
	\item All the jacks, queens and kings are removed from a deck of 52 playing cards. The remaining cards are well shuffled and then one card is drawn at random. Giving ace a value 1 similar value for other cards, find the probability that the card has a value 
		\begin{enumerate}
			\item 7
			\item greater than 7
			\item less than 7
		\end{enumerate}
		%\input{exemplar/10/13/3/30/main.tex}
  \item A Lot consists of 48 mobile phones of which 42 are good, 3 have only minor defects and 3 have major defects.Varnika will buy a phone if it is good but the trader will only buy a mobile if it has no major defects. One phone is selected at random from the lot. What is the probability that it is
\begin{enumerate}
	\item acceptable to Varnika?
            \item acceptable to the trader?
\end{enumerate}
\solution
	%\input{exemplar/10/13/3/40/main.tex}
 \item A student says that if you throw a die, it will show up 1 or not 1. Therefore, the probability of getting 1 and the probability of getting 'not 1' each is equal to $\frac{1}{2}$. Is this correct? Give reasons.\\
 \solution
        %\input{exemplar/10/13/2/9/main.tex}
   \item Four candidates A, B, C, D have ap-
plied for the assignment to coach a school cricket
team. If A is twice as likely to be selected as B, and
B and C are given about the same chance of being
selected, while C is twice as likely to be selected
as D, what are the probabilities that
\begin{enumerate}
\item C will be selected?
\item A will not be selected?
\end{enumerate}
	%\input{exemplar/11/16/3/9/main.tex}
 \item A bag contain 24 balls of which $x$ balls are red, $2x$ are white and $3x$ are blue. A ball is selected at random, What is the probability that it is
\begin{enumerate}[label=\alph*)]
\item not red ?
\item white ?
\end{enumerate}
%\input{exemplar/10/13/3/41/main.tex}
If the letters of the word ASSASSINATION are arranged at random. Find the Probability that
\begin{enumerate}[label=(\alph*)]
\item Four $S's$ come consecutively in the word
\item Two  $I's$ and two $N's$ come together
\item All $A's$ are not coming together
\item No two $A's$ are coming together
\end{enumerate}
%\input{exemplar/11/16/3/14/main.tex}
	\item One urn contains two black balls (labelled B1 and B2) and one white ball. A
	second urn contains one black ball and two white balls (labelled W1 and W2).
	Suppose the following experiment is performed. One of the two urns is chosen
	at random. Next a ball is randomly chosen from the urn. Then a second ball is
	chosen at random from the same urn without replacing the first ball.
	
	\begin{enumerate}
	\item What is the probability that two black balls are chosen?
	
	\item What is the probability that two balls of opposite colour are chosen?
	\end{enumerate}
	\solution
	%\input{exemplar/11/16/3/12/main1.tex}
\end{enumerate}

\item In a certain lottery 10,000 tickets are sold and ten equal prizes are awarded. What is the probability of not getting a prize if you buy (a) one ticket (b) two tickets (c) 10 tickets ?	
\\
\solution
		%\begin{enumerate}[label=\thesection.\arabic*,ref=\thesection.\theenumi]
	\item One card is drawn from a well-shuffled deck of 52 cards. Find the probability of getting
\begin{enumerate}
\item A king of red colour 
\item A face card 
\item A red face card
\item The jack of hearts
\item A spade
\item The queen of diamonds

\end{enumerate}
\solution
		%\input{ncert/10/15/1/14/main.tex}
	\item Five cards—the ten, jack, queen, king and ace of diamonds, are well-shuffled with their face downwards. One card is then picked up at random.
\begin{enumerate}
\item
What is the probability that the card is the queen? 
\item
If the queen is drawn and put aside, what is the probability that the second card picked up is (a) an ace? (b) a queen?\\
\end{enumerate}
\solution
		%\input{ncert/10/15/1/15/defs.tex}
	\item A bag contains $5$ red balls and some blue balls. If the probability of drawing a blue ball is double that if a red ball, determine the number of blue balls in the bag. 
		\\
\solution
		%\input{ncert/10/15/2/3/defs.tex}
	\item A card is selected from a pack of 52 cards.
 \begin{enumerate}[label=(\alph*)] 
                 \item How many points are there in the sample space?
                 \item Calculate the probability that the card is an ace of spades.
                 \item Calculate the probability that the card is (i) an ace and (ii) black card.
 \end{enumerate}
\solution
		%\input{ncert/11/16/3/4/main.tex}
\item Four cards are drawn from a well-shuffled deck of 52 cards. What is the probability of obtaining 3 diamonds and one spade.
\\
\solution
		%\input{ncert/11/16/4/2/defs.tex}
\item In a certain lottery 10,000 tickets are sold and ten equal prizes are awarded. What is the probability of not getting a prize if you buy (a) one ticket (b) two tickets (c) 10 tickets ?	
\\
\solution
		%\input{ncert/11/16/4/4/defs.tex}
		%
\item 
Out of 100 students, two sections of 40 and 60 are formed. If you and your friend are among the 100 students, what is the probability that
\begin{enumerate}
\item you both enter the same section?
\item you both enter the different sections?
\end{enumerate}
\solution
		%\input{ncert/11/16/4/5/defs.tex}
	\item 
The number lock of a suitcase has 4 wheels each labelled with ten digits i.e. from 0 to 9.The lock opens with a sequence of four digits with no repeats.What is the probability of a person getting the right sequence to open the suitcase.
\\
\solution
		%\input{ncert/11/16/4/10/defs.tex}
		%
\item 
Two cards are drawn at random and without replacement from a pack of 52 playing cards. Find the probability that both the cards are black.
\\
\solution
		%\input{ncert/12/13/2/2/defs.tex}
		\item A box of oranges is inspected by examining three randomly selected oranges drawn without replacement. If all the three oranges are good, the box is approved for sale, otherwise, it is rejected. Find the probability that a box containing 15 oranges out of which 12 are good and 3 are bad ones will be approved for sale.
		\label{ncert/12/13/2/3/defs.tex}
		\item Two balls are drawn at random with replacement from a box containing 10 black and 8 red balls. Find the probability that
		\label{ncert/12/13/2/12}
\begin{enumerate}
\item both balls are red.
\item first ball is black and second is red.
\item one of them is black and other is red.
\end{enumerate}

\item In a hostel, 60\% of the students read Hindi newspaper, 40\% read English newspaper and 20\% read both Hindi and English newspapers. A student is selected at random.
		\label{ncert/12/13/2/15}
\begin{enumerate}
\item Find the probability that she reads neither Hindi nor English newspapers.
\item If she reads Hindi newspaper, find the probability that she reads English newspaper.
\item If she reads English newspaper, find the probability that she reads Hindi newspaper.\\
\end{enumerate}
\item The probability of obtaining an even prime number on each die, when a pair of dice is rolled is 
\begin{enumerate}
    \item $0$ 
    
    \item $\frac{1}{3}$ 
    
    \item $\frac{1}{12}$ 
    
    \item $\frac{1}{36}$ 
\end{enumerate}
\solution
		%\input{ncert/12/13/2/17/defs.tex}
	\item A bag contains 4 red and 4 black balls, another bag contains 2 red and 6 black balls. One of the two bags is selected at random and a ball is drawn from the bag which is found to be red. Find the probability that the ball is drawn from the first bag.
\\
\solution
		%\input{ncert/12/13/3/2/main.tex}
  \item
  Cards with numbers 2 to 101 are placed in a box. A card is selected at random.Find the probability that the card has
\begin{enumerate}[label=(\roman*)]
	\item an even number 
	\item a square number
\end{enumerate}
\solution
%\input{exemplar/10/13/3/32/main.tex}
\item
The king, queen and jack of clubs are removed from a deck of 52 playing cards and then well shuffled. Now one card is drawn at random from the remaining cards.  Determine the probability that the card is
\begin{enumerate}[label=(\roman*)]
\item a club
\item 10 of hearts
\end{enumerate}
\solution
%\input{exemplar/10/13/3/29/main.tex}
\item A team of medical students doing their internship have to assist during surgeries
at a city hospital. The probabilities of surgeries rated as very complex, complex,
routine, simple or very simple are respectively, 0.15, 0.20, 0.31, 0.26, .08. Find
the probabilities that a particular surgery will be rated
\begin{enumerate}
	\item complex or very complex;
	\item neither very complex nor very simple;
	\item routine or complex
	\item routine or simple
\end{enumerate}
\solution
%\input{exemplar/11/16/3/8(1)/main.tex}
\item A card is selected from a pack of 52 cards.
\begin{enumerate}[label=(\alph*)]
    \item How many points are there in the sample space?
    \item Calculate the probability that the card is an ace of spades.
    \item Calculate the probability that the card is (i) an ace and (ii) black card.
\end{enumerate}
\solution
%\input{exemplar/11/16/3/4/main2.tex}
\item The probability that a non leap year selected at random will contain 53 sundays.
\\
\solution
%\input{exemplar/10/13/1/19/main.tex}
\item One of the four persons John, Rita, Aslam or Gurpreet will be promoted next
month. Consequently the sample space consists of four elementary outcomes
S = {John promoted, Rita promoted, Aslam promoted, Gurpreet promoted}
You are told that the chances of John’s promotion is same as that of Gurpreet,
Rita’s chances of promotion are twice as likely as Johns. Aslam’s chances are
four times that of John.
\begin{enumerate}
	\item Determine
	\begin{enumerate}
		\item P (John promoted)
		\item P (Rita promoted)
		\item P (Aslam promoted)
		\item P (Gurpreet promoted)
	\end{enumerate}
	\item If A = {John promoted or Gurpreet promoted}, find P (A).
\end{enumerate}
\solution
%\input{exemplar/11/16/3/10/main.tex}
\item A card is drawn from a deck of 52 cards. Find the probability of getting a king or a heart or a red card.\\
\solution
%\input{exemplar/11/16/3/15/main.tex}
\item The probability that a student will pass his examination is 0.73, the probability of
the student getting a compartment is 0.13, and the probability that the student will
either pass or get compartment is 0.96. State True or False.\\
\solution
%\input{exemplar/11/16/3/31/main.tex}
\item A card is selected from a pack of 52 cards\\
\begin{enumerate}[label=(\alph*)]
\item How many points are there in the sample space?
\item Calculate the probability that the cards is an ace of spades.
\item Calculate the probability that the card is (i) an ace (ii)black card.\\
\end{enumerate}
%\input{ncert/11/16/3/4_1/Prob_4.tex}
\item In a non-leap year, the probability of having 53 tuesdays or 53 wednesdays is\\
\solution
%\input{exemplar/11/16/3/18/main.tex}
\item There are 1000 sealed envelopes in a box, 10 of them contain a cash prize of
Rs 100 each, 100 of them contain a cash prize of Rs 50 each and 200 of them
contain a cash prize of Rs 10 each and rest do not contain any cash prize. If they
are well shuffled and an envelope is picked up out, what is the probability that it
contains no cash prize?\\
\solution
%\input{exemplar/10/13/3/34/main.tex}
\item 
A die is thrown and a card is selected at random from a deck of 52 playing cards. The probability of getting an even number on the die and a spade card.\\
\solution
%\input{exemplar/12/13/3/78/main.tex}
\item
If 4-digit numbers greater than 5,000 are randomly formed from the digits 0, 1, 3, 5, and 7, what is the probability of forming a number divisible by 5 when:
\begin{enumerate}
    \item The digits are repeated?
    \item The repetition of digits is not allowed?
\end{enumerate}
\solution
%\input{ncert/11/16/4/9/main.tex}
\item Consider the probability space $\brak{\Omega, \mathcal{G}, P}$ where $\Omega = [0,2]$ and $\mathcal{G} = \cbrak{\phi, \Omega, [0,1], (1,2]}$. Let $X$ and $Y$ be two functions on $\Omega$ defined as
\begin{align*}
    X(\omega) = 
    \begin{cases}
        1 & \text{if }\omega \in [0, 1]\\
        2 & \text{if }\omega \in (1, 2]
    \end{cases}
\end{align*}
and
\begin{align*}
    Y(\omega) = 
    \begin{cases}
        2 & \text{if }\omega \in [0, 1.5]\\
        3 & \text{if }\omega \in (1.5, 2].
    \end{cases}
\end{align*}
Then which one of the following statements is true?
\begin{enumerate}
    \item [(A)] $X$ is a random variable with respect to $\mathcal{G}$, but $Y$ is not a random variable with respect to $\mathcal{G}$.
    \item [(B)] $Y$ is a random variable with respect to $\mathcal{G}$, but $X$ is not a random variable with respect to $\mathcal{G}$.
    \item [(C)] Neither $X$ nor $Y$ is a random variable with respect to $\mathcal{G}$.
    \item [(D)] Both $X$ and $Y$ are random variables with respect to $\mathcal{G}$.
\end{enumerate} \hfill (GATE ST 2023)\\
\solution
%\input{gate/ST/2023/14/main.tex}
	\item  A die is loaded in such a way that each odd number is twice as likely to occur as
each even number. Find $P(G)$, where $G$ is the event that a number greater than
3 occurs on a single roll of the die.
\\
\solution
		%\input{exemplar/11/16/3/5/main.tex}
	\item All the jacks, queens and kings are removed from a deck of 52 playing cards. The remaining cards are well shuffled and then one card is drawn at random. Giving ace a value 1 similar value for other cards, find the probability that the card has a value 
		\begin{enumerate}
			\item 7
			\item greater than 7
			\item less than 7
		\end{enumerate}
		%\input{exemplar/10/13/3/30/main.tex}
  \item A Lot consists of 48 mobile phones of which 42 are good, 3 have only minor defects and 3 have major defects.Varnika will buy a phone if it is good but the trader will only buy a mobile if it has no major defects. One phone is selected at random from the lot. What is the probability that it is
\begin{enumerate}
	\item acceptable to Varnika?
            \item acceptable to the trader?
\end{enumerate}
\solution
	%\input{exemplar/10/13/3/40/main.tex}
 \item A student says that if you throw a die, it will show up 1 or not 1. Therefore, the probability of getting 1 and the probability of getting 'not 1' each is equal to $\frac{1}{2}$. Is this correct? Give reasons.\\
 \solution
        %\input{exemplar/10/13/2/9/main.tex}
   \item Four candidates A, B, C, D have ap-
plied for the assignment to coach a school cricket
team. If A is twice as likely to be selected as B, and
B and C are given about the same chance of being
selected, while C is twice as likely to be selected
as D, what are the probabilities that
\begin{enumerate}
\item C will be selected?
\item A will not be selected?
\end{enumerate}
	%\input{exemplar/11/16/3/9/main.tex}
 \item A bag contain 24 balls of which $x$ balls are red, $2x$ are white and $3x$ are blue. A ball is selected at random, What is the probability that it is
\begin{enumerate}[label=\alph*)]
\item not red ?
\item white ?
\end{enumerate}
%\input{exemplar/10/13/3/41/main.tex}
If the letters of the word ASSASSINATION are arranged at random. Find the Probability that
\begin{enumerate}[label=(\alph*)]
\item Four $S's$ come consecutively in the word
\item Two  $I's$ and two $N's$ come together
\item All $A's$ are not coming together
\item No two $A's$ are coming together
\end{enumerate}
%\input{exemplar/11/16/3/14/main.tex}
	\item One urn contains two black balls (labelled B1 and B2) and one white ball. A
	second urn contains one black ball and two white balls (labelled W1 and W2).
	Suppose the following experiment is performed. One of the two urns is chosen
	at random. Next a ball is randomly chosen from the urn. Then a second ball is
	chosen at random from the same urn without replacing the first ball.
	
	\begin{enumerate}
	\item What is the probability that two black balls are chosen?
	
	\item What is the probability that two balls of opposite colour are chosen?
	\end{enumerate}
	\solution
	%\input{exemplar/11/16/3/12/main1.tex}
\end{enumerate}

		%
\item 
Out of 100 students, two sections of 40 and 60 are formed. If you and your friend are among the 100 students, what is the probability that
\begin{enumerate}
\item you both enter the same section?
\item you both enter the different sections?
\end{enumerate}
\solution
		%\begin{enumerate}[label=\thesection.\arabic*,ref=\thesection.\theenumi]
	\item One card is drawn from a well-shuffled deck of 52 cards. Find the probability of getting
\begin{enumerate}
\item A king of red colour 
\item A face card 
\item A red face card
\item The jack of hearts
\item A spade
\item The queen of diamonds

\end{enumerate}
\solution
		%\input{ncert/10/15/1/14/main.tex}
	\item Five cards—the ten, jack, queen, king and ace of diamonds, are well-shuffled with their face downwards. One card is then picked up at random.
\begin{enumerate}
\item
What is the probability that the card is the queen? 
\item
If the queen is drawn and put aside, what is the probability that the second card picked up is (a) an ace? (b) a queen?\\
\end{enumerate}
\solution
		%\input{ncert/10/15/1/15/defs.tex}
	\item A bag contains $5$ red balls and some blue balls. If the probability of drawing a blue ball is double that if a red ball, determine the number of blue balls in the bag. 
		\\
\solution
		%\input{ncert/10/15/2/3/defs.tex}
	\item A card is selected from a pack of 52 cards.
 \begin{enumerate}[label=(\alph*)] 
                 \item How many points are there in the sample space?
                 \item Calculate the probability that the card is an ace of spades.
                 \item Calculate the probability that the card is (i) an ace and (ii) black card.
 \end{enumerate}
\solution
		%\input{ncert/11/16/3/4/main.tex}
\item Four cards are drawn from a well-shuffled deck of 52 cards. What is the probability of obtaining 3 diamonds and one spade.
\\
\solution
		%\input{ncert/11/16/4/2/defs.tex}
\item In a certain lottery 10,000 tickets are sold and ten equal prizes are awarded. What is the probability of not getting a prize if you buy (a) one ticket (b) two tickets (c) 10 tickets ?	
\\
\solution
		%\input{ncert/11/16/4/4/defs.tex}
		%
\item 
Out of 100 students, two sections of 40 and 60 are formed. If you and your friend are among the 100 students, what is the probability that
\begin{enumerate}
\item you both enter the same section?
\item you both enter the different sections?
\end{enumerate}
\solution
		%\input{ncert/11/16/4/5/defs.tex}
	\item 
The number lock of a suitcase has 4 wheels each labelled with ten digits i.e. from 0 to 9.The lock opens with a sequence of four digits with no repeats.What is the probability of a person getting the right sequence to open the suitcase.
\\
\solution
		%\input{ncert/11/16/4/10/defs.tex}
		%
\item 
Two cards are drawn at random and without replacement from a pack of 52 playing cards. Find the probability that both the cards are black.
\\
\solution
		%\input{ncert/12/13/2/2/defs.tex}
		\item A box of oranges is inspected by examining three randomly selected oranges drawn without replacement. If all the three oranges are good, the box is approved for sale, otherwise, it is rejected. Find the probability that a box containing 15 oranges out of which 12 are good and 3 are bad ones will be approved for sale.
		\label{ncert/12/13/2/3/defs.tex}
		\item Two balls are drawn at random with replacement from a box containing 10 black and 8 red balls. Find the probability that
		\label{ncert/12/13/2/12}
\begin{enumerate}
\item both balls are red.
\item first ball is black and second is red.
\item one of them is black and other is red.
\end{enumerate}

\item In a hostel, 60\% of the students read Hindi newspaper, 40\% read English newspaper and 20\% read both Hindi and English newspapers. A student is selected at random.
		\label{ncert/12/13/2/15}
\begin{enumerate}
\item Find the probability that she reads neither Hindi nor English newspapers.
\item If she reads Hindi newspaper, find the probability that she reads English newspaper.
\item If she reads English newspaper, find the probability that she reads Hindi newspaper.\\
\end{enumerate}
\item The probability of obtaining an even prime number on each die, when a pair of dice is rolled is 
\begin{enumerate}
    \item $0$ 
    
    \item $\frac{1}{3}$ 
    
    \item $\frac{1}{12}$ 
    
    \item $\frac{1}{36}$ 
\end{enumerate}
\solution
		%\input{ncert/12/13/2/17/defs.tex}
	\item A bag contains 4 red and 4 black balls, another bag contains 2 red and 6 black balls. One of the two bags is selected at random and a ball is drawn from the bag which is found to be red. Find the probability that the ball is drawn from the first bag.
\\
\solution
		%\input{ncert/12/13/3/2/main.tex}
  \item
  Cards with numbers 2 to 101 are placed in a box. A card is selected at random.Find the probability that the card has
\begin{enumerate}[label=(\roman*)]
	\item an even number 
	\item a square number
\end{enumerate}
\solution
%\input{exemplar/10/13/3/32/main.tex}
\item
The king, queen and jack of clubs are removed from a deck of 52 playing cards and then well shuffled. Now one card is drawn at random from the remaining cards.  Determine the probability that the card is
\begin{enumerate}[label=(\roman*)]
\item a club
\item 10 of hearts
\end{enumerate}
\solution
%\input{exemplar/10/13/3/29/main.tex}
\item A team of medical students doing their internship have to assist during surgeries
at a city hospital. The probabilities of surgeries rated as very complex, complex,
routine, simple or very simple are respectively, 0.15, 0.20, 0.31, 0.26, .08. Find
the probabilities that a particular surgery will be rated
\begin{enumerate}
	\item complex or very complex;
	\item neither very complex nor very simple;
	\item routine or complex
	\item routine or simple
\end{enumerate}
\solution
%\input{exemplar/11/16/3/8(1)/main.tex}
\item A card is selected from a pack of 52 cards.
\begin{enumerate}[label=(\alph*)]
    \item How many points are there in the sample space?
    \item Calculate the probability that the card is an ace of spades.
    \item Calculate the probability that the card is (i) an ace and (ii) black card.
\end{enumerate}
\solution
%\input{exemplar/11/16/3/4/main2.tex}
\item The probability that a non leap year selected at random will contain 53 sundays.
\\
\solution
%\input{exemplar/10/13/1/19/main.tex}
\item One of the four persons John, Rita, Aslam or Gurpreet will be promoted next
month. Consequently the sample space consists of four elementary outcomes
S = {John promoted, Rita promoted, Aslam promoted, Gurpreet promoted}
You are told that the chances of John’s promotion is same as that of Gurpreet,
Rita’s chances of promotion are twice as likely as Johns. Aslam’s chances are
four times that of John.
\begin{enumerate}
	\item Determine
	\begin{enumerate}
		\item P (John promoted)
		\item P (Rita promoted)
		\item P (Aslam promoted)
		\item P (Gurpreet promoted)
	\end{enumerate}
	\item If A = {John promoted or Gurpreet promoted}, find P (A).
\end{enumerate}
\solution
%\input{exemplar/11/16/3/10/main.tex}
\item A card is drawn from a deck of 52 cards. Find the probability of getting a king or a heart or a red card.\\
\solution
%\input{exemplar/11/16/3/15/main.tex}
\item The probability that a student will pass his examination is 0.73, the probability of
the student getting a compartment is 0.13, and the probability that the student will
either pass or get compartment is 0.96. State True or False.\\
\solution
%\input{exemplar/11/16/3/31/main.tex}
\item A card is selected from a pack of 52 cards\\
\begin{enumerate}[label=(\alph*)]
\item How many points are there in the sample space?
\item Calculate the probability that the cards is an ace of spades.
\item Calculate the probability that the card is (i) an ace (ii)black card.\\
\end{enumerate}
%\input{ncert/11/16/3/4_1/Prob_4.tex}
\item In a non-leap year, the probability of having 53 tuesdays or 53 wednesdays is\\
\solution
%\input{exemplar/11/16/3/18/main.tex}
\item There are 1000 sealed envelopes in a box, 10 of them contain a cash prize of
Rs 100 each, 100 of them contain a cash prize of Rs 50 each and 200 of them
contain a cash prize of Rs 10 each and rest do not contain any cash prize. If they
are well shuffled and an envelope is picked up out, what is the probability that it
contains no cash prize?\\
\solution
%\input{exemplar/10/13/3/34/main.tex}
\item 
A die is thrown and a card is selected at random from a deck of 52 playing cards. The probability of getting an even number on the die and a spade card.\\
\solution
%\input{exemplar/12/13/3/78/main.tex}
\item
If 4-digit numbers greater than 5,000 are randomly formed from the digits 0, 1, 3, 5, and 7, what is the probability of forming a number divisible by 5 when:
\begin{enumerate}
    \item The digits are repeated?
    \item The repetition of digits is not allowed?
\end{enumerate}
\solution
%\input{ncert/11/16/4/9/main.tex}
\item Consider the probability space $\brak{\Omega, \mathcal{G}, P}$ where $\Omega = [0,2]$ and $\mathcal{G} = \cbrak{\phi, \Omega, [0,1], (1,2]}$. Let $X$ and $Y$ be two functions on $\Omega$ defined as
\begin{align*}
    X(\omega) = 
    \begin{cases}
        1 & \text{if }\omega \in [0, 1]\\
        2 & \text{if }\omega \in (1, 2]
    \end{cases}
\end{align*}
and
\begin{align*}
    Y(\omega) = 
    \begin{cases}
        2 & \text{if }\omega \in [0, 1.5]\\
        3 & \text{if }\omega \in (1.5, 2].
    \end{cases}
\end{align*}
Then which one of the following statements is true?
\begin{enumerate}
    \item [(A)] $X$ is a random variable with respect to $\mathcal{G}$, but $Y$ is not a random variable with respect to $\mathcal{G}$.
    \item [(B)] $Y$ is a random variable with respect to $\mathcal{G}$, but $X$ is not a random variable with respect to $\mathcal{G}$.
    \item [(C)] Neither $X$ nor $Y$ is a random variable with respect to $\mathcal{G}$.
    \item [(D)] Both $X$ and $Y$ are random variables with respect to $\mathcal{G}$.
\end{enumerate} \hfill (GATE ST 2023)\\
\solution
%\input{gate/ST/2023/14/main.tex}
	\item  A die is loaded in such a way that each odd number is twice as likely to occur as
each even number. Find $P(G)$, where $G$ is the event that a number greater than
3 occurs on a single roll of the die.
\\
\solution
		%\input{exemplar/11/16/3/5/main.tex}
	\item All the jacks, queens and kings are removed from a deck of 52 playing cards. The remaining cards are well shuffled and then one card is drawn at random. Giving ace a value 1 similar value for other cards, find the probability that the card has a value 
		\begin{enumerate}
			\item 7
			\item greater than 7
			\item less than 7
		\end{enumerate}
		%\input{exemplar/10/13/3/30/main.tex}
  \item A Lot consists of 48 mobile phones of which 42 are good, 3 have only minor defects and 3 have major defects.Varnika will buy a phone if it is good but the trader will only buy a mobile if it has no major defects. One phone is selected at random from the lot. What is the probability that it is
\begin{enumerate}
	\item acceptable to Varnika?
            \item acceptable to the trader?
\end{enumerate}
\solution
	%\input{exemplar/10/13/3/40/main.tex}
 \item A student says that if you throw a die, it will show up 1 or not 1. Therefore, the probability of getting 1 and the probability of getting 'not 1' each is equal to $\frac{1}{2}$. Is this correct? Give reasons.\\
 \solution
        %\input{exemplar/10/13/2/9/main.tex}
   \item Four candidates A, B, C, D have ap-
plied for the assignment to coach a school cricket
team. If A is twice as likely to be selected as B, and
B and C are given about the same chance of being
selected, while C is twice as likely to be selected
as D, what are the probabilities that
\begin{enumerate}
\item C will be selected?
\item A will not be selected?
\end{enumerate}
	%\input{exemplar/11/16/3/9/main.tex}
 \item A bag contain 24 balls of which $x$ balls are red, $2x$ are white and $3x$ are blue. A ball is selected at random, What is the probability that it is
\begin{enumerate}[label=\alph*)]
\item not red ?
\item white ?
\end{enumerate}
%\input{exemplar/10/13/3/41/main.tex}
If the letters of the word ASSASSINATION are arranged at random. Find the Probability that
\begin{enumerate}[label=(\alph*)]
\item Four $S's$ come consecutively in the word
\item Two  $I's$ and two $N's$ come together
\item All $A's$ are not coming together
\item No two $A's$ are coming together
\end{enumerate}
%\input{exemplar/11/16/3/14/main.tex}
	\item One urn contains two black balls (labelled B1 and B2) and one white ball. A
	second urn contains one black ball and two white balls (labelled W1 and W2).
	Suppose the following experiment is performed. One of the two urns is chosen
	at random. Next a ball is randomly chosen from the urn. Then a second ball is
	chosen at random from the same urn without replacing the first ball.
	
	\begin{enumerate}
	\item What is the probability that two black balls are chosen?
	
	\item What is the probability that two balls of opposite colour are chosen?
	\end{enumerate}
	\solution
	%\input{exemplar/11/16/3/12/main1.tex}
\end{enumerate}

	\item 
The number lock of a suitcase has 4 wheels each labelled with ten digits i.e. from 0 to 9.The lock opens with a sequence of four digits with no repeats.What is the probability of a person getting the right sequence to open the suitcase.
\\
\solution
		%\begin{enumerate}[label=\thesection.\arabic*,ref=\thesection.\theenumi]
	\item One card is drawn from a well-shuffled deck of 52 cards. Find the probability of getting
\begin{enumerate}
\item A king of red colour 
\item A face card 
\item A red face card
\item The jack of hearts
\item A spade
\item The queen of diamonds

\end{enumerate}
\solution
		%\input{ncert/10/15/1/14/main.tex}
	\item Five cards—the ten, jack, queen, king and ace of diamonds, are well-shuffled with their face downwards. One card is then picked up at random.
\begin{enumerate}
\item
What is the probability that the card is the queen? 
\item
If the queen is drawn and put aside, what is the probability that the second card picked up is (a) an ace? (b) a queen?\\
\end{enumerate}
\solution
		%\input{ncert/10/15/1/15/defs.tex}
	\item A bag contains $5$ red balls and some blue balls. If the probability of drawing a blue ball is double that if a red ball, determine the number of blue balls in the bag. 
		\\
\solution
		%\input{ncert/10/15/2/3/defs.tex}
	\item A card is selected from a pack of 52 cards.
 \begin{enumerate}[label=(\alph*)] 
                 \item How many points are there in the sample space?
                 \item Calculate the probability that the card is an ace of spades.
                 \item Calculate the probability that the card is (i) an ace and (ii) black card.
 \end{enumerate}
\solution
		%\input{ncert/11/16/3/4/main.tex}
\item Four cards are drawn from a well-shuffled deck of 52 cards. What is the probability of obtaining 3 diamonds and one spade.
\\
\solution
		%\input{ncert/11/16/4/2/defs.tex}
\item In a certain lottery 10,000 tickets are sold and ten equal prizes are awarded. What is the probability of not getting a prize if you buy (a) one ticket (b) two tickets (c) 10 tickets ?	
\\
\solution
		%\input{ncert/11/16/4/4/defs.tex}
		%
\item 
Out of 100 students, two sections of 40 and 60 are formed. If you and your friend are among the 100 students, what is the probability that
\begin{enumerate}
\item you both enter the same section?
\item you both enter the different sections?
\end{enumerate}
\solution
		%\input{ncert/11/16/4/5/defs.tex}
	\item 
The number lock of a suitcase has 4 wheels each labelled with ten digits i.e. from 0 to 9.The lock opens with a sequence of four digits with no repeats.What is the probability of a person getting the right sequence to open the suitcase.
\\
\solution
		%\input{ncert/11/16/4/10/defs.tex}
		%
\item 
Two cards are drawn at random and without replacement from a pack of 52 playing cards. Find the probability that both the cards are black.
\\
\solution
		%\input{ncert/12/13/2/2/defs.tex}
		\item A box of oranges is inspected by examining three randomly selected oranges drawn without replacement. If all the three oranges are good, the box is approved for sale, otherwise, it is rejected. Find the probability that a box containing 15 oranges out of which 12 are good and 3 are bad ones will be approved for sale.
		\label{ncert/12/13/2/3/defs.tex}
		\item Two balls are drawn at random with replacement from a box containing 10 black and 8 red balls. Find the probability that
		\label{ncert/12/13/2/12}
\begin{enumerate}
\item both balls are red.
\item first ball is black and second is red.
\item one of them is black and other is red.
\end{enumerate}

\item In a hostel, 60\% of the students read Hindi newspaper, 40\% read English newspaper and 20\% read both Hindi and English newspapers. A student is selected at random.
		\label{ncert/12/13/2/15}
\begin{enumerate}
\item Find the probability that she reads neither Hindi nor English newspapers.
\item If she reads Hindi newspaper, find the probability that she reads English newspaper.
\item If she reads English newspaper, find the probability that she reads Hindi newspaper.\\
\end{enumerate}
\item The probability of obtaining an even prime number on each die, when a pair of dice is rolled is 
\begin{enumerate}
    \item $0$ 
    
    \item $\frac{1}{3}$ 
    
    \item $\frac{1}{12}$ 
    
    \item $\frac{1}{36}$ 
\end{enumerate}
\solution
		%\input{ncert/12/13/2/17/defs.tex}
	\item A bag contains 4 red and 4 black balls, another bag contains 2 red and 6 black balls. One of the two bags is selected at random and a ball is drawn from the bag which is found to be red. Find the probability that the ball is drawn from the first bag.
\\
\solution
		%\input{ncert/12/13/3/2/main.tex}
  \item
  Cards with numbers 2 to 101 are placed in a box. A card is selected at random.Find the probability that the card has
\begin{enumerate}[label=(\roman*)]
	\item an even number 
	\item a square number
\end{enumerate}
\solution
%\input{exemplar/10/13/3/32/main.tex}
\item
The king, queen and jack of clubs are removed from a deck of 52 playing cards and then well shuffled. Now one card is drawn at random from the remaining cards.  Determine the probability that the card is
\begin{enumerate}[label=(\roman*)]
\item a club
\item 10 of hearts
\end{enumerate}
\solution
%\input{exemplar/10/13/3/29/main.tex}
\item A team of medical students doing their internship have to assist during surgeries
at a city hospital. The probabilities of surgeries rated as very complex, complex,
routine, simple or very simple are respectively, 0.15, 0.20, 0.31, 0.26, .08. Find
the probabilities that a particular surgery will be rated
\begin{enumerate}
	\item complex or very complex;
	\item neither very complex nor very simple;
	\item routine or complex
	\item routine or simple
\end{enumerate}
\solution
%\input{exemplar/11/16/3/8(1)/main.tex}
\item A card is selected from a pack of 52 cards.
\begin{enumerate}[label=(\alph*)]
    \item How many points are there in the sample space?
    \item Calculate the probability that the card is an ace of spades.
    \item Calculate the probability that the card is (i) an ace and (ii) black card.
\end{enumerate}
\solution
%\input{exemplar/11/16/3/4/main2.tex}
\item The probability that a non leap year selected at random will contain 53 sundays.
\\
\solution
%\input{exemplar/10/13/1/19/main.tex}
\item One of the four persons John, Rita, Aslam or Gurpreet will be promoted next
month. Consequently the sample space consists of four elementary outcomes
S = {John promoted, Rita promoted, Aslam promoted, Gurpreet promoted}
You are told that the chances of John’s promotion is same as that of Gurpreet,
Rita’s chances of promotion are twice as likely as Johns. Aslam’s chances are
four times that of John.
\begin{enumerate}
	\item Determine
	\begin{enumerate}
		\item P (John promoted)
		\item P (Rita promoted)
		\item P (Aslam promoted)
		\item P (Gurpreet promoted)
	\end{enumerate}
	\item If A = {John promoted or Gurpreet promoted}, find P (A).
\end{enumerate}
\solution
%\input{exemplar/11/16/3/10/main.tex}
\item A card is drawn from a deck of 52 cards. Find the probability of getting a king or a heart or a red card.\\
\solution
%\input{exemplar/11/16/3/15/main.tex}
\item The probability that a student will pass his examination is 0.73, the probability of
the student getting a compartment is 0.13, and the probability that the student will
either pass or get compartment is 0.96. State True or False.\\
\solution
%\input{exemplar/11/16/3/31/main.tex}
\item A card is selected from a pack of 52 cards\\
\begin{enumerate}[label=(\alph*)]
\item How many points are there in the sample space?
\item Calculate the probability that the cards is an ace of spades.
\item Calculate the probability that the card is (i) an ace (ii)black card.\\
\end{enumerate}
%\input{ncert/11/16/3/4_1/Prob_4.tex}
\item In a non-leap year, the probability of having 53 tuesdays or 53 wednesdays is\\
\solution
%\input{exemplar/11/16/3/18/main.tex}
\item There are 1000 sealed envelopes in a box, 10 of them contain a cash prize of
Rs 100 each, 100 of them contain a cash prize of Rs 50 each and 200 of them
contain a cash prize of Rs 10 each and rest do not contain any cash prize. If they
are well shuffled and an envelope is picked up out, what is the probability that it
contains no cash prize?\\
\solution
%\input{exemplar/10/13/3/34/main.tex}
\item 
A die is thrown and a card is selected at random from a deck of 52 playing cards. The probability of getting an even number on the die and a spade card.\\
\solution
%\input{exemplar/12/13/3/78/main.tex}
\item
If 4-digit numbers greater than 5,000 are randomly formed from the digits 0, 1, 3, 5, and 7, what is the probability of forming a number divisible by 5 when:
\begin{enumerate}
    \item The digits are repeated?
    \item The repetition of digits is not allowed?
\end{enumerate}
\solution
%\input{ncert/11/16/4/9/main.tex}
\item Consider the probability space $\brak{\Omega, \mathcal{G}, P}$ where $\Omega = [0,2]$ and $\mathcal{G} = \cbrak{\phi, \Omega, [0,1], (1,2]}$. Let $X$ and $Y$ be two functions on $\Omega$ defined as
\begin{align*}
    X(\omega) = 
    \begin{cases}
        1 & \text{if }\omega \in [0, 1]\\
        2 & \text{if }\omega \in (1, 2]
    \end{cases}
\end{align*}
and
\begin{align*}
    Y(\omega) = 
    \begin{cases}
        2 & \text{if }\omega \in [0, 1.5]\\
        3 & \text{if }\omega \in (1.5, 2].
    \end{cases}
\end{align*}
Then which one of the following statements is true?
\begin{enumerate}
    \item [(A)] $X$ is a random variable with respect to $\mathcal{G}$, but $Y$ is not a random variable with respect to $\mathcal{G}$.
    \item [(B)] $Y$ is a random variable with respect to $\mathcal{G}$, but $X$ is not a random variable with respect to $\mathcal{G}$.
    \item [(C)] Neither $X$ nor $Y$ is a random variable with respect to $\mathcal{G}$.
    \item [(D)] Both $X$ and $Y$ are random variables with respect to $\mathcal{G}$.
\end{enumerate} \hfill (GATE ST 2023)\\
\solution
%\input{gate/ST/2023/14/main.tex}
	\item  A die is loaded in such a way that each odd number is twice as likely to occur as
each even number. Find $P(G)$, where $G$ is the event that a number greater than
3 occurs on a single roll of the die.
\\
\solution
		%\input{exemplar/11/16/3/5/main.tex}
	\item All the jacks, queens and kings are removed from a deck of 52 playing cards. The remaining cards are well shuffled and then one card is drawn at random. Giving ace a value 1 similar value for other cards, find the probability that the card has a value 
		\begin{enumerate}
			\item 7
			\item greater than 7
			\item less than 7
		\end{enumerate}
		%\input{exemplar/10/13/3/30/main.tex}
  \item A Lot consists of 48 mobile phones of which 42 are good, 3 have only minor defects and 3 have major defects.Varnika will buy a phone if it is good but the trader will only buy a mobile if it has no major defects. One phone is selected at random from the lot. What is the probability that it is
\begin{enumerate}
	\item acceptable to Varnika?
            \item acceptable to the trader?
\end{enumerate}
\solution
	%\input{exemplar/10/13/3/40/main.tex}
 \item A student says that if you throw a die, it will show up 1 or not 1. Therefore, the probability of getting 1 and the probability of getting 'not 1' each is equal to $\frac{1}{2}$. Is this correct? Give reasons.\\
 \solution
        %\input{exemplar/10/13/2/9/main.tex}
   \item Four candidates A, B, C, D have ap-
plied for the assignment to coach a school cricket
team. If A is twice as likely to be selected as B, and
B and C are given about the same chance of being
selected, while C is twice as likely to be selected
as D, what are the probabilities that
\begin{enumerate}
\item C will be selected?
\item A will not be selected?
\end{enumerate}
	%\input{exemplar/11/16/3/9/main.tex}
 \item A bag contain 24 balls of which $x$ balls are red, $2x$ are white and $3x$ are blue. A ball is selected at random, What is the probability that it is
\begin{enumerate}[label=\alph*)]
\item not red ?
\item white ?
\end{enumerate}
%\input{exemplar/10/13/3/41/main.tex}
If the letters of the word ASSASSINATION are arranged at random. Find the Probability that
\begin{enumerate}[label=(\alph*)]
\item Four $S's$ come consecutively in the word
\item Two  $I's$ and two $N's$ come together
\item All $A's$ are not coming together
\item No two $A's$ are coming together
\end{enumerate}
%\input{exemplar/11/16/3/14/main.tex}
	\item One urn contains two black balls (labelled B1 and B2) and one white ball. A
	second urn contains one black ball and two white balls (labelled W1 and W2).
	Suppose the following experiment is performed. One of the two urns is chosen
	at random. Next a ball is randomly chosen from the urn. Then a second ball is
	chosen at random from the same urn without replacing the first ball.
	
	\begin{enumerate}
	\item What is the probability that two black balls are chosen?
	
	\item What is the probability that two balls of opposite colour are chosen?
	\end{enumerate}
	\solution
	%\input{exemplar/11/16/3/12/main1.tex}
\end{enumerate}

		%
\item 
Two cards are drawn at random and without replacement from a pack of 52 playing cards. Find the probability that both the cards are black.
\\
\solution
		%\begin{enumerate}[label=\thesection.\arabic*,ref=\thesection.\theenumi]
	\item One card is drawn from a well-shuffled deck of 52 cards. Find the probability of getting
\begin{enumerate}
\item A king of red colour 
\item A face card 
\item A red face card
\item The jack of hearts
\item A spade
\item The queen of diamonds

\end{enumerate}
\solution
		%\input{ncert/10/15/1/14/main.tex}
	\item Five cards—the ten, jack, queen, king and ace of diamonds, are well-shuffled with their face downwards. One card is then picked up at random.
\begin{enumerate}
\item
What is the probability that the card is the queen? 
\item
If the queen is drawn and put aside, what is the probability that the second card picked up is (a) an ace? (b) a queen?\\
\end{enumerate}
\solution
		%\input{ncert/10/15/1/15/defs.tex}
	\item A bag contains $5$ red balls and some blue balls. If the probability of drawing a blue ball is double that if a red ball, determine the number of blue balls in the bag. 
		\\
\solution
		%\input{ncert/10/15/2/3/defs.tex}
	\item A card is selected from a pack of 52 cards.
 \begin{enumerate}[label=(\alph*)] 
                 \item How many points are there in the sample space?
                 \item Calculate the probability that the card is an ace of spades.
                 \item Calculate the probability that the card is (i) an ace and (ii) black card.
 \end{enumerate}
\solution
		%\input{ncert/11/16/3/4/main.tex}
\item Four cards are drawn from a well-shuffled deck of 52 cards. What is the probability of obtaining 3 diamonds and one spade.
\\
\solution
		%\input{ncert/11/16/4/2/defs.tex}
\item In a certain lottery 10,000 tickets are sold and ten equal prizes are awarded. What is the probability of not getting a prize if you buy (a) one ticket (b) two tickets (c) 10 tickets ?	
\\
\solution
		%\input{ncert/11/16/4/4/defs.tex}
		%
\item 
Out of 100 students, two sections of 40 and 60 are formed. If you and your friend are among the 100 students, what is the probability that
\begin{enumerate}
\item you both enter the same section?
\item you both enter the different sections?
\end{enumerate}
\solution
		%\input{ncert/11/16/4/5/defs.tex}
	\item 
The number lock of a suitcase has 4 wheels each labelled with ten digits i.e. from 0 to 9.The lock opens with a sequence of four digits with no repeats.What is the probability of a person getting the right sequence to open the suitcase.
\\
\solution
		%\input{ncert/11/16/4/10/defs.tex}
		%
\item 
Two cards are drawn at random and without replacement from a pack of 52 playing cards. Find the probability that both the cards are black.
\\
\solution
		%\input{ncert/12/13/2/2/defs.tex}
		\item A box of oranges is inspected by examining three randomly selected oranges drawn without replacement. If all the three oranges are good, the box is approved for sale, otherwise, it is rejected. Find the probability that a box containing 15 oranges out of which 12 are good and 3 are bad ones will be approved for sale.
		\label{ncert/12/13/2/3/defs.tex}
		\item Two balls are drawn at random with replacement from a box containing 10 black and 8 red balls. Find the probability that
		\label{ncert/12/13/2/12}
\begin{enumerate}
\item both balls are red.
\item first ball is black and second is red.
\item one of them is black and other is red.
\end{enumerate}

\item In a hostel, 60\% of the students read Hindi newspaper, 40\% read English newspaper and 20\% read both Hindi and English newspapers. A student is selected at random.
		\label{ncert/12/13/2/15}
\begin{enumerate}
\item Find the probability that she reads neither Hindi nor English newspapers.
\item If she reads Hindi newspaper, find the probability that she reads English newspaper.
\item If she reads English newspaper, find the probability that she reads Hindi newspaper.\\
\end{enumerate}
\item The probability of obtaining an even prime number on each die, when a pair of dice is rolled is 
\begin{enumerate}
    \item $0$ 
    
    \item $\frac{1}{3}$ 
    
    \item $\frac{1}{12}$ 
    
    \item $\frac{1}{36}$ 
\end{enumerate}
\solution
		%\input{ncert/12/13/2/17/defs.tex}
	\item A bag contains 4 red and 4 black balls, another bag contains 2 red and 6 black balls. One of the two bags is selected at random and a ball is drawn from the bag which is found to be red. Find the probability that the ball is drawn from the first bag.
\\
\solution
		%\input{ncert/12/13/3/2/main.tex}
  \item
  Cards with numbers 2 to 101 are placed in a box. A card is selected at random.Find the probability that the card has
\begin{enumerate}[label=(\roman*)]
	\item an even number 
	\item a square number
\end{enumerate}
\solution
%\input{exemplar/10/13/3/32/main.tex}
\item
The king, queen and jack of clubs are removed from a deck of 52 playing cards and then well shuffled. Now one card is drawn at random from the remaining cards.  Determine the probability that the card is
\begin{enumerate}[label=(\roman*)]
\item a club
\item 10 of hearts
\end{enumerate}
\solution
%\input{exemplar/10/13/3/29/main.tex}
\item A team of medical students doing their internship have to assist during surgeries
at a city hospital. The probabilities of surgeries rated as very complex, complex,
routine, simple or very simple are respectively, 0.15, 0.20, 0.31, 0.26, .08. Find
the probabilities that a particular surgery will be rated
\begin{enumerate}
	\item complex or very complex;
	\item neither very complex nor very simple;
	\item routine or complex
	\item routine or simple
\end{enumerate}
\solution
%\input{exemplar/11/16/3/8(1)/main.tex}
\item A card is selected from a pack of 52 cards.
\begin{enumerate}[label=(\alph*)]
    \item How many points are there in the sample space?
    \item Calculate the probability that the card is an ace of spades.
    \item Calculate the probability that the card is (i) an ace and (ii) black card.
\end{enumerate}
\solution
%\input{exemplar/11/16/3/4/main2.tex}
\item The probability that a non leap year selected at random will contain 53 sundays.
\\
\solution
%\input{exemplar/10/13/1/19/main.tex}
\item One of the four persons John, Rita, Aslam or Gurpreet will be promoted next
month. Consequently the sample space consists of four elementary outcomes
S = {John promoted, Rita promoted, Aslam promoted, Gurpreet promoted}
You are told that the chances of John’s promotion is same as that of Gurpreet,
Rita’s chances of promotion are twice as likely as Johns. Aslam’s chances are
four times that of John.
\begin{enumerate}
	\item Determine
	\begin{enumerate}
		\item P (John promoted)
		\item P (Rita promoted)
		\item P (Aslam promoted)
		\item P (Gurpreet promoted)
	\end{enumerate}
	\item If A = {John promoted or Gurpreet promoted}, find P (A).
\end{enumerate}
\solution
%\input{exemplar/11/16/3/10/main.tex}
\item A card is drawn from a deck of 52 cards. Find the probability of getting a king or a heart or a red card.\\
\solution
%\input{exemplar/11/16/3/15/main.tex}
\item The probability that a student will pass his examination is 0.73, the probability of
the student getting a compartment is 0.13, and the probability that the student will
either pass or get compartment is 0.96. State True or False.\\
\solution
%\input{exemplar/11/16/3/31/main.tex}
\item A card is selected from a pack of 52 cards\\
\begin{enumerate}[label=(\alph*)]
\item How many points are there in the sample space?
\item Calculate the probability that the cards is an ace of spades.
\item Calculate the probability that the card is (i) an ace (ii)black card.\\
\end{enumerate}
%\input{ncert/11/16/3/4_1/Prob_4.tex}
\item In a non-leap year, the probability of having 53 tuesdays or 53 wednesdays is\\
\solution
%\input{exemplar/11/16/3/18/main.tex}
\item There are 1000 sealed envelopes in a box, 10 of them contain a cash prize of
Rs 100 each, 100 of them contain a cash prize of Rs 50 each and 200 of them
contain a cash prize of Rs 10 each and rest do not contain any cash prize. If they
are well shuffled and an envelope is picked up out, what is the probability that it
contains no cash prize?\\
\solution
%\input{exemplar/10/13/3/34/main.tex}
\item 
A die is thrown and a card is selected at random from a deck of 52 playing cards. The probability of getting an even number on the die and a spade card.\\
\solution
%\input{exemplar/12/13/3/78/main.tex}
\item
If 4-digit numbers greater than 5,000 are randomly formed from the digits 0, 1, 3, 5, and 7, what is the probability of forming a number divisible by 5 when:
\begin{enumerate}
    \item The digits are repeated?
    \item The repetition of digits is not allowed?
\end{enumerate}
\solution
%\input{ncert/11/16/4/9/main.tex}
\item Consider the probability space $\brak{\Omega, \mathcal{G}, P}$ where $\Omega = [0,2]$ and $\mathcal{G} = \cbrak{\phi, \Omega, [0,1], (1,2]}$. Let $X$ and $Y$ be two functions on $\Omega$ defined as
\begin{align*}
    X(\omega) = 
    \begin{cases}
        1 & \text{if }\omega \in [0, 1]\\
        2 & \text{if }\omega \in (1, 2]
    \end{cases}
\end{align*}
and
\begin{align*}
    Y(\omega) = 
    \begin{cases}
        2 & \text{if }\omega \in [0, 1.5]\\
        3 & \text{if }\omega \in (1.5, 2].
    \end{cases}
\end{align*}
Then which one of the following statements is true?
\begin{enumerate}
    \item [(A)] $X$ is a random variable with respect to $\mathcal{G}$, but $Y$ is not a random variable with respect to $\mathcal{G}$.
    \item [(B)] $Y$ is a random variable with respect to $\mathcal{G}$, but $X$ is not a random variable with respect to $\mathcal{G}$.
    \item [(C)] Neither $X$ nor $Y$ is a random variable with respect to $\mathcal{G}$.
    \item [(D)] Both $X$ and $Y$ are random variables with respect to $\mathcal{G}$.
\end{enumerate} \hfill (GATE ST 2023)\\
\solution
%\input{gate/ST/2023/14/main.tex}
	\item  A die is loaded in such a way that each odd number is twice as likely to occur as
each even number. Find $P(G)$, where $G$ is the event that a number greater than
3 occurs on a single roll of the die.
\\
\solution
		%\input{exemplar/11/16/3/5/main.tex}
	\item All the jacks, queens and kings are removed from a deck of 52 playing cards. The remaining cards are well shuffled and then one card is drawn at random. Giving ace a value 1 similar value for other cards, find the probability that the card has a value 
		\begin{enumerate}
			\item 7
			\item greater than 7
			\item less than 7
		\end{enumerate}
		%\input{exemplar/10/13/3/30/main.tex}
  \item A Lot consists of 48 mobile phones of which 42 are good, 3 have only minor defects and 3 have major defects.Varnika will buy a phone if it is good but the trader will only buy a mobile if it has no major defects. One phone is selected at random from the lot. What is the probability that it is
\begin{enumerate}
	\item acceptable to Varnika?
            \item acceptable to the trader?
\end{enumerate}
\solution
	%\input{exemplar/10/13/3/40/main.tex}
 \item A student says that if you throw a die, it will show up 1 or not 1. Therefore, the probability of getting 1 and the probability of getting 'not 1' each is equal to $\frac{1}{2}$. Is this correct? Give reasons.\\
 \solution
        %\input{exemplar/10/13/2/9/main.tex}
   \item Four candidates A, B, C, D have ap-
plied for the assignment to coach a school cricket
team. If A is twice as likely to be selected as B, and
B and C are given about the same chance of being
selected, while C is twice as likely to be selected
as D, what are the probabilities that
\begin{enumerate}
\item C will be selected?
\item A will not be selected?
\end{enumerate}
	%\input{exemplar/11/16/3/9/main.tex}
 \item A bag contain 24 balls of which $x$ balls are red, $2x$ are white and $3x$ are blue. A ball is selected at random, What is the probability that it is
\begin{enumerate}[label=\alph*)]
\item not red ?
\item white ?
\end{enumerate}
%\input{exemplar/10/13/3/41/main.tex}
If the letters of the word ASSASSINATION are arranged at random. Find the Probability that
\begin{enumerate}[label=(\alph*)]
\item Four $S's$ come consecutively in the word
\item Two  $I's$ and two $N's$ come together
\item All $A's$ are not coming together
\item No two $A's$ are coming together
\end{enumerate}
%\input{exemplar/11/16/3/14/main.tex}
	\item One urn contains two black balls (labelled B1 and B2) and one white ball. A
	second urn contains one black ball and two white balls (labelled W1 and W2).
	Suppose the following experiment is performed. One of the two urns is chosen
	at random. Next a ball is randomly chosen from the urn. Then a second ball is
	chosen at random from the same urn without replacing the first ball.
	
	\begin{enumerate}
	\item What is the probability that two black balls are chosen?
	
	\item What is the probability that two balls of opposite colour are chosen?
	\end{enumerate}
	\solution
	%\input{exemplar/11/16/3/12/main1.tex}
\end{enumerate}

		\item A box of oranges is inspected by examining three randomly selected oranges drawn without replacement. If all the three oranges are good, the box is approved for sale, otherwise, it is rejected. Find the probability that a box containing 15 oranges out of which 12 are good and 3 are bad ones will be approved for sale.
		\label{ncert/12/13/2/3/defs.tex}
		\item Two balls are drawn at random with replacement from a box containing 10 black and 8 red balls. Find the probability that
		\label{ncert/12/13/2/12}
\begin{enumerate}
\item both balls are red.
\item first ball is black and second is red.
\item one of them is black and other is red.
\end{enumerate}

\item In a hostel, 60\% of the students read Hindi newspaper, 40\% read English newspaper and 20\% read both Hindi and English newspapers. A student is selected at random.
		\label{ncert/12/13/2/15}
\begin{enumerate}
\item Find the probability that she reads neither Hindi nor English newspapers.
\item If she reads Hindi newspaper, find the probability that she reads English newspaper.
\item If she reads English newspaper, find the probability that she reads Hindi newspaper.\\
\end{enumerate}
\item The probability of obtaining an even prime number on each die, when a pair of dice is rolled is 
\begin{enumerate}
    \item $0$ 
    
    \item $\frac{1}{3}$ 
    
    \item $\frac{1}{12}$ 
    
    \item $\frac{1}{36}$ 
\end{enumerate}
\solution
		%\begin{enumerate}[label=\thesection.\arabic*,ref=\thesection.\theenumi]
	\item One card is drawn from a well-shuffled deck of 52 cards. Find the probability of getting
\begin{enumerate}
\item A king of red colour 
\item A face card 
\item A red face card
\item The jack of hearts
\item A spade
\item The queen of diamonds

\end{enumerate}
\solution
		%\input{ncert/10/15/1/14/main.tex}
	\item Five cards—the ten, jack, queen, king and ace of diamonds, are well-shuffled with their face downwards. One card is then picked up at random.
\begin{enumerate}
\item
What is the probability that the card is the queen? 
\item
If the queen is drawn and put aside, what is the probability that the second card picked up is (a) an ace? (b) a queen?\\
\end{enumerate}
\solution
		%\input{ncert/10/15/1/15/defs.tex}
	\item A bag contains $5$ red balls and some blue balls. If the probability of drawing a blue ball is double that if a red ball, determine the number of blue balls in the bag. 
		\\
\solution
		%\input{ncert/10/15/2/3/defs.tex}
	\item A card is selected from a pack of 52 cards.
 \begin{enumerate}[label=(\alph*)] 
                 \item How many points are there in the sample space?
                 \item Calculate the probability that the card is an ace of spades.
                 \item Calculate the probability that the card is (i) an ace and (ii) black card.
 \end{enumerate}
\solution
		%\input{ncert/11/16/3/4/main.tex}
\item Four cards are drawn from a well-shuffled deck of 52 cards. What is the probability of obtaining 3 diamonds and one spade.
\\
\solution
		%\input{ncert/11/16/4/2/defs.tex}
\item In a certain lottery 10,000 tickets are sold and ten equal prizes are awarded. What is the probability of not getting a prize if you buy (a) one ticket (b) two tickets (c) 10 tickets ?	
\\
\solution
		%\input{ncert/11/16/4/4/defs.tex}
		%
\item 
Out of 100 students, two sections of 40 and 60 are formed. If you and your friend are among the 100 students, what is the probability that
\begin{enumerate}
\item you both enter the same section?
\item you both enter the different sections?
\end{enumerate}
\solution
		%\input{ncert/11/16/4/5/defs.tex}
	\item 
The number lock of a suitcase has 4 wheels each labelled with ten digits i.e. from 0 to 9.The lock opens with a sequence of four digits with no repeats.What is the probability of a person getting the right sequence to open the suitcase.
\\
\solution
		%\input{ncert/11/16/4/10/defs.tex}
		%
\item 
Two cards are drawn at random and without replacement from a pack of 52 playing cards. Find the probability that both the cards are black.
\\
\solution
		%\input{ncert/12/13/2/2/defs.tex}
		\item A box of oranges is inspected by examining three randomly selected oranges drawn without replacement. If all the three oranges are good, the box is approved for sale, otherwise, it is rejected. Find the probability that a box containing 15 oranges out of which 12 are good and 3 are bad ones will be approved for sale.
		\label{ncert/12/13/2/3/defs.tex}
		\item Two balls are drawn at random with replacement from a box containing 10 black and 8 red balls. Find the probability that
		\label{ncert/12/13/2/12}
\begin{enumerate}
\item both balls are red.
\item first ball is black and second is red.
\item one of them is black and other is red.
\end{enumerate}

\item In a hostel, 60\% of the students read Hindi newspaper, 40\% read English newspaper and 20\% read both Hindi and English newspapers. A student is selected at random.
		\label{ncert/12/13/2/15}
\begin{enumerate}
\item Find the probability that she reads neither Hindi nor English newspapers.
\item If she reads Hindi newspaper, find the probability that she reads English newspaper.
\item If she reads English newspaper, find the probability that she reads Hindi newspaper.\\
\end{enumerate}
\item The probability of obtaining an even prime number on each die, when a pair of dice is rolled is 
\begin{enumerate}
    \item $0$ 
    
    \item $\frac{1}{3}$ 
    
    \item $\frac{1}{12}$ 
    
    \item $\frac{1}{36}$ 
\end{enumerate}
\solution
		%\input{ncert/12/13/2/17/defs.tex}
	\item A bag contains 4 red and 4 black balls, another bag contains 2 red and 6 black balls. One of the two bags is selected at random and a ball is drawn from the bag which is found to be red. Find the probability that the ball is drawn from the first bag.
\\
\solution
		%\input{ncert/12/13/3/2/main.tex}
  \item
  Cards with numbers 2 to 101 are placed in a box. A card is selected at random.Find the probability that the card has
\begin{enumerate}[label=(\roman*)]
	\item an even number 
	\item a square number
\end{enumerate}
\solution
%\input{exemplar/10/13/3/32/main.tex}
\item
The king, queen and jack of clubs are removed from a deck of 52 playing cards and then well shuffled. Now one card is drawn at random from the remaining cards.  Determine the probability that the card is
\begin{enumerate}[label=(\roman*)]
\item a club
\item 10 of hearts
\end{enumerate}
\solution
%\input{exemplar/10/13/3/29/main.tex}
\item A team of medical students doing their internship have to assist during surgeries
at a city hospital. The probabilities of surgeries rated as very complex, complex,
routine, simple or very simple are respectively, 0.15, 0.20, 0.31, 0.26, .08. Find
the probabilities that a particular surgery will be rated
\begin{enumerate}
	\item complex or very complex;
	\item neither very complex nor very simple;
	\item routine or complex
	\item routine or simple
\end{enumerate}
\solution
%\input{exemplar/11/16/3/8(1)/main.tex}
\item A card is selected from a pack of 52 cards.
\begin{enumerate}[label=(\alph*)]
    \item How many points are there in the sample space?
    \item Calculate the probability that the card is an ace of spades.
    \item Calculate the probability that the card is (i) an ace and (ii) black card.
\end{enumerate}
\solution
%\input{exemplar/11/16/3/4/main2.tex}
\item The probability that a non leap year selected at random will contain 53 sundays.
\\
\solution
%\input{exemplar/10/13/1/19/main.tex}
\item One of the four persons John, Rita, Aslam or Gurpreet will be promoted next
month. Consequently the sample space consists of four elementary outcomes
S = {John promoted, Rita promoted, Aslam promoted, Gurpreet promoted}
You are told that the chances of John’s promotion is same as that of Gurpreet,
Rita’s chances of promotion are twice as likely as Johns. Aslam’s chances are
four times that of John.
\begin{enumerate}
	\item Determine
	\begin{enumerate}
		\item P (John promoted)
		\item P (Rita promoted)
		\item P (Aslam promoted)
		\item P (Gurpreet promoted)
	\end{enumerate}
	\item If A = {John promoted or Gurpreet promoted}, find P (A).
\end{enumerate}
\solution
%\input{exemplar/11/16/3/10/main.tex}
\item A card is drawn from a deck of 52 cards. Find the probability of getting a king or a heart or a red card.\\
\solution
%\input{exemplar/11/16/3/15/main.tex}
\item The probability that a student will pass his examination is 0.73, the probability of
the student getting a compartment is 0.13, and the probability that the student will
either pass or get compartment is 0.96. State True or False.\\
\solution
%\input{exemplar/11/16/3/31/main.tex}
\item A card is selected from a pack of 52 cards\\
\begin{enumerate}[label=(\alph*)]
\item How many points are there in the sample space?
\item Calculate the probability that the cards is an ace of spades.
\item Calculate the probability that the card is (i) an ace (ii)black card.\\
\end{enumerate}
%\input{ncert/11/16/3/4_1/Prob_4.tex}
\item In a non-leap year, the probability of having 53 tuesdays or 53 wednesdays is\\
\solution
%\input{exemplar/11/16/3/18/main.tex}
\item There are 1000 sealed envelopes in a box, 10 of them contain a cash prize of
Rs 100 each, 100 of them contain a cash prize of Rs 50 each and 200 of them
contain a cash prize of Rs 10 each and rest do not contain any cash prize. If they
are well shuffled and an envelope is picked up out, what is the probability that it
contains no cash prize?\\
\solution
%\input{exemplar/10/13/3/34/main.tex}
\item 
A die is thrown and a card is selected at random from a deck of 52 playing cards. The probability of getting an even number on the die and a spade card.\\
\solution
%\input{exemplar/12/13/3/78/main.tex}
\item
If 4-digit numbers greater than 5,000 are randomly formed from the digits 0, 1, 3, 5, and 7, what is the probability of forming a number divisible by 5 when:
\begin{enumerate}
    \item The digits are repeated?
    \item The repetition of digits is not allowed?
\end{enumerate}
\solution
%\input{ncert/11/16/4/9/main.tex}
\item Consider the probability space $\brak{\Omega, \mathcal{G}, P}$ where $\Omega = [0,2]$ and $\mathcal{G} = \cbrak{\phi, \Omega, [0,1], (1,2]}$. Let $X$ and $Y$ be two functions on $\Omega$ defined as
\begin{align*}
    X(\omega) = 
    \begin{cases}
        1 & \text{if }\omega \in [0, 1]\\
        2 & \text{if }\omega \in (1, 2]
    \end{cases}
\end{align*}
and
\begin{align*}
    Y(\omega) = 
    \begin{cases}
        2 & \text{if }\omega \in [0, 1.5]\\
        3 & \text{if }\omega \in (1.5, 2].
    \end{cases}
\end{align*}
Then which one of the following statements is true?
\begin{enumerate}
    \item [(A)] $X$ is a random variable with respect to $\mathcal{G}$, but $Y$ is not a random variable with respect to $\mathcal{G}$.
    \item [(B)] $Y$ is a random variable with respect to $\mathcal{G}$, but $X$ is not a random variable with respect to $\mathcal{G}$.
    \item [(C)] Neither $X$ nor $Y$ is a random variable with respect to $\mathcal{G}$.
    \item [(D)] Both $X$ and $Y$ are random variables with respect to $\mathcal{G}$.
\end{enumerate} \hfill (GATE ST 2023)\\
\solution
%\input{gate/ST/2023/14/main.tex}
	\item  A die is loaded in such a way that each odd number is twice as likely to occur as
each even number. Find $P(G)$, where $G$ is the event that a number greater than
3 occurs on a single roll of the die.
\\
\solution
		%\input{exemplar/11/16/3/5/main.tex}
	\item All the jacks, queens and kings are removed from a deck of 52 playing cards. The remaining cards are well shuffled and then one card is drawn at random. Giving ace a value 1 similar value for other cards, find the probability that the card has a value 
		\begin{enumerate}
			\item 7
			\item greater than 7
			\item less than 7
		\end{enumerate}
		%\input{exemplar/10/13/3/30/main.tex}
  \item A Lot consists of 48 mobile phones of which 42 are good, 3 have only minor defects and 3 have major defects.Varnika will buy a phone if it is good but the trader will only buy a mobile if it has no major defects. One phone is selected at random from the lot. What is the probability that it is
\begin{enumerate}
	\item acceptable to Varnika?
            \item acceptable to the trader?
\end{enumerate}
\solution
	%\input{exemplar/10/13/3/40/main.tex}
 \item A student says that if you throw a die, it will show up 1 or not 1. Therefore, the probability of getting 1 and the probability of getting 'not 1' each is equal to $\frac{1}{2}$. Is this correct? Give reasons.\\
 \solution
        %\input{exemplar/10/13/2/9/main.tex}
   \item Four candidates A, B, C, D have ap-
plied for the assignment to coach a school cricket
team. If A is twice as likely to be selected as B, and
B and C are given about the same chance of being
selected, while C is twice as likely to be selected
as D, what are the probabilities that
\begin{enumerate}
\item C will be selected?
\item A will not be selected?
\end{enumerate}
	%\input{exemplar/11/16/3/9/main.tex}
 \item A bag contain 24 balls of which $x$ balls are red, $2x$ are white and $3x$ are blue. A ball is selected at random, What is the probability that it is
\begin{enumerate}[label=\alph*)]
\item not red ?
\item white ?
\end{enumerate}
%\input{exemplar/10/13/3/41/main.tex}
If the letters of the word ASSASSINATION are arranged at random. Find the Probability that
\begin{enumerate}[label=(\alph*)]
\item Four $S's$ come consecutively in the word
\item Two  $I's$ and two $N's$ come together
\item All $A's$ are not coming together
\item No two $A's$ are coming together
\end{enumerate}
%\input{exemplar/11/16/3/14/main.tex}
	\item One urn contains two black balls (labelled B1 and B2) and one white ball. A
	second urn contains one black ball and two white balls (labelled W1 and W2).
	Suppose the following experiment is performed. One of the two urns is chosen
	at random. Next a ball is randomly chosen from the urn. Then a second ball is
	chosen at random from the same urn without replacing the first ball.
	
	\begin{enumerate}
	\item What is the probability that two black balls are chosen?
	
	\item What is the probability that two balls of opposite colour are chosen?
	\end{enumerate}
	\solution
	%\input{exemplar/11/16/3/12/main1.tex}
\end{enumerate}

	\item A bag contains 4 red and 4 black balls, another bag contains 2 red and 6 black balls. One of the two bags is selected at random and a ball is drawn from the bag which is found to be red. Find the probability that the ball is drawn from the first bag.
\\
\solution
		%\begin{table}[H]
	\centering
\begin{tabular}{|c|c|c|}
\hline
Random variable &Value &Definition\\ \hline
\multirow{3}{*}{X} &0 &Slips of Rs 1\\
&1 &Slips of Rs 5\\
&2 &Slips of Rs 13\\ \hline
\multirow{2}{*}{Y} &0 &Box A\\
&1 &Box B\\\hline
\end{tabular}
\caption{}
\label{tab:Distribution}
\end{table}
See \tabref{tab:Distribution}.
\begin{align}
p_{Y}\brak{k}= \begin{cases} 
      \frac{1}{3} & {k=0} \\
      \frac{2}{3 }& {k=1} 
   \end{cases}
   \\
p_{Y|X}\brak{0|0} = \frac{19}{25}\, 
p_{Y|X}\brak{0|1} = \frac{6}{25}\,
p_{Y|X}\brak{1|0} = \frac{45}{50}\,
p_{Y|X}\brak{1|2} = \frac{5}{50}
\end{align}
The desired probability is the probability that a slip drawn at random is marked other than Rs 1,
\begin{align}
&=1-p_X\brak{0}\\
&= p_X(1) + p_X(2)
\end{align}
Using Bayes theorem,
\begin{align}
&= p_Y\brak{0} \times \pr{Y=0 | X=1} + p_Y\brak{1} \times \pr{Y=1|X=2}\\
&=\frac{1}{3} \times \frac{6}{25} + \frac{2}{3} \times \frac{5}{50}\\
&=\frac{11}{75}
\end{align}

\newpage

%\tableofcontents

\bigskip

\renewcommand{\thefigure}{\theenumi}
\renewcommand{\thetable}{\theenumi}
%\renewcommand{\theequation}{\theenumi}

%\begin{abstract}
%%\boldmath
%In this letter, an algorithm for evaluating the exact analytical bit error rate  (BER)  for the piecewise linear (PL) combiner for  multiple relays is presented. Previous results were available only for upto three relays. The algorithm is unique in the sense that  the actual mathematical expressions, that are prohibitively large, need not be explicitly obtained. The diversity gain due to multiple relays is shown through plots of the analytical BER, well supported by simulations. 
%
%\end{abstract}
% IEEEtran.cls defaults to using nonbold math in the Abstract.
% This preserves the distinction between vectors and scalars. However,
% if the journal you are submitting to favors bold math in the abstract,
% then you can use LaTeX's standard command \boldmath at the very start
% of the abstract to achieve this. Many IEEE journals frown on math
% in the abstract anyway.

% Note that keywords are not normally used for peerreview papers.
%\begin{IEEEkeywords}
%Cooperative diversity, decode and forward, piecewise linear
%\end{IEEEkeywords}



% For peer review papers, you can put extra information on the cover
% page as needed:
% \ifCLASSOPTIONpeerreview
% \begin{center} \bfseries EDICS Category: 3-BBND \end{center}
% \fi
%
% For peerreview papers, this IEEEtran command inserts a page break and
% creates the second title. It will be ignored for other modes.
%\IEEEpeerreviewmaketitle




  \item
  Cards with numbers 2 to 101 are placed in a box. A card is selected at random.Find the probability that the card has
\begin{enumerate}[label=(\roman*)]
	\item an even number 
	\item a square number
\end{enumerate}
\solution
%\begin{table}[H]
	\centering
\begin{tabular}{|c|c|c|}
\hline
Random variable &Value &Definition\\ \hline
\multirow{3}{*}{X} &0 &Slips of Rs 1\\
&1 &Slips of Rs 5\\
&2 &Slips of Rs 13\\ \hline
\multirow{2}{*}{Y} &0 &Box A\\
&1 &Box B\\\hline
\end{tabular}
\caption{}
\label{tab:Distribution}
\end{table}
See \tabref{tab:Distribution}.
\begin{align}
p_{Y}\brak{k}= \begin{cases} 
      \frac{1}{3} & {k=0} \\
      \frac{2}{3 }& {k=1} 
   \end{cases}
   \\
p_{Y|X}\brak{0|0} = \frac{19}{25}\, 
p_{Y|X}\brak{0|1} = \frac{6}{25}\,
p_{Y|X}\brak{1|0} = \frac{45}{50}\,
p_{Y|X}\brak{1|2} = \frac{5}{50}
\end{align}
The desired probability is the probability that a slip drawn at random is marked other than Rs 1,
\begin{align}
&=1-p_X\brak{0}\\
&= p_X(1) + p_X(2)
\end{align}
Using Bayes theorem,
\begin{align}
&= p_Y\brak{0} \times \pr{Y=0 | X=1} + p_Y\brak{1} \times \pr{Y=1|X=2}\\
&=\frac{1}{3} \times \frac{6}{25} + \frac{2}{3} \times \frac{5}{50}\\
&=\frac{11}{75}
\end{align}

\newpage

%\tableofcontents

\bigskip

\renewcommand{\thefigure}{\theenumi}
\renewcommand{\thetable}{\theenumi}
%\renewcommand{\theequation}{\theenumi}

%\begin{abstract}
%%\boldmath
%In this letter, an algorithm for evaluating the exact analytical bit error rate  (BER)  for the piecewise linear (PL) combiner for  multiple relays is presented. Previous results were available only for upto three relays. The algorithm is unique in the sense that  the actual mathematical expressions, that are prohibitively large, need not be explicitly obtained. The diversity gain due to multiple relays is shown through plots of the analytical BER, well supported by simulations. 
%
%\end{abstract}
% IEEEtran.cls defaults to using nonbold math in the Abstract.
% This preserves the distinction between vectors and scalars. However,
% if the journal you are submitting to favors bold math in the abstract,
% then you can use LaTeX's standard command \boldmath at the very start
% of the abstract to achieve this. Many IEEE journals frown on math
% in the abstract anyway.

% Note that keywords are not normally used for peerreview papers.
%\begin{IEEEkeywords}
%Cooperative diversity, decode and forward, piecewise linear
%\end{IEEEkeywords}



% For peer review papers, you can put extra information on the cover
% page as needed:
% \ifCLASSOPTIONpeerreview
% \begin{center} \bfseries EDICS Category: 3-BBND \end{center}
% \fi
%
% For peerreview papers, this IEEEtran command inserts a page break and
% creates the second title. It will be ignored for other modes.
%\IEEEpeerreviewmaketitle




\item
The king, queen and jack of clubs are removed from a deck of 52 playing cards and then well shuffled. Now one card is drawn at random from the remaining cards.  Determine the probability that the card is
\begin{enumerate}[label=(\roman*)]
\item a club
\item 10 of hearts
\end{enumerate}
\solution
%\begin{table}[H]
	\centering
\begin{tabular}{|c|c|c|}
\hline
Random variable &Value &Definition\\ \hline
\multirow{3}{*}{X} &0 &Slips of Rs 1\\
&1 &Slips of Rs 5\\
&2 &Slips of Rs 13\\ \hline
\multirow{2}{*}{Y} &0 &Box A\\
&1 &Box B\\\hline
\end{tabular}
\caption{}
\label{tab:Distribution}
\end{table}
See \tabref{tab:Distribution}.
\begin{align}
p_{Y}\brak{k}= \begin{cases} 
      \frac{1}{3} & {k=0} \\
      \frac{2}{3 }& {k=1} 
   \end{cases}
   \\
p_{Y|X}\brak{0|0} = \frac{19}{25}\, 
p_{Y|X}\brak{0|1} = \frac{6}{25}\,
p_{Y|X}\brak{1|0} = \frac{45}{50}\,
p_{Y|X}\brak{1|2} = \frac{5}{50}
\end{align}
The desired probability is the probability that a slip drawn at random is marked other than Rs 1,
\begin{align}
&=1-p_X\brak{0}\\
&= p_X(1) + p_X(2)
\end{align}
Using Bayes theorem,
\begin{align}
&= p_Y\brak{0} \times \pr{Y=0 | X=1} + p_Y\brak{1} \times \pr{Y=1|X=2}\\
&=\frac{1}{3} \times \frac{6}{25} + \frac{2}{3} \times \frac{5}{50}\\
&=\frac{11}{75}
\end{align}

\newpage

%\tableofcontents

\bigskip

\renewcommand{\thefigure}{\theenumi}
\renewcommand{\thetable}{\theenumi}
%\renewcommand{\theequation}{\theenumi}

%\begin{abstract}
%%\boldmath
%In this letter, an algorithm for evaluating the exact analytical bit error rate  (BER)  for the piecewise linear (PL) combiner for  multiple relays is presented. Previous results were available only for upto three relays. The algorithm is unique in the sense that  the actual mathematical expressions, that are prohibitively large, need not be explicitly obtained. The diversity gain due to multiple relays is shown through plots of the analytical BER, well supported by simulations. 
%
%\end{abstract}
% IEEEtran.cls defaults to using nonbold math in the Abstract.
% This preserves the distinction between vectors and scalars. However,
% if the journal you are submitting to favors bold math in the abstract,
% then you can use LaTeX's standard command \boldmath at the very start
% of the abstract to achieve this. Many IEEE journals frown on math
% in the abstract anyway.

% Note that keywords are not normally used for peerreview papers.
%\begin{IEEEkeywords}
%Cooperative diversity, decode and forward, piecewise linear
%\end{IEEEkeywords}



% For peer review papers, you can put extra information on the cover
% page as needed:
% \ifCLASSOPTIONpeerreview
% \begin{center} \bfseries EDICS Category: 3-BBND \end{center}
% \fi
%
% For peerreview papers, this IEEEtran command inserts a page break and
% creates the second title. It will be ignored for other modes.
%\IEEEpeerreviewmaketitle




\item A team of medical students doing their internship have to assist during surgeries
at a city hospital. The probabilities of surgeries rated as very complex, complex,
routine, simple or very simple are respectively, 0.15, 0.20, 0.31, 0.26, .08. Find
the probabilities that a particular surgery will be rated
\begin{enumerate}
	\item complex or very complex;
	\item neither very complex nor very simple;
	\item routine or complex
	\item routine or simple
\end{enumerate}
\solution
%\begin{table}[H]
	\centering
\begin{tabular}{|c|c|c|}
\hline
Random variable &Value &Definition\\ \hline
\multirow{3}{*}{X} &0 &Slips of Rs 1\\
&1 &Slips of Rs 5\\
&2 &Slips of Rs 13\\ \hline
\multirow{2}{*}{Y} &0 &Box A\\
&1 &Box B\\\hline
\end{tabular}
\caption{}
\label{tab:Distribution}
\end{table}
See \tabref{tab:Distribution}.
\begin{align}
p_{Y}\brak{k}= \begin{cases} 
      \frac{1}{3} & {k=0} \\
      \frac{2}{3 }& {k=1} 
   \end{cases}
   \\
p_{Y|X}\brak{0|0} = \frac{19}{25}\, 
p_{Y|X}\brak{0|1} = \frac{6}{25}\,
p_{Y|X}\brak{1|0} = \frac{45}{50}\,
p_{Y|X}\brak{1|2} = \frac{5}{50}
\end{align}
The desired probability is the probability that a slip drawn at random is marked other than Rs 1,
\begin{align}
&=1-p_X\brak{0}\\
&= p_X(1) + p_X(2)
\end{align}
Using Bayes theorem,
\begin{align}
&= p_Y\brak{0} \times \pr{Y=0 | X=1} + p_Y\brak{1} \times \pr{Y=1|X=2}\\
&=\frac{1}{3} \times \frac{6}{25} + \frac{2}{3} \times \frac{5}{50}\\
&=\frac{11}{75}
\end{align}

\newpage

%\tableofcontents

\bigskip

\renewcommand{\thefigure}{\theenumi}
\renewcommand{\thetable}{\theenumi}
%\renewcommand{\theequation}{\theenumi}

%\begin{abstract}
%%\boldmath
%In this letter, an algorithm for evaluating the exact analytical bit error rate  (BER)  for the piecewise linear (PL) combiner for  multiple relays is presented. Previous results were available only for upto three relays. The algorithm is unique in the sense that  the actual mathematical expressions, that are prohibitively large, need not be explicitly obtained. The diversity gain due to multiple relays is shown through plots of the analytical BER, well supported by simulations. 
%
%\end{abstract}
% IEEEtran.cls defaults to using nonbold math in the Abstract.
% This preserves the distinction between vectors and scalars. However,
% if the journal you are submitting to favors bold math in the abstract,
% then you can use LaTeX's standard command \boldmath at the very start
% of the abstract to achieve this. Many IEEE journals frown on math
% in the abstract anyway.

% Note that keywords are not normally used for peerreview papers.
%\begin{IEEEkeywords}
%Cooperative diversity, decode and forward, piecewise linear
%\end{IEEEkeywords}



% For peer review papers, you can put extra information on the cover
% page as needed:
% \ifCLASSOPTIONpeerreview
% \begin{center} \bfseries EDICS Category: 3-BBND \end{center}
% \fi
%
% For peerreview papers, this IEEEtran command inserts a page break and
% creates the second title. It will be ignored for other modes.
%\IEEEpeerreviewmaketitle




\item A card is selected from a pack of 52 cards.
\begin{enumerate}[label=(\alph*)]
    \item How many points are there in the sample space?
    \item Calculate the probability that the card is an ace of spades.
    \item Calculate the probability that the card is (i) an ace and (ii) black card.
\end{enumerate}
\solution
%Let $X$ be an bernoulli rv defined as in \tabref{tab:exemplar/11/16/3/26}.  Then, 
\begin{equation}
    p =
        \frac{4}{11} 
\end{equation}
\begin{table}[H]
	\centering
	\input{exemplar/11/16/3/26/tables/Table2.tex}
	\caption{}
        \label{tab:exemplar/11/16/3/26}
\end{table}

\item The probability that a non leap year selected at random will contain 53 sundays.
\\
\solution
%\begin{table}[H]
	\centering
\begin{tabular}{|c|c|c|}
\hline
Random variable &Value &Definition\\ \hline
\multirow{3}{*}{X} &0 &Slips of Rs 1\\
&1 &Slips of Rs 5\\
&2 &Slips of Rs 13\\ \hline
\multirow{2}{*}{Y} &0 &Box A\\
&1 &Box B\\\hline
\end{tabular}
\caption{}
\label{tab:Distribution}
\end{table}
See \tabref{tab:Distribution}.
\begin{align}
p_{Y}\brak{k}= \begin{cases} 
      \frac{1}{3} & {k=0} \\
      \frac{2}{3 }& {k=1} 
   \end{cases}
   \\
p_{Y|X}\brak{0|0} = \frac{19}{25}\, 
p_{Y|X}\brak{0|1} = \frac{6}{25}\,
p_{Y|X}\brak{1|0} = \frac{45}{50}\,
p_{Y|X}\brak{1|2} = \frac{5}{50}
\end{align}
The desired probability is the probability that a slip drawn at random is marked other than Rs 1,
\begin{align}
&=1-p_X\brak{0}\\
&= p_X(1) + p_X(2)
\end{align}
Using Bayes theorem,
\begin{align}
&= p_Y\brak{0} \times \pr{Y=0 | X=1} + p_Y\brak{1} \times \pr{Y=1|X=2}\\
&=\frac{1}{3} \times \frac{6}{25} + \frac{2}{3} \times \frac{5}{50}\\
&=\frac{11}{75}
\end{align}

\newpage

%\tableofcontents

\bigskip

\renewcommand{\thefigure}{\theenumi}
\renewcommand{\thetable}{\theenumi}
%\renewcommand{\theequation}{\theenumi}

%\begin{abstract}
%%\boldmath
%In this letter, an algorithm for evaluating the exact analytical bit error rate  (BER)  for the piecewise linear (PL) combiner for  multiple relays is presented. Previous results were available only for upto three relays. The algorithm is unique in the sense that  the actual mathematical expressions, that are prohibitively large, need not be explicitly obtained. The diversity gain due to multiple relays is shown through plots of the analytical BER, well supported by simulations. 
%
%\end{abstract}
% IEEEtran.cls defaults to using nonbold math in the Abstract.
% This preserves the distinction between vectors and scalars. However,
% if the journal you are submitting to favors bold math in the abstract,
% then you can use LaTeX's standard command \boldmath at the very start
% of the abstract to achieve this. Many IEEE journals frown on math
% in the abstract anyway.

% Note that keywords are not normally used for peerreview papers.
%\begin{IEEEkeywords}
%Cooperative diversity, decode and forward, piecewise linear
%\end{IEEEkeywords}



% For peer review papers, you can put extra information on the cover
% page as needed:
% \ifCLASSOPTIONpeerreview
% \begin{center} \bfseries EDICS Category: 3-BBND \end{center}
% \fi
%
% For peerreview papers, this IEEEtran command inserts a page break and
% creates the second title. It will be ignored for other modes.
%\IEEEpeerreviewmaketitle




\item One of the four persons John, Rita, Aslam or Gurpreet will be promoted next
month. Consequently the sample space consists of four elementary outcomes
S = {John promoted, Rita promoted, Aslam promoted, Gurpreet promoted}
You are told that the chances of John’s promotion is same as that of Gurpreet,
Rita’s chances of promotion are twice as likely as Johns. Aslam’s chances are
four times that of John.
\begin{enumerate}
	\item Determine
	\begin{enumerate}
		\item P (John promoted)
		\item P (Rita promoted)
		\item P (Aslam promoted)
		\item P (Gurpreet promoted)
	\end{enumerate}
	\item If A = {John promoted or Gurpreet promoted}, find P (A).
\end{enumerate}
\solution
%\begin{table}[H]
	\centering
\begin{tabular}{|c|c|c|}
\hline
Random variable &Value &Definition\\ \hline
\multirow{3}{*}{X} &0 &Slips of Rs 1\\
&1 &Slips of Rs 5\\
&2 &Slips of Rs 13\\ \hline
\multirow{2}{*}{Y} &0 &Box A\\
&1 &Box B\\\hline
\end{tabular}
\caption{}
\label{tab:Distribution}
\end{table}
See \tabref{tab:Distribution}.
\begin{align}
p_{Y}\brak{k}= \begin{cases} 
      \frac{1}{3} & {k=0} \\
      \frac{2}{3 }& {k=1} 
   \end{cases}
   \\
p_{Y|X}\brak{0|0} = \frac{19}{25}\, 
p_{Y|X}\brak{0|1} = \frac{6}{25}\,
p_{Y|X}\brak{1|0} = \frac{45}{50}\,
p_{Y|X}\brak{1|2} = \frac{5}{50}
\end{align}
The desired probability is the probability that a slip drawn at random is marked other than Rs 1,
\begin{align}
&=1-p_X\brak{0}\\
&= p_X(1) + p_X(2)
\end{align}
Using Bayes theorem,
\begin{align}
&= p_Y\brak{0} \times \pr{Y=0 | X=1} + p_Y\brak{1} \times \pr{Y=1|X=2}\\
&=\frac{1}{3} \times \frac{6}{25} + \frac{2}{3} \times \frac{5}{50}\\
&=\frac{11}{75}
\end{align}

\newpage

%\tableofcontents

\bigskip

\renewcommand{\thefigure}{\theenumi}
\renewcommand{\thetable}{\theenumi}
%\renewcommand{\theequation}{\theenumi}

%\begin{abstract}
%%\boldmath
%In this letter, an algorithm for evaluating the exact analytical bit error rate  (BER)  for the piecewise linear (PL) combiner for  multiple relays is presented. Previous results were available only for upto three relays. The algorithm is unique in the sense that  the actual mathematical expressions, that are prohibitively large, need not be explicitly obtained. The diversity gain due to multiple relays is shown through plots of the analytical BER, well supported by simulations. 
%
%\end{abstract}
% IEEEtran.cls defaults to using nonbold math in the Abstract.
% This preserves the distinction between vectors and scalars. However,
% if the journal you are submitting to favors bold math in the abstract,
% then you can use LaTeX's standard command \boldmath at the very start
% of the abstract to achieve this. Many IEEE journals frown on math
% in the abstract anyway.

% Note that keywords are not normally used for peerreview papers.
%\begin{IEEEkeywords}
%Cooperative diversity, decode and forward, piecewise linear
%\end{IEEEkeywords}



% For peer review papers, you can put extra information on the cover
% page as needed:
% \ifCLASSOPTIONpeerreview
% \begin{center} \bfseries EDICS Category: 3-BBND \end{center}
% \fi
%
% For peerreview papers, this IEEEtran command inserts a page break and
% creates the second title. It will be ignored for other modes.
%\IEEEpeerreviewmaketitle




\item A card is drawn from a deck of 52 cards. Find the probability of getting a king or a heart or a red card.\\
\solution
%\begin{table}[H]
	\centering
\begin{tabular}{|c|c|c|}
\hline
Random variable &Value &Definition\\ \hline
\multirow{3}{*}{X} &0 &Slips of Rs 1\\
&1 &Slips of Rs 5\\
&2 &Slips of Rs 13\\ \hline
\multirow{2}{*}{Y} &0 &Box A\\
&1 &Box B\\\hline
\end{tabular}
\caption{}
\label{tab:Distribution}
\end{table}
See \tabref{tab:Distribution}.
\begin{align}
p_{Y}\brak{k}= \begin{cases} 
      \frac{1}{3} & {k=0} \\
      \frac{2}{3 }& {k=1} 
   \end{cases}
   \\
p_{Y|X}\brak{0|0} = \frac{19}{25}\, 
p_{Y|X}\brak{0|1} = \frac{6}{25}\,
p_{Y|X}\brak{1|0} = \frac{45}{50}\,
p_{Y|X}\brak{1|2} = \frac{5}{50}
\end{align}
The desired probability is the probability that a slip drawn at random is marked other than Rs 1,
\begin{align}
&=1-p_X\brak{0}\\
&= p_X(1) + p_X(2)
\end{align}
Using Bayes theorem,
\begin{align}
&= p_Y\brak{0} \times \pr{Y=0 | X=1} + p_Y\brak{1} \times \pr{Y=1|X=2}\\
&=\frac{1}{3} \times \frac{6}{25} + \frac{2}{3} \times \frac{5}{50}\\
&=\frac{11}{75}
\end{align}

\newpage

%\tableofcontents

\bigskip

\renewcommand{\thefigure}{\theenumi}
\renewcommand{\thetable}{\theenumi}
%\renewcommand{\theequation}{\theenumi}

%\begin{abstract}
%%\boldmath
%In this letter, an algorithm for evaluating the exact analytical bit error rate  (BER)  for the piecewise linear (PL) combiner for  multiple relays is presented. Previous results were available only for upto three relays. The algorithm is unique in the sense that  the actual mathematical expressions, that are prohibitively large, need not be explicitly obtained. The diversity gain due to multiple relays is shown through plots of the analytical BER, well supported by simulations. 
%
%\end{abstract}
% IEEEtran.cls defaults to using nonbold math in the Abstract.
% This preserves the distinction between vectors and scalars. However,
% if the journal you are submitting to favors bold math in the abstract,
% then you can use LaTeX's standard command \boldmath at the very start
% of the abstract to achieve this. Many IEEE journals frown on math
% in the abstract anyway.

% Note that keywords are not normally used for peerreview papers.
%\begin{IEEEkeywords}
%Cooperative diversity, decode and forward, piecewise linear
%\end{IEEEkeywords}



% For peer review papers, you can put extra information on the cover
% page as needed:
% \ifCLASSOPTIONpeerreview
% \begin{center} \bfseries EDICS Category: 3-BBND \end{center}
% \fi
%
% For peerreview papers, this IEEEtran command inserts a page break and
% creates the second title. It will be ignored for other modes.
%\IEEEpeerreviewmaketitle




\item The probability that a student will pass his examination is 0.73, the probability of
the student getting a compartment is 0.13, and the probability that the student will
either pass or get compartment is 0.96. State True or False.\\
\solution
%\begin{table}[H]
	\centering
\begin{tabular}{|c|c|c|}
\hline
Random variable &Value &Definition\\ \hline
\multirow{3}{*}{X} &0 &Slips of Rs 1\\
&1 &Slips of Rs 5\\
&2 &Slips of Rs 13\\ \hline
\multirow{2}{*}{Y} &0 &Box A\\
&1 &Box B\\\hline
\end{tabular}
\caption{}
\label{tab:Distribution}
\end{table}
See \tabref{tab:Distribution}.
\begin{align}
p_{Y}\brak{k}= \begin{cases} 
      \frac{1}{3} & {k=0} \\
      \frac{2}{3 }& {k=1} 
   \end{cases}
   \\
p_{Y|X}\brak{0|0} = \frac{19}{25}\, 
p_{Y|X}\brak{0|1} = \frac{6}{25}\,
p_{Y|X}\brak{1|0} = \frac{45}{50}\,
p_{Y|X}\brak{1|2} = \frac{5}{50}
\end{align}
The desired probability is the probability that a slip drawn at random is marked other than Rs 1,
\begin{align}
&=1-p_X\brak{0}\\
&= p_X(1) + p_X(2)
\end{align}
Using Bayes theorem,
\begin{align}
&= p_Y\brak{0} \times \pr{Y=0 | X=1} + p_Y\brak{1} \times \pr{Y=1|X=2}\\
&=\frac{1}{3} \times \frac{6}{25} + \frac{2}{3} \times \frac{5}{50}\\
&=\frac{11}{75}
\end{align}

\newpage

%\tableofcontents

\bigskip

\renewcommand{\thefigure}{\theenumi}
\renewcommand{\thetable}{\theenumi}
%\renewcommand{\theequation}{\theenumi}

%\begin{abstract}
%%\boldmath
%In this letter, an algorithm for evaluating the exact analytical bit error rate  (BER)  for the piecewise linear (PL) combiner for  multiple relays is presented. Previous results were available only for upto three relays. The algorithm is unique in the sense that  the actual mathematical expressions, that are prohibitively large, need not be explicitly obtained. The diversity gain due to multiple relays is shown through plots of the analytical BER, well supported by simulations. 
%
%\end{abstract}
% IEEEtran.cls defaults to using nonbold math in the Abstract.
% This preserves the distinction between vectors and scalars. However,
% if the journal you are submitting to favors bold math in the abstract,
% then you can use LaTeX's standard command \boldmath at the very start
% of the abstract to achieve this. Many IEEE journals frown on math
% in the abstract anyway.

% Note that keywords are not normally used for peerreview papers.
%\begin{IEEEkeywords}
%Cooperative diversity, decode and forward, piecewise linear
%\end{IEEEkeywords}



% For peer review papers, you can put extra information on the cover
% page as needed:
% \ifCLASSOPTIONpeerreview
% \begin{center} \bfseries EDICS Category: 3-BBND \end{center}
% \fi
%
% For peerreview papers, this IEEEtran command inserts a page break and
% creates the second title. It will be ignored for other modes.
%\IEEEpeerreviewmaketitle




\item A card is selected from a pack of 52 cards\\
\begin{enumerate}[label=(\alph*)]
\item How many points are there in the sample space?
\item Calculate the probability that the cards is an ace of spades.
\item Calculate the probability that the card is (i) an ace (ii)black card.\\
\end{enumerate}
%\input{ncert/11/16/3/4_1/Prob_4.tex}
\item In a non-leap year, the probability of having 53 tuesdays or 53 wednesdays is\\
\solution
%A non-leap year has a total of 365 days, and a week has 7 days.\\
So it can be expressed as 
\begin{align}
365\text{days} &=52\times 7+1 \text{day}
\end{align}
$\implies$ 52 tuesdays or wednesdays\\
Random variable X denotes the days of a week
\begin{align}
p_X\brak{k}&=\frac{1}{7}; \quad \brak{1<k<7}
\end{align}
So the probability of extra day being tuesday or wednesday is
\begin{align}
p_X\brak{3}+p_X\brak{4}&=\frac{1}{7}+\frac{1}{7}=\frac{2}{7}
\end{align}



\item There are 1000 sealed envelopes in a box, 10 of them contain a cash prize of
Rs 100 each, 100 of them contain a cash prize of Rs 50 each and 200 of them
contain a cash prize of Rs 10 each and rest do not contain any cash prize. If they
are well shuffled and an envelope is picked up out, what is the probability that it
contains no cash prize?\\
\solution
%\begin{table}[H]
	\centering
\begin{tabular}{|c|c|c|}
\hline
Random variable &Value &Definition\\ \hline
\multirow{3}{*}{X} &0 &Slips of Rs 1\\
&1 &Slips of Rs 5\\
&2 &Slips of Rs 13\\ \hline
\multirow{2}{*}{Y} &0 &Box A\\
&1 &Box B\\\hline
\end{tabular}
\caption{}
\label{tab:Distribution}
\end{table}
See \tabref{tab:Distribution}.
\begin{align}
p_{Y}\brak{k}= \begin{cases} 
      \frac{1}{3} & {k=0} \\
      \frac{2}{3 }& {k=1} 
   \end{cases}
   \\
p_{Y|X}\brak{0|0} = \frac{19}{25}\, 
p_{Y|X}\brak{0|1} = \frac{6}{25}\,
p_{Y|X}\brak{1|0} = \frac{45}{50}\,
p_{Y|X}\brak{1|2} = \frac{5}{50}
\end{align}
The desired probability is the probability that a slip drawn at random is marked other than Rs 1,
\begin{align}
&=1-p_X\brak{0}\\
&= p_X(1) + p_X(2)
\end{align}
Using Bayes theorem,
\begin{align}
&= p_Y\brak{0} \times \pr{Y=0 | X=1} + p_Y\brak{1} \times \pr{Y=1|X=2}\\
&=\frac{1}{3} \times \frac{6}{25} + \frac{2}{3} \times \frac{5}{50}\\
&=\frac{11}{75}
\end{align}

\newpage

%\tableofcontents

\bigskip

\renewcommand{\thefigure}{\theenumi}
\renewcommand{\thetable}{\theenumi}
%\renewcommand{\theequation}{\theenumi}

%\begin{abstract}
%%\boldmath
%In this letter, an algorithm for evaluating the exact analytical bit error rate  (BER)  for the piecewise linear (PL) combiner for  multiple relays is presented. Previous results were available only for upto three relays. The algorithm is unique in the sense that  the actual mathematical expressions, that are prohibitively large, need not be explicitly obtained. The diversity gain due to multiple relays is shown through plots of the analytical BER, well supported by simulations. 
%
%\end{abstract}
% IEEEtran.cls defaults to using nonbold math in the Abstract.
% This preserves the distinction between vectors and scalars. However,
% if the journal you are submitting to favors bold math in the abstract,
% then you can use LaTeX's standard command \boldmath at the very start
% of the abstract to achieve this. Many IEEE journals frown on math
% in the abstract anyway.

% Note that keywords are not normally used for peerreview papers.
%\begin{IEEEkeywords}
%Cooperative diversity, decode and forward, piecewise linear
%\end{IEEEkeywords}



% For peer review papers, you can put extra information on the cover
% page as needed:
% \ifCLASSOPTIONpeerreview
% \begin{center} \bfseries EDICS Category: 3-BBND \end{center}
% \fi
%
% For peerreview papers, this IEEEtran command inserts a page break and
% creates the second title. It will be ignored for other modes.
%\IEEEpeerreviewmaketitle




\item 
A die is thrown and a card is selected at random from a deck of 52 playing cards. The probability of getting an even number on the die and a spade card.\\
\solution
%\begin{table}[H]
	\centering
\begin{tabular}{|c|c|c|}
\hline
Random variable &Value &Definition\\ \hline
\multirow{3}{*}{X} &0 &Slips of Rs 1\\
&1 &Slips of Rs 5\\
&2 &Slips of Rs 13\\ \hline
\multirow{2}{*}{Y} &0 &Box A\\
&1 &Box B\\\hline
\end{tabular}
\caption{}
\label{tab:Distribution}
\end{table}
See \tabref{tab:Distribution}.
\begin{align}
p_{Y}\brak{k}= \begin{cases} 
      \frac{1}{3} & {k=0} \\
      \frac{2}{3 }& {k=1} 
   \end{cases}
   \\
p_{Y|X}\brak{0|0} = \frac{19}{25}\, 
p_{Y|X}\brak{0|1} = \frac{6}{25}\,
p_{Y|X}\brak{1|0} = \frac{45}{50}\,
p_{Y|X}\brak{1|2} = \frac{5}{50}
\end{align}
The desired probability is the probability that a slip drawn at random is marked other than Rs 1,
\begin{align}
&=1-p_X\brak{0}\\
&= p_X(1) + p_X(2)
\end{align}
Using Bayes theorem,
\begin{align}
&= p_Y\brak{0} \times \pr{Y=0 | X=1} + p_Y\brak{1} \times \pr{Y=1|X=2}\\
&=\frac{1}{3} \times \frac{6}{25} + \frac{2}{3} \times \frac{5}{50}\\
&=\frac{11}{75}
\end{align}

\newpage

%\tableofcontents

\bigskip

\renewcommand{\thefigure}{\theenumi}
\renewcommand{\thetable}{\theenumi}
%\renewcommand{\theequation}{\theenumi}

%\begin{abstract}
%%\boldmath
%In this letter, an algorithm for evaluating the exact analytical bit error rate  (BER)  for the piecewise linear (PL) combiner for  multiple relays is presented. Previous results were available only for upto three relays. The algorithm is unique in the sense that  the actual mathematical expressions, that are prohibitively large, need not be explicitly obtained. The diversity gain due to multiple relays is shown through plots of the analytical BER, well supported by simulations. 
%
%\end{abstract}
% IEEEtran.cls defaults to using nonbold math in the Abstract.
% This preserves the distinction between vectors and scalars. However,
% if the journal you are submitting to favors bold math in the abstract,
% then you can use LaTeX's standard command \boldmath at the very start
% of the abstract to achieve this. Many IEEE journals frown on math
% in the abstract anyway.

% Note that keywords are not normally used for peerreview papers.
%\begin{IEEEkeywords}
%Cooperative diversity, decode and forward, piecewise linear
%\end{IEEEkeywords}



% For peer review papers, you can put extra information on the cover
% page as needed:
% \ifCLASSOPTIONpeerreview
% \begin{center} \bfseries EDICS Category: 3-BBND \end{center}
% \fi
%
% For peerreview papers, this IEEEtran command inserts a page break and
% creates the second title. It will be ignored for other modes.
%\IEEEpeerreviewmaketitle




\item
If 4-digit numbers greater than 5,000 are randomly formed from the digits 0, 1, 3, 5, and 7, what is the probability of forming a number divisible by 5 when:
\begin{enumerate}
    \item The digits are repeated?
    \item The repetition of digits is not allowed?
\end{enumerate}
\solution
%\begin{table}[H]
	\centering
\begin{tabular}{|c|c|c|}
\hline
Random variable &Value &Definition\\ \hline
\multirow{3}{*}{X} &0 &Slips of Rs 1\\
&1 &Slips of Rs 5\\
&2 &Slips of Rs 13\\ \hline
\multirow{2}{*}{Y} &0 &Box A\\
&1 &Box B\\\hline
\end{tabular}
\caption{}
\label{tab:Distribution}
\end{table}
See \tabref{tab:Distribution}.
\begin{align}
p_{Y}\brak{k}= \begin{cases} 
      \frac{1}{3} & {k=0} \\
      \frac{2}{3 }& {k=1} 
   \end{cases}
   \\
p_{Y|X}\brak{0|0} = \frac{19}{25}\, 
p_{Y|X}\brak{0|1} = \frac{6}{25}\,
p_{Y|X}\brak{1|0} = \frac{45}{50}\,
p_{Y|X}\brak{1|2} = \frac{5}{50}
\end{align}
The desired probability is the probability that a slip drawn at random is marked other than Rs 1,
\begin{align}
&=1-p_X\brak{0}\\
&= p_X(1) + p_X(2)
\end{align}
Using Bayes theorem,
\begin{align}
&= p_Y\brak{0} \times \pr{Y=0 | X=1} + p_Y\brak{1} \times \pr{Y=1|X=2}\\
&=\frac{1}{3} \times \frac{6}{25} + \frac{2}{3} \times \frac{5}{50}\\
&=\frac{11}{75}
\end{align}

\newpage

%\tableofcontents

\bigskip

\renewcommand{\thefigure}{\theenumi}
\renewcommand{\thetable}{\theenumi}
%\renewcommand{\theequation}{\theenumi}

%\begin{abstract}
%%\boldmath
%In this letter, an algorithm for evaluating the exact analytical bit error rate  (BER)  for the piecewise linear (PL) combiner for  multiple relays is presented. Previous results were available only for upto three relays. The algorithm is unique in the sense that  the actual mathematical expressions, that are prohibitively large, need not be explicitly obtained. The diversity gain due to multiple relays is shown through plots of the analytical BER, well supported by simulations. 
%
%\end{abstract}
% IEEEtran.cls defaults to using nonbold math in the Abstract.
% This preserves the distinction between vectors and scalars. However,
% if the journal you are submitting to favors bold math in the abstract,
% then you can use LaTeX's standard command \boldmath at the very start
% of the abstract to achieve this. Many IEEE journals frown on math
% in the abstract anyway.

% Note that keywords are not normally used for peerreview papers.
%\begin{IEEEkeywords}
%Cooperative diversity, decode and forward, piecewise linear
%\end{IEEEkeywords}



% For peer review papers, you can put extra information on the cover
% page as needed:
% \ifCLASSOPTIONpeerreview
% \begin{center} \bfseries EDICS Category: 3-BBND \end{center}
% \fi
%
% For peerreview papers, this IEEEtran command inserts a page break and
% creates the second title. It will be ignored for other modes.
%\IEEEpeerreviewmaketitle




\item Consider the probability space $\brak{\Omega, \mathcal{G}, P}$ where $\Omega = [0,2]$ and $\mathcal{G} = \cbrak{\phi, \Omega, [0,1], (1,2]}$. Let $X$ and $Y$ be two functions on $\Omega$ defined as
\begin{align*}
    X(\omega) = 
    \begin{cases}
        1 & \text{if }\omega \in [0, 1]\\
        2 & \text{if }\omega \in (1, 2]
    \end{cases}
\end{align*}
and
\begin{align*}
    Y(\omega) = 
    \begin{cases}
        2 & \text{if }\omega \in [0, 1.5]\\
        3 & \text{if }\omega \in (1.5, 2].
    \end{cases}
\end{align*}
Then which one of the following statements is true?
\begin{enumerate}
    \item [(A)] $X$ is a random variable with respect to $\mathcal{G}$, but $Y$ is not a random variable with respect to $\mathcal{G}$.
    \item [(B)] $Y$ is a random variable with respect to $\mathcal{G}$, but $X$ is not a random variable with respect to $\mathcal{G}$.
    \item [(C)] Neither $X$ nor $Y$ is a random variable with respect to $\mathcal{G}$.
    \item [(D)] Both $X$ and $Y$ are random variables with respect to $\mathcal{G}$.
\end{enumerate} \hfill (GATE ST 2023)\\
\solution
%\begin{table}[H]
	\centering
\begin{tabular}{|c|c|c|}
\hline
Random variable &Value &Definition\\ \hline
\multirow{3}{*}{X} &0 &Slips of Rs 1\\
&1 &Slips of Rs 5\\
&2 &Slips of Rs 13\\ \hline
\multirow{2}{*}{Y} &0 &Box A\\
&1 &Box B\\\hline
\end{tabular}
\caption{}
\label{tab:Distribution}
\end{table}
See \tabref{tab:Distribution}.
\begin{align}
p_{Y}\brak{k}= \begin{cases} 
      \frac{1}{3} & {k=0} \\
      \frac{2}{3 }& {k=1} 
   \end{cases}
   \\
p_{Y|X}\brak{0|0} = \frac{19}{25}\, 
p_{Y|X}\brak{0|1} = \frac{6}{25}\,
p_{Y|X}\brak{1|0} = \frac{45}{50}\,
p_{Y|X}\brak{1|2} = \frac{5}{50}
\end{align}
The desired probability is the probability that a slip drawn at random is marked other than Rs 1,
\begin{align}
&=1-p_X\brak{0}\\
&= p_X(1) + p_X(2)
\end{align}
Using Bayes theorem,
\begin{align}
&= p_Y\brak{0} \times \pr{Y=0 | X=1} + p_Y\brak{1} \times \pr{Y=1|X=2}\\
&=\frac{1}{3} \times \frac{6}{25} + \frac{2}{3} \times \frac{5}{50}\\
&=\frac{11}{75}
\end{align}

\newpage

%\tableofcontents

\bigskip

\renewcommand{\thefigure}{\theenumi}
\renewcommand{\thetable}{\theenumi}
%\renewcommand{\theequation}{\theenumi}

%\begin{abstract}
%%\boldmath
%In this letter, an algorithm for evaluating the exact analytical bit error rate  (BER)  for the piecewise linear (PL) combiner for  multiple relays is presented. Previous results were available only for upto three relays. The algorithm is unique in the sense that  the actual mathematical expressions, that are prohibitively large, need not be explicitly obtained. The diversity gain due to multiple relays is shown through plots of the analytical BER, well supported by simulations. 
%
%\end{abstract}
% IEEEtran.cls defaults to using nonbold math in the Abstract.
% This preserves the distinction between vectors and scalars. However,
% if the journal you are submitting to favors bold math in the abstract,
% then you can use LaTeX's standard command \boldmath at the very start
% of the abstract to achieve this. Many IEEE journals frown on math
% in the abstract anyway.

% Note that keywords are not normally used for peerreview papers.
%\begin{IEEEkeywords}
%Cooperative diversity, decode and forward, piecewise linear
%\end{IEEEkeywords}



% For peer review papers, you can put extra information on the cover
% page as needed:
% \ifCLASSOPTIONpeerreview
% \begin{center} \bfseries EDICS Category: 3-BBND \end{center}
% \fi
%
% For peerreview papers, this IEEEtran command inserts a page break and
% creates the second title. It will be ignored for other modes.
%\IEEEpeerreviewmaketitle




	\item  A die is loaded in such a way that each odd number is twice as likely to occur as
each even number. Find $P(G)$, where $G$ is the event that a number greater than
3 occurs on a single roll of the die.
\\
\solution
		%\begin{table}[H]
	\centering
\begin{tabular}{|c|c|c|}
\hline
Random variable &Value &Definition\\ \hline
\multirow{3}{*}{X} &0 &Slips of Rs 1\\
&1 &Slips of Rs 5\\
&2 &Slips of Rs 13\\ \hline
\multirow{2}{*}{Y} &0 &Box A\\
&1 &Box B\\\hline
\end{tabular}
\caption{}
\label{tab:Distribution}
\end{table}
See \tabref{tab:Distribution}.
\begin{align}
p_{Y}\brak{k}= \begin{cases} 
      \frac{1}{3} & {k=0} \\
      \frac{2}{3 }& {k=1} 
   \end{cases}
   \\
p_{Y|X}\brak{0|0} = \frac{19}{25}\, 
p_{Y|X}\brak{0|1} = \frac{6}{25}\,
p_{Y|X}\brak{1|0} = \frac{45}{50}\,
p_{Y|X}\brak{1|2} = \frac{5}{50}
\end{align}
The desired probability is the probability that a slip drawn at random is marked other than Rs 1,
\begin{align}
&=1-p_X\brak{0}\\
&= p_X(1) + p_X(2)
\end{align}
Using Bayes theorem,
\begin{align}
&= p_Y\brak{0} \times \pr{Y=0 | X=1} + p_Y\brak{1} \times \pr{Y=1|X=2}\\
&=\frac{1}{3} \times \frac{6}{25} + \frac{2}{3} \times \frac{5}{50}\\
&=\frac{11}{75}
\end{align}

\newpage

%\tableofcontents

\bigskip

\renewcommand{\thefigure}{\theenumi}
\renewcommand{\thetable}{\theenumi}
%\renewcommand{\theequation}{\theenumi}

%\begin{abstract}
%%\boldmath
%In this letter, an algorithm for evaluating the exact analytical bit error rate  (BER)  for the piecewise linear (PL) combiner for  multiple relays is presented. Previous results were available only for upto three relays. The algorithm is unique in the sense that  the actual mathematical expressions, that are prohibitively large, need not be explicitly obtained. The diversity gain due to multiple relays is shown through plots of the analytical BER, well supported by simulations. 
%
%\end{abstract}
% IEEEtran.cls defaults to using nonbold math in the Abstract.
% This preserves the distinction between vectors and scalars. However,
% if the journal you are submitting to favors bold math in the abstract,
% then you can use LaTeX's standard command \boldmath at the very start
% of the abstract to achieve this. Many IEEE journals frown on math
% in the abstract anyway.

% Note that keywords are not normally used for peerreview papers.
%\begin{IEEEkeywords}
%Cooperative diversity, decode and forward, piecewise linear
%\end{IEEEkeywords}



% For peer review papers, you can put extra information on the cover
% page as needed:
% \ifCLASSOPTIONpeerreview
% \begin{center} \bfseries EDICS Category: 3-BBND \end{center}
% \fi
%
% For peerreview papers, this IEEEtran command inserts a page break and
% creates the second title. It will be ignored for other modes.
%\IEEEpeerreviewmaketitle




	\item All the jacks, queens and kings are removed from a deck of 52 playing cards. The remaining cards are well shuffled and then one card is drawn at random. Giving ace a value 1 similar value for other cards, find the probability that the card has a value 
		\begin{enumerate}
			\item 7
			\item greater than 7
			\item less than 7
		\end{enumerate}
		%Number of cards left after removing all jacks, queens and kings 
\begin{align}
N	= 52 - 4\times 3
	= 40
\end{align}
%\begin{table}[H]
%\def\arraystretch{1.2}
%\begin{tabular}{|c|c|c|}
%\hline
%	\textbf{Parameter} &\textbf{Value} &\textbf{Description}\\ \hline
%	$X$ &1-10 &Represents the value of the card picked \\ \hline
%\end{tabular}
%\end{table}
Let $1 \le X \le 10$ be the value of the card picked.  Then,
\begin{align}
	p_X(k) &= \Pr(X=k)\ \forall\ 1 \leq k \leq 10\\
	&= \frac{4\times 1}{40}\\
	&= \frac{1}{10}\\
	\therefore p_X(k) &= 
	\begin{cases}
		\frac{1}{10} & 1 \leq k \leq 10\\
		0 & \text{otherwise}
	\end{cases}
\end{align}
and
\begin{align}
	F_{X}(k) &= \sum_{m=0}^{k}p_{X}(m) \quad 1 \leq k \leq 10\\
	&= \frac{k}{10}\\
	\therefore F_{X}(k) &= 
	\begin{cases}
		0 & k \leq 0\\
		\frac{k}{10} & 1\leq k \leq 10\\
		1 & k > 10 
	\end{cases}
\end{align}
\begin{enumerate}
	\item Probability that card has value equal to 7 is
		\begin{align}
			 p_{X}(7)
			= \frac{1}{10}
		\end{align}
	\item Probability that card has value greater than 7 is
		\begin{align}
			1 - F_X(7)
			&= 1 - \frac{7}{10}
			\\
			&= \frac{3}{10}
		\end{align}
	\item Probability that card has value less than 7 is
		\begin{align}
			 F_{X}(6)
			=\frac{6}{10}
		\end{align}
\end{enumerate}

  \item A Lot consists of 48 mobile phones of which 42 are good, 3 have only minor defects and 3 have major defects.Varnika will buy a phone if it is good but the trader will only buy a mobile if it has no major defects. One phone is selected at random from the lot. What is the probability that it is
\begin{enumerate}
	\item acceptable to Varnika?
            \item acceptable to the trader?
\end{enumerate}
\solution
	%\begin{table}[H]
	\centering
\begin{tabular}{|c|c|c|}
\hline
Random variable &Value &Definition\\ \hline
\multirow{3}{*}{X} &0 &Slips of Rs 1\\
&1 &Slips of Rs 5\\
&2 &Slips of Rs 13\\ \hline
\multirow{2}{*}{Y} &0 &Box A\\
&1 &Box B\\\hline
\end{tabular}
\caption{}
\label{tab:Distribution}
\end{table}
See \tabref{tab:Distribution}.
\begin{align}
p_{Y}\brak{k}= \begin{cases} 
      \frac{1}{3} & {k=0} \\
      \frac{2}{3 }& {k=1} 
   \end{cases}
   \\
p_{Y|X}\brak{0|0} = \frac{19}{25}\, 
p_{Y|X}\brak{0|1} = \frac{6}{25}\,
p_{Y|X}\brak{1|0} = \frac{45}{50}\,
p_{Y|X}\brak{1|2} = \frac{5}{50}
\end{align}
The desired probability is the probability that a slip drawn at random is marked other than Rs 1,
\begin{align}
&=1-p_X\brak{0}\\
&= p_X(1) + p_X(2)
\end{align}
Using Bayes theorem,
\begin{align}
&= p_Y\brak{0} \times \pr{Y=0 | X=1} + p_Y\brak{1} \times \pr{Y=1|X=2}\\
&=\frac{1}{3} \times \frac{6}{25} + \frac{2}{3} \times \frac{5}{50}\\
&=\frac{11}{75}
\end{align}

\newpage

%\tableofcontents

\bigskip

\renewcommand{\thefigure}{\theenumi}
\renewcommand{\thetable}{\theenumi}
%\renewcommand{\theequation}{\theenumi}

%\begin{abstract}
%%\boldmath
%In this letter, an algorithm for evaluating the exact analytical bit error rate  (BER)  for the piecewise linear (PL) combiner for  multiple relays is presented. Previous results were available only for upto three relays. The algorithm is unique in the sense that  the actual mathematical expressions, that are prohibitively large, need not be explicitly obtained. The diversity gain due to multiple relays is shown through plots of the analytical BER, well supported by simulations. 
%
%\end{abstract}
% IEEEtran.cls defaults to using nonbold math in the Abstract.
% This preserves the distinction between vectors and scalars. However,
% if the journal you are submitting to favors bold math in the abstract,
% then you can use LaTeX's standard command \boldmath at the very start
% of the abstract to achieve this. Many IEEE journals frown on math
% in the abstract anyway.

% Note that keywords are not normally used for peerreview papers.
%\begin{IEEEkeywords}
%Cooperative diversity, decode and forward, piecewise linear
%\end{IEEEkeywords}



% For peer review papers, you can put extra information on the cover
% page as needed:
% \ifCLASSOPTIONpeerreview
% \begin{center} \bfseries EDICS Category: 3-BBND \end{center}
% \fi
%
% For peerreview papers, this IEEEtran command inserts a page break and
% creates the second title. It will be ignored for other modes.
%\IEEEpeerreviewmaketitle




 \item A student says that if you throw a die, it will show up 1 or not 1. Therefore, the probability of getting 1 and the probability of getting 'not 1' each is equal to $\frac{1}{2}$. Is this correct? Give reasons.\\
 \solution
        %\begin{table}[H]
	\centering
\begin{tabular}{|c|c|c|}
\hline
Random variable &Value &Definition\\ \hline
\multirow{3}{*}{X} &0 &Slips of Rs 1\\
&1 &Slips of Rs 5\\
&2 &Slips of Rs 13\\ \hline
\multirow{2}{*}{Y} &0 &Box A\\
&1 &Box B\\\hline
\end{tabular}
\caption{}
\label{tab:Distribution}
\end{table}
See \tabref{tab:Distribution}.
\begin{align}
p_{Y}\brak{k}= \begin{cases} 
      \frac{1}{3} & {k=0} \\
      \frac{2}{3 }& {k=1} 
   \end{cases}
   \\
p_{Y|X}\brak{0|0} = \frac{19}{25}\, 
p_{Y|X}\brak{0|1} = \frac{6}{25}\,
p_{Y|X}\brak{1|0} = \frac{45}{50}\,
p_{Y|X}\brak{1|2} = \frac{5}{50}
\end{align}
The desired probability is the probability that a slip drawn at random is marked other than Rs 1,
\begin{align}
&=1-p_X\brak{0}\\
&= p_X(1) + p_X(2)
\end{align}
Using Bayes theorem,
\begin{align}
&= p_Y\brak{0} \times \pr{Y=0 | X=1} + p_Y\brak{1} \times \pr{Y=1|X=2}\\
&=\frac{1}{3} \times \frac{6}{25} + \frac{2}{3} \times \frac{5}{50}\\
&=\frac{11}{75}
\end{align}

\newpage

%\tableofcontents

\bigskip

\renewcommand{\thefigure}{\theenumi}
\renewcommand{\thetable}{\theenumi}
%\renewcommand{\theequation}{\theenumi}

%\begin{abstract}
%%\boldmath
%In this letter, an algorithm for evaluating the exact analytical bit error rate  (BER)  for the piecewise linear (PL) combiner for  multiple relays is presented. Previous results were available only for upto three relays. The algorithm is unique in the sense that  the actual mathematical expressions, that are prohibitively large, need not be explicitly obtained. The diversity gain due to multiple relays is shown through plots of the analytical BER, well supported by simulations. 
%
%\end{abstract}
% IEEEtran.cls defaults to using nonbold math in the Abstract.
% This preserves the distinction between vectors and scalars. However,
% if the journal you are submitting to favors bold math in the abstract,
% then you can use LaTeX's standard command \boldmath at the very start
% of the abstract to achieve this. Many IEEE journals frown on math
% in the abstract anyway.

% Note that keywords are not normally used for peerreview papers.
%\begin{IEEEkeywords}
%Cooperative diversity, decode and forward, piecewise linear
%\end{IEEEkeywords}



% For peer review papers, you can put extra information on the cover
% page as needed:
% \ifCLASSOPTIONpeerreview
% \begin{center} \bfseries EDICS Category: 3-BBND \end{center}
% \fi
%
% For peerreview papers, this IEEEtran command inserts a page break and
% creates the second title. It will be ignored for other modes.
%\IEEEpeerreviewmaketitle




   \item Four candidates A, B, C, D have ap-
plied for the assignment to coach a school cricket
team. If A is twice as likely to be selected as B, and
B and C are given about the same chance of being
selected, while C is twice as likely to be selected
as D, what are the probabilities that
\begin{enumerate}
\item C will be selected?
\item A will not be selected?
\end{enumerate}
	%\begin{table}[H]
	\centering
\begin{tabular}{|c|c|c|}
\hline
Random variable &Value &Definition\\ \hline
\multirow{3}{*}{X} &0 &Slips of Rs 1\\
&1 &Slips of Rs 5\\
&2 &Slips of Rs 13\\ \hline
\multirow{2}{*}{Y} &0 &Box A\\
&1 &Box B\\\hline
\end{tabular}
\caption{}
\label{tab:Distribution}
\end{table}
See \tabref{tab:Distribution}.
\begin{align}
p_{Y}\brak{k}= \begin{cases} 
      \frac{1}{3} & {k=0} \\
      \frac{2}{3 }& {k=1} 
   \end{cases}
   \\
p_{Y|X}\brak{0|0} = \frac{19}{25}\, 
p_{Y|X}\brak{0|1} = \frac{6}{25}\,
p_{Y|X}\brak{1|0} = \frac{45}{50}\,
p_{Y|X}\brak{1|2} = \frac{5}{50}
\end{align}
The desired probability is the probability that a slip drawn at random is marked other than Rs 1,
\begin{align}
&=1-p_X\brak{0}\\
&= p_X(1) + p_X(2)
\end{align}
Using Bayes theorem,
\begin{align}
&= p_Y\brak{0} \times \pr{Y=0 | X=1} + p_Y\brak{1} \times \pr{Y=1|X=2}\\
&=\frac{1}{3} \times \frac{6}{25} + \frac{2}{3} \times \frac{5}{50}\\
&=\frac{11}{75}
\end{align}

\newpage

%\tableofcontents

\bigskip

\renewcommand{\thefigure}{\theenumi}
\renewcommand{\thetable}{\theenumi}
%\renewcommand{\theequation}{\theenumi}

%\begin{abstract}
%%\boldmath
%In this letter, an algorithm for evaluating the exact analytical bit error rate  (BER)  for the piecewise linear (PL) combiner for  multiple relays is presented. Previous results were available only for upto three relays. The algorithm is unique in the sense that  the actual mathematical expressions, that are prohibitively large, need not be explicitly obtained. The diversity gain due to multiple relays is shown through plots of the analytical BER, well supported by simulations. 
%
%\end{abstract}
% IEEEtran.cls defaults to using nonbold math in the Abstract.
% This preserves the distinction between vectors and scalars. However,
% if the journal you are submitting to favors bold math in the abstract,
% then you can use LaTeX's standard command \boldmath at the very start
% of the abstract to achieve this. Many IEEE journals frown on math
% in the abstract anyway.

% Note that keywords are not normally used for peerreview papers.
%\begin{IEEEkeywords}
%Cooperative diversity, decode and forward, piecewise linear
%\end{IEEEkeywords}



% For peer review papers, you can put extra information on the cover
% page as needed:
% \ifCLASSOPTIONpeerreview
% \begin{center} \bfseries EDICS Category: 3-BBND \end{center}
% \fi
%
% For peerreview papers, this IEEEtran command inserts a page break and
% creates the second title. It will be ignored for other modes.
%\IEEEpeerreviewmaketitle




 \item A bag contain 24 balls of which $x$ balls are red, $2x$ are white and $3x$ are blue. A ball is selected at random, What is the probability that it is
\begin{enumerate}[label=\alph*)]
\item not red ?
\item white ?
\end{enumerate}
%\begin{table}[H]
	\centering
\begin{tabular}{|c|c|c|}
\hline
Random variable &Value &Definition\\ \hline
\multirow{3}{*}{X} &0 &Slips of Rs 1\\
&1 &Slips of Rs 5\\
&2 &Slips of Rs 13\\ \hline
\multirow{2}{*}{Y} &0 &Box A\\
&1 &Box B\\\hline
\end{tabular}
\caption{}
\label{tab:Distribution}
\end{table}
See \tabref{tab:Distribution}.
\begin{align}
p_{Y}\brak{k}= \begin{cases} 
      \frac{1}{3} & {k=0} \\
      \frac{2}{3 }& {k=1} 
   \end{cases}
   \\
p_{Y|X}\brak{0|0} = \frac{19}{25}\, 
p_{Y|X}\brak{0|1} = \frac{6}{25}\,
p_{Y|X}\brak{1|0} = \frac{45}{50}\,
p_{Y|X}\brak{1|2} = \frac{5}{50}
\end{align}
The desired probability is the probability that a slip drawn at random is marked other than Rs 1,
\begin{align}
&=1-p_X\brak{0}\\
&= p_X(1) + p_X(2)
\end{align}
Using Bayes theorem,
\begin{align}
&= p_Y\brak{0} \times \pr{Y=0 | X=1} + p_Y\brak{1} \times \pr{Y=1|X=2}\\
&=\frac{1}{3} \times \frac{6}{25} + \frac{2}{3} \times \frac{5}{50}\\
&=\frac{11}{75}
\end{align}

\newpage

%\tableofcontents

\bigskip

\renewcommand{\thefigure}{\theenumi}
\renewcommand{\thetable}{\theenumi}
%\renewcommand{\theequation}{\theenumi}

%\begin{abstract}
%%\boldmath
%In this letter, an algorithm for evaluating the exact analytical bit error rate  (BER)  for the piecewise linear (PL) combiner for  multiple relays is presented. Previous results were available only for upto three relays. The algorithm is unique in the sense that  the actual mathematical expressions, that are prohibitively large, need not be explicitly obtained. The diversity gain due to multiple relays is shown through plots of the analytical BER, well supported by simulations. 
%
%\end{abstract}
% IEEEtran.cls defaults to using nonbold math in the Abstract.
% This preserves the distinction between vectors and scalars. However,
% if the journal you are submitting to favors bold math in the abstract,
% then you can use LaTeX's standard command \boldmath at the very start
% of the abstract to achieve this. Many IEEE journals frown on math
% in the abstract anyway.

% Note that keywords are not normally used for peerreview papers.
%\begin{IEEEkeywords}
%Cooperative diversity, decode and forward, piecewise linear
%\end{IEEEkeywords}



% For peer review papers, you can put extra information on the cover
% page as needed:
% \ifCLASSOPTIONpeerreview
% \begin{center} \bfseries EDICS Category: 3-BBND \end{center}
% \fi
%
% For peerreview papers, this IEEEtran command inserts a page break and
% creates the second title. It will be ignored for other modes.
%\IEEEpeerreviewmaketitle




If the letters of the word ASSASSINATION are arranged at random. Find the Probability that
\begin{enumerate}[label=(\alph*)]
\item Four $S's$ come consecutively in the word
\item Two  $I's$ and two $N's$ come together
\item All $A's$ are not coming together
\item No two $A's$ are coming together
\end{enumerate}
%\begin{table}[H]
	\centering
\begin{tabular}{|c|c|c|}
\hline
Random variable &Value &Definition\\ \hline
\multirow{3}{*}{X} &0 &Slips of Rs 1\\
&1 &Slips of Rs 5\\
&2 &Slips of Rs 13\\ \hline
\multirow{2}{*}{Y} &0 &Box A\\
&1 &Box B\\\hline
\end{tabular}
\caption{}
\label{tab:Distribution}
\end{table}
See \tabref{tab:Distribution}.
\begin{align}
p_{Y}\brak{k}= \begin{cases} 
      \frac{1}{3} & {k=0} \\
      \frac{2}{3 }& {k=1} 
   \end{cases}
   \\
p_{Y|X}\brak{0|0} = \frac{19}{25}\, 
p_{Y|X}\brak{0|1} = \frac{6}{25}\,
p_{Y|X}\brak{1|0} = \frac{45}{50}\,
p_{Y|X}\brak{1|2} = \frac{5}{50}
\end{align}
The desired probability is the probability that a slip drawn at random is marked other than Rs 1,
\begin{align}
&=1-p_X\brak{0}\\
&= p_X(1) + p_X(2)
\end{align}
Using Bayes theorem,
\begin{align}
&= p_Y\brak{0} \times \pr{Y=0 | X=1} + p_Y\brak{1} \times \pr{Y=1|X=2}\\
&=\frac{1}{3} \times \frac{6}{25} + \frac{2}{3} \times \frac{5}{50}\\
&=\frac{11}{75}
\end{align}

\newpage

%\tableofcontents

\bigskip

\renewcommand{\thefigure}{\theenumi}
\renewcommand{\thetable}{\theenumi}
%\renewcommand{\theequation}{\theenumi}

%\begin{abstract}
%%\boldmath
%In this letter, an algorithm for evaluating the exact analytical bit error rate  (BER)  for the piecewise linear (PL) combiner for  multiple relays is presented. Previous results were available only for upto three relays. The algorithm is unique in the sense that  the actual mathematical expressions, that are prohibitively large, need not be explicitly obtained. The diversity gain due to multiple relays is shown through plots of the analytical BER, well supported by simulations. 
%
%\end{abstract}
% IEEEtran.cls defaults to using nonbold math in the Abstract.
% This preserves the distinction between vectors and scalars. However,
% if the journal you are submitting to favors bold math in the abstract,
% then you can use LaTeX's standard command \boldmath at the very start
% of the abstract to achieve this. Many IEEE journals frown on math
% in the abstract anyway.

% Note that keywords are not normally used for peerreview papers.
%\begin{IEEEkeywords}
%Cooperative diversity, decode and forward, piecewise linear
%\end{IEEEkeywords}



% For peer review papers, you can put extra information on the cover
% page as needed:
% \ifCLASSOPTIONpeerreview
% \begin{center} \bfseries EDICS Category: 3-BBND \end{center}
% \fi
%
% For peerreview papers, this IEEEtran command inserts a page break and
% creates the second title. It will be ignored for other modes.
%\IEEEpeerreviewmaketitle




	\item One urn contains two black balls (labelled B1 and B2) and one white ball. A
	second urn contains one black ball and two white balls (labelled W1 and W2).
	Suppose the following experiment is performed. One of the two urns is chosen
	at random. Next a ball is randomly chosen from the urn. Then a second ball is
	chosen at random from the same urn without replacing the first ball.
	
	\begin{enumerate}
	\item What is the probability that two black balls are chosen?
	
	\item What is the probability that two balls of opposite colour are chosen?
	\end{enumerate}
	\solution
	%\begin{align}
    \label{eq:12.13.6.18.1}
	\because	\pr{A|B} &> \pr{A},\
\frac{\pr{AB}}{\pr{B}} > \pr{A}
\\
    \label{eq:12.13.6.18.2}
	\implies \pr{AB} &> \pr{A}\pr{B}
	\\
	\text{or, } \frac{\pr{AB}}{\pr{A}} &=\pr{B|A} > \pr{A}
\end{align}

\end{enumerate}

		%
\item 
Two cards are drawn at random and without replacement from a pack of 52 playing cards. Find the probability that both the cards are black.
\\
\solution
		%\begin{enumerate}[label=\thesection.\arabic*,ref=\thesection.\theenumi]
	\item One card is drawn from a well-shuffled deck of 52 cards. Find the probability of getting
\begin{enumerate}
\item A king of red colour 
\item A face card 
\item A red face card
\item The jack of hearts
\item A spade
\item The queen of diamonds

\end{enumerate}
\solution
		%\begin{table}[H]
	\centering
\begin{tabular}{|c|c|c|}
\hline
Random variable &Value &Definition\\ \hline
\multirow{3}{*}{X} &0 &Slips of Rs 1\\
&1 &Slips of Rs 5\\
&2 &Slips of Rs 13\\ \hline
\multirow{2}{*}{Y} &0 &Box A\\
&1 &Box B\\\hline
\end{tabular}
\caption{}
\label{tab:Distribution}
\end{table}
See \tabref{tab:Distribution}.
\begin{align}
p_{Y}\brak{k}= \begin{cases} 
      \frac{1}{3} & {k=0} \\
      \frac{2}{3 }& {k=1} 
   \end{cases}
   \\
p_{Y|X}\brak{0|0} = \frac{19}{25}\, 
p_{Y|X}\brak{0|1} = \frac{6}{25}\,
p_{Y|X}\brak{1|0} = \frac{45}{50}\,
p_{Y|X}\brak{1|2} = \frac{5}{50}
\end{align}
The desired probability is the probability that a slip drawn at random is marked other than Rs 1,
\begin{align}
&=1-p_X\brak{0}\\
&= p_X(1) + p_X(2)
\end{align}
Using Bayes theorem,
\begin{align}
&= p_Y\brak{0} \times \pr{Y=0 | X=1} + p_Y\brak{1} \times \pr{Y=1|X=2}\\
&=\frac{1}{3} \times \frac{6}{25} + \frac{2}{3} \times \frac{5}{50}\\
&=\frac{11}{75}
\end{align}

\newpage

%\tableofcontents

\bigskip

\renewcommand{\thefigure}{\theenumi}
\renewcommand{\thetable}{\theenumi}
%\renewcommand{\theequation}{\theenumi}

%\begin{abstract}
%%\boldmath
%In this letter, an algorithm for evaluating the exact analytical bit error rate  (BER)  for the piecewise linear (PL) combiner for  multiple relays is presented. Previous results were available only for upto three relays. The algorithm is unique in the sense that  the actual mathematical expressions, that are prohibitively large, need not be explicitly obtained. The diversity gain due to multiple relays is shown through plots of the analytical BER, well supported by simulations. 
%
%\end{abstract}
% IEEEtran.cls defaults to using nonbold math in the Abstract.
% This preserves the distinction between vectors and scalars. However,
% if the journal you are submitting to favors bold math in the abstract,
% then you can use LaTeX's standard command \boldmath at the very start
% of the abstract to achieve this. Many IEEE journals frown on math
% in the abstract anyway.

% Note that keywords are not normally used for peerreview papers.
%\begin{IEEEkeywords}
%Cooperative diversity, decode and forward, piecewise linear
%\end{IEEEkeywords}



% For peer review papers, you can put extra information on the cover
% page as needed:
% \ifCLASSOPTIONpeerreview
% \begin{center} \bfseries EDICS Category: 3-BBND \end{center}
% \fi
%
% For peerreview papers, this IEEEtran command inserts a page break and
% creates the second title. It will be ignored for other modes.
%\IEEEpeerreviewmaketitle




	\item Five cards—the ten, jack, queen, king and ace of diamonds, are well-shuffled with their face downwards. One card is then picked up at random.
\begin{enumerate}
\item
What is the probability that the card is the queen? 
\item
If the queen is drawn and put aside, what is the probability that the second card picked up is (a) an ace? (b) a queen?\\
\end{enumerate}
\solution
		%\begin{enumerate}[label=\thesection.\arabic*,ref=\thesection.\theenumi]
	\item One card is drawn from a well-shuffled deck of 52 cards. Find the probability of getting
\begin{enumerate}
\item A king of red colour 
\item A face card 
\item A red face card
\item The jack of hearts
\item A spade
\item The queen of diamonds

\end{enumerate}
\solution
		%\input{ncert/10/15/1/14/main.tex}
	\item Five cards—the ten, jack, queen, king and ace of diamonds, are well-shuffled with their face downwards. One card is then picked up at random.
\begin{enumerate}
\item
What is the probability that the card is the queen? 
\item
If the queen is drawn and put aside, what is the probability that the second card picked up is (a) an ace? (b) a queen?\\
\end{enumerate}
\solution
		%\input{ncert/10/15/1/15/defs.tex}
	\item A bag contains $5$ red balls and some blue balls. If the probability of drawing a blue ball is double that if a red ball, determine the number of blue balls in the bag. 
		\\
\solution
		%\input{ncert/10/15/2/3/defs.tex}
	\item A card is selected from a pack of 52 cards.
 \begin{enumerate}[label=(\alph*)] 
                 \item How many points are there in the sample space?
                 \item Calculate the probability that the card is an ace of spades.
                 \item Calculate the probability that the card is (i) an ace and (ii) black card.
 \end{enumerate}
\solution
		%\input{ncert/11/16/3/4/main.tex}
\item Four cards are drawn from a well-shuffled deck of 52 cards. What is the probability of obtaining 3 diamonds and one spade.
\\
\solution
		%\input{ncert/11/16/4/2/defs.tex}
\item In a certain lottery 10,000 tickets are sold and ten equal prizes are awarded. What is the probability of not getting a prize if you buy (a) one ticket (b) two tickets (c) 10 tickets ?	
\\
\solution
		%\input{ncert/11/16/4/4/defs.tex}
		%
\item 
Out of 100 students, two sections of 40 and 60 are formed. If you and your friend are among the 100 students, what is the probability that
\begin{enumerate}
\item you both enter the same section?
\item you both enter the different sections?
\end{enumerate}
\solution
		%\input{ncert/11/16/4/5/defs.tex}
	\item 
The number lock of a suitcase has 4 wheels each labelled with ten digits i.e. from 0 to 9.The lock opens with a sequence of four digits with no repeats.What is the probability of a person getting the right sequence to open the suitcase.
\\
\solution
		%\input{ncert/11/16/4/10/defs.tex}
		%
\item 
Two cards are drawn at random and without replacement from a pack of 52 playing cards. Find the probability that both the cards are black.
\\
\solution
		%\input{ncert/12/13/2/2/defs.tex}
		\item A box of oranges is inspected by examining three randomly selected oranges drawn without replacement. If all the three oranges are good, the box is approved for sale, otherwise, it is rejected. Find the probability that a box containing 15 oranges out of which 12 are good and 3 are bad ones will be approved for sale.
		\label{ncert/12/13/2/3/defs.tex}
		\item Two balls are drawn at random with replacement from a box containing 10 black and 8 red balls. Find the probability that
		\label{ncert/12/13/2/12}
\begin{enumerate}
\item both balls are red.
\item first ball is black and second is red.
\item one of them is black and other is red.
\end{enumerate}

\item In a hostel, 60\% of the students read Hindi newspaper, 40\% read English newspaper and 20\% read both Hindi and English newspapers. A student is selected at random.
		\label{ncert/12/13/2/15}
\begin{enumerate}
\item Find the probability that she reads neither Hindi nor English newspapers.
\item If she reads Hindi newspaper, find the probability that she reads English newspaper.
\item If she reads English newspaper, find the probability that she reads Hindi newspaper.\\
\end{enumerate}
\item The probability of obtaining an even prime number on each die, when a pair of dice is rolled is 
\begin{enumerate}
    \item $0$ 
    
    \item $\frac{1}{3}$ 
    
    \item $\frac{1}{12}$ 
    
    \item $\frac{1}{36}$ 
\end{enumerate}
\solution
		%\input{ncert/12/13/2/17/defs.tex}
	\item A bag contains 4 red and 4 black balls, another bag contains 2 red and 6 black balls. One of the two bags is selected at random and a ball is drawn from the bag which is found to be red. Find the probability that the ball is drawn from the first bag.
\\
\solution
		%\input{ncert/12/13/3/2/main.tex}
  \item
  Cards with numbers 2 to 101 are placed in a box. A card is selected at random.Find the probability that the card has
\begin{enumerate}[label=(\roman*)]
	\item an even number 
	\item a square number
\end{enumerate}
\solution
%\input{exemplar/10/13/3/32/main.tex}
\item
The king, queen and jack of clubs are removed from a deck of 52 playing cards and then well shuffled. Now one card is drawn at random from the remaining cards.  Determine the probability that the card is
\begin{enumerate}[label=(\roman*)]
\item a club
\item 10 of hearts
\end{enumerate}
\solution
%\input{exemplar/10/13/3/29/main.tex}
\item A team of medical students doing their internship have to assist during surgeries
at a city hospital. The probabilities of surgeries rated as very complex, complex,
routine, simple or very simple are respectively, 0.15, 0.20, 0.31, 0.26, .08. Find
the probabilities that a particular surgery will be rated
\begin{enumerate}
	\item complex or very complex;
	\item neither very complex nor very simple;
	\item routine or complex
	\item routine or simple
\end{enumerate}
\solution
%\input{exemplar/11/16/3/8(1)/main.tex}
\item A card is selected from a pack of 52 cards.
\begin{enumerate}[label=(\alph*)]
    \item How many points are there in the sample space?
    \item Calculate the probability that the card is an ace of spades.
    \item Calculate the probability that the card is (i) an ace and (ii) black card.
\end{enumerate}
\solution
%\input{exemplar/11/16/3/4/main2.tex}
\item The probability that a non leap year selected at random will contain 53 sundays.
\\
\solution
%\input{exemplar/10/13/1/19/main.tex}
\item One of the four persons John, Rita, Aslam or Gurpreet will be promoted next
month. Consequently the sample space consists of four elementary outcomes
S = {John promoted, Rita promoted, Aslam promoted, Gurpreet promoted}
You are told that the chances of John’s promotion is same as that of Gurpreet,
Rita’s chances of promotion are twice as likely as Johns. Aslam’s chances are
four times that of John.
\begin{enumerate}
	\item Determine
	\begin{enumerate}
		\item P (John promoted)
		\item P (Rita promoted)
		\item P (Aslam promoted)
		\item P (Gurpreet promoted)
	\end{enumerate}
	\item If A = {John promoted or Gurpreet promoted}, find P (A).
\end{enumerate}
\solution
%\input{exemplar/11/16/3/10/main.tex}
\item A card is drawn from a deck of 52 cards. Find the probability of getting a king or a heart or a red card.\\
\solution
%\input{exemplar/11/16/3/15/main.tex}
\item The probability that a student will pass his examination is 0.73, the probability of
the student getting a compartment is 0.13, and the probability that the student will
either pass or get compartment is 0.96. State True or False.\\
\solution
%\input{exemplar/11/16/3/31/main.tex}
\item A card is selected from a pack of 52 cards\\
\begin{enumerate}[label=(\alph*)]
\item How many points are there in the sample space?
\item Calculate the probability that the cards is an ace of spades.
\item Calculate the probability that the card is (i) an ace (ii)black card.\\
\end{enumerate}
%\input{ncert/11/16/3/4_1/Prob_4.tex}
\item In a non-leap year, the probability of having 53 tuesdays or 53 wednesdays is\\
\solution
%\input{exemplar/11/16/3/18/main.tex}
\item There are 1000 sealed envelopes in a box, 10 of them contain a cash prize of
Rs 100 each, 100 of them contain a cash prize of Rs 50 each and 200 of them
contain a cash prize of Rs 10 each and rest do not contain any cash prize. If they
are well shuffled and an envelope is picked up out, what is the probability that it
contains no cash prize?\\
\solution
%\input{exemplar/10/13/3/34/main.tex}
\item 
A die is thrown and a card is selected at random from a deck of 52 playing cards. The probability of getting an even number on the die and a spade card.\\
\solution
%\input{exemplar/12/13/3/78/main.tex}
\item
If 4-digit numbers greater than 5,000 are randomly formed from the digits 0, 1, 3, 5, and 7, what is the probability of forming a number divisible by 5 when:
\begin{enumerate}
    \item The digits are repeated?
    \item The repetition of digits is not allowed?
\end{enumerate}
\solution
%\input{ncert/11/16/4/9/main.tex}
\item Consider the probability space $\brak{\Omega, \mathcal{G}, P}$ where $\Omega = [0,2]$ and $\mathcal{G} = \cbrak{\phi, \Omega, [0,1], (1,2]}$. Let $X$ and $Y$ be two functions on $\Omega$ defined as
\begin{align*}
    X(\omega) = 
    \begin{cases}
        1 & \text{if }\omega \in [0, 1]\\
        2 & \text{if }\omega \in (1, 2]
    \end{cases}
\end{align*}
and
\begin{align*}
    Y(\omega) = 
    \begin{cases}
        2 & \text{if }\omega \in [0, 1.5]\\
        3 & \text{if }\omega \in (1.5, 2].
    \end{cases}
\end{align*}
Then which one of the following statements is true?
\begin{enumerate}
    \item [(A)] $X$ is a random variable with respect to $\mathcal{G}$, but $Y$ is not a random variable with respect to $\mathcal{G}$.
    \item [(B)] $Y$ is a random variable with respect to $\mathcal{G}$, but $X$ is not a random variable with respect to $\mathcal{G}$.
    \item [(C)] Neither $X$ nor $Y$ is a random variable with respect to $\mathcal{G}$.
    \item [(D)] Both $X$ and $Y$ are random variables with respect to $\mathcal{G}$.
\end{enumerate} \hfill (GATE ST 2023)\\
\solution
%\input{gate/ST/2023/14/main.tex}
	\item  A die is loaded in such a way that each odd number is twice as likely to occur as
each even number. Find $P(G)$, where $G$ is the event that a number greater than
3 occurs on a single roll of the die.
\\
\solution
		%\input{exemplar/11/16/3/5/main.tex}
	\item All the jacks, queens and kings are removed from a deck of 52 playing cards. The remaining cards are well shuffled and then one card is drawn at random. Giving ace a value 1 similar value for other cards, find the probability that the card has a value 
		\begin{enumerate}
			\item 7
			\item greater than 7
			\item less than 7
		\end{enumerate}
		%\input{exemplar/10/13/3/30/main.tex}
  \item A Lot consists of 48 mobile phones of which 42 are good, 3 have only minor defects and 3 have major defects.Varnika will buy a phone if it is good but the trader will only buy a mobile if it has no major defects. One phone is selected at random from the lot. What is the probability that it is
\begin{enumerate}
	\item acceptable to Varnika?
            \item acceptable to the trader?
\end{enumerate}
\solution
	%\input{exemplar/10/13/3/40/main.tex}
 \item A student says that if you throw a die, it will show up 1 or not 1. Therefore, the probability of getting 1 and the probability of getting 'not 1' each is equal to $\frac{1}{2}$. Is this correct? Give reasons.\\
 \solution
        %\input{exemplar/10/13/2/9/main.tex}
   \item Four candidates A, B, C, D have ap-
plied for the assignment to coach a school cricket
team. If A is twice as likely to be selected as B, and
B and C are given about the same chance of being
selected, while C is twice as likely to be selected
as D, what are the probabilities that
\begin{enumerate}
\item C will be selected?
\item A will not be selected?
\end{enumerate}
	%\input{exemplar/11/16/3/9/main.tex}
 \item A bag contain 24 balls of which $x$ balls are red, $2x$ are white and $3x$ are blue. A ball is selected at random, What is the probability that it is
\begin{enumerate}[label=\alph*)]
\item not red ?
\item white ?
\end{enumerate}
%\input{exemplar/10/13/3/41/main.tex}
If the letters of the word ASSASSINATION are arranged at random. Find the Probability that
\begin{enumerate}[label=(\alph*)]
\item Four $S's$ come consecutively in the word
\item Two  $I's$ and two $N's$ come together
\item All $A's$ are not coming together
\item No two $A's$ are coming together
\end{enumerate}
%\input{exemplar/11/16/3/14/main.tex}
	\item One urn contains two black balls (labelled B1 and B2) and one white ball. A
	second urn contains one black ball and two white balls (labelled W1 and W2).
	Suppose the following experiment is performed. One of the two urns is chosen
	at random. Next a ball is randomly chosen from the urn. Then a second ball is
	chosen at random from the same urn without replacing the first ball.
	
	\begin{enumerate}
	\item What is the probability that two black balls are chosen?
	
	\item What is the probability that two balls of opposite colour are chosen?
	\end{enumerate}
	\solution
	%\input{exemplar/11/16/3/12/main1.tex}
\end{enumerate}

	\item A bag contains $5$ red balls and some blue balls. If the probability of drawing a blue ball is double that if a red ball, determine the number of blue balls in the bag. 
		\\
\solution
		%\begin{enumerate}[label=\thesection.\arabic*,ref=\thesection.\theenumi]
	\item One card is drawn from a well-shuffled deck of 52 cards. Find the probability of getting
\begin{enumerate}
\item A king of red colour 
\item A face card 
\item A red face card
\item The jack of hearts
\item A spade
\item The queen of diamonds

\end{enumerate}
\solution
		%\input{ncert/10/15/1/14/main.tex}
	\item Five cards—the ten, jack, queen, king and ace of diamonds, are well-shuffled with their face downwards. One card is then picked up at random.
\begin{enumerate}
\item
What is the probability that the card is the queen? 
\item
If the queen is drawn and put aside, what is the probability that the second card picked up is (a) an ace? (b) a queen?\\
\end{enumerate}
\solution
		%\input{ncert/10/15/1/15/defs.tex}
	\item A bag contains $5$ red balls and some blue balls. If the probability of drawing a blue ball is double that if a red ball, determine the number of blue balls in the bag. 
		\\
\solution
		%\input{ncert/10/15/2/3/defs.tex}
	\item A card is selected from a pack of 52 cards.
 \begin{enumerate}[label=(\alph*)] 
                 \item How many points are there in the sample space?
                 \item Calculate the probability that the card is an ace of spades.
                 \item Calculate the probability that the card is (i) an ace and (ii) black card.
 \end{enumerate}
\solution
		%\input{ncert/11/16/3/4/main.tex}
\item Four cards are drawn from a well-shuffled deck of 52 cards. What is the probability of obtaining 3 diamonds and one spade.
\\
\solution
		%\input{ncert/11/16/4/2/defs.tex}
\item In a certain lottery 10,000 tickets are sold and ten equal prizes are awarded. What is the probability of not getting a prize if you buy (a) one ticket (b) two tickets (c) 10 tickets ?	
\\
\solution
		%\input{ncert/11/16/4/4/defs.tex}
		%
\item 
Out of 100 students, two sections of 40 and 60 are formed. If you and your friend are among the 100 students, what is the probability that
\begin{enumerate}
\item you both enter the same section?
\item you both enter the different sections?
\end{enumerate}
\solution
		%\input{ncert/11/16/4/5/defs.tex}
	\item 
The number lock of a suitcase has 4 wheels each labelled with ten digits i.e. from 0 to 9.The lock opens with a sequence of four digits with no repeats.What is the probability of a person getting the right sequence to open the suitcase.
\\
\solution
		%\input{ncert/11/16/4/10/defs.tex}
		%
\item 
Two cards are drawn at random and without replacement from a pack of 52 playing cards. Find the probability that both the cards are black.
\\
\solution
		%\input{ncert/12/13/2/2/defs.tex}
		\item A box of oranges is inspected by examining three randomly selected oranges drawn without replacement. If all the three oranges are good, the box is approved for sale, otherwise, it is rejected. Find the probability that a box containing 15 oranges out of which 12 are good and 3 are bad ones will be approved for sale.
		\label{ncert/12/13/2/3/defs.tex}
		\item Two balls are drawn at random with replacement from a box containing 10 black and 8 red balls. Find the probability that
		\label{ncert/12/13/2/12}
\begin{enumerate}
\item both balls are red.
\item first ball is black and second is red.
\item one of them is black and other is red.
\end{enumerate}

\item In a hostel, 60\% of the students read Hindi newspaper, 40\% read English newspaper and 20\% read both Hindi and English newspapers. A student is selected at random.
		\label{ncert/12/13/2/15}
\begin{enumerate}
\item Find the probability that she reads neither Hindi nor English newspapers.
\item If she reads Hindi newspaper, find the probability that she reads English newspaper.
\item If she reads English newspaper, find the probability that she reads Hindi newspaper.\\
\end{enumerate}
\item The probability of obtaining an even prime number on each die, when a pair of dice is rolled is 
\begin{enumerate}
    \item $0$ 
    
    \item $\frac{1}{3}$ 
    
    \item $\frac{1}{12}$ 
    
    \item $\frac{1}{36}$ 
\end{enumerate}
\solution
		%\input{ncert/12/13/2/17/defs.tex}
	\item A bag contains 4 red and 4 black balls, another bag contains 2 red and 6 black balls. One of the two bags is selected at random and a ball is drawn from the bag which is found to be red. Find the probability that the ball is drawn from the first bag.
\\
\solution
		%\input{ncert/12/13/3/2/main.tex}
  \item
  Cards with numbers 2 to 101 are placed in a box. A card is selected at random.Find the probability that the card has
\begin{enumerate}[label=(\roman*)]
	\item an even number 
	\item a square number
\end{enumerate}
\solution
%\input{exemplar/10/13/3/32/main.tex}
\item
The king, queen and jack of clubs are removed from a deck of 52 playing cards and then well shuffled. Now one card is drawn at random from the remaining cards.  Determine the probability that the card is
\begin{enumerate}[label=(\roman*)]
\item a club
\item 10 of hearts
\end{enumerate}
\solution
%\input{exemplar/10/13/3/29/main.tex}
\item A team of medical students doing their internship have to assist during surgeries
at a city hospital. The probabilities of surgeries rated as very complex, complex,
routine, simple or very simple are respectively, 0.15, 0.20, 0.31, 0.26, .08. Find
the probabilities that a particular surgery will be rated
\begin{enumerate}
	\item complex or very complex;
	\item neither very complex nor very simple;
	\item routine or complex
	\item routine or simple
\end{enumerate}
\solution
%\input{exemplar/11/16/3/8(1)/main.tex}
\item A card is selected from a pack of 52 cards.
\begin{enumerate}[label=(\alph*)]
    \item How many points are there in the sample space?
    \item Calculate the probability that the card is an ace of spades.
    \item Calculate the probability that the card is (i) an ace and (ii) black card.
\end{enumerate}
\solution
%\input{exemplar/11/16/3/4/main2.tex}
\item The probability that a non leap year selected at random will contain 53 sundays.
\\
\solution
%\input{exemplar/10/13/1/19/main.tex}
\item One of the four persons John, Rita, Aslam or Gurpreet will be promoted next
month. Consequently the sample space consists of four elementary outcomes
S = {John promoted, Rita promoted, Aslam promoted, Gurpreet promoted}
You are told that the chances of John’s promotion is same as that of Gurpreet,
Rita’s chances of promotion are twice as likely as Johns. Aslam’s chances are
four times that of John.
\begin{enumerate}
	\item Determine
	\begin{enumerate}
		\item P (John promoted)
		\item P (Rita promoted)
		\item P (Aslam promoted)
		\item P (Gurpreet promoted)
	\end{enumerate}
	\item If A = {John promoted or Gurpreet promoted}, find P (A).
\end{enumerate}
\solution
%\input{exemplar/11/16/3/10/main.tex}
\item A card is drawn from a deck of 52 cards. Find the probability of getting a king or a heart or a red card.\\
\solution
%\input{exemplar/11/16/3/15/main.tex}
\item The probability that a student will pass his examination is 0.73, the probability of
the student getting a compartment is 0.13, and the probability that the student will
either pass or get compartment is 0.96. State True or False.\\
\solution
%\input{exemplar/11/16/3/31/main.tex}
\item A card is selected from a pack of 52 cards\\
\begin{enumerate}[label=(\alph*)]
\item How many points are there in the sample space?
\item Calculate the probability that the cards is an ace of spades.
\item Calculate the probability that the card is (i) an ace (ii)black card.\\
\end{enumerate}
%\input{ncert/11/16/3/4_1/Prob_4.tex}
\item In a non-leap year, the probability of having 53 tuesdays or 53 wednesdays is\\
\solution
%\input{exemplar/11/16/3/18/main.tex}
\item There are 1000 sealed envelopes in a box, 10 of them contain a cash prize of
Rs 100 each, 100 of them contain a cash prize of Rs 50 each and 200 of them
contain a cash prize of Rs 10 each and rest do not contain any cash prize. If they
are well shuffled and an envelope is picked up out, what is the probability that it
contains no cash prize?\\
\solution
%\input{exemplar/10/13/3/34/main.tex}
\item 
A die is thrown and a card is selected at random from a deck of 52 playing cards. The probability of getting an even number on the die and a spade card.\\
\solution
%\input{exemplar/12/13/3/78/main.tex}
\item
If 4-digit numbers greater than 5,000 are randomly formed from the digits 0, 1, 3, 5, and 7, what is the probability of forming a number divisible by 5 when:
\begin{enumerate}
    \item The digits are repeated?
    \item The repetition of digits is not allowed?
\end{enumerate}
\solution
%\input{ncert/11/16/4/9/main.tex}
\item Consider the probability space $\brak{\Omega, \mathcal{G}, P}$ where $\Omega = [0,2]$ and $\mathcal{G} = \cbrak{\phi, \Omega, [0,1], (1,2]}$. Let $X$ and $Y$ be two functions on $\Omega$ defined as
\begin{align*}
    X(\omega) = 
    \begin{cases}
        1 & \text{if }\omega \in [0, 1]\\
        2 & \text{if }\omega \in (1, 2]
    \end{cases}
\end{align*}
and
\begin{align*}
    Y(\omega) = 
    \begin{cases}
        2 & \text{if }\omega \in [0, 1.5]\\
        3 & \text{if }\omega \in (1.5, 2].
    \end{cases}
\end{align*}
Then which one of the following statements is true?
\begin{enumerate}
    \item [(A)] $X$ is a random variable with respect to $\mathcal{G}$, but $Y$ is not a random variable with respect to $\mathcal{G}$.
    \item [(B)] $Y$ is a random variable with respect to $\mathcal{G}$, but $X$ is not a random variable with respect to $\mathcal{G}$.
    \item [(C)] Neither $X$ nor $Y$ is a random variable with respect to $\mathcal{G}$.
    \item [(D)] Both $X$ and $Y$ are random variables with respect to $\mathcal{G}$.
\end{enumerate} \hfill (GATE ST 2023)\\
\solution
%\input{gate/ST/2023/14/main.tex}
	\item  A die is loaded in such a way that each odd number is twice as likely to occur as
each even number. Find $P(G)$, where $G$ is the event that a number greater than
3 occurs on a single roll of the die.
\\
\solution
		%\input{exemplar/11/16/3/5/main.tex}
	\item All the jacks, queens and kings are removed from a deck of 52 playing cards. The remaining cards are well shuffled and then one card is drawn at random. Giving ace a value 1 similar value for other cards, find the probability that the card has a value 
		\begin{enumerate}
			\item 7
			\item greater than 7
			\item less than 7
		\end{enumerate}
		%\input{exemplar/10/13/3/30/main.tex}
  \item A Lot consists of 48 mobile phones of which 42 are good, 3 have only minor defects and 3 have major defects.Varnika will buy a phone if it is good but the trader will only buy a mobile if it has no major defects. One phone is selected at random from the lot. What is the probability that it is
\begin{enumerate}
	\item acceptable to Varnika?
            \item acceptable to the trader?
\end{enumerate}
\solution
	%\input{exemplar/10/13/3/40/main.tex}
 \item A student says that if you throw a die, it will show up 1 or not 1. Therefore, the probability of getting 1 and the probability of getting 'not 1' each is equal to $\frac{1}{2}$. Is this correct? Give reasons.\\
 \solution
        %\input{exemplar/10/13/2/9/main.tex}
   \item Four candidates A, B, C, D have ap-
plied for the assignment to coach a school cricket
team. If A is twice as likely to be selected as B, and
B and C are given about the same chance of being
selected, while C is twice as likely to be selected
as D, what are the probabilities that
\begin{enumerate}
\item C will be selected?
\item A will not be selected?
\end{enumerate}
	%\input{exemplar/11/16/3/9/main.tex}
 \item A bag contain 24 balls of which $x$ balls are red, $2x$ are white and $3x$ are blue. A ball is selected at random, What is the probability that it is
\begin{enumerate}[label=\alph*)]
\item not red ?
\item white ?
\end{enumerate}
%\input{exemplar/10/13/3/41/main.tex}
If the letters of the word ASSASSINATION are arranged at random. Find the Probability that
\begin{enumerate}[label=(\alph*)]
\item Four $S's$ come consecutively in the word
\item Two  $I's$ and two $N's$ come together
\item All $A's$ are not coming together
\item No two $A's$ are coming together
\end{enumerate}
%\input{exemplar/11/16/3/14/main.tex}
	\item One urn contains two black balls (labelled B1 and B2) and one white ball. A
	second urn contains one black ball and two white balls (labelled W1 and W2).
	Suppose the following experiment is performed. One of the two urns is chosen
	at random. Next a ball is randomly chosen from the urn. Then a second ball is
	chosen at random from the same urn without replacing the first ball.
	
	\begin{enumerate}
	\item What is the probability that two black balls are chosen?
	
	\item What is the probability that two balls of opposite colour are chosen?
	\end{enumerate}
	\solution
	%\input{exemplar/11/16/3/12/main1.tex}
\end{enumerate}

	\item A card is selected from a pack of 52 cards.
 \begin{enumerate}[label=(\alph*)] 
                 \item How many points are there in the sample space?
                 \item Calculate the probability that the card is an ace of spades.
                 \item Calculate the probability that the card is (i) an ace and (ii) black card.
 \end{enumerate}
\solution
		%\begin{table}[H]
	\centering
\begin{tabular}{|c|c|c|}
\hline
Random variable &Value &Definition\\ \hline
\multirow{3}{*}{X} &0 &Slips of Rs 1\\
&1 &Slips of Rs 5\\
&2 &Slips of Rs 13\\ \hline
\multirow{2}{*}{Y} &0 &Box A\\
&1 &Box B\\\hline
\end{tabular}
\caption{}
\label{tab:Distribution}
\end{table}
See \tabref{tab:Distribution}.
\begin{align}
p_{Y}\brak{k}= \begin{cases} 
      \frac{1}{3} & {k=0} \\
      \frac{2}{3 }& {k=1} 
   \end{cases}
   \\
p_{Y|X}\brak{0|0} = \frac{19}{25}\, 
p_{Y|X}\brak{0|1} = \frac{6}{25}\,
p_{Y|X}\brak{1|0} = \frac{45}{50}\,
p_{Y|X}\brak{1|2} = \frac{5}{50}
\end{align}
The desired probability is the probability that a slip drawn at random is marked other than Rs 1,
\begin{align}
&=1-p_X\brak{0}\\
&= p_X(1) + p_X(2)
\end{align}
Using Bayes theorem,
\begin{align}
&= p_Y\brak{0} \times \pr{Y=0 | X=1} + p_Y\brak{1} \times \pr{Y=1|X=2}\\
&=\frac{1}{3} \times \frac{6}{25} + \frac{2}{3} \times \frac{5}{50}\\
&=\frac{11}{75}
\end{align}

\newpage

%\tableofcontents

\bigskip

\renewcommand{\thefigure}{\theenumi}
\renewcommand{\thetable}{\theenumi}
%\renewcommand{\theequation}{\theenumi}

%\begin{abstract}
%%\boldmath
%In this letter, an algorithm for evaluating the exact analytical bit error rate  (BER)  for the piecewise linear (PL) combiner for  multiple relays is presented. Previous results were available only for upto three relays. The algorithm is unique in the sense that  the actual mathematical expressions, that are prohibitively large, need not be explicitly obtained. The diversity gain due to multiple relays is shown through plots of the analytical BER, well supported by simulations. 
%
%\end{abstract}
% IEEEtran.cls defaults to using nonbold math in the Abstract.
% This preserves the distinction between vectors and scalars. However,
% if the journal you are submitting to favors bold math in the abstract,
% then you can use LaTeX's standard command \boldmath at the very start
% of the abstract to achieve this. Many IEEE journals frown on math
% in the abstract anyway.

% Note that keywords are not normally used for peerreview papers.
%\begin{IEEEkeywords}
%Cooperative diversity, decode and forward, piecewise linear
%\end{IEEEkeywords}



% For peer review papers, you can put extra information on the cover
% page as needed:
% \ifCLASSOPTIONpeerreview
% \begin{center} \bfseries EDICS Category: 3-BBND \end{center}
% \fi
%
% For peerreview papers, this IEEEtran command inserts a page break and
% creates the second title. It will be ignored for other modes.
%\IEEEpeerreviewmaketitle




\item Four cards are drawn from a well-shuffled deck of 52 cards. What is the probability of obtaining 3 diamonds and one spade.
\\
\solution
		%\begin{enumerate}[label=\thesection.\arabic*,ref=\thesection.\theenumi]
	\item One card is drawn from a well-shuffled deck of 52 cards. Find the probability of getting
\begin{enumerate}
\item A king of red colour 
\item A face card 
\item A red face card
\item The jack of hearts
\item A spade
\item The queen of diamonds

\end{enumerate}
\solution
		%\input{ncert/10/15/1/14/main.tex}
	\item Five cards—the ten, jack, queen, king and ace of diamonds, are well-shuffled with their face downwards. One card is then picked up at random.
\begin{enumerate}
\item
What is the probability that the card is the queen? 
\item
If the queen is drawn and put aside, what is the probability that the second card picked up is (a) an ace? (b) a queen?\\
\end{enumerate}
\solution
		%\input{ncert/10/15/1/15/defs.tex}
	\item A bag contains $5$ red balls and some blue balls. If the probability of drawing a blue ball is double that if a red ball, determine the number of blue balls in the bag. 
		\\
\solution
		%\input{ncert/10/15/2/3/defs.tex}
	\item A card is selected from a pack of 52 cards.
 \begin{enumerate}[label=(\alph*)] 
                 \item How many points are there in the sample space?
                 \item Calculate the probability that the card is an ace of spades.
                 \item Calculate the probability that the card is (i) an ace and (ii) black card.
 \end{enumerate}
\solution
		%\input{ncert/11/16/3/4/main.tex}
\item Four cards are drawn from a well-shuffled deck of 52 cards. What is the probability of obtaining 3 diamonds and one spade.
\\
\solution
		%\input{ncert/11/16/4/2/defs.tex}
\item In a certain lottery 10,000 tickets are sold and ten equal prizes are awarded. What is the probability of not getting a prize if you buy (a) one ticket (b) two tickets (c) 10 tickets ?	
\\
\solution
		%\input{ncert/11/16/4/4/defs.tex}
		%
\item 
Out of 100 students, two sections of 40 and 60 are formed. If you and your friend are among the 100 students, what is the probability that
\begin{enumerate}
\item you both enter the same section?
\item you both enter the different sections?
\end{enumerate}
\solution
		%\input{ncert/11/16/4/5/defs.tex}
	\item 
The number lock of a suitcase has 4 wheels each labelled with ten digits i.e. from 0 to 9.The lock opens with a sequence of four digits with no repeats.What is the probability of a person getting the right sequence to open the suitcase.
\\
\solution
		%\input{ncert/11/16/4/10/defs.tex}
		%
\item 
Two cards are drawn at random and without replacement from a pack of 52 playing cards. Find the probability that both the cards are black.
\\
\solution
		%\input{ncert/12/13/2/2/defs.tex}
		\item A box of oranges is inspected by examining three randomly selected oranges drawn without replacement. If all the three oranges are good, the box is approved for sale, otherwise, it is rejected. Find the probability that a box containing 15 oranges out of which 12 are good and 3 are bad ones will be approved for sale.
		\label{ncert/12/13/2/3/defs.tex}
		\item Two balls are drawn at random with replacement from a box containing 10 black and 8 red balls. Find the probability that
		\label{ncert/12/13/2/12}
\begin{enumerate}
\item both balls are red.
\item first ball is black and second is red.
\item one of them is black and other is red.
\end{enumerate}

\item In a hostel, 60\% of the students read Hindi newspaper, 40\% read English newspaper and 20\% read both Hindi and English newspapers. A student is selected at random.
		\label{ncert/12/13/2/15}
\begin{enumerate}
\item Find the probability that she reads neither Hindi nor English newspapers.
\item If she reads Hindi newspaper, find the probability that she reads English newspaper.
\item If she reads English newspaper, find the probability that she reads Hindi newspaper.\\
\end{enumerate}
\item The probability of obtaining an even prime number on each die, when a pair of dice is rolled is 
\begin{enumerate}
    \item $0$ 
    
    \item $\frac{1}{3}$ 
    
    \item $\frac{1}{12}$ 
    
    \item $\frac{1}{36}$ 
\end{enumerate}
\solution
		%\input{ncert/12/13/2/17/defs.tex}
	\item A bag contains 4 red and 4 black balls, another bag contains 2 red and 6 black balls. One of the two bags is selected at random and a ball is drawn from the bag which is found to be red. Find the probability that the ball is drawn from the first bag.
\\
\solution
		%\input{ncert/12/13/3/2/main.tex}
  \item
  Cards with numbers 2 to 101 are placed in a box. A card is selected at random.Find the probability that the card has
\begin{enumerate}[label=(\roman*)]
	\item an even number 
	\item a square number
\end{enumerate}
\solution
%\input{exemplar/10/13/3/32/main.tex}
\item
The king, queen and jack of clubs are removed from a deck of 52 playing cards and then well shuffled. Now one card is drawn at random from the remaining cards.  Determine the probability that the card is
\begin{enumerate}[label=(\roman*)]
\item a club
\item 10 of hearts
\end{enumerate}
\solution
%\input{exemplar/10/13/3/29/main.tex}
\item A team of medical students doing their internship have to assist during surgeries
at a city hospital. The probabilities of surgeries rated as very complex, complex,
routine, simple or very simple are respectively, 0.15, 0.20, 0.31, 0.26, .08. Find
the probabilities that a particular surgery will be rated
\begin{enumerate}
	\item complex or very complex;
	\item neither very complex nor very simple;
	\item routine or complex
	\item routine or simple
\end{enumerate}
\solution
%\input{exemplar/11/16/3/8(1)/main.tex}
\item A card is selected from a pack of 52 cards.
\begin{enumerate}[label=(\alph*)]
    \item How many points are there in the sample space?
    \item Calculate the probability that the card is an ace of spades.
    \item Calculate the probability that the card is (i) an ace and (ii) black card.
\end{enumerate}
\solution
%\input{exemplar/11/16/3/4/main2.tex}
\item The probability that a non leap year selected at random will contain 53 sundays.
\\
\solution
%\input{exemplar/10/13/1/19/main.tex}
\item One of the four persons John, Rita, Aslam or Gurpreet will be promoted next
month. Consequently the sample space consists of four elementary outcomes
S = {John promoted, Rita promoted, Aslam promoted, Gurpreet promoted}
You are told that the chances of John’s promotion is same as that of Gurpreet,
Rita’s chances of promotion are twice as likely as Johns. Aslam’s chances are
four times that of John.
\begin{enumerate}
	\item Determine
	\begin{enumerate}
		\item P (John promoted)
		\item P (Rita promoted)
		\item P (Aslam promoted)
		\item P (Gurpreet promoted)
	\end{enumerate}
	\item If A = {John promoted or Gurpreet promoted}, find P (A).
\end{enumerate}
\solution
%\input{exemplar/11/16/3/10/main.tex}
\item A card is drawn from a deck of 52 cards. Find the probability of getting a king or a heart or a red card.\\
\solution
%\input{exemplar/11/16/3/15/main.tex}
\item The probability that a student will pass his examination is 0.73, the probability of
the student getting a compartment is 0.13, and the probability that the student will
either pass or get compartment is 0.96. State True or False.\\
\solution
%\input{exemplar/11/16/3/31/main.tex}
\item A card is selected from a pack of 52 cards\\
\begin{enumerate}[label=(\alph*)]
\item How many points are there in the sample space?
\item Calculate the probability that the cards is an ace of spades.
\item Calculate the probability that the card is (i) an ace (ii)black card.\\
\end{enumerate}
%\input{ncert/11/16/3/4_1/Prob_4.tex}
\item In a non-leap year, the probability of having 53 tuesdays or 53 wednesdays is\\
\solution
%\input{exemplar/11/16/3/18/main.tex}
\item There are 1000 sealed envelopes in a box, 10 of them contain a cash prize of
Rs 100 each, 100 of them contain a cash prize of Rs 50 each and 200 of them
contain a cash prize of Rs 10 each and rest do not contain any cash prize. If they
are well shuffled and an envelope is picked up out, what is the probability that it
contains no cash prize?\\
\solution
%\input{exemplar/10/13/3/34/main.tex}
\item 
A die is thrown and a card is selected at random from a deck of 52 playing cards. The probability of getting an even number on the die and a spade card.\\
\solution
%\input{exemplar/12/13/3/78/main.tex}
\item
If 4-digit numbers greater than 5,000 are randomly formed from the digits 0, 1, 3, 5, and 7, what is the probability of forming a number divisible by 5 when:
\begin{enumerate}
    \item The digits are repeated?
    \item The repetition of digits is not allowed?
\end{enumerate}
\solution
%\input{ncert/11/16/4/9/main.tex}
\item Consider the probability space $\brak{\Omega, \mathcal{G}, P}$ where $\Omega = [0,2]$ and $\mathcal{G} = \cbrak{\phi, \Omega, [0,1], (1,2]}$. Let $X$ and $Y$ be two functions on $\Omega$ defined as
\begin{align*}
    X(\omega) = 
    \begin{cases}
        1 & \text{if }\omega \in [0, 1]\\
        2 & \text{if }\omega \in (1, 2]
    \end{cases}
\end{align*}
and
\begin{align*}
    Y(\omega) = 
    \begin{cases}
        2 & \text{if }\omega \in [0, 1.5]\\
        3 & \text{if }\omega \in (1.5, 2].
    \end{cases}
\end{align*}
Then which one of the following statements is true?
\begin{enumerate}
    \item [(A)] $X$ is a random variable with respect to $\mathcal{G}$, but $Y$ is not a random variable with respect to $\mathcal{G}$.
    \item [(B)] $Y$ is a random variable with respect to $\mathcal{G}$, but $X$ is not a random variable with respect to $\mathcal{G}$.
    \item [(C)] Neither $X$ nor $Y$ is a random variable with respect to $\mathcal{G}$.
    \item [(D)] Both $X$ and $Y$ are random variables with respect to $\mathcal{G}$.
\end{enumerate} \hfill (GATE ST 2023)\\
\solution
%\input{gate/ST/2023/14/main.tex}
	\item  A die is loaded in such a way that each odd number is twice as likely to occur as
each even number. Find $P(G)$, where $G$ is the event that a number greater than
3 occurs on a single roll of the die.
\\
\solution
		%\input{exemplar/11/16/3/5/main.tex}
	\item All the jacks, queens and kings are removed from a deck of 52 playing cards. The remaining cards are well shuffled and then one card is drawn at random. Giving ace a value 1 similar value for other cards, find the probability that the card has a value 
		\begin{enumerate}
			\item 7
			\item greater than 7
			\item less than 7
		\end{enumerate}
		%\input{exemplar/10/13/3/30/main.tex}
  \item A Lot consists of 48 mobile phones of which 42 are good, 3 have only minor defects and 3 have major defects.Varnika will buy a phone if it is good but the trader will only buy a mobile if it has no major defects. One phone is selected at random from the lot. What is the probability that it is
\begin{enumerate}
	\item acceptable to Varnika?
            \item acceptable to the trader?
\end{enumerate}
\solution
	%\input{exemplar/10/13/3/40/main.tex}
 \item A student says that if you throw a die, it will show up 1 or not 1. Therefore, the probability of getting 1 and the probability of getting 'not 1' each is equal to $\frac{1}{2}$. Is this correct? Give reasons.\\
 \solution
        %\input{exemplar/10/13/2/9/main.tex}
   \item Four candidates A, B, C, D have ap-
plied for the assignment to coach a school cricket
team. If A is twice as likely to be selected as B, and
B and C are given about the same chance of being
selected, while C is twice as likely to be selected
as D, what are the probabilities that
\begin{enumerate}
\item C will be selected?
\item A will not be selected?
\end{enumerate}
	%\input{exemplar/11/16/3/9/main.tex}
 \item A bag contain 24 balls of which $x$ balls are red, $2x$ are white and $3x$ are blue. A ball is selected at random, What is the probability that it is
\begin{enumerate}[label=\alph*)]
\item not red ?
\item white ?
\end{enumerate}
%\input{exemplar/10/13/3/41/main.tex}
If the letters of the word ASSASSINATION are arranged at random. Find the Probability that
\begin{enumerate}[label=(\alph*)]
\item Four $S's$ come consecutively in the word
\item Two  $I's$ and two $N's$ come together
\item All $A's$ are not coming together
\item No two $A's$ are coming together
\end{enumerate}
%\input{exemplar/11/16/3/14/main.tex}
	\item One urn contains two black balls (labelled B1 and B2) and one white ball. A
	second urn contains one black ball and two white balls (labelled W1 and W2).
	Suppose the following experiment is performed. One of the two urns is chosen
	at random. Next a ball is randomly chosen from the urn. Then a second ball is
	chosen at random from the same urn without replacing the first ball.
	
	\begin{enumerate}
	\item What is the probability that two black balls are chosen?
	
	\item What is the probability that two balls of opposite colour are chosen?
	\end{enumerate}
	\solution
	%\input{exemplar/11/16/3/12/main1.tex}
\end{enumerate}

\item In a certain lottery 10,000 tickets are sold and ten equal prizes are awarded. What is the probability of not getting a prize if you buy (a) one ticket (b) two tickets (c) 10 tickets ?	
\\
\solution
		%\begin{enumerate}[label=\thesection.\arabic*,ref=\thesection.\theenumi]
	\item One card is drawn from a well-shuffled deck of 52 cards. Find the probability of getting
\begin{enumerate}
\item A king of red colour 
\item A face card 
\item A red face card
\item The jack of hearts
\item A spade
\item The queen of diamonds

\end{enumerate}
\solution
		%\input{ncert/10/15/1/14/main.tex}
	\item Five cards—the ten, jack, queen, king and ace of diamonds, are well-shuffled with their face downwards. One card is then picked up at random.
\begin{enumerate}
\item
What is the probability that the card is the queen? 
\item
If the queen is drawn and put aside, what is the probability that the second card picked up is (a) an ace? (b) a queen?\\
\end{enumerate}
\solution
		%\input{ncert/10/15/1/15/defs.tex}
	\item A bag contains $5$ red balls and some blue balls. If the probability of drawing a blue ball is double that if a red ball, determine the number of blue balls in the bag. 
		\\
\solution
		%\input{ncert/10/15/2/3/defs.tex}
	\item A card is selected from a pack of 52 cards.
 \begin{enumerate}[label=(\alph*)] 
                 \item How many points are there in the sample space?
                 \item Calculate the probability that the card is an ace of spades.
                 \item Calculate the probability that the card is (i) an ace and (ii) black card.
 \end{enumerate}
\solution
		%\input{ncert/11/16/3/4/main.tex}
\item Four cards are drawn from a well-shuffled deck of 52 cards. What is the probability of obtaining 3 diamonds and one spade.
\\
\solution
		%\input{ncert/11/16/4/2/defs.tex}
\item In a certain lottery 10,000 tickets are sold and ten equal prizes are awarded. What is the probability of not getting a prize if you buy (a) one ticket (b) two tickets (c) 10 tickets ?	
\\
\solution
		%\input{ncert/11/16/4/4/defs.tex}
		%
\item 
Out of 100 students, two sections of 40 and 60 are formed. If you and your friend are among the 100 students, what is the probability that
\begin{enumerate}
\item you both enter the same section?
\item you both enter the different sections?
\end{enumerate}
\solution
		%\input{ncert/11/16/4/5/defs.tex}
	\item 
The number lock of a suitcase has 4 wheels each labelled with ten digits i.e. from 0 to 9.The lock opens with a sequence of four digits with no repeats.What is the probability of a person getting the right sequence to open the suitcase.
\\
\solution
		%\input{ncert/11/16/4/10/defs.tex}
		%
\item 
Two cards are drawn at random and without replacement from a pack of 52 playing cards. Find the probability that both the cards are black.
\\
\solution
		%\input{ncert/12/13/2/2/defs.tex}
		\item A box of oranges is inspected by examining three randomly selected oranges drawn without replacement. If all the three oranges are good, the box is approved for sale, otherwise, it is rejected. Find the probability that a box containing 15 oranges out of which 12 are good and 3 are bad ones will be approved for sale.
		\label{ncert/12/13/2/3/defs.tex}
		\item Two balls are drawn at random with replacement from a box containing 10 black and 8 red balls. Find the probability that
		\label{ncert/12/13/2/12}
\begin{enumerate}
\item both balls are red.
\item first ball is black and second is red.
\item one of them is black and other is red.
\end{enumerate}

\item In a hostel, 60\% of the students read Hindi newspaper, 40\% read English newspaper and 20\% read both Hindi and English newspapers. A student is selected at random.
		\label{ncert/12/13/2/15}
\begin{enumerate}
\item Find the probability that she reads neither Hindi nor English newspapers.
\item If she reads Hindi newspaper, find the probability that she reads English newspaper.
\item If she reads English newspaper, find the probability that she reads Hindi newspaper.\\
\end{enumerate}
\item The probability of obtaining an even prime number on each die, when a pair of dice is rolled is 
\begin{enumerate}
    \item $0$ 
    
    \item $\frac{1}{3}$ 
    
    \item $\frac{1}{12}$ 
    
    \item $\frac{1}{36}$ 
\end{enumerate}
\solution
		%\input{ncert/12/13/2/17/defs.tex}
	\item A bag contains 4 red and 4 black balls, another bag contains 2 red and 6 black balls. One of the two bags is selected at random and a ball is drawn from the bag which is found to be red. Find the probability that the ball is drawn from the first bag.
\\
\solution
		%\input{ncert/12/13/3/2/main.tex}
  \item
  Cards with numbers 2 to 101 are placed in a box. A card is selected at random.Find the probability that the card has
\begin{enumerate}[label=(\roman*)]
	\item an even number 
	\item a square number
\end{enumerate}
\solution
%\input{exemplar/10/13/3/32/main.tex}
\item
The king, queen and jack of clubs are removed from a deck of 52 playing cards and then well shuffled. Now one card is drawn at random from the remaining cards.  Determine the probability that the card is
\begin{enumerate}[label=(\roman*)]
\item a club
\item 10 of hearts
\end{enumerate}
\solution
%\input{exemplar/10/13/3/29/main.tex}
\item A team of medical students doing their internship have to assist during surgeries
at a city hospital. The probabilities of surgeries rated as very complex, complex,
routine, simple or very simple are respectively, 0.15, 0.20, 0.31, 0.26, .08. Find
the probabilities that a particular surgery will be rated
\begin{enumerate}
	\item complex or very complex;
	\item neither very complex nor very simple;
	\item routine or complex
	\item routine or simple
\end{enumerate}
\solution
%\input{exemplar/11/16/3/8(1)/main.tex}
\item A card is selected from a pack of 52 cards.
\begin{enumerate}[label=(\alph*)]
    \item How many points are there in the sample space?
    \item Calculate the probability that the card is an ace of spades.
    \item Calculate the probability that the card is (i) an ace and (ii) black card.
\end{enumerate}
\solution
%\input{exemplar/11/16/3/4/main2.tex}
\item The probability that a non leap year selected at random will contain 53 sundays.
\\
\solution
%\input{exemplar/10/13/1/19/main.tex}
\item One of the four persons John, Rita, Aslam or Gurpreet will be promoted next
month. Consequently the sample space consists of four elementary outcomes
S = {John promoted, Rita promoted, Aslam promoted, Gurpreet promoted}
You are told that the chances of John’s promotion is same as that of Gurpreet,
Rita’s chances of promotion are twice as likely as Johns. Aslam’s chances are
four times that of John.
\begin{enumerate}
	\item Determine
	\begin{enumerate}
		\item P (John promoted)
		\item P (Rita promoted)
		\item P (Aslam promoted)
		\item P (Gurpreet promoted)
	\end{enumerate}
	\item If A = {John promoted or Gurpreet promoted}, find P (A).
\end{enumerate}
\solution
%\input{exemplar/11/16/3/10/main.tex}
\item A card is drawn from a deck of 52 cards. Find the probability of getting a king or a heart or a red card.\\
\solution
%\input{exemplar/11/16/3/15/main.tex}
\item The probability that a student will pass his examination is 0.73, the probability of
the student getting a compartment is 0.13, and the probability that the student will
either pass or get compartment is 0.96. State True or False.\\
\solution
%\input{exemplar/11/16/3/31/main.tex}
\item A card is selected from a pack of 52 cards\\
\begin{enumerate}[label=(\alph*)]
\item How many points are there in the sample space?
\item Calculate the probability that the cards is an ace of spades.
\item Calculate the probability that the card is (i) an ace (ii)black card.\\
\end{enumerate}
%\input{ncert/11/16/3/4_1/Prob_4.tex}
\item In a non-leap year, the probability of having 53 tuesdays or 53 wednesdays is\\
\solution
%\input{exemplar/11/16/3/18/main.tex}
\item There are 1000 sealed envelopes in a box, 10 of them contain a cash prize of
Rs 100 each, 100 of them contain a cash prize of Rs 50 each and 200 of them
contain a cash prize of Rs 10 each and rest do not contain any cash prize. If they
are well shuffled and an envelope is picked up out, what is the probability that it
contains no cash prize?\\
\solution
%\input{exemplar/10/13/3/34/main.tex}
\item 
A die is thrown and a card is selected at random from a deck of 52 playing cards. The probability of getting an even number on the die and a spade card.\\
\solution
%\input{exemplar/12/13/3/78/main.tex}
\item
If 4-digit numbers greater than 5,000 are randomly formed from the digits 0, 1, 3, 5, and 7, what is the probability of forming a number divisible by 5 when:
\begin{enumerate}
    \item The digits are repeated?
    \item The repetition of digits is not allowed?
\end{enumerate}
\solution
%\input{ncert/11/16/4/9/main.tex}
\item Consider the probability space $\brak{\Omega, \mathcal{G}, P}$ where $\Omega = [0,2]$ and $\mathcal{G} = \cbrak{\phi, \Omega, [0,1], (1,2]}$. Let $X$ and $Y$ be two functions on $\Omega$ defined as
\begin{align*}
    X(\omega) = 
    \begin{cases}
        1 & \text{if }\omega \in [0, 1]\\
        2 & \text{if }\omega \in (1, 2]
    \end{cases}
\end{align*}
and
\begin{align*}
    Y(\omega) = 
    \begin{cases}
        2 & \text{if }\omega \in [0, 1.5]\\
        3 & \text{if }\omega \in (1.5, 2].
    \end{cases}
\end{align*}
Then which one of the following statements is true?
\begin{enumerate}
    \item [(A)] $X$ is a random variable with respect to $\mathcal{G}$, but $Y$ is not a random variable with respect to $\mathcal{G}$.
    \item [(B)] $Y$ is a random variable with respect to $\mathcal{G}$, but $X$ is not a random variable with respect to $\mathcal{G}$.
    \item [(C)] Neither $X$ nor $Y$ is a random variable with respect to $\mathcal{G}$.
    \item [(D)] Both $X$ and $Y$ are random variables with respect to $\mathcal{G}$.
\end{enumerate} \hfill (GATE ST 2023)\\
\solution
%\input{gate/ST/2023/14/main.tex}
	\item  A die is loaded in such a way that each odd number is twice as likely to occur as
each even number. Find $P(G)$, where $G$ is the event that a number greater than
3 occurs on a single roll of the die.
\\
\solution
		%\input{exemplar/11/16/3/5/main.tex}
	\item All the jacks, queens and kings are removed from a deck of 52 playing cards. The remaining cards are well shuffled and then one card is drawn at random. Giving ace a value 1 similar value for other cards, find the probability that the card has a value 
		\begin{enumerate}
			\item 7
			\item greater than 7
			\item less than 7
		\end{enumerate}
		%\input{exemplar/10/13/3/30/main.tex}
  \item A Lot consists of 48 mobile phones of which 42 are good, 3 have only minor defects and 3 have major defects.Varnika will buy a phone if it is good but the trader will only buy a mobile if it has no major defects. One phone is selected at random from the lot. What is the probability that it is
\begin{enumerate}
	\item acceptable to Varnika?
            \item acceptable to the trader?
\end{enumerate}
\solution
	%\input{exemplar/10/13/3/40/main.tex}
 \item A student says that if you throw a die, it will show up 1 or not 1. Therefore, the probability of getting 1 and the probability of getting 'not 1' each is equal to $\frac{1}{2}$. Is this correct? Give reasons.\\
 \solution
        %\input{exemplar/10/13/2/9/main.tex}
   \item Four candidates A, B, C, D have ap-
plied for the assignment to coach a school cricket
team. If A is twice as likely to be selected as B, and
B and C are given about the same chance of being
selected, while C is twice as likely to be selected
as D, what are the probabilities that
\begin{enumerate}
\item C will be selected?
\item A will not be selected?
\end{enumerate}
	%\input{exemplar/11/16/3/9/main.tex}
 \item A bag contain 24 balls of which $x$ balls are red, $2x$ are white and $3x$ are blue. A ball is selected at random, What is the probability that it is
\begin{enumerate}[label=\alph*)]
\item not red ?
\item white ?
\end{enumerate}
%\input{exemplar/10/13/3/41/main.tex}
If the letters of the word ASSASSINATION are arranged at random. Find the Probability that
\begin{enumerate}[label=(\alph*)]
\item Four $S's$ come consecutively in the word
\item Two  $I's$ and two $N's$ come together
\item All $A's$ are not coming together
\item No two $A's$ are coming together
\end{enumerate}
%\input{exemplar/11/16/3/14/main.tex}
	\item One urn contains two black balls (labelled B1 and B2) and one white ball. A
	second urn contains one black ball and two white balls (labelled W1 and W2).
	Suppose the following experiment is performed. One of the two urns is chosen
	at random. Next a ball is randomly chosen from the urn. Then a second ball is
	chosen at random from the same urn without replacing the first ball.
	
	\begin{enumerate}
	\item What is the probability that two black balls are chosen?
	
	\item What is the probability that two balls of opposite colour are chosen?
	\end{enumerate}
	\solution
	%\input{exemplar/11/16/3/12/main1.tex}
\end{enumerate}

		%
\item 
Out of 100 students, two sections of 40 and 60 are formed. If you and your friend are among the 100 students, what is the probability that
\begin{enumerate}
\item you both enter the same section?
\item you both enter the different sections?
\end{enumerate}
\solution
		%\begin{enumerate}[label=\thesection.\arabic*,ref=\thesection.\theenumi]
	\item One card is drawn from a well-shuffled deck of 52 cards. Find the probability of getting
\begin{enumerate}
\item A king of red colour 
\item A face card 
\item A red face card
\item The jack of hearts
\item A spade
\item The queen of diamonds

\end{enumerate}
\solution
		%\input{ncert/10/15/1/14/main.tex}
	\item Five cards—the ten, jack, queen, king and ace of diamonds, are well-shuffled with their face downwards. One card is then picked up at random.
\begin{enumerate}
\item
What is the probability that the card is the queen? 
\item
If the queen is drawn and put aside, what is the probability that the second card picked up is (a) an ace? (b) a queen?\\
\end{enumerate}
\solution
		%\input{ncert/10/15/1/15/defs.tex}
	\item A bag contains $5$ red balls and some blue balls. If the probability of drawing a blue ball is double that if a red ball, determine the number of blue balls in the bag. 
		\\
\solution
		%\input{ncert/10/15/2/3/defs.tex}
	\item A card is selected from a pack of 52 cards.
 \begin{enumerate}[label=(\alph*)] 
                 \item How many points are there in the sample space?
                 \item Calculate the probability that the card is an ace of spades.
                 \item Calculate the probability that the card is (i) an ace and (ii) black card.
 \end{enumerate}
\solution
		%\input{ncert/11/16/3/4/main.tex}
\item Four cards are drawn from a well-shuffled deck of 52 cards. What is the probability of obtaining 3 diamonds and one spade.
\\
\solution
		%\input{ncert/11/16/4/2/defs.tex}
\item In a certain lottery 10,000 tickets are sold and ten equal prizes are awarded. What is the probability of not getting a prize if you buy (a) one ticket (b) two tickets (c) 10 tickets ?	
\\
\solution
		%\input{ncert/11/16/4/4/defs.tex}
		%
\item 
Out of 100 students, two sections of 40 and 60 are formed. If you and your friend are among the 100 students, what is the probability that
\begin{enumerate}
\item you both enter the same section?
\item you both enter the different sections?
\end{enumerate}
\solution
		%\input{ncert/11/16/4/5/defs.tex}
	\item 
The number lock of a suitcase has 4 wheels each labelled with ten digits i.e. from 0 to 9.The lock opens with a sequence of four digits with no repeats.What is the probability of a person getting the right sequence to open the suitcase.
\\
\solution
		%\input{ncert/11/16/4/10/defs.tex}
		%
\item 
Two cards are drawn at random and without replacement from a pack of 52 playing cards. Find the probability that both the cards are black.
\\
\solution
		%\input{ncert/12/13/2/2/defs.tex}
		\item A box of oranges is inspected by examining three randomly selected oranges drawn without replacement. If all the three oranges are good, the box is approved for sale, otherwise, it is rejected. Find the probability that a box containing 15 oranges out of which 12 are good and 3 are bad ones will be approved for sale.
		\label{ncert/12/13/2/3/defs.tex}
		\item Two balls are drawn at random with replacement from a box containing 10 black and 8 red balls. Find the probability that
		\label{ncert/12/13/2/12}
\begin{enumerate}
\item both balls are red.
\item first ball is black and second is red.
\item one of them is black and other is red.
\end{enumerate}

\item In a hostel, 60\% of the students read Hindi newspaper, 40\% read English newspaper and 20\% read both Hindi and English newspapers. A student is selected at random.
		\label{ncert/12/13/2/15}
\begin{enumerate}
\item Find the probability that she reads neither Hindi nor English newspapers.
\item If she reads Hindi newspaper, find the probability that she reads English newspaper.
\item If she reads English newspaper, find the probability that she reads Hindi newspaper.\\
\end{enumerate}
\item The probability of obtaining an even prime number on each die, when a pair of dice is rolled is 
\begin{enumerate}
    \item $0$ 
    
    \item $\frac{1}{3}$ 
    
    \item $\frac{1}{12}$ 
    
    \item $\frac{1}{36}$ 
\end{enumerate}
\solution
		%\input{ncert/12/13/2/17/defs.tex}
	\item A bag contains 4 red and 4 black balls, another bag contains 2 red and 6 black balls. One of the two bags is selected at random and a ball is drawn from the bag which is found to be red. Find the probability that the ball is drawn from the first bag.
\\
\solution
		%\input{ncert/12/13/3/2/main.tex}
  \item
  Cards with numbers 2 to 101 are placed in a box. A card is selected at random.Find the probability that the card has
\begin{enumerate}[label=(\roman*)]
	\item an even number 
	\item a square number
\end{enumerate}
\solution
%\input{exemplar/10/13/3/32/main.tex}
\item
The king, queen and jack of clubs are removed from a deck of 52 playing cards and then well shuffled. Now one card is drawn at random from the remaining cards.  Determine the probability that the card is
\begin{enumerate}[label=(\roman*)]
\item a club
\item 10 of hearts
\end{enumerate}
\solution
%\input{exemplar/10/13/3/29/main.tex}
\item A team of medical students doing their internship have to assist during surgeries
at a city hospital. The probabilities of surgeries rated as very complex, complex,
routine, simple or very simple are respectively, 0.15, 0.20, 0.31, 0.26, .08. Find
the probabilities that a particular surgery will be rated
\begin{enumerate}
	\item complex or very complex;
	\item neither very complex nor very simple;
	\item routine or complex
	\item routine or simple
\end{enumerate}
\solution
%\input{exemplar/11/16/3/8(1)/main.tex}
\item A card is selected from a pack of 52 cards.
\begin{enumerate}[label=(\alph*)]
    \item How many points are there in the sample space?
    \item Calculate the probability that the card is an ace of spades.
    \item Calculate the probability that the card is (i) an ace and (ii) black card.
\end{enumerate}
\solution
%\input{exemplar/11/16/3/4/main2.tex}
\item The probability that a non leap year selected at random will contain 53 sundays.
\\
\solution
%\input{exemplar/10/13/1/19/main.tex}
\item One of the four persons John, Rita, Aslam or Gurpreet will be promoted next
month. Consequently the sample space consists of four elementary outcomes
S = {John promoted, Rita promoted, Aslam promoted, Gurpreet promoted}
You are told that the chances of John’s promotion is same as that of Gurpreet,
Rita’s chances of promotion are twice as likely as Johns. Aslam’s chances are
four times that of John.
\begin{enumerate}
	\item Determine
	\begin{enumerate}
		\item P (John promoted)
		\item P (Rita promoted)
		\item P (Aslam promoted)
		\item P (Gurpreet promoted)
	\end{enumerate}
	\item If A = {John promoted or Gurpreet promoted}, find P (A).
\end{enumerate}
\solution
%\input{exemplar/11/16/3/10/main.tex}
\item A card is drawn from a deck of 52 cards. Find the probability of getting a king or a heart or a red card.\\
\solution
%\input{exemplar/11/16/3/15/main.tex}
\item The probability that a student will pass his examination is 0.73, the probability of
the student getting a compartment is 0.13, and the probability that the student will
either pass or get compartment is 0.96. State True or False.\\
\solution
%\input{exemplar/11/16/3/31/main.tex}
\item A card is selected from a pack of 52 cards\\
\begin{enumerate}[label=(\alph*)]
\item How many points are there in the sample space?
\item Calculate the probability that the cards is an ace of spades.
\item Calculate the probability that the card is (i) an ace (ii)black card.\\
\end{enumerate}
%\input{ncert/11/16/3/4_1/Prob_4.tex}
\item In a non-leap year, the probability of having 53 tuesdays or 53 wednesdays is\\
\solution
%\input{exemplar/11/16/3/18/main.tex}
\item There are 1000 sealed envelopes in a box, 10 of them contain a cash prize of
Rs 100 each, 100 of them contain a cash prize of Rs 50 each and 200 of them
contain a cash prize of Rs 10 each and rest do not contain any cash prize. If they
are well shuffled and an envelope is picked up out, what is the probability that it
contains no cash prize?\\
\solution
%\input{exemplar/10/13/3/34/main.tex}
\item 
A die is thrown and a card is selected at random from a deck of 52 playing cards. The probability of getting an even number on the die and a spade card.\\
\solution
%\input{exemplar/12/13/3/78/main.tex}
\item
If 4-digit numbers greater than 5,000 are randomly formed from the digits 0, 1, 3, 5, and 7, what is the probability of forming a number divisible by 5 when:
\begin{enumerate}
    \item The digits are repeated?
    \item The repetition of digits is not allowed?
\end{enumerate}
\solution
%\input{ncert/11/16/4/9/main.tex}
\item Consider the probability space $\brak{\Omega, \mathcal{G}, P}$ where $\Omega = [0,2]$ and $\mathcal{G} = \cbrak{\phi, \Omega, [0,1], (1,2]}$. Let $X$ and $Y$ be two functions on $\Omega$ defined as
\begin{align*}
    X(\omega) = 
    \begin{cases}
        1 & \text{if }\omega \in [0, 1]\\
        2 & \text{if }\omega \in (1, 2]
    \end{cases}
\end{align*}
and
\begin{align*}
    Y(\omega) = 
    \begin{cases}
        2 & \text{if }\omega \in [0, 1.5]\\
        3 & \text{if }\omega \in (1.5, 2].
    \end{cases}
\end{align*}
Then which one of the following statements is true?
\begin{enumerate}
    \item [(A)] $X$ is a random variable with respect to $\mathcal{G}$, but $Y$ is not a random variable with respect to $\mathcal{G}$.
    \item [(B)] $Y$ is a random variable with respect to $\mathcal{G}$, but $X$ is not a random variable with respect to $\mathcal{G}$.
    \item [(C)] Neither $X$ nor $Y$ is a random variable with respect to $\mathcal{G}$.
    \item [(D)] Both $X$ and $Y$ are random variables with respect to $\mathcal{G}$.
\end{enumerate} \hfill (GATE ST 2023)\\
\solution
%\input{gate/ST/2023/14/main.tex}
	\item  A die is loaded in such a way that each odd number is twice as likely to occur as
each even number. Find $P(G)$, where $G$ is the event that a number greater than
3 occurs on a single roll of the die.
\\
\solution
		%\input{exemplar/11/16/3/5/main.tex}
	\item All the jacks, queens and kings are removed from a deck of 52 playing cards. The remaining cards are well shuffled and then one card is drawn at random. Giving ace a value 1 similar value for other cards, find the probability that the card has a value 
		\begin{enumerate}
			\item 7
			\item greater than 7
			\item less than 7
		\end{enumerate}
		%\input{exemplar/10/13/3/30/main.tex}
  \item A Lot consists of 48 mobile phones of which 42 are good, 3 have only minor defects and 3 have major defects.Varnika will buy a phone if it is good but the trader will only buy a mobile if it has no major defects. One phone is selected at random from the lot. What is the probability that it is
\begin{enumerate}
	\item acceptable to Varnika?
            \item acceptable to the trader?
\end{enumerate}
\solution
	%\input{exemplar/10/13/3/40/main.tex}
 \item A student says that if you throw a die, it will show up 1 or not 1. Therefore, the probability of getting 1 and the probability of getting 'not 1' each is equal to $\frac{1}{2}$. Is this correct? Give reasons.\\
 \solution
        %\input{exemplar/10/13/2/9/main.tex}
   \item Four candidates A, B, C, D have ap-
plied for the assignment to coach a school cricket
team. If A is twice as likely to be selected as B, and
B and C are given about the same chance of being
selected, while C is twice as likely to be selected
as D, what are the probabilities that
\begin{enumerate}
\item C will be selected?
\item A will not be selected?
\end{enumerate}
	%\input{exemplar/11/16/3/9/main.tex}
 \item A bag contain 24 balls of which $x$ balls are red, $2x$ are white and $3x$ are blue. A ball is selected at random, What is the probability that it is
\begin{enumerate}[label=\alph*)]
\item not red ?
\item white ?
\end{enumerate}
%\input{exemplar/10/13/3/41/main.tex}
If the letters of the word ASSASSINATION are arranged at random. Find the Probability that
\begin{enumerate}[label=(\alph*)]
\item Four $S's$ come consecutively in the word
\item Two  $I's$ and two $N's$ come together
\item All $A's$ are not coming together
\item No two $A's$ are coming together
\end{enumerate}
%\input{exemplar/11/16/3/14/main.tex}
	\item One urn contains two black balls (labelled B1 and B2) and one white ball. A
	second urn contains one black ball and two white balls (labelled W1 and W2).
	Suppose the following experiment is performed. One of the two urns is chosen
	at random. Next a ball is randomly chosen from the urn. Then a second ball is
	chosen at random from the same urn without replacing the first ball.
	
	\begin{enumerate}
	\item What is the probability that two black balls are chosen?
	
	\item What is the probability that two balls of opposite colour are chosen?
	\end{enumerate}
	\solution
	%\input{exemplar/11/16/3/12/main1.tex}
\end{enumerate}

	\item 
The number lock of a suitcase has 4 wheels each labelled with ten digits i.e. from 0 to 9.The lock opens with a sequence of four digits with no repeats.What is the probability of a person getting the right sequence to open the suitcase.
\\
\solution
		%\begin{enumerate}[label=\thesection.\arabic*,ref=\thesection.\theenumi]
	\item One card is drawn from a well-shuffled deck of 52 cards. Find the probability of getting
\begin{enumerate}
\item A king of red colour 
\item A face card 
\item A red face card
\item The jack of hearts
\item A spade
\item The queen of diamonds

\end{enumerate}
\solution
		%\input{ncert/10/15/1/14/main.tex}
	\item Five cards—the ten, jack, queen, king and ace of diamonds, are well-shuffled with their face downwards. One card is then picked up at random.
\begin{enumerate}
\item
What is the probability that the card is the queen? 
\item
If the queen is drawn and put aside, what is the probability that the second card picked up is (a) an ace? (b) a queen?\\
\end{enumerate}
\solution
		%\input{ncert/10/15/1/15/defs.tex}
	\item A bag contains $5$ red balls and some blue balls. If the probability of drawing a blue ball is double that if a red ball, determine the number of blue balls in the bag. 
		\\
\solution
		%\input{ncert/10/15/2/3/defs.tex}
	\item A card is selected from a pack of 52 cards.
 \begin{enumerate}[label=(\alph*)] 
                 \item How many points are there in the sample space?
                 \item Calculate the probability that the card is an ace of spades.
                 \item Calculate the probability that the card is (i) an ace and (ii) black card.
 \end{enumerate}
\solution
		%\input{ncert/11/16/3/4/main.tex}
\item Four cards are drawn from a well-shuffled deck of 52 cards. What is the probability of obtaining 3 diamonds and one spade.
\\
\solution
		%\input{ncert/11/16/4/2/defs.tex}
\item In a certain lottery 10,000 tickets are sold and ten equal prizes are awarded. What is the probability of not getting a prize if you buy (a) one ticket (b) two tickets (c) 10 tickets ?	
\\
\solution
		%\input{ncert/11/16/4/4/defs.tex}
		%
\item 
Out of 100 students, two sections of 40 and 60 are formed. If you and your friend are among the 100 students, what is the probability that
\begin{enumerate}
\item you both enter the same section?
\item you both enter the different sections?
\end{enumerate}
\solution
		%\input{ncert/11/16/4/5/defs.tex}
	\item 
The number lock of a suitcase has 4 wheels each labelled with ten digits i.e. from 0 to 9.The lock opens with a sequence of four digits with no repeats.What is the probability of a person getting the right sequence to open the suitcase.
\\
\solution
		%\input{ncert/11/16/4/10/defs.tex}
		%
\item 
Two cards are drawn at random and without replacement from a pack of 52 playing cards. Find the probability that both the cards are black.
\\
\solution
		%\input{ncert/12/13/2/2/defs.tex}
		\item A box of oranges is inspected by examining three randomly selected oranges drawn without replacement. If all the three oranges are good, the box is approved for sale, otherwise, it is rejected. Find the probability that a box containing 15 oranges out of which 12 are good and 3 are bad ones will be approved for sale.
		\label{ncert/12/13/2/3/defs.tex}
		\item Two balls are drawn at random with replacement from a box containing 10 black and 8 red balls. Find the probability that
		\label{ncert/12/13/2/12}
\begin{enumerate}
\item both balls are red.
\item first ball is black and second is red.
\item one of them is black and other is red.
\end{enumerate}

\item In a hostel, 60\% of the students read Hindi newspaper, 40\% read English newspaper and 20\% read both Hindi and English newspapers. A student is selected at random.
		\label{ncert/12/13/2/15}
\begin{enumerate}
\item Find the probability that she reads neither Hindi nor English newspapers.
\item If she reads Hindi newspaper, find the probability that she reads English newspaper.
\item If she reads English newspaper, find the probability that she reads Hindi newspaper.\\
\end{enumerate}
\item The probability of obtaining an even prime number on each die, when a pair of dice is rolled is 
\begin{enumerate}
    \item $0$ 
    
    \item $\frac{1}{3}$ 
    
    \item $\frac{1}{12}$ 
    
    \item $\frac{1}{36}$ 
\end{enumerate}
\solution
		%\input{ncert/12/13/2/17/defs.tex}
	\item A bag contains 4 red and 4 black balls, another bag contains 2 red and 6 black balls. One of the two bags is selected at random and a ball is drawn from the bag which is found to be red. Find the probability that the ball is drawn from the first bag.
\\
\solution
		%\input{ncert/12/13/3/2/main.tex}
  \item
  Cards with numbers 2 to 101 are placed in a box. A card is selected at random.Find the probability that the card has
\begin{enumerate}[label=(\roman*)]
	\item an even number 
	\item a square number
\end{enumerate}
\solution
%\input{exemplar/10/13/3/32/main.tex}
\item
The king, queen and jack of clubs are removed from a deck of 52 playing cards and then well shuffled. Now one card is drawn at random from the remaining cards.  Determine the probability that the card is
\begin{enumerate}[label=(\roman*)]
\item a club
\item 10 of hearts
\end{enumerate}
\solution
%\input{exemplar/10/13/3/29/main.tex}
\item A team of medical students doing their internship have to assist during surgeries
at a city hospital. The probabilities of surgeries rated as very complex, complex,
routine, simple or very simple are respectively, 0.15, 0.20, 0.31, 0.26, .08. Find
the probabilities that a particular surgery will be rated
\begin{enumerate}
	\item complex or very complex;
	\item neither very complex nor very simple;
	\item routine or complex
	\item routine or simple
\end{enumerate}
\solution
%\input{exemplar/11/16/3/8(1)/main.tex}
\item A card is selected from a pack of 52 cards.
\begin{enumerate}[label=(\alph*)]
    \item How many points are there in the sample space?
    \item Calculate the probability that the card is an ace of spades.
    \item Calculate the probability that the card is (i) an ace and (ii) black card.
\end{enumerate}
\solution
%\input{exemplar/11/16/3/4/main2.tex}
\item The probability that a non leap year selected at random will contain 53 sundays.
\\
\solution
%\input{exemplar/10/13/1/19/main.tex}
\item One of the four persons John, Rita, Aslam or Gurpreet will be promoted next
month. Consequently the sample space consists of four elementary outcomes
S = {John promoted, Rita promoted, Aslam promoted, Gurpreet promoted}
You are told that the chances of John’s promotion is same as that of Gurpreet,
Rita’s chances of promotion are twice as likely as Johns. Aslam’s chances are
four times that of John.
\begin{enumerate}
	\item Determine
	\begin{enumerate}
		\item P (John promoted)
		\item P (Rita promoted)
		\item P (Aslam promoted)
		\item P (Gurpreet promoted)
	\end{enumerate}
	\item If A = {John promoted or Gurpreet promoted}, find P (A).
\end{enumerate}
\solution
%\input{exemplar/11/16/3/10/main.tex}
\item A card is drawn from a deck of 52 cards. Find the probability of getting a king or a heart or a red card.\\
\solution
%\input{exemplar/11/16/3/15/main.tex}
\item The probability that a student will pass his examination is 0.73, the probability of
the student getting a compartment is 0.13, and the probability that the student will
either pass or get compartment is 0.96. State True or False.\\
\solution
%\input{exemplar/11/16/3/31/main.tex}
\item A card is selected from a pack of 52 cards\\
\begin{enumerate}[label=(\alph*)]
\item How many points are there in the sample space?
\item Calculate the probability that the cards is an ace of spades.
\item Calculate the probability that the card is (i) an ace (ii)black card.\\
\end{enumerate}
%\input{ncert/11/16/3/4_1/Prob_4.tex}
\item In a non-leap year, the probability of having 53 tuesdays or 53 wednesdays is\\
\solution
%\input{exemplar/11/16/3/18/main.tex}
\item There are 1000 sealed envelopes in a box, 10 of them contain a cash prize of
Rs 100 each, 100 of them contain a cash prize of Rs 50 each and 200 of them
contain a cash prize of Rs 10 each and rest do not contain any cash prize. If they
are well shuffled and an envelope is picked up out, what is the probability that it
contains no cash prize?\\
\solution
%\input{exemplar/10/13/3/34/main.tex}
\item 
A die is thrown and a card is selected at random from a deck of 52 playing cards. The probability of getting an even number on the die and a spade card.\\
\solution
%\input{exemplar/12/13/3/78/main.tex}
\item
If 4-digit numbers greater than 5,000 are randomly formed from the digits 0, 1, 3, 5, and 7, what is the probability of forming a number divisible by 5 when:
\begin{enumerate}
    \item The digits are repeated?
    \item The repetition of digits is not allowed?
\end{enumerate}
\solution
%\input{ncert/11/16/4/9/main.tex}
\item Consider the probability space $\brak{\Omega, \mathcal{G}, P}$ where $\Omega = [0,2]$ and $\mathcal{G} = \cbrak{\phi, \Omega, [0,1], (1,2]}$. Let $X$ and $Y$ be two functions on $\Omega$ defined as
\begin{align*}
    X(\omega) = 
    \begin{cases}
        1 & \text{if }\omega \in [0, 1]\\
        2 & \text{if }\omega \in (1, 2]
    \end{cases}
\end{align*}
and
\begin{align*}
    Y(\omega) = 
    \begin{cases}
        2 & \text{if }\omega \in [0, 1.5]\\
        3 & \text{if }\omega \in (1.5, 2].
    \end{cases}
\end{align*}
Then which one of the following statements is true?
\begin{enumerate}
    \item [(A)] $X$ is a random variable with respect to $\mathcal{G}$, but $Y$ is not a random variable with respect to $\mathcal{G}$.
    \item [(B)] $Y$ is a random variable with respect to $\mathcal{G}$, but $X$ is not a random variable with respect to $\mathcal{G}$.
    \item [(C)] Neither $X$ nor $Y$ is a random variable with respect to $\mathcal{G}$.
    \item [(D)] Both $X$ and $Y$ are random variables with respect to $\mathcal{G}$.
\end{enumerate} \hfill (GATE ST 2023)\\
\solution
%\input{gate/ST/2023/14/main.tex}
	\item  A die is loaded in such a way that each odd number is twice as likely to occur as
each even number. Find $P(G)$, where $G$ is the event that a number greater than
3 occurs on a single roll of the die.
\\
\solution
		%\input{exemplar/11/16/3/5/main.tex}
	\item All the jacks, queens and kings are removed from a deck of 52 playing cards. The remaining cards are well shuffled and then one card is drawn at random. Giving ace a value 1 similar value for other cards, find the probability that the card has a value 
		\begin{enumerate}
			\item 7
			\item greater than 7
			\item less than 7
		\end{enumerate}
		%\input{exemplar/10/13/3/30/main.tex}
  \item A Lot consists of 48 mobile phones of which 42 are good, 3 have only minor defects and 3 have major defects.Varnika will buy a phone if it is good but the trader will only buy a mobile if it has no major defects. One phone is selected at random from the lot. What is the probability that it is
\begin{enumerate}
	\item acceptable to Varnika?
            \item acceptable to the trader?
\end{enumerate}
\solution
	%\input{exemplar/10/13/3/40/main.tex}
 \item A student says that if you throw a die, it will show up 1 or not 1. Therefore, the probability of getting 1 and the probability of getting 'not 1' each is equal to $\frac{1}{2}$. Is this correct? Give reasons.\\
 \solution
        %\input{exemplar/10/13/2/9/main.tex}
   \item Four candidates A, B, C, D have ap-
plied for the assignment to coach a school cricket
team. If A is twice as likely to be selected as B, and
B and C are given about the same chance of being
selected, while C is twice as likely to be selected
as D, what are the probabilities that
\begin{enumerate}
\item C will be selected?
\item A will not be selected?
\end{enumerate}
	%\input{exemplar/11/16/3/9/main.tex}
 \item A bag contain 24 balls of which $x$ balls are red, $2x$ are white and $3x$ are blue. A ball is selected at random, What is the probability that it is
\begin{enumerate}[label=\alph*)]
\item not red ?
\item white ?
\end{enumerate}
%\input{exemplar/10/13/3/41/main.tex}
If the letters of the word ASSASSINATION are arranged at random. Find the Probability that
\begin{enumerate}[label=(\alph*)]
\item Four $S's$ come consecutively in the word
\item Two  $I's$ and two $N's$ come together
\item All $A's$ are not coming together
\item No two $A's$ are coming together
\end{enumerate}
%\input{exemplar/11/16/3/14/main.tex}
	\item One urn contains two black balls (labelled B1 and B2) and one white ball. A
	second urn contains one black ball and two white balls (labelled W1 and W2).
	Suppose the following experiment is performed. One of the two urns is chosen
	at random. Next a ball is randomly chosen from the urn. Then a second ball is
	chosen at random from the same urn without replacing the first ball.
	
	\begin{enumerate}
	\item What is the probability that two black balls are chosen?
	
	\item What is the probability that two balls of opposite colour are chosen?
	\end{enumerate}
	\solution
	%\input{exemplar/11/16/3/12/main1.tex}
\end{enumerate}

		%
\item 
Two cards are drawn at random and without replacement from a pack of 52 playing cards. Find the probability that both the cards are black.
\\
\solution
		%\begin{enumerate}[label=\thesection.\arabic*,ref=\thesection.\theenumi]
	\item One card is drawn from a well-shuffled deck of 52 cards. Find the probability of getting
\begin{enumerate}
\item A king of red colour 
\item A face card 
\item A red face card
\item The jack of hearts
\item A spade
\item The queen of diamonds

\end{enumerate}
\solution
		%\input{ncert/10/15/1/14/main.tex}
	\item Five cards—the ten, jack, queen, king and ace of diamonds, are well-shuffled with their face downwards. One card is then picked up at random.
\begin{enumerate}
\item
What is the probability that the card is the queen? 
\item
If the queen is drawn and put aside, what is the probability that the second card picked up is (a) an ace? (b) a queen?\\
\end{enumerate}
\solution
		%\input{ncert/10/15/1/15/defs.tex}
	\item A bag contains $5$ red balls and some blue balls. If the probability of drawing a blue ball is double that if a red ball, determine the number of blue balls in the bag. 
		\\
\solution
		%\input{ncert/10/15/2/3/defs.tex}
	\item A card is selected from a pack of 52 cards.
 \begin{enumerate}[label=(\alph*)] 
                 \item How many points are there in the sample space?
                 \item Calculate the probability that the card is an ace of spades.
                 \item Calculate the probability that the card is (i) an ace and (ii) black card.
 \end{enumerate}
\solution
		%\input{ncert/11/16/3/4/main.tex}
\item Four cards are drawn from a well-shuffled deck of 52 cards. What is the probability of obtaining 3 diamonds and one spade.
\\
\solution
		%\input{ncert/11/16/4/2/defs.tex}
\item In a certain lottery 10,000 tickets are sold and ten equal prizes are awarded. What is the probability of not getting a prize if you buy (a) one ticket (b) two tickets (c) 10 tickets ?	
\\
\solution
		%\input{ncert/11/16/4/4/defs.tex}
		%
\item 
Out of 100 students, two sections of 40 and 60 are formed. If you and your friend are among the 100 students, what is the probability that
\begin{enumerate}
\item you both enter the same section?
\item you both enter the different sections?
\end{enumerate}
\solution
		%\input{ncert/11/16/4/5/defs.tex}
	\item 
The number lock of a suitcase has 4 wheels each labelled with ten digits i.e. from 0 to 9.The lock opens with a sequence of four digits with no repeats.What is the probability of a person getting the right sequence to open the suitcase.
\\
\solution
		%\input{ncert/11/16/4/10/defs.tex}
		%
\item 
Two cards are drawn at random and without replacement from a pack of 52 playing cards. Find the probability that both the cards are black.
\\
\solution
		%\input{ncert/12/13/2/2/defs.tex}
		\item A box of oranges is inspected by examining three randomly selected oranges drawn without replacement. If all the three oranges are good, the box is approved for sale, otherwise, it is rejected. Find the probability that a box containing 15 oranges out of which 12 are good and 3 are bad ones will be approved for sale.
		\label{ncert/12/13/2/3/defs.tex}
		\item Two balls are drawn at random with replacement from a box containing 10 black and 8 red balls. Find the probability that
		\label{ncert/12/13/2/12}
\begin{enumerate}
\item both balls are red.
\item first ball is black and second is red.
\item one of them is black and other is red.
\end{enumerate}

\item In a hostel, 60\% of the students read Hindi newspaper, 40\% read English newspaper and 20\% read both Hindi and English newspapers. A student is selected at random.
		\label{ncert/12/13/2/15}
\begin{enumerate}
\item Find the probability that she reads neither Hindi nor English newspapers.
\item If she reads Hindi newspaper, find the probability that she reads English newspaper.
\item If she reads English newspaper, find the probability that she reads Hindi newspaper.\\
\end{enumerate}
\item The probability of obtaining an even prime number on each die, when a pair of dice is rolled is 
\begin{enumerate}
    \item $0$ 
    
    \item $\frac{1}{3}$ 
    
    \item $\frac{1}{12}$ 
    
    \item $\frac{1}{36}$ 
\end{enumerate}
\solution
		%\input{ncert/12/13/2/17/defs.tex}
	\item A bag contains 4 red and 4 black balls, another bag contains 2 red and 6 black balls. One of the two bags is selected at random and a ball is drawn from the bag which is found to be red. Find the probability that the ball is drawn from the first bag.
\\
\solution
		%\input{ncert/12/13/3/2/main.tex}
  \item
  Cards with numbers 2 to 101 are placed in a box. A card is selected at random.Find the probability that the card has
\begin{enumerate}[label=(\roman*)]
	\item an even number 
	\item a square number
\end{enumerate}
\solution
%\input{exemplar/10/13/3/32/main.tex}
\item
The king, queen and jack of clubs are removed from a deck of 52 playing cards and then well shuffled. Now one card is drawn at random from the remaining cards.  Determine the probability that the card is
\begin{enumerate}[label=(\roman*)]
\item a club
\item 10 of hearts
\end{enumerate}
\solution
%\input{exemplar/10/13/3/29/main.tex}
\item A team of medical students doing their internship have to assist during surgeries
at a city hospital. The probabilities of surgeries rated as very complex, complex,
routine, simple or very simple are respectively, 0.15, 0.20, 0.31, 0.26, .08. Find
the probabilities that a particular surgery will be rated
\begin{enumerate}
	\item complex or very complex;
	\item neither very complex nor very simple;
	\item routine or complex
	\item routine or simple
\end{enumerate}
\solution
%\input{exemplar/11/16/3/8(1)/main.tex}
\item A card is selected from a pack of 52 cards.
\begin{enumerate}[label=(\alph*)]
    \item How many points are there in the sample space?
    \item Calculate the probability that the card is an ace of spades.
    \item Calculate the probability that the card is (i) an ace and (ii) black card.
\end{enumerate}
\solution
%\input{exemplar/11/16/3/4/main2.tex}
\item The probability that a non leap year selected at random will contain 53 sundays.
\\
\solution
%\input{exemplar/10/13/1/19/main.tex}
\item One of the four persons John, Rita, Aslam or Gurpreet will be promoted next
month. Consequently the sample space consists of four elementary outcomes
S = {John promoted, Rita promoted, Aslam promoted, Gurpreet promoted}
You are told that the chances of John’s promotion is same as that of Gurpreet,
Rita’s chances of promotion are twice as likely as Johns. Aslam’s chances are
four times that of John.
\begin{enumerate}
	\item Determine
	\begin{enumerate}
		\item P (John promoted)
		\item P (Rita promoted)
		\item P (Aslam promoted)
		\item P (Gurpreet promoted)
	\end{enumerate}
	\item If A = {John promoted or Gurpreet promoted}, find P (A).
\end{enumerate}
\solution
%\input{exemplar/11/16/3/10/main.tex}
\item A card is drawn from a deck of 52 cards. Find the probability of getting a king or a heart or a red card.\\
\solution
%\input{exemplar/11/16/3/15/main.tex}
\item The probability that a student will pass his examination is 0.73, the probability of
the student getting a compartment is 0.13, and the probability that the student will
either pass or get compartment is 0.96. State True or False.\\
\solution
%\input{exemplar/11/16/3/31/main.tex}
\item A card is selected from a pack of 52 cards\\
\begin{enumerate}[label=(\alph*)]
\item How many points are there in the sample space?
\item Calculate the probability that the cards is an ace of spades.
\item Calculate the probability that the card is (i) an ace (ii)black card.\\
\end{enumerate}
%\input{ncert/11/16/3/4_1/Prob_4.tex}
\item In a non-leap year, the probability of having 53 tuesdays or 53 wednesdays is\\
\solution
%\input{exemplar/11/16/3/18/main.tex}
\item There are 1000 sealed envelopes in a box, 10 of them contain a cash prize of
Rs 100 each, 100 of them contain a cash prize of Rs 50 each and 200 of them
contain a cash prize of Rs 10 each and rest do not contain any cash prize. If they
are well shuffled and an envelope is picked up out, what is the probability that it
contains no cash prize?\\
\solution
%\input{exemplar/10/13/3/34/main.tex}
\item 
A die is thrown and a card is selected at random from a deck of 52 playing cards. The probability of getting an even number on the die and a spade card.\\
\solution
%\input{exemplar/12/13/3/78/main.tex}
\item
If 4-digit numbers greater than 5,000 are randomly formed from the digits 0, 1, 3, 5, and 7, what is the probability of forming a number divisible by 5 when:
\begin{enumerate}
    \item The digits are repeated?
    \item The repetition of digits is not allowed?
\end{enumerate}
\solution
%\input{ncert/11/16/4/9/main.tex}
\item Consider the probability space $\brak{\Omega, \mathcal{G}, P}$ where $\Omega = [0,2]$ and $\mathcal{G} = \cbrak{\phi, \Omega, [0,1], (1,2]}$. Let $X$ and $Y$ be two functions on $\Omega$ defined as
\begin{align*}
    X(\omega) = 
    \begin{cases}
        1 & \text{if }\omega \in [0, 1]\\
        2 & \text{if }\omega \in (1, 2]
    \end{cases}
\end{align*}
and
\begin{align*}
    Y(\omega) = 
    \begin{cases}
        2 & \text{if }\omega \in [0, 1.5]\\
        3 & \text{if }\omega \in (1.5, 2].
    \end{cases}
\end{align*}
Then which one of the following statements is true?
\begin{enumerate}
    \item [(A)] $X$ is a random variable with respect to $\mathcal{G}$, but $Y$ is not a random variable with respect to $\mathcal{G}$.
    \item [(B)] $Y$ is a random variable with respect to $\mathcal{G}$, but $X$ is not a random variable with respect to $\mathcal{G}$.
    \item [(C)] Neither $X$ nor $Y$ is a random variable with respect to $\mathcal{G}$.
    \item [(D)] Both $X$ and $Y$ are random variables with respect to $\mathcal{G}$.
\end{enumerate} \hfill (GATE ST 2023)\\
\solution
%\input{gate/ST/2023/14/main.tex}
	\item  A die is loaded in such a way that each odd number is twice as likely to occur as
each even number. Find $P(G)$, where $G$ is the event that a number greater than
3 occurs on a single roll of the die.
\\
\solution
		%\input{exemplar/11/16/3/5/main.tex}
	\item All the jacks, queens and kings are removed from a deck of 52 playing cards. The remaining cards are well shuffled and then one card is drawn at random. Giving ace a value 1 similar value for other cards, find the probability that the card has a value 
		\begin{enumerate}
			\item 7
			\item greater than 7
			\item less than 7
		\end{enumerate}
		%\input{exemplar/10/13/3/30/main.tex}
  \item A Lot consists of 48 mobile phones of which 42 are good, 3 have only minor defects and 3 have major defects.Varnika will buy a phone if it is good but the trader will only buy a mobile if it has no major defects. One phone is selected at random from the lot. What is the probability that it is
\begin{enumerate}
	\item acceptable to Varnika?
            \item acceptable to the trader?
\end{enumerate}
\solution
	%\input{exemplar/10/13/3/40/main.tex}
 \item A student says that if you throw a die, it will show up 1 or not 1. Therefore, the probability of getting 1 and the probability of getting 'not 1' each is equal to $\frac{1}{2}$. Is this correct? Give reasons.\\
 \solution
        %\input{exemplar/10/13/2/9/main.tex}
   \item Four candidates A, B, C, D have ap-
plied for the assignment to coach a school cricket
team. If A is twice as likely to be selected as B, and
B and C are given about the same chance of being
selected, while C is twice as likely to be selected
as D, what are the probabilities that
\begin{enumerate}
\item C will be selected?
\item A will not be selected?
\end{enumerate}
	%\input{exemplar/11/16/3/9/main.tex}
 \item A bag contain 24 balls of which $x$ balls are red, $2x$ are white and $3x$ are blue. A ball is selected at random, What is the probability that it is
\begin{enumerate}[label=\alph*)]
\item not red ?
\item white ?
\end{enumerate}
%\input{exemplar/10/13/3/41/main.tex}
If the letters of the word ASSASSINATION are arranged at random. Find the Probability that
\begin{enumerate}[label=(\alph*)]
\item Four $S's$ come consecutively in the word
\item Two  $I's$ and two $N's$ come together
\item All $A's$ are not coming together
\item No two $A's$ are coming together
\end{enumerate}
%\input{exemplar/11/16/3/14/main.tex}
	\item One urn contains two black balls (labelled B1 and B2) and one white ball. A
	second urn contains one black ball and two white balls (labelled W1 and W2).
	Suppose the following experiment is performed. One of the two urns is chosen
	at random. Next a ball is randomly chosen from the urn. Then a second ball is
	chosen at random from the same urn without replacing the first ball.
	
	\begin{enumerate}
	\item What is the probability that two black balls are chosen?
	
	\item What is the probability that two balls of opposite colour are chosen?
	\end{enumerate}
	\solution
	%\input{exemplar/11/16/3/12/main1.tex}
\end{enumerate}

		\item A box of oranges is inspected by examining three randomly selected oranges drawn without replacement. If all the three oranges are good, the box is approved for sale, otherwise, it is rejected. Find the probability that a box containing 15 oranges out of which 12 are good and 3 are bad ones will be approved for sale.
		\label{ncert/12/13/2/3/defs.tex}
		\item Two balls are drawn at random with replacement from a box containing 10 black and 8 red balls. Find the probability that
		\label{ncert/12/13/2/12}
\begin{enumerate}
\item both balls are red.
\item first ball is black and second is red.
\item one of them is black and other is red.
\end{enumerate}

\item In a hostel, 60\% of the students read Hindi newspaper, 40\% read English newspaper and 20\% read both Hindi and English newspapers. A student is selected at random.
		\label{ncert/12/13/2/15}
\begin{enumerate}
\item Find the probability that she reads neither Hindi nor English newspapers.
\item If she reads Hindi newspaper, find the probability that she reads English newspaper.
\item If she reads English newspaper, find the probability that she reads Hindi newspaper.\\
\end{enumerate}
\item The probability of obtaining an even prime number on each die, when a pair of dice is rolled is 
\begin{enumerate}
    \item $0$ 
    
    \item $\frac{1}{3}$ 
    
    \item $\frac{1}{12}$ 
    
    \item $\frac{1}{36}$ 
\end{enumerate}
\solution
		%\begin{enumerate}[label=\thesection.\arabic*,ref=\thesection.\theenumi]
	\item One card is drawn from a well-shuffled deck of 52 cards. Find the probability of getting
\begin{enumerate}
\item A king of red colour 
\item A face card 
\item A red face card
\item The jack of hearts
\item A spade
\item The queen of diamonds

\end{enumerate}
\solution
		%\input{ncert/10/15/1/14/main.tex}
	\item Five cards—the ten, jack, queen, king and ace of diamonds, are well-shuffled with their face downwards. One card is then picked up at random.
\begin{enumerate}
\item
What is the probability that the card is the queen? 
\item
If the queen is drawn and put aside, what is the probability that the second card picked up is (a) an ace? (b) a queen?\\
\end{enumerate}
\solution
		%\input{ncert/10/15/1/15/defs.tex}
	\item A bag contains $5$ red balls and some blue balls. If the probability of drawing a blue ball is double that if a red ball, determine the number of blue balls in the bag. 
		\\
\solution
		%\input{ncert/10/15/2/3/defs.tex}
	\item A card is selected from a pack of 52 cards.
 \begin{enumerate}[label=(\alph*)] 
                 \item How many points are there in the sample space?
                 \item Calculate the probability that the card is an ace of spades.
                 \item Calculate the probability that the card is (i) an ace and (ii) black card.
 \end{enumerate}
\solution
		%\input{ncert/11/16/3/4/main.tex}
\item Four cards are drawn from a well-shuffled deck of 52 cards. What is the probability of obtaining 3 diamonds and one spade.
\\
\solution
		%\input{ncert/11/16/4/2/defs.tex}
\item In a certain lottery 10,000 tickets are sold and ten equal prizes are awarded. What is the probability of not getting a prize if you buy (a) one ticket (b) two tickets (c) 10 tickets ?	
\\
\solution
		%\input{ncert/11/16/4/4/defs.tex}
		%
\item 
Out of 100 students, two sections of 40 and 60 are formed. If you and your friend are among the 100 students, what is the probability that
\begin{enumerate}
\item you both enter the same section?
\item you both enter the different sections?
\end{enumerate}
\solution
		%\input{ncert/11/16/4/5/defs.tex}
	\item 
The number lock of a suitcase has 4 wheels each labelled with ten digits i.e. from 0 to 9.The lock opens with a sequence of four digits with no repeats.What is the probability of a person getting the right sequence to open the suitcase.
\\
\solution
		%\input{ncert/11/16/4/10/defs.tex}
		%
\item 
Two cards are drawn at random and without replacement from a pack of 52 playing cards. Find the probability that both the cards are black.
\\
\solution
		%\input{ncert/12/13/2/2/defs.tex}
		\item A box of oranges is inspected by examining three randomly selected oranges drawn without replacement. If all the three oranges are good, the box is approved for sale, otherwise, it is rejected. Find the probability that a box containing 15 oranges out of which 12 are good and 3 are bad ones will be approved for sale.
		\label{ncert/12/13/2/3/defs.tex}
		\item Two balls are drawn at random with replacement from a box containing 10 black and 8 red balls. Find the probability that
		\label{ncert/12/13/2/12}
\begin{enumerate}
\item both balls are red.
\item first ball is black and second is red.
\item one of them is black and other is red.
\end{enumerate}

\item In a hostel, 60\% of the students read Hindi newspaper, 40\% read English newspaper and 20\% read both Hindi and English newspapers. A student is selected at random.
		\label{ncert/12/13/2/15}
\begin{enumerate}
\item Find the probability that she reads neither Hindi nor English newspapers.
\item If she reads Hindi newspaper, find the probability that she reads English newspaper.
\item If she reads English newspaper, find the probability that she reads Hindi newspaper.\\
\end{enumerate}
\item The probability of obtaining an even prime number on each die, when a pair of dice is rolled is 
\begin{enumerate}
    \item $0$ 
    
    \item $\frac{1}{3}$ 
    
    \item $\frac{1}{12}$ 
    
    \item $\frac{1}{36}$ 
\end{enumerate}
\solution
		%\input{ncert/12/13/2/17/defs.tex}
	\item A bag contains 4 red and 4 black balls, another bag contains 2 red and 6 black balls. One of the two bags is selected at random and a ball is drawn from the bag which is found to be red. Find the probability that the ball is drawn from the first bag.
\\
\solution
		%\input{ncert/12/13/3/2/main.tex}
  \item
  Cards with numbers 2 to 101 are placed in a box. A card is selected at random.Find the probability that the card has
\begin{enumerate}[label=(\roman*)]
	\item an even number 
	\item a square number
\end{enumerate}
\solution
%\input{exemplar/10/13/3/32/main.tex}
\item
The king, queen and jack of clubs are removed from a deck of 52 playing cards and then well shuffled. Now one card is drawn at random from the remaining cards.  Determine the probability that the card is
\begin{enumerate}[label=(\roman*)]
\item a club
\item 10 of hearts
\end{enumerate}
\solution
%\input{exemplar/10/13/3/29/main.tex}
\item A team of medical students doing their internship have to assist during surgeries
at a city hospital. The probabilities of surgeries rated as very complex, complex,
routine, simple or very simple are respectively, 0.15, 0.20, 0.31, 0.26, .08. Find
the probabilities that a particular surgery will be rated
\begin{enumerate}
	\item complex or very complex;
	\item neither very complex nor very simple;
	\item routine or complex
	\item routine or simple
\end{enumerate}
\solution
%\input{exemplar/11/16/3/8(1)/main.tex}
\item A card is selected from a pack of 52 cards.
\begin{enumerate}[label=(\alph*)]
    \item How many points are there in the sample space?
    \item Calculate the probability that the card is an ace of spades.
    \item Calculate the probability that the card is (i) an ace and (ii) black card.
\end{enumerate}
\solution
%\input{exemplar/11/16/3/4/main2.tex}
\item The probability that a non leap year selected at random will contain 53 sundays.
\\
\solution
%\input{exemplar/10/13/1/19/main.tex}
\item One of the four persons John, Rita, Aslam or Gurpreet will be promoted next
month. Consequently the sample space consists of four elementary outcomes
S = {John promoted, Rita promoted, Aslam promoted, Gurpreet promoted}
You are told that the chances of John’s promotion is same as that of Gurpreet,
Rita’s chances of promotion are twice as likely as Johns. Aslam’s chances are
four times that of John.
\begin{enumerate}
	\item Determine
	\begin{enumerate}
		\item P (John promoted)
		\item P (Rita promoted)
		\item P (Aslam promoted)
		\item P (Gurpreet promoted)
	\end{enumerate}
	\item If A = {John promoted or Gurpreet promoted}, find P (A).
\end{enumerate}
\solution
%\input{exemplar/11/16/3/10/main.tex}
\item A card is drawn from a deck of 52 cards. Find the probability of getting a king or a heart or a red card.\\
\solution
%\input{exemplar/11/16/3/15/main.tex}
\item The probability that a student will pass his examination is 0.73, the probability of
the student getting a compartment is 0.13, and the probability that the student will
either pass or get compartment is 0.96. State True or False.\\
\solution
%\input{exemplar/11/16/3/31/main.tex}
\item A card is selected from a pack of 52 cards\\
\begin{enumerate}[label=(\alph*)]
\item How many points are there in the sample space?
\item Calculate the probability that the cards is an ace of spades.
\item Calculate the probability that the card is (i) an ace (ii)black card.\\
\end{enumerate}
%\input{ncert/11/16/3/4_1/Prob_4.tex}
\item In a non-leap year, the probability of having 53 tuesdays or 53 wednesdays is\\
\solution
%\input{exemplar/11/16/3/18/main.tex}
\item There are 1000 sealed envelopes in a box, 10 of them contain a cash prize of
Rs 100 each, 100 of them contain a cash prize of Rs 50 each and 200 of them
contain a cash prize of Rs 10 each and rest do not contain any cash prize. If they
are well shuffled and an envelope is picked up out, what is the probability that it
contains no cash prize?\\
\solution
%\input{exemplar/10/13/3/34/main.tex}
\item 
A die is thrown and a card is selected at random from a deck of 52 playing cards. The probability of getting an even number on the die and a spade card.\\
\solution
%\input{exemplar/12/13/3/78/main.tex}
\item
If 4-digit numbers greater than 5,000 are randomly formed from the digits 0, 1, 3, 5, and 7, what is the probability of forming a number divisible by 5 when:
\begin{enumerate}
    \item The digits are repeated?
    \item The repetition of digits is not allowed?
\end{enumerate}
\solution
%\input{ncert/11/16/4/9/main.tex}
\item Consider the probability space $\brak{\Omega, \mathcal{G}, P}$ where $\Omega = [0,2]$ and $\mathcal{G} = \cbrak{\phi, \Omega, [0,1], (1,2]}$. Let $X$ and $Y$ be two functions on $\Omega$ defined as
\begin{align*}
    X(\omega) = 
    \begin{cases}
        1 & \text{if }\omega \in [0, 1]\\
        2 & \text{if }\omega \in (1, 2]
    \end{cases}
\end{align*}
and
\begin{align*}
    Y(\omega) = 
    \begin{cases}
        2 & \text{if }\omega \in [0, 1.5]\\
        3 & \text{if }\omega \in (1.5, 2].
    \end{cases}
\end{align*}
Then which one of the following statements is true?
\begin{enumerate}
    \item [(A)] $X$ is a random variable with respect to $\mathcal{G}$, but $Y$ is not a random variable with respect to $\mathcal{G}$.
    \item [(B)] $Y$ is a random variable with respect to $\mathcal{G}$, but $X$ is not a random variable with respect to $\mathcal{G}$.
    \item [(C)] Neither $X$ nor $Y$ is a random variable with respect to $\mathcal{G}$.
    \item [(D)] Both $X$ and $Y$ are random variables with respect to $\mathcal{G}$.
\end{enumerate} \hfill (GATE ST 2023)\\
\solution
%\input{gate/ST/2023/14/main.tex}
	\item  A die is loaded in such a way that each odd number is twice as likely to occur as
each even number. Find $P(G)$, where $G$ is the event that a number greater than
3 occurs on a single roll of the die.
\\
\solution
		%\input{exemplar/11/16/3/5/main.tex}
	\item All the jacks, queens and kings are removed from a deck of 52 playing cards. The remaining cards are well shuffled and then one card is drawn at random. Giving ace a value 1 similar value for other cards, find the probability that the card has a value 
		\begin{enumerate}
			\item 7
			\item greater than 7
			\item less than 7
		\end{enumerate}
		%\input{exemplar/10/13/3/30/main.tex}
  \item A Lot consists of 48 mobile phones of which 42 are good, 3 have only minor defects and 3 have major defects.Varnika will buy a phone if it is good but the trader will only buy a mobile if it has no major defects. One phone is selected at random from the lot. What is the probability that it is
\begin{enumerate}
	\item acceptable to Varnika?
            \item acceptable to the trader?
\end{enumerate}
\solution
	%\input{exemplar/10/13/3/40/main.tex}
 \item A student says that if you throw a die, it will show up 1 or not 1. Therefore, the probability of getting 1 and the probability of getting 'not 1' each is equal to $\frac{1}{2}$. Is this correct? Give reasons.\\
 \solution
        %\input{exemplar/10/13/2/9/main.tex}
   \item Four candidates A, B, C, D have ap-
plied for the assignment to coach a school cricket
team. If A is twice as likely to be selected as B, and
B and C are given about the same chance of being
selected, while C is twice as likely to be selected
as D, what are the probabilities that
\begin{enumerate}
\item C will be selected?
\item A will not be selected?
\end{enumerate}
	%\input{exemplar/11/16/3/9/main.tex}
 \item A bag contain 24 balls of which $x$ balls are red, $2x$ are white and $3x$ are blue. A ball is selected at random, What is the probability that it is
\begin{enumerate}[label=\alph*)]
\item not red ?
\item white ?
\end{enumerate}
%\input{exemplar/10/13/3/41/main.tex}
If the letters of the word ASSASSINATION are arranged at random. Find the Probability that
\begin{enumerate}[label=(\alph*)]
\item Four $S's$ come consecutively in the word
\item Two  $I's$ and two $N's$ come together
\item All $A's$ are not coming together
\item No two $A's$ are coming together
\end{enumerate}
%\input{exemplar/11/16/3/14/main.tex}
	\item One urn contains two black balls (labelled B1 and B2) and one white ball. A
	second urn contains one black ball and two white balls (labelled W1 and W2).
	Suppose the following experiment is performed. One of the two urns is chosen
	at random. Next a ball is randomly chosen from the urn. Then a second ball is
	chosen at random from the same urn without replacing the first ball.
	
	\begin{enumerate}
	\item What is the probability that two black balls are chosen?
	
	\item What is the probability that two balls of opposite colour are chosen?
	\end{enumerate}
	\solution
	%\input{exemplar/11/16/3/12/main1.tex}
\end{enumerate}

	\item A bag contains 4 red and 4 black balls, another bag contains 2 red and 6 black balls. One of the two bags is selected at random and a ball is drawn from the bag which is found to be red. Find the probability that the ball is drawn from the first bag.
\\
\solution
		%\begin{table}[H]
	\centering
\begin{tabular}{|c|c|c|}
\hline
Random variable &Value &Definition\\ \hline
\multirow{3}{*}{X} &0 &Slips of Rs 1\\
&1 &Slips of Rs 5\\
&2 &Slips of Rs 13\\ \hline
\multirow{2}{*}{Y} &0 &Box A\\
&1 &Box B\\\hline
\end{tabular}
\caption{}
\label{tab:Distribution}
\end{table}
See \tabref{tab:Distribution}.
\begin{align}
p_{Y}\brak{k}= \begin{cases} 
      \frac{1}{3} & {k=0} \\
      \frac{2}{3 }& {k=1} 
   \end{cases}
   \\
p_{Y|X}\brak{0|0} = \frac{19}{25}\, 
p_{Y|X}\brak{0|1} = \frac{6}{25}\,
p_{Y|X}\brak{1|0} = \frac{45}{50}\,
p_{Y|X}\brak{1|2} = \frac{5}{50}
\end{align}
The desired probability is the probability that a slip drawn at random is marked other than Rs 1,
\begin{align}
&=1-p_X\brak{0}\\
&= p_X(1) + p_X(2)
\end{align}
Using Bayes theorem,
\begin{align}
&= p_Y\brak{0} \times \pr{Y=0 | X=1} + p_Y\brak{1} \times \pr{Y=1|X=2}\\
&=\frac{1}{3} \times \frac{6}{25} + \frac{2}{3} \times \frac{5}{50}\\
&=\frac{11}{75}
\end{align}

\newpage

%\tableofcontents

\bigskip

\renewcommand{\thefigure}{\theenumi}
\renewcommand{\thetable}{\theenumi}
%\renewcommand{\theequation}{\theenumi}

%\begin{abstract}
%%\boldmath
%In this letter, an algorithm for evaluating the exact analytical bit error rate  (BER)  for the piecewise linear (PL) combiner for  multiple relays is presented. Previous results were available only for upto three relays. The algorithm is unique in the sense that  the actual mathematical expressions, that are prohibitively large, need not be explicitly obtained. The diversity gain due to multiple relays is shown through plots of the analytical BER, well supported by simulations. 
%
%\end{abstract}
% IEEEtran.cls defaults to using nonbold math in the Abstract.
% This preserves the distinction between vectors and scalars. However,
% if the journal you are submitting to favors bold math in the abstract,
% then you can use LaTeX's standard command \boldmath at the very start
% of the abstract to achieve this. Many IEEE journals frown on math
% in the abstract anyway.

% Note that keywords are not normally used for peerreview papers.
%\begin{IEEEkeywords}
%Cooperative diversity, decode and forward, piecewise linear
%\end{IEEEkeywords}



% For peer review papers, you can put extra information on the cover
% page as needed:
% \ifCLASSOPTIONpeerreview
% \begin{center} \bfseries EDICS Category: 3-BBND \end{center}
% \fi
%
% For peerreview papers, this IEEEtran command inserts a page break and
% creates the second title. It will be ignored for other modes.
%\IEEEpeerreviewmaketitle




  \item
  Cards with numbers 2 to 101 are placed in a box. A card is selected at random.Find the probability that the card has
\begin{enumerate}[label=(\roman*)]
	\item an even number 
	\item a square number
\end{enumerate}
\solution
%\begin{table}[H]
	\centering
\begin{tabular}{|c|c|c|}
\hline
Random variable &Value &Definition\\ \hline
\multirow{3}{*}{X} &0 &Slips of Rs 1\\
&1 &Slips of Rs 5\\
&2 &Slips of Rs 13\\ \hline
\multirow{2}{*}{Y} &0 &Box A\\
&1 &Box B\\\hline
\end{tabular}
\caption{}
\label{tab:Distribution}
\end{table}
See \tabref{tab:Distribution}.
\begin{align}
p_{Y}\brak{k}= \begin{cases} 
      \frac{1}{3} & {k=0} \\
      \frac{2}{3 }& {k=1} 
   \end{cases}
   \\
p_{Y|X}\brak{0|0} = \frac{19}{25}\, 
p_{Y|X}\brak{0|1} = \frac{6}{25}\,
p_{Y|X}\brak{1|0} = \frac{45}{50}\,
p_{Y|X}\brak{1|2} = \frac{5}{50}
\end{align}
The desired probability is the probability that a slip drawn at random is marked other than Rs 1,
\begin{align}
&=1-p_X\brak{0}\\
&= p_X(1) + p_X(2)
\end{align}
Using Bayes theorem,
\begin{align}
&= p_Y\brak{0} \times \pr{Y=0 | X=1} + p_Y\brak{1} \times \pr{Y=1|X=2}\\
&=\frac{1}{3} \times \frac{6}{25} + \frac{2}{3} \times \frac{5}{50}\\
&=\frac{11}{75}
\end{align}

\newpage

%\tableofcontents

\bigskip

\renewcommand{\thefigure}{\theenumi}
\renewcommand{\thetable}{\theenumi}
%\renewcommand{\theequation}{\theenumi}

%\begin{abstract}
%%\boldmath
%In this letter, an algorithm for evaluating the exact analytical bit error rate  (BER)  for the piecewise linear (PL) combiner for  multiple relays is presented. Previous results were available only for upto three relays. The algorithm is unique in the sense that  the actual mathematical expressions, that are prohibitively large, need not be explicitly obtained. The diversity gain due to multiple relays is shown through plots of the analytical BER, well supported by simulations. 
%
%\end{abstract}
% IEEEtran.cls defaults to using nonbold math in the Abstract.
% This preserves the distinction between vectors and scalars. However,
% if the journal you are submitting to favors bold math in the abstract,
% then you can use LaTeX's standard command \boldmath at the very start
% of the abstract to achieve this. Many IEEE journals frown on math
% in the abstract anyway.

% Note that keywords are not normally used for peerreview papers.
%\begin{IEEEkeywords}
%Cooperative diversity, decode and forward, piecewise linear
%\end{IEEEkeywords}



% For peer review papers, you can put extra information on the cover
% page as needed:
% \ifCLASSOPTIONpeerreview
% \begin{center} \bfseries EDICS Category: 3-BBND \end{center}
% \fi
%
% For peerreview papers, this IEEEtran command inserts a page break and
% creates the second title. It will be ignored for other modes.
%\IEEEpeerreviewmaketitle




\item
The king, queen and jack of clubs are removed from a deck of 52 playing cards and then well shuffled. Now one card is drawn at random from the remaining cards.  Determine the probability that the card is
\begin{enumerate}[label=(\roman*)]
\item a club
\item 10 of hearts
\end{enumerate}
\solution
%\begin{table}[H]
	\centering
\begin{tabular}{|c|c|c|}
\hline
Random variable &Value &Definition\\ \hline
\multirow{3}{*}{X} &0 &Slips of Rs 1\\
&1 &Slips of Rs 5\\
&2 &Slips of Rs 13\\ \hline
\multirow{2}{*}{Y} &0 &Box A\\
&1 &Box B\\\hline
\end{tabular}
\caption{}
\label{tab:Distribution}
\end{table}
See \tabref{tab:Distribution}.
\begin{align}
p_{Y}\brak{k}= \begin{cases} 
      \frac{1}{3} & {k=0} \\
      \frac{2}{3 }& {k=1} 
   \end{cases}
   \\
p_{Y|X}\brak{0|0} = \frac{19}{25}\, 
p_{Y|X}\brak{0|1} = \frac{6}{25}\,
p_{Y|X}\brak{1|0} = \frac{45}{50}\,
p_{Y|X}\brak{1|2} = \frac{5}{50}
\end{align}
The desired probability is the probability that a slip drawn at random is marked other than Rs 1,
\begin{align}
&=1-p_X\brak{0}\\
&= p_X(1) + p_X(2)
\end{align}
Using Bayes theorem,
\begin{align}
&= p_Y\brak{0} \times \pr{Y=0 | X=1} + p_Y\brak{1} \times \pr{Y=1|X=2}\\
&=\frac{1}{3} \times \frac{6}{25} + \frac{2}{3} \times \frac{5}{50}\\
&=\frac{11}{75}
\end{align}

\newpage

%\tableofcontents

\bigskip

\renewcommand{\thefigure}{\theenumi}
\renewcommand{\thetable}{\theenumi}
%\renewcommand{\theequation}{\theenumi}

%\begin{abstract}
%%\boldmath
%In this letter, an algorithm for evaluating the exact analytical bit error rate  (BER)  for the piecewise linear (PL) combiner for  multiple relays is presented. Previous results were available only for upto three relays. The algorithm is unique in the sense that  the actual mathematical expressions, that are prohibitively large, need not be explicitly obtained. The diversity gain due to multiple relays is shown through plots of the analytical BER, well supported by simulations. 
%
%\end{abstract}
% IEEEtran.cls defaults to using nonbold math in the Abstract.
% This preserves the distinction between vectors and scalars. However,
% if the journal you are submitting to favors bold math in the abstract,
% then you can use LaTeX's standard command \boldmath at the very start
% of the abstract to achieve this. Many IEEE journals frown on math
% in the abstract anyway.

% Note that keywords are not normally used for peerreview papers.
%\begin{IEEEkeywords}
%Cooperative diversity, decode and forward, piecewise linear
%\end{IEEEkeywords}



% For peer review papers, you can put extra information on the cover
% page as needed:
% \ifCLASSOPTIONpeerreview
% \begin{center} \bfseries EDICS Category: 3-BBND \end{center}
% \fi
%
% For peerreview papers, this IEEEtran command inserts a page break and
% creates the second title. It will be ignored for other modes.
%\IEEEpeerreviewmaketitle




\item A team of medical students doing their internship have to assist during surgeries
at a city hospital. The probabilities of surgeries rated as very complex, complex,
routine, simple or very simple are respectively, 0.15, 0.20, 0.31, 0.26, .08. Find
the probabilities that a particular surgery will be rated
\begin{enumerate}
	\item complex or very complex;
	\item neither very complex nor very simple;
	\item routine or complex
	\item routine or simple
\end{enumerate}
\solution
%\begin{table}[H]
	\centering
\begin{tabular}{|c|c|c|}
\hline
Random variable &Value &Definition\\ \hline
\multirow{3}{*}{X} &0 &Slips of Rs 1\\
&1 &Slips of Rs 5\\
&2 &Slips of Rs 13\\ \hline
\multirow{2}{*}{Y} &0 &Box A\\
&1 &Box B\\\hline
\end{tabular}
\caption{}
\label{tab:Distribution}
\end{table}
See \tabref{tab:Distribution}.
\begin{align}
p_{Y}\brak{k}= \begin{cases} 
      \frac{1}{3} & {k=0} \\
      \frac{2}{3 }& {k=1} 
   \end{cases}
   \\
p_{Y|X}\brak{0|0} = \frac{19}{25}\, 
p_{Y|X}\brak{0|1} = \frac{6}{25}\,
p_{Y|X}\brak{1|0} = \frac{45}{50}\,
p_{Y|X}\brak{1|2} = \frac{5}{50}
\end{align}
The desired probability is the probability that a slip drawn at random is marked other than Rs 1,
\begin{align}
&=1-p_X\brak{0}\\
&= p_X(1) + p_X(2)
\end{align}
Using Bayes theorem,
\begin{align}
&= p_Y\brak{0} \times \pr{Y=0 | X=1} + p_Y\brak{1} \times \pr{Y=1|X=2}\\
&=\frac{1}{3} \times \frac{6}{25} + \frac{2}{3} \times \frac{5}{50}\\
&=\frac{11}{75}
\end{align}

\newpage

%\tableofcontents

\bigskip

\renewcommand{\thefigure}{\theenumi}
\renewcommand{\thetable}{\theenumi}
%\renewcommand{\theequation}{\theenumi}

%\begin{abstract}
%%\boldmath
%In this letter, an algorithm for evaluating the exact analytical bit error rate  (BER)  for the piecewise linear (PL) combiner for  multiple relays is presented. Previous results were available only for upto three relays. The algorithm is unique in the sense that  the actual mathematical expressions, that are prohibitively large, need not be explicitly obtained. The diversity gain due to multiple relays is shown through plots of the analytical BER, well supported by simulations. 
%
%\end{abstract}
% IEEEtran.cls defaults to using nonbold math in the Abstract.
% This preserves the distinction between vectors and scalars. However,
% if the journal you are submitting to favors bold math in the abstract,
% then you can use LaTeX's standard command \boldmath at the very start
% of the abstract to achieve this. Many IEEE journals frown on math
% in the abstract anyway.

% Note that keywords are not normally used for peerreview papers.
%\begin{IEEEkeywords}
%Cooperative diversity, decode and forward, piecewise linear
%\end{IEEEkeywords}



% For peer review papers, you can put extra information on the cover
% page as needed:
% \ifCLASSOPTIONpeerreview
% \begin{center} \bfseries EDICS Category: 3-BBND \end{center}
% \fi
%
% For peerreview papers, this IEEEtran command inserts a page break and
% creates the second title. It will be ignored for other modes.
%\IEEEpeerreviewmaketitle




\item A card is selected from a pack of 52 cards.
\begin{enumerate}[label=(\alph*)]
    \item How many points are there in the sample space?
    \item Calculate the probability that the card is an ace of spades.
    \item Calculate the probability that the card is (i) an ace and (ii) black card.
\end{enumerate}
\solution
%Let $X$ be an bernoulli rv defined as in \tabref{tab:exemplar/11/16/3/26}.  Then, 
\begin{equation}
    p =
        \frac{4}{11} 
\end{equation}
\begin{table}[H]
	\centering
	\input{exemplar/11/16/3/26/tables/Table2.tex}
	\caption{}
        \label{tab:exemplar/11/16/3/26}
\end{table}

\item The probability that a non leap year selected at random will contain 53 sundays.
\\
\solution
%\begin{table}[H]
	\centering
\begin{tabular}{|c|c|c|}
\hline
Random variable &Value &Definition\\ \hline
\multirow{3}{*}{X} &0 &Slips of Rs 1\\
&1 &Slips of Rs 5\\
&2 &Slips of Rs 13\\ \hline
\multirow{2}{*}{Y} &0 &Box A\\
&1 &Box B\\\hline
\end{tabular}
\caption{}
\label{tab:Distribution}
\end{table}
See \tabref{tab:Distribution}.
\begin{align}
p_{Y}\brak{k}= \begin{cases} 
      \frac{1}{3} & {k=0} \\
      \frac{2}{3 }& {k=1} 
   \end{cases}
   \\
p_{Y|X}\brak{0|0} = \frac{19}{25}\, 
p_{Y|X}\brak{0|1} = \frac{6}{25}\,
p_{Y|X}\brak{1|0} = \frac{45}{50}\,
p_{Y|X}\brak{1|2} = \frac{5}{50}
\end{align}
The desired probability is the probability that a slip drawn at random is marked other than Rs 1,
\begin{align}
&=1-p_X\brak{0}\\
&= p_X(1) + p_X(2)
\end{align}
Using Bayes theorem,
\begin{align}
&= p_Y\brak{0} \times \pr{Y=0 | X=1} + p_Y\brak{1} \times \pr{Y=1|X=2}\\
&=\frac{1}{3} \times \frac{6}{25} + \frac{2}{3} \times \frac{5}{50}\\
&=\frac{11}{75}
\end{align}

\newpage

%\tableofcontents

\bigskip

\renewcommand{\thefigure}{\theenumi}
\renewcommand{\thetable}{\theenumi}
%\renewcommand{\theequation}{\theenumi}

%\begin{abstract}
%%\boldmath
%In this letter, an algorithm for evaluating the exact analytical bit error rate  (BER)  for the piecewise linear (PL) combiner for  multiple relays is presented. Previous results were available only for upto three relays. The algorithm is unique in the sense that  the actual mathematical expressions, that are prohibitively large, need not be explicitly obtained. The diversity gain due to multiple relays is shown through plots of the analytical BER, well supported by simulations. 
%
%\end{abstract}
% IEEEtran.cls defaults to using nonbold math in the Abstract.
% This preserves the distinction between vectors and scalars. However,
% if the journal you are submitting to favors bold math in the abstract,
% then you can use LaTeX's standard command \boldmath at the very start
% of the abstract to achieve this. Many IEEE journals frown on math
% in the abstract anyway.

% Note that keywords are not normally used for peerreview papers.
%\begin{IEEEkeywords}
%Cooperative diversity, decode and forward, piecewise linear
%\end{IEEEkeywords}



% For peer review papers, you can put extra information on the cover
% page as needed:
% \ifCLASSOPTIONpeerreview
% \begin{center} \bfseries EDICS Category: 3-BBND \end{center}
% \fi
%
% For peerreview papers, this IEEEtran command inserts a page break and
% creates the second title. It will be ignored for other modes.
%\IEEEpeerreviewmaketitle




\item One of the four persons John, Rita, Aslam or Gurpreet will be promoted next
month. Consequently the sample space consists of four elementary outcomes
S = {John promoted, Rita promoted, Aslam promoted, Gurpreet promoted}
You are told that the chances of John’s promotion is same as that of Gurpreet,
Rita’s chances of promotion are twice as likely as Johns. Aslam’s chances are
four times that of John.
\begin{enumerate}
	\item Determine
	\begin{enumerate}
		\item P (John promoted)
		\item P (Rita promoted)
		\item P (Aslam promoted)
		\item P (Gurpreet promoted)
	\end{enumerate}
	\item If A = {John promoted or Gurpreet promoted}, find P (A).
\end{enumerate}
\solution
%\begin{table}[H]
	\centering
\begin{tabular}{|c|c|c|}
\hline
Random variable &Value &Definition\\ \hline
\multirow{3}{*}{X} &0 &Slips of Rs 1\\
&1 &Slips of Rs 5\\
&2 &Slips of Rs 13\\ \hline
\multirow{2}{*}{Y} &0 &Box A\\
&1 &Box B\\\hline
\end{tabular}
\caption{}
\label{tab:Distribution}
\end{table}
See \tabref{tab:Distribution}.
\begin{align}
p_{Y}\brak{k}= \begin{cases} 
      \frac{1}{3} & {k=0} \\
      \frac{2}{3 }& {k=1} 
   \end{cases}
   \\
p_{Y|X}\brak{0|0} = \frac{19}{25}\, 
p_{Y|X}\brak{0|1} = \frac{6}{25}\,
p_{Y|X}\brak{1|0} = \frac{45}{50}\,
p_{Y|X}\brak{1|2} = \frac{5}{50}
\end{align}
The desired probability is the probability that a slip drawn at random is marked other than Rs 1,
\begin{align}
&=1-p_X\brak{0}\\
&= p_X(1) + p_X(2)
\end{align}
Using Bayes theorem,
\begin{align}
&= p_Y\brak{0} \times \pr{Y=0 | X=1} + p_Y\brak{1} \times \pr{Y=1|X=2}\\
&=\frac{1}{3} \times \frac{6}{25} + \frac{2}{3} \times \frac{5}{50}\\
&=\frac{11}{75}
\end{align}

\newpage

%\tableofcontents

\bigskip

\renewcommand{\thefigure}{\theenumi}
\renewcommand{\thetable}{\theenumi}
%\renewcommand{\theequation}{\theenumi}

%\begin{abstract}
%%\boldmath
%In this letter, an algorithm for evaluating the exact analytical bit error rate  (BER)  for the piecewise linear (PL) combiner for  multiple relays is presented. Previous results were available only for upto three relays. The algorithm is unique in the sense that  the actual mathematical expressions, that are prohibitively large, need not be explicitly obtained. The diversity gain due to multiple relays is shown through plots of the analytical BER, well supported by simulations. 
%
%\end{abstract}
% IEEEtran.cls defaults to using nonbold math in the Abstract.
% This preserves the distinction between vectors and scalars. However,
% if the journal you are submitting to favors bold math in the abstract,
% then you can use LaTeX's standard command \boldmath at the very start
% of the abstract to achieve this. Many IEEE journals frown on math
% in the abstract anyway.

% Note that keywords are not normally used for peerreview papers.
%\begin{IEEEkeywords}
%Cooperative diversity, decode and forward, piecewise linear
%\end{IEEEkeywords}



% For peer review papers, you can put extra information on the cover
% page as needed:
% \ifCLASSOPTIONpeerreview
% \begin{center} \bfseries EDICS Category: 3-BBND \end{center}
% \fi
%
% For peerreview papers, this IEEEtran command inserts a page break and
% creates the second title. It will be ignored for other modes.
%\IEEEpeerreviewmaketitle




\item A card is drawn from a deck of 52 cards. Find the probability of getting a king or a heart or a red card.\\
\solution
%\begin{table}[H]
	\centering
\begin{tabular}{|c|c|c|}
\hline
Random variable &Value &Definition\\ \hline
\multirow{3}{*}{X} &0 &Slips of Rs 1\\
&1 &Slips of Rs 5\\
&2 &Slips of Rs 13\\ \hline
\multirow{2}{*}{Y} &0 &Box A\\
&1 &Box B\\\hline
\end{tabular}
\caption{}
\label{tab:Distribution}
\end{table}
See \tabref{tab:Distribution}.
\begin{align}
p_{Y}\brak{k}= \begin{cases} 
      \frac{1}{3} & {k=0} \\
      \frac{2}{3 }& {k=1} 
   \end{cases}
   \\
p_{Y|X}\brak{0|0} = \frac{19}{25}\, 
p_{Y|X}\brak{0|1} = \frac{6}{25}\,
p_{Y|X}\brak{1|0} = \frac{45}{50}\,
p_{Y|X}\brak{1|2} = \frac{5}{50}
\end{align}
The desired probability is the probability that a slip drawn at random is marked other than Rs 1,
\begin{align}
&=1-p_X\brak{0}\\
&= p_X(1) + p_X(2)
\end{align}
Using Bayes theorem,
\begin{align}
&= p_Y\brak{0} \times \pr{Y=0 | X=1} + p_Y\brak{1} \times \pr{Y=1|X=2}\\
&=\frac{1}{3} \times \frac{6}{25} + \frac{2}{3} \times \frac{5}{50}\\
&=\frac{11}{75}
\end{align}

\newpage

%\tableofcontents

\bigskip

\renewcommand{\thefigure}{\theenumi}
\renewcommand{\thetable}{\theenumi}
%\renewcommand{\theequation}{\theenumi}

%\begin{abstract}
%%\boldmath
%In this letter, an algorithm for evaluating the exact analytical bit error rate  (BER)  for the piecewise linear (PL) combiner for  multiple relays is presented. Previous results were available only for upto three relays. The algorithm is unique in the sense that  the actual mathematical expressions, that are prohibitively large, need not be explicitly obtained. The diversity gain due to multiple relays is shown through plots of the analytical BER, well supported by simulations. 
%
%\end{abstract}
% IEEEtran.cls defaults to using nonbold math in the Abstract.
% This preserves the distinction between vectors and scalars. However,
% if the journal you are submitting to favors bold math in the abstract,
% then you can use LaTeX's standard command \boldmath at the very start
% of the abstract to achieve this. Many IEEE journals frown on math
% in the abstract anyway.

% Note that keywords are not normally used for peerreview papers.
%\begin{IEEEkeywords}
%Cooperative diversity, decode and forward, piecewise linear
%\end{IEEEkeywords}



% For peer review papers, you can put extra information on the cover
% page as needed:
% \ifCLASSOPTIONpeerreview
% \begin{center} \bfseries EDICS Category: 3-BBND \end{center}
% \fi
%
% For peerreview papers, this IEEEtran command inserts a page break and
% creates the second title. It will be ignored for other modes.
%\IEEEpeerreviewmaketitle




\item The probability that a student will pass his examination is 0.73, the probability of
the student getting a compartment is 0.13, and the probability that the student will
either pass or get compartment is 0.96. State True or False.\\
\solution
%\begin{table}[H]
	\centering
\begin{tabular}{|c|c|c|}
\hline
Random variable &Value &Definition\\ \hline
\multirow{3}{*}{X} &0 &Slips of Rs 1\\
&1 &Slips of Rs 5\\
&2 &Slips of Rs 13\\ \hline
\multirow{2}{*}{Y} &0 &Box A\\
&1 &Box B\\\hline
\end{tabular}
\caption{}
\label{tab:Distribution}
\end{table}
See \tabref{tab:Distribution}.
\begin{align}
p_{Y}\brak{k}= \begin{cases} 
      \frac{1}{3} & {k=0} \\
      \frac{2}{3 }& {k=1} 
   \end{cases}
   \\
p_{Y|X}\brak{0|0} = \frac{19}{25}\, 
p_{Y|X}\brak{0|1} = \frac{6}{25}\,
p_{Y|X}\brak{1|0} = \frac{45}{50}\,
p_{Y|X}\brak{1|2} = \frac{5}{50}
\end{align}
The desired probability is the probability that a slip drawn at random is marked other than Rs 1,
\begin{align}
&=1-p_X\brak{0}\\
&= p_X(1) + p_X(2)
\end{align}
Using Bayes theorem,
\begin{align}
&= p_Y\brak{0} \times \pr{Y=0 | X=1} + p_Y\brak{1} \times \pr{Y=1|X=2}\\
&=\frac{1}{3} \times \frac{6}{25} + \frac{2}{3} \times \frac{5}{50}\\
&=\frac{11}{75}
\end{align}

\newpage

%\tableofcontents

\bigskip

\renewcommand{\thefigure}{\theenumi}
\renewcommand{\thetable}{\theenumi}
%\renewcommand{\theequation}{\theenumi}

%\begin{abstract}
%%\boldmath
%In this letter, an algorithm for evaluating the exact analytical bit error rate  (BER)  for the piecewise linear (PL) combiner for  multiple relays is presented. Previous results were available only for upto three relays. The algorithm is unique in the sense that  the actual mathematical expressions, that are prohibitively large, need not be explicitly obtained. The diversity gain due to multiple relays is shown through plots of the analytical BER, well supported by simulations. 
%
%\end{abstract}
% IEEEtran.cls defaults to using nonbold math in the Abstract.
% This preserves the distinction between vectors and scalars. However,
% if the journal you are submitting to favors bold math in the abstract,
% then you can use LaTeX's standard command \boldmath at the very start
% of the abstract to achieve this. Many IEEE journals frown on math
% in the abstract anyway.

% Note that keywords are not normally used for peerreview papers.
%\begin{IEEEkeywords}
%Cooperative diversity, decode and forward, piecewise linear
%\end{IEEEkeywords}



% For peer review papers, you can put extra information on the cover
% page as needed:
% \ifCLASSOPTIONpeerreview
% \begin{center} \bfseries EDICS Category: 3-BBND \end{center}
% \fi
%
% For peerreview papers, this IEEEtran command inserts a page break and
% creates the second title. It will be ignored for other modes.
%\IEEEpeerreviewmaketitle




\item A card is selected from a pack of 52 cards\\
\begin{enumerate}[label=(\alph*)]
\item How many points are there in the sample space?
\item Calculate the probability that the cards is an ace of spades.
\item Calculate the probability that the card is (i) an ace (ii)black card.\\
\end{enumerate}
%\input{ncert/11/16/3/4_1/Prob_4.tex}
\item In a non-leap year, the probability of having 53 tuesdays or 53 wednesdays is\\
\solution
%A non-leap year has a total of 365 days, and a week has 7 days.\\
So it can be expressed as 
\begin{align}
365\text{days} &=52\times 7+1 \text{day}
\end{align}
$\implies$ 52 tuesdays or wednesdays\\
Random variable X denotes the days of a week
\begin{align}
p_X\brak{k}&=\frac{1}{7}; \quad \brak{1<k<7}
\end{align}
So the probability of extra day being tuesday or wednesday is
\begin{align}
p_X\brak{3}+p_X\brak{4}&=\frac{1}{7}+\frac{1}{7}=\frac{2}{7}
\end{align}



\item There are 1000 sealed envelopes in a box, 10 of them contain a cash prize of
Rs 100 each, 100 of them contain a cash prize of Rs 50 each and 200 of them
contain a cash prize of Rs 10 each and rest do not contain any cash prize. If they
are well shuffled and an envelope is picked up out, what is the probability that it
contains no cash prize?\\
\solution
%\begin{table}[H]
	\centering
\begin{tabular}{|c|c|c|}
\hline
Random variable &Value &Definition\\ \hline
\multirow{3}{*}{X} &0 &Slips of Rs 1\\
&1 &Slips of Rs 5\\
&2 &Slips of Rs 13\\ \hline
\multirow{2}{*}{Y} &0 &Box A\\
&1 &Box B\\\hline
\end{tabular}
\caption{}
\label{tab:Distribution}
\end{table}
See \tabref{tab:Distribution}.
\begin{align}
p_{Y}\brak{k}= \begin{cases} 
      \frac{1}{3} & {k=0} \\
      \frac{2}{3 }& {k=1} 
   \end{cases}
   \\
p_{Y|X}\brak{0|0} = \frac{19}{25}\, 
p_{Y|X}\brak{0|1} = \frac{6}{25}\,
p_{Y|X}\brak{1|0} = \frac{45}{50}\,
p_{Y|X}\brak{1|2} = \frac{5}{50}
\end{align}
The desired probability is the probability that a slip drawn at random is marked other than Rs 1,
\begin{align}
&=1-p_X\brak{0}\\
&= p_X(1) + p_X(2)
\end{align}
Using Bayes theorem,
\begin{align}
&= p_Y\brak{0} \times \pr{Y=0 | X=1} + p_Y\brak{1} \times \pr{Y=1|X=2}\\
&=\frac{1}{3} \times \frac{6}{25} + \frac{2}{3} \times \frac{5}{50}\\
&=\frac{11}{75}
\end{align}

\newpage

%\tableofcontents

\bigskip

\renewcommand{\thefigure}{\theenumi}
\renewcommand{\thetable}{\theenumi}
%\renewcommand{\theequation}{\theenumi}

%\begin{abstract}
%%\boldmath
%In this letter, an algorithm for evaluating the exact analytical bit error rate  (BER)  for the piecewise linear (PL) combiner for  multiple relays is presented. Previous results were available only for upto three relays. The algorithm is unique in the sense that  the actual mathematical expressions, that are prohibitively large, need not be explicitly obtained. The diversity gain due to multiple relays is shown through plots of the analytical BER, well supported by simulations. 
%
%\end{abstract}
% IEEEtran.cls defaults to using nonbold math in the Abstract.
% This preserves the distinction between vectors and scalars. However,
% if the journal you are submitting to favors bold math in the abstract,
% then you can use LaTeX's standard command \boldmath at the very start
% of the abstract to achieve this. Many IEEE journals frown on math
% in the abstract anyway.

% Note that keywords are not normally used for peerreview papers.
%\begin{IEEEkeywords}
%Cooperative diversity, decode and forward, piecewise linear
%\end{IEEEkeywords}



% For peer review papers, you can put extra information on the cover
% page as needed:
% \ifCLASSOPTIONpeerreview
% \begin{center} \bfseries EDICS Category: 3-BBND \end{center}
% \fi
%
% For peerreview papers, this IEEEtran command inserts a page break and
% creates the second title. It will be ignored for other modes.
%\IEEEpeerreviewmaketitle




\item 
A die is thrown and a card is selected at random from a deck of 52 playing cards. The probability of getting an even number on the die and a spade card.\\
\solution
%\begin{table}[H]
	\centering
\begin{tabular}{|c|c|c|}
\hline
Random variable &Value &Definition\\ \hline
\multirow{3}{*}{X} &0 &Slips of Rs 1\\
&1 &Slips of Rs 5\\
&2 &Slips of Rs 13\\ \hline
\multirow{2}{*}{Y} &0 &Box A\\
&1 &Box B\\\hline
\end{tabular}
\caption{}
\label{tab:Distribution}
\end{table}
See \tabref{tab:Distribution}.
\begin{align}
p_{Y}\brak{k}= \begin{cases} 
      \frac{1}{3} & {k=0} \\
      \frac{2}{3 }& {k=1} 
   \end{cases}
   \\
p_{Y|X}\brak{0|0} = \frac{19}{25}\, 
p_{Y|X}\brak{0|1} = \frac{6}{25}\,
p_{Y|X}\brak{1|0} = \frac{45}{50}\,
p_{Y|X}\brak{1|2} = \frac{5}{50}
\end{align}
The desired probability is the probability that a slip drawn at random is marked other than Rs 1,
\begin{align}
&=1-p_X\brak{0}\\
&= p_X(1) + p_X(2)
\end{align}
Using Bayes theorem,
\begin{align}
&= p_Y\brak{0} \times \pr{Y=0 | X=1} + p_Y\brak{1} \times \pr{Y=1|X=2}\\
&=\frac{1}{3} \times \frac{6}{25} + \frac{2}{3} \times \frac{5}{50}\\
&=\frac{11}{75}
\end{align}

\newpage

%\tableofcontents

\bigskip

\renewcommand{\thefigure}{\theenumi}
\renewcommand{\thetable}{\theenumi}
%\renewcommand{\theequation}{\theenumi}

%\begin{abstract}
%%\boldmath
%In this letter, an algorithm for evaluating the exact analytical bit error rate  (BER)  for the piecewise linear (PL) combiner for  multiple relays is presented. Previous results were available only for upto three relays. The algorithm is unique in the sense that  the actual mathematical expressions, that are prohibitively large, need not be explicitly obtained. The diversity gain due to multiple relays is shown through plots of the analytical BER, well supported by simulations. 
%
%\end{abstract}
% IEEEtran.cls defaults to using nonbold math in the Abstract.
% This preserves the distinction between vectors and scalars. However,
% if the journal you are submitting to favors bold math in the abstract,
% then you can use LaTeX's standard command \boldmath at the very start
% of the abstract to achieve this. Many IEEE journals frown on math
% in the abstract anyway.

% Note that keywords are not normally used for peerreview papers.
%\begin{IEEEkeywords}
%Cooperative diversity, decode and forward, piecewise linear
%\end{IEEEkeywords}



% For peer review papers, you can put extra information on the cover
% page as needed:
% \ifCLASSOPTIONpeerreview
% \begin{center} \bfseries EDICS Category: 3-BBND \end{center}
% \fi
%
% For peerreview papers, this IEEEtran command inserts a page break and
% creates the second title. It will be ignored for other modes.
%\IEEEpeerreviewmaketitle




\item
If 4-digit numbers greater than 5,000 are randomly formed from the digits 0, 1, 3, 5, and 7, what is the probability of forming a number divisible by 5 when:
\begin{enumerate}
    \item The digits are repeated?
    \item The repetition of digits is not allowed?
\end{enumerate}
\solution
%\begin{table}[H]
	\centering
\begin{tabular}{|c|c|c|}
\hline
Random variable &Value &Definition\\ \hline
\multirow{3}{*}{X} &0 &Slips of Rs 1\\
&1 &Slips of Rs 5\\
&2 &Slips of Rs 13\\ \hline
\multirow{2}{*}{Y} &0 &Box A\\
&1 &Box B\\\hline
\end{tabular}
\caption{}
\label{tab:Distribution}
\end{table}
See \tabref{tab:Distribution}.
\begin{align}
p_{Y}\brak{k}= \begin{cases} 
      \frac{1}{3} & {k=0} \\
      \frac{2}{3 }& {k=1} 
   \end{cases}
   \\
p_{Y|X}\brak{0|0} = \frac{19}{25}\, 
p_{Y|X}\brak{0|1} = \frac{6}{25}\,
p_{Y|X}\brak{1|0} = \frac{45}{50}\,
p_{Y|X}\brak{1|2} = \frac{5}{50}
\end{align}
The desired probability is the probability that a slip drawn at random is marked other than Rs 1,
\begin{align}
&=1-p_X\brak{0}\\
&= p_X(1) + p_X(2)
\end{align}
Using Bayes theorem,
\begin{align}
&= p_Y\brak{0} \times \pr{Y=0 | X=1} + p_Y\brak{1} \times \pr{Y=1|X=2}\\
&=\frac{1}{3} \times \frac{6}{25} + \frac{2}{3} \times \frac{5}{50}\\
&=\frac{11}{75}
\end{align}

\newpage

%\tableofcontents

\bigskip

\renewcommand{\thefigure}{\theenumi}
\renewcommand{\thetable}{\theenumi}
%\renewcommand{\theequation}{\theenumi}

%\begin{abstract}
%%\boldmath
%In this letter, an algorithm for evaluating the exact analytical bit error rate  (BER)  for the piecewise linear (PL) combiner for  multiple relays is presented. Previous results were available only for upto three relays. The algorithm is unique in the sense that  the actual mathematical expressions, that are prohibitively large, need not be explicitly obtained. The diversity gain due to multiple relays is shown through plots of the analytical BER, well supported by simulations. 
%
%\end{abstract}
% IEEEtran.cls defaults to using nonbold math in the Abstract.
% This preserves the distinction between vectors and scalars. However,
% if the journal you are submitting to favors bold math in the abstract,
% then you can use LaTeX's standard command \boldmath at the very start
% of the abstract to achieve this. Many IEEE journals frown on math
% in the abstract anyway.

% Note that keywords are not normally used for peerreview papers.
%\begin{IEEEkeywords}
%Cooperative diversity, decode and forward, piecewise linear
%\end{IEEEkeywords}



% For peer review papers, you can put extra information on the cover
% page as needed:
% \ifCLASSOPTIONpeerreview
% \begin{center} \bfseries EDICS Category: 3-BBND \end{center}
% \fi
%
% For peerreview papers, this IEEEtran command inserts a page break and
% creates the second title. It will be ignored for other modes.
%\IEEEpeerreviewmaketitle




\item Consider the probability space $\brak{\Omega, \mathcal{G}, P}$ where $\Omega = [0,2]$ and $\mathcal{G} = \cbrak{\phi, \Omega, [0,1], (1,2]}$. Let $X$ and $Y$ be two functions on $\Omega$ defined as
\begin{align*}
    X(\omega) = 
    \begin{cases}
        1 & \text{if }\omega \in [0, 1]\\
        2 & \text{if }\omega \in (1, 2]
    \end{cases}
\end{align*}
and
\begin{align*}
    Y(\omega) = 
    \begin{cases}
        2 & \text{if }\omega \in [0, 1.5]\\
        3 & \text{if }\omega \in (1.5, 2].
    \end{cases}
\end{align*}
Then which one of the following statements is true?
\begin{enumerate}
    \item [(A)] $X$ is a random variable with respect to $\mathcal{G}$, but $Y$ is not a random variable with respect to $\mathcal{G}$.
    \item [(B)] $Y$ is a random variable with respect to $\mathcal{G}$, but $X$ is not a random variable with respect to $\mathcal{G}$.
    \item [(C)] Neither $X$ nor $Y$ is a random variable with respect to $\mathcal{G}$.
    \item [(D)] Both $X$ and $Y$ are random variables with respect to $\mathcal{G}$.
\end{enumerate} \hfill (GATE ST 2023)\\
\solution
%\begin{table}[H]
	\centering
\begin{tabular}{|c|c|c|}
\hline
Random variable &Value &Definition\\ \hline
\multirow{3}{*}{X} &0 &Slips of Rs 1\\
&1 &Slips of Rs 5\\
&2 &Slips of Rs 13\\ \hline
\multirow{2}{*}{Y} &0 &Box A\\
&1 &Box B\\\hline
\end{tabular}
\caption{}
\label{tab:Distribution}
\end{table}
See \tabref{tab:Distribution}.
\begin{align}
p_{Y}\brak{k}= \begin{cases} 
      \frac{1}{3} & {k=0} \\
      \frac{2}{3 }& {k=1} 
   \end{cases}
   \\
p_{Y|X}\brak{0|0} = \frac{19}{25}\, 
p_{Y|X}\brak{0|1} = \frac{6}{25}\,
p_{Y|X}\brak{1|0} = \frac{45}{50}\,
p_{Y|X}\brak{1|2} = \frac{5}{50}
\end{align}
The desired probability is the probability that a slip drawn at random is marked other than Rs 1,
\begin{align}
&=1-p_X\brak{0}\\
&= p_X(1) + p_X(2)
\end{align}
Using Bayes theorem,
\begin{align}
&= p_Y\brak{0} \times \pr{Y=0 | X=1} + p_Y\brak{1} \times \pr{Y=1|X=2}\\
&=\frac{1}{3} \times \frac{6}{25} + \frac{2}{3} \times \frac{5}{50}\\
&=\frac{11}{75}
\end{align}

\newpage

%\tableofcontents

\bigskip

\renewcommand{\thefigure}{\theenumi}
\renewcommand{\thetable}{\theenumi}
%\renewcommand{\theequation}{\theenumi}

%\begin{abstract}
%%\boldmath
%In this letter, an algorithm for evaluating the exact analytical bit error rate  (BER)  for the piecewise linear (PL) combiner for  multiple relays is presented. Previous results were available only for upto three relays. The algorithm is unique in the sense that  the actual mathematical expressions, that are prohibitively large, need not be explicitly obtained. The diversity gain due to multiple relays is shown through plots of the analytical BER, well supported by simulations. 
%
%\end{abstract}
% IEEEtran.cls defaults to using nonbold math in the Abstract.
% This preserves the distinction between vectors and scalars. However,
% if the journal you are submitting to favors bold math in the abstract,
% then you can use LaTeX's standard command \boldmath at the very start
% of the abstract to achieve this. Many IEEE journals frown on math
% in the abstract anyway.

% Note that keywords are not normally used for peerreview papers.
%\begin{IEEEkeywords}
%Cooperative diversity, decode and forward, piecewise linear
%\end{IEEEkeywords}



% For peer review papers, you can put extra information on the cover
% page as needed:
% \ifCLASSOPTIONpeerreview
% \begin{center} \bfseries EDICS Category: 3-BBND \end{center}
% \fi
%
% For peerreview papers, this IEEEtran command inserts a page break and
% creates the second title. It will be ignored for other modes.
%\IEEEpeerreviewmaketitle




	\item  A die is loaded in such a way that each odd number is twice as likely to occur as
each even number. Find $P(G)$, where $G$ is the event that a number greater than
3 occurs on a single roll of the die.
\\
\solution
		%\begin{table}[H]
	\centering
\begin{tabular}{|c|c|c|}
\hline
Random variable &Value &Definition\\ \hline
\multirow{3}{*}{X} &0 &Slips of Rs 1\\
&1 &Slips of Rs 5\\
&2 &Slips of Rs 13\\ \hline
\multirow{2}{*}{Y} &0 &Box A\\
&1 &Box B\\\hline
\end{tabular}
\caption{}
\label{tab:Distribution}
\end{table}
See \tabref{tab:Distribution}.
\begin{align}
p_{Y}\brak{k}= \begin{cases} 
      \frac{1}{3} & {k=0} \\
      \frac{2}{3 }& {k=1} 
   \end{cases}
   \\
p_{Y|X}\brak{0|0} = \frac{19}{25}\, 
p_{Y|X}\brak{0|1} = \frac{6}{25}\,
p_{Y|X}\brak{1|0} = \frac{45}{50}\,
p_{Y|X}\brak{1|2} = \frac{5}{50}
\end{align}
The desired probability is the probability that a slip drawn at random is marked other than Rs 1,
\begin{align}
&=1-p_X\brak{0}\\
&= p_X(1) + p_X(2)
\end{align}
Using Bayes theorem,
\begin{align}
&= p_Y\brak{0} \times \pr{Y=0 | X=1} + p_Y\brak{1} \times \pr{Y=1|X=2}\\
&=\frac{1}{3} \times \frac{6}{25} + \frac{2}{3} \times \frac{5}{50}\\
&=\frac{11}{75}
\end{align}

\newpage

%\tableofcontents

\bigskip

\renewcommand{\thefigure}{\theenumi}
\renewcommand{\thetable}{\theenumi}
%\renewcommand{\theequation}{\theenumi}

%\begin{abstract}
%%\boldmath
%In this letter, an algorithm for evaluating the exact analytical bit error rate  (BER)  for the piecewise linear (PL) combiner for  multiple relays is presented. Previous results were available only for upto three relays. The algorithm is unique in the sense that  the actual mathematical expressions, that are prohibitively large, need not be explicitly obtained. The diversity gain due to multiple relays is shown through plots of the analytical BER, well supported by simulations. 
%
%\end{abstract}
% IEEEtran.cls defaults to using nonbold math in the Abstract.
% This preserves the distinction between vectors and scalars. However,
% if the journal you are submitting to favors bold math in the abstract,
% then you can use LaTeX's standard command \boldmath at the very start
% of the abstract to achieve this. Many IEEE journals frown on math
% in the abstract anyway.

% Note that keywords are not normally used for peerreview papers.
%\begin{IEEEkeywords}
%Cooperative diversity, decode and forward, piecewise linear
%\end{IEEEkeywords}



% For peer review papers, you can put extra information on the cover
% page as needed:
% \ifCLASSOPTIONpeerreview
% \begin{center} \bfseries EDICS Category: 3-BBND \end{center}
% \fi
%
% For peerreview papers, this IEEEtran command inserts a page break and
% creates the second title. It will be ignored for other modes.
%\IEEEpeerreviewmaketitle




	\item All the jacks, queens and kings are removed from a deck of 52 playing cards. The remaining cards are well shuffled and then one card is drawn at random. Giving ace a value 1 similar value for other cards, find the probability that the card has a value 
		\begin{enumerate}
			\item 7
			\item greater than 7
			\item less than 7
		\end{enumerate}
		%Number of cards left after removing all jacks, queens and kings 
\begin{align}
N	= 52 - 4\times 3
	= 40
\end{align}
%\begin{table}[H]
%\def\arraystretch{1.2}
%\begin{tabular}{|c|c|c|}
%\hline
%	\textbf{Parameter} &\textbf{Value} &\textbf{Description}\\ \hline
%	$X$ &1-10 &Represents the value of the card picked \\ \hline
%\end{tabular}
%\end{table}
Let $1 \le X \le 10$ be the value of the card picked.  Then,
\begin{align}
	p_X(k) &= \Pr(X=k)\ \forall\ 1 \leq k \leq 10\\
	&= \frac{4\times 1}{40}\\
	&= \frac{1}{10}\\
	\therefore p_X(k) &= 
	\begin{cases}
		\frac{1}{10} & 1 \leq k \leq 10\\
		0 & \text{otherwise}
	\end{cases}
\end{align}
and
\begin{align}
	F_{X}(k) &= \sum_{m=0}^{k}p_{X}(m) \quad 1 \leq k \leq 10\\
	&= \frac{k}{10}\\
	\therefore F_{X}(k) &= 
	\begin{cases}
		0 & k \leq 0\\
		\frac{k}{10} & 1\leq k \leq 10\\
		1 & k > 10 
	\end{cases}
\end{align}
\begin{enumerate}
	\item Probability that card has value equal to 7 is
		\begin{align}
			 p_{X}(7)
			= \frac{1}{10}
		\end{align}
	\item Probability that card has value greater than 7 is
		\begin{align}
			1 - F_X(7)
			&= 1 - \frac{7}{10}
			\\
			&= \frac{3}{10}
		\end{align}
	\item Probability that card has value less than 7 is
		\begin{align}
			 F_{X}(6)
			=\frac{6}{10}
		\end{align}
\end{enumerate}

  \item A Lot consists of 48 mobile phones of which 42 are good, 3 have only minor defects and 3 have major defects.Varnika will buy a phone if it is good but the trader will only buy a mobile if it has no major defects. One phone is selected at random from the lot. What is the probability that it is
\begin{enumerate}
	\item acceptable to Varnika?
            \item acceptable to the trader?
\end{enumerate}
\solution
	%\begin{table}[H]
	\centering
\begin{tabular}{|c|c|c|}
\hline
Random variable &Value &Definition\\ \hline
\multirow{3}{*}{X} &0 &Slips of Rs 1\\
&1 &Slips of Rs 5\\
&2 &Slips of Rs 13\\ \hline
\multirow{2}{*}{Y} &0 &Box A\\
&1 &Box B\\\hline
\end{tabular}
\caption{}
\label{tab:Distribution}
\end{table}
See \tabref{tab:Distribution}.
\begin{align}
p_{Y}\brak{k}= \begin{cases} 
      \frac{1}{3} & {k=0} \\
      \frac{2}{3 }& {k=1} 
   \end{cases}
   \\
p_{Y|X}\brak{0|0} = \frac{19}{25}\, 
p_{Y|X}\brak{0|1} = \frac{6}{25}\,
p_{Y|X}\brak{1|0} = \frac{45}{50}\,
p_{Y|X}\brak{1|2} = \frac{5}{50}
\end{align}
The desired probability is the probability that a slip drawn at random is marked other than Rs 1,
\begin{align}
&=1-p_X\brak{0}\\
&= p_X(1) + p_X(2)
\end{align}
Using Bayes theorem,
\begin{align}
&= p_Y\brak{0} \times \pr{Y=0 | X=1} + p_Y\brak{1} \times \pr{Y=1|X=2}\\
&=\frac{1}{3} \times \frac{6}{25} + \frac{2}{3} \times \frac{5}{50}\\
&=\frac{11}{75}
\end{align}

\newpage

%\tableofcontents

\bigskip

\renewcommand{\thefigure}{\theenumi}
\renewcommand{\thetable}{\theenumi}
%\renewcommand{\theequation}{\theenumi}

%\begin{abstract}
%%\boldmath
%In this letter, an algorithm for evaluating the exact analytical bit error rate  (BER)  for the piecewise linear (PL) combiner for  multiple relays is presented. Previous results were available only for upto three relays. The algorithm is unique in the sense that  the actual mathematical expressions, that are prohibitively large, need not be explicitly obtained. The diversity gain due to multiple relays is shown through plots of the analytical BER, well supported by simulations. 
%
%\end{abstract}
% IEEEtran.cls defaults to using nonbold math in the Abstract.
% This preserves the distinction between vectors and scalars. However,
% if the journal you are submitting to favors bold math in the abstract,
% then you can use LaTeX's standard command \boldmath at the very start
% of the abstract to achieve this. Many IEEE journals frown on math
% in the abstract anyway.

% Note that keywords are not normally used for peerreview papers.
%\begin{IEEEkeywords}
%Cooperative diversity, decode and forward, piecewise linear
%\end{IEEEkeywords}



% For peer review papers, you can put extra information on the cover
% page as needed:
% \ifCLASSOPTIONpeerreview
% \begin{center} \bfseries EDICS Category: 3-BBND \end{center}
% \fi
%
% For peerreview papers, this IEEEtran command inserts a page break and
% creates the second title. It will be ignored for other modes.
%\IEEEpeerreviewmaketitle




 \item A student says that if you throw a die, it will show up 1 or not 1. Therefore, the probability of getting 1 and the probability of getting 'not 1' each is equal to $\frac{1}{2}$. Is this correct? Give reasons.\\
 \solution
        %\begin{table}[H]
	\centering
\begin{tabular}{|c|c|c|}
\hline
Random variable &Value &Definition\\ \hline
\multirow{3}{*}{X} &0 &Slips of Rs 1\\
&1 &Slips of Rs 5\\
&2 &Slips of Rs 13\\ \hline
\multirow{2}{*}{Y} &0 &Box A\\
&1 &Box B\\\hline
\end{tabular}
\caption{}
\label{tab:Distribution}
\end{table}
See \tabref{tab:Distribution}.
\begin{align}
p_{Y}\brak{k}= \begin{cases} 
      \frac{1}{3} & {k=0} \\
      \frac{2}{3 }& {k=1} 
   \end{cases}
   \\
p_{Y|X}\brak{0|0} = \frac{19}{25}\, 
p_{Y|X}\brak{0|1} = \frac{6}{25}\,
p_{Y|X}\brak{1|0} = \frac{45}{50}\,
p_{Y|X}\brak{1|2} = \frac{5}{50}
\end{align}
The desired probability is the probability that a slip drawn at random is marked other than Rs 1,
\begin{align}
&=1-p_X\brak{0}\\
&= p_X(1) + p_X(2)
\end{align}
Using Bayes theorem,
\begin{align}
&= p_Y\brak{0} \times \pr{Y=0 | X=1} + p_Y\brak{1} \times \pr{Y=1|X=2}\\
&=\frac{1}{3} \times \frac{6}{25} + \frac{2}{3} \times \frac{5}{50}\\
&=\frac{11}{75}
\end{align}

\newpage

%\tableofcontents

\bigskip

\renewcommand{\thefigure}{\theenumi}
\renewcommand{\thetable}{\theenumi}
%\renewcommand{\theequation}{\theenumi}

%\begin{abstract}
%%\boldmath
%In this letter, an algorithm for evaluating the exact analytical bit error rate  (BER)  for the piecewise linear (PL) combiner for  multiple relays is presented. Previous results were available only for upto three relays. The algorithm is unique in the sense that  the actual mathematical expressions, that are prohibitively large, need not be explicitly obtained. The diversity gain due to multiple relays is shown through plots of the analytical BER, well supported by simulations. 
%
%\end{abstract}
% IEEEtran.cls defaults to using nonbold math in the Abstract.
% This preserves the distinction between vectors and scalars. However,
% if the journal you are submitting to favors bold math in the abstract,
% then you can use LaTeX's standard command \boldmath at the very start
% of the abstract to achieve this. Many IEEE journals frown on math
% in the abstract anyway.

% Note that keywords are not normally used for peerreview papers.
%\begin{IEEEkeywords}
%Cooperative diversity, decode and forward, piecewise linear
%\end{IEEEkeywords}



% For peer review papers, you can put extra information on the cover
% page as needed:
% \ifCLASSOPTIONpeerreview
% \begin{center} \bfseries EDICS Category: 3-BBND \end{center}
% \fi
%
% For peerreview papers, this IEEEtran command inserts a page break and
% creates the second title. It will be ignored for other modes.
%\IEEEpeerreviewmaketitle




   \item Four candidates A, B, C, D have ap-
plied for the assignment to coach a school cricket
team. If A is twice as likely to be selected as B, and
B and C are given about the same chance of being
selected, while C is twice as likely to be selected
as D, what are the probabilities that
\begin{enumerate}
\item C will be selected?
\item A will not be selected?
\end{enumerate}
	%\begin{table}[H]
	\centering
\begin{tabular}{|c|c|c|}
\hline
Random variable &Value &Definition\\ \hline
\multirow{3}{*}{X} &0 &Slips of Rs 1\\
&1 &Slips of Rs 5\\
&2 &Slips of Rs 13\\ \hline
\multirow{2}{*}{Y} &0 &Box A\\
&1 &Box B\\\hline
\end{tabular}
\caption{}
\label{tab:Distribution}
\end{table}
See \tabref{tab:Distribution}.
\begin{align}
p_{Y}\brak{k}= \begin{cases} 
      \frac{1}{3} & {k=0} \\
      \frac{2}{3 }& {k=1} 
   \end{cases}
   \\
p_{Y|X}\brak{0|0} = \frac{19}{25}\, 
p_{Y|X}\brak{0|1} = \frac{6}{25}\,
p_{Y|X}\brak{1|0} = \frac{45}{50}\,
p_{Y|X}\brak{1|2} = \frac{5}{50}
\end{align}
The desired probability is the probability that a slip drawn at random is marked other than Rs 1,
\begin{align}
&=1-p_X\brak{0}\\
&= p_X(1) + p_X(2)
\end{align}
Using Bayes theorem,
\begin{align}
&= p_Y\brak{0} \times \pr{Y=0 | X=1} + p_Y\brak{1} \times \pr{Y=1|X=2}\\
&=\frac{1}{3} \times \frac{6}{25} + \frac{2}{3} \times \frac{5}{50}\\
&=\frac{11}{75}
\end{align}

\newpage

%\tableofcontents

\bigskip

\renewcommand{\thefigure}{\theenumi}
\renewcommand{\thetable}{\theenumi}
%\renewcommand{\theequation}{\theenumi}

%\begin{abstract}
%%\boldmath
%In this letter, an algorithm for evaluating the exact analytical bit error rate  (BER)  for the piecewise linear (PL) combiner for  multiple relays is presented. Previous results were available only for upto three relays. The algorithm is unique in the sense that  the actual mathematical expressions, that are prohibitively large, need not be explicitly obtained. The diversity gain due to multiple relays is shown through plots of the analytical BER, well supported by simulations. 
%
%\end{abstract}
% IEEEtran.cls defaults to using nonbold math in the Abstract.
% This preserves the distinction between vectors and scalars. However,
% if the journal you are submitting to favors bold math in the abstract,
% then you can use LaTeX's standard command \boldmath at the very start
% of the abstract to achieve this. Many IEEE journals frown on math
% in the abstract anyway.

% Note that keywords are not normally used for peerreview papers.
%\begin{IEEEkeywords}
%Cooperative diversity, decode and forward, piecewise linear
%\end{IEEEkeywords}



% For peer review papers, you can put extra information on the cover
% page as needed:
% \ifCLASSOPTIONpeerreview
% \begin{center} \bfseries EDICS Category: 3-BBND \end{center}
% \fi
%
% For peerreview papers, this IEEEtran command inserts a page break and
% creates the second title. It will be ignored for other modes.
%\IEEEpeerreviewmaketitle




 \item A bag contain 24 balls of which $x$ balls are red, $2x$ are white and $3x$ are blue. A ball is selected at random, What is the probability that it is
\begin{enumerate}[label=\alph*)]
\item not red ?
\item white ?
\end{enumerate}
%\begin{table}[H]
	\centering
\begin{tabular}{|c|c|c|}
\hline
Random variable &Value &Definition\\ \hline
\multirow{3}{*}{X} &0 &Slips of Rs 1\\
&1 &Slips of Rs 5\\
&2 &Slips of Rs 13\\ \hline
\multirow{2}{*}{Y} &0 &Box A\\
&1 &Box B\\\hline
\end{tabular}
\caption{}
\label{tab:Distribution}
\end{table}
See \tabref{tab:Distribution}.
\begin{align}
p_{Y}\brak{k}= \begin{cases} 
      \frac{1}{3} & {k=0} \\
      \frac{2}{3 }& {k=1} 
   \end{cases}
   \\
p_{Y|X}\brak{0|0} = \frac{19}{25}\, 
p_{Y|X}\brak{0|1} = \frac{6}{25}\,
p_{Y|X}\brak{1|0} = \frac{45}{50}\,
p_{Y|X}\brak{1|2} = \frac{5}{50}
\end{align}
The desired probability is the probability that a slip drawn at random is marked other than Rs 1,
\begin{align}
&=1-p_X\brak{0}\\
&= p_X(1) + p_X(2)
\end{align}
Using Bayes theorem,
\begin{align}
&= p_Y\brak{0} \times \pr{Y=0 | X=1} + p_Y\brak{1} \times \pr{Y=1|X=2}\\
&=\frac{1}{3} \times \frac{6}{25} + \frac{2}{3} \times \frac{5}{50}\\
&=\frac{11}{75}
\end{align}

\newpage

%\tableofcontents

\bigskip

\renewcommand{\thefigure}{\theenumi}
\renewcommand{\thetable}{\theenumi}
%\renewcommand{\theequation}{\theenumi}

%\begin{abstract}
%%\boldmath
%In this letter, an algorithm for evaluating the exact analytical bit error rate  (BER)  for the piecewise linear (PL) combiner for  multiple relays is presented. Previous results were available only for upto three relays. The algorithm is unique in the sense that  the actual mathematical expressions, that are prohibitively large, need not be explicitly obtained. The diversity gain due to multiple relays is shown through plots of the analytical BER, well supported by simulations. 
%
%\end{abstract}
% IEEEtran.cls defaults to using nonbold math in the Abstract.
% This preserves the distinction between vectors and scalars. However,
% if the journal you are submitting to favors bold math in the abstract,
% then you can use LaTeX's standard command \boldmath at the very start
% of the abstract to achieve this. Many IEEE journals frown on math
% in the abstract anyway.

% Note that keywords are not normally used for peerreview papers.
%\begin{IEEEkeywords}
%Cooperative diversity, decode and forward, piecewise linear
%\end{IEEEkeywords}



% For peer review papers, you can put extra information on the cover
% page as needed:
% \ifCLASSOPTIONpeerreview
% \begin{center} \bfseries EDICS Category: 3-BBND \end{center}
% \fi
%
% For peerreview papers, this IEEEtran command inserts a page break and
% creates the second title. It will be ignored for other modes.
%\IEEEpeerreviewmaketitle




If the letters of the word ASSASSINATION are arranged at random. Find the Probability that
\begin{enumerate}[label=(\alph*)]
\item Four $S's$ come consecutively in the word
\item Two  $I's$ and two $N's$ come together
\item All $A's$ are not coming together
\item No two $A's$ are coming together
\end{enumerate}
%\begin{table}[H]
	\centering
\begin{tabular}{|c|c|c|}
\hline
Random variable &Value &Definition\\ \hline
\multirow{3}{*}{X} &0 &Slips of Rs 1\\
&1 &Slips of Rs 5\\
&2 &Slips of Rs 13\\ \hline
\multirow{2}{*}{Y} &0 &Box A\\
&1 &Box B\\\hline
\end{tabular}
\caption{}
\label{tab:Distribution}
\end{table}
See \tabref{tab:Distribution}.
\begin{align}
p_{Y}\brak{k}= \begin{cases} 
      \frac{1}{3} & {k=0} \\
      \frac{2}{3 }& {k=1} 
   \end{cases}
   \\
p_{Y|X}\brak{0|0} = \frac{19}{25}\, 
p_{Y|X}\brak{0|1} = \frac{6}{25}\,
p_{Y|X}\brak{1|0} = \frac{45}{50}\,
p_{Y|X}\brak{1|2} = \frac{5}{50}
\end{align}
The desired probability is the probability that a slip drawn at random is marked other than Rs 1,
\begin{align}
&=1-p_X\brak{0}\\
&= p_X(1) + p_X(2)
\end{align}
Using Bayes theorem,
\begin{align}
&= p_Y\brak{0} \times \pr{Y=0 | X=1} + p_Y\brak{1} \times \pr{Y=1|X=2}\\
&=\frac{1}{3} \times \frac{6}{25} + \frac{2}{3} \times \frac{5}{50}\\
&=\frac{11}{75}
\end{align}

\newpage

%\tableofcontents

\bigskip

\renewcommand{\thefigure}{\theenumi}
\renewcommand{\thetable}{\theenumi}
%\renewcommand{\theequation}{\theenumi}

%\begin{abstract}
%%\boldmath
%In this letter, an algorithm for evaluating the exact analytical bit error rate  (BER)  for the piecewise linear (PL) combiner for  multiple relays is presented. Previous results were available only for upto three relays. The algorithm is unique in the sense that  the actual mathematical expressions, that are prohibitively large, need not be explicitly obtained. The diversity gain due to multiple relays is shown through plots of the analytical BER, well supported by simulations. 
%
%\end{abstract}
% IEEEtran.cls defaults to using nonbold math in the Abstract.
% This preserves the distinction between vectors and scalars. However,
% if the journal you are submitting to favors bold math in the abstract,
% then you can use LaTeX's standard command \boldmath at the very start
% of the abstract to achieve this. Many IEEE journals frown on math
% in the abstract anyway.

% Note that keywords are not normally used for peerreview papers.
%\begin{IEEEkeywords}
%Cooperative diversity, decode and forward, piecewise linear
%\end{IEEEkeywords}



% For peer review papers, you can put extra information on the cover
% page as needed:
% \ifCLASSOPTIONpeerreview
% \begin{center} \bfseries EDICS Category: 3-BBND \end{center}
% \fi
%
% For peerreview papers, this IEEEtran command inserts a page break and
% creates the second title. It will be ignored for other modes.
%\IEEEpeerreviewmaketitle




	\item One urn contains two black balls (labelled B1 and B2) and one white ball. A
	second urn contains one black ball and two white balls (labelled W1 and W2).
	Suppose the following experiment is performed. One of the two urns is chosen
	at random. Next a ball is randomly chosen from the urn. Then a second ball is
	chosen at random from the same urn without replacing the first ball.
	
	\begin{enumerate}
	\item What is the probability that two black balls are chosen?
	
	\item What is the probability that two balls of opposite colour are chosen?
	\end{enumerate}
	\solution
	%\begin{align}
    \label{eq:12.13.6.18.1}
	\because	\pr{A|B} &> \pr{A},\
\frac{\pr{AB}}{\pr{B}} > \pr{A}
\\
    \label{eq:12.13.6.18.2}
	\implies \pr{AB} &> \pr{A}\pr{B}
	\\
	\text{or, } \frac{\pr{AB}}{\pr{A}} &=\pr{B|A} > \pr{A}
\end{align}

\end{enumerate}

		\item A box of oranges is inspected by examining three randomly selected oranges drawn without replacement. If all the three oranges are good, the box is approved for sale, otherwise, it is rejected. Find the probability that a box containing 15 oranges out of which 12 are good and 3 are bad ones will be approved for sale.
		\label{ncert/12/13/2/3/defs.tex}
		\item Two balls are drawn at random with replacement from a box containing 10 black and 8 red balls. Find the probability that
		\label{ncert/12/13/2/12}
\begin{enumerate}
\item both balls are red.
\item first ball is black and second is red.
\item one of them is black and other is red.
\end{enumerate}

\item In a hostel, 60\% of the students read Hindi newspaper, 40\% read English newspaper and 20\% read both Hindi and English newspapers. A student is selected at random.
		\label{ncert/12/13/2/15}
\begin{enumerate}
\item Find the probability that she reads neither Hindi nor English newspapers.
\item If she reads Hindi newspaper, find the probability that she reads English newspaper.
\item If she reads English newspaper, find the probability that she reads Hindi newspaper.\\
\end{enumerate}
\item The probability of obtaining an even prime number on each die, when a pair of dice is rolled is 
\begin{enumerate}
    \item $0$ 
    
    \item $\frac{1}{3}$ 
    
    \item $\frac{1}{12}$ 
    
    \item $\frac{1}{36}$ 
\end{enumerate}
\solution
		%\begin{enumerate}[label=\thesection.\arabic*,ref=\thesection.\theenumi]
	\item One card is drawn from a well-shuffled deck of 52 cards. Find the probability of getting
\begin{enumerate}
\item A king of red colour 
\item A face card 
\item A red face card
\item The jack of hearts
\item A spade
\item The queen of diamonds

\end{enumerate}
\solution
		%\begin{table}[H]
	\centering
\begin{tabular}{|c|c|c|}
\hline
Random variable &Value &Definition\\ \hline
\multirow{3}{*}{X} &0 &Slips of Rs 1\\
&1 &Slips of Rs 5\\
&2 &Slips of Rs 13\\ \hline
\multirow{2}{*}{Y} &0 &Box A\\
&1 &Box B\\\hline
\end{tabular}
\caption{}
\label{tab:Distribution}
\end{table}
See \tabref{tab:Distribution}.
\begin{align}
p_{Y}\brak{k}= \begin{cases} 
      \frac{1}{3} & {k=0} \\
      \frac{2}{3 }& {k=1} 
   \end{cases}
   \\
p_{Y|X}\brak{0|0} = \frac{19}{25}\, 
p_{Y|X}\brak{0|1} = \frac{6}{25}\,
p_{Y|X}\brak{1|0} = \frac{45}{50}\,
p_{Y|X}\brak{1|2} = \frac{5}{50}
\end{align}
The desired probability is the probability that a slip drawn at random is marked other than Rs 1,
\begin{align}
&=1-p_X\brak{0}\\
&= p_X(1) + p_X(2)
\end{align}
Using Bayes theorem,
\begin{align}
&= p_Y\brak{0} \times \pr{Y=0 | X=1} + p_Y\brak{1} \times \pr{Y=1|X=2}\\
&=\frac{1}{3} \times \frac{6}{25} + \frac{2}{3} \times \frac{5}{50}\\
&=\frac{11}{75}
\end{align}

\newpage

%\tableofcontents

\bigskip

\renewcommand{\thefigure}{\theenumi}
\renewcommand{\thetable}{\theenumi}
%\renewcommand{\theequation}{\theenumi}

%\begin{abstract}
%%\boldmath
%In this letter, an algorithm for evaluating the exact analytical bit error rate  (BER)  for the piecewise linear (PL) combiner for  multiple relays is presented. Previous results were available only for upto three relays. The algorithm is unique in the sense that  the actual mathematical expressions, that are prohibitively large, need not be explicitly obtained. The diversity gain due to multiple relays is shown through plots of the analytical BER, well supported by simulations. 
%
%\end{abstract}
% IEEEtran.cls defaults to using nonbold math in the Abstract.
% This preserves the distinction between vectors and scalars. However,
% if the journal you are submitting to favors bold math in the abstract,
% then you can use LaTeX's standard command \boldmath at the very start
% of the abstract to achieve this. Many IEEE journals frown on math
% in the abstract anyway.

% Note that keywords are not normally used for peerreview papers.
%\begin{IEEEkeywords}
%Cooperative diversity, decode and forward, piecewise linear
%\end{IEEEkeywords}



% For peer review papers, you can put extra information on the cover
% page as needed:
% \ifCLASSOPTIONpeerreview
% \begin{center} \bfseries EDICS Category: 3-BBND \end{center}
% \fi
%
% For peerreview papers, this IEEEtran command inserts a page break and
% creates the second title. It will be ignored for other modes.
%\IEEEpeerreviewmaketitle




	\item Five cards—the ten, jack, queen, king and ace of diamonds, are well-shuffled with their face downwards. One card is then picked up at random.
\begin{enumerate}
\item
What is the probability that the card is the queen? 
\item
If the queen is drawn and put aside, what is the probability that the second card picked up is (a) an ace? (b) a queen?\\
\end{enumerate}
\solution
		%\begin{enumerate}[label=\thesection.\arabic*,ref=\thesection.\theenumi]
	\item One card is drawn from a well-shuffled deck of 52 cards. Find the probability of getting
\begin{enumerate}
\item A king of red colour 
\item A face card 
\item A red face card
\item The jack of hearts
\item A spade
\item The queen of diamonds

\end{enumerate}
\solution
		%\input{ncert/10/15/1/14/main.tex}
	\item Five cards—the ten, jack, queen, king and ace of diamonds, are well-shuffled with their face downwards. One card is then picked up at random.
\begin{enumerate}
\item
What is the probability that the card is the queen? 
\item
If the queen is drawn and put aside, what is the probability that the second card picked up is (a) an ace? (b) a queen?\\
\end{enumerate}
\solution
		%\input{ncert/10/15/1/15/defs.tex}
	\item A bag contains $5$ red balls and some blue balls. If the probability of drawing a blue ball is double that if a red ball, determine the number of blue balls in the bag. 
		\\
\solution
		%\input{ncert/10/15/2/3/defs.tex}
	\item A card is selected from a pack of 52 cards.
 \begin{enumerate}[label=(\alph*)] 
                 \item How many points are there in the sample space?
                 \item Calculate the probability that the card is an ace of spades.
                 \item Calculate the probability that the card is (i) an ace and (ii) black card.
 \end{enumerate}
\solution
		%\input{ncert/11/16/3/4/main.tex}
\item Four cards are drawn from a well-shuffled deck of 52 cards. What is the probability of obtaining 3 diamonds and one spade.
\\
\solution
		%\input{ncert/11/16/4/2/defs.tex}
\item In a certain lottery 10,000 tickets are sold and ten equal prizes are awarded. What is the probability of not getting a prize if you buy (a) one ticket (b) two tickets (c) 10 tickets ?	
\\
\solution
		%\input{ncert/11/16/4/4/defs.tex}
		%
\item 
Out of 100 students, two sections of 40 and 60 are formed. If you and your friend are among the 100 students, what is the probability that
\begin{enumerate}
\item you both enter the same section?
\item you both enter the different sections?
\end{enumerate}
\solution
		%\input{ncert/11/16/4/5/defs.tex}
	\item 
The number lock of a suitcase has 4 wheels each labelled with ten digits i.e. from 0 to 9.The lock opens with a sequence of four digits with no repeats.What is the probability of a person getting the right sequence to open the suitcase.
\\
\solution
		%\input{ncert/11/16/4/10/defs.tex}
		%
\item 
Two cards are drawn at random and without replacement from a pack of 52 playing cards. Find the probability that both the cards are black.
\\
\solution
		%\input{ncert/12/13/2/2/defs.tex}
		\item A box of oranges is inspected by examining three randomly selected oranges drawn without replacement. If all the three oranges are good, the box is approved for sale, otherwise, it is rejected. Find the probability that a box containing 15 oranges out of which 12 are good and 3 are bad ones will be approved for sale.
		\label{ncert/12/13/2/3/defs.tex}
		\item Two balls are drawn at random with replacement from a box containing 10 black and 8 red balls. Find the probability that
		\label{ncert/12/13/2/12}
\begin{enumerate}
\item both balls are red.
\item first ball is black and second is red.
\item one of them is black and other is red.
\end{enumerate}

\item In a hostel, 60\% of the students read Hindi newspaper, 40\% read English newspaper and 20\% read both Hindi and English newspapers. A student is selected at random.
		\label{ncert/12/13/2/15}
\begin{enumerate}
\item Find the probability that she reads neither Hindi nor English newspapers.
\item If she reads Hindi newspaper, find the probability that she reads English newspaper.
\item If she reads English newspaper, find the probability that she reads Hindi newspaper.\\
\end{enumerate}
\item The probability of obtaining an even prime number on each die, when a pair of dice is rolled is 
\begin{enumerate}
    \item $0$ 
    
    \item $\frac{1}{3}$ 
    
    \item $\frac{1}{12}$ 
    
    \item $\frac{1}{36}$ 
\end{enumerate}
\solution
		%\input{ncert/12/13/2/17/defs.tex}
	\item A bag contains 4 red and 4 black balls, another bag contains 2 red and 6 black balls. One of the two bags is selected at random and a ball is drawn from the bag which is found to be red. Find the probability that the ball is drawn from the first bag.
\\
\solution
		%\input{ncert/12/13/3/2/main.tex}
  \item
  Cards with numbers 2 to 101 are placed in a box. A card is selected at random.Find the probability that the card has
\begin{enumerate}[label=(\roman*)]
	\item an even number 
	\item a square number
\end{enumerate}
\solution
%\input{exemplar/10/13/3/32/main.tex}
\item
The king, queen and jack of clubs are removed from a deck of 52 playing cards and then well shuffled. Now one card is drawn at random from the remaining cards.  Determine the probability that the card is
\begin{enumerate}[label=(\roman*)]
\item a club
\item 10 of hearts
\end{enumerate}
\solution
%\input{exemplar/10/13/3/29/main.tex}
\item A team of medical students doing their internship have to assist during surgeries
at a city hospital. The probabilities of surgeries rated as very complex, complex,
routine, simple or very simple are respectively, 0.15, 0.20, 0.31, 0.26, .08. Find
the probabilities that a particular surgery will be rated
\begin{enumerate}
	\item complex or very complex;
	\item neither very complex nor very simple;
	\item routine or complex
	\item routine or simple
\end{enumerate}
\solution
%\input{exemplar/11/16/3/8(1)/main.tex}
\item A card is selected from a pack of 52 cards.
\begin{enumerate}[label=(\alph*)]
    \item How many points are there in the sample space?
    \item Calculate the probability that the card is an ace of spades.
    \item Calculate the probability that the card is (i) an ace and (ii) black card.
\end{enumerate}
\solution
%\input{exemplar/11/16/3/4/main2.tex}
\item The probability that a non leap year selected at random will contain 53 sundays.
\\
\solution
%\input{exemplar/10/13/1/19/main.tex}
\item One of the four persons John, Rita, Aslam or Gurpreet will be promoted next
month. Consequently the sample space consists of four elementary outcomes
S = {John promoted, Rita promoted, Aslam promoted, Gurpreet promoted}
You are told that the chances of John’s promotion is same as that of Gurpreet,
Rita’s chances of promotion are twice as likely as Johns. Aslam’s chances are
four times that of John.
\begin{enumerate}
	\item Determine
	\begin{enumerate}
		\item P (John promoted)
		\item P (Rita promoted)
		\item P (Aslam promoted)
		\item P (Gurpreet promoted)
	\end{enumerate}
	\item If A = {John promoted or Gurpreet promoted}, find P (A).
\end{enumerate}
\solution
%\input{exemplar/11/16/3/10/main.tex}
\item A card is drawn from a deck of 52 cards. Find the probability of getting a king or a heart or a red card.\\
\solution
%\input{exemplar/11/16/3/15/main.tex}
\item The probability that a student will pass his examination is 0.73, the probability of
the student getting a compartment is 0.13, and the probability that the student will
either pass or get compartment is 0.96. State True or False.\\
\solution
%\input{exemplar/11/16/3/31/main.tex}
\item A card is selected from a pack of 52 cards\\
\begin{enumerate}[label=(\alph*)]
\item How many points are there in the sample space?
\item Calculate the probability that the cards is an ace of spades.
\item Calculate the probability that the card is (i) an ace (ii)black card.\\
\end{enumerate}
%\input{ncert/11/16/3/4_1/Prob_4.tex}
\item In a non-leap year, the probability of having 53 tuesdays or 53 wednesdays is\\
\solution
%\input{exemplar/11/16/3/18/main.tex}
\item There are 1000 sealed envelopes in a box, 10 of them contain a cash prize of
Rs 100 each, 100 of them contain a cash prize of Rs 50 each and 200 of them
contain a cash prize of Rs 10 each and rest do not contain any cash prize. If they
are well shuffled and an envelope is picked up out, what is the probability that it
contains no cash prize?\\
\solution
%\input{exemplar/10/13/3/34/main.tex}
\item 
A die is thrown and a card is selected at random from a deck of 52 playing cards. The probability of getting an even number on the die and a spade card.\\
\solution
%\input{exemplar/12/13/3/78/main.tex}
\item
If 4-digit numbers greater than 5,000 are randomly formed from the digits 0, 1, 3, 5, and 7, what is the probability of forming a number divisible by 5 when:
\begin{enumerate}
    \item The digits are repeated?
    \item The repetition of digits is not allowed?
\end{enumerate}
\solution
%\input{ncert/11/16/4/9/main.tex}
\item Consider the probability space $\brak{\Omega, \mathcal{G}, P}$ where $\Omega = [0,2]$ and $\mathcal{G} = \cbrak{\phi, \Omega, [0,1], (1,2]}$. Let $X$ and $Y$ be two functions on $\Omega$ defined as
\begin{align*}
    X(\omega) = 
    \begin{cases}
        1 & \text{if }\omega \in [0, 1]\\
        2 & \text{if }\omega \in (1, 2]
    \end{cases}
\end{align*}
and
\begin{align*}
    Y(\omega) = 
    \begin{cases}
        2 & \text{if }\omega \in [0, 1.5]\\
        3 & \text{if }\omega \in (1.5, 2].
    \end{cases}
\end{align*}
Then which one of the following statements is true?
\begin{enumerate}
    \item [(A)] $X$ is a random variable with respect to $\mathcal{G}$, but $Y$ is not a random variable with respect to $\mathcal{G}$.
    \item [(B)] $Y$ is a random variable with respect to $\mathcal{G}$, but $X$ is not a random variable with respect to $\mathcal{G}$.
    \item [(C)] Neither $X$ nor $Y$ is a random variable with respect to $\mathcal{G}$.
    \item [(D)] Both $X$ and $Y$ are random variables with respect to $\mathcal{G}$.
\end{enumerate} \hfill (GATE ST 2023)\\
\solution
%\input{gate/ST/2023/14/main.tex}
	\item  A die is loaded in such a way that each odd number is twice as likely to occur as
each even number. Find $P(G)$, where $G$ is the event that a number greater than
3 occurs on a single roll of the die.
\\
\solution
		%\input{exemplar/11/16/3/5/main.tex}
	\item All the jacks, queens and kings are removed from a deck of 52 playing cards. The remaining cards are well shuffled and then one card is drawn at random. Giving ace a value 1 similar value for other cards, find the probability that the card has a value 
		\begin{enumerate}
			\item 7
			\item greater than 7
			\item less than 7
		\end{enumerate}
		%\input{exemplar/10/13/3/30/main.tex}
  \item A Lot consists of 48 mobile phones of which 42 are good, 3 have only minor defects and 3 have major defects.Varnika will buy a phone if it is good but the trader will only buy a mobile if it has no major defects. One phone is selected at random from the lot. What is the probability that it is
\begin{enumerate}
	\item acceptable to Varnika?
            \item acceptable to the trader?
\end{enumerate}
\solution
	%\input{exemplar/10/13/3/40/main.tex}
 \item A student says that if you throw a die, it will show up 1 or not 1. Therefore, the probability of getting 1 and the probability of getting 'not 1' each is equal to $\frac{1}{2}$. Is this correct? Give reasons.\\
 \solution
        %\input{exemplar/10/13/2/9/main.tex}
   \item Four candidates A, B, C, D have ap-
plied for the assignment to coach a school cricket
team. If A is twice as likely to be selected as B, and
B and C are given about the same chance of being
selected, while C is twice as likely to be selected
as D, what are the probabilities that
\begin{enumerate}
\item C will be selected?
\item A will not be selected?
\end{enumerate}
	%\input{exemplar/11/16/3/9/main.tex}
 \item A bag contain 24 balls of which $x$ balls are red, $2x$ are white and $3x$ are blue. A ball is selected at random, What is the probability that it is
\begin{enumerate}[label=\alph*)]
\item not red ?
\item white ?
\end{enumerate}
%\input{exemplar/10/13/3/41/main.tex}
If the letters of the word ASSASSINATION are arranged at random. Find the Probability that
\begin{enumerate}[label=(\alph*)]
\item Four $S's$ come consecutively in the word
\item Two  $I's$ and two $N's$ come together
\item All $A's$ are not coming together
\item No two $A's$ are coming together
\end{enumerate}
%\input{exemplar/11/16/3/14/main.tex}
	\item One urn contains two black balls (labelled B1 and B2) and one white ball. A
	second urn contains one black ball and two white balls (labelled W1 and W2).
	Suppose the following experiment is performed. One of the two urns is chosen
	at random. Next a ball is randomly chosen from the urn. Then a second ball is
	chosen at random from the same urn without replacing the first ball.
	
	\begin{enumerate}
	\item What is the probability that two black balls are chosen?
	
	\item What is the probability that two balls of opposite colour are chosen?
	\end{enumerate}
	\solution
	%\input{exemplar/11/16/3/12/main1.tex}
\end{enumerate}

	\item A bag contains $5$ red balls and some blue balls. If the probability of drawing a blue ball is double that if a red ball, determine the number of blue balls in the bag. 
		\\
\solution
		%\begin{enumerate}[label=\thesection.\arabic*,ref=\thesection.\theenumi]
	\item One card is drawn from a well-shuffled deck of 52 cards. Find the probability of getting
\begin{enumerate}
\item A king of red colour 
\item A face card 
\item A red face card
\item The jack of hearts
\item A spade
\item The queen of diamonds

\end{enumerate}
\solution
		%\input{ncert/10/15/1/14/main.tex}
	\item Five cards—the ten, jack, queen, king and ace of diamonds, are well-shuffled with their face downwards. One card is then picked up at random.
\begin{enumerate}
\item
What is the probability that the card is the queen? 
\item
If the queen is drawn and put aside, what is the probability that the second card picked up is (a) an ace? (b) a queen?\\
\end{enumerate}
\solution
		%\input{ncert/10/15/1/15/defs.tex}
	\item A bag contains $5$ red balls and some blue balls. If the probability of drawing a blue ball is double that if a red ball, determine the number of blue balls in the bag. 
		\\
\solution
		%\input{ncert/10/15/2/3/defs.tex}
	\item A card is selected from a pack of 52 cards.
 \begin{enumerate}[label=(\alph*)] 
                 \item How many points are there in the sample space?
                 \item Calculate the probability that the card is an ace of spades.
                 \item Calculate the probability that the card is (i) an ace and (ii) black card.
 \end{enumerate}
\solution
		%\input{ncert/11/16/3/4/main.tex}
\item Four cards are drawn from a well-shuffled deck of 52 cards. What is the probability of obtaining 3 diamonds and one spade.
\\
\solution
		%\input{ncert/11/16/4/2/defs.tex}
\item In a certain lottery 10,000 tickets are sold and ten equal prizes are awarded. What is the probability of not getting a prize if you buy (a) one ticket (b) two tickets (c) 10 tickets ?	
\\
\solution
		%\input{ncert/11/16/4/4/defs.tex}
		%
\item 
Out of 100 students, two sections of 40 and 60 are formed. If you and your friend are among the 100 students, what is the probability that
\begin{enumerate}
\item you both enter the same section?
\item you both enter the different sections?
\end{enumerate}
\solution
		%\input{ncert/11/16/4/5/defs.tex}
	\item 
The number lock of a suitcase has 4 wheels each labelled with ten digits i.e. from 0 to 9.The lock opens with a sequence of four digits with no repeats.What is the probability of a person getting the right sequence to open the suitcase.
\\
\solution
		%\input{ncert/11/16/4/10/defs.tex}
		%
\item 
Two cards are drawn at random and without replacement from a pack of 52 playing cards. Find the probability that both the cards are black.
\\
\solution
		%\input{ncert/12/13/2/2/defs.tex}
		\item A box of oranges is inspected by examining three randomly selected oranges drawn without replacement. If all the three oranges are good, the box is approved for sale, otherwise, it is rejected. Find the probability that a box containing 15 oranges out of which 12 are good and 3 are bad ones will be approved for sale.
		\label{ncert/12/13/2/3/defs.tex}
		\item Two balls are drawn at random with replacement from a box containing 10 black and 8 red balls. Find the probability that
		\label{ncert/12/13/2/12}
\begin{enumerate}
\item both balls are red.
\item first ball is black and second is red.
\item one of them is black and other is red.
\end{enumerate}

\item In a hostel, 60\% of the students read Hindi newspaper, 40\% read English newspaper and 20\% read both Hindi and English newspapers. A student is selected at random.
		\label{ncert/12/13/2/15}
\begin{enumerate}
\item Find the probability that she reads neither Hindi nor English newspapers.
\item If she reads Hindi newspaper, find the probability that she reads English newspaper.
\item If she reads English newspaper, find the probability that she reads Hindi newspaper.\\
\end{enumerate}
\item The probability of obtaining an even prime number on each die, when a pair of dice is rolled is 
\begin{enumerate}
    \item $0$ 
    
    \item $\frac{1}{3}$ 
    
    \item $\frac{1}{12}$ 
    
    \item $\frac{1}{36}$ 
\end{enumerate}
\solution
		%\input{ncert/12/13/2/17/defs.tex}
	\item A bag contains 4 red and 4 black balls, another bag contains 2 red and 6 black balls. One of the two bags is selected at random and a ball is drawn from the bag which is found to be red. Find the probability that the ball is drawn from the first bag.
\\
\solution
		%\input{ncert/12/13/3/2/main.tex}
  \item
  Cards with numbers 2 to 101 are placed in a box. A card is selected at random.Find the probability that the card has
\begin{enumerate}[label=(\roman*)]
	\item an even number 
	\item a square number
\end{enumerate}
\solution
%\input{exemplar/10/13/3/32/main.tex}
\item
The king, queen and jack of clubs are removed from a deck of 52 playing cards and then well shuffled. Now one card is drawn at random from the remaining cards.  Determine the probability that the card is
\begin{enumerate}[label=(\roman*)]
\item a club
\item 10 of hearts
\end{enumerate}
\solution
%\input{exemplar/10/13/3/29/main.tex}
\item A team of medical students doing their internship have to assist during surgeries
at a city hospital. The probabilities of surgeries rated as very complex, complex,
routine, simple or very simple are respectively, 0.15, 0.20, 0.31, 0.26, .08. Find
the probabilities that a particular surgery will be rated
\begin{enumerate}
	\item complex or very complex;
	\item neither very complex nor very simple;
	\item routine or complex
	\item routine or simple
\end{enumerate}
\solution
%\input{exemplar/11/16/3/8(1)/main.tex}
\item A card is selected from a pack of 52 cards.
\begin{enumerate}[label=(\alph*)]
    \item How many points are there in the sample space?
    \item Calculate the probability that the card is an ace of spades.
    \item Calculate the probability that the card is (i) an ace and (ii) black card.
\end{enumerate}
\solution
%\input{exemplar/11/16/3/4/main2.tex}
\item The probability that a non leap year selected at random will contain 53 sundays.
\\
\solution
%\input{exemplar/10/13/1/19/main.tex}
\item One of the four persons John, Rita, Aslam or Gurpreet will be promoted next
month. Consequently the sample space consists of four elementary outcomes
S = {John promoted, Rita promoted, Aslam promoted, Gurpreet promoted}
You are told that the chances of John’s promotion is same as that of Gurpreet,
Rita’s chances of promotion are twice as likely as Johns. Aslam’s chances are
four times that of John.
\begin{enumerate}
	\item Determine
	\begin{enumerate}
		\item P (John promoted)
		\item P (Rita promoted)
		\item P (Aslam promoted)
		\item P (Gurpreet promoted)
	\end{enumerate}
	\item If A = {John promoted or Gurpreet promoted}, find P (A).
\end{enumerate}
\solution
%\input{exemplar/11/16/3/10/main.tex}
\item A card is drawn from a deck of 52 cards. Find the probability of getting a king or a heart or a red card.\\
\solution
%\input{exemplar/11/16/3/15/main.tex}
\item The probability that a student will pass his examination is 0.73, the probability of
the student getting a compartment is 0.13, and the probability that the student will
either pass or get compartment is 0.96. State True or False.\\
\solution
%\input{exemplar/11/16/3/31/main.tex}
\item A card is selected from a pack of 52 cards\\
\begin{enumerate}[label=(\alph*)]
\item How many points are there in the sample space?
\item Calculate the probability that the cards is an ace of spades.
\item Calculate the probability that the card is (i) an ace (ii)black card.\\
\end{enumerate}
%\input{ncert/11/16/3/4_1/Prob_4.tex}
\item In a non-leap year, the probability of having 53 tuesdays or 53 wednesdays is\\
\solution
%\input{exemplar/11/16/3/18/main.tex}
\item There are 1000 sealed envelopes in a box, 10 of them contain a cash prize of
Rs 100 each, 100 of them contain a cash prize of Rs 50 each and 200 of them
contain a cash prize of Rs 10 each and rest do not contain any cash prize. If they
are well shuffled and an envelope is picked up out, what is the probability that it
contains no cash prize?\\
\solution
%\input{exemplar/10/13/3/34/main.tex}
\item 
A die is thrown and a card is selected at random from a deck of 52 playing cards. The probability of getting an even number on the die and a spade card.\\
\solution
%\input{exemplar/12/13/3/78/main.tex}
\item
If 4-digit numbers greater than 5,000 are randomly formed from the digits 0, 1, 3, 5, and 7, what is the probability of forming a number divisible by 5 when:
\begin{enumerate}
    \item The digits are repeated?
    \item The repetition of digits is not allowed?
\end{enumerate}
\solution
%\input{ncert/11/16/4/9/main.tex}
\item Consider the probability space $\brak{\Omega, \mathcal{G}, P}$ where $\Omega = [0,2]$ and $\mathcal{G} = \cbrak{\phi, \Omega, [0,1], (1,2]}$. Let $X$ and $Y$ be two functions on $\Omega$ defined as
\begin{align*}
    X(\omega) = 
    \begin{cases}
        1 & \text{if }\omega \in [0, 1]\\
        2 & \text{if }\omega \in (1, 2]
    \end{cases}
\end{align*}
and
\begin{align*}
    Y(\omega) = 
    \begin{cases}
        2 & \text{if }\omega \in [0, 1.5]\\
        3 & \text{if }\omega \in (1.5, 2].
    \end{cases}
\end{align*}
Then which one of the following statements is true?
\begin{enumerate}
    \item [(A)] $X$ is a random variable with respect to $\mathcal{G}$, but $Y$ is not a random variable with respect to $\mathcal{G}$.
    \item [(B)] $Y$ is a random variable with respect to $\mathcal{G}$, but $X$ is not a random variable with respect to $\mathcal{G}$.
    \item [(C)] Neither $X$ nor $Y$ is a random variable with respect to $\mathcal{G}$.
    \item [(D)] Both $X$ and $Y$ are random variables with respect to $\mathcal{G}$.
\end{enumerate} \hfill (GATE ST 2023)\\
\solution
%\input{gate/ST/2023/14/main.tex}
	\item  A die is loaded in such a way that each odd number is twice as likely to occur as
each even number. Find $P(G)$, where $G$ is the event that a number greater than
3 occurs on a single roll of the die.
\\
\solution
		%\input{exemplar/11/16/3/5/main.tex}
	\item All the jacks, queens and kings are removed from a deck of 52 playing cards. The remaining cards are well shuffled and then one card is drawn at random. Giving ace a value 1 similar value for other cards, find the probability that the card has a value 
		\begin{enumerate}
			\item 7
			\item greater than 7
			\item less than 7
		\end{enumerate}
		%\input{exemplar/10/13/3/30/main.tex}
  \item A Lot consists of 48 mobile phones of which 42 are good, 3 have only minor defects and 3 have major defects.Varnika will buy a phone if it is good but the trader will only buy a mobile if it has no major defects. One phone is selected at random from the lot. What is the probability that it is
\begin{enumerate}
	\item acceptable to Varnika?
            \item acceptable to the trader?
\end{enumerate}
\solution
	%\input{exemplar/10/13/3/40/main.tex}
 \item A student says that if you throw a die, it will show up 1 or not 1. Therefore, the probability of getting 1 and the probability of getting 'not 1' each is equal to $\frac{1}{2}$. Is this correct? Give reasons.\\
 \solution
        %\input{exemplar/10/13/2/9/main.tex}
   \item Four candidates A, B, C, D have ap-
plied for the assignment to coach a school cricket
team. If A is twice as likely to be selected as B, and
B and C are given about the same chance of being
selected, while C is twice as likely to be selected
as D, what are the probabilities that
\begin{enumerate}
\item C will be selected?
\item A will not be selected?
\end{enumerate}
	%\input{exemplar/11/16/3/9/main.tex}
 \item A bag contain 24 balls of which $x$ balls are red, $2x$ are white and $3x$ are blue. A ball is selected at random, What is the probability that it is
\begin{enumerate}[label=\alph*)]
\item not red ?
\item white ?
\end{enumerate}
%\input{exemplar/10/13/3/41/main.tex}
If the letters of the word ASSASSINATION are arranged at random. Find the Probability that
\begin{enumerate}[label=(\alph*)]
\item Four $S's$ come consecutively in the word
\item Two  $I's$ and two $N's$ come together
\item All $A's$ are not coming together
\item No two $A's$ are coming together
\end{enumerate}
%\input{exemplar/11/16/3/14/main.tex}
	\item One urn contains two black balls (labelled B1 and B2) and one white ball. A
	second urn contains one black ball and two white balls (labelled W1 and W2).
	Suppose the following experiment is performed. One of the two urns is chosen
	at random. Next a ball is randomly chosen from the urn. Then a second ball is
	chosen at random from the same urn without replacing the first ball.
	
	\begin{enumerate}
	\item What is the probability that two black balls are chosen?
	
	\item What is the probability that two balls of opposite colour are chosen?
	\end{enumerate}
	\solution
	%\input{exemplar/11/16/3/12/main1.tex}
\end{enumerate}

	\item A card is selected from a pack of 52 cards.
 \begin{enumerate}[label=(\alph*)] 
                 \item How many points are there in the sample space?
                 \item Calculate the probability that the card is an ace of spades.
                 \item Calculate the probability that the card is (i) an ace and (ii) black card.
 \end{enumerate}
\solution
		%\begin{table}[H]
	\centering
\begin{tabular}{|c|c|c|}
\hline
Random variable &Value &Definition\\ \hline
\multirow{3}{*}{X} &0 &Slips of Rs 1\\
&1 &Slips of Rs 5\\
&2 &Slips of Rs 13\\ \hline
\multirow{2}{*}{Y} &0 &Box A\\
&1 &Box B\\\hline
\end{tabular}
\caption{}
\label{tab:Distribution}
\end{table}
See \tabref{tab:Distribution}.
\begin{align}
p_{Y}\brak{k}= \begin{cases} 
      \frac{1}{3} & {k=0} \\
      \frac{2}{3 }& {k=1} 
   \end{cases}
   \\
p_{Y|X}\brak{0|0} = \frac{19}{25}\, 
p_{Y|X}\brak{0|1} = \frac{6}{25}\,
p_{Y|X}\brak{1|0} = \frac{45}{50}\,
p_{Y|X}\brak{1|2} = \frac{5}{50}
\end{align}
The desired probability is the probability that a slip drawn at random is marked other than Rs 1,
\begin{align}
&=1-p_X\brak{0}\\
&= p_X(1) + p_X(2)
\end{align}
Using Bayes theorem,
\begin{align}
&= p_Y\brak{0} \times \pr{Y=0 | X=1} + p_Y\brak{1} \times \pr{Y=1|X=2}\\
&=\frac{1}{3} \times \frac{6}{25} + \frac{2}{3} \times \frac{5}{50}\\
&=\frac{11}{75}
\end{align}

\newpage

%\tableofcontents

\bigskip

\renewcommand{\thefigure}{\theenumi}
\renewcommand{\thetable}{\theenumi}
%\renewcommand{\theequation}{\theenumi}

%\begin{abstract}
%%\boldmath
%In this letter, an algorithm for evaluating the exact analytical bit error rate  (BER)  for the piecewise linear (PL) combiner for  multiple relays is presented. Previous results were available only for upto three relays. The algorithm is unique in the sense that  the actual mathematical expressions, that are prohibitively large, need not be explicitly obtained. The diversity gain due to multiple relays is shown through plots of the analytical BER, well supported by simulations. 
%
%\end{abstract}
% IEEEtran.cls defaults to using nonbold math in the Abstract.
% This preserves the distinction between vectors and scalars. However,
% if the journal you are submitting to favors bold math in the abstract,
% then you can use LaTeX's standard command \boldmath at the very start
% of the abstract to achieve this. Many IEEE journals frown on math
% in the abstract anyway.

% Note that keywords are not normally used for peerreview papers.
%\begin{IEEEkeywords}
%Cooperative diversity, decode and forward, piecewise linear
%\end{IEEEkeywords}



% For peer review papers, you can put extra information on the cover
% page as needed:
% \ifCLASSOPTIONpeerreview
% \begin{center} \bfseries EDICS Category: 3-BBND \end{center}
% \fi
%
% For peerreview papers, this IEEEtran command inserts a page break and
% creates the second title. It will be ignored for other modes.
%\IEEEpeerreviewmaketitle




\item Four cards are drawn from a well-shuffled deck of 52 cards. What is the probability of obtaining 3 diamonds and one spade.
\\
\solution
		%\begin{enumerate}[label=\thesection.\arabic*,ref=\thesection.\theenumi]
	\item One card is drawn from a well-shuffled deck of 52 cards. Find the probability of getting
\begin{enumerate}
\item A king of red colour 
\item A face card 
\item A red face card
\item The jack of hearts
\item A spade
\item The queen of diamonds

\end{enumerate}
\solution
		%\input{ncert/10/15/1/14/main.tex}
	\item Five cards—the ten, jack, queen, king and ace of diamonds, are well-shuffled with their face downwards. One card is then picked up at random.
\begin{enumerate}
\item
What is the probability that the card is the queen? 
\item
If the queen is drawn and put aside, what is the probability that the second card picked up is (a) an ace? (b) a queen?\\
\end{enumerate}
\solution
		%\input{ncert/10/15/1/15/defs.tex}
	\item A bag contains $5$ red balls and some blue balls. If the probability of drawing a blue ball is double that if a red ball, determine the number of blue balls in the bag. 
		\\
\solution
		%\input{ncert/10/15/2/3/defs.tex}
	\item A card is selected from a pack of 52 cards.
 \begin{enumerate}[label=(\alph*)] 
                 \item How many points are there in the sample space?
                 \item Calculate the probability that the card is an ace of spades.
                 \item Calculate the probability that the card is (i) an ace and (ii) black card.
 \end{enumerate}
\solution
		%\input{ncert/11/16/3/4/main.tex}
\item Four cards are drawn from a well-shuffled deck of 52 cards. What is the probability of obtaining 3 diamonds and one spade.
\\
\solution
		%\input{ncert/11/16/4/2/defs.tex}
\item In a certain lottery 10,000 tickets are sold and ten equal prizes are awarded. What is the probability of not getting a prize if you buy (a) one ticket (b) two tickets (c) 10 tickets ?	
\\
\solution
		%\input{ncert/11/16/4/4/defs.tex}
		%
\item 
Out of 100 students, two sections of 40 and 60 are formed. If you and your friend are among the 100 students, what is the probability that
\begin{enumerate}
\item you both enter the same section?
\item you both enter the different sections?
\end{enumerate}
\solution
		%\input{ncert/11/16/4/5/defs.tex}
	\item 
The number lock of a suitcase has 4 wheels each labelled with ten digits i.e. from 0 to 9.The lock opens with a sequence of four digits with no repeats.What is the probability of a person getting the right sequence to open the suitcase.
\\
\solution
		%\input{ncert/11/16/4/10/defs.tex}
		%
\item 
Two cards are drawn at random and without replacement from a pack of 52 playing cards. Find the probability that both the cards are black.
\\
\solution
		%\input{ncert/12/13/2/2/defs.tex}
		\item A box of oranges is inspected by examining three randomly selected oranges drawn without replacement. If all the three oranges are good, the box is approved for sale, otherwise, it is rejected. Find the probability that a box containing 15 oranges out of which 12 are good and 3 are bad ones will be approved for sale.
		\label{ncert/12/13/2/3/defs.tex}
		\item Two balls are drawn at random with replacement from a box containing 10 black and 8 red balls. Find the probability that
		\label{ncert/12/13/2/12}
\begin{enumerate}
\item both balls are red.
\item first ball is black and second is red.
\item one of them is black and other is red.
\end{enumerate}

\item In a hostel, 60\% of the students read Hindi newspaper, 40\% read English newspaper and 20\% read both Hindi and English newspapers. A student is selected at random.
		\label{ncert/12/13/2/15}
\begin{enumerate}
\item Find the probability that she reads neither Hindi nor English newspapers.
\item If she reads Hindi newspaper, find the probability that she reads English newspaper.
\item If she reads English newspaper, find the probability that she reads Hindi newspaper.\\
\end{enumerate}
\item The probability of obtaining an even prime number on each die, when a pair of dice is rolled is 
\begin{enumerate}
    \item $0$ 
    
    \item $\frac{1}{3}$ 
    
    \item $\frac{1}{12}$ 
    
    \item $\frac{1}{36}$ 
\end{enumerate}
\solution
		%\input{ncert/12/13/2/17/defs.tex}
	\item A bag contains 4 red and 4 black balls, another bag contains 2 red and 6 black balls. One of the two bags is selected at random and a ball is drawn from the bag which is found to be red. Find the probability that the ball is drawn from the first bag.
\\
\solution
		%\input{ncert/12/13/3/2/main.tex}
  \item
  Cards with numbers 2 to 101 are placed in a box. A card is selected at random.Find the probability that the card has
\begin{enumerate}[label=(\roman*)]
	\item an even number 
	\item a square number
\end{enumerate}
\solution
%\input{exemplar/10/13/3/32/main.tex}
\item
The king, queen and jack of clubs are removed from a deck of 52 playing cards and then well shuffled. Now one card is drawn at random from the remaining cards.  Determine the probability that the card is
\begin{enumerate}[label=(\roman*)]
\item a club
\item 10 of hearts
\end{enumerate}
\solution
%\input{exemplar/10/13/3/29/main.tex}
\item A team of medical students doing their internship have to assist during surgeries
at a city hospital. The probabilities of surgeries rated as very complex, complex,
routine, simple or very simple are respectively, 0.15, 0.20, 0.31, 0.26, .08. Find
the probabilities that a particular surgery will be rated
\begin{enumerate}
	\item complex or very complex;
	\item neither very complex nor very simple;
	\item routine or complex
	\item routine or simple
\end{enumerate}
\solution
%\input{exemplar/11/16/3/8(1)/main.tex}
\item A card is selected from a pack of 52 cards.
\begin{enumerate}[label=(\alph*)]
    \item How many points are there in the sample space?
    \item Calculate the probability that the card is an ace of spades.
    \item Calculate the probability that the card is (i) an ace and (ii) black card.
\end{enumerate}
\solution
%\input{exemplar/11/16/3/4/main2.tex}
\item The probability that a non leap year selected at random will contain 53 sundays.
\\
\solution
%\input{exemplar/10/13/1/19/main.tex}
\item One of the four persons John, Rita, Aslam or Gurpreet will be promoted next
month. Consequently the sample space consists of four elementary outcomes
S = {John promoted, Rita promoted, Aslam promoted, Gurpreet promoted}
You are told that the chances of John’s promotion is same as that of Gurpreet,
Rita’s chances of promotion are twice as likely as Johns. Aslam’s chances are
four times that of John.
\begin{enumerate}
	\item Determine
	\begin{enumerate}
		\item P (John promoted)
		\item P (Rita promoted)
		\item P (Aslam promoted)
		\item P (Gurpreet promoted)
	\end{enumerate}
	\item If A = {John promoted or Gurpreet promoted}, find P (A).
\end{enumerate}
\solution
%\input{exemplar/11/16/3/10/main.tex}
\item A card is drawn from a deck of 52 cards. Find the probability of getting a king or a heart or a red card.\\
\solution
%\input{exemplar/11/16/3/15/main.tex}
\item The probability that a student will pass his examination is 0.73, the probability of
the student getting a compartment is 0.13, and the probability that the student will
either pass or get compartment is 0.96. State True or False.\\
\solution
%\input{exemplar/11/16/3/31/main.tex}
\item A card is selected from a pack of 52 cards\\
\begin{enumerate}[label=(\alph*)]
\item How many points are there in the sample space?
\item Calculate the probability that the cards is an ace of spades.
\item Calculate the probability that the card is (i) an ace (ii)black card.\\
\end{enumerate}
%\input{ncert/11/16/3/4_1/Prob_4.tex}
\item In a non-leap year, the probability of having 53 tuesdays or 53 wednesdays is\\
\solution
%\input{exemplar/11/16/3/18/main.tex}
\item There are 1000 sealed envelopes in a box, 10 of them contain a cash prize of
Rs 100 each, 100 of them contain a cash prize of Rs 50 each and 200 of them
contain a cash prize of Rs 10 each and rest do not contain any cash prize. If they
are well shuffled and an envelope is picked up out, what is the probability that it
contains no cash prize?\\
\solution
%\input{exemplar/10/13/3/34/main.tex}
\item 
A die is thrown and a card is selected at random from a deck of 52 playing cards. The probability of getting an even number on the die and a spade card.\\
\solution
%\input{exemplar/12/13/3/78/main.tex}
\item
If 4-digit numbers greater than 5,000 are randomly formed from the digits 0, 1, 3, 5, and 7, what is the probability of forming a number divisible by 5 when:
\begin{enumerate}
    \item The digits are repeated?
    \item The repetition of digits is not allowed?
\end{enumerate}
\solution
%\input{ncert/11/16/4/9/main.tex}
\item Consider the probability space $\brak{\Omega, \mathcal{G}, P}$ where $\Omega = [0,2]$ and $\mathcal{G} = \cbrak{\phi, \Omega, [0,1], (1,2]}$. Let $X$ and $Y$ be two functions on $\Omega$ defined as
\begin{align*}
    X(\omega) = 
    \begin{cases}
        1 & \text{if }\omega \in [0, 1]\\
        2 & \text{if }\omega \in (1, 2]
    \end{cases}
\end{align*}
and
\begin{align*}
    Y(\omega) = 
    \begin{cases}
        2 & \text{if }\omega \in [0, 1.5]\\
        3 & \text{if }\omega \in (1.5, 2].
    \end{cases}
\end{align*}
Then which one of the following statements is true?
\begin{enumerate}
    \item [(A)] $X$ is a random variable with respect to $\mathcal{G}$, but $Y$ is not a random variable with respect to $\mathcal{G}$.
    \item [(B)] $Y$ is a random variable with respect to $\mathcal{G}$, but $X$ is not a random variable with respect to $\mathcal{G}$.
    \item [(C)] Neither $X$ nor $Y$ is a random variable with respect to $\mathcal{G}$.
    \item [(D)] Both $X$ and $Y$ are random variables with respect to $\mathcal{G}$.
\end{enumerate} \hfill (GATE ST 2023)\\
\solution
%\input{gate/ST/2023/14/main.tex}
	\item  A die is loaded in such a way that each odd number is twice as likely to occur as
each even number. Find $P(G)$, where $G$ is the event that a number greater than
3 occurs on a single roll of the die.
\\
\solution
		%\input{exemplar/11/16/3/5/main.tex}
	\item All the jacks, queens and kings are removed from a deck of 52 playing cards. The remaining cards are well shuffled and then one card is drawn at random. Giving ace a value 1 similar value for other cards, find the probability that the card has a value 
		\begin{enumerate}
			\item 7
			\item greater than 7
			\item less than 7
		\end{enumerate}
		%\input{exemplar/10/13/3/30/main.tex}
  \item A Lot consists of 48 mobile phones of which 42 are good, 3 have only minor defects and 3 have major defects.Varnika will buy a phone if it is good but the trader will only buy a mobile if it has no major defects. One phone is selected at random from the lot. What is the probability that it is
\begin{enumerate}
	\item acceptable to Varnika?
            \item acceptable to the trader?
\end{enumerate}
\solution
	%\input{exemplar/10/13/3/40/main.tex}
 \item A student says that if you throw a die, it will show up 1 or not 1. Therefore, the probability of getting 1 and the probability of getting 'not 1' each is equal to $\frac{1}{2}$. Is this correct? Give reasons.\\
 \solution
        %\input{exemplar/10/13/2/9/main.tex}
   \item Four candidates A, B, C, D have ap-
plied for the assignment to coach a school cricket
team. If A is twice as likely to be selected as B, and
B and C are given about the same chance of being
selected, while C is twice as likely to be selected
as D, what are the probabilities that
\begin{enumerate}
\item C will be selected?
\item A will not be selected?
\end{enumerate}
	%\input{exemplar/11/16/3/9/main.tex}
 \item A bag contain 24 balls of which $x$ balls are red, $2x$ are white and $3x$ are blue. A ball is selected at random, What is the probability that it is
\begin{enumerate}[label=\alph*)]
\item not red ?
\item white ?
\end{enumerate}
%\input{exemplar/10/13/3/41/main.tex}
If the letters of the word ASSASSINATION are arranged at random. Find the Probability that
\begin{enumerate}[label=(\alph*)]
\item Four $S's$ come consecutively in the word
\item Two  $I's$ and two $N's$ come together
\item All $A's$ are not coming together
\item No two $A's$ are coming together
\end{enumerate}
%\input{exemplar/11/16/3/14/main.tex}
	\item One urn contains two black balls (labelled B1 and B2) and one white ball. A
	second urn contains one black ball and two white balls (labelled W1 and W2).
	Suppose the following experiment is performed. One of the two urns is chosen
	at random. Next a ball is randomly chosen from the urn. Then a second ball is
	chosen at random from the same urn without replacing the first ball.
	
	\begin{enumerate}
	\item What is the probability that two black balls are chosen?
	
	\item What is the probability that two balls of opposite colour are chosen?
	\end{enumerate}
	\solution
	%\input{exemplar/11/16/3/12/main1.tex}
\end{enumerate}

\item In a certain lottery 10,000 tickets are sold and ten equal prizes are awarded. What is the probability of not getting a prize if you buy (a) one ticket (b) two tickets (c) 10 tickets ?	
\\
\solution
		%\begin{enumerate}[label=\thesection.\arabic*,ref=\thesection.\theenumi]
	\item One card is drawn from a well-shuffled deck of 52 cards. Find the probability of getting
\begin{enumerate}
\item A king of red colour 
\item A face card 
\item A red face card
\item The jack of hearts
\item A spade
\item The queen of diamonds

\end{enumerate}
\solution
		%\input{ncert/10/15/1/14/main.tex}
	\item Five cards—the ten, jack, queen, king and ace of diamonds, are well-shuffled with their face downwards. One card is then picked up at random.
\begin{enumerate}
\item
What is the probability that the card is the queen? 
\item
If the queen is drawn and put aside, what is the probability that the second card picked up is (a) an ace? (b) a queen?\\
\end{enumerate}
\solution
		%\input{ncert/10/15/1/15/defs.tex}
	\item A bag contains $5$ red balls and some blue balls. If the probability of drawing a blue ball is double that if a red ball, determine the number of blue balls in the bag. 
		\\
\solution
		%\input{ncert/10/15/2/3/defs.tex}
	\item A card is selected from a pack of 52 cards.
 \begin{enumerate}[label=(\alph*)] 
                 \item How many points are there in the sample space?
                 \item Calculate the probability that the card is an ace of spades.
                 \item Calculate the probability that the card is (i) an ace and (ii) black card.
 \end{enumerate}
\solution
		%\input{ncert/11/16/3/4/main.tex}
\item Four cards are drawn from a well-shuffled deck of 52 cards. What is the probability of obtaining 3 diamonds and one spade.
\\
\solution
		%\input{ncert/11/16/4/2/defs.tex}
\item In a certain lottery 10,000 tickets are sold and ten equal prizes are awarded. What is the probability of not getting a prize if you buy (a) one ticket (b) two tickets (c) 10 tickets ?	
\\
\solution
		%\input{ncert/11/16/4/4/defs.tex}
		%
\item 
Out of 100 students, two sections of 40 and 60 are formed. If you and your friend are among the 100 students, what is the probability that
\begin{enumerate}
\item you both enter the same section?
\item you both enter the different sections?
\end{enumerate}
\solution
		%\input{ncert/11/16/4/5/defs.tex}
	\item 
The number lock of a suitcase has 4 wheels each labelled with ten digits i.e. from 0 to 9.The lock opens with a sequence of four digits with no repeats.What is the probability of a person getting the right sequence to open the suitcase.
\\
\solution
		%\input{ncert/11/16/4/10/defs.tex}
		%
\item 
Two cards are drawn at random and without replacement from a pack of 52 playing cards. Find the probability that both the cards are black.
\\
\solution
		%\input{ncert/12/13/2/2/defs.tex}
		\item A box of oranges is inspected by examining three randomly selected oranges drawn without replacement. If all the three oranges are good, the box is approved for sale, otherwise, it is rejected. Find the probability that a box containing 15 oranges out of which 12 are good and 3 are bad ones will be approved for sale.
		\label{ncert/12/13/2/3/defs.tex}
		\item Two balls are drawn at random with replacement from a box containing 10 black and 8 red balls. Find the probability that
		\label{ncert/12/13/2/12}
\begin{enumerate}
\item both balls are red.
\item first ball is black and second is red.
\item one of them is black and other is red.
\end{enumerate}

\item In a hostel, 60\% of the students read Hindi newspaper, 40\% read English newspaper and 20\% read both Hindi and English newspapers. A student is selected at random.
		\label{ncert/12/13/2/15}
\begin{enumerate}
\item Find the probability that she reads neither Hindi nor English newspapers.
\item If she reads Hindi newspaper, find the probability that she reads English newspaper.
\item If she reads English newspaper, find the probability that she reads Hindi newspaper.\\
\end{enumerate}
\item The probability of obtaining an even prime number on each die, when a pair of dice is rolled is 
\begin{enumerate}
    \item $0$ 
    
    \item $\frac{1}{3}$ 
    
    \item $\frac{1}{12}$ 
    
    \item $\frac{1}{36}$ 
\end{enumerate}
\solution
		%\input{ncert/12/13/2/17/defs.tex}
	\item A bag contains 4 red and 4 black balls, another bag contains 2 red and 6 black balls. One of the two bags is selected at random and a ball is drawn from the bag which is found to be red. Find the probability that the ball is drawn from the first bag.
\\
\solution
		%\input{ncert/12/13/3/2/main.tex}
  \item
  Cards with numbers 2 to 101 are placed in a box. A card is selected at random.Find the probability that the card has
\begin{enumerate}[label=(\roman*)]
	\item an even number 
	\item a square number
\end{enumerate}
\solution
%\input{exemplar/10/13/3/32/main.tex}
\item
The king, queen and jack of clubs are removed from a deck of 52 playing cards and then well shuffled. Now one card is drawn at random from the remaining cards.  Determine the probability that the card is
\begin{enumerate}[label=(\roman*)]
\item a club
\item 10 of hearts
\end{enumerate}
\solution
%\input{exemplar/10/13/3/29/main.tex}
\item A team of medical students doing their internship have to assist during surgeries
at a city hospital. The probabilities of surgeries rated as very complex, complex,
routine, simple or very simple are respectively, 0.15, 0.20, 0.31, 0.26, .08. Find
the probabilities that a particular surgery will be rated
\begin{enumerate}
	\item complex or very complex;
	\item neither very complex nor very simple;
	\item routine or complex
	\item routine or simple
\end{enumerate}
\solution
%\input{exemplar/11/16/3/8(1)/main.tex}
\item A card is selected from a pack of 52 cards.
\begin{enumerate}[label=(\alph*)]
    \item How many points are there in the sample space?
    \item Calculate the probability that the card is an ace of spades.
    \item Calculate the probability that the card is (i) an ace and (ii) black card.
\end{enumerate}
\solution
%\input{exemplar/11/16/3/4/main2.tex}
\item The probability that a non leap year selected at random will contain 53 sundays.
\\
\solution
%\input{exemplar/10/13/1/19/main.tex}
\item One of the four persons John, Rita, Aslam or Gurpreet will be promoted next
month. Consequently the sample space consists of four elementary outcomes
S = {John promoted, Rita promoted, Aslam promoted, Gurpreet promoted}
You are told that the chances of John’s promotion is same as that of Gurpreet,
Rita’s chances of promotion are twice as likely as Johns. Aslam’s chances are
four times that of John.
\begin{enumerate}
	\item Determine
	\begin{enumerate}
		\item P (John promoted)
		\item P (Rita promoted)
		\item P (Aslam promoted)
		\item P (Gurpreet promoted)
	\end{enumerate}
	\item If A = {John promoted or Gurpreet promoted}, find P (A).
\end{enumerate}
\solution
%\input{exemplar/11/16/3/10/main.tex}
\item A card is drawn from a deck of 52 cards. Find the probability of getting a king or a heart or a red card.\\
\solution
%\input{exemplar/11/16/3/15/main.tex}
\item The probability that a student will pass his examination is 0.73, the probability of
the student getting a compartment is 0.13, and the probability that the student will
either pass or get compartment is 0.96. State True or False.\\
\solution
%\input{exemplar/11/16/3/31/main.tex}
\item A card is selected from a pack of 52 cards\\
\begin{enumerate}[label=(\alph*)]
\item How many points are there in the sample space?
\item Calculate the probability that the cards is an ace of spades.
\item Calculate the probability that the card is (i) an ace (ii)black card.\\
\end{enumerate}
%\input{ncert/11/16/3/4_1/Prob_4.tex}
\item In a non-leap year, the probability of having 53 tuesdays or 53 wednesdays is\\
\solution
%\input{exemplar/11/16/3/18/main.tex}
\item There are 1000 sealed envelopes in a box, 10 of them contain a cash prize of
Rs 100 each, 100 of them contain a cash prize of Rs 50 each and 200 of them
contain a cash prize of Rs 10 each and rest do not contain any cash prize. If they
are well shuffled and an envelope is picked up out, what is the probability that it
contains no cash prize?\\
\solution
%\input{exemplar/10/13/3/34/main.tex}
\item 
A die is thrown and a card is selected at random from a deck of 52 playing cards. The probability of getting an even number on the die and a spade card.\\
\solution
%\input{exemplar/12/13/3/78/main.tex}
\item
If 4-digit numbers greater than 5,000 are randomly formed from the digits 0, 1, 3, 5, and 7, what is the probability of forming a number divisible by 5 when:
\begin{enumerate}
    \item The digits are repeated?
    \item The repetition of digits is not allowed?
\end{enumerate}
\solution
%\input{ncert/11/16/4/9/main.tex}
\item Consider the probability space $\brak{\Omega, \mathcal{G}, P}$ where $\Omega = [0,2]$ and $\mathcal{G} = \cbrak{\phi, \Omega, [0,1], (1,2]}$. Let $X$ and $Y$ be two functions on $\Omega$ defined as
\begin{align*}
    X(\omega) = 
    \begin{cases}
        1 & \text{if }\omega \in [0, 1]\\
        2 & \text{if }\omega \in (1, 2]
    \end{cases}
\end{align*}
and
\begin{align*}
    Y(\omega) = 
    \begin{cases}
        2 & \text{if }\omega \in [0, 1.5]\\
        3 & \text{if }\omega \in (1.5, 2].
    \end{cases}
\end{align*}
Then which one of the following statements is true?
\begin{enumerate}
    \item [(A)] $X$ is a random variable with respect to $\mathcal{G}$, but $Y$ is not a random variable with respect to $\mathcal{G}$.
    \item [(B)] $Y$ is a random variable with respect to $\mathcal{G}$, but $X$ is not a random variable with respect to $\mathcal{G}$.
    \item [(C)] Neither $X$ nor $Y$ is a random variable with respect to $\mathcal{G}$.
    \item [(D)] Both $X$ and $Y$ are random variables with respect to $\mathcal{G}$.
\end{enumerate} \hfill (GATE ST 2023)\\
\solution
%\input{gate/ST/2023/14/main.tex}
	\item  A die is loaded in such a way that each odd number is twice as likely to occur as
each even number. Find $P(G)$, where $G$ is the event that a number greater than
3 occurs on a single roll of the die.
\\
\solution
		%\input{exemplar/11/16/3/5/main.tex}
	\item All the jacks, queens and kings are removed from a deck of 52 playing cards. The remaining cards are well shuffled and then one card is drawn at random. Giving ace a value 1 similar value for other cards, find the probability that the card has a value 
		\begin{enumerate}
			\item 7
			\item greater than 7
			\item less than 7
		\end{enumerate}
		%\input{exemplar/10/13/3/30/main.tex}
  \item A Lot consists of 48 mobile phones of which 42 are good, 3 have only minor defects and 3 have major defects.Varnika will buy a phone if it is good but the trader will only buy a mobile if it has no major defects. One phone is selected at random from the lot. What is the probability that it is
\begin{enumerate}
	\item acceptable to Varnika?
            \item acceptable to the trader?
\end{enumerate}
\solution
	%\input{exemplar/10/13/3/40/main.tex}
 \item A student says that if you throw a die, it will show up 1 or not 1. Therefore, the probability of getting 1 and the probability of getting 'not 1' each is equal to $\frac{1}{2}$. Is this correct? Give reasons.\\
 \solution
        %\input{exemplar/10/13/2/9/main.tex}
   \item Four candidates A, B, C, D have ap-
plied for the assignment to coach a school cricket
team. If A is twice as likely to be selected as B, and
B and C are given about the same chance of being
selected, while C is twice as likely to be selected
as D, what are the probabilities that
\begin{enumerate}
\item C will be selected?
\item A will not be selected?
\end{enumerate}
	%\input{exemplar/11/16/3/9/main.tex}
 \item A bag contain 24 balls of which $x$ balls are red, $2x$ are white and $3x$ are blue. A ball is selected at random, What is the probability that it is
\begin{enumerate}[label=\alph*)]
\item not red ?
\item white ?
\end{enumerate}
%\input{exemplar/10/13/3/41/main.tex}
If the letters of the word ASSASSINATION are arranged at random. Find the Probability that
\begin{enumerate}[label=(\alph*)]
\item Four $S's$ come consecutively in the word
\item Two  $I's$ and two $N's$ come together
\item All $A's$ are not coming together
\item No two $A's$ are coming together
\end{enumerate}
%\input{exemplar/11/16/3/14/main.tex}
	\item One urn contains two black balls (labelled B1 and B2) and one white ball. A
	second urn contains one black ball and two white balls (labelled W1 and W2).
	Suppose the following experiment is performed. One of the two urns is chosen
	at random. Next a ball is randomly chosen from the urn. Then a second ball is
	chosen at random from the same urn without replacing the first ball.
	
	\begin{enumerate}
	\item What is the probability that two black balls are chosen?
	
	\item What is the probability that two balls of opposite colour are chosen?
	\end{enumerate}
	\solution
	%\input{exemplar/11/16/3/12/main1.tex}
\end{enumerate}

		%
\item 
Out of 100 students, two sections of 40 and 60 are formed. If you and your friend are among the 100 students, what is the probability that
\begin{enumerate}
\item you both enter the same section?
\item you both enter the different sections?
\end{enumerate}
\solution
		%\begin{enumerate}[label=\thesection.\arabic*,ref=\thesection.\theenumi]
	\item One card is drawn from a well-shuffled deck of 52 cards. Find the probability of getting
\begin{enumerate}
\item A king of red colour 
\item A face card 
\item A red face card
\item The jack of hearts
\item A spade
\item The queen of diamonds

\end{enumerate}
\solution
		%\input{ncert/10/15/1/14/main.tex}
	\item Five cards—the ten, jack, queen, king and ace of diamonds, are well-shuffled with their face downwards. One card is then picked up at random.
\begin{enumerate}
\item
What is the probability that the card is the queen? 
\item
If the queen is drawn and put aside, what is the probability that the second card picked up is (a) an ace? (b) a queen?\\
\end{enumerate}
\solution
		%\input{ncert/10/15/1/15/defs.tex}
	\item A bag contains $5$ red balls and some blue balls. If the probability of drawing a blue ball is double that if a red ball, determine the number of blue balls in the bag. 
		\\
\solution
		%\input{ncert/10/15/2/3/defs.tex}
	\item A card is selected from a pack of 52 cards.
 \begin{enumerate}[label=(\alph*)] 
                 \item How many points are there in the sample space?
                 \item Calculate the probability that the card is an ace of spades.
                 \item Calculate the probability that the card is (i) an ace and (ii) black card.
 \end{enumerate}
\solution
		%\input{ncert/11/16/3/4/main.tex}
\item Four cards are drawn from a well-shuffled deck of 52 cards. What is the probability of obtaining 3 diamonds and one spade.
\\
\solution
		%\input{ncert/11/16/4/2/defs.tex}
\item In a certain lottery 10,000 tickets are sold and ten equal prizes are awarded. What is the probability of not getting a prize if you buy (a) one ticket (b) two tickets (c) 10 tickets ?	
\\
\solution
		%\input{ncert/11/16/4/4/defs.tex}
		%
\item 
Out of 100 students, two sections of 40 and 60 are formed. If you and your friend are among the 100 students, what is the probability that
\begin{enumerate}
\item you both enter the same section?
\item you both enter the different sections?
\end{enumerate}
\solution
		%\input{ncert/11/16/4/5/defs.tex}
	\item 
The number lock of a suitcase has 4 wheels each labelled with ten digits i.e. from 0 to 9.The lock opens with a sequence of four digits with no repeats.What is the probability of a person getting the right sequence to open the suitcase.
\\
\solution
		%\input{ncert/11/16/4/10/defs.tex}
		%
\item 
Two cards are drawn at random and without replacement from a pack of 52 playing cards. Find the probability that both the cards are black.
\\
\solution
		%\input{ncert/12/13/2/2/defs.tex}
		\item A box of oranges is inspected by examining three randomly selected oranges drawn without replacement. If all the three oranges are good, the box is approved for sale, otherwise, it is rejected. Find the probability that a box containing 15 oranges out of which 12 are good and 3 are bad ones will be approved for sale.
		\label{ncert/12/13/2/3/defs.tex}
		\item Two balls are drawn at random with replacement from a box containing 10 black and 8 red balls. Find the probability that
		\label{ncert/12/13/2/12}
\begin{enumerate}
\item both balls are red.
\item first ball is black and second is red.
\item one of them is black and other is red.
\end{enumerate}

\item In a hostel, 60\% of the students read Hindi newspaper, 40\% read English newspaper and 20\% read both Hindi and English newspapers. A student is selected at random.
		\label{ncert/12/13/2/15}
\begin{enumerate}
\item Find the probability that she reads neither Hindi nor English newspapers.
\item If she reads Hindi newspaper, find the probability that she reads English newspaper.
\item If she reads English newspaper, find the probability that she reads Hindi newspaper.\\
\end{enumerate}
\item The probability of obtaining an even prime number on each die, when a pair of dice is rolled is 
\begin{enumerate}
    \item $0$ 
    
    \item $\frac{1}{3}$ 
    
    \item $\frac{1}{12}$ 
    
    \item $\frac{1}{36}$ 
\end{enumerate}
\solution
		%\input{ncert/12/13/2/17/defs.tex}
	\item A bag contains 4 red and 4 black balls, another bag contains 2 red and 6 black balls. One of the two bags is selected at random and a ball is drawn from the bag which is found to be red. Find the probability that the ball is drawn from the first bag.
\\
\solution
		%\input{ncert/12/13/3/2/main.tex}
  \item
  Cards with numbers 2 to 101 are placed in a box. A card is selected at random.Find the probability that the card has
\begin{enumerate}[label=(\roman*)]
	\item an even number 
	\item a square number
\end{enumerate}
\solution
%\input{exemplar/10/13/3/32/main.tex}
\item
The king, queen and jack of clubs are removed from a deck of 52 playing cards and then well shuffled. Now one card is drawn at random from the remaining cards.  Determine the probability that the card is
\begin{enumerate}[label=(\roman*)]
\item a club
\item 10 of hearts
\end{enumerate}
\solution
%\input{exemplar/10/13/3/29/main.tex}
\item A team of medical students doing their internship have to assist during surgeries
at a city hospital. The probabilities of surgeries rated as very complex, complex,
routine, simple or very simple are respectively, 0.15, 0.20, 0.31, 0.26, .08. Find
the probabilities that a particular surgery will be rated
\begin{enumerate}
	\item complex or very complex;
	\item neither very complex nor very simple;
	\item routine or complex
	\item routine or simple
\end{enumerate}
\solution
%\input{exemplar/11/16/3/8(1)/main.tex}
\item A card is selected from a pack of 52 cards.
\begin{enumerate}[label=(\alph*)]
    \item How many points are there in the sample space?
    \item Calculate the probability that the card is an ace of spades.
    \item Calculate the probability that the card is (i) an ace and (ii) black card.
\end{enumerate}
\solution
%\input{exemplar/11/16/3/4/main2.tex}
\item The probability that a non leap year selected at random will contain 53 sundays.
\\
\solution
%\input{exemplar/10/13/1/19/main.tex}
\item One of the four persons John, Rita, Aslam or Gurpreet will be promoted next
month. Consequently the sample space consists of four elementary outcomes
S = {John promoted, Rita promoted, Aslam promoted, Gurpreet promoted}
You are told that the chances of John’s promotion is same as that of Gurpreet,
Rita’s chances of promotion are twice as likely as Johns. Aslam’s chances are
four times that of John.
\begin{enumerate}
	\item Determine
	\begin{enumerate}
		\item P (John promoted)
		\item P (Rita promoted)
		\item P (Aslam promoted)
		\item P (Gurpreet promoted)
	\end{enumerate}
	\item If A = {John promoted or Gurpreet promoted}, find P (A).
\end{enumerate}
\solution
%\input{exemplar/11/16/3/10/main.tex}
\item A card is drawn from a deck of 52 cards. Find the probability of getting a king or a heart or a red card.\\
\solution
%\input{exemplar/11/16/3/15/main.tex}
\item The probability that a student will pass his examination is 0.73, the probability of
the student getting a compartment is 0.13, and the probability that the student will
either pass or get compartment is 0.96. State True or False.\\
\solution
%\input{exemplar/11/16/3/31/main.tex}
\item A card is selected from a pack of 52 cards\\
\begin{enumerate}[label=(\alph*)]
\item How many points are there in the sample space?
\item Calculate the probability that the cards is an ace of spades.
\item Calculate the probability that the card is (i) an ace (ii)black card.\\
\end{enumerate}
%\input{ncert/11/16/3/4_1/Prob_4.tex}
\item In a non-leap year, the probability of having 53 tuesdays or 53 wednesdays is\\
\solution
%\input{exemplar/11/16/3/18/main.tex}
\item There are 1000 sealed envelopes in a box, 10 of them contain a cash prize of
Rs 100 each, 100 of them contain a cash prize of Rs 50 each and 200 of them
contain a cash prize of Rs 10 each and rest do not contain any cash prize. If they
are well shuffled and an envelope is picked up out, what is the probability that it
contains no cash prize?\\
\solution
%\input{exemplar/10/13/3/34/main.tex}
\item 
A die is thrown and a card is selected at random from a deck of 52 playing cards. The probability of getting an even number on the die and a spade card.\\
\solution
%\input{exemplar/12/13/3/78/main.tex}
\item
If 4-digit numbers greater than 5,000 are randomly formed from the digits 0, 1, 3, 5, and 7, what is the probability of forming a number divisible by 5 when:
\begin{enumerate}
    \item The digits are repeated?
    \item The repetition of digits is not allowed?
\end{enumerate}
\solution
%\input{ncert/11/16/4/9/main.tex}
\item Consider the probability space $\brak{\Omega, \mathcal{G}, P}$ where $\Omega = [0,2]$ and $\mathcal{G} = \cbrak{\phi, \Omega, [0,1], (1,2]}$. Let $X$ and $Y$ be two functions on $\Omega$ defined as
\begin{align*}
    X(\omega) = 
    \begin{cases}
        1 & \text{if }\omega \in [0, 1]\\
        2 & \text{if }\omega \in (1, 2]
    \end{cases}
\end{align*}
and
\begin{align*}
    Y(\omega) = 
    \begin{cases}
        2 & \text{if }\omega \in [0, 1.5]\\
        3 & \text{if }\omega \in (1.5, 2].
    \end{cases}
\end{align*}
Then which one of the following statements is true?
\begin{enumerate}
    \item [(A)] $X$ is a random variable with respect to $\mathcal{G}$, but $Y$ is not a random variable with respect to $\mathcal{G}$.
    \item [(B)] $Y$ is a random variable with respect to $\mathcal{G}$, but $X$ is not a random variable with respect to $\mathcal{G}$.
    \item [(C)] Neither $X$ nor $Y$ is a random variable with respect to $\mathcal{G}$.
    \item [(D)] Both $X$ and $Y$ are random variables with respect to $\mathcal{G}$.
\end{enumerate} \hfill (GATE ST 2023)\\
\solution
%\input{gate/ST/2023/14/main.tex}
	\item  A die is loaded in such a way that each odd number is twice as likely to occur as
each even number. Find $P(G)$, where $G$ is the event that a number greater than
3 occurs on a single roll of the die.
\\
\solution
		%\input{exemplar/11/16/3/5/main.tex}
	\item All the jacks, queens and kings are removed from a deck of 52 playing cards. The remaining cards are well shuffled and then one card is drawn at random. Giving ace a value 1 similar value for other cards, find the probability that the card has a value 
		\begin{enumerate}
			\item 7
			\item greater than 7
			\item less than 7
		\end{enumerate}
		%\input{exemplar/10/13/3/30/main.tex}
  \item A Lot consists of 48 mobile phones of which 42 are good, 3 have only minor defects and 3 have major defects.Varnika will buy a phone if it is good but the trader will only buy a mobile if it has no major defects. One phone is selected at random from the lot. What is the probability that it is
\begin{enumerate}
	\item acceptable to Varnika?
            \item acceptable to the trader?
\end{enumerate}
\solution
	%\input{exemplar/10/13/3/40/main.tex}
 \item A student says that if you throw a die, it will show up 1 or not 1. Therefore, the probability of getting 1 and the probability of getting 'not 1' each is equal to $\frac{1}{2}$. Is this correct? Give reasons.\\
 \solution
        %\input{exemplar/10/13/2/9/main.tex}
   \item Four candidates A, B, C, D have ap-
plied for the assignment to coach a school cricket
team. If A is twice as likely to be selected as B, and
B and C are given about the same chance of being
selected, while C is twice as likely to be selected
as D, what are the probabilities that
\begin{enumerate}
\item C will be selected?
\item A will not be selected?
\end{enumerate}
	%\input{exemplar/11/16/3/9/main.tex}
 \item A bag contain 24 balls of which $x$ balls are red, $2x$ are white and $3x$ are blue. A ball is selected at random, What is the probability that it is
\begin{enumerate}[label=\alph*)]
\item not red ?
\item white ?
\end{enumerate}
%\input{exemplar/10/13/3/41/main.tex}
If the letters of the word ASSASSINATION are arranged at random. Find the Probability that
\begin{enumerate}[label=(\alph*)]
\item Four $S's$ come consecutively in the word
\item Two  $I's$ and two $N's$ come together
\item All $A's$ are not coming together
\item No two $A's$ are coming together
\end{enumerate}
%\input{exemplar/11/16/3/14/main.tex}
	\item One urn contains two black balls (labelled B1 and B2) and one white ball. A
	second urn contains one black ball and two white balls (labelled W1 and W2).
	Suppose the following experiment is performed. One of the two urns is chosen
	at random. Next a ball is randomly chosen from the urn. Then a second ball is
	chosen at random from the same urn without replacing the first ball.
	
	\begin{enumerate}
	\item What is the probability that two black balls are chosen?
	
	\item What is the probability that two balls of opposite colour are chosen?
	\end{enumerate}
	\solution
	%\input{exemplar/11/16/3/12/main1.tex}
\end{enumerate}

	\item 
The number lock of a suitcase has 4 wheels each labelled with ten digits i.e. from 0 to 9.The lock opens with a sequence of four digits with no repeats.What is the probability of a person getting the right sequence to open the suitcase.
\\
\solution
		%\begin{enumerate}[label=\thesection.\arabic*,ref=\thesection.\theenumi]
	\item One card is drawn from a well-shuffled deck of 52 cards. Find the probability of getting
\begin{enumerate}
\item A king of red colour 
\item A face card 
\item A red face card
\item The jack of hearts
\item A spade
\item The queen of diamonds

\end{enumerate}
\solution
		%\input{ncert/10/15/1/14/main.tex}
	\item Five cards—the ten, jack, queen, king and ace of diamonds, are well-shuffled with their face downwards. One card is then picked up at random.
\begin{enumerate}
\item
What is the probability that the card is the queen? 
\item
If the queen is drawn and put aside, what is the probability that the second card picked up is (a) an ace? (b) a queen?\\
\end{enumerate}
\solution
		%\input{ncert/10/15/1/15/defs.tex}
	\item A bag contains $5$ red balls and some blue balls. If the probability of drawing a blue ball is double that if a red ball, determine the number of blue balls in the bag. 
		\\
\solution
		%\input{ncert/10/15/2/3/defs.tex}
	\item A card is selected from a pack of 52 cards.
 \begin{enumerate}[label=(\alph*)] 
                 \item How many points are there in the sample space?
                 \item Calculate the probability that the card is an ace of spades.
                 \item Calculate the probability that the card is (i) an ace and (ii) black card.
 \end{enumerate}
\solution
		%\input{ncert/11/16/3/4/main.tex}
\item Four cards are drawn from a well-shuffled deck of 52 cards. What is the probability of obtaining 3 diamonds and one spade.
\\
\solution
		%\input{ncert/11/16/4/2/defs.tex}
\item In a certain lottery 10,000 tickets are sold and ten equal prizes are awarded. What is the probability of not getting a prize if you buy (a) one ticket (b) two tickets (c) 10 tickets ?	
\\
\solution
		%\input{ncert/11/16/4/4/defs.tex}
		%
\item 
Out of 100 students, two sections of 40 and 60 are formed. If you and your friend are among the 100 students, what is the probability that
\begin{enumerate}
\item you both enter the same section?
\item you both enter the different sections?
\end{enumerate}
\solution
		%\input{ncert/11/16/4/5/defs.tex}
	\item 
The number lock of a suitcase has 4 wheels each labelled with ten digits i.e. from 0 to 9.The lock opens with a sequence of four digits with no repeats.What is the probability of a person getting the right sequence to open the suitcase.
\\
\solution
		%\input{ncert/11/16/4/10/defs.tex}
		%
\item 
Two cards are drawn at random and without replacement from a pack of 52 playing cards. Find the probability that both the cards are black.
\\
\solution
		%\input{ncert/12/13/2/2/defs.tex}
		\item A box of oranges is inspected by examining three randomly selected oranges drawn without replacement. If all the three oranges are good, the box is approved for sale, otherwise, it is rejected. Find the probability that a box containing 15 oranges out of which 12 are good and 3 are bad ones will be approved for sale.
		\label{ncert/12/13/2/3/defs.tex}
		\item Two balls are drawn at random with replacement from a box containing 10 black and 8 red balls. Find the probability that
		\label{ncert/12/13/2/12}
\begin{enumerate}
\item both balls are red.
\item first ball is black and second is red.
\item one of them is black and other is red.
\end{enumerate}

\item In a hostel, 60\% of the students read Hindi newspaper, 40\% read English newspaper and 20\% read both Hindi and English newspapers. A student is selected at random.
		\label{ncert/12/13/2/15}
\begin{enumerate}
\item Find the probability that she reads neither Hindi nor English newspapers.
\item If she reads Hindi newspaper, find the probability that she reads English newspaper.
\item If she reads English newspaper, find the probability that she reads Hindi newspaper.\\
\end{enumerate}
\item The probability of obtaining an even prime number on each die, when a pair of dice is rolled is 
\begin{enumerate}
    \item $0$ 
    
    \item $\frac{1}{3}$ 
    
    \item $\frac{1}{12}$ 
    
    \item $\frac{1}{36}$ 
\end{enumerate}
\solution
		%\input{ncert/12/13/2/17/defs.tex}
	\item A bag contains 4 red and 4 black balls, another bag contains 2 red and 6 black balls. One of the two bags is selected at random and a ball is drawn from the bag which is found to be red. Find the probability that the ball is drawn from the first bag.
\\
\solution
		%\input{ncert/12/13/3/2/main.tex}
  \item
  Cards with numbers 2 to 101 are placed in a box. A card is selected at random.Find the probability that the card has
\begin{enumerate}[label=(\roman*)]
	\item an even number 
	\item a square number
\end{enumerate}
\solution
%\input{exemplar/10/13/3/32/main.tex}
\item
The king, queen and jack of clubs are removed from a deck of 52 playing cards and then well shuffled. Now one card is drawn at random from the remaining cards.  Determine the probability that the card is
\begin{enumerate}[label=(\roman*)]
\item a club
\item 10 of hearts
\end{enumerate}
\solution
%\input{exemplar/10/13/3/29/main.tex}
\item A team of medical students doing their internship have to assist during surgeries
at a city hospital. The probabilities of surgeries rated as very complex, complex,
routine, simple or very simple are respectively, 0.15, 0.20, 0.31, 0.26, .08. Find
the probabilities that a particular surgery will be rated
\begin{enumerate}
	\item complex or very complex;
	\item neither very complex nor very simple;
	\item routine or complex
	\item routine or simple
\end{enumerate}
\solution
%\input{exemplar/11/16/3/8(1)/main.tex}
\item A card is selected from a pack of 52 cards.
\begin{enumerate}[label=(\alph*)]
    \item How many points are there in the sample space?
    \item Calculate the probability that the card is an ace of spades.
    \item Calculate the probability that the card is (i) an ace and (ii) black card.
\end{enumerate}
\solution
%\input{exemplar/11/16/3/4/main2.tex}
\item The probability that a non leap year selected at random will contain 53 sundays.
\\
\solution
%\input{exemplar/10/13/1/19/main.tex}
\item One of the four persons John, Rita, Aslam or Gurpreet will be promoted next
month. Consequently the sample space consists of four elementary outcomes
S = {John promoted, Rita promoted, Aslam promoted, Gurpreet promoted}
You are told that the chances of John’s promotion is same as that of Gurpreet,
Rita’s chances of promotion are twice as likely as Johns. Aslam’s chances are
four times that of John.
\begin{enumerate}
	\item Determine
	\begin{enumerate}
		\item P (John promoted)
		\item P (Rita promoted)
		\item P (Aslam promoted)
		\item P (Gurpreet promoted)
	\end{enumerate}
	\item If A = {John promoted or Gurpreet promoted}, find P (A).
\end{enumerate}
\solution
%\input{exemplar/11/16/3/10/main.tex}
\item A card is drawn from a deck of 52 cards. Find the probability of getting a king or a heart or a red card.\\
\solution
%\input{exemplar/11/16/3/15/main.tex}
\item The probability that a student will pass his examination is 0.73, the probability of
the student getting a compartment is 0.13, and the probability that the student will
either pass or get compartment is 0.96. State True or False.\\
\solution
%\input{exemplar/11/16/3/31/main.tex}
\item A card is selected from a pack of 52 cards\\
\begin{enumerate}[label=(\alph*)]
\item How many points are there in the sample space?
\item Calculate the probability that the cards is an ace of spades.
\item Calculate the probability that the card is (i) an ace (ii)black card.\\
\end{enumerate}
%\input{ncert/11/16/3/4_1/Prob_4.tex}
\item In a non-leap year, the probability of having 53 tuesdays or 53 wednesdays is\\
\solution
%\input{exemplar/11/16/3/18/main.tex}
\item There are 1000 sealed envelopes in a box, 10 of them contain a cash prize of
Rs 100 each, 100 of them contain a cash prize of Rs 50 each and 200 of them
contain a cash prize of Rs 10 each and rest do not contain any cash prize. If they
are well shuffled and an envelope is picked up out, what is the probability that it
contains no cash prize?\\
\solution
%\input{exemplar/10/13/3/34/main.tex}
\item 
A die is thrown and a card is selected at random from a deck of 52 playing cards. The probability of getting an even number on the die and a spade card.\\
\solution
%\input{exemplar/12/13/3/78/main.tex}
\item
If 4-digit numbers greater than 5,000 are randomly formed from the digits 0, 1, 3, 5, and 7, what is the probability of forming a number divisible by 5 when:
\begin{enumerate}
    \item The digits are repeated?
    \item The repetition of digits is not allowed?
\end{enumerate}
\solution
%\input{ncert/11/16/4/9/main.tex}
\item Consider the probability space $\brak{\Omega, \mathcal{G}, P}$ where $\Omega = [0,2]$ and $\mathcal{G} = \cbrak{\phi, \Omega, [0,1], (1,2]}$. Let $X$ and $Y$ be two functions on $\Omega$ defined as
\begin{align*}
    X(\omega) = 
    \begin{cases}
        1 & \text{if }\omega \in [0, 1]\\
        2 & \text{if }\omega \in (1, 2]
    \end{cases}
\end{align*}
and
\begin{align*}
    Y(\omega) = 
    \begin{cases}
        2 & \text{if }\omega \in [0, 1.5]\\
        3 & \text{if }\omega \in (1.5, 2].
    \end{cases}
\end{align*}
Then which one of the following statements is true?
\begin{enumerate}
    \item [(A)] $X$ is a random variable with respect to $\mathcal{G}$, but $Y$ is not a random variable with respect to $\mathcal{G}$.
    \item [(B)] $Y$ is a random variable with respect to $\mathcal{G}$, but $X$ is not a random variable with respect to $\mathcal{G}$.
    \item [(C)] Neither $X$ nor $Y$ is a random variable with respect to $\mathcal{G}$.
    \item [(D)] Both $X$ and $Y$ are random variables with respect to $\mathcal{G}$.
\end{enumerate} \hfill (GATE ST 2023)\\
\solution
%\input{gate/ST/2023/14/main.tex}
	\item  A die is loaded in such a way that each odd number is twice as likely to occur as
each even number. Find $P(G)$, where $G$ is the event that a number greater than
3 occurs on a single roll of the die.
\\
\solution
		%\input{exemplar/11/16/3/5/main.tex}
	\item All the jacks, queens and kings are removed from a deck of 52 playing cards. The remaining cards are well shuffled and then one card is drawn at random. Giving ace a value 1 similar value for other cards, find the probability that the card has a value 
		\begin{enumerate}
			\item 7
			\item greater than 7
			\item less than 7
		\end{enumerate}
		%\input{exemplar/10/13/3/30/main.tex}
  \item A Lot consists of 48 mobile phones of which 42 are good, 3 have only minor defects and 3 have major defects.Varnika will buy a phone if it is good but the trader will only buy a mobile if it has no major defects. One phone is selected at random from the lot. What is the probability that it is
\begin{enumerate}
	\item acceptable to Varnika?
            \item acceptable to the trader?
\end{enumerate}
\solution
	%\input{exemplar/10/13/3/40/main.tex}
 \item A student says that if you throw a die, it will show up 1 or not 1. Therefore, the probability of getting 1 and the probability of getting 'not 1' each is equal to $\frac{1}{2}$. Is this correct? Give reasons.\\
 \solution
        %\input{exemplar/10/13/2/9/main.tex}
   \item Four candidates A, B, C, D have ap-
plied for the assignment to coach a school cricket
team. If A is twice as likely to be selected as B, and
B and C are given about the same chance of being
selected, while C is twice as likely to be selected
as D, what are the probabilities that
\begin{enumerate}
\item C will be selected?
\item A will not be selected?
\end{enumerate}
	%\input{exemplar/11/16/3/9/main.tex}
 \item A bag contain 24 balls of which $x$ balls are red, $2x$ are white and $3x$ are blue. A ball is selected at random, What is the probability that it is
\begin{enumerate}[label=\alph*)]
\item not red ?
\item white ?
\end{enumerate}
%\input{exemplar/10/13/3/41/main.tex}
If the letters of the word ASSASSINATION are arranged at random. Find the Probability that
\begin{enumerate}[label=(\alph*)]
\item Four $S's$ come consecutively in the word
\item Two  $I's$ and two $N's$ come together
\item All $A's$ are not coming together
\item No two $A's$ are coming together
\end{enumerate}
%\input{exemplar/11/16/3/14/main.tex}
	\item One urn contains two black balls (labelled B1 and B2) and one white ball. A
	second urn contains one black ball and two white balls (labelled W1 and W2).
	Suppose the following experiment is performed. One of the two urns is chosen
	at random. Next a ball is randomly chosen from the urn. Then a second ball is
	chosen at random from the same urn without replacing the first ball.
	
	\begin{enumerate}
	\item What is the probability that two black balls are chosen?
	
	\item What is the probability that two balls of opposite colour are chosen?
	\end{enumerate}
	\solution
	%\input{exemplar/11/16/3/12/main1.tex}
\end{enumerate}

		%
\item 
Two cards are drawn at random and without replacement from a pack of 52 playing cards. Find the probability that both the cards are black.
\\
\solution
		%\begin{enumerate}[label=\thesection.\arabic*,ref=\thesection.\theenumi]
	\item One card is drawn from a well-shuffled deck of 52 cards. Find the probability of getting
\begin{enumerate}
\item A king of red colour 
\item A face card 
\item A red face card
\item The jack of hearts
\item A spade
\item The queen of diamonds

\end{enumerate}
\solution
		%\input{ncert/10/15/1/14/main.tex}
	\item Five cards—the ten, jack, queen, king and ace of diamonds, are well-shuffled with their face downwards. One card is then picked up at random.
\begin{enumerate}
\item
What is the probability that the card is the queen? 
\item
If the queen is drawn and put aside, what is the probability that the second card picked up is (a) an ace? (b) a queen?\\
\end{enumerate}
\solution
		%\input{ncert/10/15/1/15/defs.tex}
	\item A bag contains $5$ red balls and some blue balls. If the probability of drawing a blue ball is double that if a red ball, determine the number of blue balls in the bag. 
		\\
\solution
		%\input{ncert/10/15/2/3/defs.tex}
	\item A card is selected from a pack of 52 cards.
 \begin{enumerate}[label=(\alph*)] 
                 \item How many points are there in the sample space?
                 \item Calculate the probability that the card is an ace of spades.
                 \item Calculate the probability that the card is (i) an ace and (ii) black card.
 \end{enumerate}
\solution
		%\input{ncert/11/16/3/4/main.tex}
\item Four cards are drawn from a well-shuffled deck of 52 cards. What is the probability of obtaining 3 diamonds and one spade.
\\
\solution
		%\input{ncert/11/16/4/2/defs.tex}
\item In a certain lottery 10,000 tickets are sold and ten equal prizes are awarded. What is the probability of not getting a prize if you buy (a) one ticket (b) two tickets (c) 10 tickets ?	
\\
\solution
		%\input{ncert/11/16/4/4/defs.tex}
		%
\item 
Out of 100 students, two sections of 40 and 60 are formed. If you and your friend are among the 100 students, what is the probability that
\begin{enumerate}
\item you both enter the same section?
\item you both enter the different sections?
\end{enumerate}
\solution
		%\input{ncert/11/16/4/5/defs.tex}
	\item 
The number lock of a suitcase has 4 wheels each labelled with ten digits i.e. from 0 to 9.The lock opens with a sequence of four digits with no repeats.What is the probability of a person getting the right sequence to open the suitcase.
\\
\solution
		%\input{ncert/11/16/4/10/defs.tex}
		%
\item 
Two cards are drawn at random and without replacement from a pack of 52 playing cards. Find the probability that both the cards are black.
\\
\solution
		%\input{ncert/12/13/2/2/defs.tex}
		\item A box of oranges is inspected by examining three randomly selected oranges drawn without replacement. If all the three oranges are good, the box is approved for sale, otherwise, it is rejected. Find the probability that a box containing 15 oranges out of which 12 are good and 3 are bad ones will be approved for sale.
		\label{ncert/12/13/2/3/defs.tex}
		\item Two balls are drawn at random with replacement from a box containing 10 black and 8 red balls. Find the probability that
		\label{ncert/12/13/2/12}
\begin{enumerate}
\item both balls are red.
\item first ball is black and second is red.
\item one of them is black and other is red.
\end{enumerate}

\item In a hostel, 60\% of the students read Hindi newspaper, 40\% read English newspaper and 20\% read both Hindi and English newspapers. A student is selected at random.
		\label{ncert/12/13/2/15}
\begin{enumerate}
\item Find the probability that she reads neither Hindi nor English newspapers.
\item If she reads Hindi newspaper, find the probability that she reads English newspaper.
\item If she reads English newspaper, find the probability that she reads Hindi newspaper.\\
\end{enumerate}
\item The probability of obtaining an even prime number on each die, when a pair of dice is rolled is 
\begin{enumerate}
    \item $0$ 
    
    \item $\frac{1}{3}$ 
    
    \item $\frac{1}{12}$ 
    
    \item $\frac{1}{36}$ 
\end{enumerate}
\solution
		%\input{ncert/12/13/2/17/defs.tex}
	\item A bag contains 4 red and 4 black balls, another bag contains 2 red and 6 black balls. One of the two bags is selected at random and a ball is drawn from the bag which is found to be red. Find the probability that the ball is drawn from the first bag.
\\
\solution
		%\input{ncert/12/13/3/2/main.tex}
  \item
  Cards with numbers 2 to 101 are placed in a box. A card is selected at random.Find the probability that the card has
\begin{enumerate}[label=(\roman*)]
	\item an even number 
	\item a square number
\end{enumerate}
\solution
%\input{exemplar/10/13/3/32/main.tex}
\item
The king, queen and jack of clubs are removed from a deck of 52 playing cards and then well shuffled. Now one card is drawn at random from the remaining cards.  Determine the probability that the card is
\begin{enumerate}[label=(\roman*)]
\item a club
\item 10 of hearts
\end{enumerate}
\solution
%\input{exemplar/10/13/3/29/main.tex}
\item A team of medical students doing their internship have to assist during surgeries
at a city hospital. The probabilities of surgeries rated as very complex, complex,
routine, simple or very simple are respectively, 0.15, 0.20, 0.31, 0.26, .08. Find
the probabilities that a particular surgery will be rated
\begin{enumerate}
	\item complex or very complex;
	\item neither very complex nor very simple;
	\item routine or complex
	\item routine or simple
\end{enumerate}
\solution
%\input{exemplar/11/16/3/8(1)/main.tex}
\item A card is selected from a pack of 52 cards.
\begin{enumerate}[label=(\alph*)]
    \item How many points are there in the sample space?
    \item Calculate the probability that the card is an ace of spades.
    \item Calculate the probability that the card is (i) an ace and (ii) black card.
\end{enumerate}
\solution
%\input{exemplar/11/16/3/4/main2.tex}
\item The probability that a non leap year selected at random will contain 53 sundays.
\\
\solution
%\input{exemplar/10/13/1/19/main.tex}
\item One of the four persons John, Rita, Aslam or Gurpreet will be promoted next
month. Consequently the sample space consists of four elementary outcomes
S = {John promoted, Rita promoted, Aslam promoted, Gurpreet promoted}
You are told that the chances of John’s promotion is same as that of Gurpreet,
Rita’s chances of promotion are twice as likely as Johns. Aslam’s chances are
four times that of John.
\begin{enumerate}
	\item Determine
	\begin{enumerate}
		\item P (John promoted)
		\item P (Rita promoted)
		\item P (Aslam promoted)
		\item P (Gurpreet promoted)
	\end{enumerate}
	\item If A = {John promoted or Gurpreet promoted}, find P (A).
\end{enumerate}
\solution
%\input{exemplar/11/16/3/10/main.tex}
\item A card is drawn from a deck of 52 cards. Find the probability of getting a king or a heart or a red card.\\
\solution
%\input{exemplar/11/16/3/15/main.tex}
\item The probability that a student will pass his examination is 0.73, the probability of
the student getting a compartment is 0.13, and the probability that the student will
either pass or get compartment is 0.96. State True or False.\\
\solution
%\input{exemplar/11/16/3/31/main.tex}
\item A card is selected from a pack of 52 cards\\
\begin{enumerate}[label=(\alph*)]
\item How many points are there in the sample space?
\item Calculate the probability that the cards is an ace of spades.
\item Calculate the probability that the card is (i) an ace (ii)black card.\\
\end{enumerate}
%\input{ncert/11/16/3/4_1/Prob_4.tex}
\item In a non-leap year, the probability of having 53 tuesdays or 53 wednesdays is\\
\solution
%\input{exemplar/11/16/3/18/main.tex}
\item There are 1000 sealed envelopes in a box, 10 of them contain a cash prize of
Rs 100 each, 100 of them contain a cash prize of Rs 50 each and 200 of them
contain a cash prize of Rs 10 each and rest do not contain any cash prize. If they
are well shuffled and an envelope is picked up out, what is the probability that it
contains no cash prize?\\
\solution
%\input{exemplar/10/13/3/34/main.tex}
\item 
A die is thrown and a card is selected at random from a deck of 52 playing cards. The probability of getting an even number on the die and a spade card.\\
\solution
%\input{exemplar/12/13/3/78/main.tex}
\item
If 4-digit numbers greater than 5,000 are randomly formed from the digits 0, 1, 3, 5, and 7, what is the probability of forming a number divisible by 5 when:
\begin{enumerate}
    \item The digits are repeated?
    \item The repetition of digits is not allowed?
\end{enumerate}
\solution
%\input{ncert/11/16/4/9/main.tex}
\item Consider the probability space $\brak{\Omega, \mathcal{G}, P}$ where $\Omega = [0,2]$ and $\mathcal{G} = \cbrak{\phi, \Omega, [0,1], (1,2]}$. Let $X$ and $Y$ be two functions on $\Omega$ defined as
\begin{align*}
    X(\omega) = 
    \begin{cases}
        1 & \text{if }\omega \in [0, 1]\\
        2 & \text{if }\omega \in (1, 2]
    \end{cases}
\end{align*}
and
\begin{align*}
    Y(\omega) = 
    \begin{cases}
        2 & \text{if }\omega \in [0, 1.5]\\
        3 & \text{if }\omega \in (1.5, 2].
    \end{cases}
\end{align*}
Then which one of the following statements is true?
\begin{enumerate}
    \item [(A)] $X$ is a random variable with respect to $\mathcal{G}$, but $Y$ is not a random variable with respect to $\mathcal{G}$.
    \item [(B)] $Y$ is a random variable with respect to $\mathcal{G}$, but $X$ is not a random variable with respect to $\mathcal{G}$.
    \item [(C)] Neither $X$ nor $Y$ is a random variable with respect to $\mathcal{G}$.
    \item [(D)] Both $X$ and $Y$ are random variables with respect to $\mathcal{G}$.
\end{enumerate} \hfill (GATE ST 2023)\\
\solution
%\input{gate/ST/2023/14/main.tex}
	\item  A die is loaded in such a way that each odd number is twice as likely to occur as
each even number. Find $P(G)$, where $G$ is the event that a number greater than
3 occurs on a single roll of the die.
\\
\solution
		%\input{exemplar/11/16/3/5/main.tex}
	\item All the jacks, queens and kings are removed from a deck of 52 playing cards. The remaining cards are well shuffled and then one card is drawn at random. Giving ace a value 1 similar value for other cards, find the probability that the card has a value 
		\begin{enumerate}
			\item 7
			\item greater than 7
			\item less than 7
		\end{enumerate}
		%\input{exemplar/10/13/3/30/main.tex}
  \item A Lot consists of 48 mobile phones of which 42 are good, 3 have only minor defects and 3 have major defects.Varnika will buy a phone if it is good but the trader will only buy a mobile if it has no major defects. One phone is selected at random from the lot. What is the probability that it is
\begin{enumerate}
	\item acceptable to Varnika?
            \item acceptable to the trader?
\end{enumerate}
\solution
	%\input{exemplar/10/13/3/40/main.tex}
 \item A student says that if you throw a die, it will show up 1 or not 1. Therefore, the probability of getting 1 and the probability of getting 'not 1' each is equal to $\frac{1}{2}$. Is this correct? Give reasons.\\
 \solution
        %\input{exemplar/10/13/2/9/main.tex}
   \item Four candidates A, B, C, D have ap-
plied for the assignment to coach a school cricket
team. If A is twice as likely to be selected as B, and
B and C are given about the same chance of being
selected, while C is twice as likely to be selected
as D, what are the probabilities that
\begin{enumerate}
\item C will be selected?
\item A will not be selected?
\end{enumerate}
	%\input{exemplar/11/16/3/9/main.tex}
 \item A bag contain 24 balls of which $x$ balls are red, $2x$ are white and $3x$ are blue. A ball is selected at random, What is the probability that it is
\begin{enumerate}[label=\alph*)]
\item not red ?
\item white ?
\end{enumerate}
%\input{exemplar/10/13/3/41/main.tex}
If the letters of the word ASSASSINATION are arranged at random. Find the Probability that
\begin{enumerate}[label=(\alph*)]
\item Four $S's$ come consecutively in the word
\item Two  $I's$ and two $N's$ come together
\item All $A's$ are not coming together
\item No two $A's$ are coming together
\end{enumerate}
%\input{exemplar/11/16/3/14/main.tex}
	\item One urn contains two black balls (labelled B1 and B2) and one white ball. A
	second urn contains one black ball and two white balls (labelled W1 and W2).
	Suppose the following experiment is performed. One of the two urns is chosen
	at random. Next a ball is randomly chosen from the urn. Then a second ball is
	chosen at random from the same urn without replacing the first ball.
	
	\begin{enumerate}
	\item What is the probability that two black balls are chosen?
	
	\item What is the probability that two balls of opposite colour are chosen?
	\end{enumerate}
	\solution
	%\input{exemplar/11/16/3/12/main1.tex}
\end{enumerate}

		\item A box of oranges is inspected by examining three randomly selected oranges drawn without replacement. If all the three oranges are good, the box is approved for sale, otherwise, it is rejected. Find the probability that a box containing 15 oranges out of which 12 are good and 3 are bad ones will be approved for sale.
		\label{ncert/12/13/2/3/defs.tex}
		\item Two balls are drawn at random with replacement from a box containing 10 black and 8 red balls. Find the probability that
		\label{ncert/12/13/2/12}
\begin{enumerate}
\item both balls are red.
\item first ball is black and second is red.
\item one of them is black and other is red.
\end{enumerate}

\item In a hostel, 60\% of the students read Hindi newspaper, 40\% read English newspaper and 20\% read both Hindi and English newspapers. A student is selected at random.
		\label{ncert/12/13/2/15}
\begin{enumerate}
\item Find the probability that she reads neither Hindi nor English newspapers.
\item If she reads Hindi newspaper, find the probability that she reads English newspaper.
\item If she reads English newspaper, find the probability that she reads Hindi newspaper.\\
\end{enumerate}
\item The probability of obtaining an even prime number on each die, when a pair of dice is rolled is 
\begin{enumerate}
    \item $0$ 
    
    \item $\frac{1}{3}$ 
    
    \item $\frac{1}{12}$ 
    
    \item $\frac{1}{36}$ 
\end{enumerate}
\solution
		%\begin{enumerate}[label=\thesection.\arabic*,ref=\thesection.\theenumi]
	\item One card is drawn from a well-shuffled deck of 52 cards. Find the probability of getting
\begin{enumerate}
\item A king of red colour 
\item A face card 
\item A red face card
\item The jack of hearts
\item A spade
\item The queen of diamonds

\end{enumerate}
\solution
		%\input{ncert/10/15/1/14/main.tex}
	\item Five cards—the ten, jack, queen, king and ace of diamonds, are well-shuffled with their face downwards. One card is then picked up at random.
\begin{enumerate}
\item
What is the probability that the card is the queen? 
\item
If the queen is drawn and put aside, what is the probability that the second card picked up is (a) an ace? (b) a queen?\\
\end{enumerate}
\solution
		%\input{ncert/10/15/1/15/defs.tex}
	\item A bag contains $5$ red balls and some blue balls. If the probability of drawing a blue ball is double that if a red ball, determine the number of blue balls in the bag. 
		\\
\solution
		%\input{ncert/10/15/2/3/defs.tex}
	\item A card is selected from a pack of 52 cards.
 \begin{enumerate}[label=(\alph*)] 
                 \item How many points are there in the sample space?
                 \item Calculate the probability that the card is an ace of spades.
                 \item Calculate the probability that the card is (i) an ace and (ii) black card.
 \end{enumerate}
\solution
		%\input{ncert/11/16/3/4/main.tex}
\item Four cards are drawn from a well-shuffled deck of 52 cards. What is the probability of obtaining 3 diamonds and one spade.
\\
\solution
		%\input{ncert/11/16/4/2/defs.tex}
\item In a certain lottery 10,000 tickets are sold and ten equal prizes are awarded. What is the probability of not getting a prize if you buy (a) one ticket (b) two tickets (c) 10 tickets ?	
\\
\solution
		%\input{ncert/11/16/4/4/defs.tex}
		%
\item 
Out of 100 students, two sections of 40 and 60 are formed. If you and your friend are among the 100 students, what is the probability that
\begin{enumerate}
\item you both enter the same section?
\item you both enter the different sections?
\end{enumerate}
\solution
		%\input{ncert/11/16/4/5/defs.tex}
	\item 
The number lock of a suitcase has 4 wheels each labelled with ten digits i.e. from 0 to 9.The lock opens with a sequence of four digits with no repeats.What is the probability of a person getting the right sequence to open the suitcase.
\\
\solution
		%\input{ncert/11/16/4/10/defs.tex}
		%
\item 
Two cards are drawn at random and without replacement from a pack of 52 playing cards. Find the probability that both the cards are black.
\\
\solution
		%\input{ncert/12/13/2/2/defs.tex}
		\item A box of oranges is inspected by examining three randomly selected oranges drawn without replacement. If all the three oranges are good, the box is approved for sale, otherwise, it is rejected. Find the probability that a box containing 15 oranges out of which 12 are good and 3 are bad ones will be approved for sale.
		\label{ncert/12/13/2/3/defs.tex}
		\item Two balls are drawn at random with replacement from a box containing 10 black and 8 red balls. Find the probability that
		\label{ncert/12/13/2/12}
\begin{enumerate}
\item both balls are red.
\item first ball is black and second is red.
\item one of them is black and other is red.
\end{enumerate}

\item In a hostel, 60\% of the students read Hindi newspaper, 40\% read English newspaper and 20\% read both Hindi and English newspapers. A student is selected at random.
		\label{ncert/12/13/2/15}
\begin{enumerate}
\item Find the probability that she reads neither Hindi nor English newspapers.
\item If she reads Hindi newspaper, find the probability that she reads English newspaper.
\item If she reads English newspaper, find the probability that she reads Hindi newspaper.\\
\end{enumerate}
\item The probability of obtaining an even prime number on each die, when a pair of dice is rolled is 
\begin{enumerate}
    \item $0$ 
    
    \item $\frac{1}{3}$ 
    
    \item $\frac{1}{12}$ 
    
    \item $\frac{1}{36}$ 
\end{enumerate}
\solution
		%\input{ncert/12/13/2/17/defs.tex}
	\item A bag contains 4 red and 4 black balls, another bag contains 2 red and 6 black balls. One of the two bags is selected at random and a ball is drawn from the bag which is found to be red. Find the probability that the ball is drawn from the first bag.
\\
\solution
		%\input{ncert/12/13/3/2/main.tex}
  \item
  Cards with numbers 2 to 101 are placed in a box. A card is selected at random.Find the probability that the card has
\begin{enumerate}[label=(\roman*)]
	\item an even number 
	\item a square number
\end{enumerate}
\solution
%\input{exemplar/10/13/3/32/main.tex}
\item
The king, queen and jack of clubs are removed from a deck of 52 playing cards and then well shuffled. Now one card is drawn at random from the remaining cards.  Determine the probability that the card is
\begin{enumerate}[label=(\roman*)]
\item a club
\item 10 of hearts
\end{enumerate}
\solution
%\input{exemplar/10/13/3/29/main.tex}
\item A team of medical students doing their internship have to assist during surgeries
at a city hospital. The probabilities of surgeries rated as very complex, complex,
routine, simple or very simple are respectively, 0.15, 0.20, 0.31, 0.26, .08. Find
the probabilities that a particular surgery will be rated
\begin{enumerate}
	\item complex or very complex;
	\item neither very complex nor very simple;
	\item routine or complex
	\item routine or simple
\end{enumerate}
\solution
%\input{exemplar/11/16/3/8(1)/main.tex}
\item A card is selected from a pack of 52 cards.
\begin{enumerate}[label=(\alph*)]
    \item How many points are there in the sample space?
    \item Calculate the probability that the card is an ace of spades.
    \item Calculate the probability that the card is (i) an ace and (ii) black card.
\end{enumerate}
\solution
%\input{exemplar/11/16/3/4/main2.tex}
\item The probability that a non leap year selected at random will contain 53 sundays.
\\
\solution
%\input{exemplar/10/13/1/19/main.tex}
\item One of the four persons John, Rita, Aslam or Gurpreet will be promoted next
month. Consequently the sample space consists of four elementary outcomes
S = {John promoted, Rita promoted, Aslam promoted, Gurpreet promoted}
You are told that the chances of John’s promotion is same as that of Gurpreet,
Rita’s chances of promotion are twice as likely as Johns. Aslam’s chances are
four times that of John.
\begin{enumerate}
	\item Determine
	\begin{enumerate}
		\item P (John promoted)
		\item P (Rita promoted)
		\item P (Aslam promoted)
		\item P (Gurpreet promoted)
	\end{enumerate}
	\item If A = {John promoted or Gurpreet promoted}, find P (A).
\end{enumerate}
\solution
%\input{exemplar/11/16/3/10/main.tex}
\item A card is drawn from a deck of 52 cards. Find the probability of getting a king or a heart or a red card.\\
\solution
%\input{exemplar/11/16/3/15/main.tex}
\item The probability that a student will pass his examination is 0.73, the probability of
the student getting a compartment is 0.13, and the probability that the student will
either pass or get compartment is 0.96. State True or False.\\
\solution
%\input{exemplar/11/16/3/31/main.tex}
\item A card is selected from a pack of 52 cards\\
\begin{enumerate}[label=(\alph*)]
\item How many points are there in the sample space?
\item Calculate the probability that the cards is an ace of spades.
\item Calculate the probability that the card is (i) an ace (ii)black card.\\
\end{enumerate}
%\input{ncert/11/16/3/4_1/Prob_4.tex}
\item In a non-leap year, the probability of having 53 tuesdays or 53 wednesdays is\\
\solution
%\input{exemplar/11/16/3/18/main.tex}
\item There are 1000 sealed envelopes in a box, 10 of them contain a cash prize of
Rs 100 each, 100 of them contain a cash prize of Rs 50 each and 200 of them
contain a cash prize of Rs 10 each and rest do not contain any cash prize. If they
are well shuffled and an envelope is picked up out, what is the probability that it
contains no cash prize?\\
\solution
%\input{exemplar/10/13/3/34/main.tex}
\item 
A die is thrown and a card is selected at random from a deck of 52 playing cards. The probability of getting an even number on the die and a spade card.\\
\solution
%\input{exemplar/12/13/3/78/main.tex}
\item
If 4-digit numbers greater than 5,000 are randomly formed from the digits 0, 1, 3, 5, and 7, what is the probability of forming a number divisible by 5 when:
\begin{enumerate}
    \item The digits are repeated?
    \item The repetition of digits is not allowed?
\end{enumerate}
\solution
%\input{ncert/11/16/4/9/main.tex}
\item Consider the probability space $\brak{\Omega, \mathcal{G}, P}$ where $\Omega = [0,2]$ and $\mathcal{G} = \cbrak{\phi, \Omega, [0,1], (1,2]}$. Let $X$ and $Y$ be two functions on $\Omega$ defined as
\begin{align*}
    X(\omega) = 
    \begin{cases}
        1 & \text{if }\omega \in [0, 1]\\
        2 & \text{if }\omega \in (1, 2]
    \end{cases}
\end{align*}
and
\begin{align*}
    Y(\omega) = 
    \begin{cases}
        2 & \text{if }\omega \in [0, 1.5]\\
        3 & \text{if }\omega \in (1.5, 2].
    \end{cases}
\end{align*}
Then which one of the following statements is true?
\begin{enumerate}
    \item [(A)] $X$ is a random variable with respect to $\mathcal{G}$, but $Y$ is not a random variable with respect to $\mathcal{G}$.
    \item [(B)] $Y$ is a random variable with respect to $\mathcal{G}$, but $X$ is not a random variable with respect to $\mathcal{G}$.
    \item [(C)] Neither $X$ nor $Y$ is a random variable with respect to $\mathcal{G}$.
    \item [(D)] Both $X$ and $Y$ are random variables with respect to $\mathcal{G}$.
\end{enumerate} \hfill (GATE ST 2023)\\
\solution
%\input{gate/ST/2023/14/main.tex}
	\item  A die is loaded in such a way that each odd number is twice as likely to occur as
each even number. Find $P(G)$, where $G$ is the event that a number greater than
3 occurs on a single roll of the die.
\\
\solution
		%\input{exemplar/11/16/3/5/main.tex}
	\item All the jacks, queens and kings are removed from a deck of 52 playing cards. The remaining cards are well shuffled and then one card is drawn at random. Giving ace a value 1 similar value for other cards, find the probability that the card has a value 
		\begin{enumerate}
			\item 7
			\item greater than 7
			\item less than 7
		\end{enumerate}
		%\input{exemplar/10/13/3/30/main.tex}
  \item A Lot consists of 48 mobile phones of which 42 are good, 3 have only minor defects and 3 have major defects.Varnika will buy a phone if it is good but the trader will only buy a mobile if it has no major defects. One phone is selected at random from the lot. What is the probability that it is
\begin{enumerate}
	\item acceptable to Varnika?
            \item acceptable to the trader?
\end{enumerate}
\solution
	%\input{exemplar/10/13/3/40/main.tex}
 \item A student says that if you throw a die, it will show up 1 or not 1. Therefore, the probability of getting 1 and the probability of getting 'not 1' each is equal to $\frac{1}{2}$. Is this correct? Give reasons.\\
 \solution
        %\input{exemplar/10/13/2/9/main.tex}
   \item Four candidates A, B, C, D have ap-
plied for the assignment to coach a school cricket
team. If A is twice as likely to be selected as B, and
B and C are given about the same chance of being
selected, while C is twice as likely to be selected
as D, what are the probabilities that
\begin{enumerate}
\item C will be selected?
\item A will not be selected?
\end{enumerate}
	%\input{exemplar/11/16/3/9/main.tex}
 \item A bag contain 24 balls of which $x$ balls are red, $2x$ are white and $3x$ are blue. A ball is selected at random, What is the probability that it is
\begin{enumerate}[label=\alph*)]
\item not red ?
\item white ?
\end{enumerate}
%\input{exemplar/10/13/3/41/main.tex}
If the letters of the word ASSASSINATION are arranged at random. Find the Probability that
\begin{enumerate}[label=(\alph*)]
\item Four $S's$ come consecutively in the word
\item Two  $I's$ and two $N's$ come together
\item All $A's$ are not coming together
\item No two $A's$ are coming together
\end{enumerate}
%\input{exemplar/11/16/3/14/main.tex}
	\item One urn contains two black balls (labelled B1 and B2) and one white ball. A
	second urn contains one black ball and two white balls (labelled W1 and W2).
	Suppose the following experiment is performed. One of the two urns is chosen
	at random. Next a ball is randomly chosen from the urn. Then a second ball is
	chosen at random from the same urn without replacing the first ball.
	
	\begin{enumerate}
	\item What is the probability that two black balls are chosen?
	
	\item What is the probability that two balls of opposite colour are chosen?
	\end{enumerate}
	\solution
	%\input{exemplar/11/16/3/12/main1.tex}
\end{enumerate}

	\item A bag contains 4 red and 4 black balls, another bag contains 2 red and 6 black balls. One of the two bags is selected at random and a ball is drawn from the bag which is found to be red. Find the probability that the ball is drawn from the first bag.
\\
\solution
		%\begin{table}[H]
	\centering
\begin{tabular}{|c|c|c|}
\hline
Random variable &Value &Definition\\ \hline
\multirow{3}{*}{X} &0 &Slips of Rs 1\\
&1 &Slips of Rs 5\\
&2 &Slips of Rs 13\\ \hline
\multirow{2}{*}{Y} &0 &Box A\\
&1 &Box B\\\hline
\end{tabular}
\caption{}
\label{tab:Distribution}
\end{table}
See \tabref{tab:Distribution}.
\begin{align}
p_{Y}\brak{k}= \begin{cases} 
      \frac{1}{3} & {k=0} \\
      \frac{2}{3 }& {k=1} 
   \end{cases}
   \\
p_{Y|X}\brak{0|0} = \frac{19}{25}\, 
p_{Y|X}\brak{0|1} = \frac{6}{25}\,
p_{Y|X}\brak{1|0} = \frac{45}{50}\,
p_{Y|X}\brak{1|2} = \frac{5}{50}
\end{align}
The desired probability is the probability that a slip drawn at random is marked other than Rs 1,
\begin{align}
&=1-p_X\brak{0}\\
&= p_X(1) + p_X(2)
\end{align}
Using Bayes theorem,
\begin{align}
&= p_Y\brak{0} \times \pr{Y=0 | X=1} + p_Y\brak{1} \times \pr{Y=1|X=2}\\
&=\frac{1}{3} \times \frac{6}{25} + \frac{2}{3} \times \frac{5}{50}\\
&=\frac{11}{75}
\end{align}

\newpage

%\tableofcontents

\bigskip

\renewcommand{\thefigure}{\theenumi}
\renewcommand{\thetable}{\theenumi}
%\renewcommand{\theequation}{\theenumi}

%\begin{abstract}
%%\boldmath
%In this letter, an algorithm for evaluating the exact analytical bit error rate  (BER)  for the piecewise linear (PL) combiner for  multiple relays is presented. Previous results were available only for upto three relays. The algorithm is unique in the sense that  the actual mathematical expressions, that are prohibitively large, need not be explicitly obtained. The diversity gain due to multiple relays is shown through plots of the analytical BER, well supported by simulations. 
%
%\end{abstract}
% IEEEtran.cls defaults to using nonbold math in the Abstract.
% This preserves the distinction between vectors and scalars. However,
% if the journal you are submitting to favors bold math in the abstract,
% then you can use LaTeX's standard command \boldmath at the very start
% of the abstract to achieve this. Many IEEE journals frown on math
% in the abstract anyway.

% Note that keywords are not normally used for peerreview papers.
%\begin{IEEEkeywords}
%Cooperative diversity, decode and forward, piecewise linear
%\end{IEEEkeywords}



% For peer review papers, you can put extra information on the cover
% page as needed:
% \ifCLASSOPTIONpeerreview
% \begin{center} \bfseries EDICS Category: 3-BBND \end{center}
% \fi
%
% For peerreview papers, this IEEEtran command inserts a page break and
% creates the second title. It will be ignored for other modes.
%\IEEEpeerreviewmaketitle




  \item
  Cards with numbers 2 to 101 are placed in a box. A card is selected at random.Find the probability that the card has
\begin{enumerate}[label=(\roman*)]
	\item an even number 
	\item a square number
\end{enumerate}
\solution
%\begin{table}[H]
	\centering
\begin{tabular}{|c|c|c|}
\hline
Random variable &Value &Definition\\ \hline
\multirow{3}{*}{X} &0 &Slips of Rs 1\\
&1 &Slips of Rs 5\\
&2 &Slips of Rs 13\\ \hline
\multirow{2}{*}{Y} &0 &Box A\\
&1 &Box B\\\hline
\end{tabular}
\caption{}
\label{tab:Distribution}
\end{table}
See \tabref{tab:Distribution}.
\begin{align}
p_{Y}\brak{k}= \begin{cases} 
      \frac{1}{3} & {k=0} \\
      \frac{2}{3 }& {k=1} 
   \end{cases}
   \\
p_{Y|X}\brak{0|0} = \frac{19}{25}\, 
p_{Y|X}\brak{0|1} = \frac{6}{25}\,
p_{Y|X}\brak{1|0} = \frac{45}{50}\,
p_{Y|X}\brak{1|2} = \frac{5}{50}
\end{align}
The desired probability is the probability that a slip drawn at random is marked other than Rs 1,
\begin{align}
&=1-p_X\brak{0}\\
&= p_X(1) + p_X(2)
\end{align}
Using Bayes theorem,
\begin{align}
&= p_Y\brak{0} \times \pr{Y=0 | X=1} + p_Y\brak{1} \times \pr{Y=1|X=2}\\
&=\frac{1}{3} \times \frac{6}{25} + \frac{2}{3} \times \frac{5}{50}\\
&=\frac{11}{75}
\end{align}

\newpage

%\tableofcontents

\bigskip

\renewcommand{\thefigure}{\theenumi}
\renewcommand{\thetable}{\theenumi}
%\renewcommand{\theequation}{\theenumi}

%\begin{abstract}
%%\boldmath
%In this letter, an algorithm for evaluating the exact analytical bit error rate  (BER)  for the piecewise linear (PL) combiner for  multiple relays is presented. Previous results were available only for upto three relays. The algorithm is unique in the sense that  the actual mathematical expressions, that are prohibitively large, need not be explicitly obtained. The diversity gain due to multiple relays is shown through plots of the analytical BER, well supported by simulations. 
%
%\end{abstract}
% IEEEtran.cls defaults to using nonbold math in the Abstract.
% This preserves the distinction between vectors and scalars. However,
% if the journal you are submitting to favors bold math in the abstract,
% then you can use LaTeX's standard command \boldmath at the very start
% of the abstract to achieve this. Many IEEE journals frown on math
% in the abstract anyway.

% Note that keywords are not normally used for peerreview papers.
%\begin{IEEEkeywords}
%Cooperative diversity, decode and forward, piecewise linear
%\end{IEEEkeywords}



% For peer review papers, you can put extra information on the cover
% page as needed:
% \ifCLASSOPTIONpeerreview
% \begin{center} \bfseries EDICS Category: 3-BBND \end{center}
% \fi
%
% For peerreview papers, this IEEEtran command inserts a page break and
% creates the second title. It will be ignored for other modes.
%\IEEEpeerreviewmaketitle




\item
The king, queen and jack of clubs are removed from a deck of 52 playing cards and then well shuffled. Now one card is drawn at random from the remaining cards.  Determine the probability that the card is
\begin{enumerate}[label=(\roman*)]
\item a club
\item 10 of hearts
\end{enumerate}
\solution
%\begin{table}[H]
	\centering
\begin{tabular}{|c|c|c|}
\hline
Random variable &Value &Definition\\ \hline
\multirow{3}{*}{X} &0 &Slips of Rs 1\\
&1 &Slips of Rs 5\\
&2 &Slips of Rs 13\\ \hline
\multirow{2}{*}{Y} &0 &Box A\\
&1 &Box B\\\hline
\end{tabular}
\caption{}
\label{tab:Distribution}
\end{table}
See \tabref{tab:Distribution}.
\begin{align}
p_{Y}\brak{k}= \begin{cases} 
      \frac{1}{3} & {k=0} \\
      \frac{2}{3 }& {k=1} 
   \end{cases}
   \\
p_{Y|X}\brak{0|0} = \frac{19}{25}\, 
p_{Y|X}\brak{0|1} = \frac{6}{25}\,
p_{Y|X}\brak{1|0} = \frac{45}{50}\,
p_{Y|X}\brak{1|2} = \frac{5}{50}
\end{align}
The desired probability is the probability that a slip drawn at random is marked other than Rs 1,
\begin{align}
&=1-p_X\brak{0}\\
&= p_X(1) + p_X(2)
\end{align}
Using Bayes theorem,
\begin{align}
&= p_Y\brak{0} \times \pr{Y=0 | X=1} + p_Y\brak{1} \times \pr{Y=1|X=2}\\
&=\frac{1}{3} \times \frac{6}{25} + \frac{2}{3} \times \frac{5}{50}\\
&=\frac{11}{75}
\end{align}

\newpage

%\tableofcontents

\bigskip

\renewcommand{\thefigure}{\theenumi}
\renewcommand{\thetable}{\theenumi}
%\renewcommand{\theequation}{\theenumi}

%\begin{abstract}
%%\boldmath
%In this letter, an algorithm for evaluating the exact analytical bit error rate  (BER)  for the piecewise linear (PL) combiner for  multiple relays is presented. Previous results were available only for upto three relays. The algorithm is unique in the sense that  the actual mathematical expressions, that are prohibitively large, need not be explicitly obtained. The diversity gain due to multiple relays is shown through plots of the analytical BER, well supported by simulations. 
%
%\end{abstract}
% IEEEtran.cls defaults to using nonbold math in the Abstract.
% This preserves the distinction between vectors and scalars. However,
% if the journal you are submitting to favors bold math in the abstract,
% then you can use LaTeX's standard command \boldmath at the very start
% of the abstract to achieve this. Many IEEE journals frown on math
% in the abstract anyway.

% Note that keywords are not normally used for peerreview papers.
%\begin{IEEEkeywords}
%Cooperative diversity, decode and forward, piecewise linear
%\end{IEEEkeywords}



% For peer review papers, you can put extra information on the cover
% page as needed:
% \ifCLASSOPTIONpeerreview
% \begin{center} \bfseries EDICS Category: 3-BBND \end{center}
% \fi
%
% For peerreview papers, this IEEEtran command inserts a page break and
% creates the second title. It will be ignored for other modes.
%\IEEEpeerreviewmaketitle




\item A team of medical students doing their internship have to assist during surgeries
at a city hospital. The probabilities of surgeries rated as very complex, complex,
routine, simple or very simple are respectively, 0.15, 0.20, 0.31, 0.26, .08. Find
the probabilities that a particular surgery will be rated
\begin{enumerate}
	\item complex or very complex;
	\item neither very complex nor very simple;
	\item routine or complex
	\item routine or simple
\end{enumerate}
\solution
%\begin{table}[H]
	\centering
\begin{tabular}{|c|c|c|}
\hline
Random variable &Value &Definition\\ \hline
\multirow{3}{*}{X} &0 &Slips of Rs 1\\
&1 &Slips of Rs 5\\
&2 &Slips of Rs 13\\ \hline
\multirow{2}{*}{Y} &0 &Box A\\
&1 &Box B\\\hline
\end{tabular}
\caption{}
\label{tab:Distribution}
\end{table}
See \tabref{tab:Distribution}.
\begin{align}
p_{Y}\brak{k}= \begin{cases} 
      \frac{1}{3} & {k=0} \\
      \frac{2}{3 }& {k=1} 
   \end{cases}
   \\
p_{Y|X}\brak{0|0} = \frac{19}{25}\, 
p_{Y|X}\brak{0|1} = \frac{6}{25}\,
p_{Y|X}\brak{1|0} = \frac{45}{50}\,
p_{Y|X}\brak{1|2} = \frac{5}{50}
\end{align}
The desired probability is the probability that a slip drawn at random is marked other than Rs 1,
\begin{align}
&=1-p_X\brak{0}\\
&= p_X(1) + p_X(2)
\end{align}
Using Bayes theorem,
\begin{align}
&= p_Y\brak{0} \times \pr{Y=0 | X=1} + p_Y\brak{1} \times \pr{Y=1|X=2}\\
&=\frac{1}{3} \times \frac{6}{25} + \frac{2}{3} \times \frac{5}{50}\\
&=\frac{11}{75}
\end{align}

\newpage

%\tableofcontents

\bigskip

\renewcommand{\thefigure}{\theenumi}
\renewcommand{\thetable}{\theenumi}
%\renewcommand{\theequation}{\theenumi}

%\begin{abstract}
%%\boldmath
%In this letter, an algorithm for evaluating the exact analytical bit error rate  (BER)  for the piecewise linear (PL) combiner for  multiple relays is presented. Previous results were available only for upto three relays. The algorithm is unique in the sense that  the actual mathematical expressions, that are prohibitively large, need not be explicitly obtained. The diversity gain due to multiple relays is shown through plots of the analytical BER, well supported by simulations. 
%
%\end{abstract}
% IEEEtran.cls defaults to using nonbold math in the Abstract.
% This preserves the distinction between vectors and scalars. However,
% if the journal you are submitting to favors bold math in the abstract,
% then you can use LaTeX's standard command \boldmath at the very start
% of the abstract to achieve this. Many IEEE journals frown on math
% in the abstract anyway.

% Note that keywords are not normally used for peerreview papers.
%\begin{IEEEkeywords}
%Cooperative diversity, decode and forward, piecewise linear
%\end{IEEEkeywords}



% For peer review papers, you can put extra information on the cover
% page as needed:
% \ifCLASSOPTIONpeerreview
% \begin{center} \bfseries EDICS Category: 3-BBND \end{center}
% \fi
%
% For peerreview papers, this IEEEtran command inserts a page break and
% creates the second title. It will be ignored for other modes.
%\IEEEpeerreviewmaketitle




\item A card is selected from a pack of 52 cards.
\begin{enumerate}[label=(\alph*)]
    \item How many points are there in the sample space?
    \item Calculate the probability that the card is an ace of spades.
    \item Calculate the probability that the card is (i) an ace and (ii) black card.
\end{enumerate}
\solution
%Let $X$ be an bernoulli rv defined as in \tabref{tab:exemplar/11/16/3/26}.  Then, 
\begin{equation}
    p =
        \frac{4}{11} 
\end{equation}
\begin{table}[H]
	\centering
	\input{exemplar/11/16/3/26/tables/Table2.tex}
	\caption{}
        \label{tab:exemplar/11/16/3/26}
\end{table}

\item The probability that a non leap year selected at random will contain 53 sundays.
\\
\solution
%\begin{table}[H]
	\centering
\begin{tabular}{|c|c|c|}
\hline
Random variable &Value &Definition\\ \hline
\multirow{3}{*}{X} &0 &Slips of Rs 1\\
&1 &Slips of Rs 5\\
&2 &Slips of Rs 13\\ \hline
\multirow{2}{*}{Y} &0 &Box A\\
&1 &Box B\\\hline
\end{tabular}
\caption{}
\label{tab:Distribution}
\end{table}
See \tabref{tab:Distribution}.
\begin{align}
p_{Y}\brak{k}= \begin{cases} 
      \frac{1}{3} & {k=0} \\
      \frac{2}{3 }& {k=1} 
   \end{cases}
   \\
p_{Y|X}\brak{0|0} = \frac{19}{25}\, 
p_{Y|X}\brak{0|1} = \frac{6}{25}\,
p_{Y|X}\brak{1|0} = \frac{45}{50}\,
p_{Y|X}\brak{1|2} = \frac{5}{50}
\end{align}
The desired probability is the probability that a slip drawn at random is marked other than Rs 1,
\begin{align}
&=1-p_X\brak{0}\\
&= p_X(1) + p_X(2)
\end{align}
Using Bayes theorem,
\begin{align}
&= p_Y\brak{0} \times \pr{Y=0 | X=1} + p_Y\brak{1} \times \pr{Y=1|X=2}\\
&=\frac{1}{3} \times \frac{6}{25} + \frac{2}{3} \times \frac{5}{50}\\
&=\frac{11}{75}
\end{align}

\newpage

%\tableofcontents

\bigskip

\renewcommand{\thefigure}{\theenumi}
\renewcommand{\thetable}{\theenumi}
%\renewcommand{\theequation}{\theenumi}

%\begin{abstract}
%%\boldmath
%In this letter, an algorithm for evaluating the exact analytical bit error rate  (BER)  for the piecewise linear (PL) combiner for  multiple relays is presented. Previous results were available only for upto three relays. The algorithm is unique in the sense that  the actual mathematical expressions, that are prohibitively large, need not be explicitly obtained. The diversity gain due to multiple relays is shown through plots of the analytical BER, well supported by simulations. 
%
%\end{abstract}
% IEEEtran.cls defaults to using nonbold math in the Abstract.
% This preserves the distinction between vectors and scalars. However,
% if the journal you are submitting to favors bold math in the abstract,
% then you can use LaTeX's standard command \boldmath at the very start
% of the abstract to achieve this. Many IEEE journals frown on math
% in the abstract anyway.

% Note that keywords are not normally used for peerreview papers.
%\begin{IEEEkeywords}
%Cooperative diversity, decode and forward, piecewise linear
%\end{IEEEkeywords}



% For peer review papers, you can put extra information on the cover
% page as needed:
% \ifCLASSOPTIONpeerreview
% \begin{center} \bfseries EDICS Category: 3-BBND \end{center}
% \fi
%
% For peerreview papers, this IEEEtran command inserts a page break and
% creates the second title. It will be ignored for other modes.
%\IEEEpeerreviewmaketitle




\item One of the four persons John, Rita, Aslam or Gurpreet will be promoted next
month. Consequently the sample space consists of four elementary outcomes
S = {John promoted, Rita promoted, Aslam promoted, Gurpreet promoted}
You are told that the chances of John’s promotion is same as that of Gurpreet,
Rita’s chances of promotion are twice as likely as Johns. Aslam’s chances are
four times that of John.
\begin{enumerate}
	\item Determine
	\begin{enumerate}
		\item P (John promoted)
		\item P (Rita promoted)
		\item P (Aslam promoted)
		\item P (Gurpreet promoted)
	\end{enumerate}
	\item If A = {John promoted or Gurpreet promoted}, find P (A).
\end{enumerate}
\solution
%\begin{table}[H]
	\centering
\begin{tabular}{|c|c|c|}
\hline
Random variable &Value &Definition\\ \hline
\multirow{3}{*}{X} &0 &Slips of Rs 1\\
&1 &Slips of Rs 5\\
&2 &Slips of Rs 13\\ \hline
\multirow{2}{*}{Y} &0 &Box A\\
&1 &Box B\\\hline
\end{tabular}
\caption{}
\label{tab:Distribution}
\end{table}
See \tabref{tab:Distribution}.
\begin{align}
p_{Y}\brak{k}= \begin{cases} 
      \frac{1}{3} & {k=0} \\
      \frac{2}{3 }& {k=1} 
   \end{cases}
   \\
p_{Y|X}\brak{0|0} = \frac{19}{25}\, 
p_{Y|X}\brak{0|1} = \frac{6}{25}\,
p_{Y|X}\brak{1|0} = \frac{45}{50}\,
p_{Y|X}\brak{1|2} = \frac{5}{50}
\end{align}
The desired probability is the probability that a slip drawn at random is marked other than Rs 1,
\begin{align}
&=1-p_X\brak{0}\\
&= p_X(1) + p_X(2)
\end{align}
Using Bayes theorem,
\begin{align}
&= p_Y\brak{0} \times \pr{Y=0 | X=1} + p_Y\brak{1} \times \pr{Y=1|X=2}\\
&=\frac{1}{3} \times \frac{6}{25} + \frac{2}{3} \times \frac{5}{50}\\
&=\frac{11}{75}
\end{align}

\newpage

%\tableofcontents

\bigskip

\renewcommand{\thefigure}{\theenumi}
\renewcommand{\thetable}{\theenumi}
%\renewcommand{\theequation}{\theenumi}

%\begin{abstract}
%%\boldmath
%In this letter, an algorithm for evaluating the exact analytical bit error rate  (BER)  for the piecewise linear (PL) combiner for  multiple relays is presented. Previous results were available only for upto three relays. The algorithm is unique in the sense that  the actual mathematical expressions, that are prohibitively large, need not be explicitly obtained. The diversity gain due to multiple relays is shown through plots of the analytical BER, well supported by simulations. 
%
%\end{abstract}
% IEEEtran.cls defaults to using nonbold math in the Abstract.
% This preserves the distinction between vectors and scalars. However,
% if the journal you are submitting to favors bold math in the abstract,
% then you can use LaTeX's standard command \boldmath at the very start
% of the abstract to achieve this. Many IEEE journals frown on math
% in the abstract anyway.

% Note that keywords are not normally used for peerreview papers.
%\begin{IEEEkeywords}
%Cooperative diversity, decode and forward, piecewise linear
%\end{IEEEkeywords}



% For peer review papers, you can put extra information on the cover
% page as needed:
% \ifCLASSOPTIONpeerreview
% \begin{center} \bfseries EDICS Category: 3-BBND \end{center}
% \fi
%
% For peerreview papers, this IEEEtran command inserts a page break and
% creates the second title. It will be ignored for other modes.
%\IEEEpeerreviewmaketitle




\item A card is drawn from a deck of 52 cards. Find the probability of getting a king or a heart or a red card.\\
\solution
%\begin{table}[H]
	\centering
\begin{tabular}{|c|c|c|}
\hline
Random variable &Value &Definition\\ \hline
\multirow{3}{*}{X} &0 &Slips of Rs 1\\
&1 &Slips of Rs 5\\
&2 &Slips of Rs 13\\ \hline
\multirow{2}{*}{Y} &0 &Box A\\
&1 &Box B\\\hline
\end{tabular}
\caption{}
\label{tab:Distribution}
\end{table}
See \tabref{tab:Distribution}.
\begin{align}
p_{Y}\brak{k}= \begin{cases} 
      \frac{1}{3} & {k=0} \\
      \frac{2}{3 }& {k=1} 
   \end{cases}
   \\
p_{Y|X}\brak{0|0} = \frac{19}{25}\, 
p_{Y|X}\brak{0|1} = \frac{6}{25}\,
p_{Y|X}\brak{1|0} = \frac{45}{50}\,
p_{Y|X}\brak{1|2} = \frac{5}{50}
\end{align}
The desired probability is the probability that a slip drawn at random is marked other than Rs 1,
\begin{align}
&=1-p_X\brak{0}\\
&= p_X(1) + p_X(2)
\end{align}
Using Bayes theorem,
\begin{align}
&= p_Y\brak{0} \times \pr{Y=0 | X=1} + p_Y\brak{1} \times \pr{Y=1|X=2}\\
&=\frac{1}{3} \times \frac{6}{25} + \frac{2}{3} \times \frac{5}{50}\\
&=\frac{11}{75}
\end{align}

\newpage

%\tableofcontents

\bigskip

\renewcommand{\thefigure}{\theenumi}
\renewcommand{\thetable}{\theenumi}
%\renewcommand{\theequation}{\theenumi}

%\begin{abstract}
%%\boldmath
%In this letter, an algorithm for evaluating the exact analytical bit error rate  (BER)  for the piecewise linear (PL) combiner for  multiple relays is presented. Previous results were available only for upto three relays. The algorithm is unique in the sense that  the actual mathematical expressions, that are prohibitively large, need not be explicitly obtained. The diversity gain due to multiple relays is shown through plots of the analytical BER, well supported by simulations. 
%
%\end{abstract}
% IEEEtran.cls defaults to using nonbold math in the Abstract.
% This preserves the distinction between vectors and scalars. However,
% if the journal you are submitting to favors bold math in the abstract,
% then you can use LaTeX's standard command \boldmath at the very start
% of the abstract to achieve this. Many IEEE journals frown on math
% in the abstract anyway.

% Note that keywords are not normally used for peerreview papers.
%\begin{IEEEkeywords}
%Cooperative diversity, decode and forward, piecewise linear
%\end{IEEEkeywords}



% For peer review papers, you can put extra information on the cover
% page as needed:
% \ifCLASSOPTIONpeerreview
% \begin{center} \bfseries EDICS Category: 3-BBND \end{center}
% \fi
%
% For peerreview papers, this IEEEtran command inserts a page break and
% creates the second title. It will be ignored for other modes.
%\IEEEpeerreviewmaketitle




\item The probability that a student will pass his examination is 0.73, the probability of
the student getting a compartment is 0.13, and the probability that the student will
either pass or get compartment is 0.96. State True or False.\\
\solution
%\begin{table}[H]
	\centering
\begin{tabular}{|c|c|c|}
\hline
Random variable &Value &Definition\\ \hline
\multirow{3}{*}{X} &0 &Slips of Rs 1\\
&1 &Slips of Rs 5\\
&2 &Slips of Rs 13\\ \hline
\multirow{2}{*}{Y} &0 &Box A\\
&1 &Box B\\\hline
\end{tabular}
\caption{}
\label{tab:Distribution}
\end{table}
See \tabref{tab:Distribution}.
\begin{align}
p_{Y}\brak{k}= \begin{cases} 
      \frac{1}{3} & {k=0} \\
      \frac{2}{3 }& {k=1} 
   \end{cases}
   \\
p_{Y|X}\brak{0|0} = \frac{19}{25}\, 
p_{Y|X}\brak{0|1} = \frac{6}{25}\,
p_{Y|X}\brak{1|0} = \frac{45}{50}\,
p_{Y|X}\brak{1|2} = \frac{5}{50}
\end{align}
The desired probability is the probability that a slip drawn at random is marked other than Rs 1,
\begin{align}
&=1-p_X\brak{0}\\
&= p_X(1) + p_X(2)
\end{align}
Using Bayes theorem,
\begin{align}
&= p_Y\brak{0} \times \pr{Y=0 | X=1} + p_Y\brak{1} \times \pr{Y=1|X=2}\\
&=\frac{1}{3} \times \frac{6}{25} + \frac{2}{3} \times \frac{5}{50}\\
&=\frac{11}{75}
\end{align}

\newpage

%\tableofcontents

\bigskip

\renewcommand{\thefigure}{\theenumi}
\renewcommand{\thetable}{\theenumi}
%\renewcommand{\theequation}{\theenumi}

%\begin{abstract}
%%\boldmath
%In this letter, an algorithm for evaluating the exact analytical bit error rate  (BER)  for the piecewise linear (PL) combiner for  multiple relays is presented. Previous results were available only for upto three relays. The algorithm is unique in the sense that  the actual mathematical expressions, that are prohibitively large, need not be explicitly obtained. The diversity gain due to multiple relays is shown through plots of the analytical BER, well supported by simulations. 
%
%\end{abstract}
% IEEEtran.cls defaults to using nonbold math in the Abstract.
% This preserves the distinction between vectors and scalars. However,
% if the journal you are submitting to favors bold math in the abstract,
% then you can use LaTeX's standard command \boldmath at the very start
% of the abstract to achieve this. Many IEEE journals frown on math
% in the abstract anyway.

% Note that keywords are not normally used for peerreview papers.
%\begin{IEEEkeywords}
%Cooperative diversity, decode and forward, piecewise linear
%\end{IEEEkeywords}



% For peer review papers, you can put extra information on the cover
% page as needed:
% \ifCLASSOPTIONpeerreview
% \begin{center} \bfseries EDICS Category: 3-BBND \end{center}
% \fi
%
% For peerreview papers, this IEEEtran command inserts a page break and
% creates the second title. It will be ignored for other modes.
%\IEEEpeerreviewmaketitle




\item A card is selected from a pack of 52 cards\\
\begin{enumerate}[label=(\alph*)]
\item How many points are there in the sample space?
\item Calculate the probability that the cards is an ace of spades.
\item Calculate the probability that the card is (i) an ace (ii)black card.\\
\end{enumerate}
%\input{ncert/11/16/3/4_1/Prob_4.tex}
\item In a non-leap year, the probability of having 53 tuesdays or 53 wednesdays is\\
\solution
%A non-leap year has a total of 365 days, and a week has 7 days.\\
So it can be expressed as 
\begin{align}
365\text{days} &=52\times 7+1 \text{day}
\end{align}
$\implies$ 52 tuesdays or wednesdays\\
Random variable X denotes the days of a week
\begin{align}
p_X\brak{k}&=\frac{1}{7}; \quad \brak{1<k<7}
\end{align}
So the probability of extra day being tuesday or wednesday is
\begin{align}
p_X\brak{3}+p_X\brak{4}&=\frac{1}{7}+\frac{1}{7}=\frac{2}{7}
\end{align}



\item There are 1000 sealed envelopes in a box, 10 of them contain a cash prize of
Rs 100 each, 100 of them contain a cash prize of Rs 50 each and 200 of them
contain a cash prize of Rs 10 each and rest do not contain any cash prize. If they
are well shuffled and an envelope is picked up out, what is the probability that it
contains no cash prize?\\
\solution
%\begin{table}[H]
	\centering
\begin{tabular}{|c|c|c|}
\hline
Random variable &Value &Definition\\ \hline
\multirow{3}{*}{X} &0 &Slips of Rs 1\\
&1 &Slips of Rs 5\\
&2 &Slips of Rs 13\\ \hline
\multirow{2}{*}{Y} &0 &Box A\\
&1 &Box B\\\hline
\end{tabular}
\caption{}
\label{tab:Distribution}
\end{table}
See \tabref{tab:Distribution}.
\begin{align}
p_{Y}\brak{k}= \begin{cases} 
      \frac{1}{3} & {k=0} \\
      \frac{2}{3 }& {k=1} 
   \end{cases}
   \\
p_{Y|X}\brak{0|0} = \frac{19}{25}\, 
p_{Y|X}\brak{0|1} = \frac{6}{25}\,
p_{Y|X}\brak{1|0} = \frac{45}{50}\,
p_{Y|X}\brak{1|2} = \frac{5}{50}
\end{align}
The desired probability is the probability that a slip drawn at random is marked other than Rs 1,
\begin{align}
&=1-p_X\brak{0}\\
&= p_X(1) + p_X(2)
\end{align}
Using Bayes theorem,
\begin{align}
&= p_Y\brak{0} \times \pr{Y=0 | X=1} + p_Y\brak{1} \times \pr{Y=1|X=2}\\
&=\frac{1}{3} \times \frac{6}{25} + \frac{2}{3} \times \frac{5}{50}\\
&=\frac{11}{75}
\end{align}

\newpage

%\tableofcontents

\bigskip

\renewcommand{\thefigure}{\theenumi}
\renewcommand{\thetable}{\theenumi}
%\renewcommand{\theequation}{\theenumi}

%\begin{abstract}
%%\boldmath
%In this letter, an algorithm for evaluating the exact analytical bit error rate  (BER)  for the piecewise linear (PL) combiner for  multiple relays is presented. Previous results were available only for upto three relays. The algorithm is unique in the sense that  the actual mathematical expressions, that are prohibitively large, need not be explicitly obtained. The diversity gain due to multiple relays is shown through plots of the analytical BER, well supported by simulations. 
%
%\end{abstract}
% IEEEtran.cls defaults to using nonbold math in the Abstract.
% This preserves the distinction between vectors and scalars. However,
% if the journal you are submitting to favors bold math in the abstract,
% then you can use LaTeX's standard command \boldmath at the very start
% of the abstract to achieve this. Many IEEE journals frown on math
% in the abstract anyway.

% Note that keywords are not normally used for peerreview papers.
%\begin{IEEEkeywords}
%Cooperative diversity, decode and forward, piecewise linear
%\end{IEEEkeywords}



% For peer review papers, you can put extra information on the cover
% page as needed:
% \ifCLASSOPTIONpeerreview
% \begin{center} \bfseries EDICS Category: 3-BBND \end{center}
% \fi
%
% For peerreview papers, this IEEEtran command inserts a page break and
% creates the second title. It will be ignored for other modes.
%\IEEEpeerreviewmaketitle




\item 
A die is thrown and a card is selected at random from a deck of 52 playing cards. The probability of getting an even number on the die and a spade card.\\
\solution
%\begin{table}[H]
	\centering
\begin{tabular}{|c|c|c|}
\hline
Random variable &Value &Definition\\ \hline
\multirow{3}{*}{X} &0 &Slips of Rs 1\\
&1 &Slips of Rs 5\\
&2 &Slips of Rs 13\\ \hline
\multirow{2}{*}{Y} &0 &Box A\\
&1 &Box B\\\hline
\end{tabular}
\caption{}
\label{tab:Distribution}
\end{table}
See \tabref{tab:Distribution}.
\begin{align}
p_{Y}\brak{k}= \begin{cases} 
      \frac{1}{3} & {k=0} \\
      \frac{2}{3 }& {k=1} 
   \end{cases}
   \\
p_{Y|X}\brak{0|0} = \frac{19}{25}\, 
p_{Y|X}\brak{0|1} = \frac{6}{25}\,
p_{Y|X}\brak{1|0} = \frac{45}{50}\,
p_{Y|X}\brak{1|2} = \frac{5}{50}
\end{align}
The desired probability is the probability that a slip drawn at random is marked other than Rs 1,
\begin{align}
&=1-p_X\brak{0}\\
&= p_X(1) + p_X(2)
\end{align}
Using Bayes theorem,
\begin{align}
&= p_Y\brak{0} \times \pr{Y=0 | X=1} + p_Y\brak{1} \times \pr{Y=1|X=2}\\
&=\frac{1}{3} \times \frac{6}{25} + \frac{2}{3} \times \frac{5}{50}\\
&=\frac{11}{75}
\end{align}

\newpage

%\tableofcontents

\bigskip

\renewcommand{\thefigure}{\theenumi}
\renewcommand{\thetable}{\theenumi}
%\renewcommand{\theequation}{\theenumi}

%\begin{abstract}
%%\boldmath
%In this letter, an algorithm for evaluating the exact analytical bit error rate  (BER)  for the piecewise linear (PL) combiner for  multiple relays is presented. Previous results were available only for upto three relays. The algorithm is unique in the sense that  the actual mathematical expressions, that are prohibitively large, need not be explicitly obtained. The diversity gain due to multiple relays is shown through plots of the analytical BER, well supported by simulations. 
%
%\end{abstract}
% IEEEtran.cls defaults to using nonbold math in the Abstract.
% This preserves the distinction between vectors and scalars. However,
% if the journal you are submitting to favors bold math in the abstract,
% then you can use LaTeX's standard command \boldmath at the very start
% of the abstract to achieve this. Many IEEE journals frown on math
% in the abstract anyway.

% Note that keywords are not normally used for peerreview papers.
%\begin{IEEEkeywords}
%Cooperative diversity, decode and forward, piecewise linear
%\end{IEEEkeywords}



% For peer review papers, you can put extra information on the cover
% page as needed:
% \ifCLASSOPTIONpeerreview
% \begin{center} \bfseries EDICS Category: 3-BBND \end{center}
% \fi
%
% For peerreview papers, this IEEEtran command inserts a page break and
% creates the second title. It will be ignored for other modes.
%\IEEEpeerreviewmaketitle




\item
If 4-digit numbers greater than 5,000 are randomly formed from the digits 0, 1, 3, 5, and 7, what is the probability of forming a number divisible by 5 when:
\begin{enumerate}
    \item The digits are repeated?
    \item The repetition of digits is not allowed?
\end{enumerate}
\solution
%\begin{table}[H]
	\centering
\begin{tabular}{|c|c|c|}
\hline
Random variable &Value &Definition\\ \hline
\multirow{3}{*}{X} &0 &Slips of Rs 1\\
&1 &Slips of Rs 5\\
&2 &Slips of Rs 13\\ \hline
\multirow{2}{*}{Y} &0 &Box A\\
&1 &Box B\\\hline
\end{tabular}
\caption{}
\label{tab:Distribution}
\end{table}
See \tabref{tab:Distribution}.
\begin{align}
p_{Y}\brak{k}= \begin{cases} 
      \frac{1}{3} & {k=0} \\
      \frac{2}{3 }& {k=1} 
   \end{cases}
   \\
p_{Y|X}\brak{0|0} = \frac{19}{25}\, 
p_{Y|X}\brak{0|1} = \frac{6}{25}\,
p_{Y|X}\brak{1|0} = \frac{45}{50}\,
p_{Y|X}\brak{1|2} = \frac{5}{50}
\end{align}
The desired probability is the probability that a slip drawn at random is marked other than Rs 1,
\begin{align}
&=1-p_X\brak{0}\\
&= p_X(1) + p_X(2)
\end{align}
Using Bayes theorem,
\begin{align}
&= p_Y\brak{0} \times \pr{Y=0 | X=1} + p_Y\brak{1} \times \pr{Y=1|X=2}\\
&=\frac{1}{3} \times \frac{6}{25} + \frac{2}{3} \times \frac{5}{50}\\
&=\frac{11}{75}
\end{align}

\newpage

%\tableofcontents

\bigskip

\renewcommand{\thefigure}{\theenumi}
\renewcommand{\thetable}{\theenumi}
%\renewcommand{\theequation}{\theenumi}

%\begin{abstract}
%%\boldmath
%In this letter, an algorithm for evaluating the exact analytical bit error rate  (BER)  for the piecewise linear (PL) combiner for  multiple relays is presented. Previous results were available only for upto three relays. The algorithm is unique in the sense that  the actual mathematical expressions, that are prohibitively large, need not be explicitly obtained. The diversity gain due to multiple relays is shown through plots of the analytical BER, well supported by simulations. 
%
%\end{abstract}
% IEEEtran.cls defaults to using nonbold math in the Abstract.
% This preserves the distinction between vectors and scalars. However,
% if the journal you are submitting to favors bold math in the abstract,
% then you can use LaTeX's standard command \boldmath at the very start
% of the abstract to achieve this. Many IEEE journals frown on math
% in the abstract anyway.

% Note that keywords are not normally used for peerreview papers.
%\begin{IEEEkeywords}
%Cooperative diversity, decode and forward, piecewise linear
%\end{IEEEkeywords}



% For peer review papers, you can put extra information on the cover
% page as needed:
% \ifCLASSOPTIONpeerreview
% \begin{center} \bfseries EDICS Category: 3-BBND \end{center}
% \fi
%
% For peerreview papers, this IEEEtran command inserts a page break and
% creates the second title. It will be ignored for other modes.
%\IEEEpeerreviewmaketitle




\item Consider the probability space $\brak{\Omega, \mathcal{G}, P}$ where $\Omega = [0,2]$ and $\mathcal{G} = \cbrak{\phi, \Omega, [0,1], (1,2]}$. Let $X$ and $Y$ be two functions on $\Omega$ defined as
\begin{align*}
    X(\omega) = 
    \begin{cases}
        1 & \text{if }\omega \in [0, 1]\\
        2 & \text{if }\omega \in (1, 2]
    \end{cases}
\end{align*}
and
\begin{align*}
    Y(\omega) = 
    \begin{cases}
        2 & \text{if }\omega \in [0, 1.5]\\
        3 & \text{if }\omega \in (1.5, 2].
    \end{cases}
\end{align*}
Then which one of the following statements is true?
\begin{enumerate}
    \item [(A)] $X$ is a random variable with respect to $\mathcal{G}$, but $Y$ is not a random variable with respect to $\mathcal{G}$.
    \item [(B)] $Y$ is a random variable with respect to $\mathcal{G}$, but $X$ is not a random variable with respect to $\mathcal{G}$.
    \item [(C)] Neither $X$ nor $Y$ is a random variable with respect to $\mathcal{G}$.
    \item [(D)] Both $X$ and $Y$ are random variables with respect to $\mathcal{G}$.
\end{enumerate} \hfill (GATE ST 2023)\\
\solution
%\begin{table}[H]
	\centering
\begin{tabular}{|c|c|c|}
\hline
Random variable &Value &Definition\\ \hline
\multirow{3}{*}{X} &0 &Slips of Rs 1\\
&1 &Slips of Rs 5\\
&2 &Slips of Rs 13\\ \hline
\multirow{2}{*}{Y} &0 &Box A\\
&1 &Box B\\\hline
\end{tabular}
\caption{}
\label{tab:Distribution}
\end{table}
See \tabref{tab:Distribution}.
\begin{align}
p_{Y}\brak{k}= \begin{cases} 
      \frac{1}{3} & {k=0} \\
      \frac{2}{3 }& {k=1} 
   \end{cases}
   \\
p_{Y|X}\brak{0|0} = \frac{19}{25}\, 
p_{Y|X}\brak{0|1} = \frac{6}{25}\,
p_{Y|X}\brak{1|0} = \frac{45}{50}\,
p_{Y|X}\brak{1|2} = \frac{5}{50}
\end{align}
The desired probability is the probability that a slip drawn at random is marked other than Rs 1,
\begin{align}
&=1-p_X\brak{0}\\
&= p_X(1) + p_X(2)
\end{align}
Using Bayes theorem,
\begin{align}
&= p_Y\brak{0} \times \pr{Y=0 | X=1} + p_Y\brak{1} \times \pr{Y=1|X=2}\\
&=\frac{1}{3} \times \frac{6}{25} + \frac{2}{3} \times \frac{5}{50}\\
&=\frac{11}{75}
\end{align}

\newpage

%\tableofcontents

\bigskip

\renewcommand{\thefigure}{\theenumi}
\renewcommand{\thetable}{\theenumi}
%\renewcommand{\theequation}{\theenumi}

%\begin{abstract}
%%\boldmath
%In this letter, an algorithm for evaluating the exact analytical bit error rate  (BER)  for the piecewise linear (PL) combiner for  multiple relays is presented. Previous results were available only for upto three relays. The algorithm is unique in the sense that  the actual mathematical expressions, that are prohibitively large, need not be explicitly obtained. The diversity gain due to multiple relays is shown through plots of the analytical BER, well supported by simulations. 
%
%\end{abstract}
% IEEEtran.cls defaults to using nonbold math in the Abstract.
% This preserves the distinction between vectors and scalars. However,
% if the journal you are submitting to favors bold math in the abstract,
% then you can use LaTeX's standard command \boldmath at the very start
% of the abstract to achieve this. Many IEEE journals frown on math
% in the abstract anyway.

% Note that keywords are not normally used for peerreview papers.
%\begin{IEEEkeywords}
%Cooperative diversity, decode and forward, piecewise linear
%\end{IEEEkeywords}



% For peer review papers, you can put extra information on the cover
% page as needed:
% \ifCLASSOPTIONpeerreview
% \begin{center} \bfseries EDICS Category: 3-BBND \end{center}
% \fi
%
% For peerreview papers, this IEEEtran command inserts a page break and
% creates the second title. It will be ignored for other modes.
%\IEEEpeerreviewmaketitle




	\item  A die is loaded in such a way that each odd number is twice as likely to occur as
each even number. Find $P(G)$, where $G$ is the event that a number greater than
3 occurs on a single roll of the die.
\\
\solution
		%\begin{table}[H]
	\centering
\begin{tabular}{|c|c|c|}
\hline
Random variable &Value &Definition\\ \hline
\multirow{3}{*}{X} &0 &Slips of Rs 1\\
&1 &Slips of Rs 5\\
&2 &Slips of Rs 13\\ \hline
\multirow{2}{*}{Y} &0 &Box A\\
&1 &Box B\\\hline
\end{tabular}
\caption{}
\label{tab:Distribution}
\end{table}
See \tabref{tab:Distribution}.
\begin{align}
p_{Y}\brak{k}= \begin{cases} 
      \frac{1}{3} & {k=0} \\
      \frac{2}{3 }& {k=1} 
   \end{cases}
   \\
p_{Y|X}\brak{0|0} = \frac{19}{25}\, 
p_{Y|X}\brak{0|1} = \frac{6}{25}\,
p_{Y|X}\brak{1|0} = \frac{45}{50}\,
p_{Y|X}\brak{1|2} = \frac{5}{50}
\end{align}
The desired probability is the probability that a slip drawn at random is marked other than Rs 1,
\begin{align}
&=1-p_X\brak{0}\\
&= p_X(1) + p_X(2)
\end{align}
Using Bayes theorem,
\begin{align}
&= p_Y\brak{0} \times \pr{Y=0 | X=1} + p_Y\brak{1} \times \pr{Y=1|X=2}\\
&=\frac{1}{3} \times \frac{6}{25} + \frac{2}{3} \times \frac{5}{50}\\
&=\frac{11}{75}
\end{align}

\newpage

%\tableofcontents

\bigskip

\renewcommand{\thefigure}{\theenumi}
\renewcommand{\thetable}{\theenumi}
%\renewcommand{\theequation}{\theenumi}

%\begin{abstract}
%%\boldmath
%In this letter, an algorithm for evaluating the exact analytical bit error rate  (BER)  for the piecewise linear (PL) combiner for  multiple relays is presented. Previous results were available only for upto three relays. The algorithm is unique in the sense that  the actual mathematical expressions, that are prohibitively large, need not be explicitly obtained. The diversity gain due to multiple relays is shown through plots of the analytical BER, well supported by simulations. 
%
%\end{abstract}
% IEEEtran.cls defaults to using nonbold math in the Abstract.
% This preserves the distinction between vectors and scalars. However,
% if the journal you are submitting to favors bold math in the abstract,
% then you can use LaTeX's standard command \boldmath at the very start
% of the abstract to achieve this. Many IEEE journals frown on math
% in the abstract anyway.

% Note that keywords are not normally used for peerreview papers.
%\begin{IEEEkeywords}
%Cooperative diversity, decode and forward, piecewise linear
%\end{IEEEkeywords}



% For peer review papers, you can put extra information on the cover
% page as needed:
% \ifCLASSOPTIONpeerreview
% \begin{center} \bfseries EDICS Category: 3-BBND \end{center}
% \fi
%
% For peerreview papers, this IEEEtran command inserts a page break and
% creates the second title. It will be ignored for other modes.
%\IEEEpeerreviewmaketitle




	\item All the jacks, queens and kings are removed from a deck of 52 playing cards. The remaining cards are well shuffled and then one card is drawn at random. Giving ace a value 1 similar value for other cards, find the probability that the card has a value 
		\begin{enumerate}
			\item 7
			\item greater than 7
			\item less than 7
		\end{enumerate}
		%Number of cards left after removing all jacks, queens and kings 
\begin{align}
N	= 52 - 4\times 3
	= 40
\end{align}
%\begin{table}[H]
%\def\arraystretch{1.2}
%\begin{tabular}{|c|c|c|}
%\hline
%	\textbf{Parameter} &\textbf{Value} &\textbf{Description}\\ \hline
%	$X$ &1-10 &Represents the value of the card picked \\ \hline
%\end{tabular}
%\end{table}
Let $1 \le X \le 10$ be the value of the card picked.  Then,
\begin{align}
	p_X(k) &= \Pr(X=k)\ \forall\ 1 \leq k \leq 10\\
	&= \frac{4\times 1}{40}\\
	&= \frac{1}{10}\\
	\therefore p_X(k) &= 
	\begin{cases}
		\frac{1}{10} & 1 \leq k \leq 10\\
		0 & \text{otherwise}
	\end{cases}
\end{align}
and
\begin{align}
	F_{X}(k) &= \sum_{m=0}^{k}p_{X}(m) \quad 1 \leq k \leq 10\\
	&= \frac{k}{10}\\
	\therefore F_{X}(k) &= 
	\begin{cases}
		0 & k \leq 0\\
		\frac{k}{10} & 1\leq k \leq 10\\
		1 & k > 10 
	\end{cases}
\end{align}
\begin{enumerate}
	\item Probability that card has value equal to 7 is
		\begin{align}
			 p_{X}(7)
			= \frac{1}{10}
		\end{align}
	\item Probability that card has value greater than 7 is
		\begin{align}
			1 - F_X(7)
			&= 1 - \frac{7}{10}
			\\
			&= \frac{3}{10}
		\end{align}
	\item Probability that card has value less than 7 is
		\begin{align}
			 F_{X}(6)
			=\frac{6}{10}
		\end{align}
\end{enumerate}

  \item A Lot consists of 48 mobile phones of which 42 are good, 3 have only minor defects and 3 have major defects.Varnika will buy a phone if it is good but the trader will only buy a mobile if it has no major defects. One phone is selected at random from the lot. What is the probability that it is
\begin{enumerate}
	\item acceptable to Varnika?
            \item acceptable to the trader?
\end{enumerate}
\solution
	%\begin{table}[H]
	\centering
\begin{tabular}{|c|c|c|}
\hline
Random variable &Value &Definition\\ \hline
\multirow{3}{*}{X} &0 &Slips of Rs 1\\
&1 &Slips of Rs 5\\
&2 &Slips of Rs 13\\ \hline
\multirow{2}{*}{Y} &0 &Box A\\
&1 &Box B\\\hline
\end{tabular}
\caption{}
\label{tab:Distribution}
\end{table}
See \tabref{tab:Distribution}.
\begin{align}
p_{Y}\brak{k}= \begin{cases} 
      \frac{1}{3} & {k=0} \\
      \frac{2}{3 }& {k=1} 
   \end{cases}
   \\
p_{Y|X}\brak{0|0} = \frac{19}{25}\, 
p_{Y|X}\brak{0|1} = \frac{6}{25}\,
p_{Y|X}\brak{1|0} = \frac{45}{50}\,
p_{Y|X}\brak{1|2} = \frac{5}{50}
\end{align}
The desired probability is the probability that a slip drawn at random is marked other than Rs 1,
\begin{align}
&=1-p_X\brak{0}\\
&= p_X(1) + p_X(2)
\end{align}
Using Bayes theorem,
\begin{align}
&= p_Y\brak{0} \times \pr{Y=0 | X=1} + p_Y\brak{1} \times \pr{Y=1|X=2}\\
&=\frac{1}{3} \times \frac{6}{25} + \frac{2}{3} \times \frac{5}{50}\\
&=\frac{11}{75}
\end{align}

\newpage

%\tableofcontents

\bigskip

\renewcommand{\thefigure}{\theenumi}
\renewcommand{\thetable}{\theenumi}
%\renewcommand{\theequation}{\theenumi}

%\begin{abstract}
%%\boldmath
%In this letter, an algorithm for evaluating the exact analytical bit error rate  (BER)  for the piecewise linear (PL) combiner for  multiple relays is presented. Previous results were available only for upto three relays. The algorithm is unique in the sense that  the actual mathematical expressions, that are prohibitively large, need not be explicitly obtained. The diversity gain due to multiple relays is shown through plots of the analytical BER, well supported by simulations. 
%
%\end{abstract}
% IEEEtran.cls defaults to using nonbold math in the Abstract.
% This preserves the distinction between vectors and scalars. However,
% if the journal you are submitting to favors bold math in the abstract,
% then you can use LaTeX's standard command \boldmath at the very start
% of the abstract to achieve this. Many IEEE journals frown on math
% in the abstract anyway.

% Note that keywords are not normally used for peerreview papers.
%\begin{IEEEkeywords}
%Cooperative diversity, decode and forward, piecewise linear
%\end{IEEEkeywords}



% For peer review papers, you can put extra information on the cover
% page as needed:
% \ifCLASSOPTIONpeerreview
% \begin{center} \bfseries EDICS Category: 3-BBND \end{center}
% \fi
%
% For peerreview papers, this IEEEtran command inserts a page break and
% creates the second title. It will be ignored for other modes.
%\IEEEpeerreviewmaketitle




 \item A student says that if you throw a die, it will show up 1 or not 1. Therefore, the probability of getting 1 and the probability of getting 'not 1' each is equal to $\frac{1}{2}$. Is this correct? Give reasons.\\
 \solution
        %\begin{table}[H]
	\centering
\begin{tabular}{|c|c|c|}
\hline
Random variable &Value &Definition\\ \hline
\multirow{3}{*}{X} &0 &Slips of Rs 1\\
&1 &Slips of Rs 5\\
&2 &Slips of Rs 13\\ \hline
\multirow{2}{*}{Y} &0 &Box A\\
&1 &Box B\\\hline
\end{tabular}
\caption{}
\label{tab:Distribution}
\end{table}
See \tabref{tab:Distribution}.
\begin{align}
p_{Y}\brak{k}= \begin{cases} 
      \frac{1}{3} & {k=0} \\
      \frac{2}{3 }& {k=1} 
   \end{cases}
   \\
p_{Y|X}\brak{0|0} = \frac{19}{25}\, 
p_{Y|X}\brak{0|1} = \frac{6}{25}\,
p_{Y|X}\brak{1|0} = \frac{45}{50}\,
p_{Y|X}\brak{1|2} = \frac{5}{50}
\end{align}
The desired probability is the probability that a slip drawn at random is marked other than Rs 1,
\begin{align}
&=1-p_X\brak{0}\\
&= p_X(1) + p_X(2)
\end{align}
Using Bayes theorem,
\begin{align}
&= p_Y\brak{0} \times \pr{Y=0 | X=1} + p_Y\brak{1} \times \pr{Y=1|X=2}\\
&=\frac{1}{3} \times \frac{6}{25} + \frac{2}{3} \times \frac{5}{50}\\
&=\frac{11}{75}
\end{align}

\newpage

%\tableofcontents

\bigskip

\renewcommand{\thefigure}{\theenumi}
\renewcommand{\thetable}{\theenumi}
%\renewcommand{\theequation}{\theenumi}

%\begin{abstract}
%%\boldmath
%In this letter, an algorithm for evaluating the exact analytical bit error rate  (BER)  for the piecewise linear (PL) combiner for  multiple relays is presented. Previous results were available only for upto three relays. The algorithm is unique in the sense that  the actual mathematical expressions, that are prohibitively large, need not be explicitly obtained. The diversity gain due to multiple relays is shown through plots of the analytical BER, well supported by simulations. 
%
%\end{abstract}
% IEEEtran.cls defaults to using nonbold math in the Abstract.
% This preserves the distinction between vectors and scalars. However,
% if the journal you are submitting to favors bold math in the abstract,
% then you can use LaTeX's standard command \boldmath at the very start
% of the abstract to achieve this. Many IEEE journals frown on math
% in the abstract anyway.

% Note that keywords are not normally used for peerreview papers.
%\begin{IEEEkeywords}
%Cooperative diversity, decode and forward, piecewise linear
%\end{IEEEkeywords}



% For peer review papers, you can put extra information on the cover
% page as needed:
% \ifCLASSOPTIONpeerreview
% \begin{center} \bfseries EDICS Category: 3-BBND \end{center}
% \fi
%
% For peerreview papers, this IEEEtran command inserts a page break and
% creates the second title. It will be ignored for other modes.
%\IEEEpeerreviewmaketitle




   \item Four candidates A, B, C, D have ap-
plied for the assignment to coach a school cricket
team. If A is twice as likely to be selected as B, and
B and C are given about the same chance of being
selected, while C is twice as likely to be selected
as D, what are the probabilities that
\begin{enumerate}
\item C will be selected?
\item A will not be selected?
\end{enumerate}
	%\begin{table}[H]
	\centering
\begin{tabular}{|c|c|c|}
\hline
Random variable &Value &Definition\\ \hline
\multirow{3}{*}{X} &0 &Slips of Rs 1\\
&1 &Slips of Rs 5\\
&2 &Slips of Rs 13\\ \hline
\multirow{2}{*}{Y} &0 &Box A\\
&1 &Box B\\\hline
\end{tabular}
\caption{}
\label{tab:Distribution}
\end{table}
See \tabref{tab:Distribution}.
\begin{align}
p_{Y}\brak{k}= \begin{cases} 
      \frac{1}{3} & {k=0} \\
      \frac{2}{3 }& {k=1} 
   \end{cases}
   \\
p_{Y|X}\brak{0|0} = \frac{19}{25}\, 
p_{Y|X}\brak{0|1} = \frac{6}{25}\,
p_{Y|X}\brak{1|0} = \frac{45}{50}\,
p_{Y|X}\brak{1|2} = \frac{5}{50}
\end{align}
The desired probability is the probability that a slip drawn at random is marked other than Rs 1,
\begin{align}
&=1-p_X\brak{0}\\
&= p_X(1) + p_X(2)
\end{align}
Using Bayes theorem,
\begin{align}
&= p_Y\brak{0} \times \pr{Y=0 | X=1} + p_Y\brak{1} \times \pr{Y=1|X=2}\\
&=\frac{1}{3} \times \frac{6}{25} + \frac{2}{3} \times \frac{5}{50}\\
&=\frac{11}{75}
\end{align}

\newpage

%\tableofcontents

\bigskip

\renewcommand{\thefigure}{\theenumi}
\renewcommand{\thetable}{\theenumi}
%\renewcommand{\theequation}{\theenumi}

%\begin{abstract}
%%\boldmath
%In this letter, an algorithm for evaluating the exact analytical bit error rate  (BER)  for the piecewise linear (PL) combiner for  multiple relays is presented. Previous results were available only for upto three relays. The algorithm is unique in the sense that  the actual mathematical expressions, that are prohibitively large, need not be explicitly obtained. The diversity gain due to multiple relays is shown through plots of the analytical BER, well supported by simulations. 
%
%\end{abstract}
% IEEEtran.cls defaults to using nonbold math in the Abstract.
% This preserves the distinction between vectors and scalars. However,
% if the journal you are submitting to favors bold math in the abstract,
% then you can use LaTeX's standard command \boldmath at the very start
% of the abstract to achieve this. Many IEEE journals frown on math
% in the abstract anyway.

% Note that keywords are not normally used for peerreview papers.
%\begin{IEEEkeywords}
%Cooperative diversity, decode and forward, piecewise linear
%\end{IEEEkeywords}



% For peer review papers, you can put extra information on the cover
% page as needed:
% \ifCLASSOPTIONpeerreview
% \begin{center} \bfseries EDICS Category: 3-BBND \end{center}
% \fi
%
% For peerreview papers, this IEEEtran command inserts a page break and
% creates the second title. It will be ignored for other modes.
%\IEEEpeerreviewmaketitle




 \item A bag contain 24 balls of which $x$ balls are red, $2x$ are white and $3x$ are blue. A ball is selected at random, What is the probability that it is
\begin{enumerate}[label=\alph*)]
\item not red ?
\item white ?
\end{enumerate}
%\begin{table}[H]
	\centering
\begin{tabular}{|c|c|c|}
\hline
Random variable &Value &Definition\\ \hline
\multirow{3}{*}{X} &0 &Slips of Rs 1\\
&1 &Slips of Rs 5\\
&2 &Slips of Rs 13\\ \hline
\multirow{2}{*}{Y} &0 &Box A\\
&1 &Box B\\\hline
\end{tabular}
\caption{}
\label{tab:Distribution}
\end{table}
See \tabref{tab:Distribution}.
\begin{align}
p_{Y}\brak{k}= \begin{cases} 
      \frac{1}{3} & {k=0} \\
      \frac{2}{3 }& {k=1} 
   \end{cases}
   \\
p_{Y|X}\brak{0|0} = \frac{19}{25}\, 
p_{Y|X}\brak{0|1} = \frac{6}{25}\,
p_{Y|X}\brak{1|0} = \frac{45}{50}\,
p_{Y|X}\brak{1|2} = \frac{5}{50}
\end{align}
The desired probability is the probability that a slip drawn at random is marked other than Rs 1,
\begin{align}
&=1-p_X\brak{0}\\
&= p_X(1) + p_X(2)
\end{align}
Using Bayes theorem,
\begin{align}
&= p_Y\brak{0} \times \pr{Y=0 | X=1} + p_Y\brak{1} \times \pr{Y=1|X=2}\\
&=\frac{1}{3} \times \frac{6}{25} + \frac{2}{3} \times \frac{5}{50}\\
&=\frac{11}{75}
\end{align}

\newpage

%\tableofcontents

\bigskip

\renewcommand{\thefigure}{\theenumi}
\renewcommand{\thetable}{\theenumi}
%\renewcommand{\theequation}{\theenumi}

%\begin{abstract}
%%\boldmath
%In this letter, an algorithm for evaluating the exact analytical bit error rate  (BER)  for the piecewise linear (PL) combiner for  multiple relays is presented. Previous results were available only for upto three relays. The algorithm is unique in the sense that  the actual mathematical expressions, that are prohibitively large, need not be explicitly obtained. The diversity gain due to multiple relays is shown through plots of the analytical BER, well supported by simulations. 
%
%\end{abstract}
% IEEEtran.cls defaults to using nonbold math in the Abstract.
% This preserves the distinction between vectors and scalars. However,
% if the journal you are submitting to favors bold math in the abstract,
% then you can use LaTeX's standard command \boldmath at the very start
% of the abstract to achieve this. Many IEEE journals frown on math
% in the abstract anyway.

% Note that keywords are not normally used for peerreview papers.
%\begin{IEEEkeywords}
%Cooperative diversity, decode and forward, piecewise linear
%\end{IEEEkeywords}



% For peer review papers, you can put extra information on the cover
% page as needed:
% \ifCLASSOPTIONpeerreview
% \begin{center} \bfseries EDICS Category: 3-BBND \end{center}
% \fi
%
% For peerreview papers, this IEEEtran command inserts a page break and
% creates the second title. It will be ignored for other modes.
%\IEEEpeerreviewmaketitle




If the letters of the word ASSASSINATION are arranged at random. Find the Probability that
\begin{enumerate}[label=(\alph*)]
\item Four $S's$ come consecutively in the word
\item Two  $I's$ and two $N's$ come together
\item All $A's$ are not coming together
\item No two $A's$ are coming together
\end{enumerate}
%\begin{table}[H]
	\centering
\begin{tabular}{|c|c|c|}
\hline
Random variable &Value &Definition\\ \hline
\multirow{3}{*}{X} &0 &Slips of Rs 1\\
&1 &Slips of Rs 5\\
&2 &Slips of Rs 13\\ \hline
\multirow{2}{*}{Y} &0 &Box A\\
&1 &Box B\\\hline
\end{tabular}
\caption{}
\label{tab:Distribution}
\end{table}
See \tabref{tab:Distribution}.
\begin{align}
p_{Y}\brak{k}= \begin{cases} 
      \frac{1}{3} & {k=0} \\
      \frac{2}{3 }& {k=1} 
   \end{cases}
   \\
p_{Y|X}\brak{0|0} = \frac{19}{25}\, 
p_{Y|X}\brak{0|1} = \frac{6}{25}\,
p_{Y|X}\brak{1|0} = \frac{45}{50}\,
p_{Y|X}\brak{1|2} = \frac{5}{50}
\end{align}
The desired probability is the probability that a slip drawn at random is marked other than Rs 1,
\begin{align}
&=1-p_X\brak{0}\\
&= p_X(1) + p_X(2)
\end{align}
Using Bayes theorem,
\begin{align}
&= p_Y\brak{0} \times \pr{Y=0 | X=1} + p_Y\brak{1} \times \pr{Y=1|X=2}\\
&=\frac{1}{3} \times \frac{6}{25} + \frac{2}{3} \times \frac{5}{50}\\
&=\frac{11}{75}
\end{align}

\newpage

%\tableofcontents

\bigskip

\renewcommand{\thefigure}{\theenumi}
\renewcommand{\thetable}{\theenumi}
%\renewcommand{\theequation}{\theenumi}

%\begin{abstract}
%%\boldmath
%In this letter, an algorithm for evaluating the exact analytical bit error rate  (BER)  for the piecewise linear (PL) combiner for  multiple relays is presented. Previous results were available only for upto three relays. The algorithm is unique in the sense that  the actual mathematical expressions, that are prohibitively large, need not be explicitly obtained. The diversity gain due to multiple relays is shown through plots of the analytical BER, well supported by simulations. 
%
%\end{abstract}
% IEEEtran.cls defaults to using nonbold math in the Abstract.
% This preserves the distinction between vectors and scalars. However,
% if the journal you are submitting to favors bold math in the abstract,
% then you can use LaTeX's standard command \boldmath at the very start
% of the abstract to achieve this. Many IEEE journals frown on math
% in the abstract anyway.

% Note that keywords are not normally used for peerreview papers.
%\begin{IEEEkeywords}
%Cooperative diversity, decode and forward, piecewise linear
%\end{IEEEkeywords}



% For peer review papers, you can put extra information on the cover
% page as needed:
% \ifCLASSOPTIONpeerreview
% \begin{center} \bfseries EDICS Category: 3-BBND \end{center}
% \fi
%
% For peerreview papers, this IEEEtran command inserts a page break and
% creates the second title. It will be ignored for other modes.
%\IEEEpeerreviewmaketitle




	\item One urn contains two black balls (labelled B1 and B2) and one white ball. A
	second urn contains one black ball and two white balls (labelled W1 and W2).
	Suppose the following experiment is performed. One of the two urns is chosen
	at random. Next a ball is randomly chosen from the urn. Then a second ball is
	chosen at random from the same urn without replacing the first ball.
	
	\begin{enumerate}
	\item What is the probability that two black balls are chosen?
	
	\item What is the probability that two balls of opposite colour are chosen?
	\end{enumerate}
	\solution
	%\begin{align}
    \label{eq:12.13.6.18.1}
	\because	\pr{A|B} &> \pr{A},\
\frac{\pr{AB}}{\pr{B}} > \pr{A}
\\
    \label{eq:12.13.6.18.2}
	\implies \pr{AB} &> \pr{A}\pr{B}
	\\
	\text{or, } \frac{\pr{AB}}{\pr{A}} &=\pr{B|A} > \pr{A}
\end{align}

\end{enumerate}

	\item A bag contains 4 red and 4 black balls, another bag contains 2 red and 6 black balls. One of the two bags is selected at random and a ball is drawn from the bag which is found to be red. Find the probability that the ball is drawn from the first bag.
\\
\solution
		%\begin{table}[H]
	\centering
\begin{tabular}{|c|c|c|}
\hline
Random variable &Value &Definition\\ \hline
\multirow{3}{*}{X} &0 &Slips of Rs 1\\
&1 &Slips of Rs 5\\
&2 &Slips of Rs 13\\ \hline
\multirow{2}{*}{Y} &0 &Box A\\
&1 &Box B\\\hline
\end{tabular}
\caption{}
\label{tab:Distribution}
\end{table}
See \tabref{tab:Distribution}.
\begin{align}
p_{Y}\brak{k}= \begin{cases} 
      \frac{1}{3} & {k=0} \\
      \frac{2}{3 }& {k=1} 
   \end{cases}
   \\
p_{Y|X}\brak{0|0} = \frac{19}{25}\, 
p_{Y|X}\brak{0|1} = \frac{6}{25}\,
p_{Y|X}\brak{1|0} = \frac{45}{50}\,
p_{Y|X}\brak{1|2} = \frac{5}{50}
\end{align}
The desired probability is the probability that a slip drawn at random is marked other than Rs 1,
\begin{align}
&=1-p_X\brak{0}\\
&= p_X(1) + p_X(2)
\end{align}
Using Bayes theorem,
\begin{align}
&= p_Y\brak{0} \times \pr{Y=0 | X=1} + p_Y\brak{1} \times \pr{Y=1|X=2}\\
&=\frac{1}{3} \times \frac{6}{25} + \frac{2}{3} \times \frac{5}{50}\\
&=\frac{11}{75}
\end{align}

\newpage

%\tableofcontents

\bigskip

\renewcommand{\thefigure}{\theenumi}
\renewcommand{\thetable}{\theenumi}
%\renewcommand{\theequation}{\theenumi}

%\begin{abstract}
%%\boldmath
%In this letter, an algorithm for evaluating the exact analytical bit error rate  (BER)  for the piecewise linear (PL) combiner for  multiple relays is presented. Previous results were available only for upto three relays. The algorithm is unique in the sense that  the actual mathematical expressions, that are prohibitively large, need not be explicitly obtained. The diversity gain due to multiple relays is shown through plots of the analytical BER, well supported by simulations. 
%
%\end{abstract}
% IEEEtran.cls defaults to using nonbold math in the Abstract.
% This preserves the distinction between vectors and scalars. However,
% if the journal you are submitting to favors bold math in the abstract,
% then you can use LaTeX's standard command \boldmath at the very start
% of the abstract to achieve this. Many IEEE journals frown on math
% in the abstract anyway.

% Note that keywords are not normally used for peerreview papers.
%\begin{IEEEkeywords}
%Cooperative diversity, decode and forward, piecewise linear
%\end{IEEEkeywords}



% For peer review papers, you can put extra information on the cover
% page as needed:
% \ifCLASSOPTIONpeerreview
% \begin{center} \bfseries EDICS Category: 3-BBND \end{center}
% \fi
%
% For peerreview papers, this IEEEtran command inserts a page break and
% creates the second title. It will be ignored for other modes.
%\IEEEpeerreviewmaketitle




  \item
  Cards with numbers 2 to 101 are placed in a box. A card is selected at random.Find the probability that the card has
\begin{enumerate}[label=(\roman*)]
	\item an even number 
	\item a square number
\end{enumerate}
\solution
%\begin{table}[H]
	\centering
\begin{tabular}{|c|c|c|}
\hline
Random variable &Value &Definition\\ \hline
\multirow{3}{*}{X} &0 &Slips of Rs 1\\
&1 &Slips of Rs 5\\
&2 &Slips of Rs 13\\ \hline
\multirow{2}{*}{Y} &0 &Box A\\
&1 &Box B\\\hline
\end{tabular}
\caption{}
\label{tab:Distribution}
\end{table}
See \tabref{tab:Distribution}.
\begin{align}
p_{Y}\brak{k}= \begin{cases} 
      \frac{1}{3} & {k=0} \\
      \frac{2}{3 }& {k=1} 
   \end{cases}
   \\
p_{Y|X}\brak{0|0} = \frac{19}{25}\, 
p_{Y|X}\brak{0|1} = \frac{6}{25}\,
p_{Y|X}\brak{1|0} = \frac{45}{50}\,
p_{Y|X}\brak{1|2} = \frac{5}{50}
\end{align}
The desired probability is the probability that a slip drawn at random is marked other than Rs 1,
\begin{align}
&=1-p_X\brak{0}\\
&= p_X(1) + p_X(2)
\end{align}
Using Bayes theorem,
\begin{align}
&= p_Y\brak{0} \times \pr{Y=0 | X=1} + p_Y\brak{1} \times \pr{Y=1|X=2}\\
&=\frac{1}{3} \times \frac{6}{25} + \frac{2}{3} \times \frac{5}{50}\\
&=\frac{11}{75}
\end{align}

\newpage

%\tableofcontents

\bigskip

\renewcommand{\thefigure}{\theenumi}
\renewcommand{\thetable}{\theenumi}
%\renewcommand{\theequation}{\theenumi}

%\begin{abstract}
%%\boldmath
%In this letter, an algorithm for evaluating the exact analytical bit error rate  (BER)  for the piecewise linear (PL) combiner for  multiple relays is presented. Previous results were available only for upto three relays. The algorithm is unique in the sense that  the actual mathematical expressions, that are prohibitively large, need not be explicitly obtained. The diversity gain due to multiple relays is shown through plots of the analytical BER, well supported by simulations. 
%
%\end{abstract}
% IEEEtran.cls defaults to using nonbold math in the Abstract.
% This preserves the distinction between vectors and scalars. However,
% if the journal you are submitting to favors bold math in the abstract,
% then you can use LaTeX's standard command \boldmath at the very start
% of the abstract to achieve this. Many IEEE journals frown on math
% in the abstract anyway.

% Note that keywords are not normally used for peerreview papers.
%\begin{IEEEkeywords}
%Cooperative diversity, decode and forward, piecewise linear
%\end{IEEEkeywords}



% For peer review papers, you can put extra information on the cover
% page as needed:
% \ifCLASSOPTIONpeerreview
% \begin{center} \bfseries EDICS Category: 3-BBND \end{center}
% \fi
%
% For peerreview papers, this IEEEtran command inserts a page break and
% creates the second title. It will be ignored for other modes.
%\IEEEpeerreviewmaketitle




\item
The king, queen and jack of clubs are removed from a deck of 52 playing cards and then well shuffled. Now one card is drawn at random from the remaining cards.  Determine the probability that the card is
\begin{enumerate}[label=(\roman*)]
\item a club
\item 10 of hearts
\end{enumerate}
\solution
%\begin{table}[H]
	\centering
\begin{tabular}{|c|c|c|}
\hline
Random variable &Value &Definition\\ \hline
\multirow{3}{*}{X} &0 &Slips of Rs 1\\
&1 &Slips of Rs 5\\
&2 &Slips of Rs 13\\ \hline
\multirow{2}{*}{Y} &0 &Box A\\
&1 &Box B\\\hline
\end{tabular}
\caption{}
\label{tab:Distribution}
\end{table}
See \tabref{tab:Distribution}.
\begin{align}
p_{Y}\brak{k}= \begin{cases} 
      \frac{1}{3} & {k=0} \\
      \frac{2}{3 }& {k=1} 
   \end{cases}
   \\
p_{Y|X}\brak{0|0} = \frac{19}{25}\, 
p_{Y|X}\brak{0|1} = \frac{6}{25}\,
p_{Y|X}\brak{1|0} = \frac{45}{50}\,
p_{Y|X}\brak{1|2} = \frac{5}{50}
\end{align}
The desired probability is the probability that a slip drawn at random is marked other than Rs 1,
\begin{align}
&=1-p_X\brak{0}\\
&= p_X(1) + p_X(2)
\end{align}
Using Bayes theorem,
\begin{align}
&= p_Y\brak{0} \times \pr{Y=0 | X=1} + p_Y\brak{1} \times \pr{Y=1|X=2}\\
&=\frac{1}{3} \times \frac{6}{25} + \frac{2}{3} \times \frac{5}{50}\\
&=\frac{11}{75}
\end{align}

\newpage

%\tableofcontents

\bigskip

\renewcommand{\thefigure}{\theenumi}
\renewcommand{\thetable}{\theenumi}
%\renewcommand{\theequation}{\theenumi}

%\begin{abstract}
%%\boldmath
%In this letter, an algorithm for evaluating the exact analytical bit error rate  (BER)  for the piecewise linear (PL) combiner for  multiple relays is presented. Previous results were available only for upto three relays. The algorithm is unique in the sense that  the actual mathematical expressions, that are prohibitively large, need not be explicitly obtained. The diversity gain due to multiple relays is shown through plots of the analytical BER, well supported by simulations. 
%
%\end{abstract}
% IEEEtran.cls defaults to using nonbold math in the Abstract.
% This preserves the distinction between vectors and scalars. However,
% if the journal you are submitting to favors bold math in the abstract,
% then you can use LaTeX's standard command \boldmath at the very start
% of the abstract to achieve this. Many IEEE journals frown on math
% in the abstract anyway.

% Note that keywords are not normally used for peerreview papers.
%\begin{IEEEkeywords}
%Cooperative diversity, decode and forward, piecewise linear
%\end{IEEEkeywords}



% For peer review papers, you can put extra information on the cover
% page as needed:
% \ifCLASSOPTIONpeerreview
% \begin{center} \bfseries EDICS Category: 3-BBND \end{center}
% \fi
%
% For peerreview papers, this IEEEtran command inserts a page break and
% creates the second title. It will be ignored for other modes.
%\IEEEpeerreviewmaketitle




\item A team of medical students doing their internship have to assist during surgeries
at a city hospital. The probabilities of surgeries rated as very complex, complex,
routine, simple or very simple are respectively, 0.15, 0.20, 0.31, 0.26, .08. Find
the probabilities that a particular surgery will be rated
\begin{enumerate}
	\item complex or very complex;
	\item neither very complex nor very simple;
	\item routine or complex
	\item routine or simple
\end{enumerate}
\solution
%\begin{table}[H]
	\centering
\begin{tabular}{|c|c|c|}
\hline
Random variable &Value &Definition\\ \hline
\multirow{3}{*}{X} &0 &Slips of Rs 1\\
&1 &Slips of Rs 5\\
&2 &Slips of Rs 13\\ \hline
\multirow{2}{*}{Y} &0 &Box A\\
&1 &Box B\\\hline
\end{tabular}
\caption{}
\label{tab:Distribution}
\end{table}
See \tabref{tab:Distribution}.
\begin{align}
p_{Y}\brak{k}= \begin{cases} 
      \frac{1}{3} & {k=0} \\
      \frac{2}{3 }& {k=1} 
   \end{cases}
   \\
p_{Y|X}\brak{0|0} = \frac{19}{25}\, 
p_{Y|X}\brak{0|1} = \frac{6}{25}\,
p_{Y|X}\brak{1|0} = \frac{45}{50}\,
p_{Y|X}\brak{1|2} = \frac{5}{50}
\end{align}
The desired probability is the probability that a slip drawn at random is marked other than Rs 1,
\begin{align}
&=1-p_X\brak{0}\\
&= p_X(1) + p_X(2)
\end{align}
Using Bayes theorem,
\begin{align}
&= p_Y\brak{0} \times \pr{Y=0 | X=1} + p_Y\brak{1} \times \pr{Y=1|X=2}\\
&=\frac{1}{3} \times \frac{6}{25} + \frac{2}{3} \times \frac{5}{50}\\
&=\frac{11}{75}
\end{align}

\newpage

%\tableofcontents

\bigskip

\renewcommand{\thefigure}{\theenumi}
\renewcommand{\thetable}{\theenumi}
%\renewcommand{\theequation}{\theenumi}

%\begin{abstract}
%%\boldmath
%In this letter, an algorithm for evaluating the exact analytical bit error rate  (BER)  for the piecewise linear (PL) combiner for  multiple relays is presented. Previous results were available only for upto three relays. The algorithm is unique in the sense that  the actual mathematical expressions, that are prohibitively large, need not be explicitly obtained. The diversity gain due to multiple relays is shown through plots of the analytical BER, well supported by simulations. 
%
%\end{abstract}
% IEEEtran.cls defaults to using nonbold math in the Abstract.
% This preserves the distinction between vectors and scalars. However,
% if the journal you are submitting to favors bold math in the abstract,
% then you can use LaTeX's standard command \boldmath at the very start
% of the abstract to achieve this. Many IEEE journals frown on math
% in the abstract anyway.

% Note that keywords are not normally used for peerreview papers.
%\begin{IEEEkeywords}
%Cooperative diversity, decode and forward, piecewise linear
%\end{IEEEkeywords}



% For peer review papers, you can put extra information on the cover
% page as needed:
% \ifCLASSOPTIONpeerreview
% \begin{center} \bfseries EDICS Category: 3-BBND \end{center}
% \fi
%
% For peerreview papers, this IEEEtran command inserts a page break and
% creates the second title. It will be ignored for other modes.
%\IEEEpeerreviewmaketitle




\item A card is selected from a pack of 52 cards.
\begin{enumerate}[label=(\alph*)]
    \item How many points are there in the sample space?
    \item Calculate the probability that the card is an ace of spades.
    \item Calculate the probability that the card is (i) an ace and (ii) black card.
\end{enumerate}
\solution
%Let $X$ be an bernoulli rv defined as in \tabref{tab:exemplar/11/16/3/26}.  Then, 
\begin{equation}
    p =
        \frac{4}{11} 
\end{equation}
\begin{table}[H]
	\centering
	\input{exemplar/11/16/3/26/tables/Table2.tex}
	\caption{}
        \label{tab:exemplar/11/16/3/26}
\end{table}

\item The probability that a non leap year selected at random will contain 53 sundays.
\\
\solution
%\begin{table}[H]
	\centering
\begin{tabular}{|c|c|c|}
\hline
Random variable &Value &Definition\\ \hline
\multirow{3}{*}{X} &0 &Slips of Rs 1\\
&1 &Slips of Rs 5\\
&2 &Slips of Rs 13\\ \hline
\multirow{2}{*}{Y} &0 &Box A\\
&1 &Box B\\\hline
\end{tabular}
\caption{}
\label{tab:Distribution}
\end{table}
See \tabref{tab:Distribution}.
\begin{align}
p_{Y}\brak{k}= \begin{cases} 
      \frac{1}{3} & {k=0} \\
      \frac{2}{3 }& {k=1} 
   \end{cases}
   \\
p_{Y|X}\brak{0|0} = \frac{19}{25}\, 
p_{Y|X}\brak{0|1} = \frac{6}{25}\,
p_{Y|X}\brak{1|0} = \frac{45}{50}\,
p_{Y|X}\brak{1|2} = \frac{5}{50}
\end{align}
The desired probability is the probability that a slip drawn at random is marked other than Rs 1,
\begin{align}
&=1-p_X\brak{0}\\
&= p_X(1) + p_X(2)
\end{align}
Using Bayes theorem,
\begin{align}
&= p_Y\brak{0} \times \pr{Y=0 | X=1} + p_Y\brak{1} \times \pr{Y=1|X=2}\\
&=\frac{1}{3} \times \frac{6}{25} + \frac{2}{3} \times \frac{5}{50}\\
&=\frac{11}{75}
\end{align}

\newpage

%\tableofcontents

\bigskip

\renewcommand{\thefigure}{\theenumi}
\renewcommand{\thetable}{\theenumi}
%\renewcommand{\theequation}{\theenumi}

%\begin{abstract}
%%\boldmath
%In this letter, an algorithm for evaluating the exact analytical bit error rate  (BER)  for the piecewise linear (PL) combiner for  multiple relays is presented. Previous results were available only for upto three relays. The algorithm is unique in the sense that  the actual mathematical expressions, that are prohibitively large, need not be explicitly obtained. The diversity gain due to multiple relays is shown through plots of the analytical BER, well supported by simulations. 
%
%\end{abstract}
% IEEEtran.cls defaults to using nonbold math in the Abstract.
% This preserves the distinction between vectors and scalars. However,
% if the journal you are submitting to favors bold math in the abstract,
% then you can use LaTeX's standard command \boldmath at the very start
% of the abstract to achieve this. Many IEEE journals frown on math
% in the abstract anyway.

% Note that keywords are not normally used for peerreview papers.
%\begin{IEEEkeywords}
%Cooperative diversity, decode and forward, piecewise linear
%\end{IEEEkeywords}



% For peer review papers, you can put extra information on the cover
% page as needed:
% \ifCLASSOPTIONpeerreview
% \begin{center} \bfseries EDICS Category: 3-BBND \end{center}
% \fi
%
% For peerreview papers, this IEEEtran command inserts a page break and
% creates the second title. It will be ignored for other modes.
%\IEEEpeerreviewmaketitle




\item One of the four persons John, Rita, Aslam or Gurpreet will be promoted next
month. Consequently the sample space consists of four elementary outcomes
S = {John promoted, Rita promoted, Aslam promoted, Gurpreet promoted}
You are told that the chances of John’s promotion is same as that of Gurpreet,
Rita’s chances of promotion are twice as likely as Johns. Aslam’s chances are
four times that of John.
\begin{enumerate}
	\item Determine
	\begin{enumerate}
		\item P (John promoted)
		\item P (Rita promoted)
		\item P (Aslam promoted)
		\item P (Gurpreet promoted)
	\end{enumerate}
	\item If A = {John promoted or Gurpreet promoted}, find P (A).
\end{enumerate}
\solution
%\begin{table}[H]
	\centering
\begin{tabular}{|c|c|c|}
\hline
Random variable &Value &Definition\\ \hline
\multirow{3}{*}{X} &0 &Slips of Rs 1\\
&1 &Slips of Rs 5\\
&2 &Slips of Rs 13\\ \hline
\multirow{2}{*}{Y} &0 &Box A\\
&1 &Box B\\\hline
\end{tabular}
\caption{}
\label{tab:Distribution}
\end{table}
See \tabref{tab:Distribution}.
\begin{align}
p_{Y}\brak{k}= \begin{cases} 
      \frac{1}{3} & {k=0} \\
      \frac{2}{3 }& {k=1} 
   \end{cases}
   \\
p_{Y|X}\brak{0|0} = \frac{19}{25}\, 
p_{Y|X}\brak{0|1} = \frac{6}{25}\,
p_{Y|X}\brak{1|0} = \frac{45}{50}\,
p_{Y|X}\brak{1|2} = \frac{5}{50}
\end{align}
The desired probability is the probability that a slip drawn at random is marked other than Rs 1,
\begin{align}
&=1-p_X\brak{0}\\
&= p_X(1) + p_X(2)
\end{align}
Using Bayes theorem,
\begin{align}
&= p_Y\brak{0} \times \pr{Y=0 | X=1} + p_Y\brak{1} \times \pr{Y=1|X=2}\\
&=\frac{1}{3} \times \frac{6}{25} + \frac{2}{3} \times \frac{5}{50}\\
&=\frac{11}{75}
\end{align}

\newpage

%\tableofcontents

\bigskip

\renewcommand{\thefigure}{\theenumi}
\renewcommand{\thetable}{\theenumi}
%\renewcommand{\theequation}{\theenumi}

%\begin{abstract}
%%\boldmath
%In this letter, an algorithm for evaluating the exact analytical bit error rate  (BER)  for the piecewise linear (PL) combiner for  multiple relays is presented. Previous results were available only for upto three relays. The algorithm is unique in the sense that  the actual mathematical expressions, that are prohibitively large, need not be explicitly obtained. The diversity gain due to multiple relays is shown through plots of the analytical BER, well supported by simulations. 
%
%\end{abstract}
% IEEEtran.cls defaults to using nonbold math in the Abstract.
% This preserves the distinction between vectors and scalars. However,
% if the journal you are submitting to favors bold math in the abstract,
% then you can use LaTeX's standard command \boldmath at the very start
% of the abstract to achieve this. Many IEEE journals frown on math
% in the abstract anyway.

% Note that keywords are not normally used for peerreview papers.
%\begin{IEEEkeywords}
%Cooperative diversity, decode and forward, piecewise linear
%\end{IEEEkeywords}



% For peer review papers, you can put extra information on the cover
% page as needed:
% \ifCLASSOPTIONpeerreview
% \begin{center} \bfseries EDICS Category: 3-BBND \end{center}
% \fi
%
% For peerreview papers, this IEEEtran command inserts a page break and
% creates the second title. It will be ignored for other modes.
%\IEEEpeerreviewmaketitle




\item A card is drawn from a deck of 52 cards. Find the probability of getting a king or a heart or a red card.\\
\solution
%\begin{table}[H]
	\centering
\begin{tabular}{|c|c|c|}
\hline
Random variable &Value &Definition\\ \hline
\multirow{3}{*}{X} &0 &Slips of Rs 1\\
&1 &Slips of Rs 5\\
&2 &Slips of Rs 13\\ \hline
\multirow{2}{*}{Y} &0 &Box A\\
&1 &Box B\\\hline
\end{tabular}
\caption{}
\label{tab:Distribution}
\end{table}
See \tabref{tab:Distribution}.
\begin{align}
p_{Y}\brak{k}= \begin{cases} 
      \frac{1}{3} & {k=0} \\
      \frac{2}{3 }& {k=1} 
   \end{cases}
   \\
p_{Y|X}\brak{0|0} = \frac{19}{25}\, 
p_{Y|X}\brak{0|1} = \frac{6}{25}\,
p_{Y|X}\brak{1|0} = \frac{45}{50}\,
p_{Y|X}\brak{1|2} = \frac{5}{50}
\end{align}
The desired probability is the probability that a slip drawn at random is marked other than Rs 1,
\begin{align}
&=1-p_X\brak{0}\\
&= p_X(1) + p_X(2)
\end{align}
Using Bayes theorem,
\begin{align}
&= p_Y\brak{0} \times \pr{Y=0 | X=1} + p_Y\brak{1} \times \pr{Y=1|X=2}\\
&=\frac{1}{3} \times \frac{6}{25} + \frac{2}{3} \times \frac{5}{50}\\
&=\frac{11}{75}
\end{align}

\newpage

%\tableofcontents

\bigskip

\renewcommand{\thefigure}{\theenumi}
\renewcommand{\thetable}{\theenumi}
%\renewcommand{\theequation}{\theenumi}

%\begin{abstract}
%%\boldmath
%In this letter, an algorithm for evaluating the exact analytical bit error rate  (BER)  for the piecewise linear (PL) combiner for  multiple relays is presented. Previous results were available only for upto three relays. The algorithm is unique in the sense that  the actual mathematical expressions, that are prohibitively large, need not be explicitly obtained. The diversity gain due to multiple relays is shown through plots of the analytical BER, well supported by simulations. 
%
%\end{abstract}
% IEEEtran.cls defaults to using nonbold math in the Abstract.
% This preserves the distinction between vectors and scalars. However,
% if the journal you are submitting to favors bold math in the abstract,
% then you can use LaTeX's standard command \boldmath at the very start
% of the abstract to achieve this. Many IEEE journals frown on math
% in the abstract anyway.

% Note that keywords are not normally used for peerreview papers.
%\begin{IEEEkeywords}
%Cooperative diversity, decode and forward, piecewise linear
%\end{IEEEkeywords}



% For peer review papers, you can put extra information on the cover
% page as needed:
% \ifCLASSOPTIONpeerreview
% \begin{center} \bfseries EDICS Category: 3-BBND \end{center}
% \fi
%
% For peerreview papers, this IEEEtran command inserts a page break and
% creates the second title. It will be ignored for other modes.
%\IEEEpeerreviewmaketitle




\item The probability that a student will pass his examination is 0.73, the probability of
the student getting a compartment is 0.13, and the probability that the student will
either pass or get compartment is 0.96. State True or False.\\
\solution
%\begin{table}[H]
	\centering
\begin{tabular}{|c|c|c|}
\hline
Random variable &Value &Definition\\ \hline
\multirow{3}{*}{X} &0 &Slips of Rs 1\\
&1 &Slips of Rs 5\\
&2 &Slips of Rs 13\\ \hline
\multirow{2}{*}{Y} &0 &Box A\\
&1 &Box B\\\hline
\end{tabular}
\caption{}
\label{tab:Distribution}
\end{table}
See \tabref{tab:Distribution}.
\begin{align}
p_{Y}\brak{k}= \begin{cases} 
      \frac{1}{3} & {k=0} \\
      \frac{2}{3 }& {k=1} 
   \end{cases}
   \\
p_{Y|X}\brak{0|0} = \frac{19}{25}\, 
p_{Y|X}\brak{0|1} = \frac{6}{25}\,
p_{Y|X}\brak{1|0} = \frac{45}{50}\,
p_{Y|X}\brak{1|2} = \frac{5}{50}
\end{align}
The desired probability is the probability that a slip drawn at random is marked other than Rs 1,
\begin{align}
&=1-p_X\brak{0}\\
&= p_X(1) + p_X(2)
\end{align}
Using Bayes theorem,
\begin{align}
&= p_Y\brak{0} \times \pr{Y=0 | X=1} + p_Y\brak{1} \times \pr{Y=1|X=2}\\
&=\frac{1}{3} \times \frac{6}{25} + \frac{2}{3} \times \frac{5}{50}\\
&=\frac{11}{75}
\end{align}

\newpage

%\tableofcontents

\bigskip

\renewcommand{\thefigure}{\theenumi}
\renewcommand{\thetable}{\theenumi}
%\renewcommand{\theequation}{\theenumi}

%\begin{abstract}
%%\boldmath
%In this letter, an algorithm for evaluating the exact analytical bit error rate  (BER)  for the piecewise linear (PL) combiner for  multiple relays is presented. Previous results were available only for upto three relays. The algorithm is unique in the sense that  the actual mathematical expressions, that are prohibitively large, need not be explicitly obtained. The diversity gain due to multiple relays is shown through plots of the analytical BER, well supported by simulations. 
%
%\end{abstract}
% IEEEtran.cls defaults to using nonbold math in the Abstract.
% This preserves the distinction between vectors and scalars. However,
% if the journal you are submitting to favors bold math in the abstract,
% then you can use LaTeX's standard command \boldmath at the very start
% of the abstract to achieve this. Many IEEE journals frown on math
% in the abstract anyway.

% Note that keywords are not normally used for peerreview papers.
%\begin{IEEEkeywords}
%Cooperative diversity, decode and forward, piecewise linear
%\end{IEEEkeywords}



% For peer review papers, you can put extra information on the cover
% page as needed:
% \ifCLASSOPTIONpeerreview
% \begin{center} \bfseries EDICS Category: 3-BBND \end{center}
% \fi
%
% For peerreview papers, this IEEEtran command inserts a page break and
% creates the second title. It will be ignored for other modes.
%\IEEEpeerreviewmaketitle




\item A card is selected from a pack of 52 cards\\
\begin{enumerate}[label=(\alph*)]
\item How many points are there in the sample space?
\item Calculate the probability that the cards is an ace of spades.
\item Calculate the probability that the card is (i) an ace (ii)black card.\\
\end{enumerate}
%\input{ncert/11/16/3/4_1/Prob_4.tex}
\item In a non-leap year, the probability of having 53 tuesdays or 53 wednesdays is\\
\solution
%A non-leap year has a total of 365 days, and a week has 7 days.\\
So it can be expressed as 
\begin{align}
365\text{days} &=52\times 7+1 \text{day}
\end{align}
$\implies$ 52 tuesdays or wednesdays\\
Random variable X denotes the days of a week
\begin{align}
p_X\brak{k}&=\frac{1}{7}; \quad \brak{1<k<7}
\end{align}
So the probability of extra day being tuesday or wednesday is
\begin{align}
p_X\brak{3}+p_X\brak{4}&=\frac{1}{7}+\frac{1}{7}=\frac{2}{7}
\end{align}



\item There are 1000 sealed envelopes in a box, 10 of them contain a cash prize of
Rs 100 each, 100 of them contain a cash prize of Rs 50 each and 200 of them
contain a cash prize of Rs 10 each and rest do not contain any cash prize. If they
are well shuffled and an envelope is picked up out, what is the probability that it
contains no cash prize?\\
\solution
%\begin{table}[H]
	\centering
\begin{tabular}{|c|c|c|}
\hline
Random variable &Value &Definition\\ \hline
\multirow{3}{*}{X} &0 &Slips of Rs 1\\
&1 &Slips of Rs 5\\
&2 &Slips of Rs 13\\ \hline
\multirow{2}{*}{Y} &0 &Box A\\
&1 &Box B\\\hline
\end{tabular}
\caption{}
\label{tab:Distribution}
\end{table}
See \tabref{tab:Distribution}.
\begin{align}
p_{Y}\brak{k}= \begin{cases} 
      \frac{1}{3} & {k=0} \\
      \frac{2}{3 }& {k=1} 
   \end{cases}
   \\
p_{Y|X}\brak{0|0} = \frac{19}{25}\, 
p_{Y|X}\brak{0|1} = \frac{6}{25}\,
p_{Y|X}\brak{1|0} = \frac{45}{50}\,
p_{Y|X}\brak{1|2} = \frac{5}{50}
\end{align}
The desired probability is the probability that a slip drawn at random is marked other than Rs 1,
\begin{align}
&=1-p_X\brak{0}\\
&= p_X(1) + p_X(2)
\end{align}
Using Bayes theorem,
\begin{align}
&= p_Y\brak{0} \times \pr{Y=0 | X=1} + p_Y\brak{1} \times \pr{Y=1|X=2}\\
&=\frac{1}{3} \times \frac{6}{25} + \frac{2}{3} \times \frac{5}{50}\\
&=\frac{11}{75}
\end{align}

\newpage

%\tableofcontents

\bigskip

\renewcommand{\thefigure}{\theenumi}
\renewcommand{\thetable}{\theenumi}
%\renewcommand{\theequation}{\theenumi}

%\begin{abstract}
%%\boldmath
%In this letter, an algorithm for evaluating the exact analytical bit error rate  (BER)  for the piecewise linear (PL) combiner for  multiple relays is presented. Previous results were available only for upto three relays. The algorithm is unique in the sense that  the actual mathematical expressions, that are prohibitively large, need not be explicitly obtained. The diversity gain due to multiple relays is shown through plots of the analytical BER, well supported by simulations. 
%
%\end{abstract}
% IEEEtran.cls defaults to using nonbold math in the Abstract.
% This preserves the distinction between vectors and scalars. However,
% if the journal you are submitting to favors bold math in the abstract,
% then you can use LaTeX's standard command \boldmath at the very start
% of the abstract to achieve this. Many IEEE journals frown on math
% in the abstract anyway.

% Note that keywords are not normally used for peerreview papers.
%\begin{IEEEkeywords}
%Cooperative diversity, decode and forward, piecewise linear
%\end{IEEEkeywords}



% For peer review papers, you can put extra information on the cover
% page as needed:
% \ifCLASSOPTIONpeerreview
% \begin{center} \bfseries EDICS Category: 3-BBND \end{center}
% \fi
%
% For peerreview papers, this IEEEtran command inserts a page break and
% creates the second title. It will be ignored for other modes.
%\IEEEpeerreviewmaketitle




\item 
A die is thrown and a card is selected at random from a deck of 52 playing cards. The probability of getting an even number on the die and a spade card.\\
\solution
%\begin{table}[H]
	\centering
\begin{tabular}{|c|c|c|}
\hline
Random variable &Value &Definition\\ \hline
\multirow{3}{*}{X} &0 &Slips of Rs 1\\
&1 &Slips of Rs 5\\
&2 &Slips of Rs 13\\ \hline
\multirow{2}{*}{Y} &0 &Box A\\
&1 &Box B\\\hline
\end{tabular}
\caption{}
\label{tab:Distribution}
\end{table}
See \tabref{tab:Distribution}.
\begin{align}
p_{Y}\brak{k}= \begin{cases} 
      \frac{1}{3} & {k=0} \\
      \frac{2}{3 }& {k=1} 
   \end{cases}
   \\
p_{Y|X}\brak{0|0} = \frac{19}{25}\, 
p_{Y|X}\brak{0|1} = \frac{6}{25}\,
p_{Y|X}\brak{1|0} = \frac{45}{50}\,
p_{Y|X}\brak{1|2} = \frac{5}{50}
\end{align}
The desired probability is the probability that a slip drawn at random is marked other than Rs 1,
\begin{align}
&=1-p_X\brak{0}\\
&= p_X(1) + p_X(2)
\end{align}
Using Bayes theorem,
\begin{align}
&= p_Y\brak{0} \times \pr{Y=0 | X=1} + p_Y\brak{1} \times \pr{Y=1|X=2}\\
&=\frac{1}{3} \times \frac{6}{25} + \frac{2}{3} \times \frac{5}{50}\\
&=\frac{11}{75}
\end{align}

\newpage

%\tableofcontents

\bigskip

\renewcommand{\thefigure}{\theenumi}
\renewcommand{\thetable}{\theenumi}
%\renewcommand{\theequation}{\theenumi}

%\begin{abstract}
%%\boldmath
%In this letter, an algorithm for evaluating the exact analytical bit error rate  (BER)  for the piecewise linear (PL) combiner for  multiple relays is presented. Previous results were available only for upto three relays. The algorithm is unique in the sense that  the actual mathematical expressions, that are prohibitively large, need not be explicitly obtained. The diversity gain due to multiple relays is shown through plots of the analytical BER, well supported by simulations. 
%
%\end{abstract}
% IEEEtran.cls defaults to using nonbold math in the Abstract.
% This preserves the distinction between vectors and scalars. However,
% if the journal you are submitting to favors bold math in the abstract,
% then you can use LaTeX's standard command \boldmath at the very start
% of the abstract to achieve this. Many IEEE journals frown on math
% in the abstract anyway.

% Note that keywords are not normally used for peerreview papers.
%\begin{IEEEkeywords}
%Cooperative diversity, decode and forward, piecewise linear
%\end{IEEEkeywords}



% For peer review papers, you can put extra information on the cover
% page as needed:
% \ifCLASSOPTIONpeerreview
% \begin{center} \bfseries EDICS Category: 3-BBND \end{center}
% \fi
%
% For peerreview papers, this IEEEtran command inserts a page break and
% creates the second title. It will be ignored for other modes.
%\IEEEpeerreviewmaketitle




\item
If 4-digit numbers greater than 5,000 are randomly formed from the digits 0, 1, 3, 5, and 7, what is the probability of forming a number divisible by 5 when:
\begin{enumerate}
    \item The digits are repeated?
    \item The repetition of digits is not allowed?
\end{enumerate}
\solution
%\begin{table}[H]
	\centering
\begin{tabular}{|c|c|c|}
\hline
Random variable &Value &Definition\\ \hline
\multirow{3}{*}{X} &0 &Slips of Rs 1\\
&1 &Slips of Rs 5\\
&2 &Slips of Rs 13\\ \hline
\multirow{2}{*}{Y} &0 &Box A\\
&1 &Box B\\\hline
\end{tabular}
\caption{}
\label{tab:Distribution}
\end{table}
See \tabref{tab:Distribution}.
\begin{align}
p_{Y}\brak{k}= \begin{cases} 
      \frac{1}{3} & {k=0} \\
      \frac{2}{3 }& {k=1} 
   \end{cases}
   \\
p_{Y|X}\brak{0|0} = \frac{19}{25}\, 
p_{Y|X}\brak{0|1} = \frac{6}{25}\,
p_{Y|X}\brak{1|0} = \frac{45}{50}\,
p_{Y|X}\brak{1|2} = \frac{5}{50}
\end{align}
The desired probability is the probability that a slip drawn at random is marked other than Rs 1,
\begin{align}
&=1-p_X\brak{0}\\
&= p_X(1) + p_X(2)
\end{align}
Using Bayes theorem,
\begin{align}
&= p_Y\brak{0} \times \pr{Y=0 | X=1} + p_Y\brak{1} \times \pr{Y=1|X=2}\\
&=\frac{1}{3} \times \frac{6}{25} + \frac{2}{3} \times \frac{5}{50}\\
&=\frac{11}{75}
\end{align}

\newpage

%\tableofcontents

\bigskip

\renewcommand{\thefigure}{\theenumi}
\renewcommand{\thetable}{\theenumi}
%\renewcommand{\theequation}{\theenumi}

%\begin{abstract}
%%\boldmath
%In this letter, an algorithm for evaluating the exact analytical bit error rate  (BER)  for the piecewise linear (PL) combiner for  multiple relays is presented. Previous results were available only for upto three relays. The algorithm is unique in the sense that  the actual mathematical expressions, that are prohibitively large, need not be explicitly obtained. The diversity gain due to multiple relays is shown through plots of the analytical BER, well supported by simulations. 
%
%\end{abstract}
% IEEEtran.cls defaults to using nonbold math in the Abstract.
% This preserves the distinction between vectors and scalars. However,
% if the journal you are submitting to favors bold math in the abstract,
% then you can use LaTeX's standard command \boldmath at the very start
% of the abstract to achieve this. Many IEEE journals frown on math
% in the abstract anyway.

% Note that keywords are not normally used for peerreview papers.
%\begin{IEEEkeywords}
%Cooperative diversity, decode and forward, piecewise linear
%\end{IEEEkeywords}



% For peer review papers, you can put extra information on the cover
% page as needed:
% \ifCLASSOPTIONpeerreview
% \begin{center} \bfseries EDICS Category: 3-BBND \end{center}
% \fi
%
% For peerreview papers, this IEEEtran command inserts a page break and
% creates the second title. It will be ignored for other modes.
%\IEEEpeerreviewmaketitle




\item Consider the probability space $\brak{\Omega, \mathcal{G}, P}$ where $\Omega = [0,2]$ and $\mathcal{G} = \cbrak{\phi, \Omega, [0,1], (1,2]}$. Let $X$ and $Y$ be two functions on $\Omega$ defined as
\begin{align*}
    X(\omega) = 
    \begin{cases}
        1 & \text{if }\omega \in [0, 1]\\
        2 & \text{if }\omega \in (1, 2]
    \end{cases}
\end{align*}
and
\begin{align*}
    Y(\omega) = 
    \begin{cases}
        2 & \text{if }\omega \in [0, 1.5]\\
        3 & \text{if }\omega \in (1.5, 2].
    \end{cases}
\end{align*}
Then which one of the following statements is true?
\begin{enumerate}
    \item [(A)] $X$ is a random variable with respect to $\mathcal{G}$, but $Y$ is not a random variable with respect to $\mathcal{G}$.
    \item [(B)] $Y$ is a random variable with respect to $\mathcal{G}$, but $X$ is not a random variable with respect to $\mathcal{G}$.
    \item [(C)] Neither $X$ nor $Y$ is a random variable with respect to $\mathcal{G}$.
    \item [(D)] Both $X$ and $Y$ are random variables with respect to $\mathcal{G}$.
\end{enumerate} \hfill (GATE ST 2023)\\
\solution
%\begin{table}[H]
	\centering
\begin{tabular}{|c|c|c|}
\hline
Random variable &Value &Definition\\ \hline
\multirow{3}{*}{X} &0 &Slips of Rs 1\\
&1 &Slips of Rs 5\\
&2 &Slips of Rs 13\\ \hline
\multirow{2}{*}{Y} &0 &Box A\\
&1 &Box B\\\hline
\end{tabular}
\caption{}
\label{tab:Distribution}
\end{table}
See \tabref{tab:Distribution}.
\begin{align}
p_{Y}\brak{k}= \begin{cases} 
      \frac{1}{3} & {k=0} \\
      \frac{2}{3 }& {k=1} 
   \end{cases}
   \\
p_{Y|X}\brak{0|0} = \frac{19}{25}\, 
p_{Y|X}\brak{0|1} = \frac{6}{25}\,
p_{Y|X}\brak{1|0} = \frac{45}{50}\,
p_{Y|X}\brak{1|2} = \frac{5}{50}
\end{align}
The desired probability is the probability that a slip drawn at random is marked other than Rs 1,
\begin{align}
&=1-p_X\brak{0}\\
&= p_X(1) + p_X(2)
\end{align}
Using Bayes theorem,
\begin{align}
&= p_Y\brak{0} \times \pr{Y=0 | X=1} + p_Y\brak{1} \times \pr{Y=1|X=2}\\
&=\frac{1}{3} \times \frac{6}{25} + \frac{2}{3} \times \frac{5}{50}\\
&=\frac{11}{75}
\end{align}

\newpage

%\tableofcontents

\bigskip

\renewcommand{\thefigure}{\theenumi}
\renewcommand{\thetable}{\theenumi}
%\renewcommand{\theequation}{\theenumi}

%\begin{abstract}
%%\boldmath
%In this letter, an algorithm for evaluating the exact analytical bit error rate  (BER)  for the piecewise linear (PL) combiner for  multiple relays is presented. Previous results were available only for upto three relays. The algorithm is unique in the sense that  the actual mathematical expressions, that are prohibitively large, need not be explicitly obtained. The diversity gain due to multiple relays is shown through plots of the analytical BER, well supported by simulations. 
%
%\end{abstract}
% IEEEtran.cls defaults to using nonbold math in the Abstract.
% This preserves the distinction between vectors and scalars. However,
% if the journal you are submitting to favors bold math in the abstract,
% then you can use LaTeX's standard command \boldmath at the very start
% of the abstract to achieve this. Many IEEE journals frown on math
% in the abstract anyway.

% Note that keywords are not normally used for peerreview papers.
%\begin{IEEEkeywords}
%Cooperative diversity, decode and forward, piecewise linear
%\end{IEEEkeywords}



% For peer review papers, you can put extra information on the cover
% page as needed:
% \ifCLASSOPTIONpeerreview
% \begin{center} \bfseries EDICS Category: 3-BBND \end{center}
% \fi
%
% For peerreview papers, this IEEEtran command inserts a page break and
% creates the second title. It will be ignored for other modes.
%\IEEEpeerreviewmaketitle




	\item  A die is loaded in such a way that each odd number is twice as likely to occur as
each even number. Find $P(G)$, where $G$ is the event that a number greater than
3 occurs on a single roll of the die.
\\
\solution
		%\begin{table}[H]
	\centering
\begin{tabular}{|c|c|c|}
\hline
Random variable &Value &Definition\\ \hline
\multirow{3}{*}{X} &0 &Slips of Rs 1\\
&1 &Slips of Rs 5\\
&2 &Slips of Rs 13\\ \hline
\multirow{2}{*}{Y} &0 &Box A\\
&1 &Box B\\\hline
\end{tabular}
\caption{}
\label{tab:Distribution}
\end{table}
See \tabref{tab:Distribution}.
\begin{align}
p_{Y}\brak{k}= \begin{cases} 
      \frac{1}{3} & {k=0} \\
      \frac{2}{3 }& {k=1} 
   \end{cases}
   \\
p_{Y|X}\brak{0|0} = \frac{19}{25}\, 
p_{Y|X}\brak{0|1} = \frac{6}{25}\,
p_{Y|X}\brak{1|0} = \frac{45}{50}\,
p_{Y|X}\brak{1|2} = \frac{5}{50}
\end{align}
The desired probability is the probability that a slip drawn at random is marked other than Rs 1,
\begin{align}
&=1-p_X\brak{0}\\
&= p_X(1) + p_X(2)
\end{align}
Using Bayes theorem,
\begin{align}
&= p_Y\brak{0} \times \pr{Y=0 | X=1} + p_Y\brak{1} \times \pr{Y=1|X=2}\\
&=\frac{1}{3} \times \frac{6}{25} + \frac{2}{3} \times \frac{5}{50}\\
&=\frac{11}{75}
\end{align}

\newpage

%\tableofcontents

\bigskip

\renewcommand{\thefigure}{\theenumi}
\renewcommand{\thetable}{\theenumi}
%\renewcommand{\theequation}{\theenumi}

%\begin{abstract}
%%\boldmath
%In this letter, an algorithm for evaluating the exact analytical bit error rate  (BER)  for the piecewise linear (PL) combiner for  multiple relays is presented. Previous results were available only for upto three relays. The algorithm is unique in the sense that  the actual mathematical expressions, that are prohibitively large, need not be explicitly obtained. The diversity gain due to multiple relays is shown through plots of the analytical BER, well supported by simulations. 
%
%\end{abstract}
% IEEEtran.cls defaults to using nonbold math in the Abstract.
% This preserves the distinction between vectors and scalars. However,
% if the journal you are submitting to favors bold math in the abstract,
% then you can use LaTeX's standard command \boldmath at the very start
% of the abstract to achieve this. Many IEEE journals frown on math
% in the abstract anyway.

% Note that keywords are not normally used for peerreview papers.
%\begin{IEEEkeywords}
%Cooperative diversity, decode and forward, piecewise linear
%\end{IEEEkeywords}



% For peer review papers, you can put extra information on the cover
% page as needed:
% \ifCLASSOPTIONpeerreview
% \begin{center} \bfseries EDICS Category: 3-BBND \end{center}
% \fi
%
% For peerreview papers, this IEEEtran command inserts a page break and
% creates the second title. It will be ignored for other modes.
%\IEEEpeerreviewmaketitle




	\item All the jacks, queens and kings are removed from a deck of 52 playing cards. The remaining cards are well shuffled and then one card is drawn at random. Giving ace a value 1 similar value for other cards, find the probability that the card has a value 
		\begin{enumerate}
			\item 7
			\item greater than 7
			\item less than 7
		\end{enumerate}
		%Number of cards left after removing all jacks, queens and kings 
\begin{align}
N	= 52 - 4\times 3
	= 40
\end{align}
%\begin{table}[H]
%\def\arraystretch{1.2}
%\begin{tabular}{|c|c|c|}
%\hline
%	\textbf{Parameter} &\textbf{Value} &\textbf{Description}\\ \hline
%	$X$ &1-10 &Represents the value of the card picked \\ \hline
%\end{tabular}
%\end{table}
Let $1 \le X \le 10$ be the value of the card picked.  Then,
\begin{align}
	p_X(k) &= \Pr(X=k)\ \forall\ 1 \leq k \leq 10\\
	&= \frac{4\times 1}{40}\\
	&= \frac{1}{10}\\
	\therefore p_X(k) &= 
	\begin{cases}
		\frac{1}{10} & 1 \leq k \leq 10\\
		0 & \text{otherwise}
	\end{cases}
\end{align}
and
\begin{align}
	F_{X}(k) &= \sum_{m=0}^{k}p_{X}(m) \quad 1 \leq k \leq 10\\
	&= \frac{k}{10}\\
	\therefore F_{X}(k) &= 
	\begin{cases}
		0 & k \leq 0\\
		\frac{k}{10} & 1\leq k \leq 10\\
		1 & k > 10 
	\end{cases}
\end{align}
\begin{enumerate}
	\item Probability that card has value equal to 7 is
		\begin{align}
			 p_{X}(7)
			= \frac{1}{10}
		\end{align}
	\item Probability that card has value greater than 7 is
		\begin{align}
			1 - F_X(7)
			&= 1 - \frac{7}{10}
			\\
			&= \frac{3}{10}
		\end{align}
	\item Probability that card has value less than 7 is
		\begin{align}
			 F_{X}(6)
			=\frac{6}{10}
		\end{align}
\end{enumerate}

  \item A Lot consists of 48 mobile phones of which 42 are good, 3 have only minor defects and 3 have major defects.Varnika will buy a phone if it is good but the trader will only buy a mobile if it has no major defects. One phone is selected at random from the lot. What is the probability that it is
\begin{enumerate}
	\item acceptable to Varnika?
            \item acceptable to the trader?
\end{enumerate}
\solution
	%\begin{table}[H]
	\centering
\begin{tabular}{|c|c|c|}
\hline
Random variable &Value &Definition\\ \hline
\multirow{3}{*}{X} &0 &Slips of Rs 1\\
&1 &Slips of Rs 5\\
&2 &Slips of Rs 13\\ \hline
\multirow{2}{*}{Y} &0 &Box A\\
&1 &Box B\\\hline
\end{tabular}
\caption{}
\label{tab:Distribution}
\end{table}
See \tabref{tab:Distribution}.
\begin{align}
p_{Y}\brak{k}= \begin{cases} 
      \frac{1}{3} & {k=0} \\
      \frac{2}{3 }& {k=1} 
   \end{cases}
   \\
p_{Y|X}\brak{0|0} = \frac{19}{25}\, 
p_{Y|X}\brak{0|1} = \frac{6}{25}\,
p_{Y|X}\brak{1|0} = \frac{45}{50}\,
p_{Y|X}\brak{1|2} = \frac{5}{50}
\end{align}
The desired probability is the probability that a slip drawn at random is marked other than Rs 1,
\begin{align}
&=1-p_X\brak{0}\\
&= p_X(1) + p_X(2)
\end{align}
Using Bayes theorem,
\begin{align}
&= p_Y\brak{0} \times \pr{Y=0 | X=1} + p_Y\brak{1} \times \pr{Y=1|X=2}\\
&=\frac{1}{3} \times \frac{6}{25} + \frac{2}{3} \times \frac{5}{50}\\
&=\frac{11}{75}
\end{align}

\newpage

%\tableofcontents

\bigskip

\renewcommand{\thefigure}{\theenumi}
\renewcommand{\thetable}{\theenumi}
%\renewcommand{\theequation}{\theenumi}

%\begin{abstract}
%%\boldmath
%In this letter, an algorithm for evaluating the exact analytical bit error rate  (BER)  for the piecewise linear (PL) combiner for  multiple relays is presented. Previous results were available only for upto three relays. The algorithm is unique in the sense that  the actual mathematical expressions, that are prohibitively large, need not be explicitly obtained. The diversity gain due to multiple relays is shown through plots of the analytical BER, well supported by simulations. 
%
%\end{abstract}
% IEEEtran.cls defaults to using nonbold math in the Abstract.
% This preserves the distinction between vectors and scalars. However,
% if the journal you are submitting to favors bold math in the abstract,
% then you can use LaTeX's standard command \boldmath at the very start
% of the abstract to achieve this. Many IEEE journals frown on math
% in the abstract anyway.

% Note that keywords are not normally used for peerreview papers.
%\begin{IEEEkeywords}
%Cooperative diversity, decode and forward, piecewise linear
%\end{IEEEkeywords}



% For peer review papers, you can put extra information on the cover
% page as needed:
% \ifCLASSOPTIONpeerreview
% \begin{center} \bfseries EDICS Category: 3-BBND \end{center}
% \fi
%
% For peerreview papers, this IEEEtran command inserts a page break and
% creates the second title. It will be ignored for other modes.
%\IEEEpeerreviewmaketitle




 \item A student says that if you throw a die, it will show up 1 or not 1. Therefore, the probability of getting 1 and the probability of getting 'not 1' each is equal to $\frac{1}{2}$. Is this correct? Give reasons.\\
 \solution
        %\begin{table}[H]
	\centering
\begin{tabular}{|c|c|c|}
\hline
Random variable &Value &Definition\\ \hline
\multirow{3}{*}{X} &0 &Slips of Rs 1\\
&1 &Slips of Rs 5\\
&2 &Slips of Rs 13\\ \hline
\multirow{2}{*}{Y} &0 &Box A\\
&1 &Box B\\\hline
\end{tabular}
\caption{}
\label{tab:Distribution}
\end{table}
See \tabref{tab:Distribution}.
\begin{align}
p_{Y}\brak{k}= \begin{cases} 
      \frac{1}{3} & {k=0} \\
      \frac{2}{3 }& {k=1} 
   \end{cases}
   \\
p_{Y|X}\brak{0|0} = \frac{19}{25}\, 
p_{Y|X}\brak{0|1} = \frac{6}{25}\,
p_{Y|X}\brak{1|0} = \frac{45}{50}\,
p_{Y|X}\brak{1|2} = \frac{5}{50}
\end{align}
The desired probability is the probability that a slip drawn at random is marked other than Rs 1,
\begin{align}
&=1-p_X\brak{0}\\
&= p_X(1) + p_X(2)
\end{align}
Using Bayes theorem,
\begin{align}
&= p_Y\brak{0} \times \pr{Y=0 | X=1} + p_Y\brak{1} \times \pr{Y=1|X=2}\\
&=\frac{1}{3} \times \frac{6}{25} + \frac{2}{3} \times \frac{5}{50}\\
&=\frac{11}{75}
\end{align}

\newpage

%\tableofcontents

\bigskip

\renewcommand{\thefigure}{\theenumi}
\renewcommand{\thetable}{\theenumi}
%\renewcommand{\theequation}{\theenumi}

%\begin{abstract}
%%\boldmath
%In this letter, an algorithm for evaluating the exact analytical bit error rate  (BER)  for the piecewise linear (PL) combiner for  multiple relays is presented. Previous results were available only for upto three relays. The algorithm is unique in the sense that  the actual mathematical expressions, that are prohibitively large, need not be explicitly obtained. The diversity gain due to multiple relays is shown through plots of the analytical BER, well supported by simulations. 
%
%\end{abstract}
% IEEEtran.cls defaults to using nonbold math in the Abstract.
% This preserves the distinction between vectors and scalars. However,
% if the journal you are submitting to favors bold math in the abstract,
% then you can use LaTeX's standard command \boldmath at the very start
% of the abstract to achieve this. Many IEEE journals frown on math
% in the abstract anyway.

% Note that keywords are not normally used for peerreview papers.
%\begin{IEEEkeywords}
%Cooperative diversity, decode and forward, piecewise linear
%\end{IEEEkeywords}



% For peer review papers, you can put extra information on the cover
% page as needed:
% \ifCLASSOPTIONpeerreview
% \begin{center} \bfseries EDICS Category: 3-BBND \end{center}
% \fi
%
% For peerreview papers, this IEEEtran command inserts a page break and
% creates the second title. It will be ignored for other modes.
%\IEEEpeerreviewmaketitle




   \item Four candidates A, B, C, D have ap-
plied for the assignment to coach a school cricket
team. If A is twice as likely to be selected as B, and
B and C are given about the same chance of being
selected, while C is twice as likely to be selected
as D, what are the probabilities that
\begin{enumerate}
\item C will be selected?
\item A will not be selected?
\end{enumerate}
	%\begin{table}[H]
	\centering
\begin{tabular}{|c|c|c|}
\hline
Random variable &Value &Definition\\ \hline
\multirow{3}{*}{X} &0 &Slips of Rs 1\\
&1 &Slips of Rs 5\\
&2 &Slips of Rs 13\\ \hline
\multirow{2}{*}{Y} &0 &Box A\\
&1 &Box B\\\hline
\end{tabular}
\caption{}
\label{tab:Distribution}
\end{table}
See \tabref{tab:Distribution}.
\begin{align}
p_{Y}\brak{k}= \begin{cases} 
      \frac{1}{3} & {k=0} \\
      \frac{2}{3 }& {k=1} 
   \end{cases}
   \\
p_{Y|X}\brak{0|0} = \frac{19}{25}\, 
p_{Y|X}\brak{0|1} = \frac{6}{25}\,
p_{Y|X}\brak{1|0} = \frac{45}{50}\,
p_{Y|X}\brak{1|2} = \frac{5}{50}
\end{align}
The desired probability is the probability that a slip drawn at random is marked other than Rs 1,
\begin{align}
&=1-p_X\brak{0}\\
&= p_X(1) + p_X(2)
\end{align}
Using Bayes theorem,
\begin{align}
&= p_Y\brak{0} \times \pr{Y=0 | X=1} + p_Y\brak{1} \times \pr{Y=1|X=2}\\
&=\frac{1}{3} \times \frac{6}{25} + \frac{2}{3} \times \frac{5}{50}\\
&=\frac{11}{75}
\end{align}

\newpage

%\tableofcontents

\bigskip

\renewcommand{\thefigure}{\theenumi}
\renewcommand{\thetable}{\theenumi}
%\renewcommand{\theequation}{\theenumi}

%\begin{abstract}
%%\boldmath
%In this letter, an algorithm for evaluating the exact analytical bit error rate  (BER)  for the piecewise linear (PL) combiner for  multiple relays is presented. Previous results were available only for upto three relays. The algorithm is unique in the sense that  the actual mathematical expressions, that are prohibitively large, need not be explicitly obtained. The diversity gain due to multiple relays is shown through plots of the analytical BER, well supported by simulations. 
%
%\end{abstract}
% IEEEtran.cls defaults to using nonbold math in the Abstract.
% This preserves the distinction between vectors and scalars. However,
% if the journal you are submitting to favors bold math in the abstract,
% then you can use LaTeX's standard command \boldmath at the very start
% of the abstract to achieve this. Many IEEE journals frown on math
% in the abstract anyway.

% Note that keywords are not normally used for peerreview papers.
%\begin{IEEEkeywords}
%Cooperative diversity, decode and forward, piecewise linear
%\end{IEEEkeywords}



% For peer review papers, you can put extra information on the cover
% page as needed:
% \ifCLASSOPTIONpeerreview
% \begin{center} \bfseries EDICS Category: 3-BBND \end{center}
% \fi
%
% For peerreview papers, this IEEEtran command inserts a page break and
% creates the second title. It will be ignored for other modes.
%\IEEEpeerreviewmaketitle




 \item A bag contain 24 balls of which $x$ balls are red, $2x$ are white and $3x$ are blue. A ball is selected at random, What is the probability that it is
\begin{enumerate}[label=\alph*)]
\item not red ?
\item white ?
\end{enumerate}
%\begin{table}[H]
	\centering
\begin{tabular}{|c|c|c|}
\hline
Random variable &Value &Definition\\ \hline
\multirow{3}{*}{X} &0 &Slips of Rs 1\\
&1 &Slips of Rs 5\\
&2 &Slips of Rs 13\\ \hline
\multirow{2}{*}{Y} &0 &Box A\\
&1 &Box B\\\hline
\end{tabular}
\caption{}
\label{tab:Distribution}
\end{table}
See \tabref{tab:Distribution}.
\begin{align}
p_{Y}\brak{k}= \begin{cases} 
      \frac{1}{3} & {k=0} \\
      \frac{2}{3 }& {k=1} 
   \end{cases}
   \\
p_{Y|X}\brak{0|0} = \frac{19}{25}\, 
p_{Y|X}\brak{0|1} = \frac{6}{25}\,
p_{Y|X}\brak{1|0} = \frac{45}{50}\,
p_{Y|X}\brak{1|2} = \frac{5}{50}
\end{align}
The desired probability is the probability that a slip drawn at random is marked other than Rs 1,
\begin{align}
&=1-p_X\brak{0}\\
&= p_X(1) + p_X(2)
\end{align}
Using Bayes theorem,
\begin{align}
&= p_Y\brak{0} \times \pr{Y=0 | X=1} + p_Y\brak{1} \times \pr{Y=1|X=2}\\
&=\frac{1}{3} \times \frac{6}{25} + \frac{2}{3} \times \frac{5}{50}\\
&=\frac{11}{75}
\end{align}

\newpage

%\tableofcontents

\bigskip

\renewcommand{\thefigure}{\theenumi}
\renewcommand{\thetable}{\theenumi}
%\renewcommand{\theequation}{\theenumi}

%\begin{abstract}
%%\boldmath
%In this letter, an algorithm for evaluating the exact analytical bit error rate  (BER)  for the piecewise linear (PL) combiner for  multiple relays is presented. Previous results were available only for upto three relays. The algorithm is unique in the sense that  the actual mathematical expressions, that are prohibitively large, need not be explicitly obtained. The diversity gain due to multiple relays is shown through plots of the analytical BER, well supported by simulations. 
%
%\end{abstract}
% IEEEtran.cls defaults to using nonbold math in the Abstract.
% This preserves the distinction between vectors and scalars. However,
% if the journal you are submitting to favors bold math in the abstract,
% then you can use LaTeX's standard command \boldmath at the very start
% of the abstract to achieve this. Many IEEE journals frown on math
% in the abstract anyway.

% Note that keywords are not normally used for peerreview papers.
%\begin{IEEEkeywords}
%Cooperative diversity, decode and forward, piecewise linear
%\end{IEEEkeywords}



% For peer review papers, you can put extra information on the cover
% page as needed:
% \ifCLASSOPTIONpeerreview
% \begin{center} \bfseries EDICS Category: 3-BBND \end{center}
% \fi
%
% For peerreview papers, this IEEEtran command inserts a page break and
% creates the second title. It will be ignored for other modes.
%\IEEEpeerreviewmaketitle




If the letters of the word ASSASSINATION are arranged at random. Find the Probability that
\begin{enumerate}[label=(\alph*)]
\item Four $S's$ come consecutively in the word
\item Two  $I's$ and two $N's$ come together
\item All $A's$ are not coming together
\item No two $A's$ are coming together
\end{enumerate}
%\begin{table}[H]
	\centering
\begin{tabular}{|c|c|c|}
\hline
Random variable &Value &Definition\\ \hline
\multirow{3}{*}{X} &0 &Slips of Rs 1\\
&1 &Slips of Rs 5\\
&2 &Slips of Rs 13\\ \hline
\multirow{2}{*}{Y} &0 &Box A\\
&1 &Box B\\\hline
\end{tabular}
\caption{}
\label{tab:Distribution}
\end{table}
See \tabref{tab:Distribution}.
\begin{align}
p_{Y}\brak{k}= \begin{cases} 
      \frac{1}{3} & {k=0} \\
      \frac{2}{3 }& {k=1} 
   \end{cases}
   \\
p_{Y|X}\brak{0|0} = \frac{19}{25}\, 
p_{Y|X}\brak{0|1} = \frac{6}{25}\,
p_{Y|X}\brak{1|0} = \frac{45}{50}\,
p_{Y|X}\brak{1|2} = \frac{5}{50}
\end{align}
The desired probability is the probability that a slip drawn at random is marked other than Rs 1,
\begin{align}
&=1-p_X\brak{0}\\
&= p_X(1) + p_X(2)
\end{align}
Using Bayes theorem,
\begin{align}
&= p_Y\brak{0} \times \pr{Y=0 | X=1} + p_Y\brak{1} \times \pr{Y=1|X=2}\\
&=\frac{1}{3} \times \frac{6}{25} + \frac{2}{3} \times \frac{5}{50}\\
&=\frac{11}{75}
\end{align}

\newpage

%\tableofcontents

\bigskip

\renewcommand{\thefigure}{\theenumi}
\renewcommand{\thetable}{\theenumi}
%\renewcommand{\theequation}{\theenumi}

%\begin{abstract}
%%\boldmath
%In this letter, an algorithm for evaluating the exact analytical bit error rate  (BER)  for the piecewise linear (PL) combiner for  multiple relays is presented. Previous results were available only for upto three relays. The algorithm is unique in the sense that  the actual mathematical expressions, that are prohibitively large, need not be explicitly obtained. The diversity gain due to multiple relays is shown through plots of the analytical BER, well supported by simulations. 
%
%\end{abstract}
% IEEEtran.cls defaults to using nonbold math in the Abstract.
% This preserves the distinction between vectors and scalars. However,
% if the journal you are submitting to favors bold math in the abstract,
% then you can use LaTeX's standard command \boldmath at the very start
% of the abstract to achieve this. Many IEEE journals frown on math
% in the abstract anyway.

% Note that keywords are not normally used for peerreview papers.
%\begin{IEEEkeywords}
%Cooperative diversity, decode and forward, piecewise linear
%\end{IEEEkeywords}



% For peer review papers, you can put extra information on the cover
% page as needed:
% \ifCLASSOPTIONpeerreview
% \begin{center} \bfseries EDICS Category: 3-BBND \end{center}
% \fi
%
% For peerreview papers, this IEEEtran command inserts a page break and
% creates the second title. It will be ignored for other modes.
%\IEEEpeerreviewmaketitle




	\item One urn contains two black balls (labelled B1 and B2) and one white ball. A
	second urn contains one black ball and two white balls (labelled W1 and W2).
	Suppose the following experiment is performed. One of the two urns is chosen
	at random. Next a ball is randomly chosen from the urn. Then a second ball is
	chosen at random from the same urn without replacing the first ball.
	
	\begin{enumerate}
	\item What is the probability that two black balls are chosen?
	
	\item What is the probability that two balls of opposite colour are chosen?
	\end{enumerate}
	\solution
	%\begin{align}
    \label{eq:12.13.6.18.1}
	\because	\pr{A|B} &> \pr{A},\
\frac{\pr{AB}}{\pr{B}} > \pr{A}
\\
    \label{eq:12.13.6.18.2}
	\implies \pr{AB} &> \pr{A}\pr{B}
	\\
	\text{or, } \frac{\pr{AB}}{\pr{A}} &=\pr{B|A} > \pr{A}
\end{align}

\end{enumerate}

\section{Multinomial}
\subsection{Formulae}
\input{app/ident.tex}
\subsection{NCERT}
\begin{enumerate}[label=\thesubsection.\arabic*,ref=\thesubsection.\theenumi]
\item A box contains 5 red marbles, 8 white marbles and 4 green marbles. One marble is taken
out of the box at random. What is the probability that the marble taken out will be 
\begin{multicols}{3}
\begin{enumerate}
    \item red ? 
    \item white ? 
    \item  not green?
\end{enumerate}
\end{multicols}
\solution
From \eqref{eq:ncert/11/16/4/1/1},
\begin{enumerate}
\item Probability that the marble taken out is red
\begin{align}
	p_{R,W,G}\brak{1,0,0} &= \frac{\comb{5}{1}\comb{8}{0}\comb{4}{0}}{\comb{17}{1}} 
	 = \frac{5}{17} 
	\end{align}
\item Probability that the marble taken out is white
\begin{align}
	p_{R,W,G}\brak{0,1,0} = \frac{\comb{5}{0}\comb{8}{1}\comb{4}{0}}{\comb{17}{1}} 
	= \frac{8}{17} 
	\end{align}
\item Probability that the marble taken out is not green
\begin{align}
1 - p_{R,W,G}\brak{0,0,1} = 1- \frac{\comb{5}{0}\comb{8}{0}\comb{4}{1}}{\comb{17}{1}} 
	  = 1 - \frac{4}{17} = \frac{13}{17} 
\end{align}
\end{enumerate}

\item A box contains 10 red marbles, 20 blue marbles and 30 green marbles. 5 marbles
are drawn from the box, what is the probability that
\begin{enumerate}
\item all will be blue?
\item atleast one will be green?
\end{enumerate}
\solution
%See \eqref{eq:ncert/11/16/4/1/1}. 
In this question, 
\begin{align}
N = 60, \, R = 10, \,B = 20,  \, G = 30, \,n = 5
\end{align}
\begin{enumerate}
\item From 
\eqref{eq:ncert/11/16/4/1/1}, 
\begin{align}
	p_{R,B,G}\brak{0,5,0}  = \frac{\comb{20}{5}}{\comb{60}{5}}
\end{align}
\item 
	Since 
\begin{align}
	p_{R,B,G}\brak{r,b,0}  = \frac{\comb{R}{r}
	\comb{B}{b}
	}
{
	\comb{R+B+G}{r+b}
}
\end{align}
The probability that at least one marble is green is given by 
\begin{align}
	1 - \sum_{r+b = n}^{}p_{R,B,G}\brak{r,b,0}  = 
	1 - \sum_{r+b = n}^{}    \frac{\comb{R}{r}
	\comb{B}{b}
	}
{
	\comb{R+B+G}{r+b}
}
	= 1 - \frac{\comb{R+B}{n}
	}
{
	\comb{R+B+G}{n}
}
\end{align}
from  \eqref{eq:ncert/11/16/4/1/5}.
Substituting numerical values, the desired probability is 
\begin{align}
1 - \frac{\comb{30}{5}}{\comb{60}{5}}
\end{align}
\end{enumerate}

\item A box of oranges is inspected by examining three randomly selected oranges drawn without replacement. If all the three oranges are good, the box is approved for sale, otherwise, it is rejected. Find the probability that a box containing 15 oranges out of which 12 are good and 3 are bad ones will be approved for sale.\\
	\solution
Choosing 
\begin{align}
R = 12, B = 3, G = 0, n = 3, r = 3, b = 0, g = 0 
\end{align}
in
\eqref{eq:ncert/11/16/4/1/1}
the desired probability is 
\begin{align}
	p_{R,B,G}\brak{3,0,0} = 
	\frac{\comb{12}{3}}
{
	\comb{15}{3}
}=
 \frac{44}{91}
\end{align}




\item A box contains 3 orange balls, 3 green balls and 2 blue balls. Three balls are drawn at random from the box without replacement. The probability of drawing 2 green balls and one blue ball is
\begin{enumerate}
\item $\frac{3}{28}$\\
\item $\frac{2}{21}$\\
\item $\frac{1}{28}$\\
\item $\frac{167}{168}$
\end{enumerate}
%The desired probability is
\begin{align}
p_{O,G,B}\brak{0,2,1} = \frac{\comb{3}{0}\comb{3}{2}\comb{2}{1}}{\comb{8}{3}}
= \frac{3}{28}
\end{align} 

\item %The desired probability is
\begin{align}
p_{R,B}\brak{1,2} = \frac{\comb{5}{1}\comb{3}{2}}{\comb{8}{3}}
= \frac{15}{56}
\end{align} 


\item A bag contains 5 red and 3 blue balls. If 3 balls are drawn at random without replacement the probability that exactly two of the three balls were red, the first ball being red is\\
%See \eqref{eq:ncert/11/16/4/1/1}. In this question,\\
As the first ball drawn is red,
\begin{align}
N=7 , R=4 , B=3 , G=0 , r=1 , b=1 , g=0
\end{align}
The desired probability is,
\begin{align}
p_{R,B}\brak {1,1} = \frac{\comb{4}{1}\comb{3}{1}}{\comb{7}{2}}
= \frac{4}{7}
\end{align}

    \end{enumerate}

    \section{Miscellaneous}
\begin{enumerate}[label=\thesection.\arabic*,ref=\thesection.\theenumi]
\item  The random variable $X$ has a probability distribution \pr{X} of the following form, where $k$ is some number 
\begin{align}
  \pr{X} =
    \begin{cases}
      k,  & x=0\\
      2k, & x=1\\
      3k, & x=2\\
      0 , & \text{otherwise}
    \end{cases}       
\end{align}
		\begin{enumerate}
			\item
 Determine the value of $k$ 

\item  Find \pr{X < 2},\pr{X \leq 2},\pr{X \geq 2}  
		\end{enumerate}
\solution
%The desired probabilities are 
% 
\begin{enumerate} 
\item $X \sim \bnm{2}{\frac{1}{2}}$
 \begin{align}
	 p_{X}(2) = \comb{2}{2}\frac{1}{2}^{2} = \frac{1}{4}
\end{align}
\item $X \sim \bnm{3}{\frac{1}{2}}$
 \begin{align}
	 p_{X}(3) = \comb{2}{0}\frac{1}{2}^{3} = \frac{1}{8}
\end{align}
\item $X \sim \bnm{4}{\frac{1}{2}}$
 \begin{align}
	 p_{X}(4) = \comb{4}{4}\frac{1}{2}^{4} = \frac{1}{16}
\end{align}
\end{enumerate}

\item State which of the following are not the probability distributions of a random 
variable. Give reasons for your answer
\renewcommand{\labelenumii}{\roman{enumii}}
\begin{enumerate}

\item \begin{table}[ht!]\centering
%\input{ncert/13/4/tables/Book.tex}
\end{table}

\item \begin{table}[ht!]\centering
%\input{ncert/13/4/tables/Book2.tex}
\end{table}

\item  \begin{table}[ht!]\centering
%\input{ncert/13/4/tables/Book3.tex}	
\end{table}

\item  \begin{table}[ht!]\centering
%\input{ncert/13/4/tables/Book5.tex}	
\end{table} 


\end{enumerate}
\item A random variable X has the following probability distribution\\

Determine

\begin{enumerate}
\begin{table}[ht!]\centering
%\input{ncert/13/4/tables/Book10.tex}
\end{table}
\item k
\item P$(X < 3)$
\item P$(X > 6)$
\item P$(0 < X < 3)$

\end{enumerate}

\item The random variable X has a probability distribution P(X) of the following form,
where k is some number :
\[P(x)=\begin{cases}
k, & \mbox{if}\; x= 0\\
2k, & \mbox{if}\; x= 1\\
3k, & \mbox{if}\; x= 2\\
0, & otherwise
\end{cases}\]
\begin{enumerate}
\item Determine the value of k.
\item Find P $(X < 2)$, P $(X \leq 2)$, P$(X \geq 2)$
\end{enumerate}
\item
A game consists of spinning an arrow which comes to rest pointing at one of the regions (1, 2 or 3) (Fig. 13.1). Are the outcomes 1, 2 and 3 equally likely to occur? Give reasons.\\
\begin{figure}[!ht]
	\begin{center}
		
		\resizebox{\columnwidth}{!}{\input{exemplar/10/13/2/6/figs/circle.tex}}
	\end{center}
	\caption{Fig.13.1}
	\label{fig:circle.tex}	
\end{figure}\\
\solution
%\begin{table}[H]
	\centering
\begin{tabular}{|c|c|c|}
\hline
Random variable &Value &Definition\\ \hline
\multirow{3}{*}{X} &0 &Slips of Rs 1\\
&1 &Slips of Rs 5\\
&2 &Slips of Rs 13\\ \hline
\multirow{2}{*}{Y} &0 &Box A\\
&1 &Box B\\\hline
\end{tabular}
\caption{}
\label{tab:Distribution}
\end{table}
See \tabref{tab:Distribution}.
\begin{align}
p_{Y}\brak{k}= \begin{cases} 
      \frac{1}{3} & {k=0} \\
      \frac{2}{3 }& {k=1} 
   \end{cases}
   \\
p_{Y|X}\brak{0|0} = \frac{19}{25}\, 
p_{Y|X}\brak{0|1} = \frac{6}{25}\,
p_{Y|X}\brak{1|0} = \frac{45}{50}\,
p_{Y|X}\brak{1|2} = \frac{5}{50}
\end{align}
The desired probability is the probability that a slip drawn at random is marked other than Rs 1,
\begin{align}
&=1-p_X\brak{0}\\
&= p_X(1) + p_X(2)
\end{align}
Using Bayes theorem,
\begin{align}
&= p_Y\brak{0} \times \pr{Y=0 | X=1} + p_Y\brak{1} \times \pr{Y=1|X=2}\\
&=\frac{1}{3} \times \frac{6}{25} + \frac{2}{3} \times \frac{5}{50}\\
&=\frac{11}{75}
\end{align}

\newpage

%\tableofcontents

\bigskip

\renewcommand{\thefigure}{\theenumi}
\renewcommand{\thetable}{\theenumi}
%\renewcommand{\theequation}{\theenumi}

%\begin{abstract}
%%\boldmath
%In this letter, an algorithm for evaluating the exact analytical bit error rate  (BER)  for the piecewise linear (PL) combiner for  multiple relays is presented. Previous results were available only for upto three relays. The algorithm is unique in the sense that  the actual mathematical expressions, that are prohibitively large, need not be explicitly obtained. The diversity gain due to multiple relays is shown through plots of the analytical BER, well supported by simulations. 
%
%\end{abstract}
% IEEEtran.cls defaults to using nonbold math in the Abstract.
% This preserves the distinction between vectors and scalars. However,
% if the journal you are submitting to favors bold math in the abstract,
% then you can use LaTeX's standard command \boldmath at the very start
% of the abstract to achieve this. Many IEEE journals frown on math
% in the abstract anyway.

% Note that keywords are not normally used for peerreview papers.
%\begin{IEEEkeywords}
%Cooperative diversity, decode and forward, piecewise linear
%\end{IEEEkeywords}



% For peer review papers, you can put extra information on the cover
% page as needed:
% \ifCLASSOPTIONpeerreview
% \begin{center} \bfseries EDICS Category: 3-BBND \end{center}
% \fi
%
% For peerreview papers, this IEEEtran command inserts a page break and
% creates the second title. It will be ignored for other modes.
%\IEEEpeerreviewmaketitle




\item Apoorv throws two dice once and computes the product of the numbers appearing on the dice. Peehu throws one die and squares the number that appears on it. Who has the better chance of getting the number 36? Why?\\
\solution
%\begin{table}[!ht]
\input{exemplar/12/13/3/41/tables/table.tex}
\caption{random variables of objects}
\label{tab:exemplar 12.13.3.41}
\end{table}
\begin{align}
\pr{X=i}=
\begin{cases}
\frac{1}{6},\text{when i=1}\\
\frac{2}{6},\text{when i=2}\\
\frac{3}{6},\text{when i=3}
\end{cases}
\end{align}

we know that that the conditional probability is defined as

                     $\pr{A|B}=\frac{\pr{A,B}}{\pr{B}}$


\begin{enumerate}
\item
The probability that a red ball will be selected is:
\begin{align}
\pr{Y=1}&=\pr{Y=1,X=1}+\pr{Y=1,X=2}+\pr{Y=1,X=3}\\
&=\pr{X=1}\times\pr{Y=1|X=1}+\pr{X=2}\times\pr{Y=1|X=2}+\pr{X=3}\times\pr{Y=1|X=3}\\
&=\frac{1}{6}\times\frac{3}{3}+\frac{2}{6}\times\frac{2}{3}+\frac{3}{6}\times0\\
&=\frac{7}{18}
\end{align}
\item
The probability that a white ball will be selected is:
\begin{align}
\pr{Y=0}&=\pr{Y=0,X=1}+\pr{Y=0,X=2}+\pr{Y=0,X=3}\\
&=\pr{X=1}\times\pr{Y=0|X=1}+\pr{X=2}\times\pr{Y=0|X=2}+\pr{X=3}\times\pr{Y=0|X=3}\\
&=\frac{1}{6}\times0+\frac{2}{6}\times\frac{1}{3}+\frac{3}{6}\times\frac{3}{3}\\
&=\frac{11}{18}
\end{align}
\end{enumerate}




\item 6 boys and 6 girls sit in a row at random. The probability that all the girls sit
together is
\begin{enumerate}
	\item $\frac{1}{432}$
	\item $\frac{12}{431}$
	\item $\frac{1}{132}$
	\item none of the above 
\end{enumerate}
			%\begin{table}[H]
	\centering
\begin{tabular}{|c|c|c|}
\hline
Random variable &Value &Definition\\ \hline
\multirow{3}{*}{X} &0 &Slips of Rs 1\\
&1 &Slips of Rs 5\\
&2 &Slips of Rs 13\\ \hline
\multirow{2}{*}{Y} &0 &Box A\\
&1 &Box B\\\hline
\end{tabular}
\caption{}
\label{tab:Distribution}
\end{table}
See \tabref{tab:Distribution}.
\begin{align}
p_{Y}\brak{k}= \begin{cases} 
      \frac{1}{3} & {k=0} \\
      \frac{2}{3 }& {k=1} 
   \end{cases}
   \\
p_{Y|X}\brak{0|0} = \frac{19}{25}\, 
p_{Y|X}\brak{0|1} = \frac{6}{25}\,
p_{Y|X}\brak{1|0} = \frac{45}{50}\,
p_{Y|X}\brak{1|2} = \frac{5}{50}
\end{align}
The desired probability is the probability that a slip drawn at random is marked other than Rs 1,
\begin{align}
&=1-p_X\brak{0}\\
&= p_X(1) + p_X(2)
\end{align}
Using Bayes theorem,
\begin{align}
&= p_Y\brak{0} \times \pr{Y=0 | X=1} + p_Y\brak{1} \times \pr{Y=1|X=2}\\
&=\frac{1}{3} \times \frac{6}{25} + \frac{2}{3} \times \frac{5}{50}\\
&=\frac{11}{75}
\end{align}

\newpage

%\tableofcontents

\bigskip

\renewcommand{\thefigure}{\theenumi}
\renewcommand{\thetable}{\theenumi}
%\renewcommand{\theequation}{\theenumi}

%\begin{abstract}
%%\boldmath
%In this letter, an algorithm for evaluating the exact analytical bit error rate  (BER)  for the piecewise linear (PL) combiner for  multiple relays is presented. Previous results were available only for upto three relays. The algorithm is unique in the sense that  the actual mathematical expressions, that are prohibitively large, need not be explicitly obtained. The diversity gain due to multiple relays is shown through plots of the analytical BER, well supported by simulations. 
%
%\end{abstract}
% IEEEtran.cls defaults to using nonbold math in the Abstract.
% This preserves the distinction between vectors and scalars. However,
% if the journal you are submitting to favors bold math in the abstract,
% then you can use LaTeX's standard command \boldmath at the very start
% of the abstract to achieve this. Many IEEE journals frown on math
% in the abstract anyway.

% Note that keywords are not normally used for peerreview papers.
%\begin{IEEEkeywords}
%Cooperative diversity, decode and forward, piecewise linear
%\end{IEEEkeywords}



% For peer review papers, you can put extra information on the cover
% page as needed:
% \ifCLASSOPTIONpeerreview
% \begin{center} \bfseries EDICS Category: 3-BBND \end{center}
% \fi
%
% For peerreview papers, this IEEEtran command inserts a page break and
% creates the second title. It will be ignored for other modes.
%\IEEEpeerreviewmaketitle




\item A card is selected from a deck of 52 cards. The probability of its being a red face card is
%\begin{table}[H]
	\centering
\begin{tabular}{|c|c|c|}
\hline
Random variable &Value &Definition\\ \hline
\multirow{3}{*}{X} &0 &Slips of Rs 1\\
&1 &Slips of Rs 5\\
&2 &Slips of Rs 13\\ \hline
\multirow{2}{*}{Y} &0 &Box A\\
&1 &Box B\\\hline
\end{tabular}
\caption{}
\label{tab:Distribution}
\end{table}
See \tabref{tab:Distribution}.
\begin{align}
p_{Y}\brak{k}= \begin{cases} 
      \frac{1}{3} & {k=0} \\
      \frac{2}{3 }& {k=1} 
   \end{cases}
   \\
p_{Y|X}\brak{0|0} = \frac{19}{25}\, 
p_{Y|X}\brak{0|1} = \frac{6}{25}\,
p_{Y|X}\brak{1|0} = \frac{45}{50}\,
p_{Y|X}\brak{1|2} = \frac{5}{50}
\end{align}
The desired probability is the probability that a slip drawn at random is marked other than Rs 1,
\begin{align}
&=1-p_X\brak{0}\\
&= p_X(1) + p_X(2)
\end{align}
Using Bayes theorem,
\begin{align}
&= p_Y\brak{0} \times \pr{Y=0 | X=1} + p_Y\brak{1} \times \pr{Y=1|X=2}\\
&=\frac{1}{3} \times \frac{6}{25} + \frac{2}{3} \times \frac{5}{50}\\
&=\frac{11}{75}
\end{align}

\newpage

%\tableofcontents

\bigskip

\renewcommand{\thefigure}{\theenumi}
\renewcommand{\thetable}{\theenumi}
%\renewcommand{\theequation}{\theenumi}

%\begin{abstract}
%%\boldmath
%In this letter, an algorithm for evaluating the exact analytical bit error rate  (BER)  for the piecewise linear (PL) combiner for  multiple relays is presented. Previous results were available only for upto three relays. The algorithm is unique in the sense that  the actual mathematical expressions, that are prohibitively large, need not be explicitly obtained. The diversity gain due to multiple relays is shown through plots of the analytical BER, well supported by simulations. 
%
%\end{abstract}
% IEEEtran.cls defaults to using nonbold math in the Abstract.
% This preserves the distinction between vectors and scalars. However,
% if the journal you are submitting to favors bold math in the abstract,
% then you can use LaTeX's standard command \boldmath at the very start
% of the abstract to achieve this. Many IEEE journals frown on math
% in the abstract anyway.

% Note that keywords are not normally used for peerreview papers.
%\begin{IEEEkeywords}
%Cooperative diversity, decode and forward, piecewise linear
%\end{IEEEkeywords}



% For peer review papers, you can put extra information on the cover
% page as needed:
% \ifCLASSOPTIONpeerreview
% \begin{center} \bfseries EDICS Category: 3-BBND \end{center}
% \fi
%
% For peerreview papers, this IEEEtran command inserts a page break and
% creates the second title. It will be ignored for other modes.
%\IEEEpeerreviewmaketitle




\item A die is loaded in such a way that each odd number is twice as likely to occur as each even number. Find P(G), where G is the event that a number greater than 3 occurs on a single roll of the die.\\
%\begin{table}[H]
	\centering
\begin{tabular}{|c|c|c|}
\hline
Random variable &Value &Definition\\ \hline
\multirow{3}{*}{X} &0 &Slips of Rs 1\\
&1 &Slips of Rs 5\\
&2 &Slips of Rs 13\\ \hline
\multirow{2}{*}{Y} &0 &Box A\\
&1 &Box B\\\hline
\end{tabular}
\caption{}
\label{tab:Distribution}
\end{table}
See \tabref{tab:Distribution}.
\begin{align}
p_{Y}\brak{k}= \begin{cases} 
      \frac{1}{3} & {k=0} \\
      \frac{2}{3 }& {k=1} 
   \end{cases}
   \\
p_{Y|X}\brak{0|0} = \frac{19}{25}\, 
p_{Y|X}\brak{0|1} = \frac{6}{25}\,
p_{Y|X}\brak{1|0} = \frac{45}{50}\,
p_{Y|X}\brak{1|2} = \frac{5}{50}
\end{align}
The desired probability is the probability that a slip drawn at random is marked other than Rs 1,
\begin{align}
&=1-p_X\brak{0}\\
&= p_X(1) + p_X(2)
\end{align}
Using Bayes theorem,
\begin{align}
&= p_Y\brak{0} \times \pr{Y=0 | X=1} + p_Y\brak{1} \times \pr{Y=1|X=2}\\
&=\frac{1}{3} \times \frac{6}{25} + \frac{2}{3} \times \frac{5}{50}\\
&=\frac{11}{75}
\end{align}

\newpage

%\tableofcontents

\bigskip

\renewcommand{\thefigure}{\theenumi}
\renewcommand{\thetable}{\theenumi}
%\renewcommand{\theequation}{\theenumi}

%\begin{abstract}
%%\boldmath
%In this letter, an algorithm for evaluating the exact analytical bit error rate  (BER)  for the piecewise linear (PL) combiner for  multiple relays is presented. Previous results were available only for upto three relays. The algorithm is unique in the sense that  the actual mathematical expressions, that are prohibitively large, need not be explicitly obtained. The diversity gain due to multiple relays is shown through plots of the analytical BER, well supported by simulations. 
%
%\end{abstract}
% IEEEtran.cls defaults to using nonbold math in the Abstract.
% This preserves the distinction between vectors and scalars. However,
% if the journal you are submitting to favors bold math in the abstract,
% then you can use LaTeX's standard command \boldmath at the very start
% of the abstract to achieve this. Many IEEE journals frown on math
% in the abstract anyway.

% Note that keywords are not normally used for peerreview papers.
%\begin{IEEEkeywords}
%Cooperative diversity, decode and forward, piecewise linear
%\end{IEEEkeywords}



% For peer review papers, you can put extra information on the cover
% page as needed:
% \ifCLASSOPTIONpeerreview
% \begin{center} \bfseries EDICS Category: 3-BBND \end{center}
% \fi
%
% For peerreview papers, this IEEEtran command inserts a page break and
% creates the second title. It will be ignored for other modes.
%\IEEEpeerreviewmaketitle





\item Determine the probability $p$,for each of following events.
\begin{enumerate}
\item An odd number appears in a single roll of dice.
\item Atleast one head appears in two tosses of fair coin.
\item A king,9 of hearts or 3 of spades appears in drawing a single card from a well shuffled deck of 52 cards.
\item The sum of 6 appears in single toss of a pair of fair dice.
\end{enumerate}
%\begin{table}[!ht]
\input{exemplar/12/13/3/41/tables/table.tex}
\caption{random variables of objects}
\label{tab:exemplar 12.13.3.41}
\end{table}
\begin{align}
\pr{X=i}=
\begin{cases}
\frac{1}{6},\text{when i=1}\\
\frac{2}{6},\text{when i=2}\\
\frac{3}{6},\text{when i=3}
\end{cases}
\end{align}

we know that that the conditional probability is defined as

                     $\pr{A|B}=\frac{\pr{A,B}}{\pr{B}}$


\begin{enumerate}
\item
The probability that a red ball will be selected is:
\begin{align}
\pr{Y=1}&=\pr{Y=1,X=1}+\pr{Y=1,X=2}+\pr{Y=1,X=3}\\
&=\pr{X=1}\times\pr{Y=1|X=1}+\pr{X=2}\times\pr{Y=1|X=2}+\pr{X=3}\times\pr{Y=1|X=3}\\
&=\frac{1}{6}\times\frac{3}{3}+\frac{2}{6}\times\frac{2}{3}+\frac{3}{6}\times0\\
&=\frac{7}{18}
\end{align}
\item
The probability that a white ball will be selected is:
\begin{align}
\pr{Y=0}&=\pr{Y=0,X=1}+\pr{Y=0,X=2}+\pr{Y=0,X=3}\\
&=\pr{X=1}\times\pr{Y=0|X=1}+\pr{X=2}\times\pr{Y=0|X=2}+\pr{X=3}\times\pr{Y=0|X=3}\\
&=\frac{1}{6}\times0+\frac{2}{6}\times\frac{1}{3}+\frac{3}{6}\times\frac{3}{3}\\
&=\frac{11}{18}
\end{align}
\end{enumerate}




\item Determine the probability p, for each of the following events.\\
(a) An odd number appears in a single toss of a fair die.\\
(b) At least one head appears in two tosses of a fair coin.\\
(c) A king, 9 of hearts, or 3 of spades appears in drawing a single card from a
well shuffled ordinary deck of 52 cards.\\
(d) The sum of 6 appears in a single toss of a pair of fair dice.
%\begin{table}[H]
	\centering
\begin{tabular}{|c|c|c|}
\hline
Random variable &Value &Definition\\ \hline
\multirow{3}{*}{X} &0 &Slips of Rs 1\\
&1 &Slips of Rs 5\\
&2 &Slips of Rs 13\\ \hline
\multirow{2}{*}{Y} &0 &Box A\\
&1 &Box B\\\hline
\end{tabular}
\caption{}
\label{tab:Distribution}
\end{table}
See \tabref{tab:Distribution}.
\begin{align}
p_{Y}\brak{k}= \begin{cases} 
      \frac{1}{3} & {k=0} \\
      \frac{2}{3 }& {k=1} 
   \end{cases}
   \\
p_{Y|X}\brak{0|0} = \frac{19}{25}\, 
p_{Y|X}\brak{0|1} = \frac{6}{25}\,
p_{Y|X}\brak{1|0} = \frac{45}{50}\,
p_{Y|X}\brak{1|2} = \frac{5}{50}
\end{align}
The desired probability is the probability that a slip drawn at random is marked other than Rs 1,
\begin{align}
&=1-p_X\brak{0}\\
&= p_X(1) + p_X(2)
\end{align}
Using Bayes theorem,
\begin{align}
&= p_Y\brak{0} \times \pr{Y=0 | X=1} + p_Y\brak{1} \times \pr{Y=1|X=2}\\
&=\frac{1}{3} \times \frac{6}{25} + \frac{2}{3} \times \frac{5}{50}\\
&=\frac{11}{75}
\end{align}

\newpage

%\tableofcontents

\bigskip

\renewcommand{\thefigure}{\theenumi}
\renewcommand{\thetable}{\theenumi}
%\renewcommand{\theequation}{\theenumi}

%\begin{abstract}
%%\boldmath
%In this letter, an algorithm for evaluating the exact analytical bit error rate  (BER)  for the piecewise linear (PL) combiner for  multiple relays is presented. Previous results were available only for upto three relays. The algorithm is unique in the sense that  the actual mathematical expressions, that are prohibitively large, need not be explicitly obtained. The diversity gain due to multiple relays is shown through plots of the analytical BER, well supported by simulations. 
%
%\end{abstract}
% IEEEtran.cls defaults to using nonbold math in the Abstract.
% This preserves the distinction between vectors and scalars. However,
% if the journal you are submitting to favors bold math in the abstract,
% then you can use LaTeX's standard command \boldmath at the very start
% of the abstract to achieve this. Many IEEE journals frown on math
% in the abstract anyway.

% Note that keywords are not normally used for peerreview papers.
%\begin{IEEEkeywords}
%Cooperative diversity, decode and forward, piecewise linear
%\end{IEEEkeywords}



% For peer review papers, you can put extra information on the cover
% page as needed:
% \ifCLASSOPTIONpeerreview
% \begin{center} \bfseries EDICS Category: 3-BBND \end{center}
% \fi
%
% For peerreview papers, this IEEEtran command inserts a page break and
% creates the second title. It will be ignored for other modes.
%\IEEEpeerreviewmaketitle




\item The probability distribution of a random variable X is given below:
\begin{table}[h]
    \centering
    \begin{tabular}{|c|c|c|c|c|}
        \hline
        X & 0 & 1 & 2 & 3 \\
        \hline
        $P(X)$ & $k$ & $\frac{k}{2}$ & $\frac{k}{4}$ & $\frac{k}{8}$ \\
        \hline
    \end{tabular}
\end{table}
\begin{enumerate}
    \item Determine the value of $k$.
    \item Determine $P(X \le 2)$ and $P(X > 2)$.
    \item Find $P(X \le 2)$ + $P(X > 2)$.
\end{enumerate}
\solution
%\begin{table}[H]
	\centering
\begin{tabular}{|c|c|c|}
\hline
Random variable &Value &Definition\\ \hline
\multirow{3}{*}{X} &0 &Slips of Rs 1\\
&1 &Slips of Rs 5\\
&2 &Slips of Rs 13\\ \hline
\multirow{2}{*}{Y} &0 &Box A\\
&1 &Box B\\\hline
\end{tabular}
\caption{}
\label{tab:Distribution}
\end{table}
See \tabref{tab:Distribution}.
\begin{align}
p_{Y}\brak{k}= \begin{cases} 
      \frac{1}{3} & {k=0} \\
      \frac{2}{3 }& {k=1} 
   \end{cases}
   \\
p_{Y|X}\brak{0|0} = \frac{19}{25}\, 
p_{Y|X}\brak{0|1} = \frac{6}{25}\,
p_{Y|X}\brak{1|0} = \frac{45}{50}\,
p_{Y|X}\brak{1|2} = \frac{5}{50}
\end{align}
The desired probability is the probability that a slip drawn at random is marked other than Rs 1,
\begin{align}
&=1-p_X\brak{0}\\
&= p_X(1) + p_X(2)
\end{align}
Using Bayes theorem,
\begin{align}
&= p_Y\brak{0} \times \pr{Y=0 | X=1} + p_Y\brak{1} \times \pr{Y=1|X=2}\\
&=\frac{1}{3} \times \frac{6}{25} + \frac{2}{3} \times \frac{5}{50}\\
&=\frac{11}{75}
\end{align}

\newpage

%\tableofcontents

\bigskip

\renewcommand{\thefigure}{\theenumi}
\renewcommand{\thetable}{\theenumi}
%\renewcommand{\theequation}{\theenumi}

%\begin{abstract}
%%\boldmath
%In this letter, an algorithm for evaluating the exact analytical bit error rate  (BER)  for the piecewise linear (PL) combiner for  multiple relays is presented. Previous results were available only for upto three relays. The algorithm is unique in the sense that  the actual mathematical expressions, that are prohibitively large, need not be explicitly obtained. The diversity gain due to multiple relays is shown through plots of the analytical BER, well supported by simulations. 
%
%\end{abstract}
% IEEEtran.cls defaults to using nonbold math in the Abstract.
% This preserves the distinction between vectors and scalars. However,
% if the journal you are submitting to favors bold math in the abstract,
% then you can use LaTeX's standard command \boldmath at the very start
% of the abstract to achieve this. Many IEEE journals frown on math
% in the abstract anyway.

% Note that keywords are not normally used for peerreview papers.
%\begin{IEEEkeywords}
%Cooperative diversity, decode and forward, piecewise linear
%\end{IEEEkeywords}



% For peer review papers, you can put extra information on the cover
% page as needed:
% \ifCLASSOPTIONpeerreview
% \begin{center} \bfseries EDICS Category: 3-BBND \end{center}
% \fi
%
% For peerreview papers, this IEEEtran command inserts a page break and
% creates the second title. It will be ignored for other modes.
%\IEEEpeerreviewmaketitle




\item %\begin{table}[H]
	\centering
\begin{tabular}{|c|c|c|}
\hline
Random variable &Value &Definition\\ \hline
\multirow{3}{*}{X} &0 &Slips of Rs 1\\
&1 &Slips of Rs 5\\
&2 &Slips of Rs 13\\ \hline
\multirow{2}{*}{Y} &0 &Box A\\
&1 &Box B\\\hline
\end{tabular}
\caption{}
\label{tab:Distribution}
\end{table}
See \tabref{tab:Distribution}.
\begin{align}
p_{Y}\brak{k}= \begin{cases} 
      \frac{1}{3} & {k=0} \\
      \frac{2}{3 }& {k=1} 
   \end{cases}
   \\
p_{Y|X}\brak{0|0} = \frac{19}{25}\, 
p_{Y|X}\brak{0|1} = \frac{6}{25}\,
p_{Y|X}\brak{1|0} = \frac{45}{50}\,
p_{Y|X}\brak{1|2} = \frac{5}{50}
\end{align}
The desired probability is the probability that a slip drawn at random is marked other than Rs 1,
\begin{align}
&=1-p_X\brak{0}\\
&= p_X(1) + p_X(2)
\end{align}
Using Bayes theorem,
\begin{align}
&= p_Y\brak{0} \times \pr{Y=0 | X=1} + p_Y\brak{1} \times \pr{Y=1|X=2}\\
&=\frac{1}{3} \times \frac{6}{25} + \frac{2}{3} \times \frac{5}{50}\\
&=\frac{11}{75}
\end{align}

\newpage

%\tableofcontents

\bigskip

\renewcommand{\thefigure}{\theenumi}
\renewcommand{\thetable}{\theenumi}
%\renewcommand{\theequation}{\theenumi}

%\begin{abstract}
%%\boldmath
%In this letter, an algorithm for evaluating the exact analytical bit error rate  (BER)  for the piecewise linear (PL) combiner for  multiple relays is presented. Previous results were available only for upto three relays. The algorithm is unique in the sense that  the actual mathematical expressions, that are prohibitively large, need not be explicitly obtained. The diversity gain due to multiple relays is shown through plots of the analytical BER, well supported by simulations. 
%
%\end{abstract}
% IEEEtran.cls defaults to using nonbold math in the Abstract.
% This preserves the distinction between vectors and scalars. However,
% if the journal you are submitting to favors bold math in the abstract,
% then you can use LaTeX's standard command \boldmath at the very start
% of the abstract to achieve this. Many IEEE journals frown on math
% in the abstract anyway.

% Note that keywords are not normally used for peerreview papers.
%\begin{IEEEkeywords}
%Cooperative diversity, decode and forward, piecewise linear
%\end{IEEEkeywords}



% For peer review papers, you can put extra information on the cover
% page as needed:
% \ifCLASSOPTIONpeerreview
% \begin{center} \bfseries EDICS Category: 3-BBND \end{center}
% \fi
%
% For peerreview papers, this IEEEtran command inserts a page break and
% creates the second title. It will be ignored for other modes.
%\IEEEpeerreviewmaketitle




\item Three persons, A, B and C, fire at a target in turn, starting with A. Their probability of hitting the target are 0.4, 0.3 and 0.2 respectively. The probability of two hits is\\
\begin{enumerate}
\item 0.024
\item 0.188
\item 0.336
\item 0.452
\end{enumerate}
\solution \\
%\begin{table}[H]
	\centering
\begin{tabular}{|c|c|c|}
\hline
Random variable &Value &Definition\\ \hline
\multirow{3}{*}{X} &0 &Slips of Rs 1\\
&1 &Slips of Rs 5\\
&2 &Slips of Rs 13\\ \hline
\multirow{2}{*}{Y} &0 &Box A\\
&1 &Box B\\\hline
\end{tabular}
\caption{}
\label{tab:Distribution}
\end{table}
See \tabref{tab:Distribution}.
\begin{align}
p_{Y}\brak{k}= \begin{cases} 
      \frac{1}{3} & {k=0} \\
      \frac{2}{3 }& {k=1} 
   \end{cases}
   \\
p_{Y|X}\brak{0|0} = \frac{19}{25}\, 
p_{Y|X}\brak{0|1} = \frac{6}{25}\,
p_{Y|X}\brak{1|0} = \frac{45}{50}\,
p_{Y|X}\brak{1|2} = \frac{5}{50}
\end{align}
The desired probability is the probability that a slip drawn at random is marked other than Rs 1,
\begin{align}
&=1-p_X\brak{0}\\
&= p_X(1) + p_X(2)
\end{align}
Using Bayes theorem,
\begin{align}
&= p_Y\brak{0} \times \pr{Y=0 | X=1} + p_Y\brak{1} \times \pr{Y=1|X=2}\\
&=\frac{1}{3} \times \frac{6}{25} + \frac{2}{3} \times \frac{5}{50}\\
&=\frac{11}{75}
\end{align}

\newpage

%\tableofcontents

\bigskip

\renewcommand{\thefigure}{\theenumi}
\renewcommand{\thetable}{\theenumi}
%\renewcommand{\theequation}{\theenumi}

%\begin{abstract}
%%\boldmath
%In this letter, an algorithm for evaluating the exact analytical bit error rate  (BER)  for the piecewise linear (PL) combiner for  multiple relays is presented. Previous results were available only for upto three relays. The algorithm is unique in the sense that  the actual mathematical expressions, that are prohibitively large, need not be explicitly obtained. The diversity gain due to multiple relays is shown through plots of the analytical BER, well supported by simulations. 
%
%\end{abstract}
% IEEEtran.cls defaults to using nonbold math in the Abstract.
% This preserves the distinction between vectors and scalars. However,
% if the journal you are submitting to favors bold math in the abstract,
% then you can use LaTeX's standard command \boldmath at the very start
% of the abstract to achieve this. Many IEEE journals frown on math
% in the abstract anyway.

% Note that keywords are not normally used for peerreview papers.
%\begin{IEEEkeywords}
%Cooperative diversity, decode and forward, piecewise linear
%\end{IEEEkeywords}



% For peer review papers, you can put extra information on the cover
% page as needed:
% \ifCLASSOPTIONpeerreview
% \begin{center} \bfseries EDICS Category: 3-BBND \end{center}
% \fi
%
% For peerreview papers, this IEEEtran command inserts a page break and
% creates the second title. It will be ignored for other modes.
%\IEEEpeerreviewmaketitle




item If two events are independent, then
\begin{enumerate}
\item they must be mutually exclusive
\item the sum of their probabilities must be equal to 1
\item (A) and (B) both are correct
\item None of the above is correct
\end{enumerate}
%Let $X$ be an bernoulli rv defined as in \tabref{tab:exemplar/11/16/3/26}.  Then, 
\begin{equation}
    p =
        \frac{4}{11} 
\end{equation}
\begin{table}[H]
	\centering
	\input{exemplar/11/16/3/26/tables/Table2.tex}
	\caption{}
        \label{tab:exemplar/11/16/3/26}
\end{table}

\item Three letters are dictated to three persons and an envelope is addressed to each of them, the letters are inserted into the envelopes at random so that each envelope contains exactly one letter. Find the probability that at least one letter in its proper envelope.\\
%\begin{table}[H]
	\centering
\begin{tabular}{|c|c|c|}
\hline
Random variable &Value &Definition\\ \hline
\multirow{3}{*}{X} &0 &Slips of Rs 1\\
&1 &Slips of Rs 5\\
&2 &Slips of Rs 13\\ \hline
\multirow{2}{*}{Y} &0 &Box A\\
&1 &Box B\\\hline
\end{tabular}
\caption{}
\label{tab:Distribution}
\end{table}
See \tabref{tab:Distribution}.
\begin{align}
p_{Y}\brak{k}= \begin{cases} 
      \frac{1}{3} & {k=0} \\
      \frac{2}{3 }& {k=1} 
   \end{cases}
   \\
p_{Y|X}\brak{0|0} = \frac{19}{25}\, 
p_{Y|X}\brak{0|1} = \frac{6}{25}\,
p_{Y|X}\brak{1|0} = \frac{45}{50}\,
p_{Y|X}\brak{1|2} = \frac{5}{50}
\end{align}
The desired probability is the probability that a slip drawn at random is marked other than Rs 1,
\begin{align}
&=1-p_X\brak{0}\\
&= p_X(1) + p_X(2)
\end{align}
Using Bayes theorem,
\begin{align}
&= p_Y\brak{0} \times \pr{Y=0 | X=1} + p_Y\brak{1} \times \pr{Y=1|X=2}\\
&=\frac{1}{3} \times \frac{6}{25} + \frac{2}{3} \times \frac{5}{50}\\
&=\frac{11}{75}
\end{align}

\newpage

%\tableofcontents

\bigskip

\renewcommand{\thefigure}{\theenumi}
\renewcommand{\thetable}{\theenumi}
%\renewcommand{\theequation}{\theenumi}

%\begin{abstract}
%%\boldmath
%In this letter, an algorithm for evaluating the exact analytical bit error rate  (BER)  for the piecewise linear (PL) combiner for  multiple relays is presented. Previous results were available only for upto three relays. The algorithm is unique in the sense that  the actual mathematical expressions, that are prohibitively large, need not be explicitly obtained. The diversity gain due to multiple relays is shown through plots of the analytical BER, well supported by simulations. 
%
%\end{abstract}
% IEEEtran.cls defaults to using nonbold math in the Abstract.
% This preserves the distinction between vectors and scalars. However,
% if the journal you are submitting to favors bold math in the abstract,
% then you can use LaTeX's standard command \boldmath at the very start
% of the abstract to achieve this. Many IEEE journals frown on math
% in the abstract anyway.

% Note that keywords are not normally used for peerreview papers.
%\begin{IEEEkeywords}
%Cooperative diversity, decode and forward, piecewise linear
%\end{IEEEkeywords}



% For peer review papers, you can put extra information on the cover
% page as needed:
% \ifCLASSOPTIONpeerreview
% \begin{center} \bfseries EDICS Category: 3-BBND \end{center}
% \fi
%
% For peerreview papers, this IEEEtran command inserts a page break and
% creates the second title. It will be ignored for other modes.
%\IEEEpeerreviewmaketitle




\end{enumerate}

\iffalse
\section{Moments}
\begin{enumerate}[label=\thesubsection.\arabic*,ref=\thesubsection.\theenumi]
	\item The mean of an rv is defined as
		\begin{align}
			\label{moment:def}
			\mu = E\brak{X} = \sum_k kp_X(k)
		\end{align}
	\item The variance of an rv is defined as
		\begin{align}
			\sigma^2= E\brak{X-\mu}^2 = \sum_k \brak{k-\mu}^2p_X(k)
			\label{var:def}
		\end{align}
			
	\item In general, the $n$th moment is defined as
		\begin{align}
			 E\brak{X}^n = \sum_k k^np_X(k)
			\label{moment:def-gen}
		\end{align}
	\item 
		\begin{align}
			 E\brak{X}^2 = \sigma^2 + \mu^2
			\label{var:moment}
		\end{align}
		\solution
			\eqref{var:def}
			can be expressed as
\begin{align}
E\sbrak{X-E\brak{X}}^2
&=E\sbrak{X^2+\sbrak{E\brak{X}}^2-2XE\brak{X}}\\
&=E\brak{X^2}+\sbrak{E\brak{X}}^2-2\sbrak{E\brak{X}}^2\\
&=E\brak{X^2}-\sbrak{E\brak{X}}^2
\end{align} 
yielding
			\eqref{var:moment}.
		\item 
			\begin{align}
				\text{var}(aX) = a^2 \text{var}\brak{X}.
			\label{var:homo-l}
			\end{align}
			\solution
\begin{align}
	\text{var}\brak{{aX}}&=E\brak{{a^2X^2}}-\sbrak{E\brak{{aX}}}^2\\
&=a^2\sbrak{E\brak{X^2}-\sbrak{E\brak{X}}^2}\\
&=a^2{\text{var}\brak{X}}{4}
\end{align}
	\item For $X \sim \bnm{n}{p}$,
		\begin{align}
			\label{moment-mean-var}
			\mu = np, \sigma^2 = np\brak{1-p}
		\end{align}
\end{enumerate}

\section{Random Algebra}
\subsection{Examples}
\input{ncert/alg.tex}
\subsection{Exercises}
\input{exemplar/alg.tex}
\section{Markov Chain}
\input{ncert/markov.tex}
\section{Continuous Distributions}
\input{probability/univar.tex}
\section{Maximum Likelihood Detection: BPSK}
\input{probability/ml.tex}
\section{Transformation of Random Variables}
\input{probability/trans.tex}
\section{Bivariate Random Variables: FSK}
\input{probability/bivar.tex}
\section{Exercises}
\input{manual/gvv_manual_ee5837.tex}
\section{Random Processes}
\input{ncert/randomprocess.tex}
\section{Error Correcting Codes}
\input{ncert/ecc.tex}

%\include{ch02} 
\backmatter
\fi
\appendix
\section{ Z-transform}
\begin{enumerate}[label=\thesection.\arabic*,ref=\thesection.\theenumi]
%From \eqref{eq:dice_xiz}, 
\item Let 
$X_i$ be independent.  For 
\begin{align}
X &= X_1+X_2+...+X_n,
\\
M_X(z)&=\prod_{i=1}^{n}M_{X_i}(z)
\end{align}

\item The $n$th moment of $X$ can be expressed as
\begin{align}
\label{eq:mgf-mean}
E\sbrak{X^n}&= \frac{d^nM_{X}(z^{-1})}{dz^n}\vert_{z=1}
\end{align}
\item {\em The $Z$-transform: }
	\iffalse
The $Z$-transform of $p_X(n)$ is defined as 
%\cite{proakis_dsp}
\begin{align}
P_X(z) = \sum_{n = -\infty}^{\infty}p_X(n)z^{-n}, \quad z \in \mathbb{C}
\label{eq:dice_xz}
\end{align}
%
From \eqref{eq:dice_pmf_xi} and \eqref{eq:dice_xz}, 
\begin{align}
P_{X_1}(z) =P_{X_2}(z) &= \frac{1}{6}\sum_{n = 1}^{6}z^{-n}
\\
&=\frac{z^{-1}\brak{1-z^{-6}}}{6\brak{1-z^{-1}}}, \quad \abs{z} > 1
\label{eq:dice_xiz}
\end{align}
upon summing up the geometric progression.  
\fi
%\cite{proakis_dsp}. 
\end{enumerate}

\iffalse
\section{Central Limit Theorem}
\subsection{Binomial}
\input{app/clt-binom.tex}
\section{Identities}
\input{app/ident.tex}
\section{Random Number Genration Using Shift Registers}
\begin{table}[H]
	\centering
\begin{tabular}{|c|c|c|}
\hline
Random variable &Value &Definition\\ \hline
\multirow{3}{*}{X} &0 &Slips of Rs 1\\
&1 &Slips of Rs 5\\
&2 &Slips of Rs 13\\ \hline
\multirow{2}{*}{Y} &0 &Box A\\
&1 &Box B\\\hline
\end{tabular}
\caption{}
\label{tab:Distribution}
\end{table}
See \tabref{tab:Distribution}.
\begin{align}
p_{Y}\brak{k}= \begin{cases} 
      \frac{1}{3} & {k=0} \\
      \frac{2}{3 }& {k=1} 
   \end{cases}
   \\
p_{Y|X}\brak{0|0} = \frac{19}{25}\, 
p_{Y|X}\brak{0|1} = \frac{6}{25}\,
p_{Y|X}\brak{1|0} = \frac{45}{50}\,
p_{Y|X}\brak{1|2} = \frac{5}{50}
\end{align}
The desired probability is the probability that a slip drawn at random is marked other than Rs 1,
\begin{align}
&=1-p_X\brak{0}\\
&= p_X(1) + p_X(2)
\end{align}
Using Bayes theorem,
\begin{align}
&= p_Y\brak{0} \times \pr{Y=0 | X=1} + p_Y\brak{1} \times \pr{Y=1|X=2}\\
&=\frac{1}{3} \times \frac{6}{25} + \frac{2}{3} \times \frac{5}{50}\\
&=\frac{11}{75}
\end{align}

\newpage

%\tableofcontents

\bigskip

\renewcommand{\thefigure}{\theenumi}
\renewcommand{\thetable}{\theenumi}
%\renewcommand{\theequation}{\theenumi}

%\begin{abstract}
%%\boldmath
%In this letter, an algorithm for evaluating the exact analytical bit error rate  (BER)  for the piecewise linear (PL) combiner for  multiple relays is presented. Previous results were available only for upto three relays. The algorithm is unique in the sense that  the actual mathematical expressions, that are prohibitively large, need not be explicitly obtained. The diversity gain due to multiple relays is shown through plots of the analytical BER, well supported by simulations. 
%
%\end{abstract}
% IEEEtran.cls defaults to using nonbold math in the Abstract.
% This preserves the distinction between vectors and scalars. However,
% if the journal you are submitting to favors bold math in the abstract,
% then you can use LaTeX's standard command \boldmath at the very start
% of the abstract to achieve this. Many IEEE journals frown on math
% in the abstract anyway.

% Note that keywords are not normally used for peerreview papers.
%\begin{IEEEkeywords}
%Cooperative diversity, decode and forward, piecewise linear
%\end{IEEEkeywords}



% For peer review papers, you can put extra information on the cover
% page as needed:
% \ifCLASSOPTIONpeerreview
% \begin{center} \bfseries EDICS Category: 3-BBND \end{center}
% \fi
%
% For peerreview papers, this IEEEtran command inserts a page break and
% creates the second title. It will be ignored for other modes.
%\IEEEpeerreviewmaketitle




\fi
\iffalse
\section{NCERT}
\begin{table}[!ht]
\input{exemplar/12/13/3/41/tables/table.tex}
\caption{random variables of objects}
\label{tab:exemplar 12.13.3.41}
\end{table}
\begin{align}
\pr{X=i}=
\begin{cases}
\frac{1}{6},\text{when i=1}\\
\frac{2}{6},\text{when i=2}\\
\frac{3}{6},\text{when i=3}
\end{cases}
\end{align}

we know that that the conditional probability is defined as

                     $\pr{A|B}=\frac{\pr{A,B}}{\pr{B}}$


\begin{enumerate}
\item
The probability that a red ball will be selected is:
\begin{align}
\pr{Y=1}&=\pr{Y=1,X=1}+\pr{Y=1,X=2}+\pr{Y=1,X=3}\\
&=\pr{X=1}\times\pr{Y=1|X=1}+\pr{X=2}\times\pr{Y=1|X=2}+\pr{X=3}\times\pr{Y=1|X=3}\\
&=\frac{1}{6}\times\frac{3}{3}+\frac{2}{6}\times\frac{2}{3}+\frac{3}{6}\times0\\
&=\frac{7}{18}
\end{align}
\item
The probability that a white ball will be selected is:
\begin{align}
\pr{Y=0}&=\pr{Y=0,X=1}+\pr{Y=0,X=2}+\pr{Y=0,X=3}\\
&=\pr{X=1}\times\pr{Y=0|X=1}+\pr{X=2}\times\pr{Y=0|X=2}+\pr{X=3}\times\pr{Y=0|X=3}\\
&=\frac{1}{6}\times0+\frac{2}{6}\times\frac{1}{3}+\frac{3}{6}\times\frac{3}{3}\\
&=\frac{11}{18}
\end{align}
\end{enumerate}




\latexprintindex
\fi


\end{document}


\newpage
\section*{About this Book}

This book introduces matrices through high school coordinate geometry. This approach makes it easier for beginners to learn Python for scientific computing. All problems in the book are from NCERT mathematics textbooks from Class 9-12.  Exercises are from CBSE board exam papers.   

The content is sufficient for industry jobs and covers nearly all matrix prerequisites for machine learning.
There is no copyright, so readers are free to print and share.  

This book is dedicated to my high school teachers, Dr. G.N. Chandwani and Dr. Anand K. Tripathi.
\begin{flushright}
\today
\end{flushright}
Github: https://github.com/gadepall/matgeo
		\\
License: https://creativecommons.org/licenses/by-sa/3.0/
\\
and
\\
https://www.gnu.org/licenses/fdl-1.3.en.html
\\
First manual appeared in January 2018
\\
First edition published on July 10, 2024
\\
In this edition, some incorrect solutions were removed.  All figures redrawn. More problems added.



