%From \eqref{eq:dice_pmf_xi},
The $Z$-transform of the first die $X_1$ is given by 
\eqref{eq:dice_xiz}.
The pmf of the second die is 
\begin{align}
\label{eq:dice_pmf_x2}
p_{X_2}(n) = 
\begin{cases}
\frac{1}{3} & 1 \le n \le 3 
\\
0 & \text{otherwise}
\end{cases}
\end{align}
yielding 
\begin{align}
M_{X_2}(z) =  \frac{1}{3}\sum_{n = 1}^{3}z^{-n}
=\frac{z^{-1}{(1-z^{-3})}}{3{(1-z^{-1})}}, {|z|} > 1
\end{align}
upon substituting in 
\eqref{eq:dice_xz}.
From 
\eqref{eq:dice_xzprod_def},
The $Z$-transform of X is given as
\begin{align}
	M_X(z) &= \frac{z^{-1}{(1-z^{-6})}}{6{(1-z^{-1})}} \times \frac{z^{-1}{(1-z^{-3})}}{3{(1-z^{-1})}}
\\
	       &= \frac{1}{18}\left[\frac{z^{-2}\brak{(1-z^{-3}-z^{-6}-z^{-9})}}{{(1-z^{-1})}^2}\right]
\end{align}
Using 
\eqref{eq:dice_xdef_props},
after some algebra, it can be shown that,
\begin{multline}
\frac{1}{18}[{{n-1}u(n-1)-{n-4}u(n-4)}
{-(n-7)u(n-7)-(n-10)u(n-10)}]\\
\system{Z}\\
\frac{1}{18} \left[{\frac{z^{-2}{1-z^{-3}-z^{-6}-z^{-9}}}{({1-z^{-1}})^2}}\right]
\end{multline}
Hence,
\iffalse
\begin{multline}
p_{X}(n) = \frac{1}{18}[{{n-1}u(n-1)-{n-4}u(n-4)-}\\
              {(n-7)u(n-7)-(n-10)u(n-10)}]
\end{multline}
\fi
\begin{align}
  p_X(n) &= 
  \begin{cases}
  0 & n \le 1
  \\
  \frac{n-1}{18} &  2 \le n \le  4
  \\
  \frac{1}{6} & 5 \le n \le 7
  \\
  \frac{10-n}{18} & 8 \le n \le 9
  \\
  0 & n \ge 10
  \end{cases}
  \end{align}
  \iffalse
hence, the probabilities are,
\begin{align}
  p_X(n) &= 
\begin{cases}
   \frac{1}{18} & n = 2 \\
   \frac{1}{9} & n = 3 \\
   \frac{1}{6} & n = 4 \\
   \frac{1}{6} & n = 5 \\
   \frac{1}{6} & n = 6 \\
   \frac{1}{6} & n = 7 \\
   \frac{1}{9} & n = 8 \\
   \frac{1}{18} & n = 9 
\end{cases}
\end{align}
\fi
See \figref{fig:mgf-sum-2}.
The experiment of rolling the dice was simulated using Python for 10000 samples.  
%
\begin{figure}[H]
\centering
\includegraphics[width=\columnwidth]{exemplar/10/13/3/23/figs/pmf.png}
\caption{Plot of $p_X(n)$.Simulations are close to the analysis. }
\label{fig:mgf-sum-2}
\end{figure}
