\begin{enumerate}[label=\thesubsection.\arabic*,ref=\thesubsection.\theenumi]
	\item
\begin{align}
	A \cup B \triangleq A+B, A \cap B \triangleq AB.
\end{align}
	\item  Boolean Axioms:  For $A \in \cbrak{0,1}$,
\begin{align}
	A + A^{\prime} = 1
	\\
	A  A^{\prime} = 0
\end{align}
\item De Morgan's Law
\begin{align}
A^{\prime}B^{\prime} &=  \brak{A+B}^{\prime}
	\label{eq:demorgan}
\end{align}
\item Axioms of Probability
	\begin{enumerate}
\item 
\begin{align}
	0 \le \pr{A} \le 1	
	\label{eq:axiom-pos}
\end{align}
\item 
\begin{align}
	\pr{1} =1	
	\label{eq:axiom-one}
\end{align}
\item If $AB = 0$, i.e. $A, B$, are mutually exclusive,
\begin{align}
	\pr{A+B} = \pr{A} + \pr{B}.
\label{eq:axiom_exclusive}
\end{align}
	\end{enumerate}
\item If $A, B$ are independent,
\begin{align}
	\pr{AB} = \pr{A}\pr{B}
\label{eq:axiom-ind}
\end{align}
\item 
\begin{align}
	\pr{A | B} = \frac{\pr{AB}}{\pr{B}}
\end{align}
\item 
	%Substituting \eqref{eq:axiom_sum_ABp} in \eqref{eq:axiom_sum_two}, 
	\begin{align}
\pr{A+B} &= \pr{A} + \pr{B} - \pr{AB} 
\label{eq:axiom_sum_AB}
\end{align}
		\begin{proof}
\begin{align}
	A &= A \brak{B+B^{\prime}} =  AB + AB^{\prime}
\\
	\implies \pr{A} &=  \pr{AB} + \pr{AB^{\prime}}
	\because
	\brak{ AB}\brak{  AB^{\prime}} = 0,
\label{eq:axiom_sum_A}
\end{align}
from \eqref{eq:axiom_exclusive}.
Similarly,
\begin{align}
A+B &= A\brak{B+B^{\prime}} + B
\\
&= B \brak{A +1} + A B^{\prime}
\\
&=B + A B^{\prime}
\\
	\implies \pr{A+B } 
	&=\pr{B} + \pr{A B^{\prime}}\quad \because B  A B^{\prime} = 0
\label{eq:axiom_sum_ApB}
\end{align}
From 
\eqref{eq:axiom_sum_A}
and 
\eqref{eq:axiom_sum_ApB},
we obtain
\eqref{eq:axiom_sum_AB}.
		\end{proof}
	\item 
From \eqref{eq:axiom_sum_A}
and 
	\eqref{eq:axiom-pos},
\begin{align}
	\label{eq:axiom-prod-ge}
	\pr{A} \ge \pr{AB}
\end{align}
\end{enumerate}
\iffalse
and 
\begin{align}
 \because BB^{\prime} = 0
\label{eq:axiom_sum_AB0}
\end{align}
Hence, $AB$ and $AB^{\prime}$ are mutually exclusive. 
\fi
\iffalse
\item 
Let A, B and C be three events.

Let X be the event that exactly one of A, B and C occurs.

Let Y be the event that at least one of A, B or C occur.

Let Z be the event that at least two of A, B or C occur.
\begin{align}
	Y &=A+B+C
\end{align}
Similarly,
\begin{align}
	Z &=AB+BC+CA
\end{align}
And,
\begin{align}
    X&=(AB^\prime C^\prime+A^\prime BC^\prime+A^\prime B^\prime C)
	\label{eq:axiom_occurrence_of_exactly_one}
\end{align}
\item 
\begin{align}
	\pr{A^{\prime}} = 1 - \pr{A}.
\end{align}
\item
\begin{align}
\pr{A^{\prime}B^{\prime}} &=  \pr{\brak{A+B}^{\prime}} 
\\
&= 1 - \pr{A+B} 
\label{eq:axiom_sum_one}
\end{align}
\item 
\begin{align}
\pr{A+B} &= \pr{B + A B^{\prime} }
\\
&=\pr{B}+\pr{ A B^{\prime} } 
\label{eq:axiom_sum_two}
\\
&\because B \brak{ A B^{\prime} } = 0
\end{align}
\item
\begin{align}
\pr{A} = \pr{AB} + \pr{AB^{\prime}}
\\
\implies 
\pr{AB^{\prime}} =  \pr{A} - \pr{AB}
\label{eq:axiom_sum_ABp}
\end{align}
\fi
