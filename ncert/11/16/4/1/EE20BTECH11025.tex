\documentclass[journal,12pt,twocolumn]{IEEEtran}
\usepackage{romannum}
\usepackage{float}
\usepackage{setspace}
\usepackage{gensymb}
\singlespacing
\usepackage[cmex10]{amsmath}
\usepackage{amsthm}
\usepackage{mathrsfs}
\usepackage{txfonts}
\usepackage{stfloats}
\usepackage{bm}
\usepackage{cite}
\usepackage{cases}
\usepackage{subfig}
\usepackage{longtable}
\usepackage{multirow}
\usepackage{enumitem}
\usepackage{mathtools}
\usepackage{steinmetz}
\usepackage{tikz}
\usepackage{circuitikz}
\usepackage{verbatim}
\usepackage{tfrupee}
\usepackage[breaklinks=true]{hyperref}
\usepackage{tkz-euclide}
\usetikzlibrary{calc,math}
\usepackage{listings}
    \usepackage{color}                                            %%
    \usepackage{array}                                            %%
    \usepackage{longtable}                                        %%
    \usepackage{calc}                                             %%
    \usepackage{multirow}                                         %%
    \usepackage{hhline}                                           %%
    \usepackage{ifthen}                                           %%
  %optionally (for landscape tables embedded in another document): %%
    \usepackage{lscape}     
\usepackage{multicol}
\usepackage{chngcntr}
\DeclareMathOperator*{\Res}{Res}
\renewcommand\thesection{\arabic{section}}
\renewcommand\thesubsection{\thesection.\arabic{subsection}}
\renewcommand\thesubsubsection{\thesubsection.\arabic{subsubsection}}

\renewcommand\thesectiondis{\arabic{section}}
\renewcommand\thesubsectiondis{\thesectiondis.\arabic{subsection}}
\renewcommand\thesubsubsectiondis{\thesubsectiondis.\arabic{subsubsection}}

% correct bad hyphenation here
\hyphenation{op-tical net-works semi-conduc-tor}
\def\inputGnumericTable{}                                 %%

\lstset{
frame=single, 
breaklines=true,
columns=fullflexible
}

\begin{document}


\newtheorem{theorem}{Theorem}[section]
\newtheorem{problem}{Problem}
\newtheorem{proposition}{Proposition}[section]
\newtheorem{lemma}{Lemma}[section]
\newtheorem{corollary}[theorem]{Corollary}
\newtheorem{example}{Example}[section]
\newtheorem{definition}[problem]{Definition}
\newcommand{\BEQA}{\begin{eqnarray}}
\newcommand{\EEQA}{\end{eqnarray}}
\newcommand{\define}{\stackrel{\triangle}{=}}

\bibliographystyle{IEEEtran}
\providecommand{\mbf}{\mathbf}
\providecommand{\pr}[1]{\ensuremath{\Pr\left(#1\right)}}
\providecommand{\qfunc}[1]{\ensuremath{Q\left(#1\right)}}
\providecommand{\sbrak}[1]{\ensuremath{{}\left[#1\right]}}
\providecommand{\lsbrak}[1]{\ensuremath{{}\left[#1\right.}}
\providecommand{\rsbrak}[1]{\ensuremath{{}\left.#1\right]}}
\providecommand{\brak}[1]{\ensuremath{\left(#1\right)}}
\providecommand{\lbrak}[1]{\ensuremath{\left(#1\right.}}
\providecommand{\rbrak}[1]{\ensuremath{\left.#1\right)}}
\providecommand{\cbrak}[1]{\ensuremath{\left\{#1\right\}}}
\providecommand{\lcbrak}[1]{\ensuremath{\left\{#1\right.}}
\providecommand{\rcbrak}[1]{\ensuremath{\left.#1\right\}}}
\theoremstyle{remark}
\newtheorem{rem}{Remark}
\newcommand{\sgn}{\mathop{\mathrm{sgn}}}
\providecommand{\abs}[1]{\left\vert#1\right\vert}
\providecommand{\res}[1]{\Res\displaylimits_{#1}} 
\providecommand{\norm}[1]{\left\lVert#1\right\rVert}
\providecommand{\mtx}[1]{\mathbf{#1}}
\providecommand{\mean}[1]{E\left[ #1 \right]}
\providecommand{\fourier}{\overset{\mathcal{F}}{ \rightleftharpoons}}
\providecommand{\system}{\overset{\mathcal{H}}{ \longleftrightarrow}}
\newcommand{\solution}{\noindent \textbf{Solution: }}
\newcommand{\cosec}{\,\text{cosec}\,}
\providecommand{\dec}[2]{\ensuremath{\overset{#1}{\underset{#2}{\gtrless}}}}
\newcommand{\myvec}[1]{\ensuremath{\begin{pmatrix}#1\end{pmatrix}}}
\newcommand{\mydet}[1]{\ensuremath{\begin{vmatrix}#1\end{vmatrix}}}
\numberwithin{equation}{subsection}
\makeatletter
\@addtoreset{figure}{problem}
\makeatother

\let\StandardTheFigure\thefigure
\let\vec\mathbf
\renewcommand{\thefigure}{\theproblem}



\def\putbox#1#2#3{\makebox[0in][l]{\makebox[#1][l]{}\raisebox{\baselineskip}[0in][0in]{\raisebox{#2}[0in][0in]{#3}}}}
     \def\rightbox#1{\makebox[0in][r]{#1}}
     \def\centbox#1{\makebox[0in]{#1}}
     \def\topbox#1{\raisebox{-\baselineskip}[0in][0in]{#1}}
     \def\midbox#1{\raisebox{-0.5\baselineskip}[0in][0in]{#1}}

\vspace{3cm}


\title{Assignment 1}
\author{Jaswanth Chowdary Madala}





% make the title area
\maketitle

\newpage

%\tableofcontents

\bigskip

\renewcommand{\thefigure}{\theenumi}
\renewcommand{\thetable}{\theenumi}



\begin{enumerate}
\item A box contains 10 red marbles, 20 blue marbles and 30 green marbles. 5 marbles
are drawn from the box, what is the probability that
\begin{enumerate}
\item all will be blue?
\item atleast one will be green?
\end{enumerate}
\textbf{Solution:}\\

\textbf{Lemma-1:}
The probability of the event where $n$ marbles drawn such that there are - $r$ red, $b$ blue, $g$ green marbles from the box which contains total of $N$ marbles - $R$ red, $B$ blue, $G$ green is $\frac{^{R}C_{r} \, ^{B}C_{b} \, ^{G}C_{g}}{^{N}C_{n}}$\\
\textbf{Proof:} \\
Total ways of choosing $r$ red balls - $^{R}C_{r}$\\
Total ways of choosing $b$ blue balls - $^{B}C_{b}$\\
Total ways of choosing $g$ green balls - $^{G}C_{g}$\\
The total number of ways of choosing $n$ marbles such that there are $r$ red, $b$ blue, $g$ green is $^{R}C_{r} \, ^{B}C_{b} \, ^{G}C_{g}$\\
The total number of ways of choosing $n$ marbles out of $N$ marbles is $^{N}C_{n}$\\
Hence the required probability is given by,
\begin{align}
&= \frac{^{P}C_{p} \times ^{Q}C_{q} \times ^{R}C_{r}}{^{N}C_{n}}
\label{eq:ncert/11/16/4/1/1}
\end{align}
\\
\textbf{Lemma-2:} 
\begin{align}
{^{R}C_{0}} \, {^{B}C_{n}} + {^{R}C_{1}} \, {^{B}C_{n-1}} + \cdots + {^{R}C_{n}} \, {^{B}C_{0}} = {^{R+B}C_{n}} \label{eq:ncert/11/16/4/1/2}
\end{align}
\textbf{Proof:} \\
We solve this by relating the LHS of the equation \eqref{eq:ncert/11/16/4/1/2} to some coefficient in a Binomial expansion. From Binomial theorem we have,
\begin{align}
\brak{a+b}^n = \sum_{k = 0}^{n} {^{n}C_{k}} \, a^r \, b^{n-r}
\end{align}
Consider the following Binomial expansions,
\begin{align}
\brak{x+1}^R = \sum_{k = 0}^{R} {^{R}C_{k}} \, x^k \label{eq:ncert/11/16/4/1/3}\\
\brak{x+1}^B = \sum_{m = 0}^{B} {^{B}C_{m}} \, x^m \label{eq:ncert/11/16/4/1/4}
\end{align}
Now lets take the product of the eqations \eqref{eq:ncert/11/16/4/1/3}, \eqref{eq:ncert/11/16/4/1/4}
\begin{align}
\brak{x+1}^R \brak{x+1}^B = \sum_{k = 0}^{R} \sum_{m = 0}^{B} {^{R}C_{k}} {^{B}C_{m}}\, x^{k+m}\\
\implies \brak{x+1}^{R+B} = \sum_{k = 0}^{R} \sum_{m = 0}^{B} {^{R}C_{k}} {^{B}C_{m}}\, x^{k+m} \label{eq:ncert/11/16/4/1/5}
\end{align} 
From the RHS of \eqref{eq:ncert/11/16/4/1/5}, the required expression, LHS of \eqref{eq:ncert/11/16/4/1/2} is the coefficient of $x^{n}$ in the above equation. The coefficient of $x^{n}$ in LHS of \eqref{eq:ncert/11/16/4/1/5} is ${^{R+B}C_{n}}$. 
\begin{align}
\implies {^{R}C_{0}} \, {^{B}C_{n}} + {^{R}C_{1}} \, {^{B}C_{n-1}} + \cdots + {^{R}C_{n}} \, {^{B}C_{0}} = {^{R+B}C_{n}} 
\end{align}


In this question, total marbles in the box are 60 - 10 red, 20 blue, 30 green. Out of which 5 balls are drawn. Hence we have,
\begin{align}
N = 60, \, R = 10, \,B = 20,  \, G = 30, \,n = 5
\end{align}
\begin{enumerate}
\item The probability that all drawn marbles are blue implies
\begin{align}
r = 0,\,  b = 5, \, G = 0
\end{align}
From \eqref{eq:ncert/11/16/4/1/1} we get the probability as,
\begin{align}
&= \frac{^{20}C_{5}}{^{60}C_{5}}
\end{align}
\item The probability that the drawn marble contains atleast 1 green, It is complement to the event where no marble drawn is green, For this event
\begin{align}
g = 0
\end{align}
From \eqref{eq:ncert/11/16/4/1/1} its probability is given by,
\begin{align}
&= \frac{{^{10}C_{0}} \, {^{20}C_{5}} + {^{10}C_{1}} \, {^{20}C_{4}} + \cdots + {^{10}C_{5}} \, {^{20}C_{0}}}{^{60}C_{5}}
\end{align}
From \eqref{eq:ncert/11/16/4/1/2} we get this as,
\begin{align}
&= \frac{^{10+20}C_{5}}{^{60}C_{5}}\\
&= \frac{^{30}C_{5}}{^{60}C_{5}}
\end{align}
Hence the required probability is given by,
\begin{align}
1 - \frac{^{30}C_{5}}{^{60}C_{5}}
\end{align}
\end{enumerate}
\end{enumerate}
\end{document}
