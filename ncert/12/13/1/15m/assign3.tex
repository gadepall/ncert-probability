\documentclass[journal,12pt,two column]{IEEEtran}
%\usepackage{setspace}
\usepackage{amssymb}
\usepackage[cmex10]{amsmath}
\usepackage{amsthm}
\usepackage[export]{adjustbox}
\usepackage{bm}
\def\inputGnumericTable{} 

\usepackage[latin1]{inputenc}                                 
\usepackage{color}                                            
\usepackage{array} 
\usepackage{longtable} 
\usepackage{calc}                                             
\usepackage{multirow}                                         
\usepackage{hhline}                                           
\usepackage{ifthen}  
\usepackage{mathtools}
\usepackage{tikz}
\usepackage{listings}
\usepackage{color}                                            %%
\usepackage{array}                                            %%
\usepackage{caption} 
\usepackage{graphicx}

\title{Assignment 3 \\ 12.13.1.15}
\author{K.SaiTeja \\ AI22BTECH11014}
\providecommand{\pr}[1]{\ensuremath{\Pr\left(#1\right)}}
\providecommand{\qfunc}[1]{\ensuremath{Q\left(#1\right)}}
\providecommand{\sbrak}[1]{\ensuremath{{}\left[#1\right]}}
\providecommand{\lsbrak}[1]{\ensuremath{{}\left[#1\right.}}
\providecommand{\rsbrak}[1]{\ensuremath{{}\left.#1\right]}}
\providecommand{\brak}[1]{\ensuremath{\left(#1\right)}}
\providecommand{\lbrak}[1]{\ensuremath{\left(#1\right.}}
\providecommand{\rbrak}[1]{\ensuremath{\left.#1\right)}}
\providecommand{\cbrak}[1]{\ensuremath{\left\{#1\right\}}}
\providecommand{\lcbrak}[1]{\ensuremath{\left\{#1\right.}}
\providecommand{\rcbrak}[1]{\ensuremath{\left.#1\right\}}}
\newcommand*{\permcomb}[4][0mu]{{{}^{#3}\mkern#1#2_{#4}}}
\newcommand*{\perm}[1][-3mu]{\permcomb[#1]{P}}
\newcommand*{\comb}[1][-1mu]{\permcomb[#1]{C}}
\renewcommand{\thetable}{\arabic{table}} 
\newcommand{\question}{\noindent \textbf{Question: }}	
\newcommand{\solution}{\noindent \textbf{Solution: }}
\begin{document}
\maketitle
\question 15)
Consider the experiment of throwing a die. If a multiple of 3 comes up, throw the die again. If any other number comes up, toss a coin. Find the conditional probability of the event the coin shows a tail, given that  at least one die shows a 3.\\

\solution
The experiment: A die is thrown: 
\begin{enumerate}
\item A multiple of 3 (i.e 3 or 6)  out of 6 outcomes(1, 2, 3, 4, 5, 6)                          
\item Any other number (i.e 1, 2, 4, 5)   then a coin is tossed (H, T)
\end{enumerate}
The sample space of the experiment is given by:
\begin{align*}\\
S = \{&(3,1), (3,2), (3,3), (3,4), (3,5), (3,6), \\
&(6,1), (6,2), (6,3), (6,4), (6,5), (6,6), \\
&1H, 2H, 4H, 5H, 1T, 2T, 4T, 5T\}
\end{align*} 
\\
Let $E$ be the event that the coin shows a tail and \\ Let $F$ be the event that at least one die shows a 3. 
\begin{center}
Then, E = \{1T, 2T, 4T, 5T\} \\
F = \{(3,1), (3,2), (3,3), (3,4), (3,5), (3,6), (6,3)\}.
\end{center}
\begin{align}
\implies EF &= \emptyset,\\ \implies \Pr(EF) &= 0.
\end{align}
\begin{center}
Therefore, the conditional probability of the event the coin shows a tail given that at least one die shows a 3 is:
\end{center}
\begin{align}
\pr{E|F} &= \frac{\pr{EF}}{\pr{F}} \\
&= \frac{0}{\Pr(F)} \\
&= 0
\end{align}

Hence, the conditional probability is 0.

\end{document}

