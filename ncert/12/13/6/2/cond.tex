		Consider the random variables $X, Y$, which denotes the first child, second child gender respectively as described in table \ref{tab:ncert/12/13/6/2/1}.
\begin{table}[h]
\centering
%%%%%%%%%%%%%%%%%%%%%%%%%%%%%%%%%%%%%%%%%%%%%%%%%%%%%%%%%%%%%%%%%%%%%%
%%                                                                  %%
%%  This is a LaTeX2e table fragment exported from Gnumeric.        %%
%%                                                                  %%
%%%%%%%%%%%%%%%%%%%%%%%%%%%%%%%%%%%%%%%%%%%%%%%%%%%%%%%%%%%%%%%%%%%%%%
\begin{tabular}{|c|l|l|}\hline
	\textbf{Variable}&\textbf{Description}&\textbf{Probability}\\\hline
A	&Person with heat attack	&\pr{A}=0.40\\\hline
$E_1$	&Person treated with meditation and yoga	&\pr{E_1}=0.50\\\hline
$E_2$	&Person treated with drug	&\pr{E_2}=0.50\\\hline
\end{tabular}

\caption{}
\label{tab:ncert/12/13/6/2/1}
\end{table}
%
The probabilities for the random variables $X,Y$ is listed in table \ref{tab:ncert/12/13/6/2/2}.
\begin{table}[h]
\centering
\input{ncert/12/13/6/2/tables/table2.tex}
\caption{}
\label{tab:ncert/12/13/6/2/2}
\end{table}
%
The probability $\pr{XY = 0}$ is given by 
\begin{align}
&= \pr{X = 0} + \pr{Y = 0} - \pr{X+Y = 0}\\
&= \frac{1}{2} +\frac{1}{2} - \frac{1}{4}\\
&= \frac{3}{4}
\end{align}
%
\begin{enumerate}
\item  The event of both children being Male is when $X +Y = 0$. The event of atleast one of the children being Male is when $XY = 0$.
\begin{align}
\{X + Y = 0\} \cap \{XY = 0\} \equiv \{X + Y = 0\}
\end{align}

The required probability is given by,
\begin{align}
\pr{X + Y = 0 \mid XY = 0}
\end{align}
\begin{align}
&= \frac{\pr{X + Y = 0}}{\pr{XY = 0}}\\
&= \frac{1}{3}
\end{align}
\item  The event of both children being Female is when $X + Y = 2$. The event of elder child being Female is when $X = 1$.
\begin{align}
\{X + Y = 2\} \cap \{X = 1\} \equiv \{X + Y = 2\}
\end{align}
The required probability is given by,
\begin{align}
\pr{X + Y = 2 \mid X = 1}
\end{align}
\begin{align}
&= \frac{\pr{X + Y = 2}}{\pr{X = 1}}\\
&= \frac{1}{2}
\end{align}
\end{enumerate}


