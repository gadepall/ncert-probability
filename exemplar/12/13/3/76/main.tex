\iffalse
\let\negmedspace\undefined
\let\negthickspace\undefined
\documentclass[journal,12pt,onecolumn]{IEEEtran}
\usepackage{cite}
\usepackage{amsmath,amssymb,amsfonts,amsthm}
\usepackage{algorithmic}
\usepackage{graphicx}
\usepackage{textcomp}
\usepackage{xcolor}
\usepackage{txfonts}
\usepackage{listings}
\usepackage{enumitem}
\usepackage{mathtools}
\usepackage{gensymb}
\usepackage[breaklinks=true]{hyperref}
\usepackage{tkz-euclide} % loads  TikZ and tkz-base
\usepackage{listings}



\newtheorem{theorem}{Theorem}[section]
\newtheorem{problem}{Problem}
\newtheorem{proposition}{Proposition}[section]
\newtheorem{lemma}{Lemma}[section]
\newtheorem{corollary}[theorem]{Corollary}
\newtheorem{example}{Example}[section]
\newtheorem{definition}[problem]{Definition}
%\newtheorem{thm}{Theorem}[section] 
%\newtheorem{defn}[thm]{Definition}
%\newtheorem{algorithm}{Algorithm}[section]
%\newtheorem{cor}{Corollary}
\newcommand{\BEQA}{\begin{eqnarray}}
\newcommand{\EEQA}{\end{eqnarray}}
\newcommand{\define}{\stackrel{\triangle}{=}}
\theoremstyle{remark}
\newtheorem{rem}{Remark}
%\bibliographystyle{ieeetr}
\begin{document}
%
\providecommand{\pr}[1]{\ensuremath{\Pr\left(#1\right)}}
\providecommand{\prt}[2]{\ensuremath{p_{#1}^{\left(#2\right)} }}        % own macro for this question
\providecommand{\qfunc}[1]{\ensuremath{Q\left(#1\right)}}
\providecommand{\sbrak}[1]{\ensuremath{{}\left[#1\right]}}
\providecommand{\lsbrak}[1]{\ensuremath{{}\left[#1\right.}}
\providecommand{\rsbrak}[1]{\ensuremath{{}\left.#1\right]}}
\providecommand{\brak}[1]{\ensuremath{\left(#1\right)}}
\providecommand{\lbrak}[1]{\ensuremath{\left(#1\right.}}
\providecommand{\rbrak}[1]{\ensuremath{\left.#1\right)}}
\providecommand{\cbrak}[1]{\ensuremath{\left\{#1\right\}}}
\providecommand{\lcbrak}[1]{\ensuremath{\left\{#1\right.}}
\providecommand{\rcbrak}[1]{\ensuremath{\left.#1\right\}}}
\newcommand{\sgn}{\mathop{\mathrm{sgn}}}
\providecommand{\abs}[1]{\left\vert#1\right\vert}
\providecommand{\res}[1]{\Res\displaylimits_{#1}} 
\providecommand{\norm}[1]{\left\lVert#1\right\rVert}
%\providecommand{\norm}[1]{\lVert#1\rVert}
\providecommand{\mtx}[1]{\mathbf{#1}}
\providecommand{\mean}[1]{E\left[ #1 \right]}
\providecommand{\cond}[2]{#1\middle|#2}
\providecommand{\fourier}{\overset{\mathcal{F}}{ \rightleftharpoons}}
\newenvironment{amatrix}[1]{%
  \left(\begin{array}{@{}*{#1}{c}|c@{}}
}{%
  \end{array}\right)
}
%\providecommand{\hilbert}{\overset{\mathcal{H}}{ \rightleftharpoons}}
%\providecommand{\system}{\overset{\mathcal{H}}{ \longleftrightarrow}}
	%\newcommand{\solution}[2]{\textbf{Solution:}{#1}}
\newcommand{\solution}{\noindent \textbf{Solution: }}
\newcommand{\cosec}{\,\text{cosec}\,}
\providecommand{\dec}[2]{\ensuremath{\overset{#1}{\underset{#2}{\gtrless}}}}
\newcommand{\myvec}[1]{\ensuremath{\begin{pmatrix}#1\end{pmatrix}}}
\newcommand{\mydet}[1]{\ensuremath{\begin{vmatrix}#1\end{vmatrix}}}
\newcommand{\myaugvec}[2]{\ensuremath{\begin{amatrix}{#1}#2\end{amatrix}}}
\providecommand{\rank}{\text{rank}}
\providecommand{\pr}[1]{\ensuremath{\Pr\left(#1\right)}}
\providecommand{\qfunc}[1]{\ensuremath{Q\left(#1\right)}}
	\newcommand*{\permcomb}[4][0mu]{{{}^{#3}\mkern#1#2_{#4}}}
\newcommand*{\perm}[1][-3mu]{\permcomb[#1]{P}}
\newcommand*{\comb}[1][-1mu]{\permcomb[#1]{C}}
\providecommand{\qfunc}[1]{\ensuremath{Q\left(#1\right)}}
\providecommand{\gauss}[2]{\mathcal{N}\ensuremath{\left(#1,#2\right)}}
\providecommand{\diff}[2]{\ensuremath{\frac{d{#1}}{d{#2}}}}
\providecommand{\myceil}[1]{\left \lceil #1 \right \rceil }
\newcommand\figref{Fig.~\ref}
\newcommand\tabref{Table~\ref}
\newcommand{\sinc}{\,\text{sinc}\,}
\newcommand{\rect}{\,\text{rect}\,}
%%
%	%\newcommand{\solution}[2]{\textbf{Solution:}{#1}}
%\newcommand{\solution}{\noindent \textbf{Solution: }}
%\newcommand{\cosec}{\,\text{cosec}\,}
%\numberwithin{equation}{section}
%\numberwithin{equation}{subsection}
%\numberwithin{problem}{section}
%\numberwithin{definition}{section}
%\makeatletter
%\@addtoreset{figure}{problem}
%\makeatother

%\let\StandardTheFigure\thefigure
\let\vec\mathbf

\bibliographystyle{IEEEtran}


\vspace{3cm}



\bigskip

\renewcommand{\thefigure}{\theenumi}
\renewcommand{\thetable}{\theenumi}
%\renewcommand{\theequation}{\theenumi}
Question: Three persons, A, B and C, fire at a target in turn, starting with A. Their probability of hitting the target are 0.4, 0.3 and 0.2 respectively. The probability of two hits is\\
(A) 0.024\\
(B) 0.188\\
(C) 0.336\\
(D) 0.452\\
\solution \\
\fi
Let $X$, $Y$ and $Z$ be random variables with definition given as under:
\begin{table}[h]
\centering
\begin{tabular}{|l|c|r|}
\hline
    Random Variable &  Description\\
    \hline
    $X$ & A hitting the target \\
    \hline
    $Y$ &  B hitting the target \\
    \hline
    $Z$ &  C hitting the target\\
    \hline
\end{tabular}
\caption{Definition of Random variables}
\end{table}
\begin{table}[h]
\centering
\begin{tabular}{|l|c|r|}
\hline
    Bernouli probability & Value & Description\\
    \hline
    $p_1$ & Probability of A hitting the target &0.4\\
    \hline
    $p_2$ &  Probability of B hitting the target&0.3 \\
    \hline
    $p_3$ &  Probability of C hitting the target&0.2\\
    \hline
\end{tabular}
\caption{Probabilities}
\end{table}

We want to find the probability of two hits, which corresponds to $S = X + Y + Z$ being equal to 2\\
PMF of $S$ using $z$-transform:\\
Applying the z-transform on both the sides\\
\begin{align}
	M_S(z) &= M_{X+Y+Z}(z)
\end{align}
Using the expectation operator:\\
\begin{align}
	E\sbrak{z^{-S}}= M_X(z)\cdot M_Y(z)\cdot M_Z(z)
\end{align}
For a Bernoulli random variable X with parameter p, the Z-transform is given by:
\begin{align}
M_X(z) = (1 - p_i) +zp_i 
\end{align}
For the random variables\\
\begin{align}
M_X(z)=(1-0.4)+0.4z=0.6+0.4z \\
M_Y(z)=(1-0.3)+0.3z=0.7+0.3z \\
M_Z(z)=(1-0.2)+0.2z=0.8+0.2z
\end{align}
Now, we can find the z-transform of S, denoted as $M_S(z)$, which is the product of the Z-transforms of $X$,$Y$ and $Z$ since they are independent\\
Coefficient of the $z^{2}$ term in $M_S(z)$ is corresponds to the probability of two hits. \\
\begin{align}
M_S(z)&=M_X(z)M_Y(z)M_Z(z)\\
 &= \prod_{k=1}^{3} (1 - p_i) +zp_i \\
&=(0.6+0.4z) (0.7+0.3z) (0.8+0.2z)\\
&=0.188z^{2}+1.208z+..\\
\therefore \text{The probability of two hits} =0.188
\end{align}

