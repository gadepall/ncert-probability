\solution
\begin{enumerate}
	\item If $E_1$ and $E_2$ are mutually exclusive events, then $E_1E_2 = \phi$.\\
	\item If $E_1$ and $E_2$ are mutually exclusive and exhaustive events, then $ E_1 E_2 = \phi $ and $ E_1 + E_2=S $ \\
	\item If $E_1$ and $E_2$ have common outcomes, this means:
\begin{align}
	E_1E_2&\neq0
\end{align}
Let $E_a$ be the outcomes that are present in $E_1$ and not in $E_2$. So,
\begin{align}
	E_a&=E_1-E_2 \label{eq:exemplar/11/16/3/43/1}
\end{align}
Let $E_b$ be the outcomes common between $E_1$ and $E_2$. So,
\begin{align}
	E_b&=E_1E_2 \label{eq:exemplar/11/16/3/43/2}
\end{align}
So, we can say that
\begin{align}
	E_1&=E_a+E_b
\end{align}
Refering to equation \eqref{eq:exemplar/11/16/3/43/1} and \eqref{eq:exemplar/11/16/3/43/2}:
\begin{align}
	E_1&=(E_1-E_2)+(E_1 E_2)
\end{align}
        \item If $E_1$ and $E_2$ are two events such that $E_1 \subset E_2$, then let E be subset of $E_2$ containing elements other than $E_1$. So,
\begin{align}
	E_1 + E &= E_2 \text{  and  } E_1 E = E_2\label{eq:exemplar/11/16/3/43/3}
\end{align}
Refering to equation \eqref{eq:exemplar/11/16/3/43/3}:
\begin{align}
	E_1 E_2 &= E_1 (E_1+E)\\
	&=(E_1 E_1)+(E_1 E)\\
	&=E_1
\end{align}
\end{enumerate}
	Hence,
\begin{multicols}{4}
\begin{enumerate}
	\item $\leftrightarrow(iv)$,\item $\leftrightarrow(iii)$,\item $\leftrightarrow(ii)$,\item $\leftrightarrow(i)$
\end{enumerate}
\end{multicols}



